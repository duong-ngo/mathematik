\documentclass[class=probandstats,crop=false]{standalone}

\begin{document}

\chapter{Probability}

\section{Properties of probabilities}

\begin{exercise}
    \par Of a group of patients having injuries, 28\% visit both a physical therapist and a chiropractor and 8\% visit neither.
    \par Say that the probability of visiting a physical therapist exceeds the probability of visiting a chiropractor by 16\%.
    \par What is the probability of a randomly selected person from this group visiting a physical therapist?
\end{exercise}

\begin{proof}[Solution]
    \par $A$ is the event \textit{patient visiting physical therapist}.
    \par $B$ is the event \textit{patient visiting chiropractor}.
    \bigskip
    \par According to assumption:
    \[
        \begin{cases}
            P(A'\cap B') = 0.08 \\
            P(A\cap B) = 0.28
        \end{cases}
    \]
    \par $P(A'\cap B') = 1 - P(A\cup B) = 1 - P(A) - P(B) + P(A\cap B)$.
    \par Therefore, $P(A) + P(B) = 1 - P(A'\cap B') + P(A\cap B) = 1 - 0.08 + 0.28 = 0.8$.
    \par Together with $P(A) - P(B) = 0.16$, we obtain that $P(A) = 0.48$.
    \par Hence probability of a randomly selected patient visiting a physical therapist is 48\%.
\end{proof}

\begin{exercise}
    \par An insurance company looks at its auto insurance customers and finds that
    \begin{enumerate}[label = (\alph*)]
        \item all insure at least one car,
        \item 85\% insure more than one car,
        \item 23\% insure a sports car,
        \item 17\% insure more than one car, including a sports car.
    \end{enumerate}
    \par Find the probability that a customer selected at random insures exactly one car and it is not a sports car.
\end{exercise}

\begin{proof}[Solution]
    \par $A$ is the event \textit{a customer insures exactly one car}.
    \par $B$ is the event \textit{a customer insures sport car}.
    \par $C$ is the event \textit{a customer insures more than one car}.
    \par According to assumption:
    \[
        \begin{cases}
            P(A) + P(C) = 1 \\
            P(B) = 0.23     \\
            P(C) = 0.85     \\
            P(B\cap C) = 0.17
        \end{cases}
    \]
    \par The event \textit{a customer insures exactly one car and not sports car} is expressed by the set $A\cap C'$
    \begin{align*}
        P(A\cap C') & = P(B'\cap C')                 \\
                    & = 1 - P(B\cup C)               \\
                    & = 1 - P(B) - P(C) + P(B\cap C) \\
                    & = 1 - 0.23 - 0.85 + 0.17       \\
                    & = 0.09.
    \end{align*}
    \par Hence the probability of a customer selected at random insures exactly one car and it is not a sports car is 9\%.
\end{proof}

\begin{exercise}
    \par Draw one card at random from a standard deck of cards.
    \par The sample space $S$ is the collection of the 52 cards.
    \par Assume that the probability set function assigns 1/52 to each of the 52 outcomes. Let
    \begin{align*}
        A & = \{ x: J, Q, K \}                           \\
        B & = \{ x: 9, 10, \text{or J and $x$ is red} \} \\
        C & = \{ x: \text{club} \}                       \\
        D & = \{ x: \text{diamond, heart, or spade} \}
    \end{align*}
    \par Find \textbf{(a)} $P(A)$, \textbf{(b)} $P(A\cap B)$, \textbf{(c)} $P(A\cup B)$, \textbf{(d)} $P(C\cup D)$, \textbf{(e)} $P(C\cap D)$.
\end{exercise}

\begin{proof}[Solution]
    \begin{enumerate}[label = \textbf{(\alph*)}]
        \item
              \[
                  P(A) = \dfrac{3 * 4}{52} = \dfrac{3}{13}.
              \]
        \item
              \[
                  P(A\cap B) = \dfrac{2}{52} = \dfrac{1}{26}.
              \]
        \item
              \[
                  P(A\cup B) = P(A) + P(B) - P(A\cap B) = \dfrac{12}{52} + \dfrac{4}{52} - \dfrac{2}{52} = \dfrac{14}{52} = \dfrac{7}{26}.
              \]
        \item
              \[
                  P(C\cup D) = 1.
              \]
        \item
              \[
                  P(C\cap D) = 0.
              \]
    \end{enumerate}
\end{proof}

\begin{exercise}
    \par A fair coin is tossed four times, and the sequence of heads and tails is observed.
    \begin{enumerate}[label = \textbf{(\alph*)}]
        \item List each of the 16 sequences in the sample space $S$.
        \item Let events $A$, $B$, $C$, and $D$ be given by $A = \{ \text{at least
                      3 heads} \}$, $B = \{\text{at most 2 heads}\}$, $C = \{\text{heads on the third toss}\}$, and $D = \{\text{1 head and 3 tails}\}$. If the probability set function assigns 1/16 to each outcome in the sample space, find:
              \begin{enumerate}[label = \textbf{(\arabic*)}]
                  \item $P(A)$
                  \item $P(A\cap B)$
                  \item $P(B)$
                  \item $P(A\cap C)$
                  \item $P(D)$
                  \item $P(A\cup C)$
                  \item $P(B\cup D)$
              \end{enumerate}
    \end{enumerate}
\end{exercise}

\begin{proof}[Solution]
    \begin{enumerate}[label = \textbf{(\alph*)}]
        \item
              \[
                  \begin{matrix}
                      HHHH & HHHT & HHTH & HHTT \\
                      HTHH & HTHT & HTTH & HTTT \\
                      THHH & THHT & THTH & THTT \\
                      TTHH & TTHT & TTTH & TTTT
                  \end{matrix}
              \]
        \item
              \begin{enumerate}[label = \textbf{(\arabic*)}]
                  \item
                        \[ P(A) = \dfrac{5}{16} \]
                  \item
                        \[ P(A\cap B) = 0 \]
                  \item
                        \[ P(B) = \dfrac{6 + 4 + 1}{16} = \dfrac{11}{16} \]
                  \item
                        \[ P(A\cap C) = \dfrac{4}{16} = \dfrac{1}{4} \]
                  \item
                        \[ P(D) = \dfrac{4}{16} = \dfrac{1}{4} \]
                  \item
                        \[ P(A\cup C) = P(A) + P(C) - P(A\cap C) = \dfrac{5}{16} + \dfrac{8}{16} - \dfrac{1}{4} = \dfrac{9}{16} \]
                  \item
                        \[ P(B\cap D) = P(D) = \dfrac{1}{4} \]
              \end{enumerate}
    \end{enumerate}
\end{proof}

\begin{exercise}
    \par Consider the trial on which a 3 is first observed in successive rolls of a six-sided die.
    \par Let A be the event that 3 is observed on the first trial.
    \par Let B be the event that at least two trials are required to observe a 3.
    \par Assuming that each side has probability 1/6, find:
    \par \textbf{(a)} $P(A)$, \textbf{(b)} $P(B)$, và \textbf{(c)} $P(A\cup B)$.
\end{exercise}

\begin{proof}[Solution]
    \begin{enumerate}[label = \textbf{(\alph*)}]
        \item
              \[ P(A) = \dfrac{1}{6} \]
        \item
              \[
                  % chktex-file 3
                  P(B) = \left(\dfrac{5}{6}\right)^{2}\dfrac{1}{6} = \dfrac{25}{216}
              \]
        \item
              \[
                  P(A\cup B) = P(A) + P(B) = \dfrac{1}{6} + \dfrac{25}{216} = \dfrac{61}{216}
              \]
    \end{enumerate}
\end{proof}

\begin{exercise}
    \par If $P(A) = 0.4$, $P(B) = 0.5$, and $P(A\cap B) = 0.3$, find \textbf{(a)} $P(A\cup B)$, \textbf{(b)} $P(A\cap B')$, \textbf{(c)} $P(A'\cup B')$.
\end{exercise}

\begin{proof}[Solution]
    \begin{enumerate}[label = \textbf{(\alph*)}]
        \item
              \[ P(A\cup B) = P(A) + P(B) - P(A\cap B) = 0.6. \]
        \item
              \[ P(A\cap B') = P(A) - P(A\cap B) = 0.1. \]
        \item
              \[ P(A'\cup B') = 1 - P(A\cap B) = 0.7. \]
    \end{enumerate}
\end{proof}

\begin{exercise}
    \par Given that $P(A\cup B) = 0.76$ and $P(A\cup B') = 0.87$, find $P(A)$.
\end{exercise}

\begin{proof}[Solution]
    \[ P(A\cup B) = P(A) + P(B) - P(A\cap B) \]
    \[ P(A\cup B') = P(A) + P(B') - P(A\cap B') \]
    \begin{align*}
        P(A\cup B) + P(A\cup B') & = 2P(A) + P(B) + P(B') - P(A\cap B) - P(A\cap B') \\
        0.76 + 0.87              & = 2P(A) + 1 - P(A)                                \\
        0.63                     & = P(A)
    \end{align*}
\end{proof}

\begin{exercise}
    \par During a visit to a primary care physician’s office, the probability of having neither lab work nor referral to a
    specialist is 0.21.
    \par Of those coming to that office, the probability of having lab work is 0.41 and the probability of
    having a referral is 0.53.
    \par What is the probability of having both lab work and a referral?
\end{exercise}

\begin{proof}[Solution]
    \par $A$ is the event \textit{having lab work}.
    \par $B$ is the event \textit{having a referral}.
    \par According to assumption:
    \[
        \begin{cases}
            P(A'\cap B') = 0.21, \\
            P(A) = 0.41,         \\
            P(B) = 0.53.
        \end{cases}
    \]
    \par $P(A'\cap B') = 1 - P(A\cup B) = 1 - P(A) - P(B) + P(A\cap B)$.
    \par Therefore, $P(A\cap B) = P(A'\cap B') + P(A) + P(B) - 1 = 0.15$.
    \par Hence the probability of having both lab work and a referral is 15\%.
\end{proof}

\begin{exercise}
    \par Roll a fair six-sided die three times.
    \par Let $A_{1} = \{\text{1 or 2 on the first roll}\}$,
    \par $A_{2} = \{\text{3 or 4 on the second roll}\}$,
    \par and $A_{3} = \{\text{5 or 6 on the third roll}\}$.
    \par It is given that $P(A_{i}) = 1/3, \; i = 1,2,3$; $P(A_{i}\cap A_{j}) = (1/3)^{2},\; i\ne j$; and $P(A_{1}\cap A_{2}\cap A_{3}) = (1/3)^{3}$.
    \begin{enumerate}[label = \textbf{(\alph*)}]
        \item Find $P(A_{1}\cup A_{2}\cup A_{3})$.
        \item Show that $P(A_{1}\cup A_{2}\cup A_{3}) = 1 - (1 - 1/3)^{3}$.
    \end{enumerate}
\end{exercise}

\begin{proof}[Solution]
    \begin{enumerate}[label = \textbf{(\alph*)}]
        \item
        \begin{align*}
            P(A_{1}\cup A_{2}\cup A_{3}) & = P(A_{1}) + P(A_{2}) + P(A_{3}) - P(A_{1}\cap A_{2}) - P(A_{2}\cap A_{3}) - P(A_{3}\cap A_{1}) + P(A_{1}\cap A_{2}\cap A_{3}) \\
                                         & = \frac{1}{3} + \frac{1}{3} + \frac{1}{3} - \frac{1}{9} - \frac{1}{9} - \frac{1}{9} + \frac{1}{27} \\
                                         & = 1 - \frac{1}{3} + \frac{1}{27} \\
                                         & = \frac{19}{27}.
        \end{align*}
        \item $P(A_{1}\cup A_{2}\cup A_{3}) = 1 - \frac{8}{27} = 1 - \left(1 - \frac{1}{3}\right)^{3}$.
    \end{enumerate}
\end{proof}

\begin{exercise}
    \par If $A$, $B$, and $C$ are any events, prove that:
    \[
        P(A\cup B\cup C) = P(A) + P(B) + P(C) - P(A\cap B) - P(A\cap C) - P(B\cap C) + P(A\cap B\cap C).
    \]
\end{exercise}

\begin{proof}
    \begin{align*}
        P(A\cup B\cup C) & = P((A\cup B)\cup C) = P(A\cup B) + P(C) - P((A\cup B)\cap C) \\
                         & = P(A) + P(B) - P(A\cap B) + P(C) - P((A\cap C)\cup (B\cap C)) \\
                         & = P(A) + P(B) + P(C) - P(A\cap B) - P(A\cap C) - P(B\cap C) + P(A\cap C\cap B\cap C) \\
                         & = P(A) + P(B) + P(C) - P(A\cap B) - P(A\cap C) - P(B\cap C) + P(A\cap B\cap C).
    \end{align*}
\end{proof}

\begin{exercise}
    \par A typical roulette wheel used in a casino has 38 slots that are numbered 1, 2, 3, \ldots 36, 0, 00, respectively.
    \par The 0 and 00 slots are colored green. Half of the remaining slots are red and half are black.
    \par Also, half of the integers between 1 and 36 inclusive are odd, half are even, and 0 and 00 are defined to be neither odd nor even.
    \par A ball is rolled around the wheel and ends up in one of the slots; we assume that each slot has equal probability of 1/38, and we are interested in the number of the slot into which the ball falls.
    \begin{enumerate}[label = \textbf{(\alph*)}]
        \item Define the sample space $S$.
        \item Let $A = \{0, 00\}$. Give the value of $P(A)$.
        \item Let $B = \{ 14, 15, 17, 18 \}$. Given the value of $P(B)$.
        \item Let $D = \{ x : \text{ $x$ is odd } \}$. Give the value of $P(D)$.
    \end{enumerate}
\end{exercise}

\begin{proof}[Solution]
    \begin{enumerate}[label = \textbf{(\alph*)}]
        \item $S = \{ 1, 2, 3, \ldots 36, 0, 00 \}$.
        \item $P(A) = \frac{2}{38} = \frac{1}{19}$.
        \item $P(B) = \frac{4}{38} = \frac{2}{19}$.
        \item $P(D) = \frac{18}{38} = \frac{9}{19}$.
    \end{enumerate}
\end{proof}

\begin{exercise}
    \par Let $x$ equal a number that is selected randomly from the closed interval from zero to one, $[0, 1]$.
    \par Use your intuition to assign values to
    \begin{enumerate}[label = \textbf{(\alph*)}]
        \item $P(\{ x: 0\le x\le 1/3 \})$.
        \item $P(\{ x: 1/3\le x\le 1 \})$.
        \item $P(\{ x: x = 1/3 \})$.
        \item $P(\{ x: 1/2 < x < 5 \})$.
    \end{enumerate}
\end{exercise}

\begin{proof}[Solution]
    \begin{enumerate}[label = \textbf{(\alph*)}]
        \item $P(\{ x: 0\le x\le 1/3 \}) = \frac{1}{3}$.
        \item $P(\{ x: 1/3\le x\le 1 \}) = \frac{2}{3}$.
        \item $P(\{ x: x = 1/3 \}) = 0$.
        \item $P(\{ x: 1/2 < x < 5 \}) = \frac{1}{2}$.
    \end{enumerate}
\end{proof}

\begin{exercise}
    \par Divide a line segment into two parts by selecting a point at random. Use your intuition to assign a probability to the event that the longer segment is at least two times longer than the shorter segment.
\end{exercise}

\begin{proof}[Solution]
    \par Let the line segment be $AB$ and $M$ be a randomly selected point on it.
    \par Without loss of generality, suppose that $MA > MB$.
    \par Let $C$ be a point on $AB$ such that $CA = 2CB$.
    \par If $M\in CA\setminus\{ C \}$, then $\frac{MA}{MB} < \frac{CA}{CB} = 2$.
    \par If $M\in CB$, then $\frac{MA}{MB}\ge \frac{CA}{CB} = 2$.
    \par Hence $\frac{MA}{MB}\ge 2$ iff $M\in BC$.
    \par So the probability of selecting $M$ on $AB$ such that $\frac{MA}{MB}\ge 2$ is $\frac{1}{3}$.
\end{proof}

\begin{exercise}
    \par Let the interval $[-r, r]$ be the base of a semicircle.
    \par If a point is selected at random from this interval, assign a probability to the event that the length of the perpendicular segment from the point to the semicircle is less than $r/2$.
\end{exercise}

\begin{proof}[Solution]
    \par On Cartesian, suppose that the semicircle has equation:
    \[
        x^{2} + y^{2} = r^{2}\quad (y \ge 0)
    \]
    \par Let $A = (-r, 0)$ and $B = (r, 0)$.
    \par Let $C = (r\cos\varphi, r\sin\varphi)$, where $\varphi\in[0,\pi]$ and $CM\perp AB$.
    \par $CM < \frac{r}{2}$ iff $\sin\varphi < \frac{1}{2}$.
    \[
        \begin{cases}
            \sin\varphi < \frac{1}{2} \\
            0\le \varphi\le \pi
        \end{cases}
        \Leftrightarrow
        0\le\varphi < \frac{\pi}{6} \text{ or } \frac{5\pi}{6} < \varphi \le\pi
        \Leftrightarrow
        \overline{OM} > \frac{r\sqrt{3}}{2} \text{ or } \overline{OM} < -\frac{r\sqrt{3}}{2}.
    \]
    \par Hence the probability of selecting $M$ such that $CM < \frac{r}{2}$ is $\frac{2r - r\sqrt{3}}{2r} = \frac{2 - \sqrt{3}}{2}$.
\end{proof}

\begin{exercise}
    \par Let $S = A_{1} \cup A_{2} \cup \cdots \cup A_{m}$, where events $A_{1}, A_{2}, \ldots, A_{m}$ are mutually exclusive and exhaustive.
    \begin{enumerate}[label = \textbf{(\alph*)}]
        \item If $P(A_{1}) = P(A_{2}) = \cdots = P(A_{m})$, show that $P(A_{i}) = 1/m$, $i = 1, 2, \ldots, m$.
        \item If $A = A_{1}\cup A_{2}\cup \cdots\cup A_{h}$, where $h < m$ and (a) holds, prove that $P(A) = h/m$.
    \end{enumerate}
\end{exercise}

\begin{proof}
    \begin{enumerate}[label = \textbf{(\alph*)}]
        \item Since $A_{1}$, $A_{2}$, \ldots $A_{m}$ are mutually exclusive and exhaustive, then $\sum^{m}_{i=1}P(A_{i}) = 1$.
            \par $P(A_{1}) = P(A_{2}) = \cdots = P(A_{m}) \Longrightarrow P(A_{1}) = P(A_{2}) = \cdots = P(A_{m}) = \frac{1}{m}$.
        \item Since $A_{1}$, $A_{2}$, \ldots $A_{h}$ are mutually exclusive, $P(A) = \sum^{h}_{i=1}P(A_{i}) = \frac[h]{m}$.
    \end{enumerate}
\end{proof}

\begin{exercise}
    \par Let $p_{n}$, $n = 0, 1, 2,\ldots$, be the probability that an automobile policyholder will file for $n$ claims in five-year period. The actuary involved makes the assumption that $p_{n+1} = (1/4)p_{n}$. What is the probability that the holder will file two or more claims during this period?
\end{exercise}

\begin{proof}[Solution]
    \begin{gather*}
        1 = \sum^{+\infty}_{n=0}p_{n}
        \Leftrightarrow 1 = \sum^{+\infty}_{n=0}p_{0}\frac{1}{4^{n}}
        \Leftrightarrow 1 = p_{0}\frac{1}{1 - \dfrac{1}{4}}
        \Leftrightarrow 1 = p_{0}\times\frac{4}{3}
        \Leftrightarrow p_{0} = \frac{3}{4}
    \end{gather*}
    \par Hence the probability that holder will file at least two claims during this period is $1 - p_{0} - p_{1} = 1 - \frac{3}{4} - \frac{3}{16} = \frac{1}{16}$.
\end{proof}


\section{Methods of Enumeration}



\section{Conditional Probability}



\section{Independent Events}



\section{Bayes's Theorem}


\end{document}
