\documentclass[class=probandstats,crop=false]{standalone}

\begin{document}

\chapter{Xác suất}

\section{Các tính chất của xác suất}

\begin{exercise}
    \par Trong một nhóm các bệnh nhân bị thương, 28\% đến gặp bác sĩ trị liệu vật lý và chấn thương chỉnh hình, 8\% không đến gặp ai. Biết rằng xác suất đến gặp bác sĩ trị liệu vật lý cao hơn xác suất đến gặp bác sĩ chấn thương chỉnh hình là 16\%.
    \par Xác suất một người được chọn ngẫu nhiên đến gặp bác sĩ trị liệu vật lý là bao nhiêu?
\end{exercise}

\begin{proof}[Lời giải]
    \par $A$ là sự kiện \textit{bệnh nhân gặp bác sĩ trị liệu vật lý}.
    \par $B$ là sự kiện \textit{bệnh nhân gặp bác sĩ chấn thương chỉnh hình}.
    \bigskip
    \par Theo giả thiết
    \[
        \begin{cases}
            P(A'\cap B') = 0.08 \\
            P(A\cap B) = 0.28
        \end{cases}
    \]
    \par $P(A'\cap B') = 1 - P(A\cup B) = 1 - P(A) - P(B) + P(A\cap B)$.
    \par Suy ra $P(A) + P(B) = 1 - P(A'\cap B') + P(A\cap B) = 1 - 0.08 + 0.28 = 0.8$.
    \par Cùng với việc $P(A) - P(B) = 0.16$, ta thu được $P(A) = 0.48$.
    \par Vậy xác suất một người được chọn ngẫu nhiên đến gặp bác sĩ trị liệu vật lý là 48\%.
\end{proof}

\begin{exercise}
    \par Một công ty bảo hiểm quan sát các khách hàng bảo hiểm xe và nhận ra
    \begin{enumerate}[label = (\alph*)]
        \item tất cả bảo hiểm ít nhất một xe
        \item 85\% bảo hiểm nhiều hơn một xe
        \item 23\% bảo hiểm một xe thể thao
        \item 17\% bảo hiểm nhiều hơn một xe, bao gồm một xe thể thao
    \end{enumerate}
    \par Tìm xác suất một khách hàng được chọn ngẫu nhiên, bảo hiểm đúng một chiếc xe và không phải xe thể thao.
\end{exercise}

\begin{proof}[Lời giải]
    \par $A$ là sự kiện \textit{bảo hiểm đúng một xe}.
    \par $B$ là sự kiện \textit{bảo hiểm xe thể thao}.
    \par $C$ là sự kiện \textit{bảo hiểm nhiều hơn một xe}.
    \par Theo giả thiết
    \[
        \begin{cases}
            P(A) + P(C) = 1 \\
            P(B) = 0.23     \\
            P(C) = 0.85     \\
            P(B\cap C) = 0.17
        \end{cases}
    \]
    \par Sự kiện \textit{bảo hiểm đúng một xe và không phải xe thể thao} được biểu thị bằng tập hợp $A\cap C'$
    \begin{align*}
        P(A\cap C') & = P(B'\cap C')                 \\
                    & = 1 - P(B\cup C)               \\
                    & = 1 - P(B) - P(C) + P(B\cap C) \\
                    & = 1 - 0.23 - 0.85 + 0.17       \\
                    & = 0.09.
    \end{align*}
    \par Vậy xác suất để một khách hàng được chọn ngẫu nhiên, bảo hiểm đúng một chiếc xe và không phải xe thể thao là 9\%.
\end{proof}

\begin{exercise}
    \par Chọn một tấm bài ngẫu nhiên từ một bộ bài chuẩn. Không gian mẫu $S$ là một bộ gồm 52 lá bài.
    \par Giả sử hàm xác suất gán giá trị 1/52 cho từng khả năng. Đặt
    \begin{align*}
        A & = \{ x: J, Q, K \}                 \\
        B & = \{ x: 9, 10, \text{hoặc J đỏ} \} \\
        C & = \{ x: \text{nhép} \}             \\
        D & = \{ x: \text{rô, cơ, bích} \}
    \end{align*}
    \par Tính \textbf{(a)} $P(A)$, \textbf{(b)} $P(A\cap B)$, \textbf{(c)} $P(A\cup B)$, \textbf{(d)} $P(C\cup D)$, \textbf{(e)} $P(C\cap D)$.
\end{exercise}

\begin{proof}[Lời giải]
    \begin{enumerate}[label = \textbf{(\alph*)}]
        \item
              \[
                  P(A) = \dfrac{3 * 4}{52} = \dfrac{3}{13}.
              \]
        \item
              \[
                  P(A\cap B) = \dfrac{2}{52} = \dfrac{1}{26}.
              \]
        \item
              \[
                  P(A\cup B) = P(A) + P(B) - P(A\cap B) = \dfrac{12}{52} + \dfrac{4}{52} - \dfrac{2}{52} = \dfrac{14}{52} = \dfrac{7}{26}.
              \]
        \item
              \[
                  P(C\cup D) = 1.
              \]
        \item
              \[
                  P(C\cap D) = 0.
              \]
    \end{enumerate}
\end{proof}

\begin{exercise}
    \par Một đồng xu cân được tung bốn lần, thứ tự ngửa và sấp được quan sát.
    \begin{enumerate}[label = \textbf{(\alph*)}]
        \item Liệt kê từng thứ tự có thể trong không gian mẫu $S$.
        \item $A$ là sự kiện \textit{có ít nhất 3 lần ngửa}, $B$ là sự kiện \textit{có nhiều nhất 2 lần ngửa}, $C$ là sự kiện \textit{ngửa trong lần tung thứ ba} và $D$ là sự kiện \textit{có 1 lần ngửa và 3 lần sấp}. Nếu hàm xác suất gán cho mỗi kết quả trong không gian mẫu giá trị là $1/16$, tìm
              \begin{enumerate}[label = \textbf{(\arabic*)}]
                  \item $P(A)$
                  \item $P(A\cap B)$
                  \item $P(B)$
                  \item $P(A\cap C)$
                  \item $P(D)$
                  \item $P(A\cup C)$
                  \item $P(B\cup D)$
              \end{enumerate}
    \end{enumerate}
\end{exercise}

\begin{proof}[Lời giải]
    \begin{enumerate}[label = \textbf{(\alph*)}]
        \item
              \[
                  \begin{matrix}
                      HHHH & HHHT & HHTH & HHTT \\
                      HTHH & HTHT & HTTH & HTTT \\
                      THHH & THHT & THTH & THTT \\
                      TTHH & TTHT & TTTH & TTTT
                  \end{matrix}
              \]
        \item
              \begin{enumerate}[label = \textbf{(\arabic*)}]
                  \item
                        \[ P(A) = \dfrac{5}{16} \]
                  \item
                        \[ P(A\cap B) = 0 \]
                  \item
                        \[ P(B) = \dfrac{6 + 4 + 1}{16} = \dfrac{11}{16} \]
                  \item
                        \[ P(A\cap C) = \dfrac{4}{16} = \dfrac{1}{4} \]
                  \item
                        \[ P(D) = \dfrac{4}{16} = \dfrac{1}{4} \]
                  \item
                        \[ P(A\cup C) = P(A) + P(C) - P(A\cap C) = \dfrac{5}{16} + \dfrac{8}{16} - \dfrac{1}{4} = \dfrac{9}{16} \]
                  \item
                        \[ P(B\cap D) = P(D) = \dfrac{1}{4} \]
              \end{enumerate}
    \end{enumerate}
\end{proof}

\begin{exercise}
    \par Xét phép thử mà trong đó, mặt 3 xuất hiện khi tung một súc sắc 6-mặt thành công.
    \par $A$ là sự kiện 3 xuất hiện trong lần thử đầu tiên.
    \par $B$ là sự kiện 3 xuất hiện sau ít nhất 2 lần thử.
    \par Giả sử mỗi mặt có xác suất là 1/6, tìm \textbf{(a)} $P(A)$, \textbf{(b)} $P(B)$, và \textbf{(c)} $P(A\cup B)$.
\end{exercise}

\begin{proof}[Lời giải]
    \begin{enumerate}[label = \textbf{(\alph*)}]
        \item
              \[ P(A) = \dfrac{1}{6} \]
        \item
              \[
                  % chktex-file 3
                  P(B) = \left(\dfrac{5}{6}\right)^{2}\dfrac{1}{6} = \dfrac{25}{216}
              \]
        \item
              \[
                  P(A\cup B) = P(A) + P(B) = \dfrac{1}{6} + \dfrac{25}{216} = \dfrac{61}{216}
              \]
    \end{enumerate}
\end{proof}

\begin{exercise}
    \par Nếu $P(A) = 0.4$, $P(B) = 0.5$, và $P(A\cap B) = 0.3$, tìm \textbf{(a)} $P(A\cup B)$, \textbf{(b)} $P(A\cap B')$, \textbf{(c)} $P(A'\cup B')$.
\end{exercise}

\begin{proof}[Lời giải]
    \begin{enumerate}[label = \textbf{(\alph*)}]
        \item
              \[ P(A\cup B) = P(A) + P(B) - P(A\cap B) = 0.6. \]
        \item
              \[ P(A\cap B') = P(A) - P(A\cap B) = 0.1. \]
        \item
              \[ P(A'\cup B') = 1 - P(A\cap B) = 0.7. \]
    \end{enumerate}
\end{proof}

\begin{exercise}
    \par Biết $P(A\cup B) = 0.76$ và $P(A\cup B') = 0.87$, tìm $P(A)$.
\end{exercise}

\begin{proof}[Lời giải]
    \[ P(A\cup B) = P(A) + P(B) - P(A\cap B) \]
    \[ P(A\cup B') = P(A) + P(B') - P(A\cap B') \]
    \begin{align*}
        P(A\cup B) + P(A\cup B') & = 2P(A) + P(B) + P(B') - P(A\cap B) - P(A\cap B') \\
        0.76 + 0.87              & = 2P(A) + 1 - P(A)                                \\
        0.63                     & = P(A)
    \end{align*}
\end{proof}

\section{Các phương pháp liệt kê}



\section{Xác suất có điều kiện}



\section{Sự kiện độc lập}



\section{Định lý Bayes}


\end{document}
