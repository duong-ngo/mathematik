\documentclass[oneside]{book}
\usepackage[left=2cm,right=2cm,top=2.5cm,bottom=2.5cm]{geometry}
\usepackage[unicode=true,hidelinks]{hyperref}

\usepackage{amsmath}
\usepackage{amsfonts}
\usepackage{amssymb}
\usepackage{amsthm}
\usepackage{amscdx}
\usepackage{mathtools}
\usepackage{mathrsfs}
\usepackage{cases}

\usepackage{fancyhdr}
\usepackage{xcolor}
\usepackage{titlesec}
\usepackage{indentfirst}
\usepackage{chngcntr}
\usepackage{caption}
\usepackage{subcaption}
\usepackage{booktabs}
\usepackage[inline]{enumitem}
\usepackage{pgf,tikz}
\usetikzlibrary{shapes}
\usetikzlibrary{shapes.arrows}
\usetikzlibrary{arrows.meta}
\usetikzlibrary{calc}
\usetikzlibrary{math}
\usetikzlibrary{decorations,
    decorations.footprints,
    decorations.fractals,
    decorations.markings,
    decorations.pathmorphing,
    decorations.pathreplacing,
    decorations.shapes,
    decorations.shapes,
    decorations.text}
\usetikzlibrary{positioning}
\usetikzlibrary{angles}
\usetikzlibrary{matrix}
\usepackage{tikz-cd}

\setcounter{chapter}{0}
\setcounter{section}{-1}
\counterwithout{section}{chapter}

\pagestyle{fancy}
\setlength{\headheight}{16pt}
\lhead{}
\rhead{\leftmark}
\lfoot{}
\cfoot{\thepage}
\rfoot{}

\titleformat{\chapter}[display]{\flushright\bf\huge}{\chaptertitlename\,\thechapter}{10pt}{}
\titleformat{\section}{\bf\Large}{\thesection}{10pt}{}
\titleformat{\subsection}{\bf\large}{\thesubsection}{10pt}{}
\titleformat{\subsubsection}{\bf\large}{\thesubsubsection}{10pt}{}

\captionsetup{labelfont={bf},labelsep=period}
\counterwithin{figure}{chapter}
\counterwithin{table}{chapter}

\theoremstyle{definition}
\newtheorem{note}{Note}
\counterwithin{note}{chapter}

\newtheorem{theorem}{Theorem}
\counterwithin{theorem}{chapter}
\newtheorem{definition}[theorem]{Definition}
\newtheorem{example}[theorem]{Example}
\newtheorem{lemma}[theorem]{Lemma}

\newtheorem{exercise}{Exercise}
\counterwithin{exercise}{section}
\newenvironment{sqcases}{%
    \matrix@check\sqcases\env@sqcases
}{%
    \endarray\right.%
}
\def\env@sqcases{%
\let\@ifnextchar\new@ifnextchar
\left\lbrack{}
\def\arraystretch{1.2}%
\array{@{}l@{\quad}l@{}}%
}

\newcommand{\tr}[1]{\left({#1}\right)}
\newcommand{\card}[1]{\left\vert{#1}\right\vert}
\newcommand{\rank}{\text{rank}}
\newcommand{\abs}[1]{\left\vert{#1}\right\vert}
\newcommand{\norm}[1]{\left\Vert{#1}\right\Vert}
\newcommand{\anglebracket}[1]{\left\langle{#1}\right\rangle}
\newcommand{\innerprod}[2]{\left\langle{#1}\vert{#2}\rangle}

\title{Fraleigh and Brand's A First Course in Abstract Algebra: Exercises}
\author{Ngo Quang Duong}
\date{\today}

\begin{document}

\maketitle

\tableofcontents

\documentclass[class=linearalgebra,crop=false]{standalone}

\begin{document}

\chapter{Kiến thức chuẩn bị}

\section*{Bài tập}

\setcounter{exercise}{0}

\begin{exercise}Chứng minh các tính chất kết hợp, giao hoán, phân phối của các phép toán hợp và giao trên tập hợp. Chứng minh công thức đối ngẫu De Morgan cho hiệu của hợp và giao của một họ tùy ý các tập hợp.
\end{exercise}

\begin{proof}Ta sử dụng các tập hợp $A, B, C$. Nhắc lại rằng, trên tập hợp các mệnh đề, phép toán $\text{and}$, $\text{or}$ có tính chất kết hợp, giao hoán và phân phối.

    \par \textit{Tính chất kết hợp}.

    \begin{gather*}
        (A\cup B)\cup C = \{ x\ |\ (x\in A \text{ or } x\in B) \text{ or } x\in C \} \\
        A\cup (B\cup C) = \{ x\ |\ x\in A \text{ or } (x\in B \text{ or } x\in C) \}
    \end{gather*}

    \par Toán tử $\text{or}$ có tính chất kết hợp, do đó $(A\cup B)\cup C = A\cup (B\cup C)$.

    \begin{gather*}
        (A\cap B)\cap C = \{ x\ |\ (x\in A \text{ and } x\in B) \text{ and } x\in C \} \\
        A\cap (B\cap C) = \{ x\ |\ x\in A \text{ and } (x\in B \text{ and } x\in C) \}
    \end{gather*}

    \par Toán tử $\text{and}$ có tính chất kết hợp, do đó $(A\cap B)\cap C = A\cap (B\cap C)$.

    \bigskip

    \par \textit{Tính chất giao hoán}.

    \begin{gather*}
        A\cup B = \{ x\ |\ x\in A \text{ or } x\in B \} \\
        B\cup A = \{ x\ |\ x\in B \text{ or } x\in A \}
    \end{gather*}

    \par Toán tử $\text{or}$ có tính chất giao hoán, do đó $A\cup B = B\cup A$.

    \begin{gather*}
        A\cap B = \{ x\ |\ x\in A \text{ and } x\in B \} \\
        B\cap A = \{ x\ |\ x\in B \text{ and } x\in A \}
    \end{gather*}

    \par Toán tử  $\text{and}$ có tính chất giao hoán, do đó $A\cap B = B\cap A$.

    \bigskip

    \par \textit{Tính chất phân phối}.

    \begin{align*}
        A\cap (B\cup C) &= \{ x\ |\ x\in A \text{ and } (x\in B \text{ or } x\in C) \} \\
                        &= \{ x\ |\ (x\in A \text{ and } x\in B) \text{ or } (x\in A \text{ and } x\in C) \} \\
                        & \text{(do phép $\text{and}, \text{or}$ có tính chất phân phối)} \\
                        &= (A\cap B)\cup (A\cap C)
    \end{align*}

    \begin{align*}
        A\cup (B\cap C) &= \{ x\ |\ x\in A \text{ or } (x\in B \text{ and } x\in C) \} \\
                        &= \{ x\ |\ (x\in A \text{ or } x\in B) \text{ and } (x\in A \text{ or } x\in C) \} \\
                        & \text{(do phép $\text{and}, \text{or}$ có tính chất phân phối)} \\
                        &= (A\cup B)\cap (A\cup C)
    \end{align*}

    \par Trong chứng minh công thức De Morgan, chúng ta sẽ sử dụng họ tập hợp $A_{i}$ với $i\in I$ nào đó.

    \begin{align*}
        X\setminus\bigcup_{i\in I} A_{i} &= \{ x\ |\ \overline{\exists i\in I, x\in A_{i}} \} \\
                                         &= \{ x\ |\ \forall i\in I, x\not\in A_{i} \} \\
                                         &= \{ x\ |\ \forall i\in I, x\in (X\setminus A_{i}) \} \\
                                         &= \bigcap_{i\in I}(X\setminus A_{i})
    \end{align*}

    \begin{align*}
        X\setminus\bigcap_{i\in I} A_{i} &= \{ x\ |\ \overline{\forall i\in I, x\in A_{i}} \} \\
                                         &= \{ x\ |\ \exists i\in I, x\not\in A_{i} \} \\
                                         &= \{ x\ |\ \exists i\in I, x\in (X\setminus A_{i}) \} \\
                                         &= \bigcup_{i\in I}(X\setminus A_{i})
    \end{align*}

\end{proof}

\begin{exercise}Chứng minh rằng
    \begin{enumerate}[itemsep=0pt,label = (\alph*)]
        \item $(A\setminus B) \cup (B\setminus A) = \emptyset \Longleftrightarrow A = B$,
        \item $A = (A\setminus B)\cup (A\cap B)$,
        \item $(A\setminus B) \cup (B\setminus A) = (A\cup B)\setminus (A\cap B)$,
        \item $A\cap (B\setminus C) = (A\cap B) \setminus (A\cap C)$,
        \item $A\cup (B\setminus A) = A\cup B$,
        \item $A\setminus (A\setminus B) = A\cap B$
    \end{enumerate}
\end{exercise}

\begin{proof}
    \begin{enumerate}[label = (\alph*)]
        \item Nếu $A = B$ thì $A\setminus B = B\setminus A = \emptyset$

        \par $\Rightarrow (A\setminus B)\cup (B\setminus A) = \emptyset$

        \par Nếu $(A\setminus B)\cup (B\setminus A) = \emptyset$ thì $A\setminus B = B\setminus A = \emptyset$

        \par $\Rightarrow A\subset B, B\subset A\Rightarrow A = B$.

        \item Đẳng thức đúng nếu $A$ là tập rỗng.

        \par Ngược lại, $A$ không rỗng, chọn $x$ là một phần tử của $A$.

        \par Có hai khả năng, $x\in B$ hoặc $x\not\in B$.

        \begin{align*}
            A &= \{ x\ |\ (x\in A \text{ and } x\in B)\text{ or }(x\in A \text{ and } x\not\in B) \} \\
              &= (A\cap B) \cup (A\setminus B)
        \end{align*}

        \item $a(x)$ là mệnh đề $x \in A$, $b(x)$ là mệnh đề $x\in B$.

        \[
            (A\setminus B)\cup (B\setminus A)=\{ x\ |\ (a(x) \text{ and } \overline{b(x)}) \text{ or } (\overline{a(x)} \text{ and } b(x)) \}
        \]

        \begin{align*}
            (a(x) \wedge \overline{b(x)}) \vee (\overline{a(x)} \wedge b(x)) &= ((a(x)\wedge \overline{b(x)})\vee \overline{a(x)}) \wedge ((a(x)\wedge \overline{b(x)})\vee b(x)) \\
            &= ((a(x)\vee \overline{a(x)}) \wedge (\overline{b(x)}\vee\overline{a(x)}))\wedge \\
            &\quad ((a(x)\vee b(x))\wedge (b(x)\vee \overline{b(x)})) \\
            &= (\overline{a(x)}\vee\overline{b(x)}) \wedge (a(x) \vee b(x)) \\
            &= \overline{a(x)\wedge b(x)} \wedge (a(x)\vee b(x)) \\
            &= (x\in A\cup B) \wedge (x\not\in A\cap B)
        \end{align*}

        $\Rightarrow (A\setminus B)\cup (B\setminus A) = (A\cup B)\setminus (A\cap B)$

        \item $a(x)$ là mệnh đề $x \in A$, $b(x)$ là mệnh đề $x\in B$, $c(x)$ là mệnh đề $x\in C$.

        \begin{align*}
            A\cap (B\setminus C) &= \{ x\ |\ a(x) \wedge (b(x) \wedge \overline{c(x)}) \} \\
                                 &= \{ x\ |\ (a(x) \wedge b(x)) \wedge (a(x) \wedge\overline{c(x)}) \} \\
                                 &= (A\cap B) \cap (A\setminus C) \\
                                 &= (A\cap B) \cap (A\setminus (A\cap C)) \\
                                 &= (A\cap B) \cap A \cap (X \setminus (A\cap C)) \\
                                 &= (A\cap B) \cap (X \setminus (A\cap C)) \\
                                 &= (A\cap B) \setminus (A\cap C)
        \end{align*}

        \item

        \begin{align*}
            A\cup (B\setminus A) &= \{ x\ |\ a(x) \vee (b(x) \wedge \overline{a(x)}) \} \\
                                 &= \{ x\ |\ (a(x) \vee b(x)) \wedge (a(x) \vee \overline{a(x)}) \} \\
                                 &= \{ x\ |\ a(x)\vee b(x) \} \\
                                 &= A\cup B
        \end{align*}

        \item

        \begin{align*}
            A\setminus (A\setminus B)&= \{ x\ |\ a(x) \wedge \overline{a(x)\wedge \overline{b(x)}} \} \\
                                     &= \{ x\ |\ a(x) \wedge (\overline{a(x)} \vee b(x)) \} \\
                                     &= \{ x\ |\ (a(x) \wedge \overline{a(x)}) \vee (a(x) \wedge b(x)) \} \\
                                     &= \{ x\ |\ a(x) \wedge b(x) \} \\
                                     &= A\cap B
        \end{align*}
    \end{enumerate}
\end{proof}

\end{document}

\chapter{First Examples}

\section{The Simplest Examples}

\section{Linear Systems with Constant Coefficients}

\chapter{Smooth Maps}

\section{Smooth Functions and Smooth Maps}

\section{Partitions of Unity}

\documentclass[class=linearalgebra,crop=false]{standalone}

\newcommand{\sgn}[1]{\text{sgn}\left({#1}\right)}
\setcounter{lemma}{0}

\begin{document}

\chapter{Định thức và hệ phương trình tuyến tính}

\par Thực hiện các phép nhân sau đây, viết các phép thế thu được thành tích của những xích rời rạc và tính dấu của chúng.

\begin{exercise}
    $
        \begin{pmatrix}
            1 & 2 & 3 & 4 & 5 \\
            2 & 4 & 5 & 1 & 3
        \end{pmatrix}
        \begin{pmatrix}
            1 & 2 & 3 & 4 & 5 \\
            4 & 3 & 5 & 1 & 2
        \end{pmatrix}
    $.
\end{exercise}

\begin{proof}[Lời giải]
    \[
        \begin{pmatrix}
            1 & 2 & 3 & 4 & 5 \\
            2 & 4 & 5 & 1 & 3
        \end{pmatrix}
        \begin{pmatrix}
            1 & 2 & 3 & 4 & 5 \\
            4 & 3 & 5 & 1 & 2
        \end{pmatrix}
        =
        \begin{pmatrix}
            4 & 3 & 5 & 1 & 2 \\
            1 & 5 & 3 & 2 & 4
        \end{pmatrix}
        \begin{pmatrix}
            1 & 2 & 3 & 4 & 5 \\
            4 & 3 & 5 & 1 & 2
        \end{pmatrix}
        =
        \begin{pmatrix}
            1 & 2 & 3 & 4 & 5 \\
            1 & 5 & 3 & 2 & 4
        \end{pmatrix}.
    \]
    \[
        \begin{pmatrix}
            1 & 2 & 3 & 4 & 5 \\
            1 & 5 & 3 & 2 & 4
        \end{pmatrix}
        =
        (1)(2,5,4)(3).
    \]
    \[
        \sgn{
            \begin{matrix}
                1 & 2 & 3 & 4 & 5 \\
                1 & 5 & 3 & 2 & 4
            \end{matrix}
        }
        = \sgn{1}\sgn{2,5,4}\sgn{3}
        = 1.
    \]
\end{proof}

\begin{exercise}
    $
        \begin{pmatrix}
            1 & 2 & 3 & 4 & 5 \\
            3 & 5 & 4 & 1 & 2
        \end{pmatrix}
        \begin{pmatrix}
            1 & 2 & 3 & 4 & 5 \\
            4 & 3 & 1 & 5 & 2
        \end{pmatrix}
    $.
\end{exercise}

\begin{proof}[Lời giải]
    \[
        \begin{pmatrix}
            1 & 2 & 3 & 4 & 5 \\
            3 & 5 & 4 & 1 & 2
        \end{pmatrix}
        \begin{pmatrix}
            1 & 2 & 3 & 4 & 5 \\
            4 & 3 & 1 & 5 & 2
        \end{pmatrix}
        =
        \begin{pmatrix}
            4 & 3 & 1 & 5 & 2 \\
            1 & 4 & 3 & 2 & 5
        \end{pmatrix}
        \begin{pmatrix}
            1 & 2 & 3 & 4 & 5 \\
            4 & 3 & 1 & 5 & 2
        \end{pmatrix}
        =
        \begin{pmatrix}
            1 & 2 & 3 & 4 & 5 \\
            1 & 4 & 3 & 2 & 5
        \end{pmatrix}.
    \]
    \[
        \begin{pmatrix}
            1 & 2 & 3 & 4 & 5 \\
            1 & 4 & 3 & 2 & 5
        \end{pmatrix}
        =
        (1)(2,4)(3)(5).
    \]
    \[
        \sgn{
            \begin{matrix}
                1 & 2 & 3 & 4 & 5 \\
                1 & 4 & 3 & 2 & 5
            \end{matrix}
        }
        = \sgn{1}\sgn{2,4}\sgn{3}\sgn{5}
        = -1.
    \]
\end{proof}

\begin{exercise}
    $(1,2)(2,3)\ldots (n-1,n)$.
\end{exercise}

\begin{lemma}\label{chapter3:cycles-product}
    $(a_{1}, a_{2}, \ldots, a_{k})(a_{k},a_{k+1}) = (a_{1},a_{2},\ldots, a_{k+1})$.
\end{lemma}

\begin{proof}[Chứng minh bổ đề]
    \par Xét dãy
        \[
            a_{1}, a_{2}, \ldots, a_{k-1}, a_{k}, a_{k+1}.
        \]
    \par Sau khi tác động bằng $(a_{k},a_{k+1})$, dãy trên trở thành:
        \[
            a_{1}, a_{2}, \ldots, a_{k-1}, a_{k+1}, a_{k}.
        \]
    \par Sau khi tác động bằng $(a_{1}, a_{2}, \ldots, a_{k})$, dãy trên (liên trên) trở thành:
        \[
            a_{2}, a_{3}, \ldots, a_{k}, a_{k+1}, a_{1}.
        \]
    \par Theo định nghĩa về xích, ta có điều phải chứng minh.
\end{proof}

\begin{proof}[Lời giải]
    \par Theo bổ đề~\ref{chapter3:cycles-product}:
        \[
            (1,2)(2,3)\ldots (n-1,n) = (1,2,\ldots,n)
            =
            \begin{pmatrix}
                1 & 2 & \cdots & n-1 & n \\
                2 & 3 & \cdots & n   & 1
            \end{pmatrix}
        \]
    \par $(1,2,\ldots, n)$ chính là một xích.
        \[
            \sgn{1,2,\ldots,n} = \sgn{1,2}\sgn{2,3}\ldots\sgn{n-1,n} = (-1){}^{n-1}.
        \]
\end{proof}

\begin{exercise}
    $(1,2,3)(2,3,4)(3,4,5)\ldots (n-2,n-1,n)$.
\end{exercise}

\begin{proof}[Lời giải]
    \par Theo bổ đề~\ref{chapter3:cycles-product}, nếu $n > 3$:
    \begin{align*}
        (1,2,3)(2,3,4)(3,4,5)\ldots (n-2,n-1,n)
        & = (1,2)(2,3)(2,3)(3,4)(3,4)(4,5) \ldots (n-2,n-1)(n-1,n) \\
        & = (1,2)(2,3){}^{2}(3,4){}^{2}\ldots (n-2,n-1){}^{2}(n-1,n) \\
        & = (1,2)(n-1,n)\qquad\text{(đây là 2 xích rời nhau)} \\
        & =
        \begin{pmatrix}
            1 & 2 & 3 & \cdots & n-2 & n-1 & n   \\
            2 & 1 & 3 & \cdots & n-2 & n   & n-1
        \end{pmatrix}.
    \end{align*}
    \par Nếu $n = 3$:
    \[
        (1,2,3) =
        \begin{pmatrix}
            1 & 2 & 3 \\
            2 & 3 & 1
        \end{pmatrix}.
    \]
    \par Trong cả hai trường hợp, dấu của phép thế (kết quả) là 1.
\end{proof}

\begin{exercise}
    Cho hai cách sắp thành dãy $a_{1}$, $a_{2}$, \ldots, $a_{n}$ và $b_{1}$, $b_{2}$, \ldots, $b_{n}$ của $n$ số tự nhiên đầu tiên. Chứng minh rằng có thể đưa cách sắp này về cách sắp kia bằng cách sử dụng không quá $n-1$ phép thế sơ cấp.
\end{exercise}

\begin{lemma}\label{chapter3:product-of-disjoint-cycles}
    Mọi phép thế đều có thể được viết dưới dạng tích của các xích rời nhau.
\end{lemma}

\begin{proof}[Chứng minh bổ đề~\ref{chapter3:product-of-disjoint-cycles}]
\end{proof}

\begin{lemma}\label{chapter3:product-of-transpositions}
    Một xích độ dài $k$ ($k > 1$) có thể viết được dưới dạng tích của $k-1$ phép thế sơ cấp.
\end{lemma}

\begin{proof}[Chứng minh bổ đề~\ref{chapter3:product-of-transpositions}]
\end{proof}

\begin{proof}
\end{proof}

\end{document}

\chapter{Cartesian products}

\section{Cartesian product topology}

\begin{problem}{IV.1.1}
Let \( \left\{ Y_{\alpha} \mid \alpha \in \mathscr{A} \right\} \) be a family of spaces. Assume that each \( Y_{\alpha} \) has a basis of cardinal number \( \le \aleph \). What is the cardinal of a basis for \( \prod_{\alpha} Y_{\alpha} \)?
\end{problem}

\begin{proof}
	% TODO
	It is \( \aleph \cdot \aleph(\mathscr{A}) \).
\end{proof}

\begin{problem}{IV.1.2}
Let \( \aleph(\mathscr{A}) \) be arbitrary and \( \prod_{\alpha} A_{\alpha} \subset \prod_{\alpha} Y_{\alpha} \). If all but at most finitely many factors \( A_{\alpha} = Y_{\alpha} \), prove \( \operatorname{Int}\left( \prod_{\alpha} A_{\alpha} \right) = \prod_{\alpha} \operatorname{Int}(A_{\alpha}) \).
\end{problem}

\begin{proof}
	Let \( \beta \in \mathscr{A} \). We will show that \( \operatorname{Int}\left( \prod_{\alpha} A_{\alpha} \right) = \prod_{\alpha} \operatorname{Int}(A_{\alpha}) \) when \( A_{\alpha} = Y_{\alpha} \) for all \(\alpha \ne \beta\).
	\begingroup
	\allowdisplaybreaks%
	\begin{align*}
		\operatorname{Int}\left( \prod_{\alpha} A_{\alpha} \right) & = \mathscr{C}\overline{\prod_{\alpha} Y_{\alpha} - \prod_{\alpha}A_{\alpha}}                         \\
		                                                           & = \mathscr{C}\overline{\mathscr{C}A_{\beta} \times \prod_{\alpha \ne \beta} Y_{\alpha}}              \\
		                                                           & = \mathscr{C}\left( \overline{\mathscr{C}A_{\beta}} \times \prod_{\alpha\ne\beta} Y_{\alpha} \right) \\
		                                                           & = \mathscr{C}\overline{\mathscr{C}A_{\beta}} \times \prod_{\alpha\ne\beta} Y_{\alpha}                \\
		                                                           & = \operatorname{Int}(A_{\beta}) \times \prod_{\alpha\ne\beta} Y_{\alpha}                             \\
		                                                           & = \prod_{\alpha} \operatorname{Int}(A_{\alpha}).
	\end{align*}
	\endgroup

	Now let \( \mathscr{B} \) be a finite subset of \( \mathscr{A} \) and \( A_{\alpha} = Y_{\alpha} \) whenever \( \alpha \notin \mathscr{B} \). From the previous case, we deduce that
	\begingroup
	\allowdisplaybreaks%
	\begin{align*}
		\operatorname{Int}\left( \prod_{\alpha} A_{\alpha} \right) & = \operatorname{Int}\left( \bigcap_{\beta \in \mathscr{B}} A_{\beta} \times \prod_{\alpha\ne\beta} Y_{\alpha} \right) \\
		                                                           & = \bigcap_{\beta \in \mathscr{B}} \operatorname{Int}\left( A_{\beta} \times \prod_{\alpha\ne\beta} Y_{\alpha} \right) \\
		                                                           & = \bigcap_{\beta \in \mathscr{B}} \operatorname{Int}(A_{\beta}) \times \prod_{\alpha\ne\beta} Y_{\alpha}              \\
		                                                           & = \prod_{\alpha} \operatorname{Int}(A_{\alpha}).
	\end{align*}
	\endgroup
\end{proof}

\begin{problem}{IV.1.3}
Let \( R \) be the real numbers with upper-limit topology (Chapter III, Section 3, Example 4). Show that \( R \times R \) is not a discrete space, but that \( A = \left\{ (x, y) \mid x + y = 1 \right\} \), as a subspace of \( R \times R \), has the discrete topology.
\end{problem}

\begin{proof}
	The singleton \( \left\{ (0, 0) \right\} \) is not open in \( R \times R \) as it doesn't contain any basic open set of the for \( \halfopenleft{a, b} \times \halfopenleft{c, d} \).

	On the other hand, if \( (x, y) \in A \) then \( \left\{ (x, y) \right\} = A \cap (\halfopenleft{x - 1, x} \times \halfopenleft{y - 1, y}) \) so \( \left\{ (x, y) \right\} \) is open in \( A \). Thus \( A \) has the discrete topology.
\end{proof}

\begin{problem}{IV.1.4}
Prove: \( \prod_{\alpha} A_{\alpha} \) is dense in \( \prod_{\alpha} Y_{\alpha} \) if and only if each \( A_{\alpha} \subset Y_{\alpha} \) is dense.
\end{problem}

\begin{proof}
	Because \( \overline{\prod_{\alpha} A_{\alpha}} = \prod_{\alpha} \overline{A_{\alpha}} \), the result follows.
\end{proof}

\section{Continuity of maps}

\begin{problem}{IV.2.1}
Prove: The cartesian product topology in \( \prod_{\alpha} Y_{\alpha} \) is the smallest topology for which all projections \( p_{\beta}: \prod_{\alpha} Y_{\alpha} \to Y_{\beta} \) are continuous.
\end{problem}

\begin{proof}
	Let \( \mathscr{T} \) be a topology on \( \prod_{\alpha} Y_{\alpha} \) such that all projections \( p_{\beta} \) are continuous. We need to show that \( \mathscr{T} \) contains the cartesian product topology.

	For each open set \( U_{\beta} \subset Y_{\beta} \), \( \left\langle U_{\beta} \right\rangle = p_{\beta}^{-1}(U_{\beta}) \in \mathscr{T} \) because \( p_{\beta} \) is continuous. Therefore the cartesian product topology is contained in \( \mathscr{T} \).

	Thus the cartesian product topology is the smallest topology for which all projections are continuous.
\end{proof}

\begin{problem}{IV.2.2}
Let \( \left\{ Y_{\alpha} \mid \alpha \in \mathscr{A} \right\} \) be a family of spaces. For each \( \mathscr{B} \subset \mathscr{A} \), let
\[
	p_{\mathscr{B}}: \prod_{\alpha \in \mathscr{A}} Y_{\alpha} \to \prod_{\beta \in \mathscr{B}} Y_{\beta}
\]

be the projection. Let \( A \subset \prod_{\alpha} Y_{\alpha} \) be closed. Prove:
\[
	A = \bigcap_{\mathscr{B} \text{ finite}} p_{\mathscr{B}}^{-1}\left( p_{\mathscr{B}}(A) \right).
\]
\end{problem}

\begin{proof}
	For each \( \mathscr{B} \subset \mathscr{A} \), \( p_{\mathscr{B}}^{-1}\left( p_{\mathscr{B}}(A) \right) \supset A \) so
	\[
		\bigcap_{\mathscr{B} \text{ finite}} p_{\mathscr{B}}^{-1}\left( p_{\mathscr{B}}(A) \right) \supset A
	\]

	Let \( c \in \bigcap_{\mathscr{B} \text{ finite}} p_{\mathscr{B}}^{-1}\left( p_{\mathscr{B}}(A) \right) \). Suppose on the contrary that \( c \notin A \). Since \( A \) is closed, there exists a basic open set \( U = \left\langle U_{\alpha_{1}}, \ldots, U_{\alpha_{n}} \right\rangle \) that contains \( c \) and is contained in \( \mathscr{C}A \). Let \( \mathscr{B} = \left\{ \alpha_{1}, \ldots, \alpha_{n} \right\} \). Because \( c \in p_{\mathscr{B}}^{-1}(p_{\mathscr{B}}(A)) \), \( p_{\mathscr{B}}(c) \in p_{\mathscr{B}}(A) \), so there exists \( f \in A \) such that \( p_{\mathscr{B}}(c) = p_{\mathscr{B}}(f) \). This means \( p_{\alpha_{i}}(f) = f(\alpha_{i}) = c(\alpha_{i}) \in U_{\alpha_{i}} \) for each \( i = 1, \ldots, n \), so \( f \in \left\langle U_{\alpha_{i}} \right\rangle \), which means \( f \in U = \bigcap^{n}_{i=1} \left\langle U_{\alpha_{i}} \right\rangle \). Therefore \( f \in A \cap U \), which contradicts \( U \subset \mathscr{C}A \). Thus \( c \in A \) and we conclude that \( \bigcap_{\mathscr{B} \text{ finite}} p_{\mathscr{B}}^{-1}\left( p_{\mathscr{B}}(A) \right) \subset A \).

	Hence \( A = \bigcap_{\mathscr{B} \text{ finite}} p_{\mathscr{B}}^{-1}\left( p_{\mathscr{B}}(A) \right) \).
\end{proof}


\section{Slices in Cartesian Products}

\begin{problem}{IV.3.1}
Let \( \mathscr{S} \) be the Sierpiński space, and \( \mathscr{S} \times \left\{ 0 \right\} \) the slice in \( \mathscr{S} \times \mathscr{S} \) parallel to the first factor. Is \( \mathscr{S} \times \left\{ 0 \right\} \) closed in \( \mathscr{S} \times \mathscr{S} \)?
\end{problem}

\begin{proof}
	The complement of \( \mathscr{S} \times \left\{0\right\} \) in \( \mathscr{S} \times \mathscr{S} \) is \( \mathscr{S} \times \left\{1\right\} \), which is not open. Hence \( \mathscr{S} \times \left\{ 0 \right\} \) is not closed in \( \mathscr{S} \times \mathscr{S} \).
\end{proof}

\section{Peano curves}


\newpage
\chapter{Rings and Fields}

\chapter{Identification Topology; Weak Topology}

\section{Identification Topology}

\begin{problem}{VI.1.1}\label{problem:VI.1.1}
Reversing the situation treated in the text, let \(X\) be a set, \( (Y, \mathscr{T}) \) a space, and \( p: X \to Y \) a surjective map. Prove:
\begin{enumerate}[label={(\alph*)}]
	\item \( \mathscr{T}_{X} = \left\{ p^{-1}(U) \mid U \text{ open in } Y \right\} \) is a topology in \( X \).
	\item \( p: (X, \mathscr{T}_{X}) \to (Y, \mathscr{T}) \) is continuous, open, and closed.
\end{enumerate}
\end{problem}

\begin{proof}
	\begin{enumerate}[label={(\alph*)}]
		\item \( \mathscr{T}_{X} \) contains \( \varnothing, X \) as \( p^{-1}(\varnothing) = \varnothing \) and \( p^{-1}(Y) = X \).

		      If \( {\left\{ U_{\alpha} \right\}}_{\alpha\in\mathscr{A}} \) is a collection of open sets in \( Y \), then
		      \[
			      \bigcup_{\alpha\in\mathscr{A}} p^{-1}(U_{\alpha}) = p^{-1}\left(\bigcup_{\alpha\in\mathscr{A}} U_{\alpha}\right)
		      \]

		      so \( \mathscr{T}_{X} \) is closed under arbitrary union.

		      If \( U_{1}, \ldots, U_{n} \) are open sets in \( Y \) then
		      \[
			      \bigcap^{n}_{i=1} p^{-1}(U_{i}) = p^{-1}\left(\bigcap^{n}_{i=1} U_{i}\right)
		      \]

		      so \( \mathscr{T}_{X} \) is closed under finite intersection.

		      Hence \( \mathscr{T}_{X} \) is a topology in \( X \).
		\item For each open set \( U \) in \( Y \), \( p^{-1}(U) \in \mathscr{T}_{X} \) so \( p \) is continuous.

		      Let \( V \) be an open set in \( X \). Then there is an open set \( U \) in \( Y \) such that \( V = p^{-1}(U) \). Hence \( p(V) = pp^{-1}(U) = U \) because \( p \) is surjective. So \( p \) is an open map.

		      Let \( W \) be a closed set in \( X \) then \( X - W \) is open and there exists an open set \( U \) in \( Y \) such that \( X - W = p^{-1}(U) \). Therefore
		      \[
			      W = X - p^{-1}(U) = p^{-1}(Y) - p^{-1}(U) = p^{-1}(Y - U)
		      \]

		      which implies that \( p(W) = pp^{-1}(Y - U) = Y - U \), which is closed in \( Y \). So \( p \) is a closed map.

		      Thus \( p \) is a continuous, open, and closed map.
	\end{enumerate}
\end{proof}

\begin{problem}{VI.1.2}
For each \( \alpha \in \mathscr{A} \), let \( p_{\alpha}: X_{\alpha} \to Y_{\alpha} \) be a continuous, open surjection. Show that \( \prod_{\alpha} p_{\alpha}: \prod_{\alpha} X_{\alpha} \to \prod_{\alpha} Y_{\alpha} \) is an identification.
\end{problem}

\begin{proof}
	For the sake of brevity, denote \( p = \prod_{\alpha} p_{\alpha} \). By definition, \( p \) is surjective.

	\( p_{Y_{\alpha}} \circ p \) is continuous for each projection \( p_{Y_{\alpha}}: \prod_{\alpha} Y_{\alpha} \to Y_{\alpha} \) so \( p \) is continuous.

	Let \( \prod_{\alpha} U_{\alpha} \) be a basic open set in \( \prod_{\alpha} X_{\alpha} \), which means \( U_{\alpha} = X_{\alpha} \) for all but finitely many \( \alpha \) and \( U_{\alpha} \) is open in \( X_{\alpha} \) for every \( \alpha \). Because \( p_{\alpha} \) is an open surjection for each \( \alpha \), the image
	\[
		p\left( \prod_{\alpha} U_{\alpha} \right) = \prod_{\alpha} p_{\alpha}(U_{\alpha})
	\]

	is open in \( \prod_{\alpha} Y_{\alpha} \) as \( p_{\alpha}(U_{\alpha}) \) is open in \( Y_{\alpha} \) and \( p_{\alpha}(U_{\alpha}) = Y_{\alpha} \) for all but finitely many \( \alpha \). Hence \( p \) is an open map.

	\( p \) is a continuous, open surjection so \( p \) is an identification.
\end{proof}

\begin{problem}{VI.1.3}
Let \( X \) be a space and \( A \subset X \) a subspace. Assume that there exists a continuous \( r: X \to A \) such that \( r\vert_{A} = 1_{A} \) (such a map is called a \textit{retraction} of \(X\) onto \(A\)). Show that \( r \) is an identification.
\end{problem}

\begin{proof}
	By definition, \( r \) is continuous and surjective. Let \( f: A \xhookrightarrow{} X \) be the inclusion map.

	\( f \) is continuous and \( r \circ f = 1_{A} \) so \( r \) is an identification.
\end{proof}

\begin{problem}{VI.1.4}\label{problem:VI.1.4}
Let \( X \) be any set. Given any family \( \left\{ (Y_{\alpha}, \mathscr{T}_{\alpha}), f_{\alpha} \mid \alpha \in \mathscr{A} \right\} \) of spaces and maps \( f_{\alpha}: X \to Y_{\alpha} \), the ``projective limit topology of \(X\) determined by this family'' is \( \bigvee_{\alpha} f_{\alpha}^{-1}(\mathscr{T}_{\alpha}) \) (see Problem~\ref{problem:III.3.8}). Prove:
\begin{enumerate}[label={(\alph*)}]
	\item If \( j: X \to \prod_{\alpha} Y_{\alpha} \) is the map \( j(x) = \left\{ f_{\alpha}(x) \right\} \), then \( \bigvee_{\alpha} f_{\alpha}^{-1}(\mathscr{T}_{\alpha}) \) is the topology in \(X\) determined by \(j\) as in Problem~\ref{problem:VI.1.1}.
	\item If whenever \( x \ne x^{\prime} \), there is some index \( \alpha \) such that \( f_{\alpha}(x) \ne f_{\alpha}(x^{\prime}) \), then \( j \) is an embedding.
\end{enumerate}
\end{problem}

\begin{proof}
	\begin{enumerate}[label={(\alph*)}]
		\item Let \( \prod_{\alpha} U_{\alpha} \) be a subbasic open set in \( \prod_{\alpha} Y_{\alpha} \) then \( U_{\alpha} = Y_{\alpha} \) for every \( \alpha \) but one \( \beta \in \mathscr{A} \).
		      \[
			      j^{-1}\left( \prod_{\alpha} U_{\alpha} \right) = \bigcap_{\alpha} f_{\alpha}^{-1}(U_{\alpha}) = f_{\beta}^{-1}(U_{\beta}) \in \bigvee_{\alpha} f_{\alpha}^{-1}(\mathscr{T}_{\alpha})
		      \]

		      Hence \( j \) is continuous, which means if \( j^{-1}(U) \) is open whenever \( U \subset \prod_{\alpha} Y_{\alpha} \) is open.

		      Let \( V \) be an open set in \( X \). According to the definition of the topology \( \bigvee_{\alpha} f_{\alpha}^{-1}(\mathscr{T}_{\alpha}) \), \( V \) can be written as a union of finite intersection of elements in \( \bigcup_{\alpha} f_{\alpha}^{-1}(\mathscr{T}_{\alpha}) \), which means
		      \[
			      V = \bigcup_{i\in I} V_{i}
		      \]

		      where each \( V_{i} \) is a finite intersection of elements in \( \bigcup_{\alpha} f_{\alpha}^{-1}(\mathscr{T}_{\alpha}) \).
		      \[
			      V_{i} = \bigcap^{n_{i}}_{k=1} f_{\alpha_{k}}^{-1}(U_{\alpha_{k}}) = \bigcap^{n_{i}}_{k=1} j^{-1}\left( U_{\alpha_{k}} \times \prod_{\alpha \ne \alpha_{k}} Y_{\alpha} \right) = j^{-1}\left( \bigcap^{n_{i}}_{k=1} U_{\alpha_{k}} \times \prod_{\alpha \ne \alpha_{k}} Y_{\alpha} \right) = j^{-1}(W_{i})
		      \]

		      where \( U_{\alpha_{k}} \) is open in \( Y_{\alpha_{k}} \). So
		      \[
			      V = \bigcup_{i\in I} j^{-1}(W_{i}) = j^{-1}\left( \bigcup_{i\in I} W_{i} \right)
		      \]

		      which means \( V \) is the preimage of an open set in \( \prod_{\alpha} Y_{\alpha} \).

		      Thus \( \bigvee_{\alpha} f_{\alpha}^{-1}(\mathscr{T}_{\alpha}) \) is the same as the topology in \( X \) determined by \( j \) as in Problem~\ref{problem:VI.1.1}.
		\item According to Problem~\ref{problem:VI.1.1}, \( j \) is continuous, open, and closed.

		      Whenever \( x \ne x^{\prime} \), there is some index \( \alpha \) such that \( f_{\alpha}(x) \ne f_{\alpha}(x^{\prime}) \), then \( j(x) \ne j(x^{\prime}) \), which implies \( j \) is injective.

		      A continuous, open, injective map is an embedding so \( j \) is an embedding.
	\end{enumerate}
\end{proof}

\section{Subspaces}

\begin{problem}{VI.2.1}
Let \(X\) have the projective limit topology (Problem~\ref{problem:VI.1.4}) determined by
\[
	\left\{ Y_{\alpha}, f_{\alpha} \mid \alpha \in \mathscr{A} \right\}
\]

and let \( A \subset X \). Prove: The subspace topology of \(A\) is the projective limit topology determined by the maps \( f_{\alpha}\vert_{A} \).
\end{problem}

\begin{proof}
	The projective limit topology on \( A \) determined by the maps \( f_{\alpha}\vert_{A} \) has subbasis
	\[
		\bigcup_{\alpha} {(f_{\alpha}\vert_{A})}^{-1}(\mathscr{T}_{\alpha})
	\]

	Let \( V \) be an open set in \( A \) (with the projective limit topology) then
	\[
		V = \bigcup_{i\in I} V_{i}
	\]

	in which each \( V_{i} \) is the intersection of finitely many elements of \( \bigcup_{\alpha} {(f_{\alpha}\vert_{A})}^{-1}(\mathscr{T}_{\alpha}) \). So there exist \( \alpha_{i_{1}}, \ldots, \alpha_{i_{n(i)}} \in \mathscr{A} \) such that
	\[
		V_{i} = \bigcap^{n(i)}_{k=1} {(f_{\alpha_{k}}\vert_{A})}^{-1}(U_{\alpha_{k}})
	\]

	Hence
	\begingroup
	\allowdisplaybreaks%
	\begin{align*}
		V_{i} & = \bigcap^{n(i)}_{k=1} (A \cap f_{\alpha_{k}}^{-1}(U_{\alpha_{k}}))                                                             \\
		      & = A \cap \bigcap^{n(i)}_{k=1} f_{\alpha_{k}}^{-1}(U_{\alpha_{k}})                                                               \\
		      & = A \cap \bigcap^{n(i)}_{k=1} j^{-1}\left( U_{\alpha_{k}} \times \prod_{\alpha \ne \alpha_{k}} Y_{\alpha} \right)               \\
		      & = A \cap j^{-1}\left( \bigcap^{n(i)}_{k=1} \left( U_{\alpha_{k}} \times \prod_{\alpha\ne\alpha_{k}} Y_{\alpha} \right) \right).
	\end{align*}
	\endgroup

	Therefore
	\begingroup
	\allowdisplaybreaks%
	\begin{align*}
		V & = A \cap \bigcup_{i\in I} j^{-1}\left( \bigcap^{n(i)}_{k=1} \left( U_{\alpha_{k}} \times \prod_{\alpha\ne\alpha_{k}} Y_{\alpha} \right) \right) \\
		  & = A \cap j^{-1}\left( \bigcup_{i\in J} \bigcap^{n(i)}_{k=1} \left( U_{\alpha_{k}} \times \prod_{\alpha\ne\alpha_{k}} Y_{\alpha} \right) \right)
	\end{align*}
	\endgroup

	Hence \( V \) is in the subspace topology of \( A \).

	Conversely, one can show that if \( V \) is in the subspace topology of \( A \), then \( V \) is also in the projective limit topology on \( A \) determinded by the maps \( f_{\alpha}\vert_{A} \).

	Thus the projective limit topology on \( A \) determinded by the maps \( f_{\alpha}\vert_{A} \) and the subspace topology on \( A \) coincide.
\end{proof}

\section{General Theorems}

\begin{problem}{VI.3.1}
Let \( p: X \to Y \) be a continuous open (or closed) surjection, and assume that each fiber \( p^{-1}(y) \) is connected. For any \( F \subset Y \), show that \( F \) is connected if and only if \( p^{-1}(F) \) is connected.
\end{problem}

\begin{proof}
	By Proposition 2.1, \( p\vert_{p^{-1}(F)}: p^{-1}(F) \to F \) is an identification because \( p \) is an identification which is also an open (or closed) map. Denote \( q = p\vert_{p^{-1}(F)} \).

	If \( p^{-1}(F) \) is connected then \( F = p(p^{-1}(F)) \) is connected, as \( p \) is a continuous surjection.

	If \( p^{-1}(F) \) is not connected then there is a continuous surjection \( h: p^{-1}(F) \to 2 \). As each fiber of \( q \) (each fiber of \(q \) is a fiber of \(p\)) is connected, the restriction of \( h \) to each fiber is a constant map. Therefore \( hq^{-1}: F \to 2 \) is a continuous surjection, according to the transgression property, which means \( F \) is not connected.
\end{proof}

\begin{problem}{VI.3.2}
Let \( X \) have the projective limit topology \( \mathscr{T} \) determined by the family
\[
	\left\{ (Y_{\alpha}, \mathscr{T}_{\alpha}), f_{\alpha} \mid \alpha \in \mathscr{A} \right\}
\]

Assume that each \( \mathscr{T}_{\alpha} \) is the projective limit topology determined by a family
\[
	\left\{ (Z_{\alpha, \beta}, \mathscr{T}_{\alpha,\beta}), g_{\alpha,\beta} \mid \beta \in \mathscr{B} \right\}.
\]

Prove: \( \mathscr{T} \) is the projective limit topology determined by
\[
	\left\{ (Z_{\alpha,\beta}, \mathscr{T}_{\alpha,\beta}), g_{\alpha,\beta} \circ f_{\alpha} \mid (\alpha, \beta) \in \mathscr{A} \times \mathscr{B} \right\}.
\]
\end{problem}

\begin{proof}
	Denote by \( \widetilde{\mathscr{T}} \) the projective limit topology determined by
	\[
		\left\{ (Z_{\alpha,\beta}, \mathscr{T}_{\alpha,\beta}), g_{\alpha,\beta} \circ f_{\alpha} \mid (\alpha, \beta) \in \mathscr{A} \times \mathscr{B} \right\}.
	\]

	Let \( h: X \to \prod_{\alpha} Y_{\alpha} \) be the map \( h(x) = {\left\{ f_{\alpha}(x) \right\}}_{\alpha} \) then
	\[
		\mathscr{T} = \left\{ h^{-1}(U) \mid U \text{ open in } \prod_{\alpha}Y_{\alpha} \right\}
	\]

	according to Problem~\ref{problem:VI.1.4}.

	For each \( \alpha \), let \( h_{\alpha}: Y_{\alpha} \to \prod_{\beta} Z_{\alpha,\beta} \) be the map \( h_{\alpha}(x) = {\left\{ g_{\alpha,\beta}(x) \right\}}_{\beta} \) then
	\[
		\mathscr{T}_{\alpha} = \left\{ h_{\alpha}^{-1}(U) \mid U \text{ open in } \prod_{\beta} Z_{\alpha,\beta} \right\}
	\]

	according to Problem~\ref{problem:VI.1.4}.

	Let \( \ell: (X, \widetilde{\mathscr{T}}) \to \prod_{\alpha,\beta} Z_{\alpha,\beta} \) be the map \( \ell(x) = {\left\{ g_{\alpha,\beta}(f_{\alpha}(x)) \right\}}_{\alpha,\beta} \) then
	\[
		\widetilde{\mathscr{T}} = \left\{ \ell^{-1}(U) \mid U \text{ open in } \prod_{\alpha,\beta} Z_{\alpha,\beta} \right\}
	\]

	according to Problem~\ref{problem:VI.1.4}.

	Note that \( f_{\alpha} = p_{\alpha} \circ h \) and \( g_{\alpha,\beta} = p_{\alpha,\beta} \circ h_{\alpha} \) in which \( p_{\alpha}: \prod_{\alpha} Y_{\alpha} \to Y_{\alpha} \) and \( p_{\alpha,\beta}: \prod_{\beta} Z_{\alpha,\beta} \to Z_{\alpha,\beta} \) are projection maps. Denote by \( q_{\alpha,\beta} \) the projection map \( \prod_{\alpha,\beta} W_{\alpha,\beta} \to W_{\alpha,\beta} \).
	\[
		\begin{tikzcd}
			&& {\prod_{\alpha} Y_{\alpha}} \\
			\\
			X && {Y_{\alpha}} && {\prod_{\beta}Z_{\alpha,\beta}} && {Z_{\alpha,\beta}}
			\arrow["{p_{\alpha}}", from=1-3, to=3-3]
			\arrow["h", from=3-1, to=1-3]
			\arrow["{f_{\alpha}}"', from=3-1, to=3-3]
			\arrow["{h_{\alpha}}"', from=3-3, to=3-5]
			\arrow["{g_{\alpha,\beta}}"', bend right, from=3-3, to=3-7]
			\arrow["{p_{\alpha,\beta}}"', from=3-5, to=3-7]
		\end{tikzcd}
	\]

	\[
		\begin{tikzcd}
			X && {\prod_{\alpha,\beta} Z_{\alpha,\beta}} && {Z_{\alpha,\beta}}
			\arrow["\ell", from=1-1, to=1-3]
			\arrow["{g_{\alpha,\beta} \circ f_{\alpha}}"', bend right, from=1-1, to=1-5]
			\arrow["{q_{\alpha,\beta}}", from=1-3, to=1-5]
		\end{tikzcd}
	\]

	Let \( U \in \mathscr{T} \) then there exists \( V \) open in \( \prod_{\alpha} Y_{\alpha} \) such that \( U = h^{-1}(V) \) (see Problem~\ref{problem:VI.1.4} and~\ref{problem:VI.1.1}). One can write \( V \) in terms of subbasic elements as follows
	\[
		V = \bigcup_{i\in I} \bigcap^{n(i)}_{k=1} p_{\alpha_{k}}^{-1}(V_{\alpha_{k}})
	\]

	in which \( V_{\alpha_{k}} \) is open in \( Y_{\alpha_{k}} \).

	As \( \mathscr{T}_{\alpha} \) is the projective limit topology on \( Y_{\alpha} \) determined by the maps \( g_{\alpha,\beta}: Y_{\alpha} \to Z_{\alpha,\beta} \), there is an open set \( W_{\alpha_{k}} \) in \( \prod_{\beta} Z_{\alpha_{k},\beta} \) such that \( V_{\alpha_{k}} = h_{\alpha}^{-1}(W_{\alpha_{k}}) \). The open set \( W_{\alpha_{k}} \) can be written in terms of subbasic elements as follows
	\[
		W_{\alpha_{k}} = \bigcup_{j \in J} \bigcap^{n(j)}_{r=1} p_{\alpha_{k},\beta_{r}}^{-1}(W_{\alpha_{k}, \beta_{r}})
	\]

	in which \( W_{\alpha_{k}, \beta_{r}} \) is open in \( Z_{\alpha_{k}, \beta_{r}} \).
	\begingroup
	\allowdisplaybreaks%
	\begin{align*}
		V             & = \bigcup_{i\in I} \bigcap^{n(i)}_{k=1} p_{\alpha_{k}}^{-1}(V_{\alpha_{k}})                                                                                                                         \\
		              & = \bigcup_{i\in I} \bigcap^{n(i)}_{k=1} p_{\alpha_{k}}^{-1}\left( h^{-1}_{\alpha_{k}}(W_{\alpha_{k}}) \right)                                                                                       \\
		              & = \bigcup_{i\in I} \bigcap^{n(i)}_{k=1} {(h_{\alpha_{k}} \circ p_{\alpha_{k}})}^{-1}(W_{\alpha_{k}})                                                                                                \\
		              & = \bigcup_{i\in I} \bigcap^{n(i)}_{k=1} {(h_{\alpha_{k}} \circ p_{\alpha_{k}})}^{-1} \left( \bigcup_{j \in J} \bigcap^{n(j)}_{r=1} p_{\alpha_{k},\beta_{r}}^{-1}(W_{\alpha_{k}, \beta_{r}}) \right) \\
		              & = \bigcup_{i\in I} \bigcap^{n(i)}_{k=1} \bigcup_{j\in J} \bigcap^{n(j)}_{r=1} {(p_{\alpha_{k},\beta_{r}} \circ h_{\alpha_{k}} \circ p_{\alpha_{k}})}^{-1}(W_{\alpha_{k},\beta_{r}})                 \\
		U = h^{-1}(V) & = \bigcup_{i\in I} \bigcap^{n(i)}_{k=1} \bigcup_{j\in J} \bigcap^{n(j)}_{r=1} {(p_{\alpha_{k},\beta_{r}} \circ h_{\alpha_{k}} \circ p_{\alpha_{k}} \circ h)}^{-1}(W_{\alpha_{k},\beta_{r}})         \\
		              & = \bigcup_{i\in I} \bigcap^{n(i)}_{k=1} \bigcup_{j\in J} \bigcap^{n(j)}_{r=1} {(g_{\alpha_{k},\beta_{r}} \circ f_{\alpha_{k}})}^{-1}(W_{\alpha_{k},\beta_{r}})                                      \\
		              & = \bigcup_{i\in I} \bigcap^{n(i)}_{k=1} \bigcup_{j\in J} \bigcap^{n(j)}_{r=1} {(q_{\alpha_{k},\beta_{r}} \circ \ell)}^{-1}(W_{\alpha_{k},\beta_{r}})                                                \\
		              & = \bigcup_{i\in I} \bigcap^{n(i)}_{k=1} \bigcup_{j\in J} \bigcap^{n(j)}_{r=1} \ell^{-1}q_{\alpha_{k},\beta_{r}}^{-1}(W_{\alpha_{k},\beta_{r}})                                                      \\
		              & = \ell^{-1}\left( \bigcup_{i\in I} \bigcap^{n(i)}_{k=1} \bigcup_{j\in J} \bigcap^{n(j)}_{r=1} q_{\alpha_{k},\beta_{r}}^{-1}(W_{\alpha_{k},\beta_{r}}) \right) \in \widetilde{\mathscr{T}}
	\end{align*}
	\endgroup

	Hence \( U \in \widetilde{\mathscr{T}} \), which means \( \mathscr{T} \subset \widetilde{\mathscr{T}} \).

	\bigskip
	Conversely, let \( U \in \widetilde{\mathscr{T}} \) then there exists \( W \) open in \( \prod_{\alpha,\beta} Z_{\alpha,\beta} \) such that \( U = \ell^{-1}(W) \).

	\( W \) can be written in terms of subbasic elements.
	\begingroup
	\allowdisplaybreaks%
	\begin{align*}
		U & = \ell^{-1}(W) = \ell^{-1}\left( \bigcup_{i\in I}\bigcap^{n(i)}_{r=1} q^{-1}_{\alpha_{r},\beta_{r}}(W_{\alpha_{r},\beta_{r}}) \right) \\
		  & = \bigcup_{i\in I}\bigcap^{n(i)}_{r=1} \ell^{-1}q^{-1}_{\alpha_{r},\beta_{r}}(W_{\alpha_{r},\beta_{r}})                               \\
		  & = \bigcup_{i\in I}\bigcap^{n(i)}_{r=1} {(q_{\alpha_{r},\beta_{r}}\circ \ell)}^{-1}(W_{\alpha_{r},\beta_{r}})                          \\
		  & = \bigcup_{i\in I}\bigcap^{n(i)}_{r=1} {(g_{\alpha_{r},\beta_{r}}\circ f_{\alpha_{r}})}^{-1}(W_{\alpha_{r},\beta_{r}})                \\
		  & = \bigcup_{i\in I}\bigcap^{n(i)}_{r=1} f_{\alpha_{r}}^{-1}(g_{\alpha_{r},\beta_{r}}^{-1}(W_{\alpha_{r},\beta_{r}})) \in \mathscr{T}
	\end{align*}
	\endgroup

	so \( \widetilde{\mathscr{T}} \subset \mathscr{T} \).

	Thus \( \mathscr{T} = \widetilde{\mathscr{T}} \).
\end{proof}

\begin{problem}{VI.3.3}
Let \(X\) have the projective limit topology determined by \( \left\{ Y_{\alpha}, f_{\alpha} \mid \alpha \in \mathscr{A} \right\} \). Prove: \( f: Z \to X \) is continuous if and only if each \( f_{\alpha} \circ f \) is continuous.
\end{problem}

\begin{proof}
	For each \( \alpha \), the map \( f_{\alpha}: X \to Y_{\alpha} \) is continuous.

	If \( f \) is continuous then each \( f_{\alpha} \circ f \) is continuous.

	Conversely, assume that each \( f_{\alpha} \circ f \) is continuous. Let \( U \) be an open set in \( X \).
	\[
		\bigcup_{\alpha} f_{\alpha}^{-1}(\mathscr{T}_{\alpha})
	\]

	is a subbasis for the projective limit topology on \( X \). Therefore \( U \) can be written as
	\[
		U = \bigcup_{\gamma} \bigcap^{n(\gamma)}_{k=1} f_{\gamma,k}^{-1}(U_{\gamma,k})
	\]

	in which \( U_{\gamma,k} \) is open in \( Y_{\gamma,k} \) so
	\begingroup
	\allowdisplaybreaks%
	\begin{align*}
		f^{-1}(U) & = f^{-1}\left( \bigcup_{\gamma} \bigcap^{n(\gamma)}_{k=1} f_{\gamma,k}^{-1}(U_{\gamma,k}) \right) \\
		          & = \bigcup_{\gamma} \bigcap^{n(\gamma)}_{k=1} f^{-1}(f_{\gamma,k}^{-1}(U_{\gamma,k}))              \\
		          & = \bigcup_{\gamma} \bigcap^{n(\gamma)}_{k=1} {(f_{\gamma,k} \circ f)}^{-1}(U_{\gamma,k})
	\end{align*}
	\endgroup

	\( {(f_{\gamma,k} \circ f)}^{-1}(U_{\gamma,k}) \) is open as \( U_{\gamma,k} \) is open in \( Y_{\gamma,k} \) and \( f_{\gamma,k} \circ f \) is continuous. Hence \( f^{-1}(U) \) is open (finite intersection then arbitrary union), so \( f \) is continuous.

	Thus \( f \) is continuous if and only if each \( f_{\alpha} \circ f \) is continuous.
\end{proof}

\section{Spaces with Equivalence Relations}

\section{Cones and Suspensions}

\section{Attaching of Spaces}

\section{The Relation \(K(f)\) of Continuous Maps}

\section{Weak Topologies}

\chapter{Operators on Inner Product Spaces}

\section{Self-Adjoint and Normal Operators}

% chapter7:sectionA:exercise1
\begin{exercise}
    Suppose $n$ is a positive integer. Define $T\in\lmap{\mathbb{F}^{n}}$ by
    \[
        T(z_{1}, \ldots, z_{n}) = (0, z_{1}, \ldots, z_{n-1}).
    \]

    Find a formula for $T^{*}(z_{1}, \ldots, z_{n})$.
\end{exercise}

\begin{proof}
    \begin{align*}
        \innerprod{(z_{1}, \ldots, z_{n}), T^{*}(w_{1}, \ldots, w_{n})} & = \innerprod{T(z_{1}, \ldots, z_{n}), (w_{1}, \ldots, w_{n})}      \\
                                                                        & = \innerprod{(0, z_{1}, \ldots, z_{n-1}), (w_{1}, \ldots, w_{n})}  \\
                                                                        & = 0 + z_{1}\conj{w_{2}} + \cdots + z_{n-1}\conj{w_{n}}             \\
                                                                        & = z_{1}\conj{w_{2}} + \cdots + z_{n-1}\conj{w_{n}} + z_{n}\conj{0} \\
                                                                        & = \innerprod{(z_{1}, \ldots, z_{n}), (w_{2}, \ldots, w_{n}, 0)}.
    \end{align*}

    Thus $T^{*}(w_{1}, \ldots, w_{n}) = (w_{2}, \ldots, w_{n}, 0)$.
\end{proof}
\newpage

% chapter7:sectionA:exercise2
\begin{exercise}
    Suppose $T\in\lmap{V, W}$. Prove that
    \[
        T = 0 \Longleftrightarrow T^{*} = 0 \Longleftrightarrow T^{*}T = 0 \Longleftrightarrow TT^{*} = 0.
    \]
\end{exercise}

\begin{proof}
    \begin{align*}
        T = 0 & \Longleftrightarrow \kernel{T} = V                                                                           \\
              & \Longleftrightarrow {(\kernel{T})}^{\bot} = \{ 0 \}                                                          \\
              & \Longleftrightarrow \range{T^{*}} = \{ 0 \}         & \text{(since $\range{T^{*}} = {(\kernel{T})}^{\bot}$)} \\
              & \Longleftrightarrow T^{*} = 0.
    \end{align*}

    $T^{*}T$ and $TT^{*}$ are self-adjoint operators.
    \begin{align*}
        T = 0     & \Longleftrightarrow \innerprod{Tv, Tv} = 0\quad\forall v\in V         \\
                  & \Longleftrightarrow \innerprod{v, T^{*}(Tv)} = 0\quad\forall v\in V   \\
                  & \Longleftrightarrow \innerprod{v, (T^{*}T)v} = 0\quad\forall v\in V   \\
                  & \Longleftrightarrow T^{*}T = 0,                                       \\
        T^{*} = 0 & \Longleftrightarrow \innerprod{T^{*}v, T^{*}v} = 0\quad\forall v\in V \\
                  & \Longleftrightarrow \innerprod{T(T^{*}v), v} = 0\quad\forall v\in V   \\
                  & \Longleftrightarrow \innerprod{(TT^{*})v, v} = 0\quad\forall v\in V   \\
                  & \Longleftrightarrow TT^{*} = 0.
    \end{align*}

    Thus $T = 0 \Longleftrightarrow T^{*} = 0 \Longleftrightarrow T^{*}T = 0 \Longleftrightarrow TT^{*} = 0$.
\end{proof}
\newpage

% chapter7:sectionA:exercise3
\begin{exercise}
    Suppose $T\in\lmap{V}$ and $\lambda\in\mathbb{F}$. Prove that
    \[
        \text{$\lambda$ is an eigenvalue of $T$} \Longleftrightarrow \text{$\conj{\lambda}$ is an eigenvalue of $T^{*}$}.
    \]
\end{exercise}

\begin{proof}
    For every $\lambda\in\mathbb{F}$, ${(T - \lambda I)}^{*} = T^{*} - \conj{\lambda}I$.

    $\kernel{(T^{*} - \conj{\lambda}I)} = {(\range{(T - \lambda I)})}^{\bot}$ and $\kernel{(T - \lambda I)} = {(\range{(T^{*} - \conj{\lambda}I)})}^{\bot}$.

    If $\lambda$ is an eigenvalue of $T$, then $\kernel{(T - \lambda I)}\ne \{0\}$, so ${(\range{(T^{*} - \conj{\lambda}I)})}^{\bot}\ne \{0\}$, and it follows that ${\range{(T^{*} - \conj{\lambda}I)}}\ne V$. Therefore $\kernel{(T^{*} - \conj{\lambda} I)}\ne \{0\}$ (follows from the fundamental theorem of linear maps). So $\conj{\lambda}$ is an eigenvalue of $T^{*}$.

    \bigskip

    If $\conj{\lambda}$ is an eigenvalue of $T^{*}$, then $\conj{\conj{\lambda}} = \lambda$ is an eigenvalue of ${(T^{*})}^{*} = T$.

    \bigskip

    Thus $\lambda$ is an eigenvalue of $T$ if and only if $\conj{\lambda}$ is an eigenvalue of $T^{*}$.
\end{proof}

\begin{proof}
    For every $\lambda\in\mathbb{F}$, ${(T - \lambda I)}^{*} = T^{*} - \conj{\lambda}I$.

    $\lambda$ is not an eigenvalue of $T$ if and only if $T - \lambda I$ is invertible.

    $T - \lambda I$ is invertible if and only if there exists $S\in\lmap{V}$ such that $S(T - \lambda I) = (T - \lambda I)S = I$.

    $S(T - \lambda I) = (T - \lambda I)S = I$ if and only if $(T^{*} - \conj{\lambda}I)S^{*} = S^{*}(T^{*} - \conj{\lambda}I) = I$.

    $T^{*} - \conj{\lambda} I$ is invertible if and only if there exists $S^{*}\in\lmap{V}$ such that $(T^{*} - \conj{\lambda}I)S^{*} = S^{*}(T^{*} - \conj{\lambda}I) = I$.

    $T^{*} - \conj{\lambda} I$ is invertible if and only if $\conj{\lambda}$ is not an eigenvalue of $T^{*}$.

    Therefore $\lambda$ is not an eigenvalue of $T$ if and only if $\conj{\lambda}$ is not an eigenvalue of $T^{*}$. So $\lambda$ is an eigenvalue of $T$ if and only if $\conj{\lambda}$ is an eigenvalue of $T^{*}$.
\end{proof}
\newpage

% chapter7:sectionA:exercise4
\begin{exercise}
    Suppose $T\in\lmap{V}$ and $U$ is a subspace of $V$. Prove that
    \[
        \text{$U$ is invariant under $T$}\Longleftrightarrow \text{$U^{\bot}$ is invariant under $T^{*}$}.
    \]
\end{exercise}

\begin{proof}
    Let $u$ be an arbitrary vector in $U$ and $w$ be an arbitrary vector in $U^{\bot}$.

    If $U$ is invariant under $T$, then $Tu\in U$ and $\innerprod{u, T^{*}w} = \innerprod{Tu, w} = 0$. It follows that $T^{*}w$ is orthogonal to every $u\in U$. Therefore $T^{*}w\in U^{\bot}$ for every $w\in U^{\bot}$, so $U^{\bot}$ is invariant under $T^{*}$.

    If $U^{\bot}$ is invariant under $T^{*}$, then $T^{*}w\in U^{\bot}$ and $\innerprod{Tu, w} = \innerprod{u, T^{*}w} = 0$. It follows that $Tu$ is orthogonal to every $w\in U^{\bot}$. Therefore $Tu\in U$ for every $u\in U$, so $U$ is invariant under $T$.
\end{proof}
\newpage

% chapter7:sectionA:exercise5
\begin{exercise}\label{chapter7:sectionA:exercise5}
    Suppose $T\in\lmap{V, W}$. Suppose $e_{1}, \ldots, e_{n}$ is an orthonormal basis of $V$ and $f_{1}, \ldots, f_{m}$ is an orthonormal basis of $W$. Prove that
    \[
        \norm{Te_{1}}^{2} + \cdots + \norm{Te_{n}}^{2} = \norm{T^{*}f_{1}}^{2} + \cdots + \norm{T^{*}f_{m}}^{2}.
    \]
\end{exercise}

\begin{quote}
    The numbers $\norm{Te_{1}}^{2}, \ldots, \norm{Te_{n}}^{2}$ in the equation above depend on the orthonormal basis $e_{1}, \ldots, e_{n}$, but the right side of the equation does not depend on $e_{1}, \ldots, e_{n}$. Thus the equation above shows that the sum on the left side does not depend on which orthonormal basis $e_{1}, \ldots, e_{n}$ is used.
\end{quote}

\begin{proof}
    According to the definition of self-adjoint
    \[
        \sum^{n}_{j=1}\norm{Te_{j}}^{2} = \sum^{n}_{j=1}\innerprod{Te_{j}, Te_{j}} = \sum^{n}_{j=1}\innerprod{e_{j}, T^{*}(Te_{j})}.
    \]

    Since $Te_{j} = \innerprod{Te_{j}, f_{1}}f_{1} + \cdots + \innerprod{Te_{j}, f_{m}}f_{m}$, then
    \begin{align*}
        \sum^{n}_{j=1}\innerprod{e_{j}, T^{*}(Te_{j})} & = \sum^{n}_{j=1}\innerprod{e_{j}, T^{*}\left( \sum^{m}_{k=1}\innerprod{Te_{j}, f_{k}}f_{k} \right)} \\
                                                       & = \sum^{n}_{j=1}\innerprod{e_{j}, \sum^{m}_{k=1}\innerprod{Te_{j}, f_{k}} T^{*}f_{k}}               \\
                                                       & = \sum^{n}_{j=1}\sum^{m}_{k=1}\conj{\innerprod{Te_{j}, f_{k}}}\innerprod{e_{j}, T^{*}f_{k}}         \\
                                                       & =  \sum^{n}_{j=1}\sum^{m}_{k=1}\abs{\innerprod{T^{*}f_{k}, e_{j}}}^{2}                              \\
                                                       & = \sum^{m}_{k=1}\sum^{n}_{j=1}\abs{\innerprod{T^{*}f_{k}, e_{j}}}^{2}.
    \end{align*}

    By Parseval's theorem
    \[
        \sum^{m}_{k=1}\sum^{n}_{j=1}\abs{\innerprod{T^{*}f_{k}, e_{j}}}^{2} =  \sum^{m}_{k=1}\norm{T^{*}f_{k}}^{2}.
    \]

    Thus
    \[
        \norm{Te_{1}}^{2} + \cdots + \norm{Te_{n}}^{2} = \norm{T^{*}f_{1}}^{2} + \cdots + \norm{T^{*}f_{m}}^{2}.
    \]
\end{proof}
\newpage

% chapter7:sectionA:exercise6
\begin{exercise}
    Suppose $T\in\lmap{V, W}$. Prove that
    \begin{enumerate}[label={(\alph*)}]
        \item $T$ is injective $\Longleftrightarrow$ $T^{*}$ is surjective.
        \item $T$ is surjective $\Longleftrightarrow$ $T^{*}$ is injective.
    \end{enumerate}
\end{exercise}

\begin{proof}
    \begin{enumerate}[label={(\alph*)}]
        \item \begin{align*}
                  \text{$T$ is injective} & \Longleftrightarrow \kernel{T} = \{0\}            \\
                                          & \Longleftrightarrow {(\kernel{T})}^{\bot} = V     \\
                                          & \Longleftrightarrow \range{T^{*}} = V             \\
                                          & \Longleftrightarrow \text{$T^{*}$ is surjective}.
              \end{align*}
        \item \begin{align*}
                  \text{$T$ is surjective} & \Longleftrightarrow \range{T} = W                \\
                                           & \Longleftrightarrow {(\range{T})}^{\bot} = \{0\} \\
                                           & \Longleftrightarrow \kernel{T^{*}} = \{0\}       \\
                                           & \Longleftrightarrow \text{$T^{*}$ is injective}.
              \end{align*}
    \end{enumerate}
\end{proof}
\newpage

% chapter7:sectionA:exercise7
\begin{exercise}\label{chapter7:sectionA:exercise7}
    Suppose $T\in\lmap{V, W}$, then
    \begin{enumerate}[label={(\alph*)}]
        \item $\dim \kernel{T^{*}} = \dim \kernel{T} + \dim W - \dim V$.
        \item $\dim \range{T^{*}} = \dim \range{T}$.
    \end{enumerate}
\end{exercise}

\begin{proof}
    \begin{enumerate}[label={(\alph*)}]
        \item Because $\kernel{T^{*}} = {(\range{T})}^{\bot}$, it follows that $\dim \kernel{T^{*}} = \dim {(\range{T})}^{\bot}$.

              Moreover, $\dim {(\range{T})}^{\bot} = \dim W - \dim \range{T}$. By the fundamental theorem of linear maps, $\dim\range{T} = \dim V - \dim\kernel{T}$.

              Thus $\dim\kernel{T^{*}} = \dim W + \dim\kernel{T} - \dim V$.
        \item By (a) and the fundamental theorem of linear maps
        \[
            \dim\range{T^{*}} = \dim W - \dim\kernel{T^{*}} = \dim V - \dim\kernel{T} = \dim\range{T}.
        \]
    \end{enumerate}
\end{proof}
\newpage

% chapter7:sectionA:exercise8
\begin{exercise}
    Suppose $A$ is an $m$-by-$n$ matrix with entries in $\mathbb{F}$. Use (b) in Exercise~\ref{chapter7:sectionA:exercise7} to prove that the row rank of $A$ equals the column rank of $A$.
\end{exercise}

\begin{proof}
    Let $T\in \lmap{\mathbb{F}^{n}, \mathbb{F}^{m}}$ defined by $T: x\mapsto Ax$. The adjoint $T^{*}\in \lmap{\mathbb{F}^{m}, \mathbb{F}^{n}}$ is $y\mapsto A^{*}y$.
    \begin{align*}
        \text{column rank of $A$} & = \dim\range{T} \\
                                  & = \dim\range{T^{*}} \\
                                  & = \text{column rank of $A^{*}$} \\
                                  & = \text{row rank of $A$}.
    \end{align*}

    Thus the row rank of $A$ equals the column rank of $A$.
\end{proof}
\newpage

% chapter7:sectionA:exercise9
\begin{exercise}
    Prove that the product of two self-adjoint operators on $V$ is self-adjoint if and only if the two operators commute.
\end{exercise}

\begin{proof}
    Let $S, T$ be two self-adjoint operators on $V$.
    \begin{align*}
        \text{$ST$ is self-adjoint} & \Longleftrightarrow ST = {(ST)}^{*} \\
                                    & \Longleftrightarrow S^{*}T^{*} = {(ST)}^{*} & \text{(since $S, T$ are self-adjoint)} \\
                                    & \Longleftrightarrow S^{*}T^{*} = T^{*}S^{*} \\
                                    & \Longleftrightarrow ST = TS & \text{(since $S, T$ are self-adjoint)}        \\
                                    & \Longleftrightarrow \text{$S$ and $T$ commute.}
    \end{align*}
\end{proof}
\newpage

% chapter7:sectionA:exercise10
\begin{exercise}
    Suppose $\mathbb{F} = \mathbb{C}$ and $T \in \lmap{V}$. Prove that $T$ is self-adjoint if and only if
    \[
        \innerprod{Tv, v} = \innerprod{T^{*}v, v}
    \]

    for all $v\in V$.
\end{exercise}

\begin{proof}
    An operator $S$ on a complex vector space $V$ is $0$ if and only if $\innerprod{Sv, v} = 0$ for all $v\in V$.
    \begin{align*}
        \text{$T$ is self-adjoint} & \Longleftrightarrow T - T^{*} = 0 \\
                                   & \Longleftrightarrow \innerprod{(T - T^{*})v, v} = 0\,\forall v\in V \\
                                   & \Longleftrightarrow \innerprod{Tv, v} = \innerprod{T^{*}v, v}\,\forall v\in V.
    \end{align*}
\end{proof}
\newpage

% chapter7:sectionA:exercise11
\begin{exercise}
    Define an operator $S: \mathbb{F}^{2}\to \mathbb{F}^{2}$ by $S(w, z) = (-z, w)$.
    \begin{enumerate}[label={(\alph*)}]
        \item Find a formula for $S^{*}$.
        \item Show that $S$ is normal but not self-adjoint.
        \item Find all eigenvalues of $S$.
    \end{enumerate}
\end{exercise}

\begin{quote}
    If $\mathbb{F} = \mathbb{R}$, then $S$ is the operator on $\mathbb{R}^{2}$ of counterclockwise rotation by $90^{\circ}$.
\end{quote}

\begin{proof}
    \begin{enumerate}[label={(\alph*)}]
        \item \begin{align*}
            \innerprod{(w, z), S^{*}(x, y)} & = \innerprod{S(w, z), (x, y)} \\
                                            & = \innerprod{(-z, w), (x, y)} \\
                                            & = -z\conj{x} + w\conj{y} \\
                                            & = w\conj{y} + z\conj{(-x)} \\
                                            & = \innerprod{(w, z), (y, -x)}.
        \end{align*}

        Hence $S^{*}(x, y) = (y, -x)$.
        \item \[
            \begin{split}
                (SS^{*})(w, z) = S(z, -w) = (w, z), \\
                (S^{*}S)(w, z) = S^{*}(-z, w) = (w, z),
            \end{split}
        \]

        so $SS^{*} = S^{*}S$, which means $S$ is normal.
        \[
            S(1, 0) = (0, 1) \ne (0, -1) = S^{*}(1, 0)
        \]

        so $S$ is not self-adjoint.
        \item
        \[
            S^{2}(w, z) = S(-z, w) = (-w, -z)
        \]

        Therefore $z^{2} + 1$ is a polynomial multiple of the minimal polynomial of $S$. On the other hand, the minimal polynomial of $S$ cannot have degree $1$. So $z^{2} + 1$ is the minimal polynomial of $S$.

        If $\mathbb{F} = \mathbb{R}$, then $S$ has no eigenvalues. If $\mathbb{F} = \mathbb{C}$, the eigenvalues of $S$ are $\iota$ and $-\iota$.
    \end{enumerate}
\end{proof}
\newpage

% chapter7:sectionA:exercise12
\begin{exercise}
    An operator $B\in\lmap{V}$ is called skew if
    \[
        B^{*} = -B
    \]

    Suppose that $T\in\lmap{V}$. Prove that $T$ is normal if and only if there exist commuting operators $A$ and $B$ such that $A$ is self-adjoint, $B$ is a skew operator, and $T = A + B$.
\end{exercise}

\begin{proof}
    Let $A = \frac{T + T^{*}}{2}$ and $B = \frac{T - T^{*}}{2}$.
    \begin{align*}
        A^{*} & = \conj{\left(\frac{1}{2}\right)}(T^{*} + {(T^{*})}^{*}) = \frac{1}{2}(T^{*} + T) = A, \\
        B^{*} & = \conj{\left(\frac{1}{2}\right)}T^{*} + \conj{\left(\frac{-1}{2}\right)}{(T^{*})}^{*} = \frac{1}{2}(T^{*} - T) = -B.
    \end{align*}

    So $A$ is a self-adjoint operator, and $B$ is a skew operator.
    \begin{align*}
        AB - BA & = \frac{(T + T^{*})(T - T^{*}) - (T - T^{*})(T + T^{*})}{4} \\
                & = \frac{(T^{2} + T^{*}T - TT^{*} - {(T^{*})}^{2}) - (T^{2} - T^{*}T + TT^{*} - {(T^{*})}^{2})}{4} \\
                & = \frac{T^{*}T - TT^{*}}{2}.
    \end{align*}

    So $A$ and $B$ commute if and only if $T$ is normal.
\end{proof}
\newpage

% chapter7:sectionA:exercise13
\begin{exercise}
    Suppose $\mathbb{F} = \mathbb{R}$. Define $\mathcal{A}\in \lmap{\lmap{V}}$ by $\mathcal{A}T = T^{*}$ for all $T\in\lmap{V}$.
    \begin{enumerate}[label={(\alph*)}]
        \item Find all eigenvalues of $\mathcal{A}$.
        \item Find the minimal polynomial of $\mathcal{A}$.
    \end{enumerate}
\end{exercise}

\begin{proof}
    \begin{enumerate}[label={(\alph*)}]
        \item Assume $\lambda\in\mathbb{R}$ is an eigenvalue of $\mathcal{A}$, and $T$ is a corresponding eigenvector.
        \[
            T^{*} = \mathcal{A}T = \lambda T.
        \]

        It follows that $T = {(T^{*})}^{*} = {(\lambda T)}^{*} = \conj{\lambda}T^{*} = \lambda T^{*}$. Therefore $T = \lambda T^{*} = \lambda^{2}T$, and $(1 - \lambda^{2})T = 0$. Because $T$ is not the zero vector, we conclude that $\lambda^{2} = 1$, which means $\lambda = 1$ or $\lambda = -1$.

        If $T$ is nonzero and self-adjoint, then $\mathcal{A}T = T = 1T$. If $T$ is nonzero and skew, then $\mathcal{A}T = -T = (-1)T$.

        If $\dim V = 1$, the only eigenvalue of $\mathcal{A}$ is $1$. Otherwise the eigenvalues of $\mathcal{A}$ are $1$ and $-1$.
        \item Find the minimal polynomial of $\mathcal{A}$.

        If $\dim V = 1$, the minimal polynomial of $\mathcal{A}$ is $(z - 1)$. Otherwise, by (a), the minimal polynomial of $\mathcal{A}$ is a polynomial multiple of $z^{2} - 1$. Moreover $\mathcal{A}^{2}T = \mathcal{A}T^{*} = {(T^{*})}^{*} = T$. Hence the minimal polynomial of $\mathcal{A}$ is $z^{2} - 1$.
    \end{enumerate}
\end{proof}
\newpage

% chapter7:sectionA:exercise14
\begin{exercise}
    Define an inner product on $\mathscr{P}_{2}(\mathbb{R})$ by $\innerprod{p, q} = \int^{1}_{0}pq$. Define an operator $T\in\lmap{\mathscr{P}_{2}(\mathbb{R})}$ by
    \[
        T(ax^{2} + bx + c) = bx.
    \]

    \begin{enumerate}[label={(\alph*)}]
        \item Show that with this inner product, the operator $T$ is not self-adjoint.
        \item The matrix of $T$ with respect to the basis $1, x, x^{2}$ is
        \[
            \begin{pmatrix}
                0 & 0 & 0 \\
                0 & 1 & 0 \\
                0 & 0 & 0
            \end{pmatrix}.
        \]

        This matrix equals its conjugate transpose, even though $T$ is not self-adjoint. Explain why this is not a contradiction.
    \end{enumerate}
\end{exercise}

\begin{proof}
    \begin{enumerate}[label={(\alph*)}]
        \item \begin{align*}
            \innerprod{ax^{2} + bx + c, T^{*}(c)} & = \innerprod{T(ax^{2} + bx + c), c} \\
            & = \innerprod{bx, c} \\
            & = \int^{1}_{0}bcx dx \\
            & = \frac{bc}{2}, \\
            \innerprod{ax^{2} + bx + c, T(c)} & = \innerprod{ax^{2} + bx + c, 0} \\
                                              & = 0.
        \end{align*}

        Choose $b, c$ such that $bc\ne 0$, then $\innerprod{ax^{2} + bx + c, T^{*}(c)}\ne \innerprod{ax^{2} + bx + c, T(c)}$. Therefore $T^{*}(c)\ne T(c)$, which implies $T$ is not self-adjoint.
        \item This is not a contradiction because $1, x, x^{2}$ is not an orthonormal basis of $\mathscr{P}_{2}(\mathbb{R})$.
    \end{enumerate}
\end{proof}
\newpage

% chapter7:sectionA:exercise15
\begin{exercise}
    Suppose $T\in\lmap{V}$ is invertible. Prove that
    \begin{enumerate}[label={(\alph*)}]
        \item $T$ is self-adjoint $\Longleftrightarrow$ $T^{-1}$ is self-adjoint;
        \item $T$ is normal $\Longleftrightarrow$ $T^{-1}$ is normal.
    \end{enumerate}
\end{exercise}

\begin{proof}
    $T$ is invertible, then there exist a unique operator $T^{-1}$ such that $TT^{-1} = T^{-1}T = I$.

    Therefore $I = {(TT^{-1})}^{*} = {(T^{-1})}^{*}T^{*}$ and $I = {(T^{-1}T)}^{*} = T^{*}{(T^{-1})}^{*}$, so
    \[
        {(T^{-1})}^{*} = {(T^{*})}^{-1}.
    \]

    \begin{enumerate}[label={(\alph*)}]
        \item Because ${(T^{-1})}^{*} = {(T^{*})}^{-1}$, it follows that $T = T^{*} \Longleftrightarrow {(T^{-1})}^{*} = T^{-1}$. So $T$ is self-adjoint if and only if $T^{-1}$ is self-adjoint.
        \item Because ${(T^{-1})}^{*} = {(T^{*})}^{-1}$, we have
        \begin{align*}
            {(T^{-1})}^{*}T^{-1} & = {(T^{*})}^{-1}T^{-1} = {(TT^{*})}^{-1}, \\
            T^{-1}{(T^{-1})}^{*} & = T^{-1}{(T^{*})}^{-1} = {(T^{*}T)}^{-1}.
        \end{align*}

        Hence $TT^{*} = T^{*}T$ if and only if ${(T^{-1})}^{*}T^{-1} = T^{-1}{(T^{-1})}^{*}$. Thus $T$ is normal if and only if $T^{-1}$ is normal.
    \end{enumerate}
\end{proof}
\newpage

% chapter7:sectionA:exercise16
\begin{exercise}
    Suppose $\mathbb{F} = \mathbb{R}$.
    \begin{enumerate}[label={(\alph*)}]
        \item Show that the set of self-adjoint operators on $V$ is a subspace of $\lmap{V}$.
        \item What is the dimension of the subspace of $\lmap{V}$ in (a) [in terms of $\dim V$]?
    \end{enumerate}
\end{exercise}

\begin{proof}
    \begin{enumerate}[label={(\alph*)}]
        \item $0 = 0^{*}$ so $0$ is in the set of self-adjoint operator.

        If $S, T$ are self-adjoint operator, then ${(S + T)}^{*} = {S^{*} + T^{*}} = S + T$ and ${(\lambda S)}^{*} = \conj{\lambda}S^{*} = \lambda S^{*} = \lambda S$. So the set of self-adjoint operators on $V$ is closed under addition and scalar multiplication.

        Thus the set of self-adjoint operators on $V$ is a subspace of $\lmap{V}$.
        \item Let $n = \dim V$ and $e_{1}, \ldots, e_{n}$ be an orthonormal basis of $V$. If $T$ is a self-adjoint operator on $V$, then the matrix of $T$ with respect to $e_{1}, \ldots, e_{n}$ is symmetric (we are working with $\mathbb{F} = \mathbb{R}$).

        Let $E_{i,i}$ be the operator on $V$ such that $E_{i,i}e_{i} = e_{i}$ and $E_{i,i}e_{j} = 0$ if $j\ne i$.

        Let $E_{i,j}$ be the operator on $V$ where $i<j$ such that $E_{i,j}e_{i} = e_{j}$, $E_{i,j}e_{j} = e_{i}$ and $E_{i,j}e_{k} = 0$ if $k\notin\{i, j\}$.

        All operators $E_{i,i}, E_{i,j}$ are self-adjoint (there are $n(n + 1)/2$ of them) and they constitute an independent list. Moreover,
        \[
            T = \innerprod{Te_{1}, e_{1}}E_{1,1} + \cdots + \innerprod{Te_{n}, e_{n}}E_{n,n} + \sum_{1\leq i < j\leq n}\innerprod{Te_{i}, e_{j}}E_{i, j} = \sum_{1\leq i\leq j\leq n}\innerprod{Te_{i}, e_{j}}E_{i,j}.
        \]

        Therefore the list $E_{i,i}, E_{i,j}$ spans the subspace of self-adjoint operators on $V$, hence it is a basis of the subspace. Thus the dimension of the subspace in (a) is $\dim V \times (\dim V + 1)/2$.
    \end{enumerate}
\end{proof}
\newpage

% chapter7:sectionA:exercise17
\begin{exercise}
    Suppose $\mathbb{F} = \mathbb{C}$. Show that the set of self-adjoint operators on $V$ is not a subspace of $\lmap{V}$.
\end{exercise}

\begin{proof}
    Let $\lambda$ be a complex number but not a real number, then for every nonzero self-adjoint operator $T$ on $V$, ${(\lambda T)}^{*} = \conj{\lambda}T^{*} = \conj{\lambda}T \ne \lambda T$. So the set of self-adjoint operators on $V$ is not closed under scalar multiplication. Thus the set of self-adjoint operators on $V$ is not a subspace of $\lmap{V}$.
\end{proof}
\newpage

% chapter7:sectionA:exercise18
\begin{exercise}
    Suppose $\dim V\geq 2$. Show that the set of normal operators on $V$ is not a subspace of $\lmap{V}$.
\end{exercise}

\begin{proof}
    Let $e_{1}, \ldots, e_{n}$ be an orthonormal basis of $V$. I define $S$ and $T$ as follows:
    \[
        Se_{i} = \begin{cases}
            -e_{2} & \text{if $i = 1$} \\
            e_{1}  & \text{if $i = 2$} \\
            0      & \text{otherwise}
        \end{cases}\qquad
        Te_{i} = \begin{cases}
            e_{2} & \text{if $i = 1$} \\
            e_{1} & \text{if $i = 2$} \\
            0 & \text{otherwise}
        \end{cases}
    \]

    \begin{align*}
        \innerprod{a_{1}e_{1} + \cdots + a_{n}e_{n}, S^{*}(b_{1}e_{1} + \cdots + b_{n}e_{n})} & = \innerprod{S(a_{1}e_{1} + \cdots + a_{n}e_{n}), b_{1}e_{1} + \cdots + b_{n}e_{n}} \\
        & = \innerprod{a_{2}e_{1} + (-a_{1})e_{2}, b_{1}e_{1} + b_{2}e_{2} + \cdots + b_{n}e_{n}} \\
        & = a_{2}\conj{b_{1}} + (-a_{1})\conj{b_{2}} \\
        & = \innerprod{a_{1}e_{1} + a_{2}e_{2} + \cdots + a_{n}e_{n}, -b_{2}e_{1} + b_{1}e_{2}} \\
        \innerprod{a_{1}e_{1} + \cdots + a_{n}e_{n}, T^{*}(b_{1}e_{1} + \cdots + b_{n}e_{n})} & = \innerprod{T(a_{1}e_{1} + \cdots + a_{n}e_{n}), b_{1}e_{1} + \cdots + b_{n}e_{n}} \\
        & = \innerprod{a_{2}e_{1} + a_{1}e_{2}, b_{1}e_{1} + b_{2}e_{2} + \cdots + b_{n}e_{n}} \\
        & = a_{2}\conj{b_{1}} + a_{1}\conj{b_{2}} \\
        & = \innerprod{a_{1}e_{1} + a_{2}e_{2} + \cdots + a_{n}e_{n}, b_{2}e_{1} + b_{1}e_{2}}
    \end{align*}

    So
    \[
        \begin{split}
            S^{*}(b_{1}e_{1} + \cdots + b_{n}e_{n}) = -b_{2}e_{1} + b_{1}e_{2} \\
            T^{*}(b_{1}e_{1} + \cdots + b_{n}e_{n}) = b_{2}e_{1} + b_{1}e_{2} \\
        \end{split}
    \]
    \begin{align*}
        (SS^{*})(b_{1}e_{1} + \cdots + b_{n}e_{n}) & = S(-b_{2}e_{1} + b_{1}e_{2}) = b_{1}e_{1} + b_{2}e_{2} \\
        (S^{*}S)(b_{1}e_{1} + \cdots + b_{n}e_{n}) & = S^{*}(-b_{1}e_{2} + b_{2}e_{1}) = b_{1}e_{1} + b_{2}e_{2} \\
        (TT^{*})(b_{1}e_{1} + \cdots + b_{n}e_{n}) & = T(b_{2}e_{1} + b_{1}e_{2}) = b_{1}e_{1} + b_{2}e_{2} \\
        (T^{*}T)(b_{1}e_{1} + \cdots + b_{n}e_{n}) & = T^{*}(b_{1}e_{2} + b_{2}e_{1}) = b_{1}e_{1} + b_{2}e_{2}
    \end{align*}

    So $S$ and $T$ are normal operators. $R = S + T$.
    \begin{align*}
        \innerprod{a_{1}e_{1} + \cdots + a_{n}e_{n}, R^{*}(b_{1}e_{1} + \cdots + b_{n}e_{n})} & = \innerprod{R(a_{1}e_{1} + \cdots + a_{n}e_{n}), b_{1}e_{1} + \cdots + b_{n}e_{n}} \\
        & = \innerprod{2a_{2}e_{1}, b_{1}e_{1} + \cdots + b_{n}e_{n}} \\
        & = 2a_{2}\conj{b_{1}} \\
        & = \innerprod{a_{1}e_{1} + \cdots + a_{n}e_{n}, 2b_{1}e_{2}}
    \end{align*}

    Hence $R^{*}(b_{1}e_{1} + \cdots + b_{n}e_{n}) = 2b_{1}e_{2}$.
    \begin{align*}
        (RR^{*})(b_{1}e_{1} + \cdots + b_{n}e_{n}) & = R(2b_{1}e_{2}) = 2b_{1}e_{1} \\
        (R^{*}R)(b_{1}e_{1} + \cdots + b_{n}e_{n}) & = R^{*}(b_{1}e_{2} + b_{2}e_{1}) = 2b_{2}e_{2}
    \end{align*}

    So $R$ and $R^{*}$ does not commute. Therefore the set of normal operators on $V$ is not closed under addition. Thus the set of normal operators on $V$ is not a subspace of $\lmap{V}$ if $\dim V\geq 2$.
\end{proof}
\newpage

% chapter7:sectionA:exercise19
\begin{exercise}
    Suppose $T\in\lmap{V}$ and $\norm{T^{*}v} \leq \norm{Tv}$ for every $v\in V$. Prove that $T$ is normal.
\end{exercise}

\begin{quote}
    This exercise fails on infinite-dimensional inner product spaces, leading to what are called hyponormal operators, which have a well-developed theory.
\end{quote}

\begin{proof}
    Let $v$ be a nonzero vector in $V$. Let $e_{1} = v/\norm{v}$ and $e_{1}, \ldots, e_{n}$ be an orthonormal basis of $V$. By Exercise~\ref{chapter7:sectionA:exercise5}, we have
    \[
        \norm{Te_{1}}^{2} + \cdots + \norm{Te_{n}}^{2} = \norm{T^{*}e_{1}}^{2} + \cdots + \norm{T^{*}e_{n}}^{2}.
    \]

    According to the hypothesis, $\norm{Te_{i}}^{2}\geq \norm{T^{*}e_{i}}^{2}$ for each $i\in\{1,\ldots, n\}$. Together with the inequality, we conclude that $\norm{Te_{i}}^{2} = \norm{Te_{i}}^{2}$ for each $i\in\{1,\ldots, n\}$.

    Therefore
    \[
        \norm{Tv} = \norm{T\left(\norm{v}\frac{v}{\norm{v}}\right)} = \norm{\norm{v}T\left(\frac{v}{\norm{v}}\right)} = \norm{\norm{v}T^{*}\left(\frac{v}{\norm{v}}\right)} = \norm{T^{*}v}.
    \]

    If $v = 0$ then $\norm{Tv} = \norm{T^{*}v} = 0$. Hence $\norm{Tv} = \norm{T^{*}v}$ for every $v\in V$. This means $T$ is normal.
\end{proof}
\newpage

% chapter7:sectionA:exercise20
\begin{exercise}
    Suppose $P\in\lmap{V}$ is such that $P^{2} = P$. Prove that the following are equivalent.
    \begin{enumerate}[label={(\alph*)}]
        \item $P$ is self-adjoint.
        \item $P$ is normal.
        \item There is a subspace $U$ of $V$ such that $P = P_{U}$.
    \end{enumerate}
\end{exercise}

\begin{proof}
    I will show that $(a) \implies (b) \implies (c) \implies (a)$.

    Suppose (a) is true, then $P^{*}P = PP = PP^{*}$, so $P$ is normal. Therefore (b) is true.

    Suppose (b) is true. Since $P^{2} = P$, then $V = \kernel{P}\oplus\range{P}$ and every vector $v$ in $V$ admits the decomposition $v = (v - Pv) + Pv$. $\kernel{P}$ is the eigenspace with respect to the eigenvalue $0$, $\range{P}$ is the eigenspace with respect to the eigenvalue $1$.

    Since $P$ is normal, then the eigenvectors with respect to different eigenvalues are orthogonal. Let $e_{1}, \ldots, e_{m}$ be an orthonormal basis of $\kernel{P}$ and $f_{1}, \ldots, f_{n}$ be an orthonormal basis of $\range{P}$, then $e_{i}$ and $f_{j}$ are orthogonal, for every $i\in\{1,\ldots, m\}$ and $j\in\{ 1,\ldots,n \}$. Therefore $e_{1}, \ldots, e_{m}$, $f_{1}, \ldots, f_{n}$ is an orthonormal basis of $V$ and each of these vectors are eigenvectors of $P$. Let $U = \operatorname{span}(f_{1}, \ldots, f_{n})$.
    \begin{align*}
        v & = \innerprod{v,e_{1}}e_{1} + \cdots + \innerprod{v,e_{m}}e_{m} + \innerprod{v, f_{1}}f_{1} + \cdots + \innerprod{v, f_{n}}f_{n} \\
        Pv & = 0 + \innerprod{v, f_{1}}Pf_{1} + \cdots + \innerprod{v, f_{n}}Pf_{n} \\
           & = \innerprod{v, e_{1}}P_{U}e_{1} + \cdots + \innerprod{v, e_{n}}P_{U}e_{n} + \innerprod{v, f_{1}}P_{U}f_{1} + \cdots + \innerprod{v, f_{n}}P_{U}f_{n} \\
           & = P_{U}v.
    \end{align*}

    Hence $P = P_{U}$, so (c) is true.

    Suppose (c) is true. $V = U\oplus U^{\bot}$.

    Let $v\in V$ and $u\in U$, then $v - u\in U^{\bot}$. For every $w\in V$
    \begin{align*}
        \innerprod{Pv, w} & = \innerprod{u, w} = \innerprod{u, P_{U}w} = \innerprod{P_{U}v, P_{U}w}, \\
        \innerprod{P^{*}v, w} & = \innerprod{v, Pw} = \innerprod{v, P_{U}w} = \innerprod{P_{U}v, P_{U}w}.
    \end{align*}

    Hence $\innerprod{Pv - P^{*}v, w} = 0$ for every $v, w\in V$. So $Pv = P^{*}v$ for every $v\in V$. Thus $P = P^{*}$, which means (a) is true.
\end{proof}
\newpage

% chapter7:sectionA:exercise21
\begin{exercise}
    Suppose $D: \mathscr{P}_{8}(\mathbb{R}) \to \mathscr{P}_{8}(\mathbb{R})$ is the differentiation operator defined by $Dp = p'$. Prove that there does not exist an inner product on $\mathscr{P}_{8}(\mathbb{R})$ that makes $D$ a normal operator.
\end{exercise}

\begin{proof}
    Assume that $D$ is a normal operator for some inner product $\innerprod{\cdot, \cdot}$ on $\mathscr{P}_{8}(\mathbb{R})$.
    \[
        \innerprod{x, D^{*}(1)} = \innerprod{Dx, 1} = \innerprod{1, 1} > 0.
    \]

    Therefore $D^{*}(1)\ne 0$. However, on the other hand
    \[
        \innerprod{D^{*}1, D^{*}1} = \innerprod{(DD^{*})1, 1} = \innerprod{(D^{*}D)1, 1} = \innerprod{D^{*}0, 1} = \innerprod{0, 1} = 0.
    \]

    This contradicts the property of definiteness of inner product so the assumption is false. Thus $D$ is not a normal operator, no matter which inner product are used on $\mathscr{P}_{8}(\mathbb{R})$.
\end{proof}
\newpage

% chapter7:sectionA:exercise22
\begin{exercise}
    Give an example of an operator $T\in\lmap{\mathbb{R}^{3}}$ such that $T$ is normal but not self-adjoint.
\end{exercise}

\begin{proof}
    Let $T(x, y, z) = (-y, x, 0)$.
    \begin{align*}
        \innerprod{(x_{1}, y_{1}, z_{1}), T^{*}(x_{2}, y_{2}, z_{2})} & = \innerprod{T(x_{1}, y_{1}, z_{1}), (x_{2}, y_{2}, z_{2})} \\
        & = \innerprod{(-y_{1}, x_{1}, 0), (x_{2}, y_{2}, z_{2})} \\
        & = (-y_{1})x_{2} + x_{1}y_{2} \\
        & = \innerprod{(x_{1}, y_{1}, z_{1}), (y_{2}, -x_{2}, 0)}
    \end{align*}

    so $T^{*}(x, y, z) = (y, -x, 0)$. Hence $T$ is not self-adjoint.
    \begin{align*}
        (TT^{*})(x, y, z) & = T(y, -x, 0) = (x, y, 0) \\
        (T^{*}T)(x, y, z) & = T^{*}(-y, x, 0) = (x, y, 0)
    \end{align*}

    so $TT^{*} = T^{*}T$. Hence $T$ is normal.
\end{proof}
\newpage

% chapter7:sectionA:exercise23
\begin{exercise}
    Suppose $T$ is a normal operator on $V$. Suppose also that $v, w \in V$ satisfy the equations
    \[
        \norm{v} = \norm{w} = 2,\quad Tv = 3v,\quad Tw = 4w.
    \]

    Show that $\norm{T(v + w)} = 10$.
\end{exercise}

\begin{proof}
    Since $\norm{v}$ and $\norm{w}$ are positive, $v$ and $w$ are nonzero. Because $Tv = 3v$ and $Tw = 4w$ so $v$ and $w$ are eigenvectors of $T$ with respect to two different eigenvalues $3$ and $4$. Therefore $v$ and $w$ are orthogonal (because eigenvectors of two different eigenvalues of a normal operator are orthogonal).
    \begin{align*}
        \norm{T(v + w)}^{2} & = \norm{Tv + Tw}^{2} = \norm{3v + 4w}^{2} \\
                            & = \norm{3v}^{2} + \norm{4w}^{2} & \text{(Pythagorean theorem)} \\
                            & = 9\cdot 4 + 16\cdot 4 = 100.
    \end{align*}

    Thus $\norm{T(v + w)} = 10$.
\end{proof}
\newpage

% chapter7:sectionA:exercise24
\begin{exercise}\label{chapter7:sectionA:exercise24}
    Suppose $T\in\lmap{V}$ and
    \[
        a_{0} + a_{1}z + a_{2}z^{2} + \cdots + a_{m-1}z^{m-1} + z^{m}
    \]

    is the minimal polynomial of $T$. Prove that the minimal polynomial of $T^{*}$ is
    \[
        \conj{a_{0}} + \conj{a_{1}}z + \conj{a_{2}}z^{2} + \cdots + \conj{a_{m-1}}z^{m-1} + z^{m}
    \]
\end{exercise}

\begin{quote}
    This exercise shows that the minimal polynomial of $T^{*}$ equals the minimal polynomial of $T$ if $\mathbb{F} = \mathbb{R}$.
\end{quote}

\begin{proof}
    Consider two polynomials $p(z)$ and $q(z) = \conj{p(\conj{z})}$. Then $p(z) = \conj{q(\conj{z})}$.
    \begin{align*}
        p(T)v = 0\,\forall v\in V & \Longleftrightarrow \innerprod{p(T)v, w} = 0,\forall v, w\in V \\
                                  & \Longleftrightarrow \innerprod{v, {(p(T))}^{*}w} = 0\,\forall v, w\in V \\
                                  & \Longleftrightarrow \innerprod{v, q(T)w} = 0\,\forall v, w\in V \\
                                  & \Longleftrightarrow q(T)w = 0\,\forall w\in V.
    \end{align*}

    Denote the minimal polynomials of $T$ and $T^{*}$ by $\mu_{T}$ and $\mu_{T^{*}}$, respectively.

    Let $p = \mu_{T}$, then $q$ is a polynomial multiple of the minimal polynomial of $T^{*}$, so $\deg \mu_{T^{*}}\leq \deg q = \deg p = \deg \mu_{T}$.

    Let $q = \mu_{T^{*}}$, then $p$ is a polynomial multiple of the minimal polynomial of $T$, so $\deg \mu_{T}\leq \deg p = \deg q = \deg \mu_{T^{*}}$.

    Therefore $\deg \mu_{T} = \deg \mu_{T^{*}}$ and $\mu_{T^{*}}(z) = \conj{\mu_{T}(\conj{z})}$ and the result follows.
\end{proof}
\newpage

% chapter7:sectionA:exercise25
\begin{exercise}
    Suppose $T \in \lmap{V}$. Prove that $T$ is diagonalizable if and only if $T^{*}$ is diagonalizable.
\end{exercise}

\begin{proof}
    If $T$ is diagonalizable, then the minimal polynomial $p_{T}$ of $T$ is a product of different monic polynomials of degree $1$
    \[
        p_{T}(z) = (z - \lambda_{1})\cdots (z - \lambda_{n})
    \]

    where $n\geq 0$. By Exercise~\ref{chapter7:sectionA:exercise24}, the minimal polynomial of $T^{*}$ is
    \[
        p_{T^{*}}(z) = (z - \conj{\lambda_{1}})\cdots (z - \conj{\lambda_{n}})
    \]

    and $\conj{\lambda_{1}}, \ldots, \conj{\lambda_{n}}$ are pairwise distinct. So $T^{*}$ is diagonalizable.

    To prove statement in the other direction, we apply the previous direction. If $T^{*}$ is diagonalizable, then $T = {(T^{*})}^{*}$ is diagonalizable.

    Thus $T$ is diagonalizable if and only if $T^{*}$ is diagonalizable.
\end{proof}
\newpage

% chapter7:sectionA:exercise26
\begin{exercise}
    Fix $u, x\in V$. Define $T\in \lmap{V}$ by $Tv = \innerprod{v, u}x$ for every $v\in V$.
    \begin{enumerate}[label={(\alph*)}]
        \item Prove that if $V$ is a real vector space, then $T$ is self-adjoint if and only if the list $u, x$ is linearly dependent.
        \item Prove that $T$ is normal if and only if the list $u, x$ is linearly dependent.
    \end{enumerate}
\end{exercise}

\begin{proof}
    \begin{align*}
        \innerprod{v, T^{*}w} & = \innerprod{Tv, w} = \innerprod{\innerprod{v, u}x, w} \\
                              & = \innerprod{v, u}\innerprod{x, w} \\
                              & = \innerprod{v, \conj{\innerprod{x, w}}u} \\
                              & = \innerprod{v, \innerprod{w, x}u}
    \end{align*}

    Therefore $T^{*}w = \innerprod{w, x}u$.
    \begin{enumerate}[label={(\alph*)}]
        \item If $T$ is self-adjoint, then $\innerprod{v, x}u = \innerprod{v, u}x$ for every $v\in V$. If either $u$ or $x$ is zero, then $u, x$ is linearly independent. If $u$ and $x$ are nonzero, then $\innerprod{v, x}$ and $\innerprod{v, u}$ are not simultaneously zero, so $u, x$ is linearly dependent.

        If $u, x$ is linearly dependent, then $u$ is a scalar multiple of $x$ or vice versa.
        \begin{itemize}
            \item If $u = \lambda x$ (notice that $\lambda\in\mathbb{R}$), then
            \[
                Tv = \innerprod{v, u}x = \innerprod{v, \lambda x}x = \innerprod{v, x}\lambda x = \innerprod{v, x}u = T^{*}v
            \]

            for every $v\in V$.
            \item If $x = \lambda u$ (notice that $\lambda\in\mathbb{R}$), then
            \[
                T^{*}v = \innerprod{v, x}u = \innerprod{v, \lambda u}u = \innerprod{v, u}\lambda u = \innerprod{v, u}x = Tv.
            \]
        \end{itemize}

        Therefore $T$ is self-adjoint.
        \item \begin{align*}
            TT^{*} = T^{*}T & \Longleftrightarrow (TT^{*})v = (T^{*}T)v\,\forall v\in V \\
                            & \Longleftrightarrow T(\innerprod{v, x}u) = T^{*}(\innerprod{v, u}x)\,\forall v\in V \\
                            & \Longleftrightarrow \innerprod{v, x}Tu = \innerprod{v, u}T^{*}x \,\forall v\in V \\
                            & \Longleftrightarrow \innerprod{v, x}\innerprod{u, u}x = \innerprod{v, u}\innerprod{x, x}u\,\forall v\in V.
        \end{align*}

        If $T$ is normal, then $\innerprod{v, x}\innerprod{u, u}x = \innerprod{v, u}\innerprod{x, x}u\,\forall v\in V$. If either $u$ or $x$ is zero, then $u, x$ is linearly dependent. If $u$ and $x$ are non zero, then $\innerprod{v, x}\innerprod{u, u}$ and $\innerprod{v, u}\innerprod{x, x}$ are not simultaneously zero, so $u, x$ is linearly dependent.

        If $u, x$ is linearly dependent, then $u$ is a scalar multiple of $x$ or vice versa.
        \begin{itemize}
            \item If $u = \lambda x$ (notice that $\lambda\in\mathbb{R}$), then
            \begin{align*}
                \innerprod{v, u}\innerprod{x, x}u & = \innerprod{v, \lambda x}\innerprod{x, x}\lambda x \\
                & = \innerprod{v, x}\innerprod{x, x}\lambda^{2} x \\
                & = \innerprod{v, x}\innerprod{\lambda x, \lambda x}x \\
                & = \innerprod{v, x}\innerprod{u, u}x
            \end{align*}

            for every $v\in V$.
            \item If $x = \lambda u$ (notice that $\lambda\in\mathbb{R}$), then
            \begin{align*}
                \innerprod{v, x}\innerprod{u, u}x & = \innerprod{v, \lambda u}\innerprod{u, u}\lambda u \\
                & = \innerprod{v, u}\innerprod{u, u}\lambda^{2}u \\
                & = \innerprod{v, u}\innerprod{\lambda u, \lambda u}u \\
                & = \innerprod{v, u}\innerprod{x, x}u.
            \end{align*}

            for every $v\in V$.
        \end{itemize}

        So $T$ is normal.
    \end{enumerate}
\end{proof}
\newpage

% chapter7:sectionA:exercise27
\begin{exercise}\label{chapter7:sectionA:exercise27}
    Suppose $T\in\lmap{V}$ is normal. Prove that
    \[
        \kernel{T^{k}} = \kernel{T}\quad\text{and}\quad \range{T^{k}} = \range{T}
    \]

    for every positive integer $k$.
\end{exercise}

\begin{proof}
    I give a proof using mathematical induction.

    The statement is true for $k = 1$, since $\kernel{T} = \kernel{T}$.

    Assume the statement is true for $n$. We have $\kernel{T^{n}}\subseteq\kernel{T^{n+1}}$. Because $T$ is normal, it follows that $T^{*}$ is also normal.
    \[
        V = \kernel{T^{*}}\oplus{(\kernel{T^{*}})}^{\bot} = \kernel{T^{*}}\oplus\range{T} = \kernel{T^{*}}\oplus\range{T^{*}}
    \]

    where ${(\kernel{T^{*}})}^{\bot} = \range{T} = \range{T^{*}}$.

    Let $v$ be a vector in $\kernel{T^{n+1}}$ then $T^{n+1}v = 0$. $T^{n+1}v = 0$ so $\innerprod{T^{n+1}v, w} = 0$ for every $w\in V$.

    Because $V$ is the orthogonal sum of $\kernel{T^{*}}$ and $\range{T^{*}}$, there exist unique vectors $u\in \kernel{T^{*}}$ and $\hat{u}\in\range{T^{*}}$ such that $w = u + \hat{u}$. Because $\hat{u}\in \range{T^{*}}$, there exists a vector $\hat{v}\in V$ such that $T^{*}\hat{v} = \hat{u}$.
    \begin{align*}
        \innerprod{T^{n}v, w} & = \innerprod{T^{n}v, u + \hat{u}} \\
                              & = \innerprod{T^{n}v, u} + \innerprod{T^{n}v, \hat{u}} \\
                              & = \innerprod{T^{n}v, u} + \innerprod{T^{n}v, T^{*}\hat{v}} \\
                              & = 0 + \innerprod{T^{n+1}v, \hat{v}} \\
                              & = \innerprod{0, \hat{v}} = 0
    \end{align*}

    where $\innerprod{T^{n}v, u} = 0$ because $T^{n}v\in\range{T}$ and $u\in\kernel{T^{*}} = {(\range{T})}^{\bot}$.

    So $\innerprod{T^{n}v, w} = 0$ for every $w\in V$, hence $T^{n}v = 0$, which precisely means $v\in\kernel{T^{n}}$. Therefore $\kernel{T^{n+1}}\subseteq \kernel{T^{n}}$.

    Hence $\kernel{T^{n}} = \kernel{T^{n+1}}$. According to the induction hypothesis, $\kernel{T^{n+1}} = \kernel{T^{n}} = \kernel{T}$.

    By the principle of mathematical induction, $\kernel{T^{k}} = \kernel{T}$ for every positive integer $k$.

    Because $T$ is normal, then so is $T^{k}$, and we have
    \begin{align*}
        \range{T^{k}} & = \range{(T^{k})}^{*} & \text{(because $T^{k}$ is normal)} \\
                      & = {(\kernel{T^{k}})}^{\bot} & \text{(range of the adjoint map)} \\
                      & = {(\kernel{T})}^{\bot} & \text{(because $\kernel{T^{k}} = \kernel{T}$)} \\
                      & = \range{T^{*}}  & \text{(range of the adjoint map)} \\
                      & = \range{T} & \text{(because $T$ is normal)}
    \end{align*}

    Thus $\kernel{T^{k}} = \kernel{T}$ and $\range{T^{k}} = \range{T}$ for every positive integer $k$.
\end{proof}
\newpage

% chapter7:sectionA:exercise28
\begin{exercise}
    Suppose $T\in\lmap{V}$ is normal. Prove that if $\lambda\in\mathbb{F}$, then the minimal polynomial of $T$ is not a polynomial multiple of ${(x - \lambda)}^{2}$.
\end{exercise}

\begin{quote}
    This exercise, together with the fundamental theorem of algebra, and the necessarily and sufficient condition of an operator to be diagonalizable give another proof for the complex spectral theorem.
\end{quote}

\begin{proof}
    Assume that the minimal polynomial of $T$ is divisible by ${(x - \lambda)}^{2}$ for some $\lambda\in\mathbb{F}$, then the minimal polynomial $\mu_{T}$ of $T$ is of the form $\mu_{T}(x) = {(x - \lambda)}^{2}p(x)$.

    Because ${(x - \lambda)}^{2}p(x)$ is the minimal polynomial of $T$, then there exists a vector $v\in V$ such that $(T - \lambda I)p(T)v \ne 0$. It follows that $p(T)v \ne 0$. On the other hand, ${(T - \lambda I)}^{2}(p(T)v) = 0$ so $p(T)v\in \kernel{{(T - \lambda I)}^{2}}$ but $p(T)v\notin \kernel{(T - \lambda I)}$.

    Since $T$ is normal, then $(T - \lambda I)$ is also normal. By Exercise~\ref{chapter7:sectionA:exercise27}, $\kernel{(T - \lambda I)} = \kernel{{(T - \lambda I)}^{2}}$. Hence $p(T)v\in \kernel{{(T - \lambda I)}^{2}}$ but $p(T)v\notin \kernel{(T - \lambda I)}$ is indeed a contradiction, so the assumption is false.

    Thus the minimal polynomial of a normal operator does not have a multiple root.
\end{proof}
\newpage

% chapter7:sectionA:exercise29
\begin{exercise}
    Prove or give a counterexample: If $T\in \lmap{V}$ and there is an orthonormal basis $e_{1}, \ldots, e_{n}$ of $V$ such that $\norm{Te_{k}} = \norm{T^{*}e_{k}}$ for each $k = 1,\ldots, n$, then $T$ is normal.
\end{exercise}

\begin{proof}
    I give a counterexample.

    Let $V = \mathbb{C}^{2}$, $e_{1}, e_{2}$ is the standard basis, and $T(x, y) = (x + y, - x - y)$ then $T^{*}(x, y) = (x - y, x - y)$.

    $\norm{Te_{1}} = \norm{T^{*}e_{1}} = \sqrt{2}$, $\norm{Te_{2}} = \norm{T^{*}e_{2}} = \sqrt{2}$. However
    \[
        \norm{T(e_{1} + e_{2})} = \norm{T(1, 1)} = \norm{(2, -2)} = \sqrt{8} \ne 0 = \norm{T^{*}(1, 1)} = \norm{T^{*}(e_{1} + e_{2})}
    \]

    so $T$ is not a normal operator.
\end{proof}
\newpage

% chapter7:sectionA:exercise30
\begin{exercise}
    Suppose that $T\in\lmap{\mathbb{F}^{3}}$ is normal and $T(1, 1, 1) = (2, 2, 2)$. Suppose $(z_{1}, z_{2}, z_{3})\in\kernel{T}$. Prove that $z_{1} + z_{2} + z_{3} = 0$.
\end{exercise}

\begin{proof}
    If $z_{1} = z_{2} = z_{3} = 0$ then $z_{1} + z_{2} + z_{3} = 0$.

    If $z_{1}, z_{2}, z_{3}$ are not simultaneously zero, then $(z_{1}, z_{2}, z_{3})$ is an eigenvector of $T$ corresponding to the eigenvalue $0$. $(1, 1, 1)$ is an eigenvector of $T$ corresponding to the eigenvalue $2$. Because $T$ is normal, it follows that the eigenvectors of $T$ with respect to different eigenvalues of $T$ are orthogonal, so
    \[
        0 = \innerprod{(z_{1}, z_{2}, z_{3}), (1, 1, 1)} = z_{1} + z_{2} + z_{3}.
    \]

    Thus $z_{1} + z_{2} + z_{3} = 0$.
\end{proof}
\newpage

% chapter7:sectionA:exercise31
\begin{exercise}\label{chapter7:sectionA:exercise31}
    Fix a positive integer $n$. In the inner product space of continuous real-valued functions on $[-\pi, \pi]$ with inner product $\innerprod{f, g} = \int^{\pi}_{-\pi}fg$, let
    \[
        V = \operatorname{span}(1, \cos x, \cos 2x, \ldots, \cos nx, \sin x, \sin 2x, \ldots, \sin nx).
    \]

    \begin{enumerate}[label={(\alph*)}]
        \item Define $D\in\lmap{V}$ by $Df = f'$. Show that $D^{*} = -D$. Conclude that $D$ is normal but not self-adjoint.
        \item Define $T\in\lmap{V}$ by $Tf = f''$. Show that $T$ is self-adjoint.
    \end{enumerate}
\end{exercise}

\begin{proof}
    An orthonormal basis of $V$ is
    \[
        \frac{1}{\sqrt{2\pi}}, \frac{\cos x}{\sqrt{\pi}}, \frac{\cos 2x}{\sqrt{\pi}}, \ldots, \frac{\cos nx}{\sqrt{\pi}}, \frac{\sin x}{\sqrt{\pi}}, \frac{\sin 2x}{\sqrt{\pi}}, \ldots, \frac{\sin nx}{\sqrt{\pi}}.
    \]

    The matrix of $D$ with respect to this basis has $(2n+1)$ rows and $(2n+1)$ columns, where
    \[
        {\mathcal{M}(D)}_{1+k, 1+2k} = k\qquad {\mathcal{M}(D)}_{1+2k, 1+k} = -k
    \]

    for $1\leq k\leq n$, and the other entries are zero. $\mathcal{M}(D) + {(\mathcal{M}(D))}^{*} = 0$ so $D^{*} = -D$. Therefore $D^{*}D = (-D)D = D(-D) = DD^{*}$ and $D^{*} = -D\ne D$, hence $D$ is normal but not self-adjoint.

    $T = D^{2}$ so $T^{*} = D^{*}D^{*} = (-D)(-D) = D^{2}$, therefore $T = T^{*}$, so $T$ is self-adjoint.
\end{proof}
\newpage

% chapter7:sectionA:exercise32
\begin{exercise}
    Suppose $T: V \to W$ is a linear map. Show that under the standard identification of $V$ with $V'$ (see 6.58) and the corresponding identification of $W$ with $W'$, the adjoint map $T^{*}: W\to V$ corresponds to the dual map $T': W'\to V'$. More precisely, show that
    \[
        T'(\varphi_{w}) = \varphi_{T^{*}w}
    \]

    for all $w\in W$, there $\varphi_{w}$ and $\varphi_{T^{*}w}$ are defined as in 6.58.
\end{exercise}

\begin{proof}
    For every vector $v\in V$, we have
    \begin{align*}
        T'(\varphi_{w})(v) & = \varphi_{w}(Tv) & \text{(definition of dual map)} \\
                           & = \innerprod{Tv, w} & \text{(definition of $\varphi_{w}$)} \\
                           & = \innerprod{v, T^{*}w} & \text{(definition of adjoint map)} \\
                           & = \varphi_{T^{*}w}(v) & \text{(definition of $\varphi_{T^{*}w}$)}
    \end{align*}

    so $T'(\varphi_{w}) = \varphi_{T^{*}w}$. Therefore the adjoint map $T^{*}$ corresponds to the dual map $T'$.
\end{proof}
\newpage

\section{The Spectral Theorem}

\section{Positive Operators}

\section{Isometries, Unitary Operators, and Matrix Factorization}

\section{Singular Value Decomposition}

\section{Consequences of Singular Value Decomposition}


\chapter{Operators on Complex Vector Spaces}

\section{Generalized Eigenvectors and Nilpotent Operators}

\section{Decomposition of an Operator}


\chapter{Multilinear Algebra and Determinants}

\section{Bilinear Forms and Quadratic Forms}

% chapter9:sectionA:exercise1
\begin{exercise}\label{chapter9:sectionA:exercise1}
    Prove that if $\beta$ is a bilinear form on $\mathbb{F}$, then there exists $c\in\mathbb{F}$ such that
    \[
        \beta(x, y) = cxy
    \]

    for all $x, y\in\mathbb{F}$.
\end{exercise}

\begin{proof}
    For all $x, y\in\mathbb{F}$
    \[
        \beta(x, y) = \beta(x\cdot 1, y\cdot 1) = x\beta(1, y\cdot 1) = xy\beta(1,1).
    \]

    Let $c = \beta(1, 1)$, then $\beta(x, y) = cxy$ for every $x, y\in\mathbb{F}$.
\end{proof}
\newpage

% chapter9:sectionA:exercise2
\begin{exercise}\label{chapter9:sectionA:exercise2}
    Let $n = \dim V$. Suppose $\beta$ is a bilinear form on $V$. Prove that there exist $\varphi_{1}, \ldots, \varphi_{n}, \tau_{1}, \ldots, \tau_{n}\in V'$ such that
    \[
        \beta(u, v) = \varphi_{1}(u)\cdot\tau_{1}(v) + \cdots + \varphi_{n}(u)\cdot\tau_{n}(v)
    \]

    for all $u, v\in V$.
\end{exercise}

\begin{quote}
    This exercise shows that if $n = \dim V$, then every bilinear form on V is of
    the form given by the last bullet point of Example 9.2.
\end{quote}

\begin{proof}
    Let $e_{1}, \ldots, e_{n}$ be a basis of $V$. For every vector $v$ in $V$, there exist unique scalars $a_{1}, \ldots, a_{n}$ such that
    \[
        v = a_{1}e_{1} + \cdots + a_{n}e_{n}.
    \]

    For every $k\in\{1,\ldots,n\}$, the mappings
    \begin{align*}
        \tau_{k}    & : v \mapsto a_{k}           \\
        \varphi_{k} & : v \mapsto \beta(v, e_{k})
    \end{align*}

    are linear functionals on $V$. From these, we have
    \begin{align*}
        \beta(u, v) & = \beta(u, \tau_{1}(v)e_{1} + \cdots + \tau_{n}(v)e_{n})                     \\
                    & = \beta(u, e_{1})\cdot\tau_{1}(v) + \cdots + \beta(u, e_{n})\cdot\tau_{n}(v) \\
                    & = \varphi_{1}(u)\cdot\tau_{1}(v) + \cdots + \varphi_{n}(u)\cdot\tau_{n}(v)
    \end{align*}

    for all $u, v\in V$.
\end{proof}
\newpage

% chapter9:sectionA:exercise3
\begin{exercise}\label{chapter9:sectionA:exercise3}
    Suppose $\beta: V\times V\to\mathbb{F}$ is a bilinear form on $V$ and also is a linear functional on $V\times V$. Prove that $\beta = 0$.
\end{exercise}

\begin{proof}
    For every $u, v\in V$
    \begin{align*}
        \beta(u, v) & = \beta(u + 0, v)                                                                   \\
                    & = \beta(u, v) + \beta(0, v)               & \text{($\beta$ is a bilinear form)}     \\
                    & = \beta(u, v + 0) + \beta(0, v)                                                     \\
                    & = \beta(u, v) + \beta(u, 0) + \beta(0, v) & \text{($\beta$ is a bilinear form)}     \\
                    & = \beta(u, v) + \beta(u, v)               & \text{($\beta$ is a linear functional)}
    \end{align*}

    so $\beta(u, v) = \beta(u, v) + (-\beta(u, v)) = 0$. Hence $\beta = 0$.
\end{proof}
\newpage

% chapter9:sectionA:exercise4
\begin{exercise}\label{chapter9:sectionA:exercise4}
    Suppose $V$ is a real inner product space and $\beta$ is a bilinear form on $V$. Show that there exists a unique operator $T\in\lmap{V}$ such that
    \[
        \beta(u, v) = \innerprod{u, Tv}
    \]

    for all $u, v\in V$.
\end{exercise}

\begin{quote}
    This exercise states that if $V$ is a real inner product space, then every bilinear form on $V$ is of the form given by the third bullet point in 9.2.
\end{quote}

\begin{proof}
    $n = \dim V$. Let $e_{1}, \ldots, e_{n}$ be an orthonormal basis of $V$. For every $u, v\in V$
    \begin{align*}
        \beta(u, v) & = \beta(\innerprod{u,e_{1}}e_{1} + \cdots + \innerprod{u,e_{n}}e_{n}, v)            \\
                    & = \innerprod{u,e_{1}}\beta(e_{1}, v) + \cdots + \innerprod{u,e_{n}}\beta(e_{n}, v).
    \end{align*}

    Let's define an operator $T$ on $V$ as follows:
    \[
        Tv = \beta(e_{1}, v)e_{1} + \cdots + \beta(e_{n}, v)e_{n}.
    \]

    Hence for every $u, v\in V$, $\beta(u, v) = \innerprod{u, Tv}$.

    Assume operators $S, T\in\lmap{V}$ satisfy $\beta(u, v) = \innerprod{u, Sv} = \innerprod{u, Tv}$ for every $u, v\in V$. It follows that for every $v\in V$, for every $u\in V$, $\innerprod{u, Sv - Tv} = 0$, so $Sv = Tv$ for every $v\in V$. Therefore $S = T$.

    Thus there exists a unique operator $T$ on $V$ such that $\beta(u, v) = \innerprod{u, Tv}$ for all $u, v\in V$.
\end{proof}
\newpage

% chapter9:sectionA:exercise5
\begin{exercise}\label{chapter9:sectionA:exercise5}
    Suppose $\beta$ is a bilinear form on a real inner product space $V$ and $T$ is the unique operator on $V$ such that $\beta(u, v) = \innerprod{u, Tv}$ for all $u, v \in V$ (see Exercise~\ref{chapter9:sectionA:exercise4}). Show that $\beta$ is an inner product on $V$ if and only if $T$ is an invertible positive operator on $V$.
\end{exercise}

\begin{proof}
    $\beta$ is a bilinear form on a real inner product space $V$. Therefore $\beta$ is also an inner product on $V$ if and only if $\beta$ is conjugate symmetric and positive definite.

    $(\Rightarrow)$ $\beta$ is an inner product on $V$.

    $\beta$ is conjugate symmetric, so $\beta$ is symmetric, since $\mathbb{F} = \mathbb{R}$. For every $u, v\in V$
    \[
        \innerprod{u, Tv} = \beta(u, v) = \beta(v, u) = \innerprod{v, Tu} = \innerprod{Tu, v} = \innerprod{u, T^{*}v}.
    \]

    So $Tv = T^{*}v$ for every $v\in V$, which implies $T$ is self-adjoint.

    $\innerprod{Tv, v} = \innerprod{v, Tv} = \beta(v, v) > 0$ for every nonzero $v\in V$, and $\beta(0, 0) = 0$.

    Hence $T$ is an invertible positive operator on $V$.

    \bigskip
    $(\Leftarrow)$ $T$ is an invertible positive operator on $V$.

    So $T$ is self-adjoint. For every $u, v\in V$
    \[
        \beta(v, u) = \innerprod{v, Tu} = \conj{\innerprod{Tu, v}} = \conj{\innerprod{u, T^{*}v}} = \conj{\innerprod{u, Tv}} = \conj{\beta(u, v)}
    \]

    which means $\beta$ is conjugate symmetric.

    For every $v\in V$
    \[
        \beta(v, v) = \innerprod{v, Tv} = \innerprod{Tv, v} \geq 0
    \]

    because $T$ is positive, so $\beta$ is positive. Moreover, $\beta(v, v) = 0$ if and only if $\innerprod{Tv, v} = 0$. $\innerprod{Tv, v} = 0$ if and only if $v = 0$ because $T$ is an invertible positive operator. Therefore $\beta$ is positive definite.

    Hence $\beta$ is an inner product on $V$.

    \bigskip
    Thus $\beta$ is an inner product on $V$ if and only if $T$ is an invertible positive operator on $V$.
\end{proof}
\newpage

% chapter9:sectionA:exercise6
\begin{exercise}\label{chapter9:sectionA:exercise6}
    Prove or give a counterexample: If $\rho$ is a symmetric bilinear form on $V$, then
    \[
        \{ v\in V : \rho(v, v) = 0 \}
    \]

    is a subspace of $V$.
\end{exercise}

\begin{proof}
    Here is a counterexample.

    $V = \mathbb{R}^{2}$ and
    \[
        \rho((x_{1}, x_{2}), (y_{1}, y_{2})) = x_{1}y_{1} - 2x_{1}y_{2} - 2x_{2}y_{1} + 3x_{2}y_{2}
    \]

    so $\rho$ is a symmetric bilinear form on $\mathbb{R}^{2}$. $(1, 1)$ and $(1, 3)$ are in $\{ v\in V : \rho(v, v) = 0 \}$.
    \[
        \rho((1, 1), (1, 1)) = \rho((3, 1), (3, 1)) = 0.
    \]

    However,
    \[
        \rho((4, 2), (4, 2)) = 4^{2} - 4\cdot 4\cdot 2 + 3\cdot 2^{2} = 3\ne 0
    \]

    so $(4, 2)$ is not in $\{ v\in V : \rho(v, v) = 0 \}$. So $\{ v\in V : \rho(v, v) = 0 \}$ is not closed under addition, which means it is not a subspace of $\mathbb{R}^{2}$.
\end{proof}
\newpage

% chapter9:sectionA:exercise7
\begin{exercise}\label{chapter9:sectionA:exercise7}
    Explain why the proof of 9.13 (diagonalization of a symmetric bilinear form by an orthonormal basis on a real inner product space) fails if the hypothesis that $\mathbb{F} = \mathbb{R}$ is dropped.
\end{exercise}

\begin{proof}
    I give a counterexample where $\mathbb{F} = \mathbb{C}$.

    Let $\rho$ be a symmetric bilinear form on a complex inner product space $\mathbb{C}^{2}$ whose matrix (with respect to the standard basis of $\mathbb{C}^{2}$) is
    \[
        \begin{pmatrix}
            1     & \iota \\
            \iota & 0
        \end{pmatrix}.
    \]

    Let $T$ be an operator on $\mathbb{C}^{2}$ whose matrix with respect to the standard basis of $\mathbb{C}^{2}$ is $\begin{pmatrix}1 & \iota \\ \iota & 0 \end{pmatrix}$. The matrix of $T^{*}$ with respect to the standard basis of $\mathbb{C}^{2}$ is
    \[
        \begin{pmatrix}
            1      & -\iota \\
            -\iota & 0
        \end{pmatrix}.
    \]

    However, $T$ is not a normal operator, because
    \[
        \mathcal{M}(T)\mathcal{M}(T^{*}) =
        \begin{pmatrix}
            2 & -\iota \\
            0 & 1
        \end{pmatrix} \ne
        \begin{pmatrix}
            2 & \iota \\
            0 & 1
        \end{pmatrix} = \mathcal{M}(T^{*})\mathcal{M}(T).
    \]

    By the complex spectral theorem, there exists not exist an orthonormal basis of $\mathbb{C}^{2}$ to which $T$ has a diagonal matrix. Hence $\rho$ is not diagonalizable by an orthonormal basis.
\end{proof}
\newpage

% chapter9:sectionA:exercise8
\begin{exercise}\label{chapter9:sectionA:exercise8}
    Find formulas for $\dim V_{\text{sym}}^{(2)}$ and $\dim V_{\text{alt}}^{(2)}$ in terms of $\dim V$.
\end{exercise}

\begin{proof}
    Let $e_{1}, \ldots, e_{\dim V}$ be a basis of $V$.

    $V^{(2)}$ is isomorphic to $\mathbb{F}^{\dim V,\dim V}$, where each bilinear form on $V$ corresponds to its matrix with respect to $e_{1}, \ldots, e_{\dim V}$.

    A symmetric bilinear form corresponds to a symmetric matrix, and the vector space of symmetric matrices with $\dim V$ columns has dimension $(\dim V)\times(\dim V + 1)/2$. Therefore
    \[
        \dim V^{(2)}_{\text{sym}} = \frac{(\dim V) \times (\dim V + 1)}{2}.
    \]

    Moreover, $\dim V^{(2)} = {(\dim V)}^{2}$ and $V^{(2)} = V^{(2)}_{\text{sym}}\oplus V^{(2)}_{\text{alt}}$ so
    \[
        \dim V^{(2)}_{\text{alt}} = {(\dim V)}^{2} - \frac{(\dim V) \times (\dim V + 1)}{2} = \frac{(\dim V)\times (\dim V - 1)}{2}.
    \]

    Thus $\dim V^{(2)}_{\text{sym}} = (\dim V)\times(\dim V + 1)/2$ and $\dim V^{(2)}_{\text{alt}} = (\dim V)\times(\dim V - 1)/2$.
\end{proof}
\newpage

% chapter9:sectionA:exercise9
\begin{exercise}\label{chapter9:sectionA:exercise9}
    Suppose that $n$ is a positive integer and $V = \{ p\in\mathscr{P}_{n}(\mathbb{R}): p(0) = p(1) \}$. Define $\alpha: V\times V\to\mathbb{R}$ by
    \[
        \alpha(p, q) = \int^{1}_{0}pq'.
    \]

    Show that $\alpha$ is an alternating bilinear form on $V$.
\end{exercise}

\begin{proof}
    For every $p_{1}, p_{2}, q \in V$
    \[
        \alpha(p_{1} + p_{2}, q) = \int^{1}_{0}(p_{1} + p_{2})q' = \int^{1}_{0}p_{1}q' + \int^{1}_{0}p_{2}q' = \alpha(p_{1}, q) + \alpha(p_{2}, q).
    \]

    For every $p, q\in V$ and $\lambda\in\mathbb{R}$
    \[
        \alpha(\lambda p, q) = \int^{1}_{0}(\lambda p)q' = \lambda\int^{1}_{0}pq' = \lambda\alpha(p, q).
    \]

    For every $p, q_{1}, q_{2}\in V$
    \[
        \alpha(p, q_{1} + q_{2}) = \int^{1}_{0}p(q_{1} + q_{2})' = \int^{1}_{0}pq_{1}' + \int^{1}_{0}pq_{2}' = \alpha(p, q_{1}) + \alpha(p, q_{2}).
    \]

    For every $p, q\in V$ and $\lambda\in\mathbb{R}$
    \[
        \alpha(p, \lambda q) = \int^{1}_{0}p(\lambda q)' = \lambda\int^{1}_{0}pq' = \lambda\alpha(p, q).
    \]

    So $\alpha$ is a bilinear form on $V$.

    For every $p\in V$
    \[
        \alpha(p, p) = \int^{1}_{0}pp' = \int^{p(1)}_{p(0)}pdp = \frac{{(p(1))}^{2} - {(p(0))}^{2}}{2} = 0.
    \]

    Hence $\alpha$ is an alternating bilinear form on $V$.
\end{proof}
\newpage

% chapter9:sectionA:exercise10
\begin{exercise}\label{chapter9:sectionA:exercise10}
    Suppose that $n$ is a positive integer and
    \[
        V = \{ p\in\mathscr{P}_{n}(\mathbb{R}): p(0) = p(1) \land p'(0) = p'(1) \}.
    \]

    Define $\rho: V\times V\to\mathbb{R}$ by
    \[
        \rho(p, q) = \int^{1}_{0}pq''.
    \]

    Show that $\rho$ is a symmetric bilinear form on $V$.
\end{exercise}

\begin{proof}
    For every $p_{1}, p_{2}, q \in V$
    \[
        \alpha(p_{1} + p_{2}, q) = \int^{1}_{0}(p_{1} + p_{2})q'' = \int^{1}_{0}p_{1}q'' + \int^{1}_{0}p_{2}q'' = \alpha(p_{1}, q) + \alpha(p_{2}, q).
    \]

    For every $p, q\in V$ and $\lambda\in\mathbb{R}$
    \[
        \alpha(\lambda p, q) = \int^{1}_{0}(\lambda p)q'' = \lambda\int^{1}_{0}pq'' = \lambda\alpha(p, q).
    \]

    For every $p, q_{1}, q_{2}\in V$
    \[
        \alpha(p, q_{1} + q_{2}) = \int^{1}_{0}p(q_{1} + q_{2})'' = \int^{1}_{0}pq_{1}'' + \int^{1}_{0}pq_{2}'' = \alpha(p, q_{1}) + \alpha(p, q_{2}).
    \]

    For every $p, q\in V$ and $\lambda\in\mathbb{R}$
    \[
        \alpha(p, \lambda q) = \int^{1}_{0}p(\lambda q)'' = \lambda\int^{1}_{0}pq'' = \lambda\alpha(p, q).
    \]

    So $\alpha$ is a bilinear form on $V$.

    For every $p, q\in V$
    \begin{align*}
        \alpha(p, q) & = \int^{1}_{0}pq'' = \int^{1}_{0}p(x)q''(x)dx         \\
                     & = \int^{x=1}_{x=0}p(x)dq'(x)                          \\
                     & = p(x)q'(x)\Big{\vert}^{x=1}_{x=0} - \int^{1}_{0}q'p' \\
                     & = -\int^{1}_{0}q'p'.
    \end{align*}

    Therefore $\alpha(q, p) = -\int^{1}_{0}p'q' = -\int^{1}_{0}q'p' = \alpha(p, q)$.

    Hence $\alpha$ is a symmetric bilinear form on $V$.
\end{proof}
\newpage

\section{Alternating Multilinear Forms}

% chapter9:sectionB:exercise1
\begin{exercise}\label{chapter9:sectionB:exercise1}
    Suppose $m$ is a positive integer. Show that $\dim V^{(m)} = {(\dim V)}^{m}$.
\end{exercise}

\begin{proof}
    Let $e_{1}, \ldots, e_{\dim V}$ be a basis of $V$ and $\varphi_{1}, \ldots, \varphi_{\dim V}$ be its dual basis.

    For each \textit{choices (duplication is allowed)}  of $m$ linear functionals from $\varphi_{1}, \ldots, \varphi_{\dim V}$, we define an $m$-linear form as follows, where $\varphi_{j_{1}}, \ldots, \varphi_{j_{m}}$ is the selected listed, $j_{1}, \ldots, j_{m}$ are from $1, \ldots, \dim V$
    \[
        \alpha_{j_{1},\ldots,j_{m}}: (v_{1}, \ldots, v_{m})\mapsto \varphi_{j_{1}}(v_{1})\cdots \varphi_{j_{m}}(v_{m})
    \]

    Assume that
    \[
        \sum x_{j_{1},\ldots,j_{m}}\alpha_{j_{1},\ldots,j_{m}} = 0
    \]

    for all $v_{1}, \ldots, v_{m}\in V$ and we use all previously defined $m$-linear forms $\alpha_{j_{1},\ldots,j_{m}}$. Plug $(v_{j_{1}}, \ldots, v_{j_{m}})\in \underbrace{V\times\cdots\times V}_{m}$ in, we obtain
    \[
        x_{j_{1},\ldots, j_{m}} = 0.
    \]

    Hence all $m$-linear forms $\alpha_{j_{1},\ldots,j_{m}}$ are linearly independent. Moreover, for every $m$-linear form $\alpha$ on $V$,
    \begin{align*}
        \alpha(v_{1}, \ldots, v_{m}) & = \alpha\left(\sum^{\dim V}_{j=1}\varphi_{j}(v_{1})e_{j}, \ldots, \sum^{\dim V}_{j=1}\varphi_{j}(v_{m})e_{j}\right)                                                     \\
                                     & = \sum^{\dim V}_{j_{m}=1}\left(\cdots\left(\sum^{\dim V}_{j_{1}=1}\alpha(e_{j_{1}}, \ldots, e_{j_{m}})\varphi_{j_{1}}(v_{1})\cdots\varphi_{j_{m}}(v_{m})\right)\right)  \\
                                     & = \sum^{\dim V}_{j_{m}=1}\left(\cdots\left(\sum^{\dim V}_{j_{1}=1}\alpha(e_{j_{1}}, \ldots, e_{j_{m}})\varphi_{j_{1},\ldots,j_{m}}(v_{1}, \ldots, v_{m})\right)\right).
    \end{align*}

    This means all $m$-linear forms $\alpha_{j_{1},\ldots,j_{m}}$ spans $V^{(m)}$. Therefore all $m$-linear forms $\alpha_{j_{1},\ldots,j_{m}}$ constitute a basis of $V^{(m)}$ and $\dim V^{(m)} = {(\dim V)}^{m}$.
\end{proof}
\newpage

% chapter9:sectionB:exercise2
\begin{exercise}\label{chapter9:sectionB:exercise2}
    Suppose $n\geq 3$ and $\alpha: \mathbb{F}^{n}\times\mathbb{F}^{n}\times\mathbb{F}^{n} \to \mathbb{F}$ is defined by
    \begin{gather*}
        \alpha((x_{1}, \ldots, x_{n}), (y_{1}, \ldots, y_{n}), (z_{1}, \ldots, z_{n})) \\
        = x_{1}y_{2}z_{3} - x_{2}y_{1}z_{3} - x_{3}y_{2}z_{1} - x_{1}y_{3}z_{2} + x_{3}y_{1}z_{2} + x_{2}y_{3}z_{1}.
    \end{gather*}

    Show that $\alpha$ is an alternating $3$-linear form on $\mathbb{F}^{n}$.
\end{exercise}

\begin{proof}
    \begin{align*}
        \alpha((x_{1}, \ldots, x_{n}), (y_{1}, \ldots, y_{n}), (z_{1}, \ldots, z_{n})) & = x_{1}(y_{2}z_{3} - y_{3}z_{2}) + x_{2}(y_{3}z_{1} - y_{1}z_{3}) + x_{3}(y_{1}z_{2} - y_{2}z_{1}) \\
        \alpha((x_{1}, \ldots, x_{n}), (y_{1}, \ldots, y_{n}), (z_{1}, \ldots, z_{n})) & = y_{1}(z_{2}x_{3} - z_{3}x_{2}) + y_{2}(z_{3}x_{1} - z_{1}x_{3}) + y_{3}(z_{1}x_{2} - z_{2}x_{1}) \\
        \alpha((x_{1}, \ldots, x_{n}), (y_{1}, \ldots, y_{n}), (z_{1}, \ldots, z_{n})) & = z_{1}(x_{2}y_{3} - x_{3}y_{2}) + z_{2}(x_{3}y_{1} - x_{1}y_{3}) + z_{3}(x_{1}y_{2} - x_{2}y_{1})
    \end{align*}

    so $\alpha$ is a $3$-linear form on $\mathbb{F}^{n}$ and $\alpha$ is alternating.
\end{proof}
\newpage

% chapter9:sectionB:exercise3
\begin{exercise}\label{chapter9:sectionB:exercise3}
    Suppose $m$ is a positive integer and $\alpha$ is an $m$-linear form on $V$ such that $\alpha(v_{1}, \ldots, v_{m}) = 0$ whenever $v_{1}, \ldots, v_{m}$ is a list of vectors in $V$ with $v_{j} = v_{j+1}$ for some $j\in\{1,\ldots,m-1\}$. Prove that $\alpha$ is an alternating $m$-linear form on $V$.
\end{exercise}

\begin{proof}
    We will show that for every positive integer $k$ less than $m$, $\alpha(v_{1}, \ldots, v_{m}) = 0$ if $v_{j} = v_{j+k}$ for every $j\in\{1,\ldots,m-k\}$.

    The statement is true for $k = 1$ due to the hypothesis.

    Assume the statement is true for all $k < \ell$ where $j + \ell < m$. Suppose that $v_{j} = v_{j + \ell}$.

    Consider the expression $\alpha(\ldots, v_{j} + v_{j+k}, \ldots, v_{j+k} + v_{j}, \ldots)$ where $v_{j} + v_{j+k}$ is at the $j$th slot, and $v_{j+k} + v_{j}$ is at the ${(j+k)}$th slot, $k < \ell$.

    We prove the anti-symmetric property first. By the induction hypothesis, for every
    \[
        (\ldots, w_{j} + w_{j+k}, \ldots, w_{j+k} + w_{j}, \ldots)\in \underbrace{V\times\cdots\times V}_{m}
    \]

    we have
    \begin{align*}
        0 = \alpha(\ldots, w_{j} + w_{j+k}, \ldots, w_{j+k} + w_{j}, \ldots) & = \alpha(\ldots, w_{j}, \ldots, w_{j+k}, \ldots) + \alpha(\ldots, w_{j+k}, \ldots, w_{j}, \ldots).
    \end{align*}

    So $\alpha(\ldots, w_{j}, \ldots, w_{j+k}, \ldots) = -\alpha(\ldots, w_{j+k}, \ldots, w_{j}, \ldots)$. Apply this result, we obtain that, if $(v_{1}, \ldots, v_{m})\in \underbrace{V\times\cdots\times V}_{m}$ such that $v_{j} = v_{j+\ell}$, then
    \[
        \alpha(\ldots, v_{j}, v_{j+1}, \ldots, v_{j+\ell}, \ldots) = -\alpha(\ldots, v_{j+1}, v_{j}, \ldots, v_{j+\ell}, \ldots).
    \]

    By the induction hypothesis, $\alpha(\ldots, v_{j+1}, v_{j}, \ldots, v_{j+\ell}, \ldots) = 0$ because $v_{j}$ is at the $(j+1)$th slot and $v_{j+\ell}$ is at the $(j+\ell)$th slot. Therefore $\alpha(\ldots, v_{j}, v_{j+1}, \ldots, v_{j+\ell}, \ldots) = 0$.

    Due to the principle of mathematical induction, for every positive integer $k < m$, for every positive integer $j\in\{1,\ldots,m-k\}$, $\alpha(v_{1}, \ldots, v_{m}) = 0$ if $v_{j} = v_{j + k}$.

    Thus $\alpha$ is an alternating $m$-linear form on $V$.
\end{proof}
\newpage

% chapter9:sectionB:exercise4
\begin{exercise}\label{chapter9:sectionB:exercise4}
    Prove or give a counterexample: If $\alpha\in V^{(4)}_{\text{alt}}$, then
    \[
        \{ (v_{1}, v_{2}, v_{3}, v_{4})\in V^{4}: \alpha(v_{1}, v_{2}, v_{3}, v_{4}) = 0 \}
    \]

    is a subspace of $V^{4}$.
\end{exercise}

\begin{proof}
    Here is a counterexample.

    $V$ is a vector space with dimension $4$. Let $e_{1}, e_{2}, e_{3}, e_{4}$ be a basis of $V$, and $\alpha$ a nonzero alternating $4$-linear form on $V$. Because $\alpha$ is alternating,
    \[
        \alpha(e_{1}, e_{1}, e_{3}, 0) = \alpha(0, e_{2}, e_{2}, e_{4}) = 0.
    \]

    However $(e_{1}, e_{1} + e_{2}, e_{2} + e_{3}, e_{4}) = (e_{1}, e_{1}, e_{3}, 0) + (0, e_{2}, e_{2}, e_{4})$ and
    \[
        \alpha(e_{1}, e_{1} + e_{2}, e_{2} + e_{3}, e_{4})\ne 0
    \]

    because $e_{1}, e_{1} + e_{2}, e_{2} + e_{3}, e_{4}$. Therefore the set
    \[
        \{ (v_{1}, v_{2}, v_{3}, v_{4})\in V^{4}: \alpha(v_{1}, v_{2}, v_{3}, v_{4}) = 0 \}
    \]

    is not closed under addition, which implies it is not a subspace of $V^{4}$.
\end{proof}
\newpage

% chapter9:sectionB:exercise5
\begin{exercise}\label{chapter9:sectionB:exercise5}
    Suppose $m$ is a positive integer and $\beta$ is an $m$-linear form on $V$. Define an $m$-linear form $\alpha$ on $V$ by
    \[
        \alpha(v_{1}, \ldots, v_{m}) = \sum_{(j_{1}, \ldots, j_{m})\in\operatorname{perm}m}(\operatorname{sign}(j_{1}, \ldots, j_{m}))\beta(v_{j_{1}}, \ldots, v_{j_{m}})
    \]

    for $v_{1}, \ldots, v_{m}\in V$. Explain why $\alpha\in V^{(m)}_{\text{alt}}$.
\end{exercise}

\begin{proof}
    For each $(j_{1}, \ldots, j_{m})\in \operatorname{perm}m$, the map
    \[
        (v_{1}, \ldots, v_{m})\mapsto (\operatorname{sign}(j_{1}, \ldots, j_{m}))\beta(v_{j_{1}}, \ldots, v_{j_{m}})
    \]

    is an $m$-linear form on $V$. Therefore
    \[
        \alpha(v_{1}, \ldots, v_{m}) = \sum_{(j_{1}, \ldots, j_{m})\in\operatorname{perm}m}(\operatorname{sign}(j_{1}, \ldots, j_{m}))\beta(v_{j_{1}}, \ldots, v_{j_{m}})
    \]

    is an $m$-linear form on $V$.

    If $v_{j} = v_{k}$ for some $j\ne k$ then for each permutation $(x_{1}, \ldots, x_{m})$ in $\operatorname{perm}m$, there exists a unique permutation $(y_{1}, \ldots, y_{m})$ in $\operatorname{perm}m$ where these two differ by a swap $j\leftrightarrow k$, hence one has $+1$ sign and the other has $-1$ sign. Moreover,
    \[
        \beta(v_{x_{1}}, \ldots, v_{x_{m}}) = \beta(v_{y_{1}}, \ldots, v_{y_{m}}).
    \]

    Therefore $\alpha(v_{1}, \ldots, v_{m}) = 0$ if $v_{j} = v_{k}$ for some $j\ne k$. Hence $\alpha$ is alternating.

    Thus $\alpha\in V^{(m)}_{\text{alt}}$.
\end{proof}
\newpage

% chapter9:sectionB:exercise6
\begin{exercise}\label{chapter9:sectionB:exercise6}
    Suppose $m$ is a positive integer and $\beta$ is an $m$-linear form on $V$. Define an $m$-linear form $\alpha$ on $V$ by
    \[
        \alpha(v_{1}, \ldots, v_{m}) = \sum_{(j_{1}, \ldots, j_{m})\in\operatorname{perm}m}\beta(v_{j_{1}}, \ldots, v_{j_{m}})
    \]

    for $v_{1}, \ldots, v_{m}\in V$. Explain why
    \[
        \alpha(v_{k_{1}}, \ldots, v_{k_{m}}) = \alpha(v_{1}, \ldots, v_{m})
    \]

    for all $v_{1}, \ldots, v_{m}\in V$ and all $(k_{1}, \ldots, k_{m})\in\operatorname{perm}m$.
\end{exercise}

\begin{proof}
    For each $(k_{1}, \ldots, k_{m})\in\operatorname{perm}m$, $\alpha(v_{k_{1}}, \ldots, v_{k_{m}})$ is the sum of all $\beta(v_{i_{1}}, \ldots, v_{i_{m}})$ where each $(i_{1}, \ldots, i_{m})$ is a permutation of $(k_{1}, \ldots, k_{m})$. On the other hand, each $(i_{1}, \ldots, i_{m})$ is also a permutation of $(1, \ldots, m)$. Therefore $\alpha(v_{k_{1}}, \ldots, v_{k_{m}}) = \alpha(v_{1}, \ldots, v_{m})$ for all $v_{1}, \ldots, v_{m}\in V$ and all $(k_{1}, \ldots, k_{m})\in\operatorname{perm}m$.
\end{proof}
\newpage

% chapter9:sectionB:exercise7
\begin{exercise}\label{chapter9:sectionB:exercise7}
    Give an example of a nonzero alternating $2$-linear form $\alpha$ on $\mathbb{R}^{3}$ and a linearly independent list $v_{1}, v_{2}$ in $\mathbb{R}^{3}$ such that $\alpha(v_{1}, v_{2}) = 0$.
\end{exercise}

\begin{quote}
    This exercise shows that 9.39 can fail if the hypothesis that $n = \dim V$ is deleted.
\end{quote}

\begin{proof}
    Here is an example.
    \[
        \alpha((x_{1}, x_{2}, x_{3}), (y_{1}, y_{2}, y_{3})) = x_{1}y_{2} - x_{2}y_{1}
    \]

    $\alpha\in {(\mathbb{R}^{3})}^{(2)}_{\text{alt}}$. However,
    \[
        \alpha((1, 0, 0), (1, 0, 1)) = 0
    \]

    although $(1, 0, 0), (1, 0, 1)$ is a linearly independent list.
\end{proof}
\newpage

\section{Determinants}

% chapter9:sectionC:exercise1
\begin{exercise}\label{chapter9:sectionC:exercise1}
    Prove or give a counterexample: $S, T\in\lmap{V}\implies \det(S + T) = \det S + \det T$.
\end{exercise}

\begin{proof}
    Here is a counterexample.

    Let $V$ be a real finite-dimensional vector space of dimension greater than $1$, then
    \[
        \det(I + I) = 2^{\dim V} \ne 1 + 1 = \det I + \det I.
    \]
\end{proof}
\newpage

% chapter9:sectionC:exercise2
\begin{exercise}\label{chapter9:sectionC:exercise2}
    Suppose the first column of a square matrix $A$ consists of all zeros except possibly the first entry $A_{1,1}$. Let $B$ be the matrix obtained from $A$ by deleting the first row and the first column of $A$. Show that $\det A = A_{1,1} \det B$.
\end{exercise}

\begin{proof}
    Let $n$ be the number of columns of $A$. By the Leibniz's determinant formula
    \begin{align*}
        \det A & = \sum_{(j_{1}, \ldots, j_{n})\in\operatorname{perm}n}(\operatorname{sign}(j_{1}, \ldots, j_{n}))A_{j_{1},1}\cdots A_{j_{n},n}
    \end{align*}

    If $j_{1}\ne 1$, then $A_{j_{1},1}\cdots A_{j_{n},n} = 0$ because $A_{i,1} = 0$ for all $i\ne 1$. Therefore
    \begin{align*}
        \det A & = A_{1,1}\sum_{(j_{2}, \ldots, j_{n})\in\operatorname{perm}(2,\ldots,n)}\operatorname{sign}(j_{2}, \ldots, j_{n})A_{j_{2},2}\cdots A_{j_{n},n} \\
               & = A_{1,1}\sum_{(i_{1},\ldots, i_{n-1})\in\operatorname{perm}(n-1)}B_{i_{1},1}\cdots B_{i_{n-1},n-1}                                            \\
               & = A_{1,1}\det B
    \end{align*}

    where $\operatorname{sign}(j_{2}, \ldots, j_{n}) = \operatorname{sign}(1, j_{2}, \ldots, j_{n})$ because $1 < j_{2}, \ldots, j_{n}$.

    Thus $\det A = A_{1,1}\det B$.
\end{proof}
\newpage

% chapter9:sectionC:exercise3
\begin{exercise}\label{chapter9:sectionC:exercise3}
    Suppose $T\in\lmap{V}$ is nilpotent. Prove that $\det(I + T) = 1$.
\end{exercise}

\begin{proof}
    Because $T$ is nilpotent, then there exists a basis $e_{1}, \ldots, e_{\dim V}$ of $V$ to which $T$ has an upper-triangular matrix whose entries on the diagonal are $0$. Therefore the matrix of $I + T$ with respect to the basis $e_{1}, \ldots, e_{\dim V}$ of $V$ is an upper-triangular matrix whose entries on the diagonal are $1$. Thus $\det(I + T) = 1$.
\end{proof}
\newpage

% chapter9:sectionC:exercise4
\begin{exercise}\label{chapter9:sectionC:exercise4}
    Suppose $S\in\lmap{V}$. Prove that $S$ is unitary if and only if $\abs{\det S} = \norm{S} = 1$.
\end{exercise}

\begin{proof}
    Let $s_{1}\geq s_{2}\geq \cdots \geq s_{\dim V}$ be the singular values of $S$, then
    \[
        \abs{\det S} = s_{1}\cdots s_{\dim V}
    \]

    and
    \[
        \norm{S} = s_{1}.
    \]

    Therefore $\abs{\det S} = \norm{S} = 1$ if and only if all singular values of $S$ are equal to $1$. Moreover, all singular values of $S$ are equal to $1$ if and only if $S$ is unitary.

    Thus $S$ is unitary if and only if $\abs{\det S} = \norm{S} = 1$.
\end{proof}
\newpage

% chapter9:sectionC:exercise5
\begin{exercise}\label{chapter9:sectionC:exercise5}
    Suppose $A$ is a block upper-triangular matrix
    \[
        A = \begin{pmatrix}
            A_{1} &        & *     \\
                  & \ddots &       \\
            0     &        & A_{m}
        \end{pmatrix},
    \]

    where each $A_{k}$ along the diagonal is a square matrix. Prove that
    \[
        \det A = (\det A_{1})\cdots (\det A_{m}).
    \]
\end{exercise}

\begin{proof}
    Firstly, for brevity (I don't want to use lengthy indices), I will prove that if
    \[
        C = \begin{pmatrix}
            A & * \\
            0 & B
        \end{pmatrix}
    \]

    then $\det C = (\det A)(\det B)$.

    Let the numbers of columns of $A, B$ be $m, n$, respectively. Let $T$ be the operator on $V = \mathbb{F}^{m+n}$ whose matrix with respect to the standard basis of $V = \mathbb{F}^{m+n}$ is $C$.

    Let $\alpha$ be a nonzero alternating $(m+n)$-linear form in $V^{(m)}_{\text{alt}}$.
    \begin{align*}
        (\det T)\alpha(e_{1}, \ldots, e_{m}, e_{m+1}, \ldots, e_{m+n}) & = \alpha(Te_{1}, \ldots, Te_{m}, Te_{m+1}, \ldots, Te_{m+n})
    \end{align*}

    Let $U = \operatorname{span}(e_{1}, \ldots, e_{n})$, then $U$ is invariant under $T$. Moreover, the matrix of $T\vert_{U}$ with respect to $e_{1}, \ldots, e_{n}$ is $A$. The following map
    \[
        (v_{1}, \ldots, v_{m})\mapsto \alpha(Tv_{1}, \ldots, Tv_{m}, Te_{m+1}, \ldots, Te_{m+n})
    \]

    is an alternating $m$-linear form on $U$. Therefore
    \begin{align*}
        \alpha(Te_{1}, \ldots, Te_{m}, Te_{m+1}, \ldots, Te_{m+n}) & = (\det T\vert_U)\alpha(e_{1}, \ldots, e_{m}, Te_{m+1}, \ldots, Te_{m+n}) \\
                                                                   & = (\det A)\alpha(e_{1}, \ldots, e_{m}, Te_{m+1}, \ldots, Te_{m+n})
    \end{align*}

    Moreover, for each $j\in\{1,\ldots,n\}$
    \begin{align*}
        Te_{m+j} & = (C_{1,m+j}e_{1} + \cdots + C_{m, m+j}e_{m}) + (C_{m+1,j}e_{m+1} + \cdots + C_{m+n,j}e_{m+n}) \\
                 & = (C_{1,m+j}e_{1} + \cdots + C_{m, m+j}e_{m}) + (B_{1,j}e_{m+1} + \cdots + B_{n,j}e_{m+n})
    \end{align*}

    and $\alpha$ is an alternating $(m+n)$-linear form on $V$, so we have
    \[
        \alpha(e_{1}, \ldots, e_{m}, Te_{m+1}, \ldots, Te_{m+n}) = \alpha(e_{1}, \ldots, e_{m}, \sum^{n}_{i=1}B_{i,1}e_{m+i}, \ldots, \sum^{n}_{i=1}B_{i,n}e_{m+i})
    \]

    Let $S$ be the operator on $W = \operatorname{span}(e_{m+1}, \ldots, e_{m+n})$ whose matrix with respect to the basis $e_{m+1}, \ldots, e_{m+n}$ is $B$, then
    \[
        \alpha(e_{1}, \ldots, e_{m}, \sum^{n}_{i=1}B_{i,1}e_{m+i}, \ldots, \sum^{n}_{i=1}B_{i,n}e_{m+i}) = \alpha(e_{1}, \ldots, e_{m}, Se_{m+1}, \ldots, Se_{m+n}).
    \]

    The following map
    \[
        (v_{1}, \ldots, v_{n})\mapsto \alpha(e_{1}, \ldots, e_{m}, Sv_{1}, \ldots, Sv_{n})
    \]

    is an alternating $n$-linear form on $W$, so
    \begin{align*}
        \alpha(e_{1}, \ldots, e_{m}, Se_{m+1}, \ldots, Se_{m+n}) & = (\det S)\alpha(e_{1}, \ldots, e_{m}, e_{m+1}, \ldots, e_{m+n})  \\
                                                                 & = (\det B)\alpha(e_{1}, \ldots, e_{m}, e_{m+1}, \ldots, e_{m+n}).
    \end{align*}

    Combine every so far together, we obtain
    \begin{align*}
        (\det C)\alpha(e_{1}, \ldots, e_{m}, e_{m+1}, \ldots, e_{m+n}) & = \alpha(Te_{1}, \ldots, Te_{m}, Te_{m+1}, \ldots, Te_{m+n})              \\
                                                                       & = (\det A)(\det B)\alpha(e_{1}, \ldots, e_{m}, e_{m+1}, \ldots, e_{m+n}).
    \end{align*}

    Hence $\det C = (\det A)(\det B)$.

    \bigskip
    Back to the original problem. I give a proof using mathematical induction on $m$.

    The statement is true for $m = 1$.

    Assume the statement is true for $m = n - 1$. Now suppose $A$ is a block upper-triangular matrix has $n$ blocks. Let $B$ be the block upper-triangular matrix obtained by removing the rows and columns of $A$ containing entries of $A_{n}$. By the previously proved result,
    \[
        \det A = (\det B)(\det A_{n}).
    \]

    According to the induction hypothesis, $\det B = (\det A_{1})\cdots (\det A_{n-1})$. Therefore $\det A = (\det A_{1})\cdots (\det A_{n})$.

    By the principle of mathematical induction, we conclude that
    \[
        \det A = (\det A_{1})\cdots (\det A_{m}).\qedhere
    \]
\end{proof}
\newpage

% chapter9:sectionC:exercise6
\begin{exercise}\label{chapter9:sectionC:exercise6}
    Suppose $A = \begin{pmatrix}v_{1} & \cdots & v_{n}\end{pmatrix}$ is an $n$-by-$n$ matrix, with $v_{k}$ denoting $k$th column of $A$. Show that if $(m_{1}, \ldots, m_{n})\in \operatorname{perm}n$, then
    \[
        \det\begin{pmatrix}v_{m_{1}} & \cdots & v_{m_{n}}\end{pmatrix} = (\operatorname{sign}(m_{1}, \ldots, m_{n}))\det A.
    \]
\end{exercise}

\begin{proof}
    Because $(u_{1}, \ldots, u_{n})\mapsto \det\begin{pmatrix}u_{1} & \cdots & u_{n}\end{pmatrix}$ is an alternating $n$-linear form on $\mathbb{F}^{n}$, then
    \begin{align*}
        \det\begin{pmatrix}v_{m_{1}} & \cdots & v_{m_{n}}\end{pmatrix} & = (\operatorname{sign}(m_{1}, \ldots, m_{n}))\det\begin{pmatrix}v_{1} & \cdots & v_{n}\end{pmatrix} \\
                                                                                                 & = (\operatorname{sign}(m_{1}, \ldots, m_{n}))\det A.\qedhere
    \end{align*}
\end{proof}
\newpage

% chapter9:sectionC:exercise7
\begin{exercise}\label{chapter9:sectionC:exercise7}
    Suppose $T\in\lmap{V}$ is invertible. Let $p$ denote the characteristic polynomial of $T$ and let $q$ denote the characteristic polynomial of $T^{-1}$. Prove that
    \[
        q(z) = \frac{1}{p(0)}z^{\dim V}p\left(\frac{1}{z}\right)
    \]

    for all nonzero $z\in\mathbb{F}$.
\end{exercise}

\begin{proof}
    For all nonzero $z\in\mathbb{F}$,
    \begin{align*}
        q(z) & = \det(zI - T^{-1})                                                         & \text{(definition of characteristic polynomial)} \\
             & = \det(T^{-1}zT - T^{-1})                                                                                                      \\
             & = \det(T^{-1})\det(zT - I)                                                  & (\det(AB) = (\det A)(\det B))                    \\
             & = \frac{1}{\det T}\det(zT - I)                                              & (1 = (\det T^{-1})(\det T))                      \\
             & = \frac{1}{\det T}{(-z)}^{\dim V}\det\left(\frac{1}{z}I - T\right)          & (\det\lambda T = \lambda^{\dim V}\det T)         \\
             & = \frac{{(-1)}^{\dim V}}{\det T}z^{\dim V}\det\left(\frac{1}{z}I - T\right)                                                    \\
             & = \frac{1}{p(0)}z^{\dim V}p\left(\frac{1}{z}\right).
    \end{align*}
\end{proof}
\newpage

% chapter9:sectionC:exercise8
\begin{exercise}\label{chapter9:sectionC:exercise8}
    Suppose $T \in \lmap{V}$ is an operator with no eigenvalues (which implies that $\mathbb{F} = \mathbb{R}$). Prove that $\det T > 0$.
\end{exercise}

\begin{proof}
    Let $p$ be the characteristic polynomial of $T$. Because $T$ has no eigenvalue, then $p$ has no root. Therefore $p(x)\ne 0$ for every $x\in\mathbb{R}$.

    Assume there exist $a, b\in\mathbb{R}$ such that $p(a) < 0$ and $p(b) > 0$. Because of the intermediate value theorem and $p$ is continuous on $\mathbb{R}$, there exists $c\in\mathbb{R}$ and $a < c < b$ such that $p(c) = 0$. This is a contradiction because $p$ has no root. Therefore, either $p(x) > 0$ for every $x\in\mathbb{R}$ or $p(x) < 0$ for every $x\in\mathbb{R}$.

    On the other hand $p$ is a monic polynomial with real coefficient, so
    \[
        \lim\limits_{x\to+\infty}\frac{p(x)}{x^{\dim V}} = 1
    \]

    which implies $\lim\limits_{x\to+\infty}p(x) = +\infty$. Hence there exists a real number $x_{0}$ such that $p(x_{0}) > 0$. Since either $p(x) > 0$ for every $x\in\mathbb{R}$ or $p(x) < 0$ for every $x\in\mathbb{R}$, we conclude that $p(x) > 0$ for all $x\in\mathbb{R}$.

    $\dim V$ is an even number, because if $\dim V$ is an odd number, then $p$ has a root. Therefore the constant term of $p$ is ${(-1)}^{\dim V}\det T = \det T$. So $\det T = p(0) > 0$.
\end{proof}
\newpage

% chapter9:sectionC:exercise9
\begin{exercise}\label{chapter9:sectionC:exercise9}
    Suppose that $V$ is a real vector space of even dimension, $T \in \lmap{V}$, and $\det T < 0$. Prove that $T$ has at least two distinct eigenvalues.
\end{exercise}

\begin{proof}
    Let $p$ be the characteristic polynomial of $T$. Because $\dim V$ is an even number, the constant term of $p$ is ${(-1)}^{\dim V} = \det T$.

    We have
    \[
        \lim\limits_{x\to{\color{red}{+\infty}}}\frac{p(x)}{x^{\dim V}} = \lim\limits_{x\to{\color{blue}{-\infty}}}\frac{p(x)}{x^{\dim V}} = 1.
    \]

    So $\lim\limits_{x\to{\color{red}{+\infty}}}p(x) = \lim\limits_{x\to{\color{blue}{-\infty}}}p(x) = +\infty$. Therefore, there are positive number $a$ and negative number $b$ such that $p(a) > 0$ and $p(b) > 0$.

    Since $\det T < 0$ and $\det T = p(0)$, it follows that $p(0) < 0$. By the intermediate value theorem, there is a positive number $c_{1}$ less than $a$ such that $p(c_{1}) = 0$ and there is a negative number $c_{2}$ larger than $b$ such that $p(c_{2}) = 0$. Moreover, $c_{1}$ and $c_{2}$ are distinct and are also eigenvalues of $T$.

    Thus $T$ has at least two distinct eigenvalues.
\end{proof}
\newpage

% chapter9:sectionC:exercise10
\begin{exercise}\label{chapter9:sectionC:exercise10}
    Suppose $V$ is a real vector space of odd dimension and $T \in \lmap{V}$. Without using the minimal polynomial, prove that $T$ has an eigenvalue.
\end{exercise}

\begin{quote}
    This result was previously proved without using determinants or the characteristic polynomial $-$ see 5.34.
\end{quote}

\begin{proof}
    $V$ is a real vector space of odd dimension, so the characteristic polynomial of $T$ has real coefficients and odd degree. Therefore the characteristic polynomial of $T$ has a root, which implies $T$ has an eigenvalue.
\end{proof}
\newpage

% chapter9:sectionC:exercise11
\begin{exercise}\label{chapter9:sectionC:exercise11}
    Prove or give a counterexample: If $\mathbb{F} = \mathbb{R}$, $T\in\lmap{V}$, and $\det T > 0$, then $T$ has a square root.
\end{exercise}

\begin{quote}
    If $\mathbb{F} = \mathbb{C}$, $T\in\lmap{V}$, and $\det T\ne 0$, then $T$ has a square root (see 8.41).
\end{quote}

\begin{proof}
    Here is a counterexample.

    On $V = \mathbb{R}^{2}$, let $T(x, y) = (-x, x - y)$. The matrix of $T$ with respect to $(1, 0), (0, 1)$ is
    \[
        A = \begin{pmatrix}
            -1 & 1  \\
            0  & -1
        \end{pmatrix}.
    \]

    $\det T = \det A = 1 > 0$. Assume $T$ has a square root $S\in\lmap{\mathbb{R}^{2}}$, let the matrix of $S$ with respect to $(1, 0), (0, 1)$ be $B$. Because $T = S^{2}$, so $B^{2} = A$, and we obtain the following system of equations
    \[
        \begin{cases}
            B_{1,1}^{2} + B_{1,2}B_{2,1} = -1, \\
            B_{2,2}^{2} + B_{1,2}B_{2,1} = -1, \\
            B_{1,2}(B_{1,1} + B_{2,2}) = 1,    \\
            B_{2,1}(B_{1,1} + B_{2,2}) = 0.
        \end{cases}
    \]

    From the last two equations, we deduce that $B_{1,1} + B_{2,2}\ne 0$ and $B_{2,1} = 0$. Therefore $-1 = B_{1,1}^{2} + B_{1,2}B_{2,1} = B_{1,1}^{2}$, which is impossible because $B_{1,1}$ is a real number.

    Hence the chosen operator $T$ does not have a square root.
\end{proof}
\newpage

% chapter9:sectionC:exercise12
\begin{exercise}\label{chapter9:sectionC:exercise12}
    Suppose $S, T\in\lmap{V}$ and $S$ is invertible. Define $p: \mathbb{F}\to\mathbb{F}$ by
    \[
        p(z) = \det(zS - T).
    \]

    Prove that $p$ is a polynomial of degree $\dim V$ and that the coefficient of $z^{\dim V}$ in this polynomial is $\det S$.
\end{exercise}

\begin{proof}
    \[
        p(z) = \det(zS - T) = \det(zIS - TS^{-1}S) = \det(zI - TS^{-1})\det(S)
    \]

    so the coefficient of $z^{\dim V}$ in this polynomial is $\det S$.
\end{proof}
\newpage

% chapter9:sectionC:exercise13
\begin{exercise}\label{chapter9:sectionC:exercise13}
    Suppose $\mathbb{F} = \mathbb{C}$, $T\in\lmap{V}$, and $n = \dim V > 2$. Let $\lambda_{1}, \ldots, \lambda_{n}$ denote the eigenvalues of $T$, with each eigenvalue included as many times as its multiplicity.
    \begin{enumerate}[label={(\alph*)}]
        \item Find a formula for the coefficient of $z^{n-2}$ in the characteristic polynomial of $T$ in terms of $\lambda_{1}, \ldots, \lambda_{n}$.
        \item Find a formula for the coefficient of $z$ in the characteristic polynomial of $T$ in terms of $\lambda_{1}, \ldots, \lambda_{n}$.
    \end{enumerate}
\end{exercise}

\begin{proof}
    The characteristic polynomial of $T$ is $(z - \lambda_{1})\cdots (z - \lambda_{n})$.

    \begin{enumerate}[label={(\alph*)}]
        \item The coefficient of $z^{n-2}$ in the characteristic polynomial of $T$ is
              \[
                  \sum_{1\leq i < j\leq n}\lambda_{i}\lambda_{j}.
              \]
        \item The coefficient of $z$ in the characteristic polynomial of $T$ is
              \[
                  {(-1)}^{n-1}\sum^{n}_{i=1}\left(\prod^{n}_{\stackrel{j=1}{j\ne i}}\lambda_{j}\right).
              \]
    \end{enumerate}
\end{proof}
\newpage

% chapter9:sectionC:exercise14
\begin{exercise}\label{chapter9:sectionC:exercise14}
    Suppose $V$ is an inner product space and $T$ is a positive operator on $V$. Prove that
    \[
        \det\sqrt{T} = \sqrt{\det T}.
    \]
\end{exercise}

\begin{proof}
    $T$ is a positive operator so $T$ is self-adjoint and hence normal. By the real and complex spectral theorems, there exists an orthonormal basis $e_{1}, \ldots, e_{\dim V}$ of $V$ which consists of eigenvectors of $T$. Let the eigenvalue of $T$ corresponding to $e_{k}$ be $\lambda_{k}$, for every $k\in\{1,\ldots,\dim V\}$.

    Because $T$ is a positive operator, all eigenvalues of $T$ are nonnegative, and $T$ has a unique positive square root $\sqrt{T}$. Moreover
    \[
        \sqrt{T}e_{k} = \sqrt{\lambda_{k}}e_{k}
    \]

    for every $k\in\{1,\ldots,\dim V\}$. Hence
    \[
        \det\sqrt{T} = \sqrt{\lambda_{1}}\cdots\sqrt{\lambda_{\dim V}} = \sqrt{\lambda_{1}\cdots\lambda_{\dim V}} = \sqrt{\det T}.
    \]

    Thus $\det\sqrt{T} = \sqrt{\det T}$.
\end{proof}
\newpage

% chapter9:sectionC:exercise15
\begin{exercise}\label{chapter9:sectionC:exercise15}
    Suppose $V$ is an inner product space and $T\in\lmap{V}$. Use the polar decomposition to give a proof that
    \[
        \abs{\det T} = \sqrt{\det(T^{*}T)}
    \]

    that is different from the proof given earlier (see 9.60).
\end{exercise}

\begin{proof}
    By the polar decomposition, there exists an unitary operator $S\in\lmap{V}$ such that $T = S\sqrt{T^{*}T}$. Since $\abs{\det S} = 1$, we have
    \[
        \abs{\det T} = \abs{\det S\sqrt{T^{*}T}} = \abs{(\det S)(\det\sqrt{T^{*}T})} = \abs{\det S}\abs{\det\sqrt{T^{*}T}} = \abs{\det\sqrt{T^{*}T}}.
    \]

    $T^{*}T$ is a positive operator on $V$. By Exercise~\ref{chapter9:sectionC:exercise14}, $\det\sqrt{T^{*}T} = \sqrt{\det(T^{*}T)}$, we obtain
    \[
        \abs{\det T} = \sqrt{\det(T^{*}T)}.
    \]
\end{proof}
\newpage

% chapter9:sectionC:exercise16
\begin{exercise}\label{chapter9:sectionC:exercise16}
    Suppose $T\in\lmap{V}$. Define $g: \mathbb{F}\to\mathbb{F}$ by $g(x) = \det(I + xT)$. Show that $g'(0) = \operatorname{tr}T$.
\end{exercise}

\begin{quote}
    Look for a clean solution to this exercise, without using the explicit but complicated formula for the determinant of a matrix.
\end{quote}

\begin{proof}
    $g$ is a polynomial function, so $g$ is differentiable.

    Let $n = \dim V$ and $e_{1}, \ldots, e_{n}$ be a basis of $V$. Let $\alpha$ be an alternating multilinear form in $V^{(\dim V)}_{\text{alt}}$ such that $\alpha(e_{1}, \ldots, e_{n}) = 1$.
    \begin{align*}
        \det(I + xT) & = (\det(I + xT))\alpha(e_{1}, \ldots, e_{n})                                                                        \\
                     & = \alpha(e_{1} + xTe_{1}, \ldots, e_{n} + xTe_{n})                                                                  \\
                     & = \alpha(e_{1}, \ldots, e_{n}) + x\alpha_{1}(e_{1}, \ldots, e_{n}) + \cdots + x^{n}\alpha_{n}(e_{1}, \ldots, e_{n}) \\
                     & = 1 + x\alpha_{1}(e_{1}, \ldots, e_{n}) + \cdots + x^{n}\alpha_{n}(e_{1}, \ldots, e_{n})
    \end{align*}

    where $\alpha_{k}(e_{1}, \ldots, e_{n})$ is defined as follows: $\operatorname{comb}(n, k)$ is the set of subsets of $k$ elements from $\{1,\ldots,n\}$
    \[
        \alpha_{k}(e_{1}, \ldots, e_{n}) = \sum_{(i_{1}, \ldots, i_{k})\in\operatorname{comb}(n,k)}\alpha(v_{1}, \ldots, v_{n})
    \]

    where $v_{j} = Te_{j}$ if $j\in\{ i_{1}, \ldots, i_{k} \}$, and otherwise, $v_{j} = e_{j}$.

    Therefore
    \begin{multline*}
        \det(I + xT) - 1 \\
        = x(\alpha(Te_{1}, \ldots, e_{n}) + \cdots + \alpha(e_{1}, \ldots, Te_{n})) + x^{2}\alpha_{2}(e_{1}, \ldots, e_{n}) + \cdots + x^{n}\alpha_{n}(e_{1}, \ldots, e_{n}).
    \end{multline*}

    Let $A$ be the matrix of $T$ with respect to $e_{1}, \ldots, e_{n}$, then
    \[
        Te_{j} = A_{1,j}e_{1} + \cdots + A_{n,j}e_{n}.
    \]

    Because $\alpha$ is an alternating multilinear form,
    \begin{align*}
        \alpha(Te_{1}, \ldots, e_{n}) + \cdots + \alpha(e_{1}, \ldots, Te_{n}) & = \alpha(A_{1,1}e_{1}, \ldots, e_{n}) + \cdots + \alpha(e_{1}, \ldots, A_{n,n}e_{n}) \\
                                                                               & = (A_{1,1} + \cdots + A_{n,n})\alpha(e_{1}, \ldots, e_{n})                           \\
                                                                               & = \operatorname{tr}T.
    \end{align*}

    Therefore
    \[
        \frac{\det(I + xT) - 1}{x} = \operatorname{tr}T + x\alpha_{2}(e_{1}, \ldots, e_{n}) + \cdots + x^{n-1}\alpha_{n}(e_{1}, \ldots,e_{n})
    \]

    so
    \[
        \lim\limits_{x\to 0}\frac{\det(I + xT) - 1}{x} = \operatorname{tr}T.
    \]

    Equivalently, $g'(0) = \operatorname{tr}T$.
\end{proof}
\newpage

% chapter9:sectionC:exercise17
\begin{exercise}\label{chapter9:sectionC:exercise17}
    Suppose $a, b, c$ are positive numbers. Find the volume of the ellipsoid
    \[
        \left\{ (x,y,z)\in\mathbb{R}^{3}: \frac{x^{2}}{a^{2}} + \frac{y^{2}}{b^{2}} + \frac{z^{2}}{c^{2}} < 1 \right\}
    \]

    by finding a set $\Omega\subseteq\mathbb{R}^{3}$ whose volume you know and an operator $T$ on $\mathbb{R}^{3}$ such that $T(\Omega)$ equals the ellipsoid above.
\end{exercise}

\begin{proof}
    Let $T$ be the operator on $\mathbb{R}^{3}$ defined by $T(x, y, z) = \left(\frac{x}{a}, \frac{y}{b}, \frac{z}{c}\right)$.

    \[
        (x, y, z)\in \left\{ (x,y,z)\in\mathbb{R}^{3}: \frac{x^{2}}{a^{2}} + \frac{y^{2}}{b^{2}} + \frac{z^{2}}{c^{2}} < 1 \right\}
    \]

    if and only if
    \[
        T(x, y, z) = \left(\frac{x}{a}, \frac{y}{b}, \frac{z}{c}\right)\in \Omega = \left\{ (x,y,z)\in\mathbb{R}^{3}: x^{2} + y^{2} + z^{2} < 1 \right\}.
    \]

    The volume of the ellipsoid (in fact, this is a sphere)
    \[
        \Omega = \left\{ (x,y,z)\in\mathbb{R}^{3}: x^{2} + y^{2} + z^{2} < 1 \right\}
    \]

    is $\frac{4}{3}\pi$. On the other hand
    \[
        \abs{T^{-1}(\Omega)} = \abs{\det T^{-1}}\abs{\Omega}
    \]

    and $\abs{\Omega} = \frac{4}{3}\pi$, $\det T^{-1} = abc$. Thus the volume of the given ellipsoid is $\frac{4}{3}\pi abc$.
\end{proof}
\newpage

% chapter9:sectionC:exercise18
\begin{exercise}\label{chapter9:sectionC:exercise18}
    Suppose that $A$ is an invertible square matrix. Prove that Hadamard's inequality (9.66) is an equality if and only if each column of $A$ is orthogonal to the other columns.
\end{exercise}

\begin{proof}
    I rewrite the proof of the Hadamard's inequality here.

    Let $n$ be the number of columns of $A$. $v_{1}, \ldots, v_{n} \in \mathbb{F}^{n}$ are the columns of $A$.

    If $A$ is not invertible, then $\abs{\det A} = 0 \leq \norm{v_{1}}\cdots \norm{v_{n}}$. Suppose $A$ is invertible. By the QR decomposition, there exists an unitary matrix $Q$ and an upper-triangular matrix $R$ whose entries on the diagonal are positive numbers such that $A = QR$.
    \begin{align*}
        \abs{\det A} & = \abs{\det Q}\abs{\det R}             \\
                     & = \abs{\det R}                         \\
                     & = \prod^{n}_{k=1}R_{k,k}               \\
                     & \leq \prod^{n}_{k=1}\norm{R_{\cdot,k}} \\
                     & = \prod^{n}_{k=1}\norm{QR_{\cdot,k}}   \\
                     & = \prod^{n}_{k=1}\norm{v_{k}}.
    \end{align*}

    Assume that $\abs{\det A} = \prod^{n}_{k=1}\norm{v_{k}}$.

    If $A$ is not invertible, then $\det A = 0$ and $\prod^{n}_{k=1}\norm{v_{k}} = 0$. $\prod^{n}_{k=1}\norm{v_{k}} = 0$ if and only if $v_{1} = \cdots = v_{n} = 0$, which implies $A = 0$. Therefore the columns of $A$ are pairwise orthogonal.

    If $A$ is invertible, the proof implies
    \[
        \prod^{n}_{k=1}R_{k,k} = \prod^{n}_{k=1}\norm{R_{\cdot,k}}.
    \]

    This happens if and only if $R$ is a diagonal matrix. Since $R$ is a diagonal matrix, the $k$th column of $A$ is $Q_{\cdot,k}R_{k,k}$. Because the columns of $Q$ are pairwise orthogonal, then $Q_{\cdot,1}R_{1,1}, \ldots, Q_{\cdot,n}R_{n,n}$ are pairwise orthogonal. Thus the columns of $A$ are pairwise orthogonal.

    \bigskip
    Assume that the columns of $A$ are pairwise orthogonal.

    If $A$ is not invertible, then the columns of $A$ are linearly dependent. Moreover, the columns of $A$ are pairwise orthogonal so there must be at least one column consisting of all $0$ (because otherwise, they are linearly independent). Therefore $\abs{\det A} = 0 = \prod^{n}_{k=1}\norm{v_{k}}$.

    If $A$ is invertible, all columns of $A$ are nonzero vectors of $\mathbb{F}^{n}$, so
    \[
        A = \prod^{n}_{k=1}\norm{v_{k}}\begin{pmatrix}
            A_{1,1}/\norm{v_{1}} & \cdots & A_{1,n}/\norm{v_{n}} \\
            \vdots               &        & \vdots               \\
            A_{n,1}/\norm{v_{1}} & \cdots & A_{n,n}/\norm{v_{n}}
        \end{pmatrix}
    \]

    where the columns of
    \[
        \begin{pmatrix}
            A_{1,1}/\norm{v_{1}} & \cdots & A_{1,n}/\norm{v_{n}} \\
            \vdots               &        & \vdots               \\
            A_{n,1}/\norm{v_{1}} & \cdots & A_{n,n}/\norm{v_{n}}
        \end{pmatrix}
    \]

    constitute an orthonormal basis of $\mathbb{F}^{n}$, which means this matrix is unitary. Therefore $\abs{\det A} = \prod^{n}_{k=1}\norm{v_{k}}$.

    Thus the Hadamard's inequality is an equality if and only if the columns of $A$ are pairwise orthogonal.
\end{proof}
\newpage

% chapter9:sectionC:exercise19
\begin{exercise}\label{chapter9:sectionC:exercise19}
    Suppose $V$ is an inner product space, $e_{1}, \ldots, e_{n}$ is an orthonormal basis of $V$, and $T\in\lmap{V}$ is a positive operator.
    \begin{enumerate}[label={(\alph*)}]
        \item Prove that $\det T\leq \prod^{n}_{k=1}\innerprod{Te_{k}, e_{k}}$.
        \item Prove that if $T$ is invertible, then the inequality in (a) is an equality if and only if $e_{k}$ is an eigenvector of $T$ for each $k = 1,\ldots,n$.
    \end{enumerate}
\end{exercise}

\begin{proof}
    \begin{enumerate}[label={(\alph*)}]
        \item Let $A$ be the matrix of $\sqrt{T}$ with respect to the basis $e_{1}, \ldots, e_{n}$ of $V$, then the $k$th column $v_{k}$ of $A$ is the coordinates of $\sqrt{T}e_{k}$ with respect to $e_{1}, \ldots, e_{n}$. By the Pythagorean theorem and the Parseval's identity (we use the standard inner products on $\mathbb{F}^{n}$ and the inner product on $V$),
              \begin{align*}
                  \norm{v_{k}}^{2} = \abs{A_{1,k}}^{2} + \cdots + \abs{A_{n,k}}^{2} = \abs{\innerprod{\sqrt{T}e_{k}, e_{1}}}^{2} + \cdots +\abs{\innerprod{\sqrt{T}e_{k}, e_{n}}}^{2} = \norm{\sqrt{T}e_{k}}^{2}.
              \end{align*}

              for every $k\in\{1,\ldots,n\}$.
              \begin{align*}
                  \det T & = {(\det \sqrt{T})}^{2}                                   & \text{(Exercise~\ref{chapter9:sectionC:exercise14})} \\
                         & \leq \prod^{n}_{k=1}\norm{v_{k}}^{2}                      & \text{(Hadamard's inequality)}                       \\
                         & = \prod^{n}_{k=1}\norm{v_{k}}^{2}                                                                                \\
                         & = \prod^{n}_{k=1}\innerprod{\sqrt{T}e_{k}, \sqrt{T}e_{k}}                                                        \\
                         & = \prod^{n}_{k=1}\innerprod{Te_{k}, e_{k}}                & \text{($T$ is a positive operator)}
              \end{align*}

              Thus $\det T\leq \prod^{n}_{k=1}\innerprod{Te_{k}, e_{k}}$.
        \item According to part (a) and the equality condition of Hadamard's inequality (Exercise~\ref{chapter9:sectionC:exercise18}), when $T$ is invertible, the inequality in (a) is an equality if and only if the columns of $\mathcal{M}(\sqrt{T}, (e_{1}, \ldots, e_{n}))$ are pairwise orthogonal.


              Suppose (a) is an equality and $T$ is invertible.

              The columns of $\mathcal{M}(\sqrt{T}, (e_{1}, \ldots, e_{n}))$ are pairwise orthogonal if and only if $\innerprod{\sqrt{T}e_{k}, \sqrt{T}e_{j}} = 0$ for all $j\ne k$. Equivalently, $\innerprod{Te_{k}, e_{j}} = 0$ for all $j\ne k$, this means the matrix of $T$ with respect to $e_{1}, \ldots, e_{n}$ is a diagonal matrix. Therefore $e_{1}, \ldots, e_{n}$ are eigenvectors of $T$.

              Suppose $e_{1}, \ldots, e_{n}$ are eigenvectors of $T$. Let $\lambda_{k}$ be the eigenvalue of $T$ corresponding to $e_{k}$ for every $k\in\{1,\ldots,n\}$, then
              \[
                  \det T = \prod^{n}_{k=1}\lambda_{k} = \prod^{n}_{k=1}\innerprod{Te_{k}, e_{k}}.
              \]

              Thus if $T$ is invertible, the inequality in (a) is an equality if and only if $e_{1}, \ldots, e_{n}$ are eigenvectors of $T$.
    \end{enumerate}
\end{proof}
\newpage

% chapter9:sectionC:exercise20
\begin{exercise}\label{chapter9:sectionC:exercise20}
    Suppose $A$ is an $n$-by-$n$ matrix, and suppose $c$ is such that $\abs{A_{j,k}}\leq c$ for all $j, k\in\{1,\ldots,n\}$. Prove that
    \[
        \abs{\det A}\leq c^{n}n^{n/2}.
    \]
\end{exercise}

\begin{quote}
    The formula for the determinant of a matrix (9.46) shows that $\abs{\det A}\leq c^{n}n{!}$. However, the estimate given by this exercise is much better. For example, if $c = 1$ and $n = 100$, then $c^{n}n! \approx 10^{158}$, but the estimate given by this exercise is the much smaller number $10^{100}$. If $n$ is an integer power of $2$, then the inequality above is sharp and cannot be improved.
\end{quote}

\begin{proof}
    Let $v_{1}, \ldots, v_{n}\in\mathbb{F}^{n}$ be the columns of $A$. By the Hadamard's inequality
    \[
        \abs{\det A}\leq \prod^{n}_{k=1}\norm{v_{k}}.
    \]

    For every $k\in\{1,\ldots,n\}$
    \[
        \norm{v_{k}}^{2} = \abs{A_{1,k}}^{2} + \cdots + \abs{A_{n,k}}^{2} \leq nc^{2}.
    \]

    Therefore
    \[
        \abs{\det A}\leq {(c\sqrt{n})}^{n} = c^{n}n^{n/2}.
    \]
\end{proof}
\newpage

% chapter9:sectionC:exercise21
\begin{exercise}\label{chapter9:sectionC:exercise21}
    Suppose $n$ is a positive integer and $\delta: \mathbb{C}^{n,n}\to \mathbb{C}$ is a function such that
    \[
        \delta(AB) = \delta(A)\cdot\delta(B)
    \]

    for all $A, B\in\mathbb{C}^{n,n}$ and $\delta(A)$ equals the product of the diagonal entries of $A$ for each diagonal matrix $A\in\mathbb{C}^{n,n}$. Prove that
    \[
        \delta(A) = \det A
    \]

    for all $A\in\mathbb{C}^{n,n}$.
\end{exercise}

\begin{quote}
    Recall that $\mathbb{C}^{n,n}$ denotes set of $n$-by-$n$ matrices with entries in $\mathbb{C}$. This exercise shows that the determinant is the unique function defined on square matrices that is multiplicative and has the desired behavior on diagonal matrices. This result is analogous to Exercise~\ref{chapter8:sectionD:exercise10}, which shows that the trace is uniquely determined by its algebraic properties.
\end{quote}

\begin{proof}
    If $A$ is invertible, then $\delta(A)\delta(A^{-1}) = \delta(I) = 1\ne 0$, so $\delta(A)\ne 0$.

    If $A$ is not invertible, then there exists $v_{1}\in\mathbb{C}^{n,1}$ and $v_{1}\ne 0$ such that $Av_{1} = 0$. Extend $v_{1}$ to a basis $v_{1}, \ldots, v_{n}$ of $\mathbb{C}^{n,1}$, let $B$ be a matrix whose $k$th column is $v_{k}$, then
    \[
        AB = \begin{pmatrix}
            0      & *      & \cdots & *      \\
            0      & *      & \cdots & *      \\
            \vdots & \vdots &        & \vdots \\
            0      & *      & \cdots & *
        \end{pmatrix}
        =
        \begin{pmatrix}
            0      & *      & \cdots & *      \\
            0      & *      & \cdots & *      \\
            \vdots & \vdots &        & \vdots \\
            0      & *      & \cdots & *
        \end{pmatrix}
        \begin{pmatrix}
            0      & 0      & \cdots & 0      \\
            0      & 1      & \cdots & 0      \\
            \vdots & \vdots &        & \vdots \\
            0      & 0      & \cdots & 1
        \end{pmatrix}.
    \]

    Therefore $\delta(A)\delta(B) = \delta(AB) = 0$. Because $\delta(B)\ne 0$ (since $B$ is invertible), it follows that $\delta(A) = 0$. Hence $\delta(A) = 0$ if $A$ is not invertible.

    By the Schur's decomposition theorem, for each $A\in\mathbb{C}^{n,n}$ there exists an unitary matrix $Q$ such that $Q^{*}AQ$ is an upper-triangular matrix.
    \[
        \delta(Q^{*}AQ) = \delta(Q^{*})\delta(A)\delta(Q) = \delta(Q^{*})\delta(Q)\delta(A) = \delta(Q^{*}Q)\delta(A) = \delta(I)\delta(A) = \delta(A).
    \]

    Suppose $A$ is not invertible, then $\delta(A) = 0 = \det A$.

    Suppose $A$ is invertible, then all entries on the diagonal of $Q^{*}AQ$ are nonzero. Let $U = Q^{*}AQ$ and choose nonzero complex numbers $\lambda_{1}, \ldots, \lambda_{n}$ such that $U_{1,1}/\lambda_{1}, \ldots, U_{n,n}/\lambda_{n}$ are pairwise distinct, then
    \[
        \begin{pmatrix}
            U_{1,1} & U_{1,2} & \cdots & U_{1,n} \\
            0       & U_{2,2} & \cdots & U_{2,n} \\
            \vdots  & \vdots  &        & \vdots  \\
            0       & 0       & \cdots & U_{n,n}
        \end{pmatrix} =
        \begin{pmatrix}
            U_{1,1}/\lambda_{1} & U_{1,2}/\lambda_{2} & \cdots & U_{1,n}/\lambda_{n} \\
            0                   & U_{2,2}/\lambda_{2} & \cdots & U_{2,n}/\lambda_{n} \\
            \vdots              & \vdots              &        & \vdots              \\
            0                   & 0                   & \cdots & U_{n,n}/\lambda_{n}
        \end{pmatrix}
        \begin{pmatrix}
            \lambda_{1} & 0           & \cdots & 0           \\
            0           & \lambda_{2} & \cdots & 0           \\
            \vdots      & \vdots      &        & \vdots      \\
            0           & 0           & \cdots & \lambda_{n}
        \end{pmatrix}.
    \]

    The upper-triangular matrix
    \[
        W = \begin{pmatrix}
            U_{1,1}/\lambda_{1} & U_{1,2}/\lambda_{2} & \cdots & U_{1,n}/\lambda_{n} \\
            0                   & U_{2,2}/\lambda_{2} & \cdots & U_{2,n}/\lambda_{n} \\
            \vdots              & \vdots              &        & \vdots              \\
            0                   & 0                   & \cdots & U_{n,n}/\lambda_{n}
        \end{pmatrix}
    \]

    has distinct entries on the diagonal, so it has distinct $n$ eigenvalues (which are precisely the entries on the diagonal). Therefore it is diagonalizable, hence there exists an invertible matrix $C$ such that $C^{-1}WC$ is a diagonal matrix.
    \[
        \delta(C^{-1}WC) = \delta(C^{-1})\delta(W)\delta(C) = \delta(C^{-1})\delta(C)\delta(W) = \delta(C^{-1}C)\delta(W) = \delta(W).
    \]

    Hence $\delta(U) = \delta(W)\lambda_{1}\cdots\lambda_{n} = U_{1,1}\cdots U_{n,n}$. Moreover, $U_{1,1}, \ldots, U_{n,n}$ are also eigenvalues of $U$, each appears as many times as its multiplicity, so $\det U =  U_{1,1}\cdots U_{n,n}$. Therefore
    \[
        \delta(A) = \delta(Q^{*}AQ) = \delta(U) = U_{1,1}\cdots U_{n,n} = \det U = \det (Q^{*}AQ) = \det A.
    \]

    Thus for all $A\in\mathbb{C}^{n,n}$, $\delta(A) = \det A$.
\end{proof}
\newpage

\section{Tensor Products}



\end{document}
