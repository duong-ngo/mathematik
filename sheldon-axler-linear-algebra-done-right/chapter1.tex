\chapter{Vector Spaces}

\section{$\mathbb{R}^{n}$ and $\mathbb{C}^{n}$}

% chapter 1/section A/exercise 1
\begin{exercise}
    Show that $\alpha + \beta = \beta + \alpha$ for all $\alpha, \beta\in \mathbb{C}$.
\end{exercise}

\begin{proof}
    Let $a_{1}, a_{2}$ be real and imaginary part of $\alpha$, $b_{1}, b_{2}$ be real and imaginary part of $\beta$. According to the definition of addition operation on $\mathbb{C}$, the real parts of $\alpha + \beta$ and $\beta + \alpha$ are $a_{1} + b_{1}$ and $b_{1} + a_{1}$, the imaginary parts of $\alpha + \beta$ and $\beta + \alpha$ are $a_{2} + b_{2}$ and $b_{2} + a_{2}$. On the other hand, addition operation on $\mathbb{R}$ is commutative, so $a_{1} + b_{1} = b_{1} + a_{1}$ and $a_{2} + b_{2} = b_{2} + a_{2}$. Hence $\alpha + \beta = \beta + \alpha$ for all $\alpha, \beta\in \mathbb{C}$
\end{proof}

% chapter 1/section A/exercise 2
\begin{exercise}
    Show that $(\alpha + \beta) + \lambda = \alpha + (\beta + \lambda)$ for all $\alpha, \beta, \lambda\in \mathbb{C}$.
\end{exercise}

\begin{proof}
    Let $a_{1}, a_{2}$ be real and imaginary part of $\alpha$, $b_{1}, b_{2}$ be real and imaginary part of $\beta$, $c_{1}, c_{2}$ be real and imaginary part of $\lambda$. According to the definition of addition operation on $\mathbb{C}$, the real parts of $(\alpha + \beta) + \lambda$ and $\alpha + (\beta + \lambda)$ are $(a_{1} + b_{1}) + c_{1}$ and $a_{1} + (b_{1} + c_{1})$, the imaginary parts of $(\alpha + \beta) + \lambda$ and $\alpha + (\beta + \lambda)$ are $(a_{2} + b_{2}) + c_{2}$ and $a_{2} + (b_{2} + c_{2})$. On the other hand, addition operation on $\mathbb{R}$ is associative, so $(a_{1} + b_{1}) + c_{1} = a_{1} + (b_{1} + c_{1})$ and $(a_{2} + b_{2}) + c_{2} = a_{2} + (b_{2} + c_{2})$. Hence $(\alpha + \beta) + \lambda = \alpha + (\beta + \lambda)$ for all $\alpha, \beta, \lambda\in \mathbb{C}$.
\end{proof}

% chapter 1/section A/exercise 3
\begin{exercise}
    Show that $(\alpha\beta)\lambda = \alpha(\beta\lambda)$ for all $\alpha, \beta, \lambda\in \mathbb{C}$.
\end{exercise}

\begin{proof}
    Let $a_{1}, a_{2}$ be real and imaginary part of $\alpha$, $b_{1}, b_{2}$ be real and imaginary part of $\beta$, $c_{1}, c_{2}$ be real and imaginary part of $\lambda$. According to the definition of multiplication operation on $\mathbb{C}$,
    \begin{align*}
        (\alpha\beta)\lambda & = ((a_{1}b_{1} - a_{2}b_{2}) + (a_{1}b_{2} + a_{2}b_{1})\iota) (c_{1} + c_{2}\iota)                                                                       \\
                             & = (a_{1}b_{1}c_{1} - a_{2}b_{2}c_{1} - a_{1}b_{2}c_{2} - a_{2}b_{1}c_{2}) + (a_{1}b_{1}c_{2} + a_{1}b_{2}c_{1} + a_{2}b_{1}c_{1} - a_{2}b_{2}c_{2})\iota, \\
        \alpha(\beta\lambda) & = (a_{1} + a_{2}\iota)((b_{1}c_{1} - b_{2}c_{2}) + (b_{1}c_{2} + b_{2}c_{1})\iota)                                                                        \\
                             & = (a_{1}b_{1}c_{1} - a_{1}b_{2}c_{2} - a_{2}b_{1}c_{2} - a_{2}b_{2}c_{1}) + (a_{1}b_{1}c_{2} + a_{1}b_{2}c_{1} + a_{2}b_{1}c_{1} - a_{2}b_{2}c_{2})\iota.
    \end{align*}

    Hence $(\alpha\beta)\lambda = \alpha(\beta\lambda)$ for all $\alpha, \beta, \lambda\in\mathbb{C}$.
\end{proof}

% chapter 1/section A/exercise 4
\begin{exercise}
    Show that $\lambda (\alpha + \beta) = \lambda\alpha + \lambda\beta$ for all $\lambda, \alpha, \beta\in\mathbb{C}$.
\end{exercise}

\begin{proof}
    Let $a_{1}, a_{2}$ be real and imaginary part of $\alpha$, $b_{1}, b_{2}$ be real and imaginary part of $\beta$, $c_{1}, c_{2}$ be real and imaginary part of $\lambda$. According to the definition of addition and multiplication operation on $\mathbb{C}$,
    \begin{align*}
        \lambda(\alpha + \beta)      & = (c_{1} + c_{2}\iota)((a_{1} + b_{1}) + (a_{2} + b_{2})\iota)                                                                \\
                                     & = (c_{1}a_{1} + c_{1}b_{1} - c_{2}a_{2} - c_{2}b_{2}) + (c_{1}a_{2} + c_{1}b_{2} + c_{2}a_{1} + c_{2}b_{1})\iota,             \\
        \lambda\alpha + \lambda\beta & = (c_{1} + c_{2}\iota)(a_{1} + a_{2}\iota) + (c_{1} + c_{2}\iota)(b_{1} + b_{2}\iota)                                         \\
                                     & = ((c_{1}a_{1} - c_{2}a_{2}) + (c_{2}a_{1} + c_{1}a_{2})\iota) + ((c_{1}b_{1} - b_{2}c_{2}) + (c_{1}b_{2} + c_{2}b_{1})\iota) \\
                                     & = (c_{1}a_{1} + c_{1}b_{1} - c_{2}a_{2} - c_{2}b_{2}) + (c_{1}a_{2} + c_{2}a_{1} + c_{1}b_{2} + c_{2}b_{1})\iota.
    \end{align*}

    Hence $\lambda(\alpha + \beta) = \lambda\alpha + \lambda\beta$ for every $\alpha, \beta, \lambda\in\mathbb{C}$.
\end{proof}

% chapter 1/section A/exercise 5
\begin{exercise}
    Show that for every $\alpha\in\mathbb{C}$, there exists a unique $\beta\in\mathbb{C}$ such that $\alpha + \beta = 0$.
\end{exercise}

\begin{proof}
    Let $a_{1}, a_{2}$ be real and imaginary part of $\alpha$, then $\beta = (-a_{1}) + (-a_{2})\iota$ satisfies $\alpha + \beta = 0$.

    Assume that complex number $\lambda$ satisfies $\alpha + \lambda = 0$, then
    \[
        \lambda = \lambda + 0 = \lambda + (\alpha + \beta) = (\lambda + \alpha) + \beta = 0 + \beta = \beta.
    \]

    Hence for every $\alpha\in \mathbb{C}$, there exists a unique $\beta\in \mathbb{C}$ such that $\alpha + \beta = 0$.
\end{proof}

% chapter 1/section A/exercise 6
\begin{exercise}
    Show that for every $\alpha\in\mathbb{C}$ with $\alpha\ne 0$, there exists a unique $\beta\in\mathbb{C}$ such that $\alpha\beta = 1$.
\end{exercise}

\begin{proof}
    Let $a_{1}, a_{2}$ be real and imaginary part of $\alpha$, then $\beta = \dfrac{a_{1}}{{a_{1}}^{2} + {a_{2}}^{2}} + \frac{-a_{2}}{{a_{1}}^{2} + {a_{2}}^{2}}\iota$ satisfies $\alpha\beta = 1$.

    Assume that complex number $\lambda$ satisfies $\alpha\lambda = 1$, then
    \[
        \lambda = \lambda\cdot 1 = \lambda(\alpha\beta) = (\lambda\alpha)\beta = 1\cdot\beta = \beta.
    \]

    Hence for every nonzero $\alpha\in\mathbb{C}$, there exists a unique $\beta\in\mathbb{C}$ such that $\alpha\beta = 1$.
\end{proof}

% chapter 1/section A/exercise 7
\begin{exercise}
    Show that
    \[
        \frac{-1 + \sqrt{3}\iota}{2}
    \]

    is a cube root of $1$ (meaning that its cube equals $1$).
\end{exercise}

\begin{proof}
    \begin{align*}
        {\left(\frac{-1 + \sqrt{3}\iota}{2}\right)}^{3} & = \frac{{(-1 + \sqrt{3}\iota)}^{3}}{8}                                                                      \\
                                                        & = \frac{(-1) + 3\cdot{(-1)}^{2}\sqrt{3}\iota + 3\cdot (-1){(\sqrt{3}\iota)}^{2} + {(\sqrt{3}\iota)}^{3}}{8} \\
                                                        & = \frac{(-1) + 3\sqrt{3}\iota + 9 + (-3\sqrt{3}\iota)}{8}                                                   \\
                                                        & = \frac{8}{8}                                                                                               \\
                                                        & = 1.
    \end{align*}
\end{proof}

% chapter 1/section A/exercise 8
\begin{exercise}
    Find two distinct square roots of $\iota$.
\end{exercise}

\begin{proof}
    We will find complex roots of the equation $z^{2} = \iota$.

    Let $z = a + b\iota$, where $a$ and $b$ are real numbers. $z^{2} = {(a + b\iota)}^{2} = (a^{2} - b^{2}) + 2ab\iota = \iota$. From this, we deduce that the real and imaginary part of both sides are identical, therefore $a^{2} - b^{2} = 0$ and $2ab = 1$.

    Since $a^{2} - b^{2} = 0$, then $a = b$ or $a = -b$. If $a = b$, then $2a^{2} = 2b^{2} = 1$, equivalently, $a = b = \dfrac{\sqrt{2}}{2}$ or $a = b = \dfrac{-\sqrt{2}}{2}$. Otherwise, $a = -b$, then $1 = 2ab = -2a^{2} < 0$, which is impossible.

    Hence the complex square roots of $\iota$ are $\dfrac{\sqrt{2}(1 + \iota)}{2}$ and $\dfrac{-\sqrt{2}(1 + \iota)}{2}$.
\end{proof}

% chapter 1/section A/exercise 9
\begin{exercise}
    Find $x\in \mathbb{R}^{4}$ such that
    \[
        (4, -3, 1, 7) + 2x = (5, 9, -6, 8).
    \]
\end{exercise}

\begin{proof}
    \begin{align*}
        2x & = (5, 9, -6, 8) - (4, -3, 1, 7) \\
           & = (1, 12, -7, 1)
    \end{align*}

    So $x = \left(\dfrac{1}{2}, 6, \dfrac{-7}{2}, \dfrac{1}{2}\right)$.
\end{proof}

% chapter 1/section A/exercise 10
\begin{exercise}
    Explain why there does not exist $\lambda\in\mathbb{C}$ such that
    \[
        \lambda (2 - 3\iota, 5 + 4\iota, -6 + 7\iota) = (12 - 5\iota, 7 + 22\iota, -32 - 9\iota).
    \]
\end{exercise}

\begin{proof}
    Assume that there does exists such a complex number $\lambda$, then
    \[
        \begin{split}
            \abs{\lambda}\cdot\abs{2 - 3\iota} = \abs{12 - 5\iota} \\
            \abs{\lambda}\cdot\abs{5 + 4\iota} = \abs{7 + 22\iota} \\
            \abs{\lambda}\cdot\abs{-6 + 7\iota} = \abs{-32 - 9\iota}
        \end{split}
    \]

    equivalently,
    \[
        \begin{split}
            \abs{\lambda}\sqrt{13} = 13,         \\
            \abs{\lambda}\sqrt{41} = \sqrt{533}, \\
            \abs{\lambda}\sqrt{85} = \sqrt{1105}.
        \end{split}
    \]

    But there is no complex number $\lambda$ satisfying the above three equations, since $\dfrac{13}{\sqrt{13}}\ne \dfrac{\sqrt{533}}{\sqrt{41}}$. Thus there does not exist $\lambda\in\mathbb{C}$ such that $\lambda (2 - 3\iota, 5 + 4\iota, -6 + 7\iota) = (12 - 5\iota, 7 + 22\iota, -32 - 9\iota)$.
\end{proof}

% chapter 1/section A/exercise 11
\begin{exercise}
    Show that $(x + y) + z = x + (y + z)$ for all $x, y, z\in \mathbb{F}^{n}$.
\end{exercise}

\begin{proof}
    The $i$th component of $(x + y) + z$ and $x + (y + z)$ are $(x_{i} + y_{i}) + z_{i}$ and $x_{i} + (y_{i} + z_{i})$. Since addition operation on $\mathbb{F}$ is associative, then $(x_{i} + y_{i}) + z_{i}$ and $x_{i} + (y_{i} + z_{i})$ are equal. So $(x_{i} + y_{i}) + z_{i} = x_{i} + (y_{i} + z_{i})$ for every $i\in\mathbb{N}$ and $i\leq n$. Thus $(x + y) + z = x + (y + z)$ for all $x, y, z\in \mathbb{F}^{n}$.
\end{proof}

% chapter 1/section A/exercise 12
\begin{exercise}
    Show that $(ab)x = a(bx)$ for all $x\in\mathbb{F}^{n}$ and all $a, b\in\mathbb{F}$.
\end{exercise}

\begin{proof}
    The $i$th component of $(ab)x$ and $a(bx)$ are $(ab)x_{i}$ and $a(bx_{i})$. Since multiplication operation on $\mathbb{F}$ is associative, then $(ab)x_{i} = a(bx_{i})$ for every $a, b, x_{i}\in\mathbb{F}$. So $(ab)x_{i} = a(bx_{i})$ for every $i\in\mathbb{N}$ and $i\leq n$. Thus $(ab)x = a(bx)$ for all $x\in\mathbb{F}^{n}$ and all $a, b\in\mathbb{F}$.
\end{proof}

% chapter 1/section A/exercise 13
\begin{exercise}
    Show that $1x = x$ for all $x\in\mathbb{F}^{n}$.
\end{exercise}

\begin{proof}
    For every $i\in\mathbb{N}$ and $i\leq n$, the $i$th component of $1x$ is $1x_{i} = x_{i}$, which is equal to the $i$th component of $x$. Thus $1x = x$ for all $x\in\mathbb{F}^{n}$.
\end{proof}

% chapter 1/section A/exercise 14
\begin{exercise}
    Show that $\lambda (x + y) = \lambda x + \lambda y$ for all $\lambda\in\mathbb{F}$ and all $x, y\in \mathbb{F}^{n}$.
\end{exercise}

\begin{proof}
    For every $i\in\mathbb{N}$ and $i\leq n$, the $i$th component of $\lambda (x + y)$ is $\lambda (x_{i} + y_{i}) = \lambda x_{i} + \lambda y_{i}$, which is also the $i$th component of $\lambda x + \lambda y$. Thus $\lambda (x + y) = \lambda x + \lambda y$.
\end{proof}

% chapter 1/section A/exercise 15
\begin{exercise}
    Show that $(a + b)x = ax + bx$ for all $a, b\in\mathbb{F}$ and all $x\in \mathbb{F}^{n}$.
\end{exercise}

\begin{proof}
    For every $i\in\mathbb{N}$ and $i\leq n$, the $i$th component of $(a + b)x$ is $(a + b)x_{i} = a x_{i} + b x_{i}$, which is also the $i$th component of $a x + b x$. Thus $(a + b)x = a x + b x$.
\end{proof}

\section{Definition of Vector Space}

% chapter 1/section B/exercise 1
\begin{exercise}
    Prove that $-(-v) = v$ for every $v\in V$.
\end{exercise}

\begin{proof}
    $-v$ is the additive inverse of $v$, and $v$ is the additive inverse of $-v$. On the other hand, $-(-v)$ is the additive inverse of $-v$, then due to the uniqueness of additive inverse, we conclude that $-(-v) = v$ for every $v\in V$.
\end{proof}

% chapter 1/section B/exercise 2
\begin{exercise}
    Suppose $a\in\mathbb{F}, v\in V$, and $av = 0$. Prove that $a = 0$ or $v = 0$.
\end{exercise}

\begin{proof}
    Assume that $a\ne 0$ and $v\ne 0$. Then there exists $a^{-1}$ such that $aa^{-1} = a^{-1}a = 1$. Hence
    \[
        0 = av = a^{-1}(av) = (a^{-1}a)v = 1v = v
    \]

    which contradicts the assumption. Thus $av = 0$ implies $a = 0$ or $v = 0$.
\end{proof}

% chapter 1/section B/exercise 3
\begin{exercise}
    Suppose $v, w\in V$. Explain why there exists a unique $x\in V$ such that $v + 3x = w$.
\end{exercise}

\begin{proof}
    $v + 3x = w$ is equivalent to $3x = w - v$, which is equivalent to $x = \frac{1}{3}w + \frac{-1}{3}v$, which uniquely determines $x$. Thus there exists a unique $x\in V$ such that $v + 3x = w$.
\end{proof}

% chapter 1/section B/exercise 4
\begin{exercise}
    The empty set is not a vector space. The empty set fails to satisfy only one of the requirements listed in the definition of a vector space (1.20). Which one?
\end{exercise}

\begin{proof}
    The empty set is not a vector space because it has no element, therefore no additive identity.
\end{proof}

% chapter 1/section B/exercise 5
\begin{exercise}
    Show that in the definition of a vector space (1.20), the additive inverse condition can be replaced with the condition that
    \[
        0v = 0\text{ for all }v\in V
    \]

    Here the $0$ on the left side is the number $0$, and the $0$ on the right side is the additive identity of $V$.
\end{exercise}

\begin{proof}
    We will show that the additive inverse condition can be derived from the new definition.

    Let $v$ be an arbitrary vector of $V$, then
    \[
        0 = 0v = (1 + (-1)v) = 1v + (-1)v = v + (-1)v.
    \]

    So $(-1)v$ is an additive inverse of $v$. Therefore every vector of $V$ has an additive inverse. Thus the new definition of vector space is equivalent to the original one.
\end{proof}

% chapter 1/section B/exercise 6
\begin{exercise}
    Let $\infty$ and $-\infty$ denote two distinct objects, neither of which is in $\mathbb{R}$. Define an addition and scalar multiplication on $\mathbb{R}\cup \{ \infty, -\infty \}$ as you could guess from the notation. Specifically, the sum and product of two real numbers is as usual, and for $t\in\mathbb{R}$ define
    \[
        t\infty = \begin{cases}
            -\infty & \text{if $t < 0$}, \\
            0       & \text{if $t = 0$}, \\
            \infty  & \text{if $t > 0$},
        \end{cases}
        \qquad
        t(-\infty) = \begin{cases}
            \infty  & \text{if $t < 0$}, \\
            0       & \text{if $t = 0$}, \\
            -\infty & \text{if $t > 0$},
        \end{cases}
    \]

    and
    \begin{align*}
        t + \infty         & = \infty + t = \infty + \infty = \infty,           \\
        t + (-\infty)      & = (-\infty) + t = (-\infty) + (-\infty) = -\infty, \\
        \infty + (-\infty) & = (-\infty) + \infty = 0.
    \end{align*}

    With these operations of addition and scalar multiplication, is $\mathbb{R}\cup \{ \infty, -\infty \}$ a vector space over $\mathbb{R}$? Explain.
\end{exercise}

\begin{proof}
    No, it is not a real vector space. Because the associativity of addition is not satisfied. Let $t$ be a nonzero real number, then
    \begin{align*}
        (\infty + (-\infty)) + t & = 0 + t = t,              \\
        \infty + ((-\infty) + t) & = \infty + (-\infty) = 0.
    \end{align*}
\end{proof}

% chapter 1/section B/exercise 7
\begin{exercise}
    Suppose $S$ is a nonempty set. Let $V^{S}$ denote the set of functions from $S$ to $V$. Define a natural addition and scalar multiplication on $V^{S}$, and show that $V^{S}$ is a vector with these definitions.
\end{exercise}

\begin{proof}
    Let $f$, $g$ be arbitrary functions from $S$ to $V$. We define $f + g$ and $\lambda f$ as the following:
    \begin{align*}
        (f + g)(x)     & = f(x) + g(x)  & \text{for all $x\in S$},                                 \\
        (\lambda f)(x) & = \lambda f(x) & \text{for all $\lambda\in\mathbb{F}$, and all $x\in S$}.
    \end{align*}

    \begin{itemize}
        \item \textbf{commutativity} is satisfied, since
              \[
                  (f + g)(x) = f(x) + g(x) = g(x) + f(x) = (g + f)(x) \quad\text{for all $f, g\in V^{S}$, and all $x\in S$}.
              \]
        \item \textbf{associativity} is satisfied, since
              \begin{align*}
                  ((f + g) + h)(x) & = (f + g)(x) + h(x)                                                           \\
                                   & = (f(x) + g(x)) + h(x)                                                        \\
                                   & = f(x) + (g(x) + h(x))                                                        \\
                                   & = f(x) + (g + h)(x)                                                           \\
                                   & = (f + (g + h))(x) \quad \text{for all $f, g, h\in V^{S}$, and all $x\in S$.}
              \end{align*}
              \begin{align*}
                  ((\lambda_{1}\lambda_{2})f)(x) & = (\lambda_{1}\lambda_{2})f(x)                                                                                                      \\
                                                 & = \lambda_{1}(\lambda_{2}f(x))                                                                                                      \\
                                                 & = \lambda_{1}(\lambda_{2}f)(x)                                                                                                      \\
                                                 & = (\lambda_{1}(\lambda_{2}f))(x) \quad \text{for all $f\in V^{S}$, all $\lambda_{1}, \lambda_{2}\in \mathbb{F}$, and all $x\in S$.}
              \end{align*}
        \item \textbf{additive identity} is satisfied, since with element $0\in V^{S}$ such that $0(x) = 0$ for all $x\in S$, then
              \[
                  (f + 0)(x) = f(x) + 0(x) = f(x) + 0 = f(x) = 0 + f(x) = 0(x) + f(x) = (0 + f)(x)
              \]

              for all $f\in V^{S}$, and all $x\in S$.
        \item \textbf{additive inverse} is satisfied, since with $-f\in V^{S}$ such that $(-f)(x) = -f(x)$ for all $x\in S$, then
              \[
                  (f + (-f))(x) = f(x) + (-f)(x) = f(x) + (-f(x)) = 0
              \]

              for all $x\in S$.
        \item \textbf{multiplicative identity} is satisfied, since for all $f\in V^{S}$ and all $x\in S$,
              \[
                  (1\cdot f)(x) = 1\cdot f(x) = f(x).
              \]
        \item \textbf{distributivity properties} are satisfied, since for all $\lambda_{1}, \lambda_{2}\in \mathbb{F}$, all $f, g\in V^{S}$, and all $x\in S$,
              \[
                  (\lambda_{1}(f + g))(x) = \lambda_{1}(f + g)(x) = \lambda_{1}(f(x) + g(x)) = (\lambda_{1}f)(x) + (\lambda_{1}g)(x) = (\lambda_{1}f + \lambda_{1}g)(x).
              \]
              \[
                  ((\lambda_{1} + \lambda_{2})f)(x) = (\lambda_{1} + \lambda_{2})f(x) = \lambda_{1}f(x) + \lambda_{2}f(x) = (\lambda_{1}f)(x) + (\lambda_{2}f)(x) = (\lambda_{1}f + \lambda_{2}f)(x).
              \]
    \end{itemize}
\end{proof}

% chapter 1/section B/exercise 8
\begin{exercise}
    Suppose $V$ is a real vector space.
    \begin{itemize}
        \item The \textit{complexification} of $V$, denoted by $V_{\mathbb{C}}$, equals $V\times V$. An element of $V_{\mathbb{C}}$ is an ordered pair $(u, v)$, where $u, v\in V$, but we write this as $u + \iota v$.
        \item Addition on $V_{\mathbb{C}}$ is defined by
              \[
                  (u_{1} + \iota v_{1}) + (u_{2} + \iota v_{2}) = (u_{1} + u_{2}) + \iota (v_{1} + v_{2})
              \]

              for all $u_{1}, v_{1}, u_{2}, v_{2}\in V$.
        \item Complex scalar multiplication on $V_{\mathbb{C}}$ is defined by
              \[
                  (a + b\iota) (u + \iota v) = (au - bv) + \iota(av + bu)
              \]

              for all $a, b\in\mathbb{R}$ and all $u, v\in V$.
    \end{itemize}

    Prove that with the definitions of addition and scalar multiplication as above, $V_{\mathbb{C}}$ is a complex vector space.
\end{exercise}

\begin{proof}
    \begin{itemize}
        \item Addition on $V_{\mathbb{C}}$ is commutative, since for all $(u, v), (u', v')\in V_{\mathbb{C}}$,
              \[
                  (u, v) + (u', v') = (u + u', v + v') = (u' + u, v' + v) = (u', v') + (u, v).
              \]
        \item Addition on $V_{\mathbb{C}}$ is associative, since for all $(u, v), (u', v'), (u'', v'')\in V_{\mathbb{C}}$,
              \begin{align*}
                  ((u, v) + (u', v')) + (u'', v'') & = (u + u', v + v') + (u'', v'')     \\
                                                   & = ((u + u') + u'', (v + v') + v'')  \\
                                                   & = (u + (u' + u''), v + (v' + v''))  \\
                                                   & = (u, v) + (u' + u'', v' + v'')     \\
                                                   & = (u, v) + ((u', v') + (u'', v'')).
              \end{align*}
        \item Addition on $V_{\mathbb{C}}$ has identity element, which is $(0, 0)$, since for all $(u, v)\in V_{\mathbb{C}}$,
              \[
                  (u, v) + (0, 0) = (u + 0, v + 0) = (u, v).
              \]
        \item Every element of $V_{\mathbb{C}}$ has an additive inverse. For every $(u, v)\in V_{\mathbb{C}}$,
              \[
                  (u, v) + (-u, -v) = (u + (-u), v + (-v)) = (0, 0).
              \]
        \item For all $(u, v)\in V_{\mathbb{C}}$
              \[
                  1(u, v) = (1 + 0\iota) (u, v) = (1u - 0v, 1v + 0u) = (u, v).
              \]
        \item For all $(u, v)\in V_{\mathbb{C}}$, all $z_{1} = a_{1} + b_{1}\iota, z_{2} = a_{2} + b_{2}\iota\in \mathbb{C}$
              \begin{align*}
                  (z_{1}z_{2})(u, v) & = (a_{1}a_{2} - b_{1}b_{2} + (a_{1}b_{2} + a_{2}b_{1})\iota)(u, v)                                                    \\
                                     & = ((a_{1}a_{2} - b_{1}b_{2})u - (a_{1}b_{2} + a_{2}b_{1})v, (a_{1}a_{2} - b_{1}b_{2})v + (a_{1}b_{2} + a_{2}b_{1})u), \\
                  z_{1}(z_{2}(u, v)) & = (a_{1} + b_{1}\iota)(a_{2}u - b_{2}v, a_{2}v + b_{2}u)                                                              \\
                                     & = (a_{1}(a_{2}u - b_{2}v) - b_{1}(a_{2}v + b_{2}u), a_{1}(a_{2}v + b_{2}u) + b_{1}(a_{2}u - b_{2}v))                  \\
                                     & = ((a_{1}a_{2} - b_{1}b_{2})u - (a_{1}b_{2} + a_{2}b_{1})v, (a_{1}a_{2} - b_{1}b_{2})v + (a_{1}b_{2} + a_{2}b_{1})u).
              \end{align*}

              Hence $(z_{1}z_{2})(u, v) = z_{1}(z_{2}(u, v))$.
        \item Addition and scalar multiplication on $V_{\mathbb{C}}$ are distributive. For all $(u, v), (u', v')\in V_{\mathbb{C}}$, all $z_{1} = a_{1} + b_{1}\iota, z_{2} = a_{2} + b_{2}\iota\in \mathbb{C}$
              \begin{align*}
                  (z_{1} + z_{2})(u, v) & = ((a_{1} + a_{2}) + (b_{1} + b_{2})\iota)(u, v)                             \\
                                        & = ((a_{1} + a_{2})u - (b_{1} + b_{2})v, (a_{1} + a_{2})v + (b_{1} + b_{2})u) \\
                                        & = (a_{1}u - b_{1}v, a_{1}v + b_{1}u) + (a_{2}u - b_{2}v, a_{2}v + b_{2}u)    \\
                                        & = z_{1}(u, v) + z_{2}(u, v).
              \end{align*}
              \begin{align*}
                  z_{1}((u, v) + (u', v')) & = (a_{1} + b_{1}\iota)(u + u', v + v')                                        \\
                                           & = (a_{1}(u + u') - b_{1}(v + v'), a_{1}(v + v') + b_{1}(u + u'))              \\
                                           & = (a_{1}u - b_{1}v, a_{1}v + b_{1}u) + (a_{1}u' - b_{1}v', a_{1}v' + b_{1}u') \\
                                           & = z_{1}(u, v) + z_{2}(u', v').
              \end{align*}
    \end{itemize}
\end{proof}

\section{Subspaces}

% chapter 1/section C/exercise 1
\begin{exercise}
    For each of the following subsets of $\mathbb{F}^{3}$, determine whether it is a subspace of $\mathbb{F}^{3}$.
    \begin{enumerate}[label={(\alph*)}]
        \item $\{ (x_{1}, x_{2}, x_{3})\in \mathbb{F}^{3}: x_{1} + 2x_{2} + 3x_{3} = 0 \}$
        \item $\{ (x_{1}, x_{2}, x_{3})\in \mathbb{F}^{3}: x_{1} + 2x_{2} + 3x_{3} = 4 \}$
        \item $\{ (x_{1}, x_{2}, x_{3})\in \mathbb{F}^{3}: x_{1}x_{2}x_{3} = 0 \}$
        \item $\{ (x_{1}, x_{2}, x_{3})\in \mathbb{F}^{3}: x_{1} = 5x_{3} \}$
    \end{enumerate}
\end{exercise}

\begin{proof}
    \begin{enumerate}[label={(\alph*)}]
        \item Yes. This is a subspace of $\mathbb{F}^{3}$.
        \item No. This is not a subspace of $\mathbb{F}^{3}$.

              Because $(4, 0, 0)$ is an element of the subset but $(8, 0, 0) = 2(4, 0, 0)$ is not.
        \item No. This is not a subspace of $\mathbb{F}^{3}$.

              Because $(1, 1, 0)$ and $(0, 0, 1)$ are elements of the subset, but $(1, 1, 1) = (1, 1, 0) + (0, 0, 1)$ is not.
        \item Yes. This is a subspace of $\mathbb{F}^{3}$.
    \end{enumerate}
\end{proof}

% chapter 1/section C/exercise 2
\begin{exercise}
    Verify all assertions about subspaces in Example 1.35.
    \begin{enumerate}[label={(\alph*)}]
        \item If $b\in\mathbb{F}$, then
              \[
                  \{ (x_{1}, x_{2}, x_{3}, x_{4})\in \mathbb{F}^{4}: x_{3} = 5x_{4} + b \}
              \]

              is a subspace of $\mathbb{F}^{4}$ if and only if $b = 0$.
        \item The set of continuous real-valued functions on the interval $[0, 1]$ is a subspace of $\mathbb{R}^{[0,1]}$.
        \item The set of differentiable real-valued functions on $\mathbb{R}$ is a subspace of $\mathbb{R}^{\mathbb{R}}$.
        \item The set of differentiable real-valued functions $f$ on the interval $(0, 3)$ such that $f'(2) = b$ is a subspace of $\mathbb{R}^{(0, 3)}$ if and only if $b = 0$.
        \item The set of all sequences of complex numbers with limit $0$ is a subspace of $\mathbb{C}^{\infty}$.
    \end{enumerate}
\end{exercise}

\begin{proof}
    \begin{enumerate}[label={(\alph*)}]
        \item If $b = 0$.

              $(0, 0, 0, 0)$ is in the subset. If $x$ and $y$ are in the subset, then so is $x + y$, because $x_{3} + y_{3} = 5x_{4} + 5y_{4} = 5(x_{4} + y_{4})$. If $x$ is in the subset, then so is $ax$ for every $a\in\mathbb{F}$, since $ax_{3} = 4ax_{4}$. So the subset is a subspace of $\mathbb{F}^{4}$.

              If the subset is a subspace of $\mathbb{F}^{4}$, then $(0, 0, 0, 0)$ is in the subset, then $0 = 5\cdot 0 + b$, so $b = 0$.
        \item The zero function $0: [0, 1]\to \mathbb{R}$ where $0(x) = 0$ for every $x\in [0, 1]$ is in the subset.

              If $f, g$ are continuous real-valued functions on $[0, 1]$ then so is $f + g$ and $\lambda\cdot f$ (for every real number $\lambda$).

              So the subset is a subspace of $\mathbb{R}^{[0,1]}$.
        \item The zero function $0: \mathbb{R}\to \mathbb{R}$ where $0(x) = 0$ for every $x\in \mathbb{R}$ is in the subset.

              If $f, g$ are differentiable real-valued functions on $\mathbb{R}$ then so is $f + g$ and $\lambda\cdot f$ (for every real number $\lambda$).

              So the subset is a subspace of $\mathbb{R}^{\mathbb{R}}$.
        \item If $b = 0$.

              The zero function $0: (0, 3)\to \mathbb{R}$ where $0(x) = 0$ for every $x\in (0, 3)$ is in the subset. If $f, g$ are differentiable real-valued functions on $(0, 3)$ such that $f'(2) = g'(2) = 0$ then so is $f + g$ and $\lambda\cdot f$ (for every real number $\lambda$) because $(f + g)'(2) = f'(2) + g'(2) = 0$ and $(\lambda\cdot f)'(2) = \lambda \cdot f'(2) = 0$. So the subset is a subspace of $\mathbb{R}^{(0, 3)}$.

              If the subset is a subspace of $\mathbb{R}^{(0, 3)}$ then the zero function $0: (0, 3)\to \mathbb{R}$ where $0(x) = 0$ for every $x\in (0, 3)$ is in the set. Therefore $b = 0'(2) = 0$.
        \item The constant sequence ${\left\{0\right\}}^{\infty}_{n=1}$ is in the subset.

              If two sequences of complex numbers ${\left\{a_{n}\right\}}^{\infty}_{n=1}$, ${\left\{b_{n}\right\}}^{\infty}_{n=1}$ have limit $0$ then so is  ${\left\{a_{n} + b_{n}\right\}}^{\infty}_{n=1}$ and  ${\left\{k a_{n}\right\}}^{\infty}_{n=1}$ (for every $k\in\mathbb{C}$).

              Hence the set of all sequences of complex numbers with limit $0$ is a subspace of $\mathbb{C}^{\infty}$.
    \end{enumerate}
\end{proof}

% chapter 1/section C/exercise 3
\begin{exercise}
    Show that the set of differentiable real-valued functions $f$ on the interval $(-4, 4)$ such that $f'(-1) = 3f(2)$ is a subspace of $\mathbb{R}^{(-4, 4)}$.
\end{exercise}

\begin{proof}
    The zero function $0: (-4, 4)\to \mathbb{R}$ where $0(x) = 0$ for every $x\in (-4, 4)$ is in the set.

    If $f, g$ are in the set, then $(f + g)'(-1) = f'(-1) + g'(-1) = 3f(2) + 3g(2) = 3 (f + g)(2)$, and $(\lambda\cdot f)'(-1) = \lambda\cdot f'(-1) = \lambda\cdot 3 f(2) = 3 (\lambda\cdot f)(2)$, which means $f + g$ and $\lambda\cdot f$ (for every $\lambda\in\mathbb{R}$) are also in the set.

    Hence the set of differentiable real-valued functions $f$ on the interval $(-4, 4)$ such that $f'(-1) = 3f(2)$ is a subspace of $\mathbb{R}^{(-4, 4)}$.
\end{proof}

% chapter 1/section C/exercise 4
\begin{exercise}
    Suppose $b\in\mathbb{R}$. Show that the set of continuous real-valued functions $f$ on the interval $[0, 1]$ such that $\int^{1}_{0} f = b$ is a subspace of $\mathbb{R}^{[0,1]}$ if and only if $b = 0$.
\end{exercise}

\begin{proof}
    If $b = 0$.

    The zero function $0: [0, 1]\to \mathbb{R}$ where $0(x) = 0$ for every $x\in [0, 1]$ is in the set. If $f, g$ are in the set, then
    \begin{align*}
        \int^{1}_{0}(f + g)(x)dx          & = \int^{1}_{0}(f(x) + g(x))dx = \int^{1}_{0}f(x)dx + \int^{1}_{0}g(x)dx = 0 + 0 = 0, \\
        \int^{1}_{0}(\lambda\cdot f)(x)dx & = \int^{1}_{0}\lambda\cdot f(x)dx = \lambda\int^{1}_{0}f(x)dx = 0.
    \end{align*}

    for every $\lambda \in \mathbb{R}$. So the set is a subspace of $\mathbb{R}^{[0, 1]}$.

    If the set is a subspace of $\mathbb{R}^{[0, 1]}$.
\end{proof}

% chapter 1/section C/exercise 5
\begin{exercise}
    Is $\mathbb{R}^{2}$ a subspace of the complex vector space $\mathbb{C}^{2}$?
\end{exercise}

\begin{proof}
    If the two vector spaces are over the field of real numbers, then yes.

    If the two vector spaces are over the field of complex numbers, then no, because $\mathbb{R}^{2}$ will not be closed under scalar multiplication.
\end{proof}

% chapter 1/section C/exercise 6
\begin{exercise}
    \begin{enumerate}[label={(\alph*)}]
        \item Is $\{ (a, b, c)\in\mathbb{R}^{3}: a^{3} = b^{3} \}$ a subspace of $\mathbb{R}^{3}$?
        \item Is $\{ (a, b, c)\in\mathbb{C}^{3}: a^{3} = b^{3} \}$ a subspace of $\mathbb{C}^{3}$?
    \end{enumerate}
\end{exercise}

\begin{proof}
    \begin{enumerate}[label={(\alph*)}]
        \item Yes. In $\mathbb{R}$, $a^{3} = b^{3}$ is equivalent to $a = b$. So the given subset is a subspace of $\mathbb{R}^{3}$.
        \item No. $(1, 1, 0)$ and $(e^{2\pi\iota/3}, 1, 0)$ are in the given subset, but $(1 + e^{2\pi\iota/3}, 2, 0)$ is not.
    \end{enumerate}
\end{proof}

% chapter 1/section C/exercise 7
\begin{exercise}
    Prove or give a counterexample: If $U$ is a nonempty subset of $\mathbb{R}^{2}$ such that $U$ is closed under addition and under taking additive inverses (meaning $-u\in U$ whenever $u\in U$), then $U$ is a subspace of $\mathbb{R}^{2}$.
\end{exercise}

\begin{proof}
    No. Here is a counterexample.
    \[
        U = \mathbb{Q}^{2}
    \]

    This nonempty subset of $\mathbb{R}^{2}$ is closed under addition and additive inversion, but is not closed under scalar multiplication. $(1, 0)\in U$, but $\sqrt{2}(1, 0)\notin U$.
\end{proof}

% chapter 1/section C/exercise 8
\begin{exercise}
    Give an example of a nonempty subset $U$ of $\mathbb{R}^{2}$ such that $U$ is closed under scalar multiplication, but $U$ is not a subspace of $\mathbb{R}^{2}$.
\end{exercise}

\begin{proof}
    \[
        U = \{ (x, 0): x\in\mathbb{R} \} \cup \{ (0, x): x\in\mathbb{R} \}
    \]

    is a nonempty subset of $\mathbb{R}^{2}$ but is not closed under addition. So this subset $U$ is not a subspace of $\mathbb{R}^{2}$.
\end{proof}

% chapter 1/section C/exercise 9
\begin{exercise}
    A function $f: \mathbb{R}\to\mathbb{R}$ is called periodic if there exists a positive number $p$ such that $f(x) = f(x + p)$ for all $x\in\mathbb{R}$. Is the set of periodic functions from $\mathbb{R}$ to $\mathbb{R}$ a subspace of $\mathbb{R}^{\mathbb{R}}$? Explain.
\end{exercise}

\begin{proof}
    No.

    The set is not closed under addition. Because
    \[
        f+g: x\mapsto \underbrace{x - \left\lfloor x\right\rfloor}_{f(x)} + \underbrace{\frac{1}{10}\sin(x)}_{g(x)}
    \]

    is not periodic.
\end{proof}

% chapter 1/section C/exercise 10
\begin{exercise}
    Suppose $V_{1}$ and $V_{2}$ are subspaces of $V$. Prove that the intersection $V_{1}\cap V_{2}$ is a subspace of $V$.
\end{exercise}

\begin{proof}
    Since $V_{1}$ and $V_{2}$ are subspaces of $V$, then the additive identity of $V$ is in $V_{1}$ and $V_{2}$, therefore the additive identity is also in $V_{1}\cap V_{2}$. So $V_{1}\cap V_{2}$ is nonempty. $V_{1}\cap V_{2}$ is closed under addition and scalar multiplication because $V_{1}$ and $V_{2}$ are closed under addition and scalar multiplication. Hence $V_{1}\cap V_{2}$ is a subspace of $V$.
\end{proof}

% chapter 1/section C/exercise 11
\begin{exercise}
    Prove that the intersection of every collection of subspaces of $V$ is a subspace of $V$.
\end{exercise}

\begin{proof}
    Let ${\{V_{i}\}}_{i\in I}$ where $I\ne\varnothing$ is a collection of subspaces of $V$.

    Since for all $i\in I$, the additive identity is in $V_{i}$, then the additive identity is in $\bigcap_{i\in I} V_{i}$.

    If $v, w$ are in $\bigcap_{i\in I}V_{i}$, then $v, w$ are in $V_{i}$ for all $i\in I$. So $v + w$ is in $V_{i}$ for all $i\in I$. So $v+w\in \bigcap_{i\in I}V_{i}$. Therefore $\bigcap_{i\in I}V_{i}$ is closed under addition.

    If $v$ is in $\bigcap_{i\in I}V_{i}$, then $v$ is in $V_{i}$ for all $i\in I$. So $\lambda\cdot v$ is in $V_{i}$ for all $i\in I$ and for all $\lambda\in\mathbb{F}$. So $\lambda\cdot v\in \bigcap_{i\in I}V_{i}$ for all $\lambda\in\mathbb{F}$. Therefore $\bigcap_{i\in I}V_{i}$ is closed under scalar multiplication.

    Thus $\bigcap_{i\in I} V_{i}$ is a subspace of $V$.
\end{proof}

% chapter 1/section C/exercise 12
\begin{exercise}
    Prove that the union of two subspaces of $V$ is a subspace of $V$ if and only if one of the subspaces is contained in the other.
\end{exercise}

\begin{proof}
    Let $V_{1}$ and $V_{2}$ be two subspaces of $V$.

    If $V_{1}\subseteq V_{2}$ or $V_{2}\subseteq V_{1}$ then $V_{1}\cup V_{2}$ is also a subspace of $V$.

    Suppose that $V_{1}$ and $V_{2}$ satisfy: $V_{1}\cup V_{2}$ is a subspace of $V$. Since $V_{1} + V_{2}$ is the smallest subspace containing $V_{1}, V_{2}$ and $V_{1}\cup V_{2}$ is the smallest set containing $V_{1}, V_{2}$. Together with $V_{1}\cup V_{2}$ being a subspace of $V$, we deduce that $V_{1}\cup V_{2} = V_{1} + V_{2}$.

    Assume that none of the two subspaces contains the other, then there exists $v_{1}\in V_{1}, v_{1}\notin V_{2}$ and $v_{2}\in V_{2}, v_{2}\notin V_{1}$. So $v_{1} + v_{2}$ is not in $V_{1}$ or $V_{2}$ (because $(-v_{1}) + v_{1} + v_{2}\notin V_{1}$ and $v_{1} + v_{2} + (-v_{2})\notin V_{2}$). Hence $v_{1} + v_{2}\notin V_{1}\cup V_{2}$, which contradicts $V_{1}\cup V_{2} = V_{1} + V_{2}$. Therefore $V_{1}\subseteq V_{2}$ or $V_{2}\subseteq V_{1}$.
\end{proof}

% chapter 1/section C/exercise 13
\begin{exercise}
    Prove that the union of three subspaces of $V$ is a subspace of $V$ if and only if one of the subspaces contains the other two. ($V$ is a vector space over a field with characteristic other than two.)
\end{exercise}

\begin{proof}
    Let $V_{1}, V_{2}, V_{3}$ be three subspaces of $V$.

    If one of these three subspaces contains the other two, then their union is also a subspace of $V$.

    Suppose that the three subspaces satisfy: $V_{1}\cup V_{2}\cap V_{3}$ is a subspace of $V$. Since $V_{1} + V_{2} + V_{3}$ is the smallest subspace containing $V_{1}, V_{2}, V_{3}$ and $V_{1}\cup V_{2}\cup V_{3}$ is the smallest set containing $V_{1}, V_{2}, V_{3}$. Together with $V_{1}\cup V_{2}\cup V_{3}$ being a subspace of $V$, we deduce that $V_{1}\cup V_{2}\cup V_{3} = V_{1} + V_{2} + V_{3}$.

    If $V_{1}$ does not contain $V_{2}$ and $V_{2}$ does not contain $V_{1}$, then there exist $v_{1.2}\in V_{1}$, $v_{1.2}\notin V_{2}$, $v_{2.1}\in V_{2}$, $v_{2,1}\notin V_{1}$.

    Let $a, b$ be a nonzero scalar such that $a - b = 1$, then $av_{1.2} + v_{2.1}\notin V_{1}\cup V_{2}$. On the other hand, $av_{1.2} + v_{2.1}\in V_{1} + V_{2}\subseteq V_{1} + V_{2} + V_{3} = V_{1}\cup V_{2}\cup V_{3}$. So $av_{1.2} + v_{2.1}\in V_{3}$. Similarly, $bv_{1.2} + v_{2.1}\in V_{3}$. Then $v_{1.2} = (a - b)v_{1.2} = av_{1.2} + v_{2.1} - (bv_{1.2} + v_{2.1})\in V_{3}$. Therefore $V_{1}\setminus V_{2}\subseteq V_{3}$. Analogously, $V_{2}\setminus V_{1}\subseteq V_{3}$.

    Let $u$ be an element of $V_{1}\cap V_{2}$, $w$ be an element of $V_{1}\setminus V_{2}\subseteq V_{3}$. Then $u + w\notin V_{2}$ and $u + w\in V_{1}$. So $u + w\in V_{1}\setminus V_{2}\subseteq V_{3}$. Therefore $(u + w) + (-w)\in V_{3}$, and $u\in V_{3}$ for every $u\in V_{1}\cap V_{2}$. So $V_{1}\cap V_{2}\subseteq V_{3}$.

    Hence $V_{1} = (V_{1}\setminus V_{2})\cup (V_{1}\cap V_{2})\subseteq V_{3}$ and $V_{2} = (V_{2}\setminus V_{1})\cup (V_{1}\cap V_{2})\subseteq V_{3}$.

    Otherwise, $V_{1}\subseteq V_{2}$ or $V_{2}\subseteq V_{1}$. By applying Exercise 1.C.12, we obtain the desire result.
\end{proof}

% chapter 1/section C/exercise 14
\begin{exercise}
    Suppose
    \[
        U = \{ (x, -x, 2x)\in \mathbb{F}^{3}: x\in\mathbb{F} \}\text{ and } W = \{ (x, x, 2x)\in \mathbb{F}^{3}: x\in\mathbb{F} \}
    \]

    Describe $U + W$ using symbols, and also give a description of $U + W$ that uses no symbols.
\end{exercise}

\begin{proof}
    \begin{align*}
        U + W & = \{ (x, -x, 2x) + (y, y, 2y)\in \mathbb{F}^{3}: x, y\in\mathbb{F} \} \\
              & = \{ (x+y, -x+y, 2(x+y))\in \mathbb{F}^{3}: x, y\in\mathbb{F} \}      \\
              & = \{ (x, y, 2x)\in\mathbb{F}^{3}: x, y\in\mathbb{F} \}
    \end{align*}

    $U + W$ is a subspace of $\mathbb{F}^{3}$ where for every $v\in U + W$, the 3rd component of $v$ is twice the 1st component of $v$.
\end{proof}

% chapter 1/section C/exercise 15
\begin{exercise}
    Suppose $U$ is a subspace of $V$. What is $U + U$?
\end{exercise}

\begin{proof}
    $U + U$ is $U$.
\end{proof}

% chapter 1/section C/exercise 16
\begin{exercise}
    Is the operation of addition on the subspaces of $V$ commutative? In other words, if $U$ and $W$ are subspaces of $V$, is $U + W = W + U$?
\end{exercise}

\begin{proof}
    Yes. Because addition in a vector space is commutative, then every element of $U + W$ is an element of $W + U$ and vice versa.
\end{proof}

% chapter 1/section C/exercise 17
\begin{exercise}
    Is the operation of addition on the subspaces of $V$ associative? In other words, if $V_{1}$, $V_{2}$, $V_{3}$ are subspaces of $V$, is
    \[
        (V_{1} + V_{2}) + V_{3} = V_{1} + (V_{2} + V_{3})?
    \]
\end{exercise}

\begin{proof}
    Yes. Because addition in a vector space is associative, then every element of $(V_{1} + V_{2}) + V_{3}$ is an element of $V_{1} + (V_{2} + V_{3})$ and vice versa.
\end{proof}

% chapter 1/section C/exercise 18
\begin{exercise}
    Does the operation of addition on the subspaces of $V$ have an additive identity? Which subspaces have additive inverses?
\end{exercise}

\begin{proof}
    Yes, the operation of addition on the subspaces of $V$ have an additive identity. Such an additive identity is the zero subspace, which contains the zero vector only.

    Only the zero subspace has additive inverse, because the sum of a nonzero subspace $W$ with any subspace is never the zero subspace (since the sum contains $W$).
\end{proof}

% chapter 1/section C/exercise 19
\begin{exercise}
    Prove or give a counterexample: If $V_{1}, V_{2}, U$ are subspaces of $V$ such that
    \[
        V_{1} + U = V_{2} + U,
    \]

    then $V_{1} = V_{2}$.
\end{exercise}

\begin{proof}
    I give a counterexample.

    Let $V = \mathbb{F}^{2}$, $U = \{ (x, 0)\in \mathbb{F}^{2}: x\in\mathbb{F} \}$, $V_{1} = \{ (0, x)\in \mathbb{F}^{2}: x\in\mathbb{F} \}$, $V_{2} = \{ (x, y)\in \mathbb{F}^{2}: x\in\mathbb{F} \}$.

    $V_{1} + U = V$ and $V_{2} + U = V$, but $V_{1}\ne V_{2}$.
\end{proof}

% chapter 1/section C/exercise 20
\begin{exercise}
    Suppose
    \[
        U = \{ (x, x, y, y)\in\mathbb{F}^{4}: x, y\in\mathbb{F} \}.
    \]

    Find a subspace $W$ of $\mathbb{F}^{4}$ such that $\mathbb{F}^{4} = U\oplus W$.
\end{exercise}

\begin{proof}
    \[
        (x_{1}, x_{2}, x_{3}, x_{4}) = (x_{2}, x_{2}, x_{4}, x_{4}) + (x_{1} - x_{2}, 0, x_{3} - x_{4}, 0)
    \]

    Let $W = \{ (x, 0, y, 0)\in\mathbb{F}^{4}: x, y\in\mathbb{F} \}$ then $\mathbb{F}^{4} = U\oplus W$.
\end{proof}

% chapter 1/section C/exercise 21
\begin{exercise}
    Suppose
    \[
        U = \{ (x, y, x+y, x-y, 2x)\in\mathbb{F}^{5}: x, y\in\mathbb{F} \}.
    \]

    Find a subspace $W$ of $\mathbb{F}^{5}$ such that $\mathbb{F}^{5} = U\oplus W$.
\end{exercise}

\begin{proof}
    \[
        (x_{1}, x_{2}, x_{3}, x_{4}, x_{5}) = (x_{1}, x_{2}, x_{1} + x_{2}, x_{1} - x_{2}, 2x_{1}) + (0, 0, x_{3} - x_{1} - x_{2}, x_{4} - x_{1} + x_{2}, x_{5} - 2x_{1})
    \]

    Let $W = \{ (0, 0, x, y, z)\in\mathbb{F}^{5}: x, y, z\in\mathbb{F} \}$, then $U + W = \mathbb{F}^{5}$ and $U\cap W = \{ (0, 0, 0, 0, 0) \}$. Hence $\mathbb{F}^{5} = U\oplus W$.
\end{proof}

% chapter 1/section C/exercise 22
\begin{exercise}
    Suppose
    \[
        U = \{ (x, y, x+y, x-y, 2x)\in\mathbb{F}^{5}: x, y\in\mathbb{F} \}.
    \]

    Find three subspaces $W_{1}, W_{2}, W_{3}$ of $\mathbb{F}^{5}$, none of which equals $\{0\}$, such that $\mathbb{F}^{5} = U\oplus W_{1}\oplus W_{2}\oplus W_{3}$.
\end{exercise}

\begin{proof}
    Let $W_{1} = \{ (0, 0, x, 0, 0)\in\mathbb{F}^{5}: x\in\mathbb{F} \}$, $W_{2} = \{ (0, 0, 0, x, 0)\in\mathbb{F}^{5}: x\in\mathbb{F} \}$, $W_{3} = \{ (0, 0, 0, x, 0)\in\mathbb{F}^{5}: x\in\mathbb{F} \}$, then $\mathbb{F}^{5} = U\oplus W_{1}\oplus W_{2}\oplus W_{3}$.
\end{proof}

% chapter 1/section C/exercise 23
\begin{exercise}
    Prove or give a counterexample: If $V_{1}$, $V_{2}$, $U$ are subspaces of $V$ such that
    \[
        V = V_{1}\oplus U\text{ and } V = V_{2}\oplus U,
    \]

    then $V_{1} = V_{2}$.
\end{exercise}

\begin{proof}
    I give a counterexample.

    Let $V = \mathbb{F}^{2}$, $U = \{ (x, 0)\in \mathbb{F}^{2}: x\in\mathbb{F} \}$, $V_{1} = \{ (0, x)\in \mathbb{F}^{2}: x\in\mathbb{F} \}$, $V_{2} = \{ (x, x)\in \mathbb{F}^{2}: x\in\mathbb{F} \}$.

    $V_{1} \oplus U = V$ and $V_{2} \oplus U = V$, but $V_{1}\ne V_{2}$.
\end{proof}

% chapter 1/section C/exercise 24
\begin{exercise}
    A function $f: \mathbb{R}\to \mathbb{R}$ is called \textit{even} if
    \[
        f(-x) = f(x)
    \]

    for all $x\in\mathbb{R}$. A function $f: \mathbb{R}\to \mathbb{R}$ is called \textit{odd} if
    \[
        f(-x) = -f(x)
    \]

    for all $x\in\mathbb{R}$. Let $V_{e}$ denote the set of real-valued even functions on $\mathbb{R}$ and let $V_{o}$ denote the set of real-valued odd functions on $\mathbb{R}$. Show that $\mathbb{R}^{\mathbb{R}} = V_{e}\oplus V_{o}$.
\end{exercise}

\begin{proof}
    Let $f$ be a real-valued functions on $\mathbb{R}$. For all $x\in\mathbb{R}$,
    \[
        f(x) = \frac{1}{2}(f(x) + f(-x)) + \frac{1}{2}(f(x) - f(-x))
    \]

    Let $f_{e}(x) = \frac{1}{2}(f(x) + f(-x))$ and $f_{o}(x) = \frac{1}{2}(f(x) - f(-x))$. Since
    \[
        f_{e}(-x) = \frac{1}{2}(f(-x) + f(x)) = f_{e}(x)\qquad f_{o}(-x) = \frac{1}{2}(f(-x) - f(x)) = -f_{o}(x)
    \]

    then $f_{e}\in V_{e}$ and $f_{o}\in V_{o}$. So $\mathbb{R}^{\mathbb{R}} = V_{e} + V_{o}$.

    Let $g$ be an even and odd function, then $g(-x) = g(x) = -g(x)$ for all $x\in\mathbb{R}$, which implies that $g(x) = 0$ for all $x\in\mathbb{R}$. Equivalently, $g$ is the zero function. Therefore $V_{e}\cap V_{o} = \{ 0 \}$.

    Thus $\mathbb{R}^{\mathbb{R}} = V_{e}\oplus V_{o}$.
\end{proof}
