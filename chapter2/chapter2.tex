\documentclass[class=linearalgebra,crop=false]{standalone}

\begin{document}

\setcounter{exercise}{0}

\chapter{Ma trận và ánh xạ tuyến tính}

\begin{exercise}
    Tính tích của hai ma trận sau đây:
    \[
        \begin{pmatrix}
            0 & 4 & 7 & 1 \\
            2 & 1 & 7 & 6 \\
            1 & 0 & 8 & 3 \\
            0 & 1 & 9 & 6
        \end{pmatrix}
        \begin{pmatrix}
            -2 & 8 & -5 & 4 \\
            7  & 8 &  5 & 5 \\
            0  & 3 &  8 & 4 \\
            -8 & 9 & -8 & 9
        \end{pmatrix}
    \]
\end{exercise}

\begin{proof}[Lời giải]
    \[
        \begin{pmatrix}
            20  & 62 & -68 & 57 \\
            -45 & 99 & 3   & 95 \\
            -26 & 59 & 35  & 63 \\
            -41 & 89 & 29  & 95 
        \end{pmatrix}
    \]
\end{proof}

\par Tính các lũy thừa sau đây

\begin{exercise}
    \[
        \begin{pmatrix}
            \cos\phi & -\sin\phi \\
            \sin\phi &  \cos\phi
        \end{pmatrix}^{n},\quad
        \begin{pmatrix}
            \lambda & 1 \\
                  0 & \lambda
        \end{pmatrix}^{n}.
    \]
\end{exercise}

\begin{proof}[Lời giải]
    \begin{enumerate}
        \item 
            \[
                A = \begin{pmatrix}
                    \cos\phi & -\sin\phi \\
                    \sin\phi & \cos\phi
                \end{pmatrix}
            \]
            \par Ta sẽ chứng minh bằng quy nạp rằng:
            \[
                A^{n} =
                \begin{pmatrix}
                    \cos(n\phi) & -\sin(n\phi) \\
                    \sin(n\phi) & \cos(n\phi)
                \end{pmatrix}
                \tag{*}
            \]
            \par (*) đúng với $n = 1$, giả sử (*) cũng đúng với $n = k$.
            \begin{align*}
                A^{k+1} & =
                \begin{pmatrix}
                    \cos(k\phi) & -\sin(k\phi) \\
                    \sin(k\phi) & \cos(k\phi)
                \end{pmatrix}
                \begin{pmatrix}
                    \cos\phi & -\sin\phi \\
                    \sin\phi & \cos\phi
                \end{pmatrix} \\
                        & =
                \begin{pmatrix}
                    \cos(k\phi)\cos\phi - \sin(k\phi)\sin\phi & -\sin(k\phi)\cos\phi - \cos(k\phi)\sin\phi \\
                    \sin(k\phi)\cos\phi + \cos(k\phi)\sin\phi & \cos(k\phi)\cos\phi - \sin(k\phi)\sin\phi
                \end{pmatrix} \\
                        & = 
                \begin{pmatrix}
                    \cos((k+1)\phi) & -\sin((k+1)\phi) \\
                    \sin((k+1)\phi) & \cos((k+1)\phi)
                \end{pmatrix}
            \end{align*}
            \par Vậy (*) vẫn đúng với $n = k+1$. Theo nguyên lý quy nạp, (*) đúng với mọi $n\in\mathbb{N}$.
        \item
            \[
                B =
                \begin{pmatrix}
                    \lambda & 1 \\
                    0       & \lambda
                \end{pmatrix}
            \]
            \par Ta sẽ chứng minh bằng quy nạp rằng:
            \[
                B^{n} =
                \begin{pmatrix}
                    \lambda^{n} & n\lambda^{n-1} \\
                    0           & \lambda^{n}
                \end{pmatrix}
                \tag{**}
            \]
            \par (**) đúng với $n=1$, giả sử (**) cũng đúng với $n = k$.
            \begin{align*}
                A^{k+1} & =
                \begin{pmatrix}
                    \lambda^{k} & k\lambda^{k-1} \\
                    0           & \lambda^{k}
                \end{pmatrix}
                \begin{pmatrix}
                    \lambda & 1       \\
                    0       & \lambda
                \end{pmatrix}
                =
                \begin{pmatrix}
                    \lambda^{k+1} & (k+1)\lambda^{k} \\
                    0             & \lambda^{k+1}
                \end{pmatrix}
            \end{align*}
    \end{enumerate}
\end{proof}

\begin{exercise}
    \[
        \begin{pmatrix}
             a_{1} & 0      & \cdots & 0      \\
                 0 & a_{2}  & \cdots & 0      \\
            \vdots & \vdots & \ddots & \vdots \\
                 0 & 0      & \cdots & a_{n}
        \end{pmatrix}^{k}
    \]
\end{exercise}

\begin{proof}[Lời giải]
    Từ định nghĩa tích ma trận, ta có đẳng thức sau:
    \[
        \begin{pmatrix}
            x_{1}  & 0      & \cdots & 0      \\
            0      & x_{2}  & \cdots & 0      \\
            \vdots & \vdots & \ddots & \vdots \\
            0      & 0      & \cdots & x_{n}
        \end{pmatrix}
        \begin{pmatrix}
            y_{1}  & 0      & \cdots & 0      \\
            0      & y_{2}  & \cdots & 0      \\
            \vdots & \vdots & \ddots & \vdots \\
            0      & 0      & \cdots & y_{n}
        \end{pmatrix}
        =
        \begin{pmatrix}
            x_{1}y_{1}  & 0      & \cdots & 0      \\
            0      & x_{2}y_{2}  & \cdots & 0      \\
            \vdots & \vdots & \ddots & \vdots \\
            0      & 0      & \cdots & x_{n}y_{n}
        \end{pmatrix}
    \]
    \par Áp dụng vào bài toán, ta được:
    \[
        \begin{pmatrix}
            a_{1}  & 0      & \cdots & 0      \\
            0      & a_{2}  & \cdots & 0      \\
            \vdots & \vdots & \ddots & \vdots \\
            0      & 0      & \cdots & a_{n}
        \end{pmatrix}^{k}
        =
        \begin{pmatrix}
            a^{k}_{1}  & 0      & \cdots & 0      \\
            0      & a^{k}_{2}  & \cdots & 0      \\
            \vdots & \vdots & \ddots & \vdots \\
            0      & 0      & \cdots & a^{k}_{n}
        \end{pmatrix}
    \]
\end{proof}

\begin{exercise}
    \[
        \begin{pmatrix}
            1 & 1 & 0 & 0 & \cdots & 0 & 0 \\
            0 & 1 & 1 & 0 & \cdots & 0 & 0 \\
            0 & 0 & 1 & 1 & \cdots & 0 & 0 \\
            \vdots & \vdots & \vdots & \vdots & \ddots & \vdots & \vdots \\
            0 & 0 & 0 & 0 & \cdots & 1 & 1 \\
            0 & 0 & 0 & 0 & \cdots & 0 & 1
        \end{pmatrix}^{n-1}
    \]
\end{exercise}

\begin{proof}[Lời giải]
\end{proof}

\begin{exercise}
    Cho hai ma trận $A$ và $B$ với các yếu tố trong $\mathbb{F}$. Chứng minh rằng nếu các tích $AB$ và $BA$ đều có nghĩa và $AB = BA$, thì $A$ và $B$ là các ma trận vuông cùng cỡ.
\end{exercise}

\begin{proof}
    Tích $AB$ và $BA$ có nghĩa, tức là số cột của $A$ bằng số hàng của $B$ và số cột của $B$ bằng số hàng của $A$.
    \par Vậy $A\in M(m\times n,\mathbb{F})$ thì $B\in M(n\times m, \mathbb{F})$.
    \par Suy ra $AB\in M(m\times m,\mathbb{F})$, $BA\in M(n\times n,\mathbb{F})$.
    \par Mà $AB = BA$ nên $m = n$.
    \par Do đó $A$ và $B$ là hai ma trận vuông cùng cỡ.
\end{proof}

\begin{exercise}
    Ma trận tích $AB$ sẽ thay đổi thế nào nếu ta
    \begin{enumerate}[label = (\alph*)]
        \item đổi chỗ các hàng thứ $i$ và thứ $j$ của ma trận $A$?
        \item cộng vào hàng thứ $i$ của $A$ tích của vô hướng $c$ với hàng thứ $j$ của $A$?
        \item đổi chỗ các cột thứ $i$ và thứ $j$ của ma trận $B$.
        \item cộng vào cột thứ $i$ của $B$ tích của vô hướng $c$ với cột thứ $j$ của $B$?
    \end{enumerate}
\end{exercise}

\begin{proof}[Lời giải]
    \begin{enumerate}[label = (\alph*)]
        \item Hàng thứ $i$ và thứ $j$ của ma trận tích đổi chỗ.
        \item Hàng thứ $i$ của $AB$ được cộng thêm tích của vô hướng $c$ với hàng thứ $j$ của $AB$.
        \item Cột thứ $i$ và thứ $j$ của ma trận tích đổi chỗ.
        \item Cột thứ $i$ của $AB$ được cộng thêm tích của vô hướng $c$ với cột thứ $j$ của $AB$.
    \end{enumerate}
\end{proof}

\begin{exercise}
    \textit{Vết} của một ma trận vuông là tổng của tất cả các yếu tố nằm trên đường chéo chính của ma trận đó. Chứng minh rằng vết của $AB$ bằng vết của $BA$.
\end{exercise}

\begin{proof}
\end{proof}

\begin{exercise}
    Chứng minh rằng nếu $A$ và $B$ là các ma trận vuông cùng cấp, với $AB\ne BA$, thì
    \begin{enumerate}[label = (\alph*)]
        \item $(A+B){}^{2}\ne A^{2} + 2AB + B^{2}$,
        \item $(A+B)(A-B)\ne A^{2} - B^{2}$.
    \end{enumerate}
\end{exercise}

\begin{proof}
    \begin{enumerate}[label = (\alph*)]
        \item
             \begin{align*}
                 (A+B){}^{2} & = (A+B)(A+B) \\
                             & = A^{2} + AB + BA + B^{2} \\
                             & \ne A^{2} + AB + AB + B^{2} \\
                             & = A^{2} + 2AB + B^{2}
             \end{align*}
        \item
             \begin{align*}
                 (A+B)(A-B) & = A^{2} - AB + BA - B^{2} \\
                            & = A^{2} - B^{2} + (BA - AB) \\
                            & \ne A^{2} - B^{2}
             \end{align*}
    \end{enumerate}
\end{proof}

\begin{exercise}
    Chứng minh rằng nếu $A$ và $B$ là các ma trận vuông với $AB = BA$ thì
    \[
        (A+B){}^{n} = A^{n} + nA^{n-1}B + \frac{n(n-1)}{2}A^{n-2}B^{2} + \cdots + B^{n}.
    \]
\end{exercise}

\begin{proof}
\end{proof}

\end{document}
