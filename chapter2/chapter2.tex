\documentclass[class=linearalgebra,crop=false]{standalone}

\begin{document}

\setcounter{exercise}{0}

\chapter{Ma trận và ánh xạ tuyến tính}

\begin{exercise}
    Tính tích của hai ma trận sau đây:
    \[
        \begin{pmatrix}
            0 & 4 & 7 & 1 \\
            2 & 1 & 7 & 6 \\
            1 & 0 & 8 & 3 \\
            0 & 1 & 9 & 6
        \end{pmatrix}
        \begin{pmatrix}
            -2 & 8 & -5 & 4 \\
            7  & 8 & 5  & 5 \\
            0  & 3 & 8  & 4 \\
            -8 & 9 & -8 & 9
        \end{pmatrix}
    \]
\end{exercise}

\begin{proof}[Lời giải]
    \[
        \begin{pmatrix}
            20  & 62 & -68 & 57 \\
            -45 & 99 & 3   & 95 \\
            -26 & 59 & 35  & 63 \\
            -41 & 89 & 29  & 95
        \end{pmatrix}
    \]
\end{proof}

\par Tính các lũy thừa sau đây

\begin{exercise}
    \[
        \begin{pmatrix}
            \cos\phi & -\sin\phi \\
            \sin\phi & \cos\phi
        \end{pmatrix}^{n},\quad
        \begin{pmatrix}
            \lambda & 1       \\
            0       & \lambda
        \end{pmatrix}^{n}.
    \]
\end{exercise}

\begin{proof}[Lời giải]
    \begin{enumerate}
        \item
              \[
                  A = \begin{pmatrix}
                      \cos\phi & -\sin\phi \\
                      \sin\phi & \cos\phi
                  \end{pmatrix}
              \]
              \par Ta sẽ chứng minh bằng quy nạp rằng:
              \[
                  A^{n} =
                  \begin{pmatrix}
                      \cos(n\phi) & -\sin(n\phi) \\
                      \sin(n\phi) & \cos(n\phi)
                  \end{pmatrix}
                  \tag{*}
              \]
              \par (*) đúng với $n = 1$, giả sử (*) cũng đúng với $n = k$.
              \begin{align*}
                  A^{k+1} & =
                  \begin{pmatrix}
                      \cos(k\phi) & -\sin(k\phi) \\
                      \sin(k\phi) & \cos(k\phi)
                  \end{pmatrix}
                  \begin{pmatrix}
                      \cos\phi & -\sin\phi \\
                      \sin\phi & \cos\phi
                  \end{pmatrix}  \\
                          & =
                  \begin{pmatrix}
                      \cos(k\phi)\cos\phi - \sin(k\phi)\sin\phi & -\sin(k\phi)\cos\phi - \cos(k\phi)\sin\phi \\
                      \sin(k\phi)\cos\phi + \cos(k\phi)\sin\phi & \cos(k\phi)\cos\phi - \sin(k\phi)\sin\phi
                  \end{pmatrix} \\
                          & =
                  \begin{pmatrix}
                      \cos((k+1)\phi) & -\sin((k+1)\phi) \\
                      \sin((k+1)\phi) & \cos((k+1)\phi)
                  \end{pmatrix}
              \end{align*}
              \par Vậy (*) vẫn đúng với $n = k+1$. Theo nguyên lý quy nạp, (*) đúng với mọi $n\in\mathbb{N}$.
        \item
              \[
                  B =
                  \begin{pmatrix}
                      \lambda & 1       \\
                      0       & \lambda
                  \end{pmatrix}
              \]
              \par Ta sẽ chứng minh bằng quy nạp rằng:
              \[
                  B^{n} =
                  \begin{pmatrix}
                      \lambda^{n} & n\lambda^{n-1} \\
                      0           & \lambda^{n}
                  \end{pmatrix}
                  \tag{**}
              \]
              \par (**) đúng với $n=1$, giả sử (**) cũng đúng với $n = k$.
              \begin{align*}
                  A^{k+1} & =
                  \begin{pmatrix}
                      \lambda^{k} & k\lambda^{k-1} \\
                      0           & \lambda^{k}
                  \end{pmatrix}
                  \begin{pmatrix}
                      \lambda & 1       \\
                      0       & \lambda
                  \end{pmatrix}
                  =
                  \begin{pmatrix}
                      \lambda^{k+1} & (k+1)\lambda^{k} \\
                      0             & \lambda^{k+1}
                  \end{pmatrix}
              \end{align*}
              \par Vậy (**) vẫn đúng với $n = k+1$. Theo nguyên lý quy nạp, (**) đúng với mọi $n\in\mathbb{N}$.
    \end{enumerate}
\end{proof}

\begin{exercise}
    \[
        \begin{pmatrix}
            a_{1}  & 0      & \cdots & 0      \\
            0      & a_{2}  & \cdots & 0      \\
            \vdots & \vdots & \ddots & \vdots \\
            0      & 0      & \cdots & a_{n}
        \end{pmatrix}^{k}
    \]
\end{exercise}

\begin{proof}[Lời giải]
    Từ định nghĩa tích ma trận, ta có đẳng thức sau:
    \[
        \begin{pmatrix}
            x_{1}  & 0      & \cdots & 0      \\
            0      & x_{2}  & \cdots & 0      \\
            \vdots & \vdots & \ddots & \vdots \\
            0      & 0      & \cdots & x_{n}
        \end{pmatrix}
        \begin{pmatrix}
            y_{1}  & 0      & \cdots & 0      \\
            0      & y_{2}  & \cdots & 0      \\
            \vdots & \vdots & \ddots & \vdots \\
            0      & 0      & \cdots & y_{n}
        \end{pmatrix}
        =
        \begin{pmatrix}
            x_{1}y_{1} & 0          & \cdots & 0          \\
            0          & x_{2}y_{2} & \cdots & 0          \\
            \vdots     & \vdots     & \ddots & \vdots     \\
            0          & 0          & \cdots & x_{n}y_{n}
        \end{pmatrix}
    \]
    \par Áp dụng vào bài toán, ta được:
    \[
        \begin{pmatrix}
            a_{1}  & 0      & \cdots & 0      \\
            0      & a_{2}  & \cdots & 0      \\
            \vdots & \vdots & \ddots & \vdots \\
            0      & 0      & \cdots & a_{n}
        \end{pmatrix}^{k}
        =
        \begin{pmatrix}
            a^{k}_{1} & 0         & \cdots & 0         \\
            0         & a^{k}_{2} & \cdots & 0         \\
            \vdots    & \vdots    & \ddots & \vdots    \\
            0         & 0         & \cdots & a^{k}_{n}
        \end{pmatrix}
    \]
\end{proof}

\begin{exercise}
    Ma trận sau đây có $n$ hàng, $n$ cột.
    \[
        \begin{pmatrix}
            1      & 1      & 0      & 0      & \cdots & 0      & 0      \\
            0      & 1      & 1      & 0      & \cdots & 0      & 0      \\
            0      & 0      & 1      & 1      & \cdots & 0      & 0      \\
            \vdots & \vdots & \vdots & \vdots & \ddots & \vdots & \vdots \\
            0      & 0      & 0      & 0      & \cdots & 1      & 1      \\
            0      & 0      & 0      & 0      & \cdots & 0      & 1
        \end{pmatrix}^{n-1}
    \]
\end{exercise}

\begin{proof}[Lời giải]
    \[
        A =
        \begin{pmatrix}
            1      & 1      & 0      & 0      & \cdots & 0               \\
            0      & 1      & 1      & 0      & \cdots & 0      & 0      \\
            0      & 0      & 1      & 1      & \cdots & 0      & 0      \\
            \vdots & \vdots & \vdots & \vdots & \ddots & \vdots & \vdots \\
            0      & 0      & 0      & 0      & \cdots & 1      & 1      \\
            0      & 0      & 0      & 0      & \cdots & 0      & 1
        \end{pmatrix}
    \]
    \par Ma trận $A$ được viết lại như sau:
    \[
        A =
        \begin{pmatrix}
            \binom{1}{0} & \binom{1}{1} & 0            & 0            & \cdots & 0            & 0            \\
            0            & \binom{1}{0} & \binom{1}{1} & 0            & \cdots & 0            & 0            \\
            0            & 0            & \binom{1}{0} & \binom{1}{1} & \cdots & 0            & 0            \\
            \vdots       & \vdots       & \vdots       & \vdots       & \ddots & \vdots       & \vdots       \\
            0            & 0            & 0            & 0            & \ddots & \binom{1}{0} & \binom{1}{1} \\
            0            & 0            & 0            & 0            & \cdots & 0            & \binom{1}{0}
        \end{pmatrix}
    \]
    \par Sử dụng quy tắc nhân ma trận và đẳng thức Pascal, ta được:
    \[
        A^{2} =
        \begin{pmatrix}
            \binom{2}{0} & \binom{2}{1} & \binom{2}{2} & 0            & \cdots & 0            & 0            \\
            0            & \binom{2}{0} & \binom{2}{1} & \binom{2}{2} & \cdots & 0            & 0            \\
            0            & 0            & \binom{2}{0} & \binom{2}{1} & \cdots & 0            & 0            \\
            \vdots       & \vdots       & \vdots       & \vdots       & \ddots & \vdots       & \vdots       \\
            0            & 0            & 0            & 0            & \cdots & \binom{2}{0} & \binom{2}{1} \\
            0            & 0            & 0            & 0            & \cdots & 0            & \binom{2}{0}
        \end{pmatrix}
    \]
    \par Nhân liên tiếp với $A$, đến cuối cùng ta được:
    \[
        A^{n-1} =
        \begin{pmatrix}
            \binom{n-1}{0} & \binom{n-1}{1} & \binom{n-1}{2} & \binom{n-1}{3} & \cdots & \binom{n-1}{n-2} & \binom{n-1}{n-1} \\
            0              & \binom{n-1}{0} & \binom{n-1}{1} & \binom{n-1}{2} & \cdots & \binom{n-1}{n-3} & \binom{n-1}{n-2} \\
            0              & 0              & \binom{n-1}{0} & \binom{n-1}{1} & \cdots & \binom{n-1}{n-4} & \binom{n-1}{n-3} \\
            \vdots         & \vdots         & \vdots         & \vdots         & \ddots & \vdots           & \vdots           \\
            0              & 0              & 0              & 0              & \cdots & \binom{n-1}{0}   & \binom{n-1}{1}   \\
            0              & 0              & 0              & 0              & \cdots & 0                & \binom{n-1}{0}
        \end{pmatrix}
    \]
\end{proof}

\begin{exercise}
    Cho hai ma trận $A$ và $B$ với các yếu tố trong $\mathbb{F}$. Chứng minh rằng nếu các tích $AB$ và $BA$ đều có nghĩa và $AB = BA$, thì $A$ và $B$ là các ma trận vuông cùng cỡ.
\end{exercise}

\begin{proof}
    Tích $AB$ và $BA$ có nghĩa, tức là số cột của $A$ bằng số hàng của $B$ và số cột của $B$ bằng số hàng của $A$.
    \par Vậy $A\in M(m\times n,\mathbb{F})$ thì $B\in M(n\times m, \mathbb{F})$.
    \par Suy ra $AB\in M(m\times m,\mathbb{F})$, $BA\in M(n\times n,\mathbb{F})$.
    \par Mà $AB = BA$ nên $m = n$.
    \par Do đó $A$ và $B$ là hai ma trận vuông cùng cỡ.
\end{proof}

\begin{exercise}
    Ma trận tích $AB$ sẽ thay đổi thế nào nếu ta
    \begin{enumerate}[label = (\alph*)]
        \item đổi chỗ các hàng thứ $i$ và thứ $j$ của ma trận $A$?
        \item cộng vào hàng thứ $i$ của $A$ tích của vô hướng $c$ với hàng thứ $j$ của $A$?
        \item đổi chỗ các cột thứ $i$ và thứ $j$ của ma trận $B$.
        \item cộng vào cột thứ $i$ của $B$ tích của vô hướng $c$ với cột thứ $j$ của $B$?
    \end{enumerate}
\end{exercise}

\begin{proof}[Lời giải]
    \begin{enumerate}[label = (\alph*)]
        \item Hàng thứ $i$ và thứ $j$ của ma trận tích đổi chỗ.
        \item Hàng thứ $i$ của $AB$ được cộng thêm tích của vô hướng $c$ với hàng thứ $j$ của $AB$.
        \item Cột thứ $i$ và thứ $j$ của ma trận tích đổi chỗ.
        \item Cột thứ $i$ của $AB$ được cộng thêm tích của vô hướng $c$ với cột thứ $j$ của $AB$.
    \end{enumerate}
\end{proof}

\begin{exercise}\label{trace-of-products}
    \textit{Vết} của một ma trận vuông là tổng của tất cả các yếu tố nằm trên đường chéo chính của ma trận đó. Chứng minh rằng vết của $AB$ bằng vết của $BA$.
\end{exercise}

\begin{proof}
    Tích $AB$ và $BA$ có nghĩa, vậy là số cột của $A$ bằng số hàng của $B$, số cột của $B$ bằng số hàng của $A$.
    \par Giả sử $A\in M(m\times n, \mathbb{F})$. Khi đó, $B\in M(n\times m, \mathbb{F})$. Ta đặt
    \[
        A =
        \begin{pmatrix}
            a_{11} & a_{12} & \cdots & a_{1n} \\
            a_{21} & a_{22} & \cdots & a_{2n} \\
            \vdots & \vdots & \ddots & \vdots \\
            a_{m1} & a_{m2} & \cdots & a_{mn}
        \end{pmatrix}
        \quad
        B =
        \begin{pmatrix}
            b_{11} & b_{12} & \cdots & b_{1m} \\
            b_{21} & b_{22} & \cdots & b_{2m} \\
            \vdots & \vdots & \ddots & \vdots \\
            b_{n1} & b_{n2} & \cdots & b_{nm}
        \end{pmatrix}
    \]
    \par Kí hiệu $(AB){}_{ij}$ là yếu tố hàng $i$, cột $j$ của $AB$. Kí hiệu $Tr(X)$ là vết của ma trận $X$.
    \[
        Tr(AB) = \sum^{m}_{i=1}(AB){}_{ii} = \sum^{m}_{i=1}\sum^{n}_{j=1}a_{ij}b_{ji}.
    \]
    \[
        Tr(BA) = \sum^{n}_{i=1}(BA){}_{ii} = \sum^{n}_{i=1}\sum^{m}_{j=1}b_{ij}a_{ji}.
    \]
    \par Đổi thứ tự lấy tổng, ta được:
    \[
        \sum^{m}_{i=1}\sum^{n}_{j=1}a_{ij}b_{ji} = \sum^{n}_{i=1}\sum^{m}_{j=1}b_{ij}a_{ji}.
    \]
    \par Do đó, $Tr(AB) = Tr(BA)$.
\end{proof}

\begin{exercise}
    Chứng minh rằng nếu $A$ và $B$ là các ma trận vuông cùng cấp, với $AB\ne BA$, thì
    \begin{enumerate}[label = (\alph*)]
        \item $(A+B){}^{2}\ne A^{2} + 2AB + B^{2}$,
        \item $(A+B)(A-B)\ne A^{2} - B^{2}$.
    \end{enumerate}
\end{exercise}

\begin{proof}
    \begin{enumerate}[label = (\alph*)]
        \item
              \begin{align*}
                  (A+B){}^{2} & = (A+B)(A+B)                \\
                              & = A^{2} + AB + BA + B^{2}   \\
                              & \ne A^{2} + AB + AB + B^{2} \\
                              & = A^{2} + 2AB + B^{2}
              \end{align*}
        \item
              \begin{align*}
                  (A+B)(A-B) & = A^{2} - AB + BA - B^{2}   \\
                             & = A^{2} - B^{2} + (BA - AB) \\
                             & \ne A^{2} - B^{2}
              \end{align*}
    \end{enumerate}
\end{proof}

\begin{exercise}
    Chứng minh rằng nếu $A$ và $B$ là các ma trận vuông với $AB = BA$ thì
    \[
        (A+B){}^{n} = A^{n} + nA^{n-1}B + \frac{n(n-1)}{2}A^{n-2}B^{2} + \cdots + B^{n}.
        \tag{(*)}
    \]
\end{exercise}

\begin{proof}Ta chứng minh bằng quy nạp theo $n$.
    \par (*) đúng với $n = 1$.
    \par Giả sử (*) đúng với $n = k$.
    \begin{align*}
        (A+B){}^{k}               & = \sum^{k}_{j=0}\binom{k}{j}A^{k-j}B^{j}                      \\
        \Rightarrow (A+B){}^{k+1} & = \sum^{k}_{j=0}A^{k+1-j}B^{j} + \sum^{k}_{j=0}A^{k-j}B^{j+1}
    \end{align*}
    \par Sử dụng đẳng thức Pascal $\binom{n}{k} = \binom{n-1}{k} + \binom{n-1}{k-1}$, ta suy ra
    \[
        (A+B){}^{k+1} = A^{k+1} + \sum^{k+1}_{j=1}\binom{k+1}{j}A^{k+1-j}B^{j} = \sum^{k+1}_{j=0}\binom{k+1}{j}A^{k+1-j}B^{j}.
    \]
    \par Vậy (*) đúng với $n = k + 1$.
    \par Do đó (*) với mọi số tự nhiên $n$.
\end{proof}

\begin{exercise}
    Hai ma trận vuông $A$ và $B$ được gọi là \textit{giao hoán} với nhau nếu $AB = BA$. Chứng minh rằng $A$ giao hoán với mọi ma trận vuông cùng cỡ với nó nếu và chỉ nếu nó là một ma trận vô hướng, tức là $A = cE$ trong đó $c\in\mathbb{F}$ và $E$ là ma trận đơn vị cùng cỡ với $A$.
\end{exercise}

\begin{proof}
    $(\Rightarrow)$
    \par $A = cE$. $B = (b_{ij}){}_{n\times n}$.
    \[
        \begin{pmatrix}
            b_{11} & b_{12} & \cdots & b_{1n} \\
            b_{21} & b_{22} & \cdots & b_{2n} \\
            \cdots & \cdots & \ddots & \vdots \\
            b_{n1} & b_{n2} & \cdots & b_{nn}
        \end{pmatrix}
    \]
    \par $(\Leftarrow)$
    \par $A = (a_{ij}){}_{n\times n}$. $S_{ij}$ là ma trận vuông cỡ $n$ với yếu tố hàng $i$ cột $j$ và hàng $j$ cột $i$ bằng đơn vị, các yếu tố còn lại bằng $0$. Không mất tính tổng quát, giả sử $i < j$.
    \[
        AS_{ij} =
        \begin{pmatrix}
            * & *      & \cdots & *      & * \\
            * & a_{ij} & *      & a_{ii} & * \\
            * & *      & \ddots & *      & * \\
            * & a_{jj} & *      & a_{ji} & * \\
            * & *      & \cdots & *      & *
        \end{pmatrix}
        \qquad
        S_{ij}A =
        \begin{pmatrix}
            * & *      & \cdots & *      & * \\
            * & a_{ji} & *      & a_{jj} & * \\
            * & *      & \ddots & *      & * \\
            * & a_{ii} & *      & a_{ij} & * \\
            * & *      & \cdots & *      & *
        \end{pmatrix}
    \]
    \par trong đó $*$ là các phần tử không.
    \par $AS_{ij} = S_{ij}A$. Đồng nhất hệ số hai ma trận tích này, ta được $a_{ii} = a_{jj}$ và $a_{ij} = a_{ji}$.
    \par Như vậy, $A$ có các yếu tố trên đường chéo chính bằng nhau, các yếu tố ngoài đường chéo chính bằng nhau.
    \par Ta viết lại ma trận $A$ như sau:
    \[
        A =
        \begin{pmatrix}
            a      & b      & \cdots & b      \\
            b      & a      & \cdots & b      \\
            \vdots & \vdots & \ddots & \vdots \\
            b      & b      & \cdots & a
        \end{pmatrix}
    \]
    \par $E_{1}$ là ma trận vuông cỡ $n$ có yếu tố hàng $1$, cột $1$ bằng đơn vị, các yếu tố còn lại bằng không.
    \[
        AE_{1} =
        \begin{pmatrix}
            a      & 0      & \cdots & 0      \\
            b      & 0      & \cdots & 0      \\
            \vdots & \vdots & \ddots & \vdots \\
            b      & 0      & \cdots & 0
        \end{pmatrix}
        \qquad
        E_{1}A =
        \begin{pmatrix}
            a      & b      & \cdots & b      \\
            0      & 0      & \cdots & 0      \\
            \vdots & \vdots & \ddots & \vdots \\
            0      & 0      & \cdots & 0
        \end{pmatrix}
    \]
    \par $AE_{1} = E_{1}A$. Đồng nhất hệ số ta được $b = 0$.
    \par Vậy $A = aE$, trong đó $E$ là ma trận đơn vị cùng cỡ với $A$.
\end{proof}

\begin{exercise}
    Ma trận vuông $A$ được gọi là \textit{ma trận chéo} nếu các yếu tố nằm ngoài đường chéo chính của nó đều bằng không. Chứng minh rằng ma trận vuông $A$ giao hoán với mọi ma trận chéo cùng cỡ với nó nếu và chỉ nếu chính $A$ là một ma trận chéo.
\end{exercise}

\begin{proof}
    $(\Rightarrow)$
    \par $A$ và $B$ là các ma trận chéo.
    \[
        \begin{pmatrix}
            a_{11} & 0      & \cdots & 0      \\
            0      & a_{22} & \cdots & 0      \\
            \vdots & \vdots & \ddots & \vdots \\
            0      & 0      & \vdots & a_{nn}
        \end{pmatrix}
        \begin{pmatrix}
            b_{11} & 0      & \cdots & 0      \\
            0      & b_{22} & \cdots & 0      \\
            \vdots & \vdots & \ddots & \vdots \\
            0      & 0      & \cdots & b_{nn}
        \end{pmatrix}
        =
        \begin{pmatrix}
            a_{11}b_{11} & 0            & \cdots & 0          \\
            0            & a_{22}b_{22} & \cdots & 0          \\
            \vdots       & \vdots       & \ddots & \vdots     \\
            0            & 0            & \cdots & a_{n}b_{n}
        \end{pmatrix}
    \]
    \[
        \begin{pmatrix}
            b_{11} & 0      & \cdots & 0      \\
            0      & b_{22} & \cdots & 0      \\
            \vdots & \vdots & \ddots & \vdots \\
            0      & 0      & \vdots & b_{nn}
        \end{pmatrix}
        \begin{pmatrix}
            a_{11} & 0      & \cdots & 0      \\
            0      & a_{22} & \cdots & 0      \\
            \vdots & \vdots & \ddots & \vdots \\
            0      & 0      & \cdots & a_{nn}
        \end{pmatrix}
        =
        \begin{pmatrix}
            a_{11}b_{11} & 0            & \cdots & 0          \\
            0            & a_{22}b_{22} & \cdots & 0          \\
            \vdots       & \vdots       & \ddots & \vdots     \\
            0            & 0            & \cdots & a_{n}b_{n}
        \end{pmatrix}
    \]
    \par Như vậy, $A$ giao hoán với mọi ma trận chéo.
    \par $(\Leftarrow)$
    \par $A = (a_{ij}){}_{n\times n}$. $E_{i}$ là ma trận vuông cỡ $n$ với yếu tố hàng $i$ cột $i$ bằng đơn vị và các yếu tố còn lại bằng không. $E_{i}$ là ma trận chéo.
    \[
        AE_{i} =
        \begin{pmatrix}
            0      & \cdots & a_{1i} & \cdots & 0      \\
            \vdots & \ddots & \vdots & \ddots & \vdots \\
            0      & \cdots & a_{ii} & \cdots & 0      \\
            \vdots & \ddots & \vdots & \ddots & \vdots \\
            0      & \cdots & a_{ni} & \cdots & 0
        \end{pmatrix}
    \]
    \[
        E_{i}A =
        \begin{pmatrix}
            0      & \cdots & 0      & \cdots & 0      \\
            \vdots & \ddots & \vdots & \ddots & \vdots \\
            a_{i1} & \cdots & a_{ii} & \cdots & a_{in} \\
            \vdots & \ddots & \vdots & \ddots & \vdots \\
            0      & \cdots & 0      & \cdots & 0
        \end{pmatrix}
    \]
    \par Do $AE_{i} = E_{i}A$ nên $a_{ij} = a_{ji} = 0, \forall j\ne i$.
    \par Như vậy, $A$ là một ma trận chéo.
\end{proof}

\begin{exercise}
    Chứng minh rằng nếu $A$ là một ma trận chéo với các yếu tố trên đường chéo chính đôi một khác nhau, thì mọi ma trận giao hoán với $A$ cũng là một ma trận chéo.
\end{exercise}

\begin{proof}
    \[
        A =
        \begin{pmatrix}
            a_{1}  & 0      & \cdots & 0      \\
            0      & a_{2}  & \cdots & 0      \\
            \vdots & \vdots & \ddots & \vdots \\
            0      & 0      & \vdots & a_{n}
        \end{pmatrix}
    \]
    \par trong đó, $a_{i}\ne a_{j},\forall i\ne j$.
    \par $B = (b_{ij}){}_{n\times n}$.
    \[
        AB =
        \begin{pmatrix}
            a_{1}b_{11} & a_{1}b_{12} & \cdots & a_{1}b_{1n} \\
            a_{2}b_{21} & a_{2}b_{22} & \cdots & a_{2}b_{2n} \\
            \vdots      & \vdots      & \ddots & \vdots      \\
            a_{n}b_{n1} & a_{n}b_{n2} & \cdots & a_{n}b_{nn}
        \end{pmatrix}
        \qquad
        BA =
        \begin{pmatrix}
            a_{1}b_{11} & a_{2}b_{12} & \cdots & a_{n}b_{1n} \\
            a_{1}b_{21} & a_{2}b_{22} & \cdots & a_{n}b_{2n} \\
            \vdots      & \vdots      & \ddots & \vdots      \\
            a_{1}b_{n1} & a_{2}b_{n2} & \cdots & a_{n}b_{nn}
        \end{pmatrix}
    \]
    \par $AB = BA$ thì $a_{i}b_{ij} = a_{j}b_{ij},\forall i\ne j$. Vì $a_{i} \ne a_{j}$ nên $b_{ij} = 0,\forall i\ne j$.
    \par Vậy $B$ là một ma trận chéo. Tức là mọi ma trận giao hoán với $A$ cũng là ma trận chéo.
\end{proof}

\begin{exercise}
    Gọi $D = \text{diag}(a_{1},a_{2},\ldots,a_{n})$ là ma trận chéo với các yếu tố trên đường chéo chính lần lượt bằng $a_{1}, a_{2}, \ldots, a_{n}$. Chứng minh rằng nhân $D$ với $A$ từ bên trái có nghĩa là nhân các hàng của $A$ theo thứ tự với $a_{1}, a_{2},\ldots,a_{n}$; còn nhân $D$ với $A$ từ bên phải có nghĩa là nhân với các cột của $A$ theo thứ tự với $a_{1}, a_{2}, \ldots, a_{n}$.
\end{exercise}

\begin{proof}
    $A = (a_{ij}){}_{n\times n}$.
    \[
        DA =
        \begin{pmatrix}
            a_{1}  & 0      & \cdots & 0      \\
            0      & a_{2}  & \cdots & 0      \\
            \vdots & \vdots & \ddots & \vdots \\
            0      & 0      & \cdots & a_{n}
        \end{pmatrix}
        \begin{pmatrix}
            a_{11} & a_{12} & \cdots & a_{1n} \\
            a_{21} & a_{22} & \cdots & a_{2n} \\
            \vdots & \vdots & \ddots & \vdots \\
            a_{n1} & a_{n2} & \cdots & a_{nn}
        \end{pmatrix}
        =
        \begin{pmatrix}
            a_{1}a_{11} & a_{1}a_{12} & \cdots & a_{1}a_{1n} \\
            a_{2}a_{21} & a_{2}a_{22} & \cdots & a_{2}a_{2n} \\
            \vdots      & \vdots      & \ddots & \vdots      \\
            a_{n}a_{n1} & a_{n}a_{n2} & \cdots & a_{n}a_{nn}
        \end{pmatrix}
    \]
    \[
        AD =
        \begin{pmatrix}
            a_{11} & a_{12} & \cdots & a_{1n} \\
            a_{21} & a_{22} & \cdots & a_{2n} \\
            \vdots & \vdots & \ddots & \vdots \\
            a_{n1} & a_{n2} & \cdots & a_{nn}
        \end{pmatrix}
        \begin{pmatrix}
            a_{1}  & 0      & \cdots & 0      \\
            0      & a_{2}  & \cdots & 0      \\
            \vdots & \vdots & \ddots & \vdots \\
            0      & 0      & \cdots & a_{n}
        \end{pmatrix}
        =
        \begin{pmatrix}
            a_{1}a_{11} & a_{2}a_{12} & \cdots & a_{n}a_{1n} \\
            a_{1}a_{21} & a_{2}a_{22} & \cdots & a_{n}a_{2n} \\
            \vdots      & \vdots      & \ddots & \vdots      \\
            a_{1}a_{n1} & a_{2}a_{n2} & \cdots & a_{n}a_{nn}
        \end{pmatrix}
    \]
\end{proof}

\begin{exercise}
    Tìm tất cả các ma trận giao hoán với ma trận sau đây:
    \[
        \begin{pmatrix}
            3 & 1 & 0 \\
            0 & 3 & 1 \\
            0 & 0 & 3
        \end{pmatrix}
    \]
\end{exercise}

\begin{proof}[Lời giải]
    Đặt ma trận cần tìm là $A = (a_{ij}){}_{n\times n}$.
    \[
        \begin{pmatrix}
            3 & 1 & 0 \\
            0 & 3 & 1 \\
            0 & 0 & 3
        \end{pmatrix}
        \begin{pmatrix}
            a_{11} & a_{12} & a_{13} \\
            a_{21} & a_{22} & a_{23} \\
            a_{31} & a_{32} & a_{33}
        \end{pmatrix}=
        \begin{pmatrix}
            3a_{11} + a_{21} & 3a_{12} + a_{22} & 3a_{13} + a_{23} \\
            3a_{21} + a_{31} & 3a_{22} + a_{32} & 3a_{23} + a_{33} \\
            3a_{31}          & 3a_{32}          & 3a_{33}
        \end{pmatrix}
    \]
    \[
        \begin{pmatrix}
            a_{11} & a_{12} & a_{13} \\
            a_{21} & a_{22} & a_{23} \\
            a_{31} & a_{32} & a_{33}
        \end{pmatrix}
        \begin{pmatrix}
            3 & 1 & 0 \\
            0 & 3 & 1 \\
            0 & 0 & 3
        \end{pmatrix}=
        \begin{pmatrix}
            3a_{11} & a_{11} + 3a_{12} & a_{12} + 3a_{13} \\
            3a_{21} & a_{21} + 3a_{22} & a_{22} + 3a_{23} \\
            3a_{31} & a_{31} + 3a_{32} & a_{32} + 3a_{33}
        \end{pmatrix}
    \]
    \par Đồng nhất hệ số hai ma trận tích:
    \[
        \Rightarrow
        \begin{cases}
            3a_{11} + a_{21} = 3a_{11} \Leftrightarrow a_{21} = 0               \\
            3a_{12} + a_{22} = a_{11} + 3a_{12} \Leftrightarrow a_{11} = a_{22} \\
            3a_{13} + a_{23} = a_{12} + 3a_{13} \Leftrightarrow a_{12} = a_{23} \\
            3a_{21} + a_{31} = 3a_{21} \Leftrightarrow a_{31} = 0               \\
            3a_{22} + a_{32} = a_{21} + 3a_{22} \Leftrightarrow a_{21} = a_{32} \\
            3a_{23} + a_{33} = a_{22} + 3a_{23} \Leftrightarrow a_{22} = a_{33} \\
            3a_{32} = a_{31} + 3a_{32} \Leftrightarrow a_{31} = 0               \\
            3a_{33} = a_{32} + 3a_{33} \Leftrightarrow a_{32} = 0
        \end{cases}
    \]
    \par Vậy các ma trận giao hoán với ma trận
    $\begin{pmatrix}
            3 & 1 & 0 \\
            0 & 3 & 1 \\
            0 & 0 & 3
        \end{pmatrix}$
    có dạng:
    \[
        \begin{pmatrix}
            a & b & c \\
            0 & a & b \\
            0 & 0 & a
        \end{pmatrix}
    \]
\end{proof}

\begin{exercise}\label{characteristic-poly}
    Chứng minh rằng ma trận
    $A = \begin{pmatrix}a & b \\ c & d\end{pmatrix}$ thỏa mãn phương trình
    \[
        X^{2} - (a+d)X + (ad-bc) = 0
    \]
\end{exercise}

\begin{proof}
    \[
        A^{2} =
        \begin{pmatrix}
            a^{2}+bc & b(a+d)   \\
            c(a+d)   & d^{2}+bc
        \end{pmatrix}
    \]
    \begin{align*}
        A^{2} - (a+d)A + (ad-bc) & = \begin{pmatrix}a^{2}+bc & b(a+d) \\ c(a+d) & d^{2}+bc\end{pmatrix} - (a+d)\begin{pmatrix}a & b \\ c & d\end{pmatrix} + (ad-bc)\begin{pmatrix}1 & 0 \\ 0 & 1\end{pmatrix} \\
                                 & = \begin{pmatrix}a^{2}+bc - a(a+d) + (ad-bc) & b(a+d) - b(a+d) \\ c(a+d) - c(a+d) & d^{2}+bc - d(a+d) + (ad-bc)\end{pmatrix}                                                                       \\
                                 & = \begin{pmatrix}0 & 0 \\ 0 & 0 \end{pmatrix} = 0.
    \end{align*}
\end{proof}

\begin{exercise}\label{minimal-poly}
    Chứng minh rằng đối với mỗi ma trận vuông $A$, tồn tại một đa thức khác không $f(X)$ sao cho $f(A) = 0$. Hơn nữa, mọi đa thức có tính chất đó đều là bội của một đa thức $f_{0}(X)$ như thế.
\end{exercise}

\begin{proof}
    Giả sử $A\in M(n\times n,\mathbb{F})$.
    \par $M(n\times n,\mathbb{F})$ là một không gian vector với số chiều là $n^{2}$.
    \par Do đó hệ $(E_{n}, A, A^{2}, \ldots, A^{n^{2}})$ phụ thuộc tuyến tính, tức là tồn tại các yếu tố $a_{0}, a_{1}, a_{2}, \ldots, a_{n^{2}}$ trong trường $\mathbb{F}$, không đồng thời bằng không sao cho:
    \[
        a_{0} + a_{1}A + a_{2}A^{2} + \cdots + a_{n^{2}}A^{n^{2}} = 0
    \]
    \par Điều đó có nghĩa là đa thức $f(X) = a_{0} + a_{1}X + a_{2}X^{2} + \cdots + a_{n^{2}}X^{n^{2}}$ khác không và thỏa mãn $f(A) = 0$.
    \par Như vậy tập hợp các đa thức khác không nhận $A$ làm nghiệm là tập khác rỗng. Theo nguyên lý sắp thứ tự tốt, tập hợp này chứa đa thức có bậc nhỏ nhất. Đặt đa thức như vậy là $f_{0}(X)$.
    \par Giả sử đa thức $p(X)$ thỏa mãn $p(A) = 0$. Ta thực hiện phép chia cho đa thức $f_{0}(X)$.
    \[
        p(X) = f_{0}(X)q(X) + r(X)
    \]
    \par trong đó $\deg r(X) < \deg f_{0}(X)$.
    \par $p(A) = 0\Rightarrow f_{0}(A)q(A) + r(A) = 0 \Rightarrow r(A) = 0$.
    \par Đa thức $r(X)$ có bậc bé hơn $f_{0}(X)$, do đó $r(X) = 0$. Vậy $p(X)$ là bội của $f_{0}(X)$.
\end{proof}

\begin{exercise}
    Giả sử $n$ không chia hết cho đặc số $p$ của trường $\mathbb{F}$. Chứng minh rằng không tồn tại các ma trận $A, B\in M(n\times n, \mathbb{F})$ sao cho $AB - BA = E_{n}$.
\end{exercise}

\begin{proof}
    Ánh xạ vết $Tr: M(n\times n, \mathbb{F})\rightarrow \mathbb{F}$ là một ánh xạ tuyến tính.
    \par Theo Bài~\ref{trace-of-products}, $Tr(AB - BA) = Tr(AB) - Tr(BA) = 0$.
    \par Mà $Tr(E_{n}) = n$ và $n$ không là bội của đặc số của trường $\mathbb{F}$ nên $n\ne 0$. Do đó $Tr(AB - BA)\ne Tr(E_{n})$.
    \par Điều đó cũng có nghĩa là $AB - BA \ne E_{n}$.
\end{proof}

\begin{exercise}
    Giả sử $A$ là một ma trận vuông cỡ 2 và $k$ là một số nguyên $\ge 2$. Chứng minh rằng $A^{k} = 0$ nếu và chỉ nếu $A^{2} = 0$.
\end{exercise}

\begin{proof}
    $A = \begin{pmatrix}a & b \\ c & d\end{pmatrix}$
    \par $(\Rightarrow)$
    \par $A^{2} = 0$ thì $A^{k} = 0,\forall k\ge 2$.
    \par $(\Leftarrow)$
    \par Nếu $k = 2$, ta có ngay điều phải chứng minh.
    \par Nếu $k > 2$.
    \par Theo Bài~\ref{characteristic-poly}:
    \[
        A^{2} - (a+d)A + (ad - bc) = 0
    \]
    \begin{enumerate}[label = Trường hợp \arabic*:,itemindent=2cm]
        \item $ad - bc = 0, a + d = 0$.
              \[
                  \Rightarrow A^{2} = 0
              \]
        \item $ad - bc = 0, a + d \ne 0$.
              \[
                  \Rightarrow A^{2} = (a+d)A \Rightarrow A^{k} = (a+d)A^{k-1} \Rightarrow A^{k-1} = 0\Rightarrow A^{k-2} = 0 \Rightarrow \cdots \Rightarrow A^{2} = 0
              \]
        \item $ad - bc \ne 0$.
              \par $A^{k} = 0\Rightarrow A^{k+1} = 0$.
              \[
                  \Rightarrow A^{k+1} - (a+d)A^{k} + (ad-bc)A^{k-1} = 0 \Rightarrow A^{k-1} = 0 \Rightarrow \cdots \Rightarrow A^{2} = 0.
              \]
    \end{enumerate}
    \par Tóm lại, trong tất cả các trường hợp, ta đều có $A^{2} = 0$.
\end{proof}

\begin{exercise}
    Tìm tất cả các ma trận vuông $A$ cỡ 2 sao cho $A^{2} = 0$.
\end{exercise}

\begin{proof}[Lời giải]
    $A = \begin{pmatrix}a & b \\ c & d\end{pmatrix}\Rightarrow A^{2} = \begin{pmatrix}a^{2}+bc & b(a+d) \\ c(a+d) & d^{2}+bc\end{pmatrix}$.
    \[
        A^{2} = 0
        \Rightarrow
        \begin{cases}
            a^{2}+bc = 0 \\
            d^{2}+bc = 0 \\
            b(a + d) = 0 \\
            c(a + d) = 0 \\
        \end{cases}
    \]
    \begin{enumerate}[label = Trường hợp \arabic*:,itemindent=2cm]
        \item $a + d \ne 0$.
              \par $b(a+d) = c(a+d) = 0$ và $a+d\ne 0$ nên $b = c = 0$.
              \par $a^{2} + bc = d^{2} + bc = b = c = 0$ nên $a = b = c = d = 0$.
        \item $a + d = 0$.
              \par Nếu $a = d = 0$ thì $bc = 0$.
              \par $A = \begin{pmatrix}0 & t \\ 0 & 0\end{pmatrix}$ hoặc $A = \begin{pmatrix}0 & 0 \\ t & 0\end{pmatrix}$.
              \par Nếu $a, d\ne 0$ thì
              \par $A = \begin{pmatrix}t & u \\ \frac{-t^{2}}{u} & -t\end{pmatrix}$ trong đó $t \ne 0$.
    \end{enumerate}
    \par Như vậy $A$ thuộc một trong các dạng sau:
    \[
        \begin{pmatrix}
            0 & t \\
            0 & 0
        \end{pmatrix}
        \qquad
        \begin{pmatrix}
            0 & 0 \\
            t & 0
        \end{pmatrix}
        \qquad
        \begin{pmatrix}
            t                & u  \\
            \frac{-t^{2}}{u} & -t
        \end{pmatrix}
    \]
\end{proof}

\begin{exercise}
    Tìm tất cả các ma trận vuông $A$ cỡ 2 sao cho $A^{2} = E_{2}$.
\end{exercise}

\begin{proof}[Lời giải]
    $A = \begin{pmatrix}a & b \\ c & d\end{pmatrix}\Rightarrow A^{2} = \begin{pmatrix}a^{2}+bc & b(a+d) \\ c(a+d) & d^{2}+bc\end{pmatrix}$.
    \begin{enumerate}[label = Trường hợp \arabic*:,itemindent=2cm]
        \item $a + d\ne 0$.
              \par Suy ra $b = c = 0$, $a^{2} = d^{2} = 1$.
              \par $(a,d)\in\{  (1,1), (-1,-1), (1,-1), (-1,1) \}$.
        \item $a + d = 0$.
              \par $a = t, d = -t$, suy ra $bc = 1 - t^{2}$.
              \par Nếu $t^{2} = 1$ thì $bc = 0$.
              \par Nếu $t^{2} \ne 1$ thì $b = u, c = \frac{1-t^{2}}{u}$.
    \end{enumerate}
    \par Như vậy $A$ thuộc một trong các dạng sau:
    \[
        \begin{pmatrix}
            1 & 0 \\
            0 & 1
        \end{pmatrix}
        \quad
        \begin{pmatrix}
            -1 & 0  \\
            0  & -1
        \end{pmatrix}
        \quad
        \begin{pmatrix}
            1 & 0  \\
            t & -1
        \end{pmatrix}
        \quad
        \begin{pmatrix}
            1 & t  \\
            0 & -1
        \end{pmatrix}
        \quad
        \begin{pmatrix}
            -1 & 0 \\
            t  & 1
        \end{pmatrix}
        \quad
        \begin{pmatrix}
            -1 & t \\
            0  & 1
        \end{pmatrix}
    \]
    \[
        \begin{pmatrix}
            t                 & u  \\
            \frac{1-t^{2}}{u} & -t
        \end{pmatrix} \quad (t\ne\pm{1})
    \]
\end{proof}

\begin{exercise}
    Giải phương trình $AX = 0$, trong đó $A$ là ma trận vuông cỡ 2 đã cho còn $X$ là ma trận vuông cỡ 2 cần tìm.
\end{exercise}

\begin{proof}[Lời giải]
    $A = \begin{pmatrix}a & b \\ c & d\end{pmatrix}$; $X = \begin{pmatrix}x_{11} & x_{12} \\ x_{21} & x_{22}\end{pmatrix}$.
    \[
        AX =
        \begin{pmatrix}
            ax_{11}+bx_{21} & ax_{12}+bx_{22} \\
            cx_{11}+dx_{21} & cx_{12}+dx_{22}
        \end{pmatrix}
    \]
    \begin{enumerate}[label = Trường hợp \arabic*:,itemindent=2cm]
        \item $a = b = c = d = 0$.
              \par Mọi ma trận vuông cỡ 2 đều thỏa mãn.
        \item $a = b = c = 0, d\ne 0$.
              \par $x_{21} = x_{22} = 0$.
        \item $a = b = d = 0, c\ne 0$.
              \par $x_{11} = x_{12} = 0$.
        \item $a = c = d = 0, b\ne 0$.
              \par $x_{21} = x_{22} = 0$.
        \item $b = c = d = 0, a\ne 0$.
              \par $x_{11} = x_{12} = 0$.
        \item $a = b = 0; c, d\ne 0$.
              \par $x_{11} = dp, x_{21} = -cp, x_{12} = dq, x_{22} = -cq$.
        \item $a, b\ne 0; c = d = 0$.
              \par $x_{11} = bp, x_{21} = -ap, x_{12} = bq, x_{22} = -aq$.
        \item $a = c = 0; b, d\ne 0$.
              \par $x_{21} = x_{22} = 0$.
        \item $a, c\ne 0; b = d = 0$.
              \par $x_{11} = x_{12} = 0$.
        \item $a = d = 0; b, c\ne 0$.
              \par $x_{21} = x_{22} = x_{11} = x_{12} = 0$.
        \item $a, d\ne 0; b = c = 0$.
              \par $x_{11} = x_{12} = x_{21} = x_{22} = 0$.
        \item $a = 0\vee b = 0 \vee c = 0 \vee d = 0$.
              \par $x_{11} = x_{12} = x_{21} = x_{22} = 0$.
        \item $a,b,c,d\ne 0$.
              \par Nếu $ad - bc = 0$ thì $x_{11} = bp, x_{21} = -ap, x_{12} = bq, x_{22} = -aq$.
              \par Nếu $ad - bc \ne 0$ thì $x_{11} = x_{12} = x_{21} = x_{22} = 0$.
    \end{enumerate}
\end{proof}

\begin{exercise}
    Tìm ma trận nghịch đảo (nếu có) của ma trận
    \[
        A =
        \begin{pmatrix}
            a & b \\
            c & d
        \end{pmatrix}
    \]
\end{exercise}

\begin{proof}
    Theo Bài~\ref{characteristic-poly}:
    \[
        A^{2} - (a+d)A + (ad-bc) = 0
    \]
    \begin{enumerate}[label = Trường hợp \arabic*:,itemindent=2cm]
        \item $ad - bc = 0$.
              \par Giả sử $A$ khả nghịch.
              \[
                  A^{-1}(A^{2} - (a+d)A) = 0 \Rightarrow A = (a+d)E
              \]
              \par Đồng nhất hệ số của $A$ và $(a+d)E$, ta suy ra $a = b = c =d = 0$. Ma trận này không khả nghịch.
              \par Vậy khi $ad - bc = 0$ thì $A$ không khả nghịch.
        \item $ad - bc \ne 0$
              \par Giả sử $A$ khả nghịch.
              \[
                  A^{-1}(A^{2} - (a+d)A + (ad-bc)) = 0 \Rightarrow A^{-1} = \frac{1}{ad-bc}((a+d) - A) = \frac{1}{ad-bc}\begin{pmatrix}d & -b \\ -c & a\end{pmatrix}
              \]
              \par Kiểm tra lại, ta thấy ma trận này thỏa mãn.
    \end{enumerate}
    \par Như vậy, khi $ad - bc = 0$, ma trận $A$ không khả nghịch. Khi $ad - bc \ne 0$, $A$ khả nghịch và
    \[
        A^{-1} = \frac{1}{ad-bc}\begin{pmatrix}d & -b \\ -c & a\end{pmatrix}
    \]
\end{proof}

\begin{exercise}
    Giả sử $V = V_{1}\oplus V_{2}$, trong đó $V_{1}$ có cơ sở $(\alpha_{1}, \ldots, \alpha_{k})$, $V_{2}$ có cơ sở $(\alpha_{k+1}, \ldots, \alpha_{n})$. Tìm ma trận của phép chiếu lên $V_{1}$ theo phương $V_{2}$ đối với cơ sở $(\alpha_{1}, \ldots, \alpha_{n})$.
\end{exercise}

\begin{proof}[Lời giải]
    $pr_{V_{1}}$ là phép chiếu lên $V_{1}$ theo phương $V_{2}$.
    \[
        pr_{V_{1}}(\alpha_{i}) =
        \begin{cases}
            \alpha_{i} (i = \overline{1, k}) \\
            0 (i = \overline{k+1,n})
        \end{cases}
    \]
    \par Vậy ma trận của $pr_{V_{1}}$ đối với cơ sở $(\alpha_{1}, \ldots, \alpha_{n})$ của $V$ và cơ sở $(\alpha_{1}, \ldots, \alpha_{k})$ của $V_{1}$ là:
    \[
        \begin{pmatrix}
            1      & 0      & \cdots & 0      & \cdots & 0      \\
            0      & 1      & \cdots & 0      & \cdots & 0      \\
            \vdots & \vdots & \ddots & \vdots & \ddots & \vdots \\
            0      & 0      & \cdots & 1      & \cdots & 0      \\
            \vdots & \vdots & \ddots & \vdots & \ddots & \vdots \\
            0      & 0      & \cdots & 0      & \cdots & 0
        \end{pmatrix}
    \]
\end{proof}

\begin{exercise}
    Chứng minh rằng nếu $V = V_{1}\oplus V_{2}$, thì $V$ đẳng cấu với tích trực tiếp $V_{1}\times V_{2}$.
\end{exercise}

\begin{proof}
    $v, \alpha, \beta \in V$. Giả sử
    \[
        v = \underbrace{v_{1}}_{\in V_{1}} + \underbrace{v_{2}}_{\in V_{2}}
        \qquad
        \alpha = \underbrace{\alpha_{1}}_{\in V_{1}} + \underbrace{\alpha_{2}}_{\in V_{2}}
        \qquad
        \beta = \underbrace{\beta_{1}}_{\in V_{1}} + \underbrace{\beta_{2}}_{\in V_{2}}
    \]
    \par Ta xét ánh xạ:
    \[
        \begin{split}
            \phi:\ & V_{1}\oplus V_{2} \to V_{1}\times V_{2} \\
            & v = v_{1} + v_{2} \mapsto (v_{1}, v_{2})
        \end{split}
    \]
    \par Ánh xạ này là một đồng cấu tuyến tính vì:
    \[
        \begin{split}
            \phi(\alpha + \beta) & = (\alpha_{1} + \beta_{1}, \alpha_{2} + \beta_{2}) = (\alpha_{1}, \alpha_{2}) + (\beta_{1}, \beta_{2}) = \phi(\alpha) + \phi(\beta) \\
            \phi(a\alpha) & = (a\alpha_{1}, a\alpha_{2}) = a(\alpha_{1}, \alpha_{2}) = a\phi(\alpha)
        \end{split}
    \]
    \par Ánh xạ này là đơn cấu vì nếu $\phi(\alpha) = \phi(\beta)$ khi và chỉ khi $\alpha_{1} = \beta_{1}, \alpha_{2} = \beta_{2}$, tức là $\alpha = \beta$ (theo định nghĩa tổng trực tiếp).
    \par Ánh xạ này là toàn cấu vì $\forall (v_{1}, v_{2})\in V_{1}\times V_{2}$ thì $\phi(v_{1} + v_{2}) = (v_{1}, v_{2})$.
    \par Do đó $\phi$ là đẳng cấu.
    \par Như vậy, $V = V_{1}\oplus V_{2} \cong V_{1}\times V_{2}$.
\end{proof}

\begin{exercise}
    Chứng minh rằng tồn tại duy nhất tự đồng cấu $\mathbb{R}_{3}\to\mathbb{R}_{3}$ chuyển các vector $\alpha_{1} = (2,3,5)$, $\alpha_{2} = (0, 1, 2)$, $\alpha_{3} = (1, 0, 0)$ tương ứng thành các vector $\beta_{1} = (1, 1, 1)$, $\beta_{2} = (1, 1, -1)$, $\beta_{3} = (2, 1, 2)$. Tìm ma trận của $f$ đối với cơ sở chính tắc của không gian.
\end{exercise}

\begin{proof}
    Xét ràng buộc tuyến tính sau:
    \[
        a_{1}\alpha_{1} + a_{2}\alpha_{2} + a_{3}\alpha_{3} = (0,0,0)
    \]
    \begin{align*}
        \Leftrightarrow & (2a_{1} + a_{3}, 3a_{1} + a_{2}, 5a_{1} + 2a_{3}) = (0,0,0)                                   \\
        \Leftrightarrow & (2a_{1} + a_{3}, 3a_{1} + a_{2}, 5a_{1} + 2a_{3} - 2a_{1} - a_{3} - 3a_{1} - a_{2}) = (0,0,0) \\
        \Leftrightarrow & (2a_{1} + a_{3}, 3a_{1} + a_{2}, a_{3} - a_{2}) = (0,0,0)                                     \\
        \Leftrightarrow & (2a_{1} + a_{3}, a_{1} + a_{2} - a_{3}, a_{3} - a_{2}) = (0,0,0)                              \\
        \Leftrightarrow & (2a_{1} + a_{3}, a_{1}, a_{3} - a_{2}) = (0,0,0)                                              \\
        \Leftrightarrow & (a_{3}, a_{1}, a_{2}) = (0,0,0)
    \end{align*}
    \par Vậy hệ $(\alpha_{1}, \alpha_{2}, \alpha_{3})$ độc lập tuyến tính.
    \par Không gian vector $\mathbb{R}_{3}$ có số chiều là 3, do đó $(\alpha_{1}, \alpha_{2}, \alpha_{3})$ là một cơ sở của $\mathbb{R}_{3}$.
    \par Một ánh xạ tuyến tính được xác định duy nhất bởi tác động của nó lên một cơ sở, do đó tồn tại duy nhất tự đồng cấu $f: \mathbb{R}_{3}\to\mathbb{R}_{3}$ chuyển $\alpha_{1}\mapsto\beta_{1}$, $\alpha_{2}\mapsto\beta_{2}$, $\alpha_{3}\mapsto\beta_{3}$.
    \[
        \begin{cases}
            \beta_{1} = 1\alpha_{1} + (-2)\alpha_{2} + (-1)\alpha_{3} \\
            \beta_{2} = 3\alpha_{1} + (-8)\alpha_{2} + (-5)\alpha_{3} \\
            \beta_{3} = 0\alpha_{1} + 1\alpha_{2} + 2\alpha_{3}
        \end{cases}
    \]
    \par Như vậy ma trận của tự đồng cấu $f$ đối với cơ sở $(\alpha_{1}, \alpha_{2}, \alpha_{3})$ là:
    \[
        A =
        \begin{pmatrix}
            1  & 3  & 0 \\
            -2 & -8 & 1 \\
            -1 & -5 & 2
        \end{pmatrix}
    \]
    \par Ma trận chuyển từ cơ sở $(\alpha_{1}, \alpha_{2}, \alpha_{3})$ sang cơ sở chính tắc là:
    \[
        C =
        \begin{pmatrix}
            0 & 2  & -1 \\
            0 & -5 & 3  \\
            1 & -4 & 2
        \end{pmatrix}
    \]
    \[
        \begin{cases}
            f(e_{1}) = f(\alpha_{3}) = \beta_{3} = (2,1,2)                                                               \\
            f(e_{2}) = f(2\alpha_{1} - 5\alpha_{2} - 4\alpha_{3}) = 2\beta_{1} - 5\beta_{2} - 4\beta_{3} = (-11, -7, -1) \\
            f(e_{3}) = f(-\alpha_{1} + 3\alpha_{2} + 2\alpha_{3}) = -\beta_{1} + 3\beta_{2} + 2\beta_{3} = (6, 4, 0)
        \end{cases}
    \]
    \par Vậy ma trận của $f$ đối với cơ sở chính tắc là:
    \[
        \begin{pmatrix}
            2 & -11 & 6 \\
            1 & -7  & 4 \\
            2 & -1  & 0
        \end{pmatrix}
    \]
\end{proof}

\begin{exercise}
    Tự đồng cấu $f$ của không gian vector $\mathbb{F}^{n}$ chuyển các vector độc lập tuyến tính $(\alpha_{1}, \ldots, \alpha_{n})$ tương ứng thành các vector $\beta_{1}, \ldots, \beta_{n}$. Chứng minh rằng ma trận $M(f)$ của $f$ đối với một cơ sở nào đó $(e_{1},\ldots, e_{n})$ thỏa mãn hệ thức $M(f) = BA^{-1}$, trong đó các cột của ma trận $A$ và ma trận $B$ là tọa độ tương ứng của các vector $\alpha_{1}, \ldots, \alpha_{n}$ và $\beta_{1}, \ldots, \beta_{n}$ đối với cơ sở $(e_{1}, \ldots, e_{n})$.
\end{exercise}

\begin{proof}
    Theo định nghĩa của $M(f)$:
    \[
        (f(e_{1}) \ldots f(e_{n})) = (e_{1} \ldots e_{n})M(f)
    \]
    \par $(\alpha_{1}, \ldots, \alpha_{n})$ độc lập tuyến tính trong không gian vector $\mathbb{F}^{n}$ nên $(\alpha_{1}, \ldots, \alpha_{n})$ là một cơ sở.
    \par $A$ là ma trận mà các cột lần lượt là tọa độ của các vector $\alpha_{1}, \ldots, \alpha_{n}$ đối với cơ sở $(e_{1}, \ldots, e_{n})$.
    \par $B$ là ma trận mà các cột lần lượt là tọa độ của các vector $\beta_{1}, \ldots, \beta_{n}$ đối với cơ sở $(e_{1}, \ldots, e_{n})$.
    \[
        \begin{split}
            (\alpha_{1} \ldots \alpha_{n}) = (e_{1} \ldots e_{n})A \\
            (\beta_{1} \ldots \beta_{n}) = (e_{1} \ldots e_{n})B
        \end{split}
    \]
    \begin{align*}
        (f(\alpha_{1})\ldots f(\alpha_{n})) & = (f(e_{1})\ldots f(e_{n}))A \\
                                            & = (e_{1}\ldots e_{n})M(f)A
    \end{align*}
    \par Bên cạnh đó:
    \begin{align*}
        (f(\alpha_{1})\ldots f(\alpha_{n})) & = (\beta_{1} \ldots \beta_{n}) \\
                                            & = (e_{1} \ldots e_{n})B
    \end{align*}
    \par Suy ra $B = M(f)A$. Mà $A$ là ma trận chuyển từ cơ sở $(e_{1}, \ldots, e_{n})$ sang $(\alpha_{1}, \ldots, \alpha_{n})$ nên $A$ khả nghịch.
    \par Do vậy $M(f) = M(f)AA^{-1} = BA^{-1}$.
\end{proof}

\begin{exercise}
    Chứng minh rằng phép nhân với ma trận $\begin{pmatrix}a & b \\ c & d\end{pmatrix}$
    \begin{enumerate}[label = (\alph*)]
        \item từ bên trái,
        \item từ bên phải
    \end{enumerate}
    \par là các tự đồng cấu của không gian các ma trận vuông cỡ 2. Hãy tìm ma trận của tự đồng cấu đó đối với cơ sở gồm các ma trận sau đây:
    \[
        \begin{pmatrix}
            1 & 0 \\
            0 & 0
        \end{pmatrix},
        \begin{pmatrix}
            0 & 1 \\
            0 & 0
        \end{pmatrix},
        \begin{pmatrix}
            0 & 0 \\
            1 & 0
        \end{pmatrix},
        \begin{pmatrix}
            0 & 0 \\
            0 & 1
        \end{pmatrix}.
    \]
\end{exercise}

\begin{proof}
    Phép nhân ma trận có tính phân phối với phép cộng ma trận nên:
    \[
        \begin{pmatrix}
            a & b \\
            c & d
        \end{pmatrix}
        \left(
        \begin{pmatrix}
            x_{11} & x_{12} \\
            x_{21} & x_{22}
        \end{pmatrix}
        +
        \begin{pmatrix}
            y_{11} & y_{12} \\
            y_{21} & y_{22}
        \end{pmatrix}
        \right)
        =
        \begin{pmatrix}
            a & b \\
            c & d
        \end{pmatrix}
        \begin{pmatrix}
            x_{11} & x_{12} \\
            x_{21} & x_{22}
        \end{pmatrix}
        +
        \begin{pmatrix}
            a & b \\
            c & d
        \end{pmatrix}
        \begin{pmatrix}
            y_{11} & y_{12} \\
            y_{21} & y_{22}
        \end{pmatrix}
    \]
    \par Bên cạnh đó, mọi ma trận đều giao hoán với ma trận vô hướng nên:
    \begin{align*}
        \begin{pmatrix}
            a & b \\
            c & d
        \end{pmatrix}
        \begin{pmatrix}
            tx_{11} & tx_{12} \\
            tx_{21} & tx_{22}
        \end{pmatrix} & =
        \begin{pmatrix}
            a & b \\
            c & d
        \end{pmatrix}
        \begin{pmatrix}
            t & 0 \\
            0 & t
        \end{pmatrix}
        \begin{pmatrix}
            x_{11} & x_{12} \\
            x_{21} & x_{22}
        \end{pmatrix} \\
        & =
        \begin{pmatrix}
            t & 0 \\
            0 & t
        \end{pmatrix}
        \begin{pmatrix}
            a & b \\
            c & d
        \end{pmatrix}
        \begin{pmatrix}
            x_{11} & x_{12} \\
            x_{21} & x_{22}
        \end{pmatrix} \\
        & =
        t
        \begin{pmatrix}
            a & b \\
            c & d
        \end{pmatrix}
        \begin{pmatrix}
            x_{11} & x_{12} \\
            x_{21} & x_{22}
        \end{pmatrix}
    \end{align*}
    \par Như vậy, phép nhân với một ma trận vuông cỡ 2 cho trước từ bên trái là một tự đồng cấu của không gian $M(2\times 2,\mathbb{F})$.
    \par Điều tương tự cũng đúng với phép nhân với một ma trận vuông cỡ 2 cho trước từ bên phải.
    \begin{enumerate}[label = (\alph*)]
        \item
            \[
                \begin{cases}
                    \begin{pmatrix}
                        a & b \\
                        c & d
                    \end{pmatrix}
                    \begin{pmatrix}
                        1 & 0 \\
                        0 & 0
                    \end{pmatrix}=
                    \begin{pmatrix}
                        a & 0 \\
                        c & 0
                    \end{pmatrix} \\
                    \begin{pmatrix}
                        a & b \\
                        c & d
                    \end{pmatrix}
                    \begin{pmatrix}
                        0 & 1 \\
                        0 & 0
                    \end{pmatrix}=
                    \begin{pmatrix}
                        0 & a \\
                        0 & c
                    \end{pmatrix} \\
                    \begin{pmatrix}
                        a & b \\
                        c & d
                    \end{pmatrix}
                    \begin{pmatrix}
                        0 & 0 \\
                        1 & 0
                    \end{pmatrix}=
                    \begin{pmatrix}
                        b & 0 \\
                        d & 0
                    \end{pmatrix} \\
                    \begin{pmatrix}
                        a & b \\
                        c & d
                    \end{pmatrix}
                    \begin{pmatrix}
                        0 & 0 \\
                        0 & 1
                    \end{pmatrix}=
                    \begin{pmatrix}
                        0 & b \\
                        0 & d
                    \end{pmatrix}
                \end{cases}
            \]
            \par Vậy ma trận của tự đồng cấu này đối với cơ sở chính tắc của $M(2\times 2,\mathbb{F})$ là:
            \[
                \begin{pmatrix}
                    a & 0 & b & 0 \\
                    0 & a & 0 & b \\
                    c & 0 & d & 0 \\
                    0 & c & 0 & d
                \end{pmatrix}
            \]
        \item
            \[
                \begin{cases}
                    \begin{pmatrix}
                        1 & 0 \\
                        0 & 0
                    \end{pmatrix}
                    \begin{pmatrix}
                        a & b \\
                        c & d
                    \end{pmatrix}=
                    \begin{pmatrix}
                        a & b \\
                        0 & 0
                    \end{pmatrix} \\
                    \begin{pmatrix}
                        0 & 1 \\
                        0 & 0
                    \end{pmatrix}
                    \begin{pmatrix}
                        a & b \\
                        c & d
                    \end{pmatrix}=
                    \begin{pmatrix}
                        c & d \\
                        0 & 0
                    \end{pmatrix} \\
                    \begin{pmatrix}
                        0 & 0 \\
                        1 & 0
                    \end{pmatrix}
                    \begin{pmatrix}
                        a & b \\
                        c & d
                    \end{pmatrix}=
                    \begin{pmatrix}
                        0 & 0 \\
                        a & b
                    \end{pmatrix} \\
                    \begin{pmatrix}
                        0 & 0 \\
                        0 & 1
                    \end{pmatrix}
                    \begin{pmatrix}
                        a & b \\
                        c & d
                    \end{pmatrix}=
                    \begin{pmatrix}
                        0 & 0 \\
                        c & d
                    \end{pmatrix}
                \end{cases}
            \]
            \par Vậy ma trận của tự đồng cấu này đối với cơ sở chính tắc của $M(2\times 2,\mathbb{F})$ là:
            \[
                \begin{pmatrix}
                    a & c & 0 & 0 \\
                    b & d & 0 & 0 \\
                    0 & 0 & a & c \\
                    0 & 0 & b & d
                \end{pmatrix}
            \]
    \end{enumerate}
\end{proof}

\begin{exercise}
    Chứng minh rằng đạo hàm là một tự đồng cấu của không gian vector các đa thức hệ số thực có bậc không vượt quá $n$. Tìm ma trận của tự đồng cấu đó với các cơ sở sau đây:
    \begin{enumerate}[label = (\alph*)]
        \item $(1, X, \ldots, X^{n})$.
        \item $(1, (X-c), \ldots, \frac{(X-c){}^{n}}{n!})$.
    \end{enumerate}
\end{exercise}

\begin{proof}
    \[
        \begin{split}
            a(X) = a_{0} + a_{1}X + \cdots + a_{n}X^{n} \\
            b(X) = b_{0} + b_{1}X + \cdots + b_{n}X^{n}
        \end{split}
    \]
    \begin{align*}
        \frac{d}{dX}(a(X) + b(X)) & = \frac{d}{dX}\left(\sum^{n}_{k=0}(a_{k}+b_{k})x^{k}\right) \\
                                  & = \sum^{n}_{k=1}k(a_{k}+b_{k})X^{k-1} \\
                                  & = \sum^{n}_{k=1}ka_{k}X^{k-1} + \sum^{n}_{k=1}b_{k}X^{k-1} \\
                                  & = \frac{d}{dX}a(X) + \frac{d}{dX}b(X).
    \end{align*}
    \begin{align*}
        \frac{d}{dX}(ta(X)) & = \frac{d}{dX}\sum^{n}_{k=0}ta_{k}X^{k} \\
                            & = t\frac{d}{dX}\sum^{n}_{k=1}ka_{k}X^{k-1} \\
                            & = t\frac{d}{dX}a(X).
    \end{align*}
    \par Bên cạnh đó
    \[
        \deg \frac{d}{dX}a(X) < \deg a(X).
    \]
    \par Do đó đạo hàm là một tự đồng cấu của không gian $\mathbb{R}[X]{}_{n}$.
    \begin{enumerate}[label = (\alph*)]
        \item Ma trận của đạo hàm đối với cơ sở $(1, X, \ldots, X^{n})$ là:
            \[
                \begin{pmatrix}
                    0 & 1 & 0 & 0 & \cdots & 0 \\
                    0 & 0 & 2 & 0 & \cdots & 0 \\
                    0 & 0 & 0 & 3 & \cdots & 0 \\
                    \vdots & \vdots & \vdots & \vdots & \ddots & \vdots \\
                    0 & 0 & 0 & 0 & \cdots & n \\
                    0 & 0 & 0 & 0 & \cdots & 0
                \end{pmatrix}
            \]
        \item Ma trận của đạo hàm đối với cơ sở $(1, (X-c), \ldots, \frac{(X-c){}^{n}}{n!})$ là:
            \[
                \begin{pmatrix}
                    0 & \frac{1}{0!} & 0 & 0 & \cdots & 0 \\
                    0 & 0 & \frac{1}{1!} & 0 & \cdots & 0 \\
                    0 & 0 & 0 & \frac{1}{2!} & \cdots & 0 \\
                    \vdots & \vdots & \vdots & \vdots & \ddots & \vdots \\
                    0 & 0 & 0 & 0 & \cdots & \frac{1}{(n-1)!} \\
                    0 & 0 & 0 & 0 & \cdots & 0
                \end{pmatrix}
            \]
    \end{enumerate}
\end{proof}

\begin{exercise}
    Ma trận của một tự đồng cấu đối với cơ sở $(e_{1},\ldots,e_{n})$ thay đổi thế nào nếu ta đổi chỗ các vector $e_{i}$ và $e_{j}$.
\end{exercise}

\begin{proof}[Lời giải]
    Đổi chỗ $f(e_{i})$, $f(e_{j})$ thì hai cột thứ $i$, $j$ của $M(f)$ đổi chỗ.
    \par Đổi chỗ $e_{i}$, $e_{j}$ thì hai hàng thứ $i$, $j$ đổi chỗ.
    \par Ma trận mới được dựng như sau:
    \begin{itemize}
        \item Đổi chỗ cột $i$ và $j$.
        \item Đổi chỗ hàng $i$ và $j$.
    \end{itemize}
    \par hoặc nói cách khác:
    \begin{itemize}
        \item Yếu tố hàng $i$ cột $i$ và hàng $j$ cột $i$ đổi chỗ.
        \item Yếu tố hàng $i$ cột $j$ và hàng $j$ cột $i$ đổi chỗ.
        \item Yếu tố hàng $i$ cột $k$ và hàng $j$ cột $k$ đổi chỗ $(k\not\in\{ i, j \})$.
        \item Yếu tố hàng $k$ cột $i$ và hàng $k$ cột $j$ đổi chỗ $(k\not\in\{ i, j \})$.
    \end{itemize}
\end{proof}

\begin{exercise}
    Tự đồng cấu $f$ có ma trận
    \[
        A =
        \begin{pmatrix}
            1 & 2 & 0  & 1 \\
            3 & 0 & -1 & 2 \\
            2 & 5 & 3  & 1 \\
            1 & 2 & 1  & 3
        \end{pmatrix}
    \]
    \par đối với cơ sở $(e_{1}, e_{2}, e_{3}, e_{4})$. Hãy tìm ma trận của $f$ đối với cơ sở $(e_{1}, e_{1} + e_{2}, e_{1} + e_{2} + e_{3}, e_{1} + e_{2} + e_{3} + e_{4})$.
\end{exercise}

\begin{proof}[Lời giải]
    Ma trận chuyển từ cơ sở $(e_{1}, e_{2}, e_{3}, e_{4})$ sang $(e_{1}, e_{1} + e_{2}, e_{1} + e_{2} + e_{3}, e_{1} + e_{2} + e_{3} + e_{4})$ là:
    \[
        C =
        \begin{pmatrix}
            1 & 1 & 1 & 1 \\
            0 & 1 & 1 & 1 \\
            0 & 0 & 1 & 1 \\
            0 & 0 & 0 & 1
        \end{pmatrix}
    \]
    \par Ma trận chuyển từ cơ sở $(e_{1}, e_{1} + e_{2}, e_{1} + e_{2} + e_{3}, e_{1} + e_{2} + e_{3} + e_{4}) = (e'_{1}, e'_{2}, e'_{3}, e'_{4})$ sang $(e_{1}, e_{2}, e_{3}, e_{4})$ là:
    \[
        C^{-1} =
        \begin{pmatrix}
            1 & -1 & 0  & 0  \\
            0 & 1  & -1 & 0  \\
            0 & 0  & 1  & -1 \\
            0 & 0  & 0  & 1
        \end{pmatrix}
    \]
    \begin{align*}
        f(e_{1}) & = 1e_{1} + 3e_{2} + 2e_{3} + 1e_{4} \\
                 & = 1e'_{1} + 3(e'_{2} - e'_{1}) + 2(e'_{3} - e'_{2}) + 1(e'_{4} - e'_{3}) \\
                 & = (-2)e'_{1} + 1e'_{2} + 1e'_{3} + 1e'_{4} \\
        f(e_{2}) & = 2e_{1} + 0e_{2} + 5e_{3} + 2e_{4} \\
                 & = 2e'_{1} + 0(e'_{2} - e'_{1}) + 5(e'_{3} - e'_{2}) + 2(e'_{4} - e'_{3}) \\
                 & = 2e'_{1} + (-5)e'_{2} + 3e'_{3} + 2e'_{4} \\
        f(e_{3}) & = 0e_{1} + (-1)e_{2} + 3e_{3} + 1e_{4} \\
                 & = 0e'_{1} + (-1)(e'_{2} - e'_{1}) + 3(e'_{3} - e'_{2}) + 1(e'_{4} - e'_{3}) \\
                 & = 1e'_{1} + (-4)e'_{2} + 2e'_{3} + 1e'_{4} \\
        f(e_{4}) & = 1e_{1} + 2e_{2} + 1e_{3} + 3e_{4} \\
                 & = 1e'_{1} + 2(e'_{2} - e'_{1}) + 1(e'_{3} - e'_{2}) + 3(e'_{4} - e'_{3}) \\
                 & = (-1)e'_{1} + 1e'_{2} + (-2)e'_{3} + 3e'_{4}
    \end{align*}
    \begin{align*}
        f(e'_{1}) & = f(e_{1}) \\
                  & = (-2)e'_{1} + 1e'_{2} + 1e'_{3} + 1e'_{4} \\
        f(e'_{2}) & = f(e_{1}) + f(e_{2}) \\
                  & = 0e'_{1} + (-4)e'_{2} + 4e'_{3} + 2e'_{4} \\
        f(e'_{3}) & = f(e_{1}) + f(e_{2}) + f(e_{3}) \\
                  & = 1e'_{1} + (-8)e'_{2} + 6e'_{3} + 4e'_{4} \\
        f(e'_{4}) & = f(e_{1}) + f(e_{2}) + f(e_{3}) + f(e_{4}) \\
                  & = 0e'_{1} + (-7)e'_{2} + 4e'_{3} + 7e'_{4}
    \end{align*}
    \par Vậy ma trận của tự đồng cấu $f$ đối với cơ sở $(e'_{1}, e'_{2}, e'_{3}, e'_{4})$ là:
    \[
        \begin{pmatrix}
            -2 & 0  & 1  & 0  \\
            1  & -4 & -8 & -7 \\
            1  & 4  & 6  & 4  \\
            1  & 2  & 4  & 7
        \end{pmatrix}
    \]
\end{proof}

\begin{exercise}
    Tự đồng cấu $\phi$ có ma trận
    \[
        \begin{pmatrix}
            15 & -11 & 5 \\
            20 & -15 & 8 \\
            8  & -7  & 6
        \end{pmatrix}
    \]
    \par đối với cơ sở $(e_{1}, e_{2}, e_{3})$. Hãy tìm ma trận của $\phi$ đối với cơ sở gồm các vector $\epsilon_{1} = 2e_{1} + 3e_{2} + e_{3}$, $\epsilon_{2} = 3e_{1} + 4e_{2} + e_{3}$, $\epsilon_{3} = e_{1} + 2e_{2} + 2e_{3}$.
\end{exercise}

\begin{proof}[Lời giải]
    \[
        \begin{cases}
            e_{1} & = (-6)\epsilon_{1} + 4\epsilon_{2} + 1\epsilon_{3} \\
            e_{2} & = 5\epsilon_{1} + (-3)\epsilon_{2} + (-1)\epsilon_{3} \\
            e_{3} & = (-2)\epsilon_{1} + 1\epsilon_{2} + 1\epsilon_{3}
        \end{cases}
    \]
    \begin{align*}
        f(e_{1}) & = 15e_{1} + 20e_{2} + 8e_{3} \\
                 & = (-6)\epsilon_{1} + 8\epsilon_{2} + 3\epsilon_{3} \\
        f(e_{2}) & = (-11)e_{1} + (-15)e_{2} + (-7)e_{3} \\
                 & = 5\epsilon_{1} + (-6)\epsilon_{2} + (-3)\epsilon_{3} \\
        f(e_{3}) & = 5e_{1} + 8e_{2} + 6e_{3} \\
                 & = (-2)\epsilon_{1} + 2\epsilon_{2} + 3\epsilon_{3}
    \end{align*}
    \begin{align*}
        f(\epsilon_{1}) & = 2f(e_{1}) + 3f(e_{2}) + f(e_{3}) \\
                        & = \epsilon_{1} \\
        f(\epsilon_{2}) & = 3f(e_{1}) + 4f(e_{2}) + f(e_{3}) \\
                        & = 2\epsilon_{2} \\
        f(\epsilon_{3}) & = f(e_{1}) + 2f(e_{2}) + 2f(e_{3}) \\
                        & = 3\epsilon_{3} \\
    \end{align*}
    \par Vậy ma trận của tự đồng cấu $\phi$ đối với cơ sở $(\epsilon_{1}, \epsilon_{2}, \epsilon_{3})$ là:
    \[
        \begin{pmatrix}
            1 & 0 & 0 \\
            0 & 2 & 0 \\
            0 & 0 & 3
        \end{pmatrix}
    \]
\end{proof}

\begin{exercise}
    Đồng cấu $\phi: \mathbb{C}_{3}\to\mathbb{C}_{3}$ có ma trận
    \[
        \begin{pmatrix}
            1  & -18 & 15 \\
            -1 & -22 & 20 \\
            1  & -25 & 22
        \end{pmatrix}
    \]
    \par đối với cơ sở gồm các vector $\alpha_{1} = (8, -6, 7)$, $\alpha_{2} = (-16, 7, -13)$, $\alpha_{3} = (9, -3, 7)$. Tìm ma trận của $\phi$ đối với cơ sở gồm các vector
    \[
        \beta_{1} = (1, -2, 1),\quad\beta_{2} = (3, -1, 2),\quad\beta_{3} = (2, 1, 2).
    \]
\end{exercise}

\begin{proof}[Lời giải]
    \[
        \begin{cases}
            \beta_{1} & = 1\alpha_{1} + 1\alpha_{2} + 1\alpha_{3} \\
            \beta_{2} & = 1\alpha_{1} + 2\alpha_{2} + 3\alpha_{3} \\
            \beta_{3} & = (-3)\alpha_{1} + (-5)\alpha_{2} + (-6)\alpha_{3} \\
        \end{cases}
    \]
    \[
        \begin{cases}
            \alpha_{1} & = 3\beta_{1} + 1\beta_{2} + 1\beta_{3} \\
            \alpha_{2} & = (-3)\beta_{1} + (-3)\beta_{2} + (-2)\beta_{3} \\
            \alpha_{3} & = 1\beta_{1} + 2\beta_{2} + 1\beta_{3}
        \end{cases}
    \]
    \begin{align*}
        \phi(\alpha_{1}) & = 1\alpha_{1} + (-1)\alpha_{2} + 1\alpha_{3} \\
                         & = 7\beta_{1} + 6\beta_{2} + 4\beta_{3} \\
        \phi(\alpha_{2}) & = (-18)\alpha_{1} + (-22)\alpha_{2} + (-25)\alpha_{3} \\
                         & = (-13)\beta_{1} + (-2)\beta_{2} + 1\beta_{3} \\
        \phi(\alpha_{3}) & = 15\alpha_{1} + 20\alpha_{2} + 22\alpha_{3} \\
                         & = 7\beta_{1} + (-1)\beta_{2} + (-3)\beta_{3}
    \end{align*}
    \begin{align*}
        \phi(\beta_{1}) & = \phi(\alpha_{1}) + \phi(\alpha_{2}) + \phi(\alpha_{3}) \\
                        & = 1\beta_{1} + 3\beta_{2} + 2\beta_{3} \\
        \phi(\beta_{2}) & = \phi(\alpha_{1}) + 2\phi(\alpha_{2}) + 3\phi(\alpha_{3}) \\
                        & = 2\beta_{1} + (-1)\beta_{2} + (-3)\beta_{3} \\
        \phi(\beta_{3}) & = (-3)\phi(\alpha_{1}) + (-5)\phi(\alpha_{2}) + (-6)\phi(\alpha_{3}) \\
                        & = 2\beta_{1} + (-2)\beta_{2} + 1\beta_{3}
    \end{align*}
    \par Vậy ma trận của đồng cấu $\phi$ đối với cơ sở $(\beta_{1}, \beta_{2}, \beta_{3})$ là:
    \[
        \begin{pmatrix}
            1 & 2 & 2 \\
            3 & -1 & -2 \\
            2 & -3 & 1
        \end{pmatrix}
    \]
\end{proof}

\begin{exercise}
    Chứng minh rằng các ma trận của một tự đồng cấu đối với hai cơ sở của không gian là trùng nhau nếu và chỉ nếu ma trận chuyển giữa hai cơ sở đó giao hoán với ma trận của đồng cấu đã cho đối với mỗi cơ sở nói trên.
\end{exercise}

\begin{proof}
    Gọi $A$ là ma trận của tự đồng cấu đối với cơ sở thứ nhất, $B$ là ma trận của tự đồng cấu đối với cơ sở thứ hai và $C$ là ma trận chuyển từ cơ sở thứ nhất sang cơ sở thứ hai.
    \par Theo định lý 2.13, $B = C^{-1}AC$.
    \par $(\Rightarrow)$
    \par $AC = CA, BC = CB$.
    \par Suy ra $C^{-1}AC = A$. Mà $B = C^{-1}AC$ nên $A = B$.
    \par $(\Leftarrow)$
    \par $A = B$.
    \par $B = C^{-1}AC$ suy ra $CB = AC$.
    \par Mà $A = B$ nên $CA = AC$ và $CB = BC$. Điều này có nghĩa là $C$ giao hoán với cả $A$ và $B$.
\end{proof}

\begin{exercise}
    Tự đồng cấu $\phi\in End(\mathbb{R}_{2})$ có ma trận $\begin{pmatrix}3 & 5 \\ 4 & 3\end{pmatrix}$ đối với cơ sở gồm $\alpha_{1} = (1, 2)$, $\alpha_{2} = (2, 3)$, và tự đồng cấu $\psi\in End(\mathbb{R}_{2})$ có ma trận $\begin{pmatrix}4 & 6 \\ 6 & 9 \end{pmatrix}$ đối với cơ sở gồm $\beta_{1} = (3, 1)$, $\beta_{2} = (4, 2)$. Tìm ma trận của $\phi + \psi$ đối với cơ sở $(\beta_{1}, \beta_{2})$.
\end{exercise}

\begin{proof}[Lời giải]
    \par Ma trận chuyển từ cơ sở $(\alpha_{1}, \alpha_{2})$ sang $(\beta_{1}, \beta_{2})$ là:
    \[
        \begin{pmatrix}
            -7 & -8 \\
            5  & 6
        \end{pmatrix}
    \]
    \par Ma trận chuyển từ cơ sở $(\beta_{1}, \beta_{2})$ sang $(\alpha_{1}, \alpha_{2})$ là:
    \[
        \begin{pmatrix}
            -3 & -4 \\
            \frac{5}{2} & \frac{7}{2}
        \end{pmatrix}
    \]
    \par Ma trận của tự đồng cấu $\phi$ đối với cơ sở $(\beta_{1}, \beta_{2})$ là:
    \[
        \begin{pmatrix}
            -3 & -4 \\
            \frac{5}{2} & \frac{7}{2}
        \end{pmatrix}
        \begin{pmatrix}
            3 & 5 \\
            4 & 3
        \end{pmatrix}
        \begin{pmatrix}
            -7 & -8 \\
            5 & 6
        \end{pmatrix}
        =
        \begin{pmatrix}
            40 & 38 \\
            -\frac{71}{2} & -34
        \end{pmatrix}
    \]
    \par Vậy ma trận của tự đồng cấu $\phi + \psi$ đối với cơ sở $(\beta_{1}, \beta_{2})$ là:
    \[
        \begin{pmatrix}
            40 & 38 \\
            -\frac{71}{2} & -34
        \end{pmatrix}
        +
        \begin{pmatrix}
            4 & 6 \\
            6 & 9
        \end{pmatrix}
        =
        \begin{pmatrix}
            44 & 44 \\
            -\frac{59}{2} & -25
        \end{pmatrix}
    \]
\end{proof}

\begin{exercise}
    Tự đồng cấu $\phi\in End(\mathbb{R}_{2})$ có ma trận $\begin{pmatrix} 2 & -1 \\ 5 & -3 \end{pmatrix}$ đối với cơ sở $\alpha_{1} = (-3, 7)$, $\alpha_{2} = (1, -2)$, và tự đồng cấu $\psi\in End(\mathbb{R}_{2})$ có ma trận $\begin{pmatrix} 1 & 3 \\ 2 & 7 \end{pmatrix}$ đối với cơ sở gồm $\beta_{1} = (6, -7)$, $\beta_{2} = (-5, 6)$. Tìm ma trận của $\phi\psi$ đối với cơ sở chính tắc của $\mathbb{R}_{2}$.
\end{exercise}

\begin{proof}[Lời giải]
    Ma trận chuyển từ cơ sở $(\alpha_{1}, \alpha_{2})$ sang cơ sở chính tắc là:
    \[
        \begin{pmatrix}
            2 & 1 \\
            7 & 3
        \end{pmatrix}
    \]
    \par Ma trận của tự đồng cấu $\phi$ đối với cơ sở chính tắc là:
    \[
        \begin{pmatrix}
            -3 & 1 \\
            7  & -2
        \end{pmatrix}
        \begin{pmatrix}
            2 & -1 \\
            5 & -3
        \end{pmatrix}
        \begin{pmatrix}
            2 & 1 \\
            7 & 3
        \end{pmatrix}=
        \begin{pmatrix}
            -2 & -1 \\
            1  & 1
        \end{pmatrix}
    \]
    \par Ma trận chuyển từ cơ sở $(\beta_{1}, \beta_{2})$ sang cơ sở chính tắc là:
    \[
        \begin{pmatrix}
            6 & 7 \\
            5 & 6
        \end{pmatrix}
    \]
    \par Ma trận của tự đồng cấu $\psi$ đối với cơ sở chính tắc là:
    \[
        \begin{pmatrix}
            6 & -7 \\
            -5 & 6
        \end{pmatrix}
        \begin{pmatrix}
            1 & 3 \\
            2 & 7
        \end{pmatrix}
        \begin{pmatrix}
            6 & 7 \\
            5 & 6
        \end{pmatrix}=
        \begin{pmatrix}
            -203 & -242 \\
            177 & 211
        \end{pmatrix}
    \]
    \par Ma trận của $\phi\psi$ đối với cơ sở chính tắc là:
    \[
        \begin{pmatrix}
            -2 & -1 \\
            1 & 1
        \end{pmatrix}
        \begin{pmatrix}
            -203 & -242 \\
            177 & 211
        \end{pmatrix}=
        \begin{pmatrix}
            229 & 273 \\
            -26 & -31
        \end{pmatrix}
    \]
\end{proof}

\end{document}
