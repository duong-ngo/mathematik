\documentclass[class=linearalgebra,crop=false]{standalone}

\setcounter{exercise}{0}

\begin{document}

\chapter{Không gian vector}

\begin{exercise}Xét xem các tập hợp sau đây có lập thành $\mathbb{F}$--không gian vector hay không đối với các phép toán thông thường (được định nghĩa theo từng thành phần):
    \begin{enumerate}[itemsep=0pt,topsep=0pt,label = (\alph*)]
        \item Tập hợp tất cả các dãy $(x_{1},\ldots,x_{n})\in\mathbb{F}_{n}$ thỏa mãn điều kiện $x_{1} + \cdots + x_{n} = 0$.
        \item Tập hợp tất cả các dãy $(x_{1},\ldots,x_{n})\in\mathbb{F}_{n}$ thỏa mãn điều kiện $x_{1} + \cdots + x_{n} = 1$.
        \item Tập hợp tất cả các dãy $(x_{1},\ldots,x_{n})\in\mathbb{F}_{n}$ thỏa mãn điều kiện $x_{1} = x_{n} = -1$.
        \item Tập hợp tất cả các dãy $(x_{1},\ldots,x_{n})\in\mathbb{F}_{n}$ thỏa mãn điều kiện $x_{1} = x_{3} = x_{5} = \cdots$, $x_{2} = x_{4} = x_{6} = \cdots$.
        \item Tập hợp các ma trận vuông $(a_{ij}){}_{n\times n}$ \textit{đối xứng} cỡ $n$, nghĩa là các ma trận thỏa mãn $a_{ij} = a_{ji}$, với $1\le i, n\le n$.
    \end{enumerate}
\end{exercise}

\begin{proof}
    \begin{enumerate}[label = (\alph*)]
        \item Đây đúng là một không gian vector với các phép toán thông thường.
        \item Đây không phải một không gian vector với các phép toán thông thường vì phép cộng không đóng.
        \item Đây không phải một không gian vector với các phép toán thông thường vì phép cộng không đóng.
        \item Đây đúng là một không gian vector với các phép toán thông thường.
        \item Đây đúng là một không gian vector với các phép toán thông thường.
    \end{enumerate}
\end{proof}

\begin{exercise}Tập hợp tất cả các dãy $(x_{1},\ldots, x_{n})\in\mathbb{R}_{n}$ với tất cả các thành phần $x_{1}, \ldots, x_{n}$ đều nguyên có lập thành một $\mathbb{R}$--không gian vector hay không?
\end{exercise}

\begin{proof}Không. Bởi phép nhân vector với vô hướng không đóng trong tập này.
\end{proof}

\begin{exercise}Với các phép toán thông thường, $\mathbb{Q}$ có là một $\mathbb{R}$--không gian vector hay không? $\mathbb{R}$ có là một $\mathbb{C}$ không gian vector hay không?
\end{exercise}

\begin{proof}$\mathbb{Q}$ không phải là một $\mathbb{R}$--không gian vector vì phép nhân một số hữu tỷ với một số thực có thể là một số vô tỷ. Ví dụ: $\sqrt{2}\cdot 1 = \sqrt{2}\not\in\mathbb{Q}$.
    \par $\mathbb{R}$ không phải là một $\mathbb{C}$--không gian vector vì phép nhân một số thực với một số phức có thể là một số phức nhưng lại không phải số thực. Ví dụ $\iota\cdot 1 = \iota\not\in\mathbb{R}$.
\end{proof}

\begin{exercise}Chứng minh rằng nhóm $\mathbb{Z}$ không đẳng cấu với nhóm cộng của bất kỳ một không gian vector trên bất kỳ trường nào.
\end{exercise}

\begin{proof}
\end{proof}

\end{document}
