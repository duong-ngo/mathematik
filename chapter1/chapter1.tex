\documentclass[class=linearalgebra,crop=false]{standalone}

\setcounter{exercise}{0}
\newtheorem{lemma}{Bổ đề}
\setcounter{lemma}{0}

\begin{document}

\chapter{Không gian vector}

\begin{exercise}Xét xem các tập hợp sau đây có lập thành $\mathbb{F}$--không gian vector hay không đối với các phép toán thông thường (được định nghĩa theo từng thành phần):
    \begin{enumerate}[itemsep=0pt,topsep=0pt,label = (\alph*)]
        \item Tập hợp tất cả các dãy $(x_{1},\ldots,x_{n})\in\mathbb{F}_{n}$ thỏa mãn điều kiện $x_{1} + \cdots + x_{n} = 0$.
        \item Tập hợp tất cả các dãy $(x_{1},\ldots,x_{n})\in\mathbb{F}_{n}$ thỏa mãn điều kiện $x_{1} + \cdots + x_{n} = 1$.
        \item Tập hợp tất cả các dãy $(x_{1},\ldots,x_{n})\in\mathbb{F}_{n}$ thỏa mãn điều kiện $x_{1} = x_{n} = -1$.
        \item Tập hợp tất cả các dãy $(x_{1},\ldots,x_{n})\in\mathbb{F}_{n}$ thỏa mãn điều kiện $x_{1} = x_{3} = x_{5} = \cdots$, $x_{2} = x_{4} = x_{6} = \cdots$.
        \item Tập hợp các ma trận vuông $(a_{ij}){}_{n\times n}$ \textit{đối xứng} cỡ $n$, nghĩa là các ma trận thỏa mãn $a_{ij} = a_{ji}$, với $1\le i, n\le n$.
    \end{enumerate}
\end{exercise}

\begin{proof}
    \begin{enumerate}[label = (\alph*)]
        \item Đây đúng là một không gian vector với các phép toán thông thường.
        \item Đây không phải một không gian vector với các phép toán thông thường vì phép cộng không đóng.
        \item Đây không phải một không gian vector với các phép toán thông thường vì phép cộng không đóng.
        \item Đây đúng là một không gian vector với các phép toán thông thường.
        \item Đây đúng là một không gian vector với các phép toán thông thường.
    \end{enumerate}
\end{proof}

\begin{exercise}Tập hợp tất cả các dãy $(x_{1},\ldots, x_{n})\in\mathbb{R}_{n}$ với tất cả các thành phần $x_{1}, \ldots, x_{n}$ đều nguyên có lập thành một $\mathbb{R}$--không gian vector hay không?
\end{exercise}

\begin{proof}Không. Bởi phép nhân vector với vô hướng không đóng trong tập này.
\end{proof}

\begin{exercise}Với các phép toán thông thường, $\mathbb{Q}$ có là một $\mathbb{R}$--không gian vector hay không? $\mathbb{R}$ có là một $\mathbb{C}$ không gian vector hay không?
\end{exercise}

\begin{proof}$\mathbb{Q}$ không phải là một $\mathbb{R}$--không gian vector vì phép nhân một số hữu tỷ với một số thực có thể là một số vô tỷ. Ví dụ: $\sqrt{2}\cdot 1 = \sqrt{2}\not\in\mathbb{Q}$.
    \par $\mathbb{R}$ không phải là một $\mathbb{C}$--không gian vector vì phép nhân một số thực với một số phức có thể là một số phức nhưng lại không phải số thực. Ví dụ $\iota\cdot 1 = \iota\not\in\mathbb{R}$.
\end{proof}

\begin{exercise}Chứng minh rằng nhóm $\mathbb{Z}$ không đẳng cấu với nhóm cộng của bất kỳ một không gian vector trên bất kỳ trường nào.
\end{exercise}

\begin{lemma}Xét không gian vector $(V, \mathbb{F})$. $a\cdot\alpha = 0$ khi và chỉ khi $a = 0$ hoặc $\alpha$ là vector-không.
\end{lemma}

\begin{proof}[Chứng minh bổ đề]$(\Rightarrow)$ $a = 0$ thì:
    \[ 0\cdot\alpha + 0 = 0\cdot\alpha = (0 + 0)\cdot\alpha = 0\cdot\alpha + 0\cdot\alpha \]
    \par Theo luật giản ước, $0\cdot\alpha = 0$.
    \par Nếu $\alpha = 0$ thì
    \[ a\cdot 0 + 0 = a\cdot 0 = a\cdot (0 + 0) = a\cdot 0 + a\cdot 0 \]
    \par Theo luật giản ước, $a\cdot 0 = 0$.
    \bigskip
    \par $(\Leftarrow)$ $a\cdot\alpha = 0$.
    \par Giả sử $a$ khác không. Khi đó $a$ khả nghịch.
    \[ \Rightarrow (a^{-1}a)\alpha = 0 \Rightarrow \alpha = 0\]
    \par Chứng minh hoàn tất.
\end{proof}

\begin{proof}Giả sử phản chứng, tồn tại một không gian vector $V$ trên trường $\mathbb{F}$ đẳng cấu với $\mathbb{Z}$.
    \par Khi đó, tồn tại một đồng cấu $\varphi: \mathbb{Z}\rightarrow V$ sao cho:
    \[ \varphi(x) + \varphi(y) = \varphi(x + y)\quad\forall x, y\in\mathbb{Z} \]
    \par và đồng cấu này là song ánh -- đẳng cấu nhóm.
    \par Một đồng cấu nhóm biến phần tử trung lập của nhóm này thành phần tử trung lập của nhóm kia, do đó $\varphi(0)$ là vector-không trên $V$.
    \par $\varphi$ là đẳng cấu nên các phần tử của $V$ liệt kê được, bao gồm:
    \[ V = \{ \cdots, \varphi(-2), \varphi(-1), \varphi(0), \varphi(1), \varphi(2), \cdots \} \]
    \par $\mathbb{F}$ có thể là một trường chỉ gồm phần tử không với các phép toán được định nghĩa như sau:
    \begin{align*}
        0 + 0 = 0 \\
        0\cdot 0 = 0
    \end{align*}
    \par Trong trường này, phần tử đơn vị trùng với phần tử không, do đó mọi vector trên trường này đều là vector-không. Điều này mâu thuẫn với giả sử vì $\mathbb{Z}$ vô hạn đếm được, trong khi $V$ chỉ có đúng một phần tử.
    \par Do vậy, trường $\mathbb{F}$ phải có phần tử khác không.
    \par $n$ là một số nguyên dương.
    \begin{align*}
        \varphi(n)&=\underbrace{\varphi(1) + \cdots + \varphi(1)}_{n} \\
                  &=(\underbrace{1 + \cdots + 1}_{n})\varphi(1) =a_{n}\varphi(1) \\
        \varphi(-n)&=-a_{n}\varphi(1)
    \end{align*}
    \par Như vậy $a_{1} = 1$, quy ước $a_{0} = 0$. Các vector của $V$ được tương ứng một-một với $a_{n}$ -- đây là một song ánh.
    \par Nếu $\text{Char}(\mathbb{F}) = p$ thì $a_{n + p} = a_{n}$, mâu thuẫn với song ánh. Do đó $\text{Char}(\mathbb{F}) = 0$.
    \par $a_{m}$ là một phần tử khác không và khác đơn vị, phần tử đối của đơn vị ($m > 0$). Như vậy $a_{m}$ khả nghịch. Giả sử phần tử khả nghịch là $a_{k}$.
    \par Nếu $a_{k}a_{m} = a_{km} = 1 \Leftrightarrow km = 1 \Leftrightarrow k = m = 1 \vee k = m = -1$, mâu thuẫn với phép chọn $a_{m}$.
    \par Vậy giả sử phản chứng là sai.
    \par Do đó ta kết luận $\mathbb{Z}$ không đẳng cấu với bất cứ không gian vector nào, trên bất cứ trường nào.
\end{proof}

\begin{exercise}Chứng minh rằng nhóm abel $A$ đối với phép cộng $+$ có thể trở thành một không gian vector trên trường $\mathbb{F}_{p}$ nếu và chỉ nếu
    \[ px = \underbrace{x + x + \cdots + x}_{p} = 0,\quad \forall x\in A. \]
\end{exercise}

\begin{proof}
\end{proof}

\begin{exercise}Xét xem các vector sau đây độc lập hay phụ thuộc tuyến tính trong $\mathbb{R}_{4}$:
    \begin{enumerate}[label = (\alph*)]
        \item $e_{1} = (-1, -2, 1, 2)$, $e_{2} = (0, -1, 2, 3)$, $e_{3} = (1, 4, 1, 2)$, $e_{4} = (-1, 0, 1, 3)$.
        \item $\alpha_{1} = (-1, 1, 0, 1)$, $\alpha_{2} = (1, 0, 1, 1)$, $\alpha_{3} = (-3, 1, -2, -1)$.
    \end{enumerate}
\end{exercise}

\begin{proof}
    \begin{enumerate}[label = (\alph*)]
        \item Xét ràng buộc tuyến tính $x_{1}e_{1} + x_{2}e_{2} + x_{3}e_{3} + x_{4}e_{4} = (0,0,0,0)$. Để tìm $x_{1}, x_{2}, x_{3}, x_{4}$, ta giải hệ phương trình thuần nhất sau:
            \begin{align*}
                &\begin{cases}
                    (-1)x_{1} + 0x_{2} + 1x_{3} + (-1)x_{4} = 0 \\
                    (-2)x_{1} + (-1)x_{2} + 4x_{3} + 0x_{4} = 0 \\
                    1x_{1} + 2x_{2} + 1x_{3} + 1x_{4} = 0 \\
                    2x_{1} + 3x_{2} + 2x_{3} + 3x_{4} = 0
                \end{cases} \\
                \Leftrightarrow\quad&
                \begin{cases}
                    0x_{1} + 3x_{2} + 4x_{3} + 1x_{4} = 0 \\
                    0x_{1} + 2x_{2} + 6x_{3} + 3x_{4} = 0 \\
                    0x_{1} + x_{2} + 0x_{3} + (-1)x_{4} = 0 \\
                    2x_{1} + 3x_{2} + 2x_{3} + 3x_{4} = 0
                \end{cases} \\
                \Leftrightarrow\quad&
                \begin{cases}
                    0x_{1} + 0x_{2} + 4x_{3} + 1x_{4} = 0 \\
                    0x_{1} + 0x_{2} + 6x_{3} + 5x_{4} = 0 \\
                    0x_{1} + x_{2} + 0x_{3} + (-1)x_{4} = 0 \\
                    2x_{1} + 3x_{2} + 2x_{3} + 3x_{4} = 0
                \end{cases} \\
                \Leftrightarrow\quad&
                \begin{cases}
                    0x_{1} + 0x_{2} + 12x_{3} + 3x_{4} = 0 \\
                    0x_{1} + 0x_{2} + 12x_{3} + 10x_{4} = 0 \\
                    0x_{1} + x_{2} + 0x_{3} + (-1)x_{4} = 0 \\
                    2x_{1} + 3x_{2} + 2x_{3} + 3x_{4} = 0
                \end{cases} \\
                \Leftrightarrow\quad&
                \begin{cases}
                    0x_{1} + 0x_{2} + 0x_{3} + (-7)x_{4} = 0 \\
                    0x_{1} + 0x_{2} + 12x_{3} + 10x_{4} = 0 \\
                    0x_{1} + x_{2} + 0x_{3} + (-1)x_{4} = 0 \\
                    2x_{1} + 3x_{2} + 2x_{3} + 3x_{4} = 0
                \end{cases}
            \end{align*}
            \par Hệ phương trình này chỉ có nghiệm tầm thường $(x_{1}, x_{2}, x_{3}, x_{4}) = (0, 0, 0, 0)$, kéo theo ràng buộc tuyến tính tầm thường. Do đó hệ độc lập tuyến tính.
        \item Xét ràng buộc tuyến tính $x_{1}\alpha_{1} + x_{2}\alpha_{2} + x_{3}\alpha_{3} = (0, 0, 0)$. Để tìm $x_{1}, x_{2}, x_{3}$, ta giải hệ phương trình thuần nhất sau:
            \begin{align*}
                &\begin{cases}
                    (-1)x_{1} + 1x_{2} + (-3)x_{3} = 0 \\
                    1x_{1} + 0x_{2} + 1x_{3} = 0 \\
                    0x_{1} + 1x_{2} + (-2)x_{3} = 0 \\
                    1x_{1} + 1x_{2} + (-1)x_{3} = 0
                \end{cases} \\
                \Leftrightarrow\quad&
                \begin{cases}
                    0x_{1} + 2x_{2} + (-4)x_{3} = 0 \\
                    0x_{1} + (-1)x_{2} + 2x_{3} = 0 \\
                    0x_{1} + 1x_{2} + (-2)x_{3} = 0 \\
                    1x_{1} + 1x_{2} + (-1)x_{3} = 0
                \end{cases} \\
                \Leftrightarrow\quad&
                \begin{cases}
                    0x_{1} + 1x_{2} + (-2)x_{3} = 0 \\
                    1x_{1} + 1x_{2} + (-1)x_{3} = 0
                \end{cases}
            \end{align*}
            \par Hệ phương trình này có nghiệm không tầm thường $(x_{1}, x_{2}, x_{3}) = (0, 2, 1)$. Do đó hệ phụ thuộc tuyến tính.
    \end{enumerate}
\end{proof}

\begin{exercise}Chứng minh rằng hai hệ vector sau đây là các cơ sở của $\mathbb{C}_{3}$. Tìm ma trận chuyển từ cơ sở thứ nhất sang cơ sở thứ hai:
    \par $e_{1} = (1, 2, 1), e_{2} = (2, 3, 3), e_{3} = (3, 7, 1)$;
    \par $e'_{1} = (3, 1, 4), e'_{2} = (5, 2, 1), e'_{3} = (1, 1, -6)$.
\end{exercise}

\begin{proof}Xét ràng buộc tuyến tính $x_{1}e_{1} + x_{2}e_{2} + x_{3}e_{3} = (0, 0, 0)$.
    \begin{align*}
        &\begin{cases}
            1x_{1} + 2x_{2} + 3x_{3} = 0 \\
            2x_{1} + 3x_{2} + 7x_{3} = 0 \\
            1x_{1} + 3x_{2} + 1x_{3} = 0
        \end{cases} \\
        \Leftrightarrow\quad&
        \begin{cases}
            1x_{1} + 2x_{2} + 3x_{3} = 0 \\
            0x_{1} + (-1)x_{2} + 1x_{3} = 0 \\
            0x_{1} + 1x_{2} + (-2)x_{3} = 0
        \end{cases} \\
        \Leftrightarrow\quad&
        \begin{cases}
            1x_{1} + 2x_{2} + 3x_{3} = 0 \\
            0x_{1} + (-1)x_{2} + 1x_{3} = 0 \\
            0x_{1} + 0x_{2} + (-1)x_{3} = 0
        \end{cases}
    \end{align*}
    \par Hệ phương trình trên chỉ có nghiệm tầm thường $(x_{1}, x_{2}, x_{3}) = (0, 0, 0)$ do đó hệ $e_{1}, e_{2}, e_{3}$ độc lập tuyến tính và cực đại (vì $\dim\mathbb{C}_{3} = 3$) nên hệ này cũng là một cơ sở của $\mathbb{C}_{3}$.

    \bigskip
    \par Xét ràng buộc tuyến tính $x_{1}e'_{1} + x_{2}e'_{2} + x_{3}e'_{3} = (0, 0, 0)$.
    \begin{align*}
        &\begin{cases}
            3x_{1} + 5x_{2} + 1x_{3} = 0 \\
            1x_{1} + 2x_{2} + 1x_{3} = 0 \\
            4x_{1} + 1x_{2} + (-6)x_{3} = 0
        \end{cases} \\
        \Leftrightarrow\quad&
        \begin{cases}
            3x_{1} + 5x_{2} + 1x_{3} = 0 \\
            0x_{1} + 1x_{2} + 2x_{3} = 0 \\
            0x_{1} + (-6)x_{2} + (-8)x_{3} = 0
        \end{cases} \\
        \Leftrightarrow\quad&
        \begin{cases}
            3x_{1} + 5x_{2} + 1x_{3} = 0 \\
            0x_{1} + 1x_{2} + 2x_{3} = 0 \\
            0x_{1} + 0x_{2} + 4x_{3} = 0
        \end{cases}
    \end{align*}
    \par Hệ phương trình trên chỉ có nghiệm tầm thường $(x_{1}, x_{2}, x_{3}) = (0, 0, 0)$ do đó hệ $e_{1}, e_{2}, e_{3}$ độc lập tuyến tính và cực đại (vì $\dim\mathbb{C}_{3} = 3$) nên hệ này cũng là một cơ sở của $\mathbb{C}_{3}$.

    \par $(e_{1}, e_{2}, e_{3})$ là một cơ sở của $\mathbb{C}_{3}$ nên vector $(a_{1}, a_{2}, a_{3})$ biểu thị tuyến tính được duy nhất theo cơ sở này.
    \par Bằng việc giải hệ phương trình sau:
    \[
        \begin{cases}
            1x_{1} + 2x_{2} + 3x_{3} = a_{1} \\
            2x_{1} + 3x_{2} + 7x_{3} = a_{2} \\
            1x_{1} + 3x_{2} + 1x_{3} = a_{3}
        \end{cases}
    \]
    \par ta thu được nghiệm:
    \[
        (x_{1}, x_{2}, x_{3}) = (-18a_{1} + 7a_{2} + 5a_{3}, 5a_{1} - 2a_{2} - a_{3}, 3a_{1} - a_{2} - a_{3})
    \]
    \par Thay số, ta biểu diễn được $(e'_{1}, e'_{2}, e'_{3})$ qua $(e_{1}, e_{2}, e_{3})$ như sau:
    \[
        \begin{cases}
            e'_{1} = (-27)e_{1} + 9e_{2} + 4e_{3} \\
            e'_{2} = (-71)e_{1} + 20e_{2} + 12e_{3} \\
            e'_{3} = (-41)e_{1} + 9e_{2} + 8e_{3}
        \end{cases}
    \]
    \par Như vậy, ma trận chuyển cơ sở $(e_{1}, e_{2}, e_{3}) \rightarrow (e'_{1}, e'_{2}, e'_{3})$ là:
    \[
        \begin{pmatrix}
            -27 & 9 & 4 \\
            -71 & 20 & 12 \\
            -41 & 9 & 8
        \end{pmatrix}
    \]
\end{proof}

\begin{exercise}Chứng minh rằng hai hệ vector sau đây là các cơ sở của $\mathbb{C}_{4}$. Tìm mối liên hệ giữa tọa độ của cùng một vector trong hai cơ sở đó:
    \par $e_{1} = (1, 1, 1, 1), e_{2} = (1, 2, 1, 1), e_{3} = (1, 1, 2, 1), e_{4} = (1, 3, 2, 3)$;
    \par $e'_{1} = (1, 0, 3, 3), e'_{2} = (2, 3, 5, 4), e'_{3} = (2, 2, 5, 4), e'_{4} = (2, 3, 4, 4)$.
\end{exercise}

\begin{proof}Xét ràng buộc tuyến tính $x_{1}e_{1} + x_{2}e_{2} + x_{3}e_{3} + x_{4}e_{4} = (0,0,0,0)$.
    \begin{align*}
        &\begin{cases}
            1x_{1} + 1x_{2} + 1x_{3} + 1x_{4} = 0 \\
            1x_{1} + 2x_{2} + 1x_{3} + 3x_{4} = 0 \\
            1x_{1} + 1x_{2} + 2x_{3} + 2x_{4} = 0 \\
            1x_{1} + 1x_{2} + 1x_{3} + 3x_{4} = 0
        \end{cases} \\
        \Leftrightarrow\quad&
        \begin{cases}
            1x_{1} + 1x_{2} + 1x_{3} + 1x_{4} = 0 \\
            0x_{1} + 1x_{2} + 0x_{3} + 2x_{4} = 0 \\
            0x_{1} + 0x_{2} + 1x_{3} + 1x_{4} = 0 \\
            0x_{1} + 0x_{2} + 0x_{3} + 2x_{4} = 0
        \end{cases}
    \end{align*}

    \par Hệ phương trình trên chỉ có nghiệm tầm thường $(x_{1}, x_{2}, x_{3}, x_{4}) = (0, 0, 0, 0)$.
    \par Do đó hệ vector $e_{1}, e_{2}, e_{3}, e_{4}$ độc lập tuyến tính.
    \par Số chiều của không gian vector $\mathbb{C}_{4}$ bằng 4 nên hệ vector $e_{1}, e_{2}, e_{3}, e_{4}$ độc lập tuyến tính cực đại, nên cũng là cơ sở của $\mathbb{C}_{4}$.

    \bigskip
    \par Xét ràng buộc tuyến tính $x_{1}e'_{1} + x_{2}e'_{2} + x_{3}e'_{3} + x_{4}e'_{4} = (0,0,0,0)$.
    \begin{align*}
        &\begin{cases}
            1x_{1} + 2x_{2} + 2x_{3} + 2x_{4} = 0 \\
            0x_{1} + 3x_{2} + 2x_{3} + 3x_{4} = 0 \\
            3x_{1} + 5x_{2} + 5x_{3} + 4x_{4} = 0 \\
            3x_{1} + 4x_{2} + 4x_{3} + 4x_{4} = 0
        \end{cases} \\
        \Leftrightarrow\quad&
        \begin{cases}
            1x_{1} + 2x_{2} + 2x_{3} + 2x_{4} = 0 \\
            0x_{1} + 3x_{2} + 2x_{3} + 3x_{4} = 0 \\
            0x_{1} + (-1)x_{2} + (-1)x_{3} + (-2)x_{4} = 0 \\
            0x_{1} + (-2)x_{2} + (-2)x_{3} + (-2)x_{4} = 0
        \end{cases} \\
        \Leftrightarrow\quad&
        \begin{cases}
            1x_{1} + 2x_{2} + 2x_{3} + 2x_{4} = 0 \\
            0x_{1} + 3x_{2} + 2x_{3} + 3x_{4} = 0 \\
            0x_{1} + 0x_{2} + (-1)x_{3} + (-3)x_{4} = 0 \\
            0x_{1} + 0x_{2} + 0x_{3} + 2x_{4} = 0
        \end{cases}
    \end{align*}

    \par Hệ phương trình trên chỉ có nghiệm tầm thường $(x_{1}, x_{2}, x_{3}, x_{4}) = (0, 0, 0, 0)$.
    \par Do đó hệ vector $e'_{1}, e'_{2}, e'_{3}, e'_{4}$ độc lập tuyến tính.
    \par Số chiều của không gian vector $\mathbb{C}_{4}$ bằng 4 nên hệ vector $e'_{1}, e'_{2}, e'_{3}, e'_{4}$ độc lập tuyến tính cực đại, nên cũng là cơ sở của $\mathbb{C}_{4}$.

    \bigskip
    \par Từ việc giải các hệ phương trình tuyến tính, ta được:
    \[
        \begin{cases}
            e'_{1} = 1e_{1} + (-3)e_{2} + 1e_{3} + 1e_{4} \\
            e'_{2} = 0e_{1} + (-1)e_{2} + 2e_{3} + 1e_{4} \\
            e'_{3} = 1e_{1} + (-2)e_{2} + 2e_{3} + 1e_{4} \\
            e'_{4} = 1e_{1} + (-1)e_{2} + 1e_{3} + 1e_{4}
        \end{cases}
    \]
    \[
        \begin{cases}
            e_{1} = (-1)e'_{1} + (-2)e'_{2} + 2e'_{3} + 1e'_{4} \\
            e_{2} = (-1)e'_{1} + 0e'_{2} + 0e'_{3} + 1e'_{4} \\
            e_{3} = 1e'_{1} + (-1)e'_{2} + 2e'_{3} + 0e'_{4} \\
            e_{4} = 1e'_{1} + 2e'_{2} + (-3)e'_{3} + 1e'_{4} \\
        \end{cases}
    \]

    \par Với một vector bất kỳ $\alpha = (a_{1}, a_{2}, a_{3}, a_{4})$:

    \begin{align*}
        \alpha &= (a_{1} - a_{2} - a_{3} + a_{4})e_{1} + (a_{2} - a_{4})e_{2} + \left(\frac{1}{2}a_{1} + a_{3} - \frac{1}{2}a_{4} \right)e_{3} + \left(\frac{1}{2}a_{4} - \frac{1}{2}a_{1}\right)e_{4} \\
        \alpha &= (-2a_{1} + a_{4})e'_{1} + \left(\frac{-9}{2}a_{1} + a_{2} + a_{3} + \frac{1}{2}a_{4}\right)e_{2} + \left(\frac{9}{2}a_{1} - a_{2} - \frac{3}{2}a_{4}\right)e'_{3} + \left(\frac{3}{2}a_{1} - a_{3} + \frac{1}{2}a_{4}\right)e'_{4} \\
    \end{align*}
\end{proof}

\begin{exercise}Xét xem các tập hợp hàm số thực sau đây có lập thành không gian vector đối với các phép toán thông thường hay không? Nếu có, hãy tìm số chiều của các không gian đó.
    \begin{enumerate}[label*= (\alph*),itemsep=0pt]
        \item Tập $\mathbb{R}[X]$ các đa thức của một ẩn $X$.
        \item Tập $C^{\infty}(\mathbb{R})$ các hàm thực khả vi vô hạn trên $\mathbb{R}$.
        \item Tập $C^{0}(\mathbb{R})$ các hàm thực liên tục trên $\mathbb{R}$.
        \item Tập các hàm thực bị chặn trên $\mathbb{R}$.
        \item Tập các hàm $f: \mathbb{R}\rightarrow\mathbb{R}$ sao cho $\sup_{x\in\mathbb{R}}|f(x)| \le 1$.
        \item Tập các hàm $f: \mathbb{R}\rightarrow\mathbb{R}$ thỏa mãn điều kiện $f(0) = 0$.
        \item Tập các hàm $f: \mathbb{R}\rightarrow\mathbb{R}$ thỏa mãn điều kiện $f(0) = -1$.
        \item Tập các hàm thực đơn điệu trên $\mathbb{R}$.
    \end{enumerate}
\end{exercise}

\begin{proof}
    \begin{enumerate}[label*= (\alph*),itemsep=0pt]
        \item $\mathbb{R}[X]$ là một không gian vector trên $\mathbb{R}$.
            \par Xét hệ vector $(1, X, X^{2}, \ldots , X^{n})$
            \par Hệ này là độc lập tuyến tính vì $\displaystyle\sum^{n}_{k=0}a_{k}X^{k} = 0$ khi và chỉ khi $a_{k} = 0, \forall k\in\{0, 1, \ldots, n\}$
            \par Với $n$ tự nhiên, bất kỳ thì hệ $(1, X, X^{2}, \ldots, X^{n})$ độc lập tuyến tính.
            \par Giả sử $\mathbb{R}[X]$ hữu hạn sinh, $\dim\mathbb{R}[X] = m, m\in\mathbb{N}$.
            \par Mà hệ vector $(1, X, X^{2}, \ldots, X^{m})$ độc lập tuyến tính, gồm $m + 1$ phần tử. Điều này mâu thuẫn.
            \par Do đó $\dim\mathbb{R}[X] = \infty$.
        \item $C^{\infty}(\mathbb{R})$ là một không gian vector trên $\mathbb{R}$.
            \par $C^{\infty}(\mathbb{R})\supset\mathbb{R}[X]$ nên $\dim C^{\infty}(\mathbb{R}) = \infty$.
        \item $C^{0}(\mathbb{R})$ là một không gian vector trên $\mathbb{R}$.
            \par $C^{0}(\mathbb{R})\supset\mathbb{R}[X]$ nên $\dim C^{0}(\mathbb{R}) = \infty$.
        \item Tập các hàm thực bị chặn trên $\mathbb{R}$ là một không gian vector.
            \par Xét dãy hàm $f_{n} = \cos (nx)$. Từng hàm này đều bị chặn. Cụ thể là $|f_{n}(x)| \le 1$.
            \par $f_{n}(x)$ là một đa thức bậc $n$ với biến $\cos x$.
            \par Bằng quy nạp theo $n$, ta chứng minh được hệ $(f_{0}, f_{1}, \ldots, f_{n})$ độc lập tuyến tính.
            \par Như vậy không gian vector các hàm thực bị chặn trên $\mathbb{R}$ có số chiều là $\infty$.
        \item Tập các hàm này không lập thành không gian vector. Lấy ví dụ:
            \par $f(x) = \cos x \Rightarrow 2f(x) = 2\cos x$. Mà $\sup_{x\in\mathbb{R}}|2\cos x| = 2$. Tức là phép nhân vector với số thực không đóng trên tập này.
        \item Tập các hàm $f: \mathbb{R}\rightarrow\mathbb{R}$ thỏa mãn điều kiện $f(0) = 0$ tạo thành một không gian vector trên $\mathbb{R}$.
            \par Hệ $(X, X^{2}, \ldots, X^{n})$ độc lập tuyến tính với mọi $n\in\mathbb{N}$
            \par Do đó không gian vector này có số chiều là $\infty$.
        \item Tập các hàm $f: \mathbb{R}\rightarrow\mathbb{R}$ thỏa mãn điều kiện $f(0) = -1$ không tạo thành không gian vector trên $\mathbb{R}$.
            \par Lấy ví dụ, $f(x) = -\cos (x)\Rightarrow g(x) = 2f(x) = -2\cos (x)\Rightarrow g(0) = -2$.
        \item Tập các hàm thực đơn điệu trên $\mathbb{R}$ không tạo thành một không gian vector trên $\mathbb{R}$.
            \par Lấy ví dụ, $f(x) = x^{3}, g(x) = -3x$, hàm $f(x) + g(x) = x^{3} - 3x$ không phải hàm đơn điệu trên $\mathbb{R}$.
    \end{enumerate}
\end{proof}

\end{document}
