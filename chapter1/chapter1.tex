\documentclass[class=linearalgebra,crop=false]{standalone}

\setcounter{exercise}{0}
\newtheorem{lemma}{Bổ đề}
\setcounter{lemma}{0}

\begin{document}

\chapter{Không gian vector}

\begin{exercise}Xét xem các tập hợp sau đây có lập thành $\mathbb{F}$--không gian vector hay không đối với các phép toán thông thường (được định nghĩa theo từng thành phần):
    \begin{enumerate}[itemsep=0pt,topsep=0pt,label = (\alph*)]
        \item Tập hợp tất cả các dãy $(x_{1},\ldots,x_{n})\in\mathbb{F}_{n}$ thỏa mãn điều kiện $x_{1} + \cdots + x_{n} = 0$.
        \item Tập hợp tất cả các dãy $(x_{1},\ldots,x_{n})\in\mathbb{F}_{n}$ thỏa mãn điều kiện $x_{1} + \cdots + x_{n} = 1$.
        \item Tập hợp tất cả các dãy $(x_{1},\ldots,x_{n})\in\mathbb{F}_{n}$ thỏa mãn điều kiện $x_{1} = x_{n} = -1$.
        \item Tập hợp tất cả các dãy $(x_{1},\ldots,x_{n})\in\mathbb{F}_{n}$ thỏa mãn điều kiện $x_{1} = x_{3} = x_{5} = \cdots$, $x_{2} = x_{4} = x_{6} = \cdots$.
        \item Tập hợp các ma trận vuông $(a_{ij}){}_{n\times n}$ \textit{đối xứng} cỡ $n$, nghĩa là các ma trận thỏa mãn $a_{ij} = a_{ji}$, với $1\le i, n\le n$.
    \end{enumerate}
\end{exercise}

\begin{proof}
    \begin{enumerate}[label = (\alph*)]
        \item Đây đúng là một không gian vector với các phép toán thông thường.
        \item Đây không phải một không gian vector với các phép toán thông thường vì phép cộng không đóng.
        \item Đây không phải một không gian vector với các phép toán thông thường vì phép cộng không đóng.
        \item Đây đúng là một không gian vector với các phép toán thông thường.
        \item Đây đúng là một không gian vector với các phép toán thông thường.
    \end{enumerate}
\end{proof}

\begin{exercise}Tập hợp tất cả các dãy $(x_{1},\ldots, x_{n})\in\mathbb{R}_{n}$ với tất cả các thành phần $x_{1}, \ldots, x_{n}$ đều nguyên có lập thành một $\mathbb{R}$--không gian vector hay không?
\end{exercise}

\begin{proof}Không. Bởi phép nhân vector với vô hướng không đóng trong tập này.
\end{proof}

\begin{exercise}Với các phép toán thông thường, $\mathbb{Q}$ có là một $\mathbb{R}$--không gian vector hay không? $\mathbb{R}$ có là một $\mathbb{C}$ không gian vector hay không?
\end{exercise}

\begin{proof}$\mathbb{Q}$ không phải là một $\mathbb{R}$--không gian vector vì phép nhân một số hữu tỷ với một số thực có thể là một số vô tỷ. Ví dụ: $\sqrt{2}\cdot 1 = \sqrt{2}\not\in\mathbb{Q}$.
    \par $\mathbb{R}$ không phải là một $\mathbb{C}$--không gian vector vì phép nhân một số thực với một số phức có thể là một số phức nhưng lại không phải số thực. Ví dụ $\iota\cdot 1 = \iota\not\in\mathbb{R}$.
\end{proof}

\begin{exercise}Chứng minh rằng nhóm $\mathbb{Z}$ không đẳng cấu với nhóm cộng của bất kỳ một không gian vector trên bất kỳ trường nào.
\end{exercise}

\begin{lemma}Xét không gian vector $(V, \mathbb{F})$. $a\cdot\alpha = 0$ khi và chỉ khi $a = 0$ hoặc $\alpha$ là vector-không.
\end{lemma}

\begin{proof}[Chứng minh bổ đề]$(\Rightarrow)$ $a = 0$ thì:
    \[ 0\cdot\alpha + 0 = 0\cdot\alpha = (0 + 0)\cdot\alpha = 0\cdot\alpha + 0\cdot\alpha \]
    \par Theo luật giản ước, $0\cdot\alpha = 0$.
    \par Nếu $\alpha = 0 thì$
    \[ a\cdot 0 + 0 = a\cdot 0 = a\cdot (0 + 0) = a\cdot 0 + a\cdot 0 \]
    \par Theo luật giản ước, $a\cdot 0 = 0$.
    \bigskip
    \par $(\Leftarrow)$ $a\cdot\alpha = 0$.
    \par Giả sử $a$ khác không. Khi đó $a$ khả nghịch.
    \[ \Rightarrow (a^{-1}a)\alpha = 0 \Rightarrow \alpha = 0\]
    \par Chứng minh hoàn tất.
\end{proof}

\begin{proof}Giả sử phản chứng, tồn tại một không gian vector $V$ trên trường $\mathbb{F}$ đẳng cấu với $\mathbb{Z}$.
    \par Khi đó, tồn tại một đồng cấu $\varphi: \mathbb{Z}\rightarrow V$ sao cho:
    \[ \varphi(x) + \varphi(y) = \varphi(x + y)\quad\forall x, y\in\mathbb{Z} \]
    \par và đồng cấu này là song ánh -- đẳng cấu nhóm.
    \par Một đồng cấu nhóm biến phần tử trung lập của nhóm này thành phần tử trung lập của nhóm kia, do đó $\varphi(0)$ là vector-không trên $V$.
    \par $\varphi$ là đẳng cấu nên các phần tử của $V$ liệt kê được, bao gồm:
    \[ V = \{ \cdots, \varphi(-2), \varphi(-1), \varphi(0), \varphi(1), \varphi(2), \cdots \} \]
    \par $\mathbb{F}$ có thể là một trường chỉ gồm phần tử không với các phép toán được định nghĩa như sau:
    \begin{align*}
        0 + 0 = 0 \\
        0\cdot 0 = 0
    \end{align*}
    \par Trong trường này, phần tử đơn vị trùng với phần tử không, do đó mọi vector trên trường này đều là vector-không. Điều này mâu thuẫn với giả sử vì $\mathbb{Z}$ vô hạn đếm được, trong khi $V$ chỉ có đúng một phần tử.
    \par Do vậy, trường $\mathbb{F}$ phải có phần tử khác không.
    \par $n$ là một số nguyên dương.
    \begin{align*}
        \varphi(n)&=\underbrace{\varphi(1) + \cdots + \varphi(1)}_{n} \\
                  &=(\underbrace{1 + \cdots + 1}_{n})\varphi(1) =a_{n}\varphi(1) \\
        \varphi(-n)&=-a_{n}\varphi(1)
    \end{align*}
    \par Như vậy $a_{1} = 1$, quy ước $a_{0} = 0$. Các vector của $V$ được tương ứng một-một với $a_{n}$ -- đây là một song ánh.
    \par Nếu $\text{Char}(\mathbb{F}) = p$ thì $a_{n + p} = a_{n}$, mâu thuẫn với song ánh. Do đó $\text{Char}(\mathbb{F}) = 0$.
    \par $a_{m}$ là một phần tử khác không và khác đơn vị, phần tử đối của đơn vị ($m > 0$). Như vậy $a_{m}$ khả nghịch. Giả sử phần tử khả nghịch là $a_{k}$.
    \par Nếu $a_{k}a_{m} = a_{km} = 1 \Leftrightarrow km = 1 \Leftrightarrow k = m = 1 \vee k = m = -1$, mâu thuẫn với phép chọn $a_{m}$.
    \par Vậy giả sử phản chứng là sai.
    \par Do đó ta kết luận $\mathbb{Z}\not\cong (V, \mathbb{F}), \forall V, \mathbb{F}$.
\end{proof}

\end{document}
