\documentclass[class=linearalgebra,crop=false]{standalone}

\setcounter{exercise}{0}
\newtheorem{lemma}{Bổ đề}
\setcounter{lemma}{0}

\begin{document}

\chapter{Không gian vector}

\begin{exercise}Xét xem các tập hợp sau đây có lập thành $\mathbb{F}$--không gian vector hay không đối với các phép toán thông thường (được định nghĩa theo từng thành phần):
    \begin{enumerate}[itemsep=0pt,topsep=0pt,label = (\alph*)]
        \item Tập hợp tất cả các dãy $(x_{1},\ldots,x_{n})\in\mathbb{F}_{n}$ thỏa mãn điều kiện $x_{1} + \cdots + x_{n} = 0$.
        \item Tập hợp tất cả các dãy $(x_{1},\ldots,x_{n})\in\mathbb{F}_{n}$ thỏa mãn điều kiện $x_{1} + \cdots + x_{n} = 1$.
        \item Tập hợp tất cả các dãy $(x_{1},\ldots,x_{n})\in\mathbb{F}_{n}$ thỏa mãn điều kiện $x_{1} = x_{n} = -1$.
        \item Tập hợp tất cả các dãy $(x_{1},\ldots,x_{n})\in\mathbb{F}_{n}$ thỏa mãn điều kiện $x_{1} = x_{3} = x_{5} = \cdots$, $x_{2} = x_{4} = x_{6} = \cdots$.
        \item Tập hợp các ma trận vuông $(a_{ij}){}_{n\times n}$ \textit{đối xứng} cỡ $n$, nghĩa là các ma trận thỏa mãn $a_{ij} = a_{ji}$, với $1\le i, n\le n$.
    \end{enumerate}
\end{exercise}

\begin{proof}
    \begin{enumerate}[label = (\alph*)]
        \item Đây đúng là một không gian vector với các phép toán thông thường.
        \item Đây không phải một không gian vector với các phép toán thông thường vì phép cộng không đóng.
        \item Đây không phải một không gian vector với các phép toán thông thường vì phép cộng không đóng.
        \item Đây đúng là một không gian vector với các phép toán thông thường.
        \item Đây đúng là một không gian vector với các phép toán thông thường.
    \end{enumerate}
\end{proof}

\begin{exercise}Tập hợp tất cả các dãy $(x_{1},\ldots, x_{n})\in\mathbb{R}_{n}$ với tất cả các thành phần $x_{1}, \ldots, x_{n}$ đều nguyên có lập thành một $\mathbb{R}$--không gian vector hay không?
\end{exercise}

\begin{proof}Không. Bởi phép nhân vector với vô hướng không đóng trong tập này.
\end{proof}

\begin{exercise}Với các phép toán thông thường, $\mathbb{Q}$ có là một $\mathbb{R}$--không gian vector hay không? $\mathbb{R}$ có là một $\mathbb{C}$ không gian vector hay không?
\end{exercise}

\begin{proof}$\mathbb{Q}$ không phải là một $\mathbb{R}$--không gian vector vì phép nhân một số hữu tỷ với một số thực có thể là một số vô tỷ. Ví dụ: $\sqrt{2}\cdot 1 = \sqrt{2}\not\in\mathbb{Q}$.
    \par $\mathbb{R}$ không phải là một $\mathbb{C}$--không gian vector vì phép nhân một số thực với một số phức có thể là một số phức nhưng lại không phải số thực. Ví dụ $\iota\cdot 1 = \iota\not\in\mathbb{R}$.
\end{proof}

\begin{exercise}Chứng minh rằng nhóm $\mathbb{Z}$ không đẳng cấu với nhóm cộng của bất kỳ một không gian vector trên bất kỳ trường nào.
\end{exercise}

\begin{lemma}Xét không gian vector $(V, \mathbb{F})$. $a\cdot\alpha = 0$ khi và chỉ khi $a = 0$ hoặc $\alpha$ là vector-không.
\end{lemma}

\begin{proof}[Chứng minh bổ đề]$(\Rightarrow)$ $a = 0$ thì:
    \[ 0\cdot\alpha + 0 = 0\cdot\alpha = (0 + 0)\cdot\alpha = 0\cdot\alpha + 0\cdot\alpha \]
    \par Theo luật giản ước, $0\cdot\alpha = 0$.
    \par Nếu $\alpha = 0$ thì
    \[ a\cdot 0 + 0 = a\cdot 0 = a\cdot (0 + 0) = a\cdot 0 + a\cdot 0 \]
    \par Theo luật giản ước, $a\cdot 0 = 0$.
    \bigskip
    \par $(\Leftarrow)$ $a\cdot\alpha = 0$.
    \par Giả sử $a$ khác không. Khi đó $a$ khả nghịch.
    \[ \Rightarrow (a^{-1}a)\alpha = 0 \Rightarrow \alpha = 0\]
    \par Chứng minh hoàn tất.
\end{proof}

\begin{proof}Giả sử phản chứng, tồn tại một không gian vector $V$ trên trường $\mathbb{F}$ đẳng cấu với $\mathbb{Z}$.
    \par Khi đó, tồn tại một đồng cấu $\varphi: \mathbb{Z}\rightarrow V$ sao cho:
    \[ \varphi(x) + \varphi(y) = \varphi(x + y)\quad\forall x, y\in\mathbb{Z} \]
    \par và đồng cấu này là song ánh -- đẳng cấu nhóm.
    \par Một đồng cấu nhóm biến phần tử trung lập của nhóm này thành phần tử trung lập của nhóm kia, do đó $\varphi(0)$ là vector-không trên $V$.
    \par $\varphi$ là đẳng cấu nên các phần tử của $V$ liệt kê được, bao gồm:
    \[ V = \{ \cdots, \varphi(-2), \varphi(-1), \varphi(0), \varphi(1), \varphi(2), \cdots \} \]
    \par $\mathbb{F}$ có thể là một trường chỉ gồm phần tử không với các phép toán được định nghĩa như sau:
    \begin{align*}
        0 + 0 = 0 \\
        0\cdot 0 = 0
    \end{align*}
    \par Trong trường này, phần tử đơn vị trùng với phần tử không, do đó mọi vector trên trường này đều là vector-không. Điều này mâu thuẫn với giả sử vì $\mathbb{Z}$ vô hạn đếm được, trong khi $V$ chỉ có đúng một phần tử.
    \par Do vậy, trường $\mathbb{F}$ phải có phần tử khác không.
    \par $n$ là một số nguyên dương.
    \begin{align*}
        \varphi(n)&=\underbrace{\varphi(1) + \cdots + \varphi(1)}_{n} \\
                  &=(\underbrace{1 + \cdots + 1}_{n})\varphi(1) =a_{n}\varphi(1) \\
        \varphi(-n)&=-a_{n}\varphi(1)
    \end{align*}
    \par Như vậy $a_{1} = 1$, quy ước $a_{0} = 0$. Các vector của $V$ được tương ứng một-một với $a_{n}$ -- đây là một song ánh.
    \par Nếu $\text{Char}(\mathbb{F}) = p$ thì $a_{n + p} = a_{n}$, mâu thuẫn với song ánh. Do đó $\text{Char}(\mathbb{F}) = 0$.
    \par $a_{m}$ là một phần tử khác không và khác đơn vị, phần tử đối của đơn vị ($m > 0$). Như vậy $a_{m}$ khả nghịch. Giả sử phần tử khả nghịch là $a_{k}$.
    \par Nếu $a_{k}a_{m} = a_{km} = 1 \Leftrightarrow km = 1 \Leftrightarrow k = m = 1 \vee k = m = -1$, mâu thuẫn với phép chọn $a_{m}$.
    \par Vậy giả sử phản chứng là sai.
    \par Do đó ta kết luận $\mathbb{Z}$ không đẳng cấu với bất cứ không gian vector nào, trên bất cứ trường nào.
\end{proof}

\begin{exercise}Chứng minh rằng nhóm abel $A$ đối với phép cộng $+$ có thể trở thành một không gian vector trên trường $\mathbb{F}_{p}$ nếu và chỉ nếu
    \[ px = \underbrace{x + x + \cdots + x}_{p} = 0,\quad \forall x\in A. \]
\end{exercise}

\begin{proof}Một nhóm abel $A$ với phép cộng đã thỏa mãn 4 tiên đề đầu tiên của không gian vector.
    \par($\Rightarrow$) $A$ là một không gian vector trên trường $\mathbb{F}_{p}$.
        \[ \underbrace{x + x + \cdots + x}_{p} = \underbrace{1x + 1x + \cdots + 1x}_{p} = \underbrace{(1 + 1 + \cdots + 1)}_{p}x = 0x = 0 \]
    \par($\Leftarrow$) $\underbrace{x + x + \cdots + x}_{p} = 0$
    \par Ta định nghĩa phép nhân $nx$ với $x\in A, n\in\mathbb{Z}$ như sau:
    \[
        nx =
        \begin{cases}
            \underbrace{x + x + \cdots + x}_{n}, & n > 0 \\
            0, & n = 0 \\
            \underbrace{(-x) + (-x) + \cdots + (-x)}_{-n}, & n < 0
        \end{cases}
    \]
    \par Cùng với điều kiện $\underbrace{x + \cdots + x}_{p} = 0$, ta suy ra $nx = (n\bmod{p})x$
    \par Ta kiểm tra 4 tiên đề còn lại:
    \begin{enumerate}[label= (V\arabic*)]
        \setcounter{enumi}{4}
        \item $(a + b)x$
            \par Nếu $a > 0, b > 0$:
            \[ (a + b)x = \underbrace{x + x + \cdots + x}_{a+b} = \underbrace{x + \cdots + x}_{a} + \underbrace{x + \cdots + x}_{b} = ax + bx \]
            \par Nếu $a < 0, b < 0$:
            \[ (a + b)x = \underbrace{(-x) + \cdots + (-x)}_{-a-b} = \underbrace{(-x) + \cdots + (-x)}_{-a} + \underbrace{(-x) + \cdots + (-x)}_{-b} = ax + bx \]
            \par Nếu $a = 0$ hoặc $b = 0$
            \[ (a + b)x = ax + bx \]
            \par Nếu $a > 0, b < 0$ và $a + b > 0$
            \[ (a + b)x = \underbrace{x + \cdots + x}_{a+b} = \underbrace{x + \cdots + x}_{a} + \underbrace{(-x) + \cdots + (-x)}_{-b} = ax + bx \]
            \par Nếu $a > 0, b < 0$ và $a + b < 0$
            \[ (a + b)x = \underbrace{(-x) + \cdots + (-x)}_{-a-b} = \underbrace{x + \cdots + x}_{a} + \underbrace{(-x) + \cdots + (-x)}_{b} = ax + bx \]
            \par Nếu $a > 0, b < 0$ và $a + b = 0$
            \[ (a + b)x = 0 = \underbrace{(x + \cdots + x)}_{a} + \underbrace{(-x) + \cdots + (-x)}_{-b} = ax + bx \]
            \par Trường hợp $a < 0, b > 0$ được chứng minh tương tự.
        \item $a(x + y)$
            \par Nếu $a = 0$:
            \[ a(x + y) = 0 = 0 + 0 = 0x + 0y = ax + ay \]
            \par Nếu $a > 0$:
            \[ a(x + y) = \underbrace{(x+y)+\cdots+(x+y)}_{a} = \underbrace{x + \cdots + x}_{a} + \underbrace{y + \cdots + y}_{a} = ax + ay \]
            \par Nếu $a < 0$:
            \[ a(x + y) = \underbrace{(-x-y)+\cdots+(-x-y)}_{-a} = \underbrace{(-x) + \cdots + (-x)}_{-a} + \underbrace{(-y) + \cdots + (-y)}_{-a} = ax + ay \]
        \item $(ab)x$
            \par Nếu $a = 0$ hoặc $b = 0$:
            \[ (ab)x = 0 = a(bx) \]
            \par Nếu $a > 0, b > 0$:
            \[ (ab)x = \underbrace{x + \cdots + x}_{ab} = \underbrace{\underbrace{(x + \cdots + x)}_{b} + \cdots + \underbrace{(x + \cdots + x)}_{b}}_{a} = a(bx) \]
            \par Các trường hợp còn lại được quy về $a\ge 0, b\ge 0$ vì $nx = (n\bmod p)x$.
        \item $1x = x$
    \end{enumerate}
    \par Do đó $A$ là một không gian vector trên trường $\mathbb{F}_{p}$.
\end{proof}

\begin{exercise}Xét xem các vector sau đây độc lập hay phụ thuộc tuyến tính trong $\mathbb{R}_{4}$:
    \begin{enumerate}[label = (\alph*)]
        \item $e_{1} = (-1, -2, 1, 2)$, $e_{2} = (0, -1, 2, 3)$, $e_{3} = (1, 4, 1, 2)$, $e_{4} = (-1, 0, 1, 3)$.
        \item $\alpha_{1} = (-1, 1, 0, 1)$, $\alpha_{2} = (1, 0, 1, 1)$, $\alpha_{3} = (-3, 1, -2, -1)$.
    \end{enumerate}
\end{exercise}

\begin{proof}
    \begin{enumerate}[label = (\alph*)]
        \item Xét ràng buộc tuyến tính $x_{1}e_{1} + x_{2}e_{2} + x_{3}e_{3} + x_{4}e_{4} = (0,0,0,0)$. Để tìm $x_{1}, x_{2}, x_{3}, x_{4}$, ta giải hệ phương trình thuần nhất sau:
            \begin{align*}
                &\begin{cases}
                    (-1)x_{1} + 0x_{2} + 1x_{3} + (-1)x_{4} = 0 \\
                    (-2)x_{1} + (-1)x_{2} + 4x_{3} + 0x_{4} = 0 \\
                    1x_{1} + 2x_{2} + 1x_{3} + 1x_{4} = 0 \\
                    2x_{1} + 3x_{2} + 2x_{3} + 3x_{4} = 0
                \end{cases} \\
                \Leftrightarrow\quad&
                \begin{cases}
                    0x_{1} + 3x_{2} + 4x_{3} + 1x_{4} = 0 \\
                    0x_{1} + 2x_{2} + 6x_{3} + 3x_{4} = 0 \\
                    0x_{1} + x_{2} + 0x_{3} + (-1)x_{4} = 0 \\
                    2x_{1} + 3x_{2} + 2x_{3} + 3x_{4} = 0
                \end{cases} \\
                \Leftrightarrow\quad&
                \begin{cases}
                    0x_{1} + 0x_{2} + 4x_{3} + 1x_{4} = 0 \\
                    0x_{1} + 0x_{2} + 6x_{3} + 5x_{4} = 0 \\
                    0x_{1} + x_{2} + 0x_{3} + (-1)x_{4} = 0 \\
                    2x_{1} + 3x_{2} + 2x_{3} + 3x_{4} = 0
                \end{cases} \\
                \Leftrightarrow\quad&
                \begin{cases}
                    0x_{1} + 0x_{2} + 12x_{3} + 3x_{4} = 0 \\
                    0x_{1} + 0x_{2} + 12x_{3} + 10x_{4} = 0 \\
                    0x_{1} + x_{2} + 0x_{3} + (-1)x_{4} = 0 \\
                    2x_{1} + 3x_{2} + 2x_{3} + 3x_{4} = 0
                \end{cases} \\
                \Leftrightarrow\quad&
                \begin{cases}
                    0x_{1} + 0x_{2} + 0x_{3} + (-7)x_{4} = 0 \\
                    0x_{1} + 0x_{2} + 12x_{3} + 10x_{4} = 0 \\
                    0x_{1} + x_{2} + 0x_{3} + (-1)x_{4} = 0 \\
                    2x_{1} + 3x_{2} + 2x_{3} + 3x_{4} = 0
                \end{cases}
            \end{align*}
            \par Hệ phương trình này chỉ có nghiệm tầm thường $(x_{1}, x_{2}, x_{3}, x_{4}) = (0, 0, 0, 0)$, kéo theo ràng buộc tuyến tính tầm thường. Do đó hệ độc lập tuyến tính.
        \item Xét ràng buộc tuyến tính $x_{1}\alpha_{1} + x_{2}\alpha_{2} + x_{3}\alpha_{3} = (0, 0, 0)$. Để tìm $x_{1}, x_{2}, x_{3}$, ta giải hệ phương trình thuần nhất sau:
            \begin{align*}
                &\begin{cases}
                    (-1)x_{1} + 1x_{2} + (-3)x_{3} = 0 \\
                    1x_{1} + 0x_{2} + 1x_{3} = 0 \\
                    0x_{1} + 1x_{2} + (-2)x_{3} = 0 \\
                    1x_{1} + 1x_{2} + (-1)x_{3} = 0
                \end{cases} \\
                \Leftrightarrow\quad&
                \begin{cases}
                    0x_{1} + 2x_{2} + (-4)x_{3} = 0 \\
                    0x_{1} + (-1)x_{2} + 2x_{3} = 0 \\
                    0x_{1} + 1x_{2} + (-2)x_{3} = 0 \\
                    1x_{1} + 1x_{2} + (-1)x_{3} = 0
                \end{cases} \\
                \Leftrightarrow\quad&
                \begin{cases}
                    0x_{1} + 1x_{2} + (-2)x_{3} = 0 \\
                    1x_{1} + 1x_{2} + (-1)x_{3} = 0
                \end{cases}
            \end{align*}
            \par Hệ phương trình này có nghiệm không tầm thường $(x_{1}, x_{2}, x_{3}) = (0, 2, 1)$. Do đó hệ phụ thuộc tuyến tính.
    \end{enumerate}
\end{proof}

\begin{exercise}Chứng minh rằng hai hệ vector sau đây là các cơ sở của $\mathbb{C}_{3}$. Tìm ma trận chuyển từ cơ sở thứ nhất sang cơ sở thứ hai:
    \par $e_{1} = (1, 2, 1), e_{2} = (2, 3, 3), e_{3} = (3, 7, 1)$;
    \par $e'_{1} = (3, 1, 4), e'_{2} = (5, 2, 1), e'_{3} = (1, 1, -6)$.
\end{exercise}

\begin{proof}Xét ràng buộc tuyến tính $x_{1}e_{1} + x_{2}e_{2} + x_{3}e_{3} = (0, 0, 0)$.
    \begin{align*}
        &\begin{cases}
            1x_{1} + 2x_{2} + 3x_{3} = 0 \\
            2x_{1} + 3x_{2} + 7x_{3} = 0 \\
            1x_{1} + 3x_{2} + 1x_{3} = 0
        \end{cases} \\
        \Leftrightarrow\quad&
        \begin{cases}
            1x_{1} + 2x_{2} + 3x_{3} = 0 \\
            0x_{1} + (-1)x_{2} + 1x_{3} = 0 \\
            0x_{1} + 1x_{2} + (-2)x_{3} = 0
        \end{cases} \\
        \Leftrightarrow\quad&
        \begin{cases}
            1x_{1} + 2x_{2} + 3x_{3} = 0 \\
            0x_{1} + (-1)x_{2} + 1x_{3} = 0 \\
            0x_{1} + 0x_{2} + (-1)x_{3} = 0
        \end{cases}
    \end{align*}
    \par Hệ phương trình trên chỉ có nghiệm tầm thường $(x_{1}, x_{2}, x_{3}) = (0, 0, 0)$ do đó hệ $e_{1}, e_{2}, e_{3}$ độc lập tuyến tính và cực đại (vì $\dim\mathbb{C}_{3} = 3$) nên hệ này cũng là một cơ sở của $\mathbb{C}_{3}$.

    \bigskip
    \par Xét ràng buộc tuyến tính $x_{1}e'_{1} + x_{2}e'_{2} + x_{3}e'_{3} = (0, 0, 0)$.
    \begin{align*}
        &\begin{cases}
            3x_{1} + 5x_{2} + 1x_{3} = 0 \\
            1x_{1} + 2x_{2} + 1x_{3} = 0 \\
            4x_{1} + 1x_{2} + (-6)x_{3} = 0
        \end{cases} \\
        \Leftrightarrow\quad&
        \begin{cases}
            3x_{1} + 5x_{2} + 1x_{3} = 0 \\
            0x_{1} + 1x_{2} + 2x_{3} = 0 \\
            0x_{1} + (-6)x_{2} + (-8)x_{3} = 0
        \end{cases} \\
        \Leftrightarrow\quad&
        \begin{cases}
            3x_{1} + 5x_{2} + 1x_{3} = 0 \\
            0x_{1} + 1x_{2} + 2x_{3} = 0 \\
            0x_{1} + 0x_{2} + 4x_{3} = 0
        \end{cases}
    \end{align*}
    \par Hệ phương trình trên chỉ có nghiệm tầm thường $(x_{1}, x_{2}, x_{3}) = (0, 0, 0)$ do đó hệ $e_{1}, e_{2}, e_{3}$ độc lập tuyến tính và cực đại (vì $\dim\mathbb{C}_{3} = 3$) nên hệ này cũng là một cơ sở của $\mathbb{C}_{3}$.

    \par $(e_{1}, e_{2}, e_{3})$ là một cơ sở của $\mathbb{C}_{3}$ nên vector $(a_{1}, a_{2}, a_{3})$ biểu thị tuyến tính được duy nhất theo cơ sở này.
    \par Bằng việc giải hệ phương trình sau:
    \[
        \begin{cases}
            1x_{1} + 2x_{2} + 3x_{3} = a_{1} \\
            2x_{1} + 3x_{2} + 7x_{3} = a_{2} \\
            1x_{1} + 3x_{2} + 1x_{3} = a_{3}
        \end{cases}
    \]
    \par ta thu được nghiệm:
    \[
        (x_{1}, x_{2}, x_{3}) = (-18a_{1} + 7a_{2} + 5a_{3}, 5a_{1} - 2a_{2} - a_{3}, 3a_{1} - a_{2} - a_{3})
    \]
    \par Thay số, ta biểu diễn được $(e'_{1}, e'_{2}, e'_{3})$ qua $(e_{1}, e_{2}, e_{3})$ như sau:
    \[
        \begin{cases}
            e'_{1} = (-27)e_{1} + 9e_{2} + 4e_{3} \\
            e'_{2} = (-71)e_{1} + 20e_{2} + 12e_{3} \\
            e'_{3} = (-41)e_{1} + 9e_{2} + 8e_{3}
        \end{cases}
    \]
    \par Như vậy, ma trận chuyển cơ sở $(e_{1}, e_{2}, e_{3}) \rightarrow (e'_{1}, e'_{2}, e'_{3})$ là:
    \[
        \begin{pmatrix}
            -27 & 9 & 4 \\
            -71 & 20 & 12 \\
            -41 & 9 & 8
        \end{pmatrix}
    \]
\end{proof}

\begin{exercise}Chứng minh rằng hai hệ vector sau đây là các cơ sở của $\mathbb{C}_{4}$. Tìm mối liên hệ giữa tọa độ của cùng một vector trong hai cơ sở đó:
    \par $e_{1} = (1, 1, 1, 1), e_{2} = (1, 2, 1, 1), e_{3} = (1, 1, 2, 1), e_{4} = (1, 3, 2, 3)$;
    \par $e'_{1} = (1, 0, 3, 3), e'_{2} = (2, 3, 5, 4), e'_{3} = (2, 2, 5, 4), e'_{4} = (2, 3, 4, 4)$.
\end{exercise}

\begin{proof}Xét ràng buộc tuyến tính $x_{1}e_{1} + x_{2}e_{2} + x_{3}e_{3} + x_{4}e_{4} = (0,0,0,0)$.
    \begin{align*}
        &\begin{cases}
            1x_{1} + 1x_{2} + 1x_{3} + 1x_{4} = 0 \\
            1x_{1} + 2x_{2} + 1x_{3} + 3x_{4} = 0 \\
            1x_{1} + 1x_{2} + 2x_{3} + 2x_{4} = 0 \\
            1x_{1} + 1x_{2} + 1x_{3} + 3x_{4} = 0
        \end{cases} \\
        \Leftrightarrow\quad&
        \begin{cases}
            1x_{1} + 1x_{2} + 1x_{3} + 1x_{4} = 0 \\
            0x_{1} + 1x_{2} + 0x_{3} + 2x_{4} = 0 \\
            0x_{1} + 0x_{2} + 1x_{3} + 1x_{4} = 0 \\
            0x_{1} + 0x_{2} + 0x_{3} + 2x_{4} = 0
        \end{cases}
    \end{align*}

    \par Hệ phương trình trên chỉ có nghiệm tầm thường $(x_{1}, x_{2}, x_{3}, x_{4}) = (0, 0, 0, 0)$.
    \par Do đó hệ vector $e_{1}, e_{2}, e_{3}, e_{4}$ độc lập tuyến tính.
    \par Số chiều của không gian vector $\mathbb{C}_{4}$ bằng 4 nên hệ vector $e_{1}, e_{2}, e_{3}, e_{4}$ độc lập tuyến tính cực đại, nên cũng là cơ sở của $\mathbb{C}_{4}$.

    \bigskip
    \par Xét ràng buộc tuyến tính $x_{1}e'_{1} + x_{2}e'_{2} + x_{3}e'_{3} + x_{4}e'_{4} = (0,0,0,0)$.
    \begin{align*}
        &\begin{cases}
            1x_{1} + 2x_{2} + 2x_{3} + 2x_{4} = 0 \\
            0x_{1} + 3x_{2} + 2x_{3} + 3x_{4} = 0 \\
            3x_{1} + 5x_{2} + 5x_{3} + 4x_{4} = 0 \\
            3x_{1} + 4x_{2} + 4x_{3} + 4x_{4} = 0
        \end{cases} \\
        \Leftrightarrow\quad&
        \begin{cases}
            1x_{1} + 2x_{2} + 2x_{3} + 2x_{4} = 0 \\
            0x_{1} + 3x_{2} + 2x_{3} + 3x_{4} = 0 \\
            0x_{1} + (-1)x_{2} + (-1)x_{3} + (-2)x_{4} = 0 \\
            0x_{1} + (-2)x_{2} + (-2)x_{3} + (-2)x_{4} = 0
        \end{cases} \\
        \Leftrightarrow\quad&
        \begin{cases}
            1x_{1} + 2x_{2} + 2x_{3} + 2x_{4} = 0 \\
            0x_{1} + 3x_{2} + 2x_{3} + 3x_{4} = 0 \\
            0x_{1} + 0x_{2} + (-1)x_{3} + (-3)x_{4} = 0 \\
            0x_{1} + 0x_{2} + 0x_{3} + 2x_{4} = 0
        \end{cases}
    \end{align*}

    \par Hệ phương trình trên chỉ có nghiệm tầm thường $(x_{1}, x_{2}, x_{3}, x_{4}) = (0, 0, 0, 0)$.
    \par Do đó hệ vector $e'_{1}, e'_{2}, e'_{3}, e'_{4}$ độc lập tuyến tính.
    \par Số chiều của không gian vector $\mathbb{C}_{4}$ bằng 4 nên hệ vector $e'_{1}, e'_{2}, e'_{3}, e'_{4}$ độc lập tuyến tính cực đại, nên cũng là cơ sở của $\mathbb{C}_{4}$.

    \bigskip
    \par Từ việc giải các hệ phương trình tuyến tính, ta được:
    \[
        \begin{cases}
            e'_{1} = 1e_{1} + (-3)e_{2} + 1e_{3} + 1e_{4} \\
            e'_{2} = 0e_{1} + (-1)e_{2} + 2e_{3} + 1e_{4} \\
            e'_{3} = 1e_{1} + (-2)e_{2} + 2e_{3} + 1e_{4} \\
            e'_{4} = 1e_{1} + (-1)e_{2} + 1e_{3} + 1e_{4}
        \end{cases}
    \]
    \[
        \begin{cases}
            e_{1} = (-1)e'_{1} + (-2)e'_{2} + 2e'_{3} + 1e'_{4} \\
            e_{2} = (-1)e'_{1} + 0e'_{2} + 0e'_{3} + 1e'_{4} \\
            e_{3} = 1e'_{1} + (-1)e'_{2} + 2e'_{3} + 0e'_{4} \\
            e_{4} = 1e'_{1} + 2e'_{2} + (-3)e'_{3} + 1e'_{4} \\
        \end{cases}
    \]

    \par Với một vector bất kỳ $\alpha = (a_{1}, a_{2}, a_{3}, a_{4})$:

    \begin{align*}
        \alpha &= (a_{1} - a_{2} - a_{3} + a_{4})e_{1} + (a_{2} - a_{4})e_{2} + \left(\frac{1}{2}a_{1} + a_{3} - \frac{1}{2}a_{4} \right)e_{3} + \left(\frac{1}{2}a_{4} - \frac{1}{2}a_{1}\right)e_{4} \\
        \alpha &= (-2a_{1} + a_{4})e'_{1} + \left(\frac{-9}{2}a_{1} + a_{2} + a_{3} + \frac{1}{2}a_{4}\right)e_{2} + \left(\frac{9}{2}a_{1} - a_{2} - \frac{3}{2}a_{4}\right)e'_{3} + \left(\frac{3}{2}a_{1} - a_{3} + \frac{1}{2}a_{4}\right)e'_{4} \\
    \end{align*}
\end{proof}

\begin{exercise}Xét xem các tập hợp hàm số thực sau đây có lập thành không gian vector đối với các phép toán thông thường hay không? Nếu có, hãy tìm số chiều của các không gian đó.
    \begin{enumerate}[label*= (\alph*),itemsep=0pt]
        \item Tập $\mathbb{R}[X]$ các đa thức của một ẩn $X$.
        \item Tập $C^{\infty}(\mathbb{R})$ các hàm thực khả vi vô hạn trên $\mathbb{R}$.
        \item Tập $C^{0}(\mathbb{R})$ các hàm thực liên tục trên $\mathbb{R}$.
        \item Tập các hàm thực bị chặn trên $\mathbb{R}$.
        \item Tập các hàm $f: \mathbb{R}\rightarrow\mathbb{R}$ sao cho $\sup_{x\in\mathbb{R}}|f(x)| \le 1$.
        \item Tập các hàm $f: \mathbb{R}\rightarrow\mathbb{R}$ thỏa mãn điều kiện $f(0) = 0$.
        \item Tập các hàm $f: \mathbb{R}\rightarrow\mathbb{R}$ thỏa mãn điều kiện $f(0) = -1$.
        \item Tập các hàm thực đơn điệu trên $\mathbb{R}$.
    \end{enumerate}
\end{exercise}

\begin{proof}
    \begin{enumerate}[label*= (\alph*),itemsep=0pt]
        \item $\mathbb{R}[X]$ là một không gian vector trên $\mathbb{R}$.
            \par Xét hệ vector $(1, X, X^{2}, \ldots , X^{n})$
            \par Hệ này là độc lập tuyến tính vì $\displaystyle\sum^{n}_{k=0}a_{k}X^{k} = 0$ khi và chỉ khi $a_{k} = 0, \forall k\in\{0, 1, \ldots, n\}$
            \par Với $n$ tự nhiên, bất kỳ thì hệ $(1, X, X^{2}, \ldots, X^{n})$ độc lập tuyến tính.
            \par Giả sử $\mathbb{R}[X]$ hữu hạn sinh, $\dim\mathbb{R}[X] = m, m\in\mathbb{N}$.
            \par Mà hệ vector $(1, X, X^{2}, \ldots, X^{m})$ độc lập tuyến tính, gồm $m + 1$ phần tử. Điều này mâu thuẫn.
            \par Do đó $\dim\mathbb{R}[X] = \infty$.
        \item $C^{\infty}(\mathbb{R})$ là một không gian vector trên $\mathbb{R}$.
            \par $C^{\infty}(\mathbb{R})\supset\mathbb{R}[X]$ nên $\dim C^{\infty}(\mathbb{R}) = \infty$.
        \item $C^{0}(\mathbb{R})$ là một không gian vector trên $\mathbb{R}$.
            \par $C^{0}(\mathbb{R})\supset\mathbb{R}[X]$ nên $\dim C^{0}(\mathbb{R}) = \infty$.
        \item Tập các hàm thực bị chặn trên $\mathbb{R}$ là một không gian vector.
            \par Xét dãy hàm $f_{n} = \cos (nx)$. Từng hàm này đều bị chặn. Cụ thể là $|f_{n}(x)| \le 1$.
            \par $f_{n}(x)$ là một đa thức bậc $n$ với biến $\cos x$.
            \par Bằng quy nạp theo $n$, ta chứng minh được hệ $(f_{0}, f_{1}, \ldots, f_{n})$ độc lập tuyến tính.
            \par Như vậy không gian vector các hàm thực bị chặn trên $\mathbb{R}$ có số chiều là $\infty$.
        \item Tập các hàm này không lập thành không gian vector. Lấy ví dụ:
            \par $f(x) = \cos x \Rightarrow 2f(x) = 2\cos x$. Mà $\sup_{x\in\mathbb{R}}|2\cos x| = 2$. Tức là phép nhân vector với số thực không đóng trên tập này.
        \item Tập các hàm $f: \mathbb{R}\rightarrow\mathbb{R}$ thỏa mãn điều kiện $f(0) = 0$ tạo thành một không gian vector trên $\mathbb{R}$.
            \par Hệ $(X, X^{2}, \ldots, X^{n})$ độc lập tuyến tính với mọi $n\in\mathbb{N}$
            \par Do đó không gian vector này có số chiều là $\infty$.
        \item Tập các hàm $f: \mathbb{R}\rightarrow\mathbb{R}$ thỏa mãn điều kiện $f(0) = -1$ không tạo thành không gian vector trên $\mathbb{R}$.
            \par Lấy ví dụ, $f(x) = -\cos (x)\Rightarrow g(x) = 2f(x) = -2\cos (x)\Rightarrow g(0) = -2$.
        \item Tập các hàm thực đơn điệu trên $\mathbb{R}$ không tạo thành một không gian vector trên $\mathbb{R}$.
            \par Lấy ví dụ, $f(x) = x^{3}, g(x) = -3x$, hàm $f(x) + g(x) = x^{3} - 3x$ không phải hàm đơn điệu trên $\mathbb{R}$.
    \end{enumerate}
\end{proof}

\begin{exercise}Định nghĩa hai phép toán cộng và nhân với vô hướng trên tập hợp
    \[ V = \{ (x, y)\in\mathbb{R}\times\mathbb{R}\mid y > 0 \} \]
    \par như sau:
    \begin{align*}
        (x, y) + (u, v) &= (x + u, yv) \\
                a(x, y) &= (ax, y^{a})
    \end{align*}
    \par Xét xem $V$ có là một không gian vector thực đối với hai phép toán đó không. Nếu có, hãy tìm một cơ sở của không gian ấy.
\end{exercise}

\begin{proof}Đầu tiên, $V$ đóng với hai phép toán đã được định nghĩa. Bây giờ ta kiểm tra 8 tiên đề:
    \begin{enumerate}[label = (V\arabic*)]
        \item
            \begin{align*}
                ((x_{1}, y_{1}) + (x_{2}, y_{2})) + (x_{3}, y_{3}) &= (x_{1} + x_{2}, y_{1}y_{2}) + (x_{3}, y_{3}) \\
                                                                   &= ((x_{1} + x_{2}) + x_{3}, (y_{1}y_{2})y_{3}) \\
                                                                   &= (x_{1} + (x_{2} + x_{3}), y_{1}(y_{1}y_{3})) \\
                                                                   &= (x_{1}, y_{1}) + (x_{2} + x_{3}, y_{2}y_{3}) \\
                                                                   &= (x_{1}, y_{1}) + ((x_{2}, y_{2}) + (x_{3}, y_{3}))
            \end{align*}
        \item Phần tử trung lập là $(0, 1)$.
            \begin{align*}
                (x, y) + (0, 1) &= (x + 0, y) = (x, y) \\
                (0, 1) + (x, y) &= (0 + x, y) = (x, y)
            \end{align*}
        \item
            \begin{align*}
                (x, y) + (-x, y^{-1}) &= (x + (-x), yy^{-1}) = (0, 1) \\
                (-x, y^{-1}) + (x, y) &= (-x + x, y^{-1}y) = (0, 1)
            \end{align*}
        \item
            \begin{align*}
                (x_{1}, y_{1}) + (x_{2}, y_{2}) &= (x_{1} + x_{2}, y_{1}y_{2}) \\
                                                &= (x_{2} + x_{1}, y_{2}y_{1}) \\
                                                &= (x_{2}, y_{2}) + (x_{1}, y_{1})
            \end{align*}
        \item
            \begin{align*}
                (a + b)(x, y) &= ((a + b)x, y^{a + b}) \\
                              &= (ax + bx, y^{a}y^{b}) \\
                              &= (ax, y^{a}) + (bx, y^{b}) \\
                              &= a(x, y) + b(x, y)
            \end{align*}
        \item
            \begin{align*}
                a((x_{1}, y_{1}) + (x_{2}, y_{2})) &= a(x_{1} + x_{2}, y_{1}y_{2}) \\
                                                   &= (a(x_{1} + x_{2}), y^{a}_{1}y^{a}_{2}) \\
                                                   &= (ax_{1} + ax_{2}, y^{a}_{1}y^{a}_{2}) \\
                                                   &= (ax_{1}, y^{a}_{1}) + (ax_{2}, y^{a}_{2}) \\
                                                   &= a(x_{1},y_{1}) + a(x_{2}, y_{2})
            \end{align*}
        \item
            \begin{align*}
                a(b(x, y)) &= a(bx, y^{b}) \\
                           &= (a(bx), (y^{b}){}^{a}) \\
                           &= ((ab)x, y^{ab}) \\
                           &= ab(x, y)
            \end{align*}
        \item
            \begin{align*}
                1(x, y) &= (1x, y^{1}) = (x, y)
            \end{align*}
    \end{enumerate}
    \par Như vậy, $V$ là một không gian vector thực.

    \begin{align*}
        (x, y) &= (x, 1) + (0, y) \\
               &= (x, 1^{x}) + (0, e^{\ln y}) \\
               &= x(1, 1) + \ln y (0, e)
    \end{align*}
    \par Xét hệ vector $((1, 1), (0, e))$ và ràng buộc tuyến tính $a(1, 1) + b(0, e) = (0, 1)$
    \begin{align*}
        &a(1, 1) + b(0, e) = (0, 1) \\
        \Leftrightarrow\quad&(a, 1) + (0, e^{b}) = (0, 1) \\
        \Leftrightarrow\quad&(a, e^{b}) = (0, 1) \\
        \Leftrightarrow\quad&
        \begin{cases}
            a = 0 \\
            e^{b} = 1
        \end{cases}\Leftrightarrow a = b = 0
    \end{align*}
    \par Như vậy, $((1, 1), (0, e))$ là một hệ sinh độc lập tuyến tính của $V$, do đó đây cũng là một cơ sở của $V$.
\end{proof}

\begin{exercise}Ma trận chuyển từ một cơ sở sang một cơ sở khác thay đổi thế nào nếu:
    \begin{enumerate}[label = (\alph*),itemsep=0pt]
        \item Đổi chỗ hai vector trong cơ sở thứ nhất?
        \item Đổi chỗ hai vector trong cơ sở thứ hai?
        \item Đổi các vector trong mỗi cơ sở theo thứ tự ngược lại?
    \end{enumerate}
\end{exercise}

\begin{proof}Lấy hai cơ sở $(\alpha_{1},\ldots, \alpha_{n})$, $(\beta_{1},\ldots,\beta_{n})$.
    \par Ta xét ma trận chuyển cơ sở $(c_{ij}){}_{n\times n}$ chuyển từ cơ sở thứ nhất sang cơ sở thứ hai.
    \[
        \beta_{j} = \sum^{n}_{i=1}c_{ij}\alpha_{i}
    \]
    \begin{enumerate}[label = (\alph*),itemsep=0pt]
        \item Đổi chỗ hai vector $\alpha_{i}$ và $\alpha_{j}$ thì hai hàng thứ $i$ và $j$ trong ma trận chuyển cơ sở đổi chỗ.
        \item Đổi chỗ hai vector $\beta_{i}$ và $\beta_{j}$ thì hai cột thứ $i$ và $j$ trong ma trận chuyển cơ sở đổi chỗ.
        \item Phép đảo thứ tự vector là hợp thành của nhiều phép đổi chỗ hai vector. Như vậy, khi đảo các vector $\alpha_{i}$ theo thứ tự ngược lại thì thứ tự các vector hàng ngược lại. Sau khi đảo tiếp các vector $\beta_{i}$ theo thứ tự ngược lại thì thứ tự các vector cột ngược lại.
            \par Cụ thể hơn, $(c_{ij}){}_{n\times n} \mapsto (d_{ij}){}_{n\times n}$ thì:
            \[ d_{ij} = c_{(n+1-i)(n+1-j)} \]
    \end{enumerate}
\end{proof}

\begin{exercise}Chứng minh rằng hai hệ vector $(1, X, X^{2}, \ldots, X^{n})$ và $(1, (X - a), (X - a){}^{2}, \ldots, (X - a){}^{n})$, trong đó $a$ là một số thực, là các cơ sở của không gian $\mathbb{R}[X]{}_{n}$ các đa thức hệ số thực với bậc không vượt quá $n$. Tìm ma trận chuyển từ cơ sở thứ nhất sang cơ sở thứ hai.
\end{exercise}

\begin{proof}Lấy một đa thức có bậc không quá $n$:
    \[ P(X) = a_{0} + a_{1}X + a_{2}X^{2} + \cdots + a_{n}X^{n} \]
    \par Rõ ràng $P(X)$ biểu thị tuyến tính được theo hệ vector $(1, X, X^{2}, \ldots, X^{n})$.
    \par Giả sử $P(X) = b_{0} + b_{1}X + b_{2}X^{2} + \cdots + b_{n}X^{n}$.
    \par Đồng nhất hai đa thức, ta được $a_{i} = b_{i}, \forall i\in\{0, 1, \ldots, n\}$. Như vậy $P(X)$ biểu thị tuyến tính được theo hệ vector $(1, X, X^{2}, \ldots, X^{n})$ theo cách duy nhất. Do đó hệ này là một cơ sở và $\dim_{\mathbb{R}}\mathbb{R}[X]{}_{n} = n + 1$.

    \par Nếu $a = 0$ thì hệ vector $(1, (X - a), (X - a){}^{2}, \ldots, (X - a){}^{n})$ trở thành hệ $(1, X, X^{2}, \ldots, X^{n})$, đây là một cơ sở.
    \par Do đó ta chỉ xem xét trường hợp $a\ne 0$.
    \par Ta xét ràng buộc tuyến tính $\displaystyle\sum^{n}_{k=0}x_{k}(X-a){}^{k} = 0$
    \par Khai triển ràng buộc trên bằng khai triển Newton, hệ số của hạng tử $X^{k}$ bằng:
    \[ \sum^{n}_{i=k}x_{i}\binom{i}{i-k}(-a){}^{i-k} \]
    \par Từ ràng buộc tuyến tính và khai triển trên, ta đồng nhất hệ số và thu được hệ phương trình tuyến tính thuần nhất (theo thứ tự, đồng nhất các hệ số của $X^{n}, X^{n-1}, \ldots, X, 1$):
    \[
        \begin{cases}
            x_{n}\dbinom{n}{0} = 0 \\
            x_{n}\dbinom{n}{1}(-a){}^{1} + x_{n-1}\dbinom{n-1}{0}(-a){}^{0} = 0 \\
            x_{n}\dbinom{n}{2}(-a){}^{2} + x_{n-1}\dbinom{n-1}{1}(-a){}^{1} + x_{n-2}\dbinom{n-2}{0}(-a){}^{0} = 0 \\
            \vdots \\
            x_{n}\dbinom{n}{k}(-a){}^{k} + x_{n-1}\dbinom{n-1}{k-1}(-a){}^{k-1} + \cdots + x_{n-k}\dbinom{n-k}{0}(-a){}^{0} = 0 \\
            \vdots \\
            x_{n}\dbinom{n}{n}(-a){}^{n} + x_{n-1}\dbinom{n-1}{n-1}(-a){}^{n-1} + \cdots + x_{0}\dbinom{0}{0}(-a){}^{0} = 0
        \end{cases}
    \]
    \par Giải phương trình trên, thay vào phương trình dưới, ta thu được nghiệm duy nhất của hệ là $(x_{n}, x_{n-1},\ldots, x_{0}) = (0,0,\ldots, 0)$.
    \par Như vậy $x_{n} = x_{n-1} = \cdots = x_{1} = x_{0} = 0$ nên hệ vector $(1, (X - a), (X - a){}^{2}, \ldots, (X - a){}^{n})$ độc lập tuyến tính.
    \par Không gian vector $\mathbb{R}[X]{}_{n}$ hữu hạn sinh nên $(1, (X-a), (X-a){}^{2},\ldots, (X-a){}^{n})$ cũng là một cơ sở của $\mathbb{R}[X]{}_{n}$.

    \bigskip
    \par Bằng khai triển Newton, ta thu được ma trận chuyển cơ sở từ $(1, X, X^{2}, \ldots, X^{n})$ sang $(1, (X-a), (X-a){}^{2}, \ldots, (X-a){}^{n})$ là:
    \[
        \begin{pmatrix}
            1 & \binom{1}{1}(-a) & \binom{2}{2}(-a){}^{2} & \cdots & \binom{n}{n}(-a){}^{n} \\
            0 & 1 & \binom{2}{1}(-a){}^{1} & \cdots & \binom{n}{n-1}(-a){}^{n-1} \\
            0 & 0 & 1 & \cdots & \binom{n}{n-2}(-a){}^{n-2} \\
            \vdots & \vdots & \vdots & \ddots & \vdots \\
            0 & 0 & 0 & \cdots & 1
        \end{pmatrix}
    \]
    \par Các yếu tố của ma trận $(c_{ij}){}_{(n+1)\times(n+1)}$, trong đó, $0\le i, j \le n$ này có thể được xác định nhanh chóng như sau:
    \[
        \begin{cases}
            c_{ij} = \dbinom{j}{i}(-a){}^{j - i}&,\quad\text{if $i\le j$}\\
            c_{ij} = 0&,\quad\text{if $i > j$}
        \end{cases}
    \]
\end{proof}

\begin{exercise}Tìm các tọa độ của đa thức $f(X) = a_{0} + a_{1}X + \cdots + a_{n}X^{n}$ đối với hai cơ sở nói trên.
\end{exercise}

\begin{proof}Tọa độ của $f(X)$ đối với cơ sở $(1, X, \ldots, X^{n})$ là:
    \[ (a_{0}, a_{1}, \ldots, a_{n}) \]
    \par Theo bài toán trên, $(1, (X - a), (X - a){}^{2}, \ldots, (X - a){}^{n})$ cũng là một cơ sở của không gian vector $\mathbb{R}[X]{}_{n}$.
    \par Từng vector trong cơ sở thứ nhất có thể biểu thị tuyến tính duy nhất theo cơ sở thứ hai.
    \[
        \begin{cases}
            1 &= 1 \\
            X &= a + (X-a) \\
            X^{2} &= a^{2} + 2a(X-a) + (X-a){}^{2} \\
            X^{3} &= a^{3} + 3a^{2}(X-a) + 3a(X-a){}^{2} + (X-a){}^{3} \\
            &\vdots \\
            X^{n} &= a^{n} + \binom{n}{1}a^{n-1}(X-a) + \binom{n}{2}a^{n-2}(X-a){}^{2} + \cdots + \binom{n}{n-1}(X-a){}^{n-1} + (X-a){}^{n}
        \end{cases}
    \]
    \par Nhân hai vế của phương trình thứ $k$ với $a_{k}$, rồi cộng vế theo vế, ta thu được
    \[
        f(X) = \sum^{n}_{k=0}b_{k}(X-a){}^{k}
    \]
    \par trong đó:
    \[
        b_{k} = \sum^{n}_{i=k}a_{i}\binom{k+i}{i}a^{i}
    \]
\end{proof}

\begin{exercise}Cho không gian vector con $L$ của không gian $\mathbb{R}[X]$ các đa thức hệ số thực. Chứng minh rằng nếu $L$ chứa ít nhất một đa thức bậc $k$ với mọi $k = 0, 1,\ldots, n$, nhưng không chứa đa thức nào với bậc lớn hơn $n$ thì $L$ chính là không gian con $\mathbb{R}[X]{}_{n}$ tất cả các đa thức bậc không vượt quá $n$.
\end{exercise}

\begin{proof}Đặt các đa thức bậc $0, 1, 2,\ldots, n$ trong $L$ lần lượt là:
    \[
        \begin{cases}
            f_{0}(X) = a_{00} \\
            f_{1}(X) = a_{10} + a_{11}X \\
            f_{2}(X) = a_{20} + a_{21}X + a_{22}X^{2} \\
            \vdots \\
            f_{k}(X) = a_{k0} + a_{k1}X + a_{k2}X^{2} + \cdots + a_{k(k-1)}X^{k-1} + a_{kk}X^{k} \\
            \vdots \\
            f_{n}(X) = a_{n0} + a_{n1}X + a_{n2}X^{2} + \cdots + a_{n(n-1)}X^{n-1} + a_{nn}X^{n}
        \end{cases}
    \]
    \par trong đó, $a_{ii}\ne 0, \forall i\in\{0, 1, 2,\ldots, n\}$
    \par Ta sẽ chứng minh $X^{k}$ có thể biểu thị tuyến tính được theo hệ vector $(f_{0}(X), f_{1}(X), \ldots, f_{n}(X))$, với mọi $k\in\{0,1,\ldots, n\}$. Cụ thể hơn, ta chứng minh điều này bằng quy nạp.
    \[ X^{0} = 1 = a^{-1}_{00}a_{00} = a^{-1}_{00}f_{0}(X) \]
    \par Giả sử điều trên đúng từ $0$ đến $k-1$, với $k-1 < n$.
    \begin{align*}
        &f_{k}(X) = a_{k0} + a_{k1}X + a_{k2}X^{2} + \cdots + a_{k(k-1)}X^{k-1} + a_{kk}X^{k} \\
        \Rightarrow\quad&X^{k} = a^{-1}_{kk}(f_{k}(X) - a_{k0} - a_{k1}X - a_{k2}X^{2} - \cdots a_{k(k-1)}X^{k-1})
    \end{align*}
    \par Theo giả thiết quy nạp, các đa thức $1, X, \ldots, X^{k-1}$ biểu thị tuyến tính được theo $f_{0}(X)$, $f_{1}(X)$, $\ldots$, $f_{n}(X)$ nên đẳng thức trên cho thấy giả thiết vẫn đúng với $k$.
    \par Như vậy, hệ $(1, X, \ldots, X^{n})$ biểu thị tuyến tính được theo $(f_{0}(X), f_{1}(X), \ldots, f_{n}(X))$.
    \par Xét một đa thức bất kỳ với bậc không quá $n$:
        \[ f(X) = a_{0} + a_{1}X + a_{2}X^{2} + \cdots + a_{n}X^{n} \]
    \par Hiển nhiên $f(X)$ biểu thị tuyến tính được theo $(1, X, \ldots, X^{n})$. Quan hệ \textit{biểu thị tuyến tính} có tính bắc cầu nên $f(X)$ cũng biểu thị tuyến tính được theo $(f_{0}(X), f_{1}(X), \ldots, f_{n}(X))$.
    \par Do đó $\mathcal{L}(\{f_{0}(X), f_{1}(X),\ldots, f_{n}(X)\}) = \mathbb{R}[X]{}_{n}$.
    \par $L$ không chứa đa thức nào có bậc lớn hơn $n$, cùng với điều trên, ta kết luận $L = \mathbb{R}[X]{}_{n}$.
\end{proof}

\begin{exercise}Chứng minh rằng tập hợp các vector $(x_{1}, x_{2}, \ldots, x_{n})\in\mathbb{R}_{n}$ thỏa mãn hệ thức $x_{1} + 2x_{2} + \cdots + nx_{n} = 0$ là một không gian vector con của $\mathbb{R}_{n}$. Tìm số chiều và một cơ sở của không gian vector con đó.
\end{exercise}

\begin{proof}Ta chứng minh bài toán tổng quát hơn: \textit{tập hợp $U$ các vector $(x_{1}, x_{2}, \ldots, x_{n})\in\mathbb{R}_{n}$ thỏa mãn hệ thức $\displaystyle\sum^{n}_{i=1}a_{i}x_{i} = 0$ là một không gian vector con của $\mathbb{R}_{n}$}.
    \par $\mathbf{x} = (x_{1}, x_{2}, \ldots, x_{n}), \mathbf{y} = (y_{1}, y_{2}, \ldots, y_{n})$.
    \par $\mathbf{x} + \mathbf{y} = (x_{1} + y_{1}, x_{2} + y_{2}, \ldots, x_{n} + y_{n})$.
    \par $\displaystyle\sum^{n}_{i=0}a_{i}(x_{i} + y_{i}) = \displaystyle\sum^{n}_{i=0}a_{i}x_{i} + \displaystyle\sum^{n}_{i=0}a_{i}y_{i} = 0 + 0 = 0$.
    \par $\displaystyle\sum^{n}_{i=0}a_{i}(ax_{i}) = a\displaystyle\sum^{n}_{i=0}a_{i}x_{i} = a\cdot 0 = 0$.
    \par Do đó $U$ là một không gian vector con của $\mathbb{R}_{n}$.
    \par Xét không gian con $W = \mathcal{L}(\{(a_{1},\ldots, a_{n})\})$. $\dim W = 1$.
    \par $\mathbf{x} = (x_{1}, \ldots, x_{n}) \in \mathbb{R}_{n}$.
        \[
            \underbrace{(x_{1},\ldots, x_{n})}_{\in\mathbb{R}_{n}} = \underbrace{\left(x_{1} - \frac{a_{1}\sum^{n}_{i=1} a_{i}x_{i}}{\sum^{n}_{i=1} a^{2}_{i}},\ldots, x_{n} - \frac{a_{n}\sum^{n}_{i=1}a_{i}x_{i}}{\sum^{n}_{i=1}a^{2}_{i}}\right)}_{\in U} + \underbrace{\frac{\sum^{n}_{i=1}a_{i}x_{i}}{\sum^{n}_{i=1}a^{2}_{i}}(a_{1},\ldots,a_{n})}_{\in W}
        \]
    \par Do đó $\mathbb{R}_{n} = U + W$. Mà $U\cap W = \{ (0,\ldots, 0) \}$ nên $\mathbb{R}_{n} = U\oplus W$.
    \par $\dim\mathbb{R}_{n} = \dim U + \dim W \Rightarrow \dim U = n - \dim W$.
    \par Nếu $(a_{1},\ldots, a_{n}) = (0,\ldots,0)$ thì $\dim W = 0 \Rightarrow \dim U = n$.
    \par Lúc này một cơ sở của $U$ là cơ sở của $\mathbb{R}_{n}$.
    \par Còn nếu $(a_{1},\ldots, a_{n}) \ne (0,\ldots,0)$ thì $\dim W = 1\Rightarrow \dim U = n - 1$.
    \par Không mất tính tổng quát, giả sử $a_{1}\ne 0$, xét hệ vector sau:
    \[
        \begin{cases}
            (-a_{2}, a_{1}, 0, \ldots, 0) \\
            (-a_{3}, 0, a_{1}, \ldots, 0) \\
            \ddots \\
            (-a_{n}, 0, 0, \ldots, a_{1})
        \end{cases}
    \]
    \par Xét biểu thị tuyến tính:
        \[ b_{2}(-a_{2}, a_{1}, 0, \ldots, 0) + b_{3}(-a_{3}, 0, a_{1}, \ldots, 0) + \cdots + b_{n}(-a_{n}, 0, 0, \ldots, a_{1}) = (0,0,0,\ldots, 0) \]
    \par Đồng nhất các yếu tố, ta suy ra $b_{2}a_{1} = b_{3}a_{1} = \cdots = b_{n}a_{1} = 0$ nên $b_{2} = b_{3} = \cdots = b_{n}$ nên hệ vector trên độc lập tuyến tính. Số vector trong hệ này bằng $n - 1$, bằng số chiều của $U$ nên đây là một cơ sở của $U$.
\end{proof}

\begin{exercise}Tìm tất cả các $\mathbb{F}_{2}$-không gian vector con một và hai chiều của $\mathbb{F}^{3}_{2}$. Giải bài toán tương tự đối với không gian $\mathbb{F}^{3}_{p}$, trong đó $p$ là một số nguyên tố.
\end{exercise}

\begin{proof}
\end{proof}

\begin{exercise}Chứng minh rằng các ma trận vuông đối xứng cỡ $n$ với các yếu tố trong trường $\mathbb{F}$ lập thành một không gian vector con của $M(n\times n,\mathbb{F})$. Tìm số chiều và một cơ sở của $\mathbb{F}$-không gian vector con đó.
\end{exercise}

\begin{proof}$P = (p_{ij}){}_{n\times n}, Q = (q_{ij}){}_{n\times n}$ là hai ma trận đối xứng.
    \par $P + Q = (p_{ij} + q_{ij}){}_{n\times n}$.
    \par Vì $p_{ij} = p_{ji}, q_{ij} = q_{ji}$ nên $p_{ij} + q_{ij} = p_{ji} + q_{ji}$, do đó $P + Q$ cũng là ma trận đối xứng.
    \par $aP = (ap_{ij}){}_{n\times n}$.
    \par Vì $p_{ij} = p_{ji}$ nên $ap_{ij} = ap_{ji}$, do đó $aP$ cũng là ma trận đối xứng.
    \par Như vậy tập hợp các ma trận đối xứng cỡ $n$ với các yếu tố trong trường $\mathbb{F}$ là một không gian vector con của $M(n\times n,\mathbb{F})$.
    \par Đặt $S_{ij}$ là ma trận vuông cỡ $n$ sao cho yếu tố hàng $i$, cột $j$ và yếu tố hàng $j$, cột $i$ bằng $1$, các yếu tố còn lại bằng $0$.
    \par $P = \displaystyle\sum_{1\le i\le j \le n}p_{ij}S_{ij}$.
    \par Xét biểu thị tuyến tính $0 = \displaystyle\sum_{1\le i\le j\le n}s_{ij}S_{ij}$.
    \[
        \Rightarrow
        \begin{pmatrix}
            0 & 0 & \cdots & 0 \\
            0 & 0 & \cdots & 0 \\
            \vdots & \vdots & \ddots & \vdots \\
            0 & 0 & \cdots & 0
        \end{pmatrix} =
        \begin{pmatrix}
            s_{11} & s_{12} & \cdots & s_{1n} \\
            s_{12} & s_{22} & \cdots & s_{2n} \\
            \vdots & \vdots & \ddots & \vdots \\
            s_{1n} & s_{2n} & \cdots & s_{nn}
        \end{pmatrix}
    \]
    \par Đồng nhất các yếu tố, ta suy ra $s_{ij} = 0,\forall i, j$, tức là hệ $\{S_{ij}\}$ độc lập tuyến tính.
    \par $P$ luôn biểu thị tuyến tính được theo hệ độc lập tuyến tính $\{S_{ij}\}$ nên đây chính là một cơ sở.
    \par Số chiều của không gian vector các ma trận vuông đối xứng cỡ $n$ là $\dfrac{n(n + 1)}{2} = \dbinom{n+1}{2}$.
\end{proof}

\begin{exercise}Chứng minh rằng các ma trận vuông phản đối xứng cỡ $n$ lập thành một không gian vector con của $M(n\times n,\mathbb{F})$. Tìm số chiều và một cơ sở của $\mathbb{F}$-không gian vector con đó.
\end{exercise}

\begin{proof}
\end{proof}

\end{document}
