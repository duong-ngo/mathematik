\documentclass[class=nhvh-linear-algebra,crop=false]{standalone}

\setcounter{exercise}{0}

\begin{document}

\chapter{Cấu trúc của tự đồng cấu}

\par Nếu không nói gì thêm, ta hiểu các ma trận có các yếu tố thuộc trường số phức.

\begin{exercise}
    \par Tìm giá trị riêng và vector riêng của các tự đồng cấu có ma trận sau đây đối với một cơ sở nào đó của không gian:
    \begin{enumerate}[label = (\alph*)]
        \item $
                  \begin{pmatrix}
                      2  & -1 & 2  \\
                      5  & -3 & 3  \\
                      -1 & 0  & -2
                  \end{pmatrix}
              $,
        \item $
                  \begin{pmatrix}
                      4 & -5 & 2 \\
                      5 & -7 & 3 \\
                      6 & -9 & 4
                  \end{pmatrix}
              $,
        \item $
                  \begin{pmatrix}
                      1  & -3 & 3  \\
                      -2 & -6 & 13 \\
                      -1 & -4 & 8
                  \end{pmatrix}
              $,
        \item $
                  \begin{pmatrix}
                      1 & -3 & 4 \\
                      4 & -7 & 8 \\
                      6 & -7 & 7
                  \end{pmatrix}
              $,
        \item $
                  \begin{pmatrix}
                      1 & 0 & 0 & 0 \\
                      0 & 0 & 0 & 0 \\
                      0 & 0 & 0 & 0 \\
                      1 & 0 & 0 & 1
                  \end{pmatrix}
              $,
        \item $
                  \begin{pmatrix}
                      3 & -1 & 0 & 0  \\
                      1 & 1  & 0 & 0  \\
                      3 & 0  & 5 & -3 \\
                      4 & -1 & 3 & -1
                  \end{pmatrix}
              $.
    \end{enumerate}
\end{exercise}

\begin{proof}[Lời giải]
    \begin{enumerate}[label = (\alph*)]
        \item Đa thức đặc trưng của tự đồng cấu này là:
              \[
                  \begin{vmatrix}
                      2 - X & -1     & 2      \\
                      5     & -3 - X & 3      \\
                      -1    & 0      & -2 - X
                  \end{vmatrix}
                  = -(X+1){}^{3}.
              \]
              \par Tự đồng cấu có giá trị riêng duy nhất $\lambda_{1} = -1$.
              \par Các vector riêng ứng với giá trị riêng $\lambda_{1} = -1$ là nghiệm không tầm thường của hệ phương trình tuyến tính thuần nhất sau:
              \[
                  \begin{pmatrix}
                      3  & -1 & 2  \\
                      5  & -2 & 3  \\
                      -1 & 0  & -1
                  \end{pmatrix}
                  \begin{pmatrix}
                      x \\
                      y \\
                      z
                  \end{pmatrix}
                  =
                  \begin{pmatrix}
                      0 \\
                      0 \\
                      0
                  \end{pmatrix}
                  \Longleftrightarrow
                  \begin{cases}
                      3x - y + 2z = 0 \\
                      -x - z = 0
                  \end{cases}
              \]
              \par Nghiệm tổng quát của hệ phương trình tuyến tính này là $(x, y, z) = (a, a, -a)$.
              \par Vậy các vector riêng của tự đồng cấu là các vector $a(1, 1, -1)$, trong đó $a\ne 0$.
        \item Đa thức đặc trưng của tự đồng cấu này là:
              \[
                  \begin{vmatrix}
                      4 - X & -5     & 2     \\
                      5     & -7 - X & 3     \\
                      6     & -9     & 4 - X
                  \end{vmatrix}
                  = X^{2}(1 - X).
              \]
              \par Tự đồng cấu có hai giá trị riêng $\lambda_{1} = 0$, $\lambda_{2} = 1$.
              \par Các vector riêng ứng với giá trị riêng $\lambda_{1} = 0$ là nghiệm không tầm thường của hệ phương trình tuyến tính thuần nhất sau:
              \[
                  \begin{pmatrix}
                      4 & -5 & 2 \\
                      5 & -7 & 3 \\
                      6 & -9 & 4
                  \end{pmatrix}
                  \begin{pmatrix}
                      x \\
                      y \\
                      z
                  \end{pmatrix}
                  =
                  \begin{pmatrix}
                      0 \\
                      0 \\
                      0
                  \end{pmatrix}
                  \Longleftrightarrow
                  \begin{cases}
                      3y - 2z = 0 \\
                      x - 2y + z = 0
                  \end{cases}.
              \]
              \par Nghiệm tổng quát của hệ phương trình tuyến tính này là $(x, y, z) = (a, 2a, 3a)$, suy ra các vector riêng của tự đồng cấu ứng với giá trị riêng $\lambda_{1} = 0$ là các vector $a(1, 2, 3)$, trong đó $a \ne 0$.
              \bigskip
              \par Các vector riêng ứng với giá trị riêng $\lambda_{2} = 1$ là nghiệm không tầm thường của hệ phương trình tuyến tính thuần nhất sau:
              \[
                  \begin{pmatrix}
                      3 & -5 & 2 \\
                      4 & -7 & 3 \\
                      5 & -9 & 4
                  \end{pmatrix}
                  \begin{pmatrix}
                      x \\
                      y \\
                      z
                  \end{pmatrix}
                  =
                  \begin{pmatrix}
                      0 \\
                      0 \\
                      0
                  \end{pmatrix}
                  \Longleftrightarrow
                  \begin{cases}
                      x - 2y + z = 0 \\
                      y - z = 0
                  \end{cases}.
              \]
              \par Nghiệm tổng quát của hệ phương trình tuyến tính này là $(x, y, z) = (a, a, a)$, suy ra các vector riêng của tự đồng cấu ứng với giá trị riêng $\lambda_{2} = 1$ là các vector $a(1, 1, 1)$, trong đó $a \ne 0$.
        \item Đa thức đặc trưng của tự đồng cấu này là:
              \[
                  \begin{vmatrix}
                      1 - X & -3     & 3     \\
                      -2    & -6 - X & 13    \\
                      -1    & -4     & 8 - X
                  \end{vmatrix}
                  = (1-X){}^{3}.
              \]
              \par Tự đồng cấu có một giá trị riêng $\lambda_{1} = 1$.
              \par Các vector riêng ứng với giá trị riêng $\lambda_{1}$ là nghiệm không tầm thường của hệ phương trình tuyến tính thuần nhất:
              \[
                  \begin{pmatrix}
                      0  & -3 & 3  \\
                      -2 & -7 & 13 \\
                      -1 & -4 & 7
                  \end{pmatrix}
                  \begin{pmatrix}
                      x \\
                      y \\
                      z
                  \end{pmatrix}
                  =
                  \begin{pmatrix}
                      0 \\
                      0 \\
                      0
                  \end{pmatrix}
                  \Longleftrightarrow
                  \begin{cases}
                      y = z \\
                      x = 3y
                  \end{cases}.
              \]
              \par Nghiệm tổng quát của hệ phương trình tuyến tính này là $(x, y, z) = (3a, a, a)$, suy ra các vector riêng ứng với giá trị riêng $\lambda_{1} = 1$ là $a(3, 1, 1)$
        \item Đa thức đặc trưng của tự đồng cấu này là:
              \[
                  \begin{vmatrix}
                      1 - X & -3     & 4     \\
                      4     & -7 - X & 8     \\
                      6     & -7     & 7 - X
                  \end{vmatrix}
                  = (3 - X)(X + 1){}^{2}.
              \]
              \par Tự đồng cấu có hai giá trị riêng $\lambda_{1} = 3$, $\lambda_{2} = -1$.
              \par Các vector riêng ứng với giá trị riêng $\lambda_{1} = 3$ là nghiệm của hệ phương trình tuyến tính thuần nhất:
              \[
                  \begin{pmatrix}
                      -2 & -3  & 4 \\
                      4  & -10 & 8 \\
                      6  & -7  & 4
                  \end{pmatrix}
                  \begin{pmatrix}
                      x \\
                      y \\
                      z
                  \end{pmatrix}
                  =
                  \begin{pmatrix}
                      0 \\
                      0 \\
                      0
                  \end{pmatrix}
                  \Longleftrightarrow
                  \begin{cases}
                      -y + z = 0 \\
                      6x - 7y + 4z = 0
                  \end{cases}.
              \]
              \par Nghiệm tổng quát của hệ phương trình tuyến tính này là $(x, y, z) = (a, 2a, 2a)$, suy ra các vector riêng ứng với giá trị riêng $\lambda_{1} = 3$ là $a(1, 2, 2)$, trong đó $a \ne 0$.
              \bigskip
              \par Các vector riêng ứng với giá trị riêng $\lambda_{2} = -1$ là nghiệm không tầm thường của hệ phương trình tuyến tính thuần nhất:
              \[
                  \begin{pmatrix}
                      2 & -3 & 4 \\
                      4 & -6 & 8 \\
                      6 & -7 & 8
                  \end{pmatrix}
                  \begin{pmatrix}
                      x \\
                      y \\
                      z
                  \end{pmatrix}
                  =
                  \begin{pmatrix}
                      0 \\
                      0 \\
                      0
                  \end{pmatrix}
                  \Longleftrightarrow
                  \begin{cases}
                      y - 2z = 0 \\
                      2x - y = 0
                  \end{cases}.
              \]
              \par Nghiệm tổng quát của hệ phương trình tuyến tính này là $(x, y, z) = (a, 2a, a)$, suy ra các vector riêng ứng với giá trị riêng $\lambda_{2} = -1$ là $a(1, 2, 1)$, trong đó $a \ne 0$.
        \item Đa thức đặc trưng của tự đồng cấu này là:
              \[
                  \begin{vmatrix}
                      1 - X & 0  & 0  & 0  \\
                      0     & -X & 0  & 0  \\
                      0     & 0  & -X & 0  \\
                      1     & 0  & 0  & -X
                  \end{vmatrix}
                  = X^{2}(X-1){}^{2}.
              \]
              \par Tự đồng cấu có hai giá trị riêng $\lambda_{1} = 0$, $\lambda_{2} = 1$.
              \par Các vector riêng tương ứng với giá trị riêng $\lambda_{1} = 0$ là nghiệm không tầm thường của hệ phương trình tuyến tính thuần nhất:
              \[
                  \begin{pmatrix}
                      1 & 0 & 0 & 0 \\
                      0 & 0 & 0 & 0 \\
                      0 & 0 & 0 & 0 \\
                      1 & 0 & 0 & 1
                  \end{pmatrix}
                  \begin{pmatrix}
                      x \\
                      y \\
                      z \\
                      w
                  \end{pmatrix}
                  =
                  \begin{pmatrix}
                      0 \\
                      0 \\
                      0 \\
                      0
                  \end{pmatrix}
                  \Longleftrightarrow
                  \begin{cases}
                      x = 0 \\
                      x + w = 0
                  \end{cases}.
              \]
              \par Nghiệm tổng quát của hệ phương trình tuyến tính trên là $(x, y, z, w) = (0, a, b, 0)$, do đó các vector riêng ứng với giá trị riêng $\lambda_{1} = 0$ là $a(0, 1, 0, 0) + b(0, 0, 1, 0)$, trong đó $(a, b)\ne (0, 0)$.
              \bigskip
              \par Các vector riêng ứng với giá trị riêng $\lambda_{2} = 1$ là nghiệm không tầm thường của hệ phương trình tuyến tính thuần nhất:
              \[
                  \begin{pmatrix}
                      0 & 0  & 0  & 0 \\
                      0 & -1 & 0  & 0 \\
                      0 & 0  & -1 & 0 \\
                      1 & 0  & 0  & 0
                  \end{pmatrix}
                  \begin{pmatrix}
                      x \\
                      y \\
                      z \\
                      w
                  \end{pmatrix}
                  =
                  \begin{pmatrix}
                      0 \\
                      0 \\
                      0 \\
                      0
                  \end{pmatrix}
                  \Longleftrightarrow
                  \begin{cases}
                      y = 0 \\
                      z = 0 \\
                      x = 0
                  \end{cases}.
              \]
              \par Nghiệm tổng quát của hệ phương trình tuyến tính trên là $(x, y, z, w) = (0, 0, 0, a)$, do đó các vector riêng ứng với giá trị riêng $\lambda_{2} = 1$ là $a(0, 0, 0, 1)$, trong đó $a\ne 0$.
        \item Đa thức đặc trưng của tự đồng cấu này là:
              \[
                  \begin{vmatrix}
                      3 - X & -1    & 0     & 0      \\
                      1     & 1 - X & 0     & 0      \\
                      3     & 0     & 5 - X & -3     \\
                      4     & -1    & 3     & -1 - X
                  \end{vmatrix}
                  = (X - 2){}^{4}.
              \]
              \par Tự đồng cấu có một giá trị riêng là $\lambda_{1} = 2$.
              \par Các vector riêng ứng với giá trị riêng $\lambda_{1} = 2$ là nghiệm không tầm thường của hệ phương trình tuyến tính thuần nhất:
              \[
                  \begin{pmatrix}
                      1 & -1 & 0 & 0  \\
                      1 & -1 & 0 & 0  \\
                      3 & 0  & 3 & -3 \\
                      4 & -1 & 3 & -3
                  \end{pmatrix}
                  \begin{pmatrix}
                      x \\
                      y \\
                      z \\
                      w
                  \end{pmatrix}
                  =
                  \begin{pmatrix}
                      0 \\
                      0 \\
                      0 \\
                      0
                  \end{pmatrix}
                  \Longleftrightarrow
                  \begin{cases}
                      x = y \\
                      x + z - w = 0
                  \end{cases}.
              \]
              \par Nghiệm tổng quát của hệ phương trình tuyến tính trên là $(x, y, z, w) = (b-a, b-a, a, b)$, do đó các vector riêng ứng với các giá trị riêng $\lambda_{1} = 2$ là $(b-a)(1, 1, 0, 0) + a(0, 0, 1, 0) + b(0, 0, 0, 1)$, trong đó $(a, b)\ne (0, 0)$.
    \end{enumerate}
\end{proof}

\begin{exercise}
    \par Chứng minh rằng nếu tự đồng cấu $\varphi$ của không gian vector $n$ chiều $V$ có $n$ giá trị riêng khác nhau và $\psi$ là một tự đồng cấu giao hoán với $\varphi$, thì mỗi vector riêng của $\varphi$ cũng là một vector riêng của $\psi$ và $V$ có một cơ sở gồm toàn vector riêng của $\psi$.
\end{exercise}

\begin{proof}
    \par Gọi $n$ giá trị riêng phân biệt của $\varphi$ là $\lambda_{1}$, $\lambda_{2}$, \ldots, $\lambda_{n}$. $\alpha_{1}$, $\alpha_{2}$, \ldots, $\alpha_{n}$ lần lượt là các vector riêng tương ứng với các giá trị riêng $\lambda_{1}$, $\lambda_{2}$, \ldots, $\lambda_{n}$.
    \par Xét ràng buộc tuyến tính:
    \begin{align*}
                         & a_{1}\alpha_{1} + a_{2}\alpha_{2} + \cdots + a_{n}\alpha_{n} = 0                                                                                                                                                                                         \\
        \Rightarrow\quad & a_{1}\varphi(\alpha_{1}) + a_{2}\varphi(\alpha_{2}) + \cdots + a_{n}\varphi(\alpha_{n}) = 0                                                                                                                                                              \\
        \Rightarrow\quad & a_{1}\lambda_{1}\alpha_{1} + a_{2}\lambda_{2}\alpha_{2} + \cdots + a_{n}\lambda_{n}\alpha_{n} = 0                                                                                                                                                        \\
        \Rightarrow\quad & a_{1}(\lambda_{1} - \lambda_{n})\alpha_{1} + a_{2}(\lambda_{2} - \lambda_{n})\alpha_{2} + \cdots + a_{n-1}(\lambda_{n-1} - \lambda_{n})\alpha_{n-1} = 0 \tag{1}                                                                                          \\
        \Rightarrow\quad & a_{1}(\lambda_{1} - \lambda_{n})\varphi(\alpha_{1}) + a_{2}(\lambda_{2} - \lambda_{n})\varphi(\alpha_{2}) + \cdots + a_{n-1}(\lambda_{n-1} - \lambda_{n})\varphi(\alpha_{n-1}) = 0                                                                       \\
        \Rightarrow\quad & a_{1}(\lambda_{1} - \lambda_{n})\lambda_{1}\alpha_{1} + a_{2}(\lambda_{2} - \lambda_{n})\lambda_{2}\alpha_{2} + \cdots + a_{n-1}(\lambda_{n-1} - \lambda_{n})\lambda_{n-1}\alpha_{n-1} = 0                                                               \\
        \Rightarrow\quad & a_{1}(\lambda_{1} - \lambda_{n})(\lambda_{1} - \lambda_{n-1})\alpha_{1} + a_{2}(\lambda_{2} - \lambda_{n})(\lambda_{2} - \lambda_{n-1})\alpha_{2} + \cdots + a_{n-2}(\lambda_{n-2} - \lambda_{n})(\lambda_{n-2} - \lambda_{n-1})\alpha_{n-2} = 0 \tag{2} \\
                         & \ddots
    \end{align*}
    \par Tiếp tục quá trình này, ta thu được các ràng buộc tuyến tính sau:
    \begin{align*}
         & a_{1}\prod_{1 < i \le n}(\lambda_{1} - \lambda_{i})\cdot\alpha_{1}                                                                                  & = 0 \\
         & a_{1}\prod_{2 < i \le n}(\lambda_{1} - \lambda_{i})\cdot\alpha_{1} + a_{2}\prod_{2 < i\le n}(\lambda_{2} - \lambda_{i})\cdot\alpha_{2}              & = 0 \\
         & \vdots                                                                                                                                                    \\
         & a_{1}(\lambda_{1} - \lambda_{n})\alpha_{1} + a_{2}(\lambda_{2} - \lambda_{n})\alpha_{2} + \cdots + a_{n-1}(\lambda_{n-1} - \lambda_{n})\alpha_{n-1} & = 0 \\
         & a_{1}\alpha_{1} + a_{2}\alpha_{2} + \cdots + a_{n}\alpha_{n}                                                                                        & = 0
    \end{align*}
    \par Vì các giá trị riêng đôi một khác nhau và các vector riêng khác không nên $a_{1} = a_{2} = \cdots = a_{n} = 0$.
    \par Do đó các vector $\alpha_{1}$, $\alpha_{2}$, \ldots, $\alpha_{n}$ độc lập tuyến tính.
    \par Mà $\dim V = n$ nên $\alpha_{1}$, $\alpha_{2}$, \ldots, $\alpha_{n}$ là một cơ sở của $V$.
    \par Như vậy ma trận của $\varphi$ đối với cơ sở $\alpha_{1}$, $\alpha_{2}$, \ldots, $\alpha_{n}$ là ma trận chéo sau đây:
    \[
        \begin{pmatrix}
            \lambda_{1} & 0           & \cdots & 0           \\
            0           & \lambda_{2} & \cdots & 0           \\
            \vdots      & \vdots      & \ddots & \vdots      \\
            0           & 0           & \cdots & \lambda_{n}
        \end{pmatrix}.
    \]
    \par Đặt ma trận của $\psi$ đối với cơ sở $\alpha_{1}$, $\alpha_{2}$, \ldots, $\alpha_{n}$ là $(a_{ij}){}_{n\times n}$. Do $\varphi\psi = \psi\varphi$ nên:
    \begingroup
    \allowdisplaybreaks{}
    \begin{align*}
        \begin{pmatrix}
            \lambda_{1} & 0           & \cdots & 0           \\
            0           & \lambda_{2} & \cdots & 0           \\
            \vdots      & \vdots      & \ddots & \vdots      \\
            0           & 0           & \cdots & \lambda_{n}
        \end{pmatrix}
        \begin{pmatrix}
            a_{11} & a_{12} & \cdots & a_{1n} \\
            a_{21} & a_{22} & \cdots & a_{2n} \\
            \vdots & \vdots & \ddots & \vdots \\
            a_{n1} & a_{n2} & \cdots & a_{nn}
        \end{pmatrix}
         & =
        \begin{pmatrix}
            a_{11} & a_{12} & \cdots & a_{1n} \\
            a_{21} & a_{22} & \cdots & a_{2n} \\
            \vdots & \vdots & \ddots & \vdots \\
            a_{n1} & a_{n2} & \cdots & a_{nn}
        \end{pmatrix}
        \begin{pmatrix}
            \lambda_{1} & 0           & \cdots & 0           \\
            0           & \lambda_{2} & \cdots & 0           \\
            \vdots      & \vdots      & \ddots & \vdots      \\
            0           & 0           & \cdots & \lambda_{n}
        \end{pmatrix} \\
        \Longleftrightarrow
        \begin{pmatrix}
            \lambda_{1}a_{11} & \lambda_{1}a_{12} & \cdots & \lambda_{1}a_{1n} \\
            \lambda_{2}a_{21} & \lambda_{2}a_{22} & \cdots & \lambda_{2}a_{2n} \\
            \vdots            & \vdots            & \ddots & \vdots            \\
            \lambda_{n}a_{n1} & \lambda_{n}a_{n2} & \cdots & \lambda_{n}a_{nn}
        \end{pmatrix}
         & =
        \begin{pmatrix}
            \lambda_{1}a_{11} & \lambda_{2}a_{12} & \cdots & \lambda_{n}a_{1n} \\
            \lambda_{1}a_{21} & \lambda_{2}a_{22} & \cdots & \lambda_{n}a_{2n} \\
            \vdots            & \vdots            & \ddots & \vdots            \\
            \lambda_{1}a_{n1} & \lambda_{2}a_{n2} & \cdots & \lambda_{n}a_{nn}
        \end{pmatrix}.
    \end{align*}
    \endgroup
    \par Đồng nhất các yếu tố của hai ma trận trên, ta được $a_{ij}\lambda_{i} = a_{ij}\lambda_{j}$.
    \par Nếu $i\ne j$ thì $\lambda_{i}\ne\lambda_{j}$, kéo theo $a_{ij} = 0, \forall i\ne j$.
    \par Như vậy, ma trận của $\psi$ đối với cơ sở $\alpha_{1}$, $\alpha_{2}$, \ldots, $\alpha_{n}$ là:
    \[
        \begin{pmatrix}
            a_{11} & 0      & \cdots & 0      \\
            0      & a_{22} & \cdots & 0      \\
            \vdots & \vdots & \ddots & \vdots \\
            0      & 0      & \cdots & a_{nn}
        \end{pmatrix}.
    \]
    \par Suy ra $\psi(\alpha_{1}) = a_{11}\alpha_{1}$, $\psi(\alpha_{2}) = a_{22}\alpha_{2}$, \ldots, $\psi(\alpha_{n}) = a_{nn}\alpha_{n}$.
    \par Do đó, vector riêng của $\varphi$ cũng là vector riêng của $\psi$ và $V$ có một cơ sở gồm toàn vector riêng của $\psi$.
\end{proof}

\begin{exercise}
    \par Xác định xem những tự đồng cấu được cho bởi các ma trận sau đối với một cơ sở nào đó của không gian vector $V$ có chéo hóa được không. Nếu có, hãy xác định cơ sở, đối với nó ma trận của tự đồng cấu có dạng chéo và xác định ma trận này.
    \begin{enumerate}[label = (\alph*)]
        \item $
                  \begin{pmatrix}
                      -1 & 3 & -1 \\
                      -3 & 5 & -1 \\
                      -3 & 3 & 1
                  \end{pmatrix}
              $,
        \item $
                  \begin{pmatrix}
                      6 & -5 & -3 \\
                      3 & -2 & -2 \\
                      2 & -2 & 0
                  \end{pmatrix}
              $,
        \item $
                  \begin{pmatrix}
                      1 & 1  & 1  & 1  \\
                      1 & 1  & -1 & -1 \\
                      1 & -1 & 1  & -1 \\
                      1 & -1 & -1 & 1
                  \end{pmatrix}
              $,
        \item $
                  \begin{pmatrix}
                      4 & -3  & 1 & 2 \\
                      5 & -8  & 5 & 4 \\
                      6 & -12 & 8 & 5 \\
                      4 & -3  & 2 & 2
                  \end{pmatrix}
              $.
    \end{enumerate}
\end{exercise}

\begin{proof}
    \begin{enumerate}[label = (\alph*)]
        \item Tự đồng cấu $f$ có đa thức đặc trưng:
              \[
                  p_{f}(X) =
                  \begin{vmatrix}
                      -1-X & 3   & -1  \\
                      -3   & 5-X & -1  \\
                      -3   & 3   & 1-X
                  \end{vmatrix}
                  = (1-X)(2-X){}^{2}.
              \]
              \par $p_{f}(X)$ có nghiệm $\lambda_{1} = 1$ (bội 1), $\lambda_{2} = 2$ (bội 2).
              \par Không gian con riêng tương ứng với giá trị riêng $\lambda_{1} = 1$ là không gian nghiệm của hệ phương trình tuyến tính thuần nhất sau:
              \[
                  \begin{pmatrix}
                      -2 & 3 & -1 \\
                      -3 & 4 & -1 \\
                      -3 & 3 & 0
                  \end{pmatrix}
                  \begin{pmatrix}
                      x \\ y \\ z
                  \end{pmatrix}
                  =
                  \begin{pmatrix}
                      0 \\ 0 \\ 0
                  \end{pmatrix}
                  \Longleftrightarrow
                  \begin{cases}
                      -x + y = 0 \\
                      -2x + 3y - z = 0
                  \end{cases}
              \]
              \par Do đó, $\ker (f - 1\cdot \text{id}_{V}) = \text{span}\{ (1, 1, 1) \}$.
              \par Không gian con riêng tương ứng với giá trị riêng $\lambda_{2} = 2$ là không gian nghiệm của hệ phương trình tuyến tính thuần nhất sau:
              \[
                  \begin{pmatrix}
                      -3 & 3 & -1 \\
                      -3 & 3 & -1 \\
                      -3 & 3 & -1
                  \end{pmatrix}
                  \begin{pmatrix}
                      x \\ y \\ z
                  \end{pmatrix}
                  =
                  \begin{pmatrix}
                      0 \\ 0 \\ 0
                  \end{pmatrix}
                  \Longleftrightarrow
                  -3x + 3y - z = 0
              \]
              \par Do đó, $\ker (f - 2\cdot \text{id}_{V}) = \text{span}\{ (1, 0, -3), (0, 1, 3) \}$.
              \bigskip
              \par $\dim\ker (f - 1\cdot\text{id}_{V}) = 1$, $\dim\ker (f - 2\cdot\text{id}_{V}) = 2$ nên theo định lý 3.7, $f$ chéo hóa được.
              \par Cơ sở của $V$ sao cho ma trận của $f$ có dạng chéo là:
              \[
                  \begin{pmatrix}
                      1 \\ 1 \\ 1
                  \end{pmatrix},
                  \begin{pmatrix}
                      1 \\ 0 \\ -3
                  \end{pmatrix},
                  \begin{pmatrix}
                      0 \\ 1 \\ 3
                  \end{pmatrix},
              \]
              \par Khi đó, ma trận của $f$ đối với cơ sở mới là:
              \[
                  \begin{pmatrix}
                      1 & 0 & 0 \\
                      0 & 2 & 0 \\
                      0 & 0 & 2
                  \end{pmatrix}.
              \]
        \item Tự đồng cấu $f$ có đa thức đặc trưng:
              \[
                  p_{f}(X) =
                  \begin{vmatrix}
                      6-X & -5   & -3 \\
                      3   & -2-X & -2 \\
                      2   & -2   & -X
                  \end{vmatrix}
                  = (2-X)(1-X){}^{2}.
              \]
              \par $p_{f}(X)$ có nghiệm $\lambda_{1} = 2$ (bội 1), $\lambda_{2} = 1$ (bội 2).
              \par $\ker (f - 2\cdot\text{id}_{V})$ là không gian nghiệm của hệ phương trình tuyến tính thuần nhất:
              \[
                  \begin{pmatrix}
                      4 & -5 & -3 \\
                      3 & -4 & -2 \\
                      2 & -2 & -2
                  \end{pmatrix}
                  \begin{pmatrix}
                      x \\ y \\ z
                  \end{pmatrix}
                  =
                  \begin{pmatrix}
                      0 \\ 0 \\ 0
                  \end{pmatrix}
                  \Longleftrightarrow
                  \begin{cases}
                      -y + z = 0 \\
                      x - y - z = 0
                  \end{cases}.
              \]
              \par Do đó $\ker (f - 2\cdot\text{id}_{V}) = \text{span}\{ (2, 1, 1) \}$.
              \bigskip
              \par $\ker (f - 1\cdot\text{id}_{V})$ là không gian nghiệm của hệ phương trình tuyến tính thuần nhất:
              \[
                  \begin{pmatrix}
                      5 & -5 & -3 \\
                      3 & -3 & -2 \\
                      2 & -2 & -1
                  \end{pmatrix}
                  \begin{pmatrix}
                      x \\ y \\ z
                  \end{pmatrix}
                  =
                  \begin{pmatrix}
                      0 \\ 0 \\ 0
                  \end{pmatrix}
                  \Longleftrightarrow
                  \begin{cases}
                      2x - 2y - z = 0 \\
                      x  - y - z = 0
                  \end{cases}.
              \]
              \par Do đó $\ker (f - 1\cdot\text{id}_{V}) = \text{span}\{ (1, 1, 0) \}$.
              \bigskip
              \par $\dim\ker (f - 2\cdot\text{id}_{V}) = 1 = 3 - 2$, $\dim\ker (f - 1\cdot\text{id}_{V}) = 1 \ne 3 - 1$.
              \par Theo định lý 3.7, tự đồng cấu $f$ không chéo hóa được.
        \item Tự đồng cấu $f$ có đa thức đặc trưng:
              \[
                  p_{f}(X) =
                  \begin{vmatrix}
                      1-X & 1   & 1   & 1   \\
                      1   & 1-X & -1  & -1  \\
                      1   & -1  & 1-X & -1  \\
                      1   & -1  & -1  & 1-X
                  \end{vmatrix}
                  = (X+2)(X-2){}^{3}.
              \]
              \par $p_{f}(X)$ có các nghiệm $\lambda_{1} = -2$ (bội 1) và $\lambda_{2} = 2$ (bội 3).
              \par $\ker (f - (-2)\text{id}_{V})$ là không gian nghiệm của hệ phương trình tuyến tính thuần nhất:
              \[
                  \begin{pmatrix}
                      3 & 1  & 1  & 1  \\
                      1 & 3  & -1 & -1 \\
                      1 & -1 & 3  & -1 \\
                      1 & -1 & -1 & 3
                  \end{pmatrix}
                  \begin{pmatrix}
                      x \\ y \\ z \\ w
                  \end{pmatrix}
                  =
                  \begin{pmatrix}
                      0 \\ 0 \\ 0 \\ 0
                  \end{pmatrix}
                  \Longleftrightarrow
                  \begin{cases}
                      4y - 4w = 0 \\
                      4z - 4w = 0 \\
                      x - y - z + 3w = 0
                  \end{cases}.
              \]
              \par Do đó $\ker (f - (-2)\cdot\text{id}_{V}) = \text{span}\{ (-1, 1, 1, 1) \}$.
              \bigskip
              \par $\ker (f - 2\cdot\text{id}_{V})$ là không gian nghiệm của hệ phương trình tuyến tính thuần nhất:
              \[
                  \begin{pmatrix}
                      -1 & 1  & 1  & 1  \\
                      1  & -1 & -1 & -1 \\
                      1  & -1 & -1 & -1 \\
                      1  & -1 & -1 & -1
                  \end{pmatrix}
                  \begin{pmatrix}
                      x \\ y \\ z \\ w
                  \end{pmatrix}
                  =
                  \begin{pmatrix}
                      0 \\ 0 \\ 0 \\ 0
                  \end{pmatrix}
                  \Longleftrightarrow
                  x - y - z - w = 0.
              \]
              \par Do đó, $\ker (f - 2\cdot\text{id}_{V}) = \text{span}\{ (1, 1, 0, 0), (1, 0, 1, 0), (1, 0, 0, 1) \}$.
              \par $\dim\ker (f - (-2)\cdot\text{id}_{V}) = 1 = 4 - 3$, $\dim\ker (f - 2\cdot\text{id}_{V}) = 3 = 4 - 1$.
              \par Theo định lý 3.7, $f$ chéo hóa được.
              \par Cơ sở của $V$ sao cho ma trận của $f$ có dạng chéo là:
              \[
                  \begin{pmatrix}
                      -1 \\ 1 \\ 1 \\ 1
                  \end{pmatrix},
                  \begin{pmatrix}
                      1 \\ 1 \\ 0 \\ 0
                  \end{pmatrix},
                  \begin{pmatrix}
                      1 \\ 0 \\ 1 \\ 0
                  \end{pmatrix},
                  \begin{pmatrix}
                      1 \\ 0 \\ 0 \\ 1
                  \end{pmatrix}.
              \]
              \par Khi đó, ma trận của $f$ là:
              \[
                  \begin{pmatrix}
                      -2 & 0 & 0 & 0 \\
                      0  & 2 & 0 & 0 \\
                      0  & 0 & 2 & 0 \\
                      0  & 0 & 0 & 2
                  \end{pmatrix}.
              \]
        \item Tự đồng cấu $f$ có đa thức đặc trưng:
              \[
                  p_{f}(X) =
                  \begin{vmatrix}
                      4-X & -3   & 1   & 2   \\
                      5   & -8-X & 5   & 4   \\
                      6   & -12  & 8-X & 5   \\
                      4   & -3   & 2   & 2-X
                  \end{vmatrix}
                  = (X - 1)(X^{3} - 5X^{2} + 2X + 11)
              \]
              \par $p_{f}(X)$ có 4 nghiệm thực phân biệt. Do đó $f$ chéo hóa được.
    \end{enumerate}
\end{proof}

\begin{exercise}
    \par Cho ma trận vuông cỡ $n$:
    \[
        A =
        \begin{pmatrix}
            0      & 0      & \cdots & 0      & 1      \\
            0      & 0      & \cdots & 1      & 0      \\
            \vdots & \vdots & \ddots & \vdots & \vdots \\
            0      & 1      & \cdots & 0      & 0      \\
            1      & 0      & \cdots & 0      & 0
        \end{pmatrix}.
    \]
    \par Tìm ma trận khả nghịch $T$ sao cho $B = T^{-1}AT$ là một ma trận chéo, và tìm $B$.
\end{exercise}

\begin{proof}[Lời giải]
    \par $f \in \text{End}(V = \mathbb{C}^{n})$ có ma trận $A$ đối với cơ sở nào đó.
    \begin{enumerate}[label = \textbf{Trường hợp \arabic*.}, itemindent=2cm]
        \item $n$ chẵn.
              \par Đặt $n = 2k$.
              \par Áp dụng liên tiếp khai triển Laplace cho hàng đầu tiên và hàng cuối cùng:
              \[
                  \det(A_{2k} - XI_{2k}) = (X^{2} - 1)\det(A_{2k-2} - XI_{2k-2}) = \cdots = (X^{2} - 1){}^{k} = (X - 1){}^{k}(X + 1){}^{k}.
              \]
              \par Như vậy, đa thức đặc trưng của $A$ có nghiệm $\lambda_{1} = 1$ (bội $k$) và $\lambda_{2} = -1$ (bội $k$).
              \par $\ker (f - 1\cdot\text{id}_{V})$ là không gian nghiệm của hệ phương trình tuyến tính thuần nhất:
              \[
                  \begin{pmatrix}
                      -1     & \cdots & 0      & 0      & \cdots & 1      \\
                      \vdots & \ddots & \vdots & \vdots & \ddots & \vdots \\
                      0      & \cdots & -1     & 1      & \cdots & 0      \\
                      0      & \cdots & 1      & -1     & \cdots & 0      \\
                      \vdots & \ddots & \vdots & \vdots & \ddots & \vdots \\
                      1      & \cdots & 0      & 0      & \cdots & -1
                  \end{pmatrix}
                  \begin{pmatrix}
                      x_{1} \\ \vdots \\  x_{k} \\ x_{k+1} \\ \vdots \\ x_{2k}
                  \end{pmatrix}
                  =
                  \begin{pmatrix}
                      0 \\ \vdots \\ 0 \\ 0 \\ \vdots \\ 0
                  \end{pmatrix}
                  \Longleftrightarrow
                  \begin{cases}
                      x_{1} - x_{2k} = 0   \\
                      x_{2} - x_{2k-1} = 0 \\
                      \vdots               \\
                      x_{k} - x_{k+1} = 0
                  \end{cases}
              \]
              \[
                  \Rightarrow\ker (f - 1\cdot\text{id}_{V}) = \text{span}\left\{
                  \begin{pmatrix}
                      1 \\ \vdots \\ 0 \\ 0 \\ \vdots \\ 1
                  \end{pmatrix},
                  \cdots,
                  \begin{pmatrix}
                      0 \\ \vdots \\ 1 \\ 1 \\ \vdots \\ 0
                  \end{pmatrix}
                  \right\}
              \]
              \par $\ker (f - (-1)\cdot\text{id}_{V})$ là không gian nghiệm của hệ phương trình tuyến tính thuần nhất:
              \[
                  \begin{pmatrix}
                      1      & \cdots & 0      & 0      & \cdots & 1      \\
                      \vdots & \ddots & \vdots & \vdots & \ddots & \vdots \\
                      0      & \cdots & 1      & 1      & \cdots & 0      \\
                      0      & \cdots & 1      & 1      & \cdots & 0      \\
                      \vdots & \ddots & \vdots & \vdots & \ddots & \vdots \\
                      1      & \cdots & 0      & 0      & \cdots & 1
                  \end{pmatrix}
                  \begin{pmatrix}
                      x_{1} \\ \vdots \\  x_{k} \\ x_{k+1} \\ \vdots \\ x_{2k}
                  \end{pmatrix}
                  =
                  \begin{pmatrix}
                      0 \\ \vdots \\ 0 \\ 0 \\ \vdots \\ 0
                  \end{pmatrix}
                  \Longleftrightarrow
                  \begin{cases}
                      x_{1} + x_{2k} = 0   \\
                      x_{2} + x_{2k-1} = 0 \\
                      \vdots               \\
                      x_{k} + x_{k+1} = 0
                  \end{cases}
              \]
              \[
                  \Rightarrow\ker (f - (-1)\cdot\text{id}_{V}) = \text{span}\left\{
                  \begin{pmatrix}
                      1 \\ \vdots \\ 0 \\ 0 \\ \vdots \\ -1
                  \end{pmatrix},
                  \cdots,
                  \begin{pmatrix}
                      0 \\ \vdots \\ 1 \\ -1 \\ \vdots \\ 0
                  \end{pmatrix}
                  \right\}
              \]
              \par Mà $\dim\ker (f - 1\cdot\text{id}_{V}) = k$, $\dim\ker (f - (-1)\cdot\text{id}_{V}) = k$ nên theo định lý 3.7, $A$ chéo hóa được.
              \par $B = T^{-1}AT$, trong đó:
              \[
                  T = \begin{pmatrix}
                      1      & \cdots & 0      & 0      & \cdots & 1      \\
                      \vdots & \ddots & \vdots & \vdots & \ddots & \vdots \\
                      0      & \cdots & 1      & 1      & \cdots & 0      \\
                      0      & \cdots & 1      & -1     & \cdots & 0      \\
                      \vdots & \ddots & \vdots & \vdots & \ddots & \vdots \\
                      1      & \cdots & 0      & 0      & \cdots & -1
                  \end{pmatrix}
                  \qquad
                  B = \begin{pmatrix}
                      1      & \cdots & 0      & 0      & \cdots & 0      \\
                      \vdots & \ddots & \vdots & \vdots & \ddots & \vdots \\
                      0      & \cdots & 1      & 0      & \cdots & 0      \\
                      0      & \cdots & 0      & -1     & \cdots & 0      \\
                      \vdots & \ddots & \vdots & \vdots & \ddots & \vdots \\
                      0      & \cdots & 0      & 0      & \cdots & -1
                  \end{pmatrix}
              \]
        \item $n$ lẻ.
              \par Đặt $n = 2k - 1$.
              \par Áp dụng liên tiếp khai triển Laplace cho hàng đầu tiên và hàng cuối cùng:
              \[
                  \det(A_{2k-1} - XI_{2k-1}) = (X^{2} - 1)\det(A_{2k-3} - XI_{2k-3}) = \cdots = (X^{2} - 1){}^{k-1}(1 - X) = (-1)(X-1){}^{k}(X+1){}^{k-1}
              \]
              \par Đa thức đặc trưng của $A$ có nghiệm $\lambda_{1} = 1$ (bội $k$) và $\lambda_{2} = -1$ (bội $k - 1$).
              \par $\ker (f - 1\cdot\text{id}_{V})$ là không gian nghiệm của hệ phương trình tuyến tính thuần nhất:
              \[
                  \begin{pmatrix}
                      -1     & \cdots & 0      & 0      & 0      & \cdots & 1      \\
                      \vdots & \ddots & \vdots & \vdots & \vdots & \ddots & \vdots \\
                      0      & \cdots & -1     & 0      & 1      & \cdots & 0      \\
                      0      & \cdots & 0      & 0      & 0      & \cdots & 0      \\
                      0      & \cdots & 1      & 0      & -1     & \cdots & 0      \\
                      \vdots & \ddots & \vdots & \vdots & \vdots & \ddots & \vdots \\
                      1      & \cdots & 0      & 0      & 0      & 0      & -1
                  \end{pmatrix}
                  \begin{pmatrix}
                      x_{1} \\ \vdots \\ x_{k-1} \\ x_{k} \\ x_{k+1} \\ \vdots \\ x_{2k-1}
                  \end{pmatrix}
                  =
                  \begin{pmatrix}
                      0 \\ \vdots \\ 0 \\ 0 \\ 0 \\ \vdots \\ 0
                  \end{pmatrix}
                  \Longleftrightarrow
                  \begin{cases}
                      x_{1} - x_{2k-1} = 0 \\
                      x_{2} - x_{2k-2} = 0 \\
                      \vdots               \\
                      x_{k-1} - x_{k} = 0  \\
                      0x_{k} = 0
                  \end{cases}
              \]
              \[
                  \Rightarrow\ker (f - 1\cdot\text{id}_{V}) = \text{span}\left\{
                  \begin{pmatrix}
                      1 \\ \vdots \\ 0 \\ 0 \\ 0 \\ \vdots \\ 1
                  \end{pmatrix}
                  \cdots
                  \begin{pmatrix}
                      0 \\ \vdots \\ 1 \\ 0 \\ 1 \\ \vdots \\ 0
                  \end{pmatrix},
                  \begin{pmatrix}
                      0 \\ \vdots \\ 0 \\ 1 \\ 0 \\ \vdots \\ 0
                  \end{pmatrix}
                  \right\}.
              \]
              \par $\ker (f - (-1)\cdot\text{id}_{V})$ là không gian nghiệm của hệ phương trình tuyến tính thuần nhất:
              \[
                  \begin{pmatrix}
                      1      & \cdots & 0      & 0      & 0      & \cdots & 1      \\
                      \vdots & \ddots & \vdots & \vdots & \vdots & \ddots & \vdots \\
                      0      & \cdots & 1      & 0      & 1      & \cdots & 0      \\
                      0      & \cdots & 0      & 2      & 0      & \cdots & 0      \\
                      0      & \cdots & 1      & 0      & 1      & \cdots & 0      \\
                      \vdots & \ddots & \vdots & \vdots & \vdots & \ddots & \vdots \\
                      1      & \cdots & 0      & 0      & 0      & 0      & 1
                  \end{pmatrix}
                  \begin{pmatrix}
                      x_{1} \\ \vdots \\ x_{k-1} \\ x_{k} \\ x_{k+1} \\ \vdots \\ x_{2k-1}
                  \end{pmatrix}
                  =
                  \begin{pmatrix}
                      0 \\ \vdots \\ 0 \\ 0 \\ 0 \\ \vdots \\ 0
                  \end{pmatrix}
                  \Longleftrightarrow
                  \begin{cases}
                      x_{1} + x_{2k-1} = 0 \\
                      x_{2} + x_{2k-2} = 0 \\
                      \vdots               \\
                      x_{k-1} + x_{k} = 0  \\
                      2x_{k} = 0
                  \end{cases}
              \]
              \[
                  \Rightarrow\ker (f - 1\cdot\text{id}_{V}) = \text{span}\left\{
                  \begin{pmatrix}
                      1 \\ \vdots \\ 0 \\ 0 \\ 0 \\ \vdots \\ -1
                  \end{pmatrix}
                  \cdots
                  \begin{pmatrix}
                      0 \\ \vdots \\ 1 \\ 0 \\ -1 \\ \vdots \\ 0
                  \end{pmatrix}
                  \right\}.
              \]
              \par Mà $\dim\ker (f - 1\cdot\text{id}_{V}) = k$, $\dim\ker (f - (-1)\cdot\text{id}_{V}) = k - 1$ nên theo định lý 3.7, $A$ chéo hóa được.
              \par $B = T^{-1}AT$, trong đó:
              \[
                  T =
                  \begin{pmatrix}
                      1      & \cdots & 0      & 0      & 0      & \cdots & 1      \\
                      \vdots & \ddots & \vdots & \vdots & \vdots & \ddots & \vdots \\
                      0      & \cdots & 1      & 0      & 1      & \cdots & 0      \\
                      0      & \cdots & 0      & 1      & 0      & \cdots & 0      \\
                      0      & \cdots & 1      & 0      & -1     & \cdots & 0      \\
                      \vdots & \ddots & \vdots & \vdots & \vdots & \ddots & \vdots \\
                      1      & \cdots & 0      & 0      & 0      & 0      & -1
                  \end{pmatrix}
                  \quad
                  B =
                  \begin{pmatrix}
                      1      & \cdots & 0      & 0      & 0      & \cdots & 0      \\
                      \vdots & \ddots & \vdots & \vdots & \vdots & \ddots & \vdots \\
                      0      & \cdots & 1      & 0      & 0      & \cdots & 0      \\
                      0      & \cdots & 0      & 1      & 0      & \cdots & 0      \\
                      0      & \cdots & 1      & 0      & -1     & \cdots & 0      \\
                      \vdots & \ddots & \vdots & \vdots & \vdots & \ddots & \vdots \\
                      0      & \cdots & 0      & 0      & 0      & 0      & -1
                  \end{pmatrix}.
              \]
    \end{enumerate}
\end{proof}

\begin{exercise}
    \par Ma trận $A$ có các vô hướng $a_{1}$, $a_{2}$, \ldots, $a_{n}$ nằm trên đường chéo thứ hai (theo thứ tự từ hàng một tới hàng $n$) còn tất cả các yếu tố khác bằng 0. Tìm điều kiện để $A$ chéo hóa được.
\end{exercise}

\begin{proof}[Lời giải]
\end{proof}

\end{document}
