% chktex-file 9
% chktex-file 17
\chapter{The Euclidean Topology}

\section{Euclidean Topology}

% chapter 2/section 1/exercise 1
\begin{exercise}
	Prove that if $a, b\in \mathbb{R}$ with $a < b$ then neither $[a, b)$ nor $(a, b]$ is an open subset of $\mathbb{R}$. Also show that neither is a closed subset of $\mathbb{R}$.
\end{exercise}

\begin{proof}
	$a\in [a, b)$. Assume that there exist $c, d\in \mathbb{R}$ such that $a\in (c, d)\subseteq [a, b)$. However, $c < \frac{c+a}{2} < a < d$, then $\frac{c+a}{2}\in (c, d)$ and $\frac{c+a}{2}\notin [a, b)$. So $(c, d)\nsubseteq [a, b)$, which means the assumption is false. Therefore $[a, b)$ is not open.

									The complement of $[a, b)$ in $\mathbb{R}$ is $(-\infty, a)\cup [b, +\infty)$. Assume that there exist $c, d\in \mathbb{R}$ such that $b\in (c, d)\subseteq (-\infty, a)\cup [b, +\infty)$. Let $x = \max\{ a, c \}$, and $y = \frac{x + b}{2}$. Then $a < y$, $c < y$, and $y < b$. So $y\in (c, d)$ but $y\notin (-\infty, a)\cup [b, +\infty)$. So $(c, d)\nsubseteq (-\infty, a)\cup [b, +\infty)$, which means $(-\infty, a)\cup [b, +\infty)$ is not open. Therefore $[a, b)$ is not closed.

																	$b\in (a, b]$. Assume that there exist $c, d\in \mathbb{R}$ such that $b\in (c, d)\subseteq (a, b]$. However, $b, c < \frac{b+d}{2} < d$, then $\frac{b+d}{2}\in (c, d)$ and $\frac{b+d}{2}\notin (a, b]$. So $(c, d)\nsubseteq (a, b]$, which means the assumption is false. Therefore $(a, b]$ is not open.

												The complement of $(a, b]$ in $\mathbb{R}$ is $(-\infty, a]\cup (b, +\infty)$. Assume that there exist $c, d\in \mathbb{R}$ such that $a\in (c, d)\subseteq (-\infty, a]\cup (b, +\infty)$. Let $x = \min\{ b, d \}$ and $y = \frac{a+x}{2}$. Then $x\leq b, d$, $a < y < b$, $c < y < d$, so $y\in (c, d)$ but $y\notin (-\infty, a]\cup (b, +\infty)$. So $(c, d)\nsubseteq (-\infty, a]\cup (b, +\infty)$, which means $(-\infty, a]\cup (b, +\infty)$ is not open. Therefore, $(a, b]$ is not closed.

	Thus $[a, b)$ and $(a, b]$ are not open nor closed.
\end{proof}
\newpage

% chapter 2/section 1/exercise 2
\begin{exercise}
	Prove that the sets $[a, +\infty)$ and $(-\infty, a]$ are closed subsets of $\mathbb{R}$.
\end{exercise}

\begin{proof}
	The complement of $[a, +\infty)$ in $\mathbb{R}$ is $(-\infty, a)$, which is an open set. So $[a, +\infty)$ is a closed subset of $\mathbb{R}$. On the other hand, $[a, +\infty)$ is not an open subset of $\mathbb{R}$ because there do not exist an open interval $(c, d)$ such that $a\in (c, d)\subseteq [a, +\infty)$.

										The complement of $(-\infty, a]$ in $\mathbb{R}$ is $(a, +\infty)$, which is an open set. So $(-\infty, a]$ is a closed subset of $\mathbb{R}$. On the other hand, $(-\infty, a]$ is not an open subset of $\mathbb{R}$ because there do not exist an open interval $(c, d)$ such that $a\in (c, d)\subseteq (-\infty, a]$.
\end{proof}
\newpage

% chapter 2/section 1/exercise 3
\begin{exercise}
	Show, by example, that the union of an infinite number of closed subsets of $\mathbb{R}$ is not necessarily a closed subset of $\mathbb{R}$.
\end{exercise}

\begin{proof}
	I give two examples.

	Example 1. $\bigcup_{q\in\mathbb{Q}} \{ q \} = \mathbb{Q}$. Although any singleton subset of $\mathbb{R}$ is closed, but $\mathbb{Q}$ is not a closed subset of $\mathbb{R}$.

	Example 2. $\bigcup_{x\in\mathbb{R}, x < 1} (-\infty, x] = (-\infty, 1)$. $(-\infty, x]$ is closed because its complement $(x, +\infty)$ is open. $(-\infty, 1)$ is not closed, because its complement $[1, +\infty)$ is not open.
\end{proof}
\newpage

% chapter 2/section 1/exercise 4
\begin{exercise}
	Prove each of the following statements.
	\begin{enumerate}[label={(\roman*)}]
		\item The set $\mathbb{Z}$ of all integers is not an open subset of $\mathbb{R}$.
		\item The set $\mathbb{P}$ of all prime numbers is a closed subset of $\mathbb{R}$ but not an open subset of $\mathbb{R}$.
		\item The set $\mathbb{I}$ of all irrational numbers is neither a closed subset nor an open subset of $\mathbb{R}$.
	\end{enumerate}
\end{exercise}

\begin{proof}
	\begin{enumerate}[label={(\roman*)}]
		\item Assume that there exists an open interval $(a, b)\subseteq \mathbb{Z}$. Let $x$ be an element of $(a, b)$, then $x$ is an integer. If $x+1\in (a, b)$, then so does $x+\frac{1}{2}$. But $x+\frac{1}{2}$ is not an integer. If $x+1\notin (a, b)$, then $x < b < x+1$, $\frac{x+b}{2}$ is not an integer. So the assumption is false. Therefore, $\mathbb{Z}$ does not contain any open interval. Hence $\mathbb{Z}$ is not an open subset of $\mathbb{R}$.
		\item The complement of $\mathbb{P}$ in $\mathbb{R}$ is $(-\infty, 2)\cup\bigcup_{n\in\mathbb{Z}_{>0}} (p_{n}, p_{n+1})$, where $p_{n}$ is the $n$th prime. So $\mathbb{P}$ is closed.

		      Similar to (i), $\mathbb{P}$ does not contain any open interval. So $\mathbb{P}$ is not open.
		\item Assume that there exists an open interval $(a, b)\subseteq \mathbb{Q}$. Due to the density of $\mathbb{Q}$ in $\mathbb{R}$, there exists a rational number $q$ in $(a - \sqrt{2}, b - \sqrt{2})$, so the irrational number $q + \sqrt{2}$ is in $(a, b)$. Therefore $(a, b)\nsubseteq \mathbb{Q}$, which means the assumption is false. So $\mathbb{Q}$ does not contain any open interval. Hence $\mathbb{Q}$ is not an open subset of $\mathbb{R}$, and the complement $\mathbb{I} = \mathbb{R}\setminus\mathbb{Q}$ is not a closed subset of $\mathbb{R}$.

		      Assume that there exists an open interval $(a, b)\subseteq \mathbb{I}$. Due to the density of $\mathbb{Q}$ in $\mathbb{R}$, there exists a rational number in $(a, b)$. Therefore $(a, b)\nsubseteq \mathbb{I}$, which means the assumption is false. So $\mathbb{I}$ does not contain any open interval. Hence $\mathbb{I}$ is not an open subset of $\mathbb{R}$.
	\end{enumerate}
\end{proof}
\newpage

% chapter 2/section 1/exercise 5
\begin{exercise}
	If $F$ is a non-empty finite subset of $\mathbb{R}$, show that $F$ is closed in $\mathbb{R}$ but that $F$ is not open in $\mathbb{R}$.
\end{exercise}

\begin{proof}
	$F = \bigcup_{x\in F}\{ x \}$. Since $\{ x \}$ is a closed subset of $\mathbb{R}$ for any real number $x$ and $F$ is a non-empty finite subset, then $F$ is a union of finitely many closed subsets of $\mathbb{R}$. Therefore $F$ is closed in $\mathbb{R}$.

	Assume that $F$ contains an open interval $(a, b)$. Since $F$ is a non-empty finite subset of $\mathbb{R}$ and $\mathbb{R}$ is a totally ordered set, then $F$ has a least element. Let $c$ be the least element of $F$, then $a < c < b$. On the other hand, $a < \frac{a+c}{2} < c < b$, so $\frac{a+c}{2}\notin F$, which contradicts $(a, b)\subseteq F$. Hence the assumption is false, and $F$ does not contain any open interval. Therefore $F$ is not open in $\mathbb{R}$.
\end{proof}
\newpage

% chapter 2/section 1/exercise 6
\begin{exercise}
	If $F$ is a non-empty countable subset of $\mathbb{R}$, prove that $F$ is not an open set, but that $F$ may or may not be a closed set depending on the choice of $F$.
\end{exercise}

\begin{proof}
	If $F$ is a non-empty finite subset of $\mathbb{R}$, then $F$ is not open in $\mathbb{R}$, according to the previous exercise.

	If $F$ is a countably infinite subset of $\mathbb{R}$, then $F$ does not contain any open interval, because any open interval is uncountably infinite. Therefore $F$ is not open in $\mathbb{R}$.

	$F$ may be a closed set. For example: when $F$ is $\mathbb{Z}$.

	$F$ may not be a closed set. For example: when $F$ is $\mathbb{Q}$.
\end{proof}
\newpage

% chapter 2/section 1/exercise 7
\begin{exercise}
	\begin{enumerate}[label={(\roman*)}]
		\item Let $S = \{ 0, 1, 1/2, 1/3, 1/4, 1/5, \ldots, 1/n, \ldots \}$. Prove that the set $S$ is closed in the euclidean topology on $\mathbb{R}$.
		\item Is the set $T = \{ 1, 1/2, 1/3, 1/4, 1/5, \ldots, 1/n, \ldots \}$ closed in $\mathbb{R}$?
		\item Is the set $\{ \sqrt{2}, 2\sqrt{2}, 3\sqrt{2}, \ldots, n\sqrt{2} \}$ closed in $\mathbb{R}$?
	\end{enumerate}
\end{exercise}

\begin{proof}
	\begin{enumerate}[label={(\roman*)}]
		\item Let $x\in \mathbb{R}\setminus S$. $x\ne 0$, because $0\in S$. If $x < 0$, then $x\in (x-1, \frac{x}{2})$, where $(x-1, \frac{x}{2})\subseteq \mathbb{R}\setminus S$. If $x > 0$, then $\frac{1}{x}$ is not a positive integer, because $x\notin S$. Let $n$ be the integral part of $\frac{1}{x}$, then $n < \dfrac{1}{x} < n+1$.

		      If $n = 0$, then $x > 1$, $x\in \left(\frac{1+x}{2}, 1+x\right)\subseteq \mathbb{R}\setminus S$.

		      If $n > 0$, then $\frac{1}{n+1} < x < \frac{1}{n}$. Let $a = \frac{1}{2}\left(x + \frac{1}{n+1} \right)$ and $b = \frac{1}{2}\left(x + \frac{1}{n}\right)$, then $x\in (a, b)\subseteq \mathbb{R}\setminus S$.

		      Hence $\mathbb{R}\setminus S$ is open, which means $S$ is closed in $\mathbb{R}$.
		\item $0\in \mathbb{R}\setminus T$. Assume that there exists an open interval $(a, b)$ such that $0\in (a, b)\subseteq \mathbb{R}\setminus T$. Then $b > 0$. According to Archimedean property, there exists an integer $n$ such that $\frac{1}{b} < n$. Since $0 < \frac{1}{b} < n$, then $0 < \frac{1}{n} < b$. Therefore, $\frac{1}{n}\in (a, b)$. But this contradicts $\frac{1}{n}\in T$. So the assumption is false.

		      Hence $\mathbb{R}\setminus T$ is not open, and $T$ is not closed.
		\item Let $A = \{ \sqrt{2}, 2\sqrt{2}, 3\sqrt{2}, \ldots, n\sqrt{2}, \ldots \}$. Let $x\in \mathbb{R}\setminus A$, then $x\ne\sqrt{2}$. $\sqrt{2}$ is the least element of $A$.

		      If $x < \sqrt{2}$, then $x\in \left(x-1, \frac{1}{2}(x+\sqrt{2})\right)\subseteq \mathbb{R}\setminus A$.

		      If $x > \sqrt{2}$, then $\frac{x}{\sqrt{2}} > 1$. Since $x\notin A$, then $\frac{x}{\sqrt{2}}$ is not a positive integer. Let $n$ be the integral part of $\frac{x}{\sqrt{2}}$, then $1\leq n < \frac{x}{\sqrt{2}} < n+1$. So $n\sqrt{2} < x < (n+1)\sqrt{2}$. Let $a = \frac{1}{2}(x + n\sqrt{2})$ and $b = \frac{1}{2}(x + (n+1)\sqrt{2})$, then $n\sqrt{2} < a, b < (n+1)\sqrt{2}$. Therefore $x\in (a, b)\subseteq \mathbb{R}\setminus A$.

		      Hence $\mathbb{R}\setminus A$ is open, and $A$ is closed in $\mathbb{R}$.
	\end{enumerate}
\end{proof}
\newpage

% chapter 2/section 1/exercise 8
\begin{exercise}
	\begin{enumerate}[label={(\roman*)}]
		\item Let $(X, \tau)$ be a topological space. A subset $S$ of $X$ is said to be an {\color{red}$F_{\sigma}$-set} if it is the union of a countable number of closed sets. Prove that all open intervals $(a, b)$ and all closed intervals $[a, b]$, are $F_{\sigma}$-sets in $\mathbb{R}$.
		\item Let $(X, \tau)$ be a topological space. A subset $S$ of $X$ is said to be an {\color{red}$G_{\sigma}$-set} if it is the intersection of a countable number of open sets. Prove that all open intervals $(a, b)$ and all closed intervals $[a, b]$, are $G_{\sigma}$-sets in $\mathbb{R}$.
		\item Prove that the set $\mathbb{Q}$ of rationals is an $F_{\sigma}$-set in $\mathbb{R}$.
		\item Verify that the complement of an $F_{\sigma}$-set is a $G_{\sigma}$-set and the complement of a $G_{\sigma}$-set is an $F_{\sigma}$-set.
	\end{enumerate}
\end{exercise}

\begin{proof}
	\begin{enumerate}[label={(\roman*)}]
		\item Every closed interval $[a, b]$ is an $F_{\sigma}$-set.

		      Let's consider an open interval $(a, b)$. For every positive integer $n \geq 2$, we define $a_{n}$ and $b_{n}$ as follows
		      \[
			      a_{n} = a + \frac{b-a}{n}
			      \qquad
			      b_{n} = b - \frac{b-a}{n}
		      \]

		      Then $\bigcup_{n\in\mathbb{Z}_{\geq 2}} [a_{n}, b_{n}] \subseteq (a, b)$.

		      Let $x$ be an element of $(a, b)$, then $x = a + \frac{x - a}{b - a}(b - a) = b - \frac{b - x}{b - a}(b - a)$.

		      If $n\geq \max\left\{ \frac{b-a}{x-a}, \frac{b-a}{b-x} \right\}$, then $a_{n}\leq x\leq b_{n}$. So there exists a positive integer $n$ such that $x\in [a_{n}, b_{n}]$. Hence $(a, b)\subseteq \bigcup_{n\in\mathbb{Z}_{\geq 2}} [a_{n}, b_{n}]$.

		      Therefore
		      \[
			      (a, b) = \bigcup_{n\in\mathbb{Z}_{\geq 2}} [a_{n}, b_{n}]
		      \]

		      which means every open interval $(a, b)$ is an $F_{\sigma}$-set.
		\item Every open interval $(a, b)$ is a $G_{\sigma}$-set.

		      Let's consider a closed interval $[a, b]$. For every positive integer, we define $a_{n}$ and $b_{n}$ as follows
		      \[
			      a_{n} = a - \frac{1}{n}
			      \qquad
			      b_{n} = b + \frac{1}{n}
		      \]

		      Then $[a, b]\subseteq \bigcap_{n\in\mathbb{Z}_{>0}} (a_{n}, b_{n})$, and $a = \sup a_{n}$, $b = \inf b_{n}$.

		      Let $x$ be an element of $\bigcap_{n\in\mathbb{Z}_{> 0}} (a_{n}, b_{n})$, then $x$ is an upper bound of ${(a_{n})}_{n\in\mathbb{Z}_{>0}}$, $x$ is a lower bound of ${(b_{n})}_{n\in\mathbb{Z}_{> 0}}$. According to the definition of supremum and infimum, $a\leq x\leq b$. Therefore $\bigcap_{n\in\mathbb{Z}_{>0}} (a_{n}, b_{n})\subseteq [a, b]$.

		      So $\bigcap_{n\in\mathbb{Z}_{>0}} (a_{n}, b_{n}) = [a, b]$, which means every closed interval $[a, b]$ is a $G_{\sigma}$-set.
		\item Since $\mathbb{Q}$ is a countably infinite set, then $\mathbb{Q}$ is the union of its singleton subsets (which are closed). Therefore $\mathbb{Q}$ is an $F_{\sigma}$-set.
		\item If $X$ is an $F_{\sigma}$-set then according to the De Morgan's formula, the complement of $X$ is an intersection of a countable number of open sets, which means the complement of $X$ is a $G_{\sigma}$-set.

		      If $X$ is a $G_{\sigma}$-set then according to the De Morgan's formula, the complement of $X$ is a union of of countable number of closed sets, which means the complement of $X$ is an $F_{\sigma}$-set.
	\end{enumerate}
\end{proof}
\newpage

\section{Basis for a Topology}

% chapter 2/section 2/exercise 1
\begin{exercise}
	In this exercise you will prove that disc $\{ \anglebracket{x,y} \mid x^{2} + y^{2} < 1 \}$ is an open subset of $\mathbb{R}^{2}$, and then that every open disc in the plane is an open set.

	\begin{enumerate}[label={(\roman*)}]
		\item Let $\anglebracket{a, b}$ be any point in the disc $D = \{ \anglebracket{x, y} \mid x^{2} + y^{2} < 1 \}$. Put $r = \sqrt{a^{2} + b^{2}}$. Let $R_{\anglebracket{x, y}}$ be the open rectangle with vertices at the points $\anglebracket{a\pm\frac{1-r}{8}, b\pm\frac{1-r}{8}}$. Verify that $R_{\anglebracket{a, b}}\subset D$.
		\item Using (i) show that
		      \[
			      D = \bigcup_{\anglebracket{a, b}\in D} R_{\anglebracket{a, b}}.
		      \]
		\item Deduce from (ii) that $D$ is an open set in $\mathbb{R}^{2}$.
		\item Show that every disc $\{ \anglebracket{x, y} \mid {(x-a)}^{2} + {(y-b)}^{2} < c^{2}, a, b, c\in \mathbb{R} \}$ is open in $\mathbb{R}^{2}$.
	\end{enumerate}
\end{exercise}

\begin{proof}
	\begin{enumerate}[label={(\roman*)}]
		\item Let $\anglebracket{x, y}$ be a point in $R_{\anglebracket{a, b}}$. Then

		      Let $a_{1} = a - \frac{1-r}{8}$, $a_{2} = a + \frac{1-r}{8}$, $b_{1} = b - \frac{1-r}{8}$, $b_{2} = b + \frac{1-r}{8}$.
		      \begin{align*}
			      {\left(a \pm \frac{1-r}{8}\right)}^{2} + {\left(b \pm \frac{1-r}{8}\right)}^{2} & = a^{2} + b^{2} + \frac{{(1-r)}^{2}}{32} \pm \frac{1-r}{8}(a \pm b)                                                \\
			                                                                                      & = r^{2} + \frac{{(1-r)}^{2}}{32} \pm \frac{1-r}{8}(a + b)                                                          \\
			                                                                                      & \leq r^{2} + \frac{{(1-r)}^{2}}{32} + \frac{\sqrt{2}r(1-r)}{8}      & {(a\pm b)}^{2}\leq 2(a^{2} + b^{2}) = 2r^{2} \\
			                                                                                      & < r^{2} + {(1-r)}^{2} + 2r(1-r) = 1.
		      \end{align*}

		      So $\anglebracket{a_{1}, b_{1}}$, $\anglebracket{a_{1}, b_{2}}$, $\anglebracket{a_{2}, b_{1}}$, $\anglebracket{a_{2}, b_{2}}$ are in the open disc but not in the open rectangle $R_{\anglebracket{a, b}}$. Let $\anglebracket{x, y}$ be a point in the open rectangle $R_{\anglebracket{a, b}}$. Then
		      \[
			      a - \frac{1-r}{8}\leq x\leq a + \frac{1-r}{8},\qquad b - \frac{1-r}{8}\leq y\leq b + \frac{1-r}{8}.
		      \]

		      \begin{align*}
			      \sqrt{x^{2} + y^{2}} & \leq \sqrt{{(x-a)}^{2} + {(y-b)}^{2}} + \sqrt{a^{2} + b^{2}} \\
			                           & \leq \frac{\sqrt{2}(1-r)}{8} + r                             \\
			                           & < (1 - r) + r = 1.
		      \end{align*}

		      Hence $\anglebracket{x, y}\in D$ for every $x\in R_{\anglebracket{a, b}}$, which means $R_{\anglebracket{a, b}}\subseteq D$. Besides, the vertices of the open rectangle $R_{\anglebracket{a, b}}$ (they are not in the open rectangle) are in the open disc $D$. Therefore $R_{\anglebracket{a, b}}\subset D$.
		\item Due to (i)
		      \[
			      \bigcup_{\anglebracket{a, b}\in D} R_{\anglebracket{a, b}}\subseteq D.
		      \]

		      Let $\anglebracket{c, d}$ be a point in $D$, then $\anglebracket{c, d}\in R_{\anglebracket{c, d}}\subset D$. So $D\subseteq \bigcup_{\anglebracket{a, b}\in D} R_{\anglebracket{a, b}}$.

		      Thus $D = \bigcup_{\anglebracket{a, b}\in D} R_{\anglebracket{a, b}}$.
		\item Since $R_{\anglebracket{a, b}}$ is an open set in Euclidean topology for every point $\anglebracket{a, b}$. Therefore $D$ is an open set.
		\item The disc $\{ \anglebracket{x, y} \mid {(x-a)}^{2} + {(y - b)}^{2} < c^{2}, a, b, c\in\mathbb{R} \}$
		      \begin{itemize}
			      \item is open in $\mathbb{R}^{2}$ if $c = 0$, because when $c = 0$, the disc is the empty set.
			      \item is open in $\mathbb{R}^{2}$ if $c\ne 0$, because when $c\ne 0$, the disc is the image of $D$ under the composition $\tau\circ h$ of the homothety $h: \anglebracket{x, y}\mapsto \anglebracket{cx, cy}$ and the translation $\tau: \anglebracket{x, y} \mapsto \anglebracket{x+a, y+b}$.
		      \end{itemize}
	\end{enumerate}
\end{proof}
\newpage

% chapter 2/section 2/exercise 2
\begin{exercise}
	In this exercise you will show that the collection of all open discs in $\mathbb{R}^{2}$ is a basis for a topology on $\mathbb{R}^{2}$. [Later we shall see that this is the euclidean topology.]
	\begin{enumerate}[label={(\roman*)}]
		\item Let $D_{1}$ and $D_{2}$ be any open discs in $\mathbb{R}^{2}$ with $D_{1}\cap D_{2}\ne \varnothing$. If $\anglebracket{a, b}$ is any point in $D_{1}\cap D_{2}$, show that there exists an open disc $D_{\anglebracket{a,b}}$ with center $\anglebracket{a, b}$ such that $D_{\anglebracket{a, b}}\subset D_{1}\cap D_{2}$.
		\item Show that
		      \[
			      D_{1}\cap D_{2} = \bigcup_{\anglebracket{a, b}\in D_{1}\cap D_{2}} D_{\anglebracket{a, b}}.
		      \]
		\item Using (ii) and Proposition 2.2.8, prove that the collection of all open discs in $\mathbb{R}^{2}$ is a basis for a topology on $\mathbb{R}^{2}$.
	\end{enumerate}
\end{exercise}

\begin{proof}
	\begin{enumerate}[label={(\roman*)}]
		\item Let $\anglebracket{a_{1}, b_{1}}$, $\anglebracket{a_{2}, b_{2}}$ be the centers of $D_{1}$, $D_{2}$, respectively. Let $r_{1}$, $r_{2}$ be the radii of $D_{1}$, $D_{2}$, respectively.
		      \[
			      \begin{split}
				      D_{1} = \{ \anglebracket{x, y}\mid {(x - a_{1})}^{2} + {(y - b_{1})}^{2} < {r_{1}}^{2} \}, \\
				      D_{2} = \{ \anglebracket{x, y}\mid {(x - a_{2})}^{2} + {(y - b_{2})}^{2} < {r_{2}}^{2} \}.
			      \end{split}
		      \]

		      Let $r$ be a positive real number which is strictly less than
		      \[
			      \min\left\{ r_{1} - \sqrt{{(a-a_{1})}^{2} + {(b-b_{1})}^{2}}, r_{2} - \sqrt{{(a-a_{2})}^{2} + {(b-b_{2})}^{2}} \right\}.
		      \]

		      Let $\anglebracket{x, y}$ be a point in the open disc of center $\anglebracket{a, b}$ and radius $r$.
		      \begin{align*}
			      \sqrt{{(x-a_{1})}^{2} + {(y-b_{1})}^{2}} & \leq \sqrt{{(x-a)}^{2} + {(y-b)}^{2}} + \sqrt{{(a-a_{1})}^{2} + {(b-b_{1})}^{2}} < r_{1} \\
			      \sqrt{{(x-a_{2})}^{2} + {(y-b_{2})}^{2}} & \leq \sqrt{{(x-a)}^{2} + {(y-b)}^{2}} + \sqrt{{(a-a_{2})}^{2} + {(b-b_{2})}^{2}} < r_{2} \\
		      \end{align*}

		      So $\anglebracket{x, y}$ is an element of $D_{1}$ and $D_{2}$. Therefore the open disc whose center is $\anglebracket{a, b}$ and radius is $r$ is a subset of $D_{1}\cap D_{2}$.
		\item According to (i), $\bigcup_{\anglebracket{a, b}\in D_{1}\cap D_{2}} D_{\anglebracket{a, b}}\subseteq D_{1}\cap D_{2}$. On the other hand, for every $\anglebracket{c, d}\in D_{1}\cap D_{2}$, there exists an open disc whose center is $\anglebracket{c, d}$ and the open disc is contained in $D_{1}\cap D_{2}$. So $D_{1}\cap D_{2}\subseteq \bigcup_{\anglebracket{a, b}\in D_{1}\cap D_{2}} D_{\anglebracket{a, b}}$. Hence
		      \[
			      D_{1}\cap D_{2} = \bigcup_{\anglebracket{a, b}\in D_{1}\cap D_{2}} D_{\anglebracket{a, b}}.
		      \]
		\item The union of all open discs is the entire $\mathbb{R}^{2}$, since every point $\anglebracket{x, y}$ is contained in an open disc, for example, $\{ \anglebracket{u, v} \mid u^{2} + v^{2} < x^{2} + y^{2} + 1 \}$.

		      According to (ii), the intersection of any two open discs, if not empty, is an union of open discs.

		      Due to Proposition 2.2.8, the collection of all open discs in $\mathbb{R}^{2}$ is a basis for a topology on $\mathbb{R}^{2}$.
	\end{enumerate}
\end{proof}
\newpage

% chapter 2/section 2/exercise 3
\begin{exercise}
	Let $\mathcal{B}$ be the collection of all open intervals $(a, b)$ in $\mathbb{R}$ with $a < b$ and $a$ and $b$ rational numbers. Prove that $\mathcal{B}$ is a basis for the euclidean topology on $\mathbb{R}$. [Compare this with Proposition 2.2.1 and Example 2.2.3 where $a$ and $b$ were not necessarily rational.]
\end{exercise}

\begin{proof}
	Let $(x, y)$ be an open interval in $\mathbb{R}$, and $t$ be an element of $(x, y)$.

	Every member $(a, b)$ of $\mathcal{B}$ such that $x\leq a < b \leq y$ is a subset of $(x, y)$. Therefore
	\[
		\bigcup_{\substack{(a, b)\in\mathcal{B} \\ x\leq a < b\leq y }} (a, b) \subseteq (x, y).
	\]

	Due to the density of $\mathbb{Q}$ in $\mathbb{R}$, there exist rational numbers $a$, $b$ such that $x < a < t$ and $t < b < y$. So $\mathcal{B}\ni (a, b)\subseteq (x, y)$. Therefore
	\[
		(x, y)\subseteq \bigcup_{\substack{(a, b)\in\mathcal{B} \\ x\leq a < b\leq y }} (a, b).
	\]

	So
	\[
		(x, y) = \bigcup_{\substack{(a, b)\in\mathcal{B} \\ x\leq a < b\leq y }} (a, b).
	\]

	On the other hand, the collection of all open intervals in $\mathbb{R}$ is a basis for the Euclidean topology on $\mathbb{R}$. So $\mathcal{B}$ is a basis for the Euclidean topology on $\mathbb{R}$.
\end{proof}
\newpage

% chapter 2/section 2/exercise 4
\begin{exercise}
	A topological space $(X, \tau)$ is said to satisfy the {\color{red}second axiom of countability} or to be {\color{red}second countable} if there exists a basis $\mathcal{B}$ for $\tau$, where $\mathcal{B}$ consists of only a countable number of sets.
	\begin{enumerate}[label={(\roman*)}]
		\item Using Exercise 3 above show that $\mathbb{R}$ satisfies the second axiom of countability.
		\item Prove that the discrete topology on an uncountable set does not satisfy the second axiom of countability.
		\item Prove that $\mathbb{R}^{n}$ satisfies the second axiom of countability, for each positive integer $n$.
		\item Let $(X, \tau)$ be the set of all integers with finite-closed topology. Does the space $(X, \tau)$ satisfy the second axiom of countability?
	\end{enumerate}
\end{exercise}

\begin{proof}
	\begin{enumerate}[label={(\roman*)}]
		\item Let $\mathcal{B}$ be the collection of all open intervals $(a, b)$ in $\mathbb{R}$ where $a < b$, $a$ and $b$ are rational numbers. According to Exercise 3, $\mathcal{B}$ is a basis for the Euclidean topology on $\mathbb{R}$. Therefore $\mathbb{R}$ with the Euclidean topology is second countable.
		\item Let $Y$ be a basis for the discrete topology on $X$. Let $x$ be an element of $X$. Since $Y$ is a basis for the discrete topology on $X$, the singleton subset $\{ x \}$ of $X$ is a union of members of $Y$. Therefore, $\{ x \}$ is a member of $X$ for every element $x$ of $X$. Since $X$ is uncountable, then $Y$ is also uncountable (it contains every singleton subset of $X$). Hence the discrete topology on an uncountable set does not satisfy the second axiom of countability.
		\item The collection $\mathcal{B}$ of all open intervals $\{ \anglebracket{x_{1}, \ldots, x_{n}} \mid a_{1} < x_{1} < b_{1}, \ldots, a_{n} < x_{n} < b_{n} \}$ is a basis for the Euclidean topology on $\mathbb{R}^{n}$.

		      Let $\mathcal{B'}$ be the set of open boxes
		      \[
			      \{ \anglebracket{x_{1},\ldots, x_{n}} \mid c_{1} < x_{1} < d_{1}, \ldots, c_{n} < x_{n} < d_{n} \}
		      \]

		      where $c_{i}$, $d_{i}$ are rational numbers for every $i\in \{ 1, \ldots, n \}$.
		      \[
			      \{ \anglebracket{x_{1},\ldots, x_{n}} \mid a_{i} < x_{i} < b_{i} \} = \bigcup_{C\in\mathcal{B'}} C
		      \]

		      where $C\subseteq \{ \anglebracket{x_{1},\ldots, x_{n}} \mid a_{i} < x_{i} < b_{i} \} $. So $\mathcal{B'}$ is a basis for the Euclidean topology on $\mathbb{R}^{n}$. On the other hand, $\mathcal{B'}$ is countably infinite, since we can establish a bijection from $\mathcal{B'}$ to ${(\mathbb{Q}\times\mathbb{Q})}^{n}$. Hence the Euclidean topology on $\mathbb{R}^{n}$ satified the second axiom of countability, for each positive integer $n$.
		\item Yes.

		      Let $A_{n}$ be the set of subsets of $n$ elements of $\mathbb{Z}$.

		      If $n = 0$, $A_{0}$ has $1$ element. If $n$ is a positive integer, $A_{n}$ is a countably infinite set. Since $A_{n}$ is countably infinite for every positive integer $n$, then we can label every element of $A_{n}$ (establish a bijection from $A_{n}$ to $\mathbb{N}\cup\{0\}$) as follows
		      \begin{align*}
			      A_{0}  & = \{ a_{0,0} = \varnothing \},                     \\
			      A_{1}  & = \{ a_{1,0}, a_{1,1}, a_{1,2}, a_{1,3},\ldots \}, \\
			      A_{2}  & = \{ a_{2,0}, a_{2,1}, a_{2,2}, a_{2,3},\ldots \}, \\
			      A_{3}  & = \{ a_{3,0}, a_{3,1}, a_{3,2}, a_{3,3},\ldots \}, \\
			      \ldots &
		      \end{align*}

		      The union $\bigcup^{\infty}_{n=0} A_{n}$ is countably infinite, since we can list all the elements as follows
		      \begin{align*}
			       & a_{0,0}                             \\
			       & a_{1,0}                             \\
			       & a_{1,1}, a_{2,0},                   \\
			       & a_{1,2}, a_{2,1}, a_{3,0},          \\
			       & a_{1,3}, a_{2,2}, a_{3,1}, a_{4,0}, \\
			       & \ldots
		      \end{align*}

		      So the set of all finite subsets of $\mathbb{Z}$ is countably infinite. Therefore the set of all subsets of $\mathbb{Z}$ of which complement is finite is also countably infinite --- Equivalently, the set of all open sets in the finite-closed topology on $\mathbb{Z}$ is countably infinite. So the space $(X, \tau)$ satisfies the second axiom of countability.
	\end{enumerate}
\end{proof}
\newpage

% chapter 2/section 2/exercise 5
\begin{exercise}
	Prove the following statements.
	\begin{enumerate}[label={(\roman*)}]
		\item Let $m$ and $c$ be real numbers. Then the line $L = \{ \anglebracket{x, y} \mid y = mx + c \}$ is a closed subset of $\mathbb{R}^{2}$.
		\item Let $\mathbb{S}^{1}$ be the unit circle given by $\mathbb{S}^{1} = \{ \anglebracket{x, y}\in \mathbb{R}^{2} \mid x^{2} + y^{2} = 1 \}$. Then $\mathbb{S}^{1}$ is a closed subset of $\mathbb{R}^{2}$.
		\item Let $\mathbb{S}^{n}$ be the unit $n$-sphere given by
		      \[
			      \mathbb{S}^{n} = \{ \anglebracket{x_{1}, x_{2}, \ldots, x_{n}, x_{n+1}}\in \mathbb{R}^{n+1} \mid {x_{1}}^{2} + {x_{2}}^{2} + \cdots + {x_{n+1}}^{2} = 1 \}.
		      \]

		      Then $\mathbb{S}^{n}$ is a closed subset of $\mathbb{R}^{n+1}$.
		\item Let $B^{n}$ be the closed unit $n$-ball given by
		      \[
			      B^{n} = \{ \anglebracket{x_{1}, x_{2}, \ldots, x_{n}}\in \mathbb{R}^{n} \mid {x_{1}}^{2} + {x_{2}}^{2} + \cdots + {x_{n}}^{2} \leq 1 \}.
		      \]

		      Then $B^{n}$ is a closed subset of $\mathbb{R}^{n}$.
		\item The curve $C = \{ \anglebracket{x, y}\in \mathbb{R}^{2} \mid xy = 1 \}$ is a closed subset of $\mathbb{R}^{2}$.
	\end{enumerate}
\end{exercise}

\begin{proof}
	\begin{enumerate}[label={(\roman*)}]
		\item The complement of $L$ in $\mathbb{R}^{2}$ is the following set
		      \[
			      \{ \anglebracket{x, y}\in\mathbb{R}^{2} \mid y > mx + c \} \cup \{ \anglebracket{x, y}\in\mathbb{R}^{2} \mid y < mx + c \}.
		      \]

		      If a point $\anglebracket{x_{0}, y_{0}}$ satisfies $y_{0} \ne mx_{0} + c$. The orthogonal projection of $\anglebracket{x_{0}, y_{0}}$ onto $L$ is
		      \[
			      \anglebracket{x_{0} - \frac{m(mx_{0} - y_{0} + c)}{m^{2} + 1}, y_{0} + \frac{mx_{0} - y_{0} + c}{m^{2} + 1}}
		      \]

		      The distance between $\anglebracket{x_{0}, y_{0}}$ and $L$ is
		      \[
			      r = \frac{\abs{mx_{0} - y_{0} + c}}{\sqrt{m^{2} + 1}}
		      \]

		      So the open disc of which center is $\anglebracket{x_{0}, y_{0}}$ and radius $r' < r$ is contained in $\mathbb{R}^{2}\setminus L$. Hence $\mathbb{R}^{2}\setminus L$ is open, and $L$ is closed.
		\item The complement of $\mathbb{S}^{1}$ in $\mathbb{R}^{2}$ is the following set
		      \[
			      \{ \anglebracket{x, y}\in\mathbb{R}^{2} \mid x^{2} + y^{2} < 1 \} \cup \{ \anglebracket{x, y}\in\mathbb{R}^{2} \mid x^{2} + y^{2} > 1 \}
		      \]

		      $\{ \anglebracket{x, y}\in\mathbb{R}^{2} \mid x^{2} + y^{2} < 1 \}$ is an open disc, and according to Exercise 1, it is an open subset of $\mathbb{R}^{2}$.

		      Let $\anglebracket{x_{0}, y_{0}}$ be a point such that ${x_{0}}^{2} + {y_{0}}^{2} > 1$. Let $t$ be a real number such that $0 < t < 1$.
		      \[
			      {\left(\frac{x_{0}}{\sqrt{{x_{0}}^{2} + {y_{0}}^{2}}}\right)}^{2} + {\left(\frac{y_{0}}{\sqrt{{x_{0}}^{2} + {y_{0}}^{2}}}\right)}^{2} = 1.
		      \]
		      \[
			      \anglebracket{x_{1}, y_{1}} = \anglebracket{(1-t)x_{0} + \frac{tx_{0}}{\sqrt{{x_{0}}^{2} + {y_{0}}^{2}}}, (1-t)y_{0} + \frac{ty_{0}}{\sqrt{{x_{0}}^{2} + {y_{0}}^{2}}}}.
		      \]

		      The distance between $\anglebracket{x_{0}, y_{0}}$ and $\anglebracket{x_{1}, y_{1}}$ is
		      \[
			      r = \sqrt{{\left(-tx_{0} + \frac{tx_{0}}{\sqrt{{x_{0}}^{2} + {y_{0}}^{2}}}\right)}^{2} + {\left(-ty_{0} + \frac{ty_{0}}{\sqrt{{x_{0}}^{2} + {y_{0}}^{2}}}\right)}^{2}}.
		      \]

		      The open disc of which center is $\anglebracket{x_{0}, y_{0}}$ and radius $r$ is contained in $\{ \anglebracket{x, y}\in\mathbb{R}^{2} \mid x^{2} + y^{2} > 1 \}$. So $\{ \anglebracket{x, y}\in\mathbb{R}^{2} \mid x^{2} + y^{2} > 1 \}$ is an open set of $\mathbb{R}^{2}$.

		      Hence $\mathbb{R}^{2}\setminus \mathbb{S}^{1}$ is an open set of $\mathbb{R}^{2}$.
		\item The complement of $\mathbb{S}^{n}$ in $\mathbb{R}^{n+1}$ is the following set
		      \begin{multline*}
			      \{ \anglebracket{x_{1}, x_{2}, \ldots, x_{n}, x_{n+1}} \in \mathbb{R}^{n+1} \mid {x_{1}}^{2} + {x_{2}}^{2} + \cdots + {x_{n+1}}^{2} < 1 \} \\
			      \cup \\
			      \{ \anglebracket{x_{1}, x_{2}, \ldots, x_{n}, x_{n+1}} \in \mathbb{R}^{n+1} \mid {x_{1}}^{2} + {x_{2}}^{2} + \cdots + {x_{n+1}}^{2} > 1 \}.
		      \end{multline*}

		      Let $\anglebracket{c_{1}, c_{2}, \ldots, c_{n}, c_{n+1}}$ be a point in $\{ \anglebracket{x_{1}, x_{2}, \ldots, x_{n}, x_{n+1}} \in \mathbb{R}^{n+1} \mid {x_{1}}^{2} + {x_{2}}^{2} + \cdots + {x_{n+1}}^{2} < 1 \}$. We define the box $B_{\anglebracket{c_{1}, c_{2}, \ldots, c_{n}, c_{n+1}}}$ as follows
		      \[
			      B_{\anglebracket{c_{1}, c_{2}, \ldots, c_{n}, c_{n+1}}} = \left\{ \anglebracket{x_{1}, x_{2}, \ldots, x_{n}, x_{n+1}} \mid c_{i} - \frac{1-r}{\sqrt{n+1}} < x_{i} < c_{i} + \frac{1-r}{\sqrt{n+1}} \right\}
		      \]

		      where $r = \sqrt{{c_{1}}^{2} + \cdots + {c_{n+1}}^{2}}$. Let $\anglebracket{z_{1}, z_{2}, \ldots, z_{n}, z_{n+1}}$ be a point in $B_{\anglebracket{c_{1}, c_{2}, \ldots, c_{n}, c_{n+1}}}$.
		      \begin{align*}
			      \sqrt{{z_{1}}^{2} + \cdots+ {z_{n+1}}^{2}} & \leq \sqrt{{(z_{1} - c_{1})}^{2} + \cdots + {(z_{n+1} - c_{n+1})}^{2}} + \sqrt{{c_{1}}^{2} + \cdots + {c_{n+1}}^{2}} \\
			                                                 & < \sqrt{(n+1){\left(\frac{1-r}{\sqrt{n+1}}\right)}^{2}} + r                                                          \\
			                                                 & = (1 - r) + r < 1.
		      \end{align*}

		      So $B_{\anglebracket{c_{1}, c_{2}, \ldots, c_{n}, c_{n+1}}}$ is contained in $\{ \anglebracket{x_{1}, x_{2}, \ldots, x_{n}, x_{n+1}} \in \mathbb{R}^{n+1} \mid {x_{1}}^{2} + {x_{2}}^{2} + \cdots + {x_{n+1}}^{2} < 1 \}$ for every point $\anglebracket{c_{1}, c_{2}, \ldots, c_{n}, c_{n+1}}$ within $\{ \anglebracket{x_{1}, x_{2}, \ldots, x_{n}, x_{n+1}} \in \mathbb{R}^{n+1} \mid {x_{1}}^{2} + {x_{2}}^{2} + \cdots + {x_{n+1}}^{2} < 1 \}$. So $\{ \anglebracket{x_{1}, x_{2}, \ldots, x_{n}, x_{n+1}} \in \mathbb{R}^{n+1} \mid {x_{1}}^{2} + {x_{2}}^{2} + \cdots + {x_{n+1}}^{2} < 1 \}$ is an open set of $\mathbb{R}^{n+1}$.

		      Let $\anglebracket{a_{1}, \ldots, a_{n}, a_{n+1}}$ be a point in $ \{ \anglebracket{x_{1}, x_{2}, \ldots, x_{n}, x_{n+1}} \in \mathbb{R}^{n+1} \mid {x_{1}}^{2} + {x_{2}}^{2} + \cdots + {x_{n+1}}^{2} > 1 \}$. The following point
		      \[
			      \anglebracket{\frac{a_{1}}{\sqrt{{a_{1}}^{2} + \cdots + {a_{n}}^{2} + {a_{n+1}}^{2}}}, \ldots, \frac{a_{n+1}}{\sqrt{{a_{1}}^{2} + \cdots + {a_{n}}^{2} + {a_{n+1}}^{2}}}}
		      \]

		      is in the sphere $\mathbb{S}^{n}$. Let $t$ be a real number such that $0 < t < 1$.
		      \[
			      \anglebracket{b_{1}, \ldots, b_{n+1}} = \anglebracket{(1-t)a_{1} + \frac{ta_{1}}{\sqrt{{a_{1}}^{2} + \cdots + {a_{n}}^{2} + {a_{n+1}}^{2}}}, \ldots, (1-t)a_{n+1} + \frac{ta_{n+1}}{\sqrt{{a_{1}}^{2} + \cdots + {a_{n}}^{2} + {a_{n+1}}^{2}}}}.
		      \]

		      The distance between $\anglebracket{a_{1}, \ldots, a_{n+1}}$ and $\anglebracket{b_{1}, \ldots, b_{n+1}}$ is
		      \[
			      r = \sqrt{{\left(-ta_{1} + \frac{ta_{1}}{\sqrt{{a_{1}}^{2} + \cdots + {a_{n}}^{2} + {a_{n+1}}^{2}}}\right)}^{2} + \cdots + {\left(-ta_{n+1} + \frac{ta_{n+1}}{\sqrt{{a_{1}}^{2} + \cdots + {a_{n}}^{2} + {a_{n+1}}^{2}}}\right)}^{2}}
		      \]

		      Then the following set
		      \[
			      \left\{ \anglebracket{x_{1}, \ldots, x_{n+1}} \mid \sqrt{{(x_{1} - a_{1})}^{2} + \cdots + {(x_{n+1} - a_{n+1})}^{2}} < r \right\}
		      \]

		      is an open set and is contained in $\{ \anglebracket{x_{1}, x_{2}, \ldots, x_{n}, x_{n+1}} \in \mathbb{R}^{n+1} \mid {x_{1}}^{2} + {x_{2}}^{2} + \cdots + {x_{n+1}}^{2} > 1 \}$. Hence $\{ \anglebracket{x_{1}, x_{2}, \ldots, x_{n}, x_{n+1}} \in \mathbb{R}^{n+1} \mid {x_{1}}^{2} + {x_{2}}^{2} + \cdots + {x_{n+1}}^{2} > 1 \}$ is an open set in $\mathbb{R}^{n+1}$.

		      Thus $\mathbb{R}^{n+1}\setminus\mathbb{S}^{n}$ is an open set in $\mathbb{R}^{n+1}$, and $\mathbb{S}^{n}$ is a closed set in $\mathbb{R}^{n+1}$.
		\item The complement of $B^{n}$ in $\mathbb{R}^{n}$ is the following set
		      \[
			      \{ \anglebracket{x{1}, x_{2}, \ldots, x_{n}}\in \mathbb{R}^{n} \mid {x_{1}}^{2} + {x_{2}}^{2} + \cdots + {x_{n}}^{2} > 1 \}.
		      \]

		      According to (iii), $\mathbb{R}^{n}\setminus B^{n}$ is an open set in $\mathbb{R}^{n}$. Therefore $B^{n}$ is a closed set in $\mathbb{R}^{n}$.
		\item The complement of $C$ in $\mathbb{R}^{2}$ is the following set
		      \[
			      \{ \anglebracket{x, y}\in \mathbb{R}^{2} \mid xy > 1 \} \cup \{ \anglebracket{x, y}\in \mathbb{R}^{2} \mid xy < 1 \}
		      \]

		      \begin{enumerate}[label={\textbf{Case \arabic*.}},itemindent=1cm]
			      \item $\anglebracket{x_{0}, y_{0}} \in \{ \anglebracket{x, y}\in \mathbb{R}^{2} \mid xy > 1 \}$ and $x_{0} > 0$, $y_{0} > 0$

			            then every point in the following open rectangle is in $\{ \anglebracket{x, y}\in \mathbb{R}^{2} \mid xy > 1 \}$
			            \[
				            \left\{ \anglebracket{x, y}\in \mathbb{R}^{2} \mid \frac{1}{2}\left(x_{0} + \frac{1}{y_{0}}\right) < x < \frac{1}{2}\left(3x_{0} - \frac{1}{y_{0}}\right) \wedge \frac{1}{2}\left(\frac{1}{x_{0}} + y_{0}\right) < y < \frac{1}{2}\left(3y_{0} - \frac{1}{x_{0}}\right)\right\}.
			            \]

			      \item $\anglebracket{x_{0}, y_{0}} \in \{ \anglebracket{x, y}\in \mathbb{R}^{2} \mid xy > 1 \}$ and $x_{0} < 0$, $y_{0} < 0$

			            then every point in the following open rectangle is in $\{ \anglebracket{x, y}\in \mathbb{R}^{2} \mid xy > 1 \}$
			            \[
				            \left\{ \anglebracket{x, y}\in \mathbb{R}^{2} \mid \frac{1}{2}\left(3x_{0} - \frac{1}{y_{0}}\right) < x < \frac{1}{2}\left(x_{0} + \frac{1}{y_{0}}\right) \wedge \frac{1}{2}\left(3y_{0} - \frac{1}{x_{0}}\right) < y < \frac{1}{2}\left(\frac{1}{x_{0}} + y_{0}\right)\right\}.
			            \]
		      \end{enumerate}

		      So $\{ \anglebracket{x, y}\in \mathbb{R}^{2} \mid xy > 1 \}$ is an open set in $\mathbb{R}^{2}$.

		      \begin{enumerate}[label={\textbf{Case \arabic*.}},itemindent=1cm]
			      \item $\anglebracket{x_{0}, y_{0}} \in \{ \anglebracket{x, y}\in \mathbb{R}^{2} \mid 0 < xy < 1 \}$ and $x_{0} > 0$, $y_{0} > 0$

			            then every point in the following open rectangle is in $\{ \anglebracket{x, y}\in \mathbb{R}^{2} \mid 0 < xy < 1 \}$
			            \[
				            \left\{ \anglebracket{x, y}\in \mathbb{R}^{2} \mid \frac{x_{0}}{2} < x < \frac{x_{0}}{2} + \frac{x_{0}}{\sqrt{x_{0}y_{0}}} \wedge \frac{y_{0}}{2} < y < \frac{y_{0}}{2} + \frac{y_{0}}{\sqrt{x_{0}y_{0}}} \right\}.
			            \]

			      \item $\anglebracket{x_{0}, y_{0}} \in \{ \anglebracket{x, y}\in \mathbb{R}^{2} \mid xy < 1 \}$ and $x_{0} < 0$, $y_{0} < 0$

			            then every point in the following open rectangle is in $\{ \anglebracket{x, y}\in \mathbb{R}^{2} \mid 0 < xy < 1 \}$
			            \[
				            \left\{ \anglebracket{x, y}\in \mathbb{R}^{2} \mid \frac{x_{0}}{2} + \frac{x_{0}}{\sqrt{x_{0}y_{0}}} < x < \frac{x_{0}}{2} \wedge \frac{y_{0}}{2} + \frac{y_{0}}{\sqrt{x_{0}y_{0}}} < y < \frac{y_{0}}{2} \right\}.
			            \]

			      \item $\anglebracket{x_{0}, y_{0}}\in \{ \anglebracket{x, y}\in \mathbb{R}^{2} \mid xy < 1 \}$ and $x_{0} < 0$, $y_{0} > 0$

			            then every point in the folowing open rectangle is in $\{ \anglebracket{x, y}\in \mathbb{R}^{2} \mid xy < 1 \}$
			            \[
				            \left\{ \anglebracket{x, y}\in\mathbb{R}^{2} \mid 2x_{0} < x < 0 \wedge 0 < y < 2y_{0} \right\}.
			            \]

			      \item $\anglebracket{x_{0}, y_{0}}\in \{ \anglebracket{x, y}\in \mathbb{R}^{2} \mid xy < 1 \}$ and $x_{0} > 0$, $y_{0} < 0$

			            then every point in the folowing open rectangle is in $\{ \anglebracket{x, y}\in \mathbb{R}^{2} \mid xy < 1 \}$
			            \[
				            \left\{ \anglebracket{x, y}\in\mathbb{R}^{2} \mid 0 < x < 2x_{0} \wedge 2y_{0} < y < 0 \right\}.
			            \]

			      \item $\anglebracket{x_{0}, y_{0}}\in \{ \anglebracket{x, y}\in \mathbb{R}^{2} \mid xy < 1 \}$ and $x_{0} = 0$, $y_{0} > 0$

			            then every point in the folowing open rectangle is in $\{ \anglebracket{x, y}\in \mathbb{R}^{2} \mid xy < 1 \}$
			            \[
				            \left\{ \anglebracket{x, y}\in \mathbb{R}^{2} \mid \frac{-1}{2y_{0}} < x < \frac{1}{2y_{0}} \wedge \frac{y_{0}}{2} < y < \frac{3y_{0}}{2} \right\}.
			            \]
			      \item $\anglebracket{x_{0}, y_{0}}\in \{ \anglebracket{x, y}\in \mathbb{R}^{2} \mid xy < 1 \}$ and $x_{0} = 0$, $y_{0} < 0$

			            then every point in the folowing open rectangle is in $\{ \anglebracket{x, y}\in \mathbb{R}^{2} \mid xy < 1 \}$
			            \[
				            \left\{ \anglebracket{x, y}\in \mathbb{R}^{2} \mid \frac{1}{2y_{0}} < x < \frac{-1}{2y_{0}} \wedge \frac{3y_{0}}{2} < y < \frac{y_{0}}{2} \right\}.
			            \]
			      \item $\anglebracket{x_{0}, y_{0}}\in \{ \anglebracket{x, y}\in \mathbb{R}^{2} \mid xy < 1 \}$ and $x_{0} > 0$, $y_{0} = 0$

			            then every point in the folowing open rectangle is in $\{ \anglebracket{x, y}\in \mathbb{R}^{2} \mid xy < 1 \}$
			            \[
				            \left\{ \anglebracket{x, y}\in \mathbb{R}^{2} \mid \frac{x_{0}}{2} < x < \frac{3x_{0}}{2} \wedge \frac{-1}{2x_{0}} < y < \frac{1}{2x_{0}} \right\}.
			            \]
			      \item $\anglebracket{x_{0}, y_{0}}\in \{ \anglebracket{x, y}\in \mathbb{R}^{2} \mid xy < 1 \}$ and $x_{0} < 0$, $y_{0} = 0$

			            then every point in the folowing open rectangle is in $\{ \anglebracket{x, y}\in \mathbb{R}^{2} \mid xy < 1 \}$
			            \[
				            \left\{ \anglebracket{x, y}\in \mathbb{R}^{2} \mid \frac{3x_{0}}{2} < x < \frac{x_{0}}{2} \wedge \frac{1}{2x_{0}} < y < \frac{-1}{2x_{0}}\right\}.
			            \]
			      \item $\anglebracket{x_{0}, y_{0}} = \anglebracket{0, 0}$

			            then every point in the folowing open rectangle is in $\{ \anglebracket{x, y}\in \mathbb{R}^{2} \mid xy < 1 \}$
			            \[
				            \left\{ \anglebracket{x, y}\in \mathbb{R}^{2} \mid -1 < x < 1 \wedge -1 < y < 1 \right\}.
			            \]
		      \end{enumerate}

		      So $\{ \anglebracket{x, y}\in \mathbb{R}^{2} \mid xy < 1 \}$ is an open set in $\mathbb{R}^{2}$.

		      Hence the complement of $C$ in $\mathbb{R}^{2}$ is an open set in $\mathbb{R}^{2}$, therefore $C$ is a closed set in $\mathbb{R}^{2}$.
	\end{enumerate}
\end{proof}
\newpage

% chapter 2/section 2/exercise 6
\begin{exercise}
	Let $\mathcal{B}_{1}$ be a basis for a topology $\tau_{1}$ on a set $X$ and $\mathcal{B}_{2}$  a basis for a topology on a set $Y$. The set $X\times Y$ consists of all ordered pairs $\anglebracket{x, y}$, $x\in X$ and $y\in Y$. Let $\mathcal{B}$ be the collection of subsets of $X\times Y$ consisting of all the sets $B_{1}\times B_{2}$ where $B_{1}\in \mathcal{B}_{1}$ and $B_{2}\in \mathcal{B}_{2}$. Prove that $\mathcal{B}$ is a basis for a topology on $X\times Y$. The topology so defined is called the {\color{red}product topology} on $X\times Y$.
\end{exercise}

\begin{proof}
	Since $\mathcal{B}_{1}$ is a basis for a topology $\tau_{1}$ on a set $X$ and $\mathcal{B}_{2}$ is a basis for a topology on a set $Y$, then
	\begin{align*}
		X\times Y & = \left(\bigcup_{B_{1}\in\mathcal{B}_{1}} B_{1}\right) \times \left(\bigcup_{B_{2}\in \mathcal{B}_{2}} B_{2}\right).
	\end{align*}

	If $\anglebracket{x_{1}, x_{2}}$ is an element of $\bigcup_{\substack{B_{1}\in \mathcal{B}_{1} \\ B_{2}\in \mathcal{B}_{2}}} (B_{1}\times B_{2})$ then there exist $B_{1}\times B_{2} \in \mathcal{B}$ such that $x_{1}\in B_{1}$ and $x_{2}\in B_{2}$. On the other hand, $\anglebracket{x_{1}, x_{2}}$ is also an element of $X\times Y$. Therefore
	\[
		\bigcup_{\substack{B_{1}\in \mathcal{B}_{1} \\ B_{2}\in \mathcal{B}_{2}}} (B_{1}\times B_{2}) \subseteq \left(\bigcup_{B_{1}\in\mathcal{B}_{1}} B_{1}\right) \times \left(\bigcup_{B_{2}\in \mathcal{B}_{2}} B_{2}\right).
	\]

	If $\anglebracket{x_{1}, x_{2}}$ is an element of $\left(\bigcup_{B_{1}\in\mathcal{B}_{1}} B_{1}\right) \times \left(\bigcup_{B_{2}\in \mathcal{B}_{2}} B_{2}\right)$, then there exist $B_{1}\in \mathcal{B}_{1}$ and $B_{2}\in \mathcal{B}_{2}$ such that $x_{1}\in B_{1}$ and $x_{2}\in B_{2}$. Therefore $\anglebracket{x_{1}, x_{2}}\in B_{1}\times B_{2}$, which means $\anglebracket{x_{1}, x_{2}} \in \bigcup_{\substack{B_{1}\in \mathcal{B}_{1} \\ B_{2}\in \mathcal{B}_{2}}} (B_{1}\times B_{2})$. Therefore
	\[
		\left(\bigcup_{B_{1}\in\mathcal{B}_{1}} B_{1}\right) \times \left(\bigcup_{B_{2}\in \mathcal{B}_{2}} B_{2}\right) \subseteq \bigcup_{\substack{B_{1}\in \mathcal{B}_{1} \\ B_{2}\in \mathcal{B}_{2}}} (B_{1}\times B_{2}).
	\]

	Hence
	\[
		X\times Y = \bigcup_{\substack{B_{1}\in \mathcal{B}_{1} \\ B_{2}\in \mathcal{B}_{2}}} (B_{1}\times B_{2}).
	\]

	Let $A_{1}\times A_{2}$ and $B_{1}\times B_{2}$ be two members of $\mathcal{B}$.
	\[
		(A_{1}\times A_{2})\cap (B_{1}\times B_{2}) = (A_{1}\cap B_{1})\times (A_{2}\cap B_{2}).
	\]

	Since $\mathcal{B}_{1}$ is a basis for a topology $\tau_{1}$ on a set $X$ and $\mathcal{B}_{2}$ is a basis for a topology on a set $Y$, then $A_{1}\cap B_{1}$ is a union of members of $\mathcal{B}_{1}$ and $A_{2}\cap B_{2}$ is a union of members of $\mathcal{B}_{2}$. So $(A_{1}\times A_{2})\cap (B_{1}\times B_{2})$ is a union of members of $\mathcal{B}$.

	Thus, due to Proposition 2.2.8, $\mathcal{B}$ is a basis for a topology on $X\times Y$.
\end{proof}
\newpage

% chapter 2/section 2/exercise 7
\begin{exercise}
	Using Exercise 3 above and Exercise 2.1 \#8, prove that every open subset of $\mathbb{R}$ is an $F_{\sigma}$-set and a $G_{\sigma}$-set.
\end{exercise}

\begin{proof}
	Let $A$ be an open subset of $\mathbb{R}$. Then $A$ is the intersection of a countable number of open set, therefore $A$ is a $G_{\sigma}$-set.

	If $A$ is the empty set, then $A$ is indeed an $F_{\sigma}$-set.

	Since $A$ is an open subset of $\mathbb{R}$, then $A$ is a union of open intervals in $\mathbb{R}$. On the other hand, if two open intervals are not disjoint, then their union is an open interval. Therefore, $A$ is a union of disjoint open intervals.
	\[
		A = \bigcup_{i\in I} O_{i}
	\]

	where $O_{i}$ ($i\in I$) are disjoint open intervals. Due to the density of $\mathbb{Q}$ in $\mathbb{R}$, each $O_{i}$ contains a rational number $q_{i}$. Moreover, for every rational number $q_{i}$, $q_{i}\notin O_{j}$, where $i\ne j$ and $j\in I$. Hence there is a bijection from ${\{ q_{i} \}}_{i\in I}$ to ${\{ O_{i} \}}_{i\in I}$. On the other hand, ${\{ q_{i} \}}_{i\in I}$ is countable (a subset of $\mathbb{Q}$, which is countably infinite), so ${\{ O_{i} \}}_{i\in I}$ is countable.

	According to Exercise 2.1 \#8, every open interval is an $F_{\sigma}$-set (union of a countable number of closed sets). Therefore, $A$ is an $F_{\sigma}$-set.
\end{proof}
\newpage

\section{Basis for a Given Topology}

% chapter 2/section 3/exercise 1
\begin{exercise}
	Determine whether or not each of the following collections is a basis for the euclidean topology on $\mathbb{R}^{2}$:
	\begin{enumerate}[label={(\roman*)}]
		\item the collection of all ``open'' squares with sides parallel to the axes;
		\item the collection of all ``open'' discs;
		\item the collection of all ``open'' squares;
		\item the collection of all ``open'' rectangles;
		\item the collection of all ``open'' triangles.
	\end{enumerate}
\end{exercise}

\begin{proof}
	\begin{enumerate}[label={(\roman*)}]
		\item Yes.
		\item Yes.
		\item Yes.
		\item Yes.
		\item Yes.
	\end{enumerate}
\end{proof}
\newpage

% chapter 2/section 3/exercise 2
\begin{exercise}
	\begin{enumerate}[label={(\roman*)}]
		\item Let $\mathcal{B}$ be a basis for a topology $\tau$ on a non-empty set $X$. If $\mathcal{B}_{1}$ is a collection of subsets of $X$ such that $\tau\supseteq \mathcal{B}_{1} \supseteq \mathcal{B}$, prove that $\mathcal{B}_{1}$ is also a basis for $\tau$.
		\item Deduce from (i) that there exist an uncountable number of distinct bases for the euclidean topology on $\mathbb{R}$.
	\end{enumerate}
\end{exercise}

\begin{proof}
	\begin{enumerate}[label={(\roman*)}]
		\item Let $A$ be an open set in $(X, \tau)$. Therefore, $A$ is a union of open sets of $\mathcal{B}$. Since $\tau \supseteq \mathcal{B}_{1} \supseteq \mathcal{B}$, then $A$ is also a union of open sets of $\mathcal{B}_{1}$. Hence $\mathcal{B}_{1}$ is also a basis for $\tau$.
		\item Let $\mathcal{B}$ be the collection of open intervals $(a, b)$ where $a$ and $b$ are rational numbers. Then $\mathcal{B}$ is a basis for the euclidean topology on $\mathbb{R}$.

		      According to (i), for every positive real number $r$, $\mathcal{B} \cup \{ (0, r) \}$ is a basis for the euclidean topology on $\mathbb{R}$. Therefore, there exist an uncountable number of distinct bases for the euclidean topology on $\mathbb{R}$.
	\end{enumerate}
\end{proof}
\newpage

% chapter 2/section 3/exercise 3
\begin{exercise}
	Let $\mathcal{B} = \{ (a, b] \mid a, b\in\mathbb{R}, a < b \}$. As seen in Example 2.3.1, $\mathcal{B}$ is a basis for a topology $\tau$ on $\mathbb{R}$ and $\tau$ is $\boxed{\text{not}}$ the euclidean topology on $\mathbb{R}$. Nevertheless, show that each interval $(a, b)$ is open in $(\mathbb{R}, \tau)$.
\end{exercise}

\begin{proof}
	For every positive integer $n$, let $I_{n} = \left( a, b - \frac{b-a}{n} \right]$. The supremum of the set of ${\left\{b - \frac{b-a}{n}\right\}}_{n\in\mathbb{Z}_{>0}}$ is $b$, but $b$ is not the maximum element of this set. On the other hand
	\[
		(a, b) = \bigcup_{n\in\mathbb{Z}_{>0}} I_{n} = \bigcup_{n\in\mathbb{Z}_{>0}} \left( a, b - \frac{b-a}{n} \right]
	\]

	Therefore each open interval $(a, b)$ is open in $(\mathbb{R}, \tau)$.
\end{proof}
\newpage

% chapter 2/section 3/exercise 4
\begin{exercise}
	Let $C[0, 1]$ be the set of all continuous real-valued functions on $[0, 1]$.
	\begin{enumerate}[label={(\roman*)}]
		\item Show that the collection $\mathcal{M}$, where $\mathcal{M} = \{ M(f, \varepsilon) \mid f\in C[0, 1] \text{ and $\varepsilon$ is a positive real number} \}$ and $M(f, \varepsilon) = \{ g \mid g\in C[0, 1] \text{ and } \int^{1}_{0} \abs{f(x) - g(x)}dx < \varepsilon \}$, is a basis for a topology $\tau_{1}$ on $C[0, 1]$.
		\item Show that the collection $\mathcal{U}$, where $\mathcal{U} = \{ U(f, \varepsilon) \mid  f\in C[0, 1] \text{ and $\varepsilon$ is a positive real number} \}$ and $U(f, \varepsilon) = \{ g \mid g\in C[0, 1] \text{ and $\sup_{x\in [0,1]}\abs{f(x) - g(x)} < \varepsilon$ } \}$, is a basis for a topology $\tau_{2}$ on $C[0, 1]$.
		\item Prove that $\tau_{1}\ne \tau_{2}$.
	\end{enumerate}
\end{exercise}

\begin{proof}
	\begin{enumerate}[label={(\roman*)}]
		\item Due to the definition of $\mathcal{M}$ and $M(f, \varepsilon)$, we conclude that $C[0, 1] \supseteq \bigcup_{M(f, \varepsilon)\in \mathcal{M}} M(f, \varepsilon)$.
		      \[
			      C[0, 1] = \bigcup_{f\in C[0, 1]} \{ f \} \subseteq \bigcup_{f\in C[0, 1]} M(f, 1) \subseteq \bigcup_{M(f, \varepsilon)\in \mathcal{M}} M(f, \varepsilon).
		      \]

		      So $C[0, 1] = \bigcup_{M(f, \varepsilon)\in \mathcal{M}} M(f, \varepsilon)$. On the other hand
		      \begin{align*}
			      M(f_{1}, \varepsilon_{1})\cap M(f_{2}, \varepsilon_{2}) & = \left\{ f \mid f\in C[0, 1] \text{ and } \int^{1}_{0}\abs{f(x) - f_{1}(x)}dx < \varepsilon_{1} \text{ and } \int^{1}_{0}\abs{f(x) - f_{2}(x)}dx < \varepsilon_{2} \right\}
		      \end{align*}

              If $M(f_{1}, \varepsilon_{1})\cap M(f_{2}, \varepsilon_{2})\ne \varnothing$, let $g$ be an element of it. Let
              \[
                \begin{split}
                    d_{1} = \int^{1}_{0}\abs{g(x) - f_{1}(x)}dx, \\
                    d_{2} = \int^{1}_{0}\abs{g(x) - f_{2}(x)}dx.
                \end{split}
              \]

              Then $d_{1} < \varepsilon_{1}$ and $d_{2} < \varepsilon_{2}$, and $M(g, \varepsilon_{1} - d_{1})\subseteq M(f_{1}, \varepsilon_{1})$, $M(g, \varepsilon_{2} - d_{2})\subseteq M(f_{2}, \varepsilon_{2})$. Let $\varepsilon = \min\{ \varepsilon_{1} - d_{1}, \varepsilon_{1} - d_{2} \}$, then $M(g, \varepsilon_{1})\subseteq M(f_{1}, \varepsilon_{1})\cap M(f_{2}, \varepsilon_{2})$. This holds for any element $g$ in $M(f_{1}, \varepsilon_{1})\cap M(f_{2}, \varepsilon_{2})$.

              So $M(f_{1}, \varepsilon_{1})\cap M(f_{2}, \varepsilon_{2})$ is a union of elements of $\mathcal{M}$.

              Hence $\mathcal{M}$ is a basis for a topology on $C[0, 1]$.
		\item Due to the definition of $\mathcal{U}$ and $U(f, \varepsilon)$, we conclude that $C[0, 1]\supseteq \bigcup_{U(f, \varepsilon)\in \mathcal{U}} U(f, \varepsilon)$.
        \[
            C[0, 1] = \bigcup_{f\in C[0, 1]} \{ f \} \subseteq \bigcup_{f\in C[0, 1]} U(f, 1) \subseteq \bigcup_{U(f, \varepsilon)\in \mathcal{U}} U(f, \varepsilon).
        \]

        So $C[0, 1] = \bigcup_{U(f, \varepsilon)\in \mathcal{U}} U(f, \varepsilon)$. On the other hand
        \[
            U(f_{1}, \varepsilon_{1})\cap U(f_{2}, \varepsilon_{2}) = \left\{ f \mid f\in C[0, 1] \text{ and } \sup_{x\in [0, 1]} \abs{f(x) - f_{1}(x)} < \varepsilon_{1} \text{ and } \sup_{x\in [0, 1]} \abs{f(x) - f_{2}(x)} < \varepsilon_{2} \right\}
        \]

        If $U(f_{1}, \varepsilon_{1})\cap U(f_{2}, \varepsilon_{2})\ne \varnothing$, let $g$ be an element of it. Let
              \[
                \begin{split}
                    d_{1} = \sup_{x\in [0, 1]}\abs{g(x) - f_{1}(x)}dx, \\
                    d_{2} = \sup_{x\in [0, 1]}\abs{g(x) - f_{2}(x)}dx.
                \end{split}
              \]

              Then $d_{1} < \varepsilon_{1}$ and $d_{2} < \varepsilon_{2}$, and $U(g, \varepsilon_{1} - d_{1})\subseteq U(f_{1}, \varepsilon_{1})$, $U(g, \varepsilon_{2} - d_{2})\subseteq U(f_{2}, \varepsilon_{2})$. Let $\varepsilon = \min\{ \varepsilon_{1} - d_{1}, \varepsilon_{1} - d_{2} \}$, then $U(g, \varepsilon_{1})\subseteq U(f_{1}, \varepsilon_{1})\cap U(f_{2}, \varepsilon_{2})$. This holds for any element $g$ in $U(f_{1}, \varepsilon_{1})\cap U(f_{2}, \varepsilon_{2})$.

              So $U(f_{1}, \varepsilon_{1})\cap U(f_{2}, \varepsilon_{2})$ is a union of elements of $\mathcal{U}$.

              Hence $\mathcal{U}$ is a basis for a topology on $C[0, 1]$.
		\item % pending
	\end{enumerate}
\end{proof}
\newpage
