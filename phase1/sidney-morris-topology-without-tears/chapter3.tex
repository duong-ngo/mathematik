\chapter{Limit Points}

\section{Limit Points and Closure}

% chapter 3/section 1/exercise 1
\begin{exercise}
    \begin{enumerate}[label={(\alph*)}]
        \item In Example 3.1.2, find all the limit points of the following sets:
              \begin{enumerate}[label={(\roman*)}]
                  \item $\{a\}$,
                  \item $\{b,c\}$,
                  \item $\{a,c,d\}$,
                  \item $\{b,d,e\}$.
              \end{enumerate}
        \item Hence, find the closure of each of the above sets.
        \item Now find the closure of each of the above sets using the method of Example 3.1.11.
    \end{enumerate}
\end{exercise}

\begin{proof}
    \begin{enumerate}[label={(\alph*)}]
        \item \begin{enumerate}[label={(\roman*)}]
                  \item $\{a\}$ has no limit points.
                  \item The limit points of $\{b,c\}$ are $b$, $d$, $e$.
                  \item The limit points of $\{a,c,d\}$ are $b$, $c$, $d$, $e$.
                  \item The limit points of $\{b,d,e\}$ are $b$, $c$, $e$.
              \end{enumerate}
        \item \begin{enumerate}[label={(\roman*)}]
                  \item The closure of $\{a\}$ is $\{a\}$.
                  \item The closure of $\{b,c\}$ is $\{b,c,d,e\}$.
                  \item The closure of $\{a,c,d\}$ is $X$.
                  \item The closure of $\{b,d,e\}$ is $\{b,c,d,e\}$.
              \end{enumerate}
        \item The closed sets of $(X, \tau)$ are $\varnothing$, $X$, $\{a\}$, $\{b,c,d,e\}$, $\{a,b,e\}$, $\{b,e\}$.

              \begin{enumerate}[label={(\roman*)}]
                  \item The closure of $\{a\}$ is $\{a\}$.
                  \item The closure of $\{b,c\}$ is $\{b,c,d,e\}$.
                  \item The closure of $\{a,c,d\}$ is $X$.
                  \item The closure of $\{b,d,e\}$ is $\{b,c,d,e\}$.
              \end{enumerate}
    \end{enumerate}
\end{proof}
\newpage

% chapter 3/section 1/exercise 2
\begin{exercise}
    Let $(\mathbb{Z}, \tau)$ be the set of integers with the finite-closed topology. List the set of limit points of the following sets:
    \begin{enumerate}[label={(\roman*)}]
        \item $A = \{1,2,3,\ldots,10\}$.
        \item An infinite subset $E$ of $\mathbb{Z}$.
    \end{enumerate}
\end{exercise}

\begin{proof}
    \begin{enumerate}[label={(\roman*)}]
        \item Let $n\in\mathbb{Z}$. Let $S = \mathbb{Z} \setminus (A\cup \{n\})$, then $S$ is an open set of $(\mathbb{Z}, \tau)$ since the complement of $S$ is a finite set. On the other hand, $S\cap A = \varnothing$.

              Thus $A$ does not have any limit points.
        \item Let $n\in\mathbb{Z}$. If $S$ is an open set of $(\mathbb{Z}, \tau)$ containing $n$, then $\mathbb{Z}\setminus S$ is finite. Assume that $S\cap E$ has less than two elements, then the complement of $S$ is infinite (a superset of $E\setminus (S\cap E)$), which contradicts $S$ being an open set. Therefore $S\cap E$ has at least two elements, which means $S\cap E$ is not empty and $S\cap E$ contains at least one element other than $n$. So every open set containing $n$ intersects $E$ non-trivally.

              Thus every integer is a limit point of $E$.
    \end{enumerate}
\end{proof}
\newpage

% chapter 3/section 1/exercise 3
\begin{exercise}
\end{exercise}

\begin{proof}
\end{proof}
\newpage

% chapter 3/section 1/exercise 4
\begin{exercise}
\end{exercise}

\begin{proof}
\end{proof}
\newpage

% chapter 3/section 1/exercise 5
\begin{exercise}
\end{exercise}

\begin{proof}
\end{proof}
\newpage

% chapter 3/section 1/exercise 6
\begin{exercise}
\end{exercise}

\begin{proof}
\end{proof}
\newpage

\section{Neighborhoods}



\section{Connectedness}

