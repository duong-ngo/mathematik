\chapter{Topological Spaces}

\section{Topology}

% chapter 1/section 1/exercise 1
\begin{exercise}
	Let $X = \{ a, b, c, d, e, f \}$. Determine whether or not each of the following collections of subsets of $X$ is a topology on $X$
	\begin{enumerate}[label={(\alph*)}]
		\item $\tau_{1} = \{ X, \varnothing, \{ a \}, \{ a, f \}, \{ b, f \}, \{ a, b, f \} \}$.
		\item $\tau_{2} = \{ X, \varnothing, \{ a, b, f \}, \{ a, b, d \}, \{ a, b, d, f \} \}$.
		\item $\tau_{3} = \{ X, \varnothing, \{ f \}, \{ e, f \}, \{ a, f \} \}$.
	\end{enumerate}
\end{exercise}

\begin{proof}
	\begin{enumerate}[label={(\alph*)}]
		\item $\tau_{1}$ is a topology on $X$.
		\item $\tau_{1}$ is not a topology on $X$, since $\{ a, b, f \}\cap\{ a, b, d \} = \{ a, b \}\notin \tau_{2}$.
		\item $\tau_{3}$ is not a topology on $X$, since $\{ e, f \}\cup\{ a, f \} = \{ a, e, f \}\notin \tau_{3}$.
	\end{enumerate}
\end{proof}
\newpage

% chapter 1/section 1/exercise 2
\begin{exercise}
	Let $X = \{ a, b, d, d, e, f \}$. Which of the followig collections of subsets of $X$ is a topology on $X$? (Justify your answer.)
	\begin{enumerate}[label={(\alph*)}]
		\item $\tau_{1} = \{ X, \varnothing, \{ c \}, \{ b, d, e \}, \{ b, c, d, e \}, \{ b \} \}$;
		\item $\tau_{2} = \{ X, \varnothing, \{ a \}, \{ b, d, e \}, \{ a, b, d \}, \{ a, b, d, e \} \}$;
		\item $\tau_{3} = \{ X, \varnothing, \{ b \}, \{ a, b, c \}, \{ d, e, f \}, \{ b, d, e, f \} \}$.
	\end{enumerate}
\end{exercise}

\begin{proof}
	\begin{enumerate}[label={(\alph*)}]
		\item $\tau_{1}$ is not a topology on $X$, since $\{ c \}\cup\{ b \} = \{ b, c \}\notin \tau_{1}$.
		\item $\tau_{2}$ is not a topology on $X$, since $\{ b, d, e \}\cap \{ a, b, d \} = \{ b, d \}\notin\tau_{2}$.
		\item $\tau_{3}$ is a topology on $X$, since
		      \begin{itemize}
			      \item the union of any two sets in $\tau_{3}$ is in $\tau_{3}$ (this argument is valid because $\tau_{3}$ is finite)
			            \begin{align*}
				            \{ b \} \cup \{ a, b, c \}          & = \{ a, b, c \}\in \tau_{3},    \\
				            \{ b \} \cup \{ d, e, f \}          & = \{ b, d, e, f \}\in \tau_{3}, \\
				            \{ b \} \cup \{ b, d, e, f \}       & = \{ b, d, e, f \}\in \tau_{3}, \\
				            \{ a, b, c \} \cup \{ d, e, f \}    & = X\in \tau_{3},                \\
				            \{ a, b, c \} \cup \{ b, d, e, f \} & = X\in \tau_{3},                \\
				            \{ d, e, f \} \cup \{ b, d, e, f \} & = \{ b, d, e, f \}\in \tau_{3}.
			            \end{align*}
			      \item the intersection of any two sets in $\tau_{3}$ is in $\tau_{3}$
			            \begin{align*}
				            \{ b \} \cap \{ a, b, c \}          & = \{ b \}\in \tau_{3},       \\
				            \{ b \} \cap \{ d, e, f \}          & = \varnothing\in \tau_{3},   \\
				            \{ b \} \cap \{ b, d, e, f \}       & = \{ b \}\in \tau_{3},       \\
				            \{ a, b, c \} \cap \{ d, e, f \}    & = \varnothing\in \tau_{3},   \\
				            \{ a, b, c \} \cup \{ b, d, e, f \} & = \{ b \}\in \tau_{3},       \\
				            \{ d, e, f \} \cup \{ b, d, e, f \} & = \{ d, e, f \}\in \tau_{3}.
			            \end{align*}
		      \end{itemize}
	\end{enumerate}
\end{proof}
\newpage

% chapter 1/section 1/exercise 3
\begin{exercise}
	If $X = \{ a, b, c, d, e, f \}$ and $\tau$ is the discrete topology on $X$, which of the following statments are true?
	\begin{multicols}{4}
		\begin{enumerate}[label={(\alph*)}]
			\item $X\in \tau$;
			\item $\{ X \}\in \tau$;
			\item $\{\varnothing\}\in \tau$;
			\item $\varnothing\in \tau$;
			\item $\varnothing\in X$;
			\item $\{\varnothing\}\in X$;
			\item $\{ a \}\in \tau$;
			\item $a\in \tau$;
			\item $\varnothing\subseteq X$;
			\item $\{ a \}\in X$;
			\item $\{ \varnothing \}\subseteq X$;
			\item $a\in X$;
			\item $X\subseteq \tau$;
			\item $\{ a \}\subseteq \tau$;
			\item $\{ X \}\subseteq \tau$;
			\item $a\subseteq \tau$.
		\end{enumerate}
	\end{multicols}
\end{exercise}

\begin{proof}
	\begin{enumerate}[label={(\alph*)}]
		\item $X\in \tau$ --- true.
		\item $\{ X \}\in \tau$ --- false.
		\item $\{\varnothing\}\in \tau$ --- false.
		\item $\varnothing\in \tau$ --- false.
		\item $\varnothing\in X$ --- false.
		\item $\{\varnothing\}\in X$ --- false.
		\item $\{ a \}\in \tau$ --- true.
		\item $a\in \tau$ --- false.
		\item $\varnothing\subseteq X$ --- true.
		\item $\{ a \}\in X$ --- false.
		\item $\{ \varnothing \}\subseteq X$ --- false.
		\item $a\in X$ --- true.
		\item $X\subseteq \tau$ --- true.
		\item $\{ a \}\subseteq \tau$ --- false.
		\item $\{ X \}\subseteq \tau$ --- true.
		\item $a\subseteq \tau$ --- false.
	\end{enumerate}
\end{proof}
\newpage

% chapter 1/section 1/exercise 4
\begin{exercise}
	Let $(X, \tau)$ be any topological space. Verify that \color{blue}{the intersection of any finite number of members of $\tau$ is a member of $\tau$}.
\end{exercise}

\begin{proof}
	The intersection of $0$ members of $\tau$ is $\varnothing$, which is a member of $\tau$.

	Assume that the intersection of $n$ members of $\tau$ is a member of $\tau$ (where $n$ is a nonnegative integer). From $(n+1)$ members of $\tau$, we pick out $n$ members. Due to the induction hypothesis, the intersection of these $n$ members is a member of $\tau$ --- denote the intersection by $A$. Due to the definition of topological space, the intersection of $A$ and the rest member is again a member of $\tau$. Therefore, the intersection of $(n+1)$ members of $\tau$ is a member of $\tau$.

	Thus, according to the principle of mathematical induction, the intersection of any finite number of members of $\tau$ is a member of $\tau$.
\end{proof}
\newpage

% chapter 1/section 1/exercise 5
\begin{exercise}
	Let $\mathbb{R}$ be the set of all real numbers. Prove that each of the following collections of subsets of $\mathbb{R}$ is a topology.
	\begin{enumerate}[label={(\roman*)}]
		\item $\tau_{1}$ consists of $\mathbb{R}$, $\varnothing$, and every interval $(-n, n)$, for $n$ any positive integer, where $(-n, n)$ denotes that set $\{ x\in\mathbb{R} : -n < x < n \}$;
		\item $\tau_{2}$ consists of $\mathbb{R}$, $\varnothing$, and every interval $[-n, n]$, for $n$ any positive integer, where $[-n, n]$ denotes that set $\{ x\in\mathbb{R} : -n \leq x \leq n \}$;
		      % chktex-file 9
		      % chktex-file 17
		\item $\tau_{3}$ consists of $\mathbb{R}$, $\varnothing$, and every interval $[n, \infty)$, for $n$ any positive integer, where $[n, \infty)$ denotes that set $\{ x\in\mathbb{R} : n\leq x \}$.
	\end{enumerate}
\end{exercise}

\begin{proof}
	\begin{enumerate}[label={(\roman*)}]
		\item A union of a finite number of members of $\tau_{1}$ is a member of $\tau_{1}$, because
		      \[
			      \bigcup^{m}_{k=1} (-n_{k}, n_{k}) = (-n, n)
		      \]

		      where $n = \max\{ n_{1}, n_{2}, \ldots, n_{m} \}$. A union of infinite members of $\tau_{1}$ is $\mathbb{R}$, because
		      \begin{itemize}
			      \item $\bigcup^{\infty}_{k=1} (-n_{k}, n_{k})\subseteq \mathbb{R}$
			      \item Let $x$ be a real number. Due to the Archimedean property, there exists a natural number $n$ such that $x < n$. From infinite members of $\tau_{1}$, there exists a natural number $n_{k}$ such that $n\leq n_{k}$, so that $x\in (-n, n)\subseteq (-n_{k}, n_{k})$. So $\mathbb{R}\subseteq \bigcup^{\infty}_{k=1} (-n_{k}, n_{k})$.
		      \end{itemize}

		      The intersection of any two members of $\tau_{1}$ is a member of $\tau_{1}$, because
		      \[
			      (-n_{1}, n_{1})\cap (-n_{2}, n_{2}) = (-n, n)
		      \]

		      where $n = \min\{ n_{1}, n_{2} \}$.

		      Hence $\tau_{1}$ is a topology on $\mathbb{R}$.
		\item A union of a finite number of members of $\tau_{2}$ is a member of $\tau_{2}$, because
		      \[
			      \bigcup^{m}_{k=1} [-n_{k}, n_{k}] = [-n, n]
		      \]

		      where $n = \max\{ n_{1}, n_{2}, \ldots, n_{m} \}$. A union of infinite members of $\tau_{2}$ is $\mathbb{R}$, because
		      \begin{itemize}
			      \item $\bigcup^{\infty}_{k=1} [-n_{k}, n_{k}]\subseteq \mathbb{R}$
			      \item Let $x$ be a real number. Due to the Archimedean property, there exists a natural number $n$ such that $x < n$. From infinite members of $\tau_{2}$, there exists a natural number $n_{k}$ such that $n\leq n_{k}$, so that $x\in [-n, n]\subseteq [-n_{k}, n_{k}]$. So $\mathbb{R}\subseteq \bigcup^{\infty}_{k=1} [-n_{k}, n_{k}]$.
		      \end{itemize}

		      The intersection of any two members of $\tau_{2}$ is a member of $\tau_{2}$, because
		      \[
			      [-n_{1}, n_{1}]\cap [-n_{2}, n_{2}] = [-n, n]
		      \]

		      where $n = \min\{ n_{1}, n_{2} \}$.

		      Hence $\tau_{2}$ is a topology on $\mathbb{R}$.
		\item A union of members of $\tau_{3}$, $\bigcup_{n\in A} [n, \infty)$ (where $A\subseteq \mathbb{N}$) is a member of $\tau_{3}$, because
		      \begin{itemize}
			      \item if $A = \varnothing$, then the union is the empty set.
			      \item if $A\ne\varnothing$, then $A$ has a least element --- let it be $a$, then $\bigcup_{n\in A} [n, \infty) = [a, \infty)$.
		      \end{itemize}

		      The intersection of any two members of $\tau_{3}$ is a member of $\tau_{3}$, because
		      \[
			      [n_{1}, \infty) \cap [n_{2}, \infty) = [n, \infty)
		      \]

		      where $n = \min\{ n_{1}, n_{2} \}$.

		      Hence $\tau_{3}$ is a topology on $\mathbb{R}$.
	\end{enumerate}
\end{proof}
\newpage

% chapter 1/section 1/exercise 6
\begin{exercise}
	Let $\mathbb{N}$ be the set of all positive integers. Prove that each of the following collections of subsets of $\mathbb{N}$ is a topology.
	\begin{enumerate}[label={(\roman*)}]
		\item $\tau_{1}$ consists of $\mathbb{N}$, $\varnothing$, and every set $\{ 1, 2, \ldots, n \}$, for $n$ any positive integer. (This is called the {\color{red}{initial segment topology}}.)
		\item $\tau_{2}$ consists of $\mathbb{N}$, $\varnothing$, and every set $\{ n, n+1, \ldots \}$, for $n$ any positive integer. (This is called the {\color{red}{final segment topology}}.)
	\end{enumerate}
\end{exercise}

\begin{proof}
	\begin{enumerate}[label={(\roman*)}]
		\item A union of a finite number of members of $\tau_{1}$ is a member of $\tau_{1}$, because
		      \[
			      \bigcup^{m}_{k=1} \{ 1, 2, \ldots, n_{k} \} = \{ 1, 2, \ldots, n \}
		      \]

		      where $n = \max\{ n_{1}, n_{2}, \ldots, n_{k} \}$. A union of infinite members of $\tau_{1}$ is $\mathbb{N}$, which is a member of $\tau_{1}$.

		      The intersection of any two members of $\tau_{1}$ is a member of $\tau_{1}$, because
		      \[
			      \{ 1, 2, \ldots, n_{1} \} \cap \{ 1, 2, \ldots, n_{2} \} = \{ 1, 2, \ldots, n \}
		      \]

		      where $n = \min\{ n_{1}, n_{2} \}$.

		      Hence $\tau_{1}$ is a topology on $\mathbb{N}$.
		\item A union of members of $\tau_{2}$ is a member of $\tau_{2}$, because
		      \begin{itemize}
			      \item if the union has no member, then the union is the empty set, which is a member of $\tau_{2}$.
			      \item if the union is not empty, let's say $\bigcup_{n\in A} \{ n, n + 1, \ldots \}$, where $\varnothing\subset A\subseteq \mathbb{N}$, then $A$ has a least element --- let it be $a$, then $\bigcup_{n\in A} \{ n, n + 1, \ldots \} = \{ a, a+1, \ldots \}$, which is a member of $\tau_{2}$.
		      \end{itemize}

		      The intersection of any two members of $\tau_{2}$ is a member of $\tau_{2}$, because
		      \[
			      \{ n_{1}, n_{1} + 1, \ldots \}\cap \{ n_{2}, n_{2} + 1, \ldots \} = \{ n, n+1, \ldots \}
		      \]

		      where $n = \min\{ n_{1}, n_{2} \}$.

		      Hence $\tau_{2}$ is a topology on $\mathbb{N}$.
	\end{enumerate}
\end{proof}
\newpage

% chapter 1/section 1/exercise 7
\begin{exercise}
	List all possible topologies on the following sets:
	\begin{enumerate}[label={(\alph*)}]
		\item $X = \{ a, b \}$.
		\item $Y = \{ a, b, c \}$.
	\end{enumerate}
\end{exercise}

\begin{proof}
	\begin{enumerate}[label={(\alph*)}]
		\item All possible topologies on $X$ are
		      \begin{itemize}
			      \item $\tau_{1} = \{ \varnothing, X \}$ (indiscrete topology).
			      \item $\tau_{2} = \{ \varnothing, \{ a \}, X \}$.
			      \item $\tau_{3} = \{ \varnothing, \{ b \}, X \}$.
			      \item $\tau_{4} = \{ \varnothing, \{ a \}, \{ b \}, X \}$ (discrete topology).
		      \end{itemize}
		\item All possible topologies on $Y$ are
		      \begin{itemize}
			      \item $\tau_{1} = \{ \varnothing, Y \}$ (indiscrete topology).
			      \item $\tau_{2} = \{ \varnothing, \{ a \}, Y \}$.
			      \item $\tau_{3} = \{ \varnothing, \{ b \}, Y \}$.
			      \item $\tau_{4} = \{ \varnothing, \{ c \}, Y \}$.
			      \item $\tau_{5} = \{ \varnothing, \{ a \}, \{ b \}, \{ a, b \}, Y \}$.
			      \item $\tau_{6} = \{ \varnothing, \{ b \}, \{ c \}, \{ b, c \}, Y \}$.
			      \item $\tau_{7} = \{ \varnothing, \{ a \}, \{ c \}, \{ a, c \}, Y \}$.
			      \item $\tau_{8} = \{ \varnothing, \{ a \}, \{ b \}, \{ a, b \}, \{ a, c \}, Y \}$.
			      \item $\tau_{9} = \{ \varnothing, \{ a \}, \{ b \}, \{ a, b \}, \{ b, c \}, Y \}$.
			      \item $\tau_{10} = \{ \varnothing, \{ a \}, \{ b \}, \{ a, b \}, \{ b, c \}, \{ a, c \}, Y \}$.
			      \item $\tau_{11} = \{ \varnothing, \{ b \}, \{ c \}, \{ b, c \}, \{ a, b \}, Y \}$.
			      \item $\tau_{12} = \{ \varnothing, \{ b \}, \{ c \}, \{ b, c \}, \{ a, c \}, Y \}$.
			      \item $\tau_{13} = \{ \varnothing, \{ b \}, \{ c \}, \{ a, b \}, \{ b, c \}, \{ a, c \}, Y \}$.
			      \item $\tau_{14} = \{ \varnothing, \{ a \}, \{ c \}, \{ a, c \}, \{ b, c \}, Y \}$.
			      \item $\tau_{15} = \{ \varnothing, \{ a \}, \{ c \}, \{ a, c \}, \{ a, b \}, Y \}$.
			      \item $\tau_{16} = \{ \varnothing, \{ a \}, \{ c \}, \{ a, b \}, \{ b, c \}, \{ a, c \}, Y \}$.
			      \item $\tau_{17} = \{ \varnothing, \{ a \}, \{ a, b \}, Y \}$.
			      \item $\tau_{18} = \{ \varnothing, \{ a \}, \{ a, c \}, Y \}$.
			      \item $\tau_{19} = \{ \varnothing, \{ a \}, \{ a, b \}, \{ a, c \}, Y \}$.
			      \item $\tau_{20} = \{ \varnothing, \{ a \}, \{ b, c \}, Y \}$.
			      \item $\tau_{21} = \{ \varnothing, \{ b \}, \{ a, b \} Y \}$.
			      \item $\tau_{22} = \{ \varnothing, \{ b \}, \{ b, c \}, Y \}$.
			      \item $\tau_{23} = \{ \varnothing, \{ b \}, \{ a, b \}, \{ b, c \}, Y \}$.
			      \item $\tau_{24} = \{ \varnothing, \{ b \}, \{ a, c \}, Y \}$.
			      \item $\tau_{25} = \{ \varnothing, \{ c \}, \{ b, c \}, Y \}$.
			      \item $\tau_{26} = \{ \varnothing, \{ c \}, \{ a, c \}, Y \}$.
			      \item $\tau_{27} = \{ \varnothing, \{ c \}, \{ b, c \}, \{ a, c \}, Y \}$.
			      \item $\tau_{28} = \{ \varnothing, \{ c \}, \{ a, b \}, Y \}$.
			      \item $\tau_{29} = \{ \varnothing, \{ a \}, \{ b \}, \{ c \}, \{ a, b \}, \{ b, c \}, \{ a, c \}, Y \}$ (discrete topology).
		      \end{itemize}
	\end{enumerate}
\end{proof}
\newpage

% chapter 1/section 1/exercise 8
\begin{exercise}
	Let $X$ be an infinite set and $\tau$ a topology on $X$. If every infinite subset of $X$ is in $\tau$, prove that $\tau$ is the discrete topology.
\end{exercise}

\begin{proof}
	Let $x$ be an element of $X$. $A = X\setminus\{ x \}$ is an infinite set.

	Because $A$ is an infinite set, then $A$ contains a countable subset $B = \{ x_{1}, x_{2}, x_{3}, \ldots \}$. Let $C = \{ x_{2}, x_{4}, x_{6}, \ldots \}$, then $C$, $A\setminus C$ is a partition of $A$, where both $C$ and $A\setminus C$ are infinite sets.

	According to the hypothesis, $\{ x \}\cup C$ and $\{ x \}\cup (A\setminus C)$ are members of $\tau$, since they are infinite subsets of $X$. Therefore, the intersection of these two members is again a member of $\tau$. So $\{ x \}$ is a member of $\tau$ --- Furthermore, $x$ is an arbitrary element of $X$, so $\tau$ contains all singleton subsets of $X$.

	Hence $\tau$ is the discrete topology.
\end{proof}
\newpage

% chapter 1/section 1/exercise 9
\begin{exercise}
	Let $\mathbb{R}$ be the set of all real numbers. Precisely three of the following ten collections of subsets of $\mathbb{R}$ are topologies. Identify these and justify your answer.
	\begin{enumerate}[label={(\roman*)}]
		\item $\tau_{1}$ consists of $\mathbb{R}$, $\varnothing$, and every interval $(a, b)$, for $a$ and $b$ any real numbers with $a < b$;
		\item $\tau_{2}$ consists of $\mathbb{R}$, $\varnothing$, and every interval $(-r, r)$, for $r$ any positive real number;
		\item $\tau_{3}$ consists of $\mathbb{R}$, $\varnothing$, and every interval $(-r, r)$, for $r$ any positive rational number;
		\item $\tau_{4}$ consists of $\mathbb{R}$, $\varnothing$, and every interval $[-r, r]$, for $r$ any positive rational number;
		\item $\tau_{5}$ consists of $\mathbb{R}$, $\varnothing$, and every interval $(-r, r)$, for $r$ any positive irrational number;
		\item $\tau_{6}$ consists of $\mathbb{R}$, $\varnothing$, and every interval $[-r, r]$, for $r$ any positive irrational number;
		\item $\tau_{7}$ consists of $\mathbb{R}$, $\varnothing$, and every interval $[-r, r)$, for $r$ any positive real number;
		\item $\tau_{8}$ consists of $\mathbb{R}$, $\varnothing$, and every interval $(-r, r]$, for $r$ any positive real number;
		\item $\tau_{9}$ consists of $\mathbb{R}$, $\varnothing$, every interval $[-r, r]$, and every interval $(-r, r)$, for $r$ any positive real number;
		\item $\tau_{10}$ consists of $\mathbb{R}$, $\varnothing$, every interval $[-n, n]$, and every interval $(-r, r)$, for $n$ any positive integer and $r$ any positive real number.
	\end{enumerate}
\end{exercise}

\begin{proof}
	In this list, only $\tau_{2}$, $\tau_{9}$, $\tau_{10}$ are topologies on $\mathbb{R}$.

	\begin{enumerate}[label={(\roman*)}]
		\item \textbf{$\tau_{1}$ is not a topology on $\mathbb{R}$}. Because the union of $(1, 2)$ and $(3, 4)$ is not a member of $\tau_{1}$.
		\item A union of members of $\tau_{2}$ is of the form
		      \[
			      A = \bigcup_{r\in S} (-r, r)
		      \]

		      where $S\subseteq \mathbb{R}_{> 0}$. Then $\bigcup_{r\in S} (-r, r)\subseteq \mathbb{R}$.
		      \begin{enumerate}[label={\textbf{Case \arabic*.}},itemindent=1cm]
			      \item $S = \varnothing$.

			            Then the union is the empty set, which is a member of $\tau_{2}$.
			      \item $S$ is bounded from above.

			            According to the axiom of completeness, $S$ has a least upper bound. Let $u = \sup S$. Let $u'$ be a real number such that $u' < u$, then $u'$ is not the least upper bound of $S$, so there exists $a\in S$ such that $u' < a$. Therefore $u'$ is not an upper bound of $A$.

			            $u$ is indeed an upper bound of $A$, and every real number strictly less than $u$ is not an upper bound of $A$. So $u = \sup A$. On the other hand, $-u$ is a lower bound of $A$. Let $v$ be a real number such that $v > -u$, then $-v < u$. Because $u = \sup A$, then $-v < \sup A$, and there exists $a\in A$ such that $a < -v$, which is equivalent to $v > -a\geq -\abs{a}$. So $v$ is not a lower bound of $A$. Therefore, $-u$ is the greatest lower bound of $A$. Hence $A \subseteq (-u, u)$.

			            Let $x$ be an element of $(-u, u)$, then $-u < x < u$ and $-u < -\abs{x}\leq \abs{x} < u$. Then $\abs{x}$ is not an upper bound of $S$, so there exists $s\in S$ such that $\abs{x} < s$. Therefore, $x\in (-s, s)\subseteq A$. Hence $(-u, u)\subseteq A$.

			            So $A = (-u, u)$, which is a member of $\tau_{2}$.
			      \item $S$ is not bounded from above.

			            Then $A$ is not bounded from above.

			            Because $A$ is not bounded from above, and $r\in A\Longleftrightarrow -r\in A$ so $A$ is not bounded from below either. Let $x$ be a real number. Since $A$ is unbounded, then there exist real numbers $r$ and $r'$ in $A$ such that $r' < x < r$. Let $r_{0} = \max\{ \abs{r}, \abs{r'} \}$ ($r_{0}\in A$, because $r, r', -r, -r'\in A$). Since $S$ is not bounded from above, there exists $s\in S$ such that $r_{0} < s$, then $x\in (-r_{0}, r_{0})\subseteq (-s, s)\subseteq A$. So $\mathbb{R}\subseteq A$.

			            Hence $A = \mathbb{R}$, which is a member of $\tau_{2}$.
		      \end{enumerate}

		      Hence the union of arbitrary members in $\tau_{2}$ is again a member of $\tau_{2}$.

		      Let's consider the intersection of arbitrary two members $A, B$ in $\tau_{2}$
		      \begin{enumerate}[label={\textbf{Case \arabic*.}},itemindent=1cm]
			      \item $A = \varnothing$ or $B = \varnothing$.

			            $A\cap B = \varnothing$, which is a member of $\tau_{2}$.
			      \item $A = \mathbb{R}$ or $B = \mathbb{R}$.

			            If $A = \mathbb{R}$, then $A\cap B = B$, which is a member of $\tau_{2}$.

			            If $B = \mathbb{R}$, then $A\cap B = A$, which is a member of $\tau_{2}$.
			      \item $A = (-r_{1}, r_{1})$ and $B = (-r_{2}, r_{2})$.

			            Let $r = \min\{ r_{1}, r_{2} \}$, then $A\cap B = (-r, r)$, which is a member of $\tau_{2}$.
		      \end{enumerate}

		      Hence the intersection of any two members of $\tau_{2}$ is a member of $\tau_{2}$.

		      Thus \textbf{$\tau_{2}$ is a topology on $\mathbb{R}$}.
		\item Let $S = \{ r\in\mathbb{Q} \mid 0 < r < \sqrt{2} \}$. From the definition of $S$, we deduce that
		      \[
			      \bigcup_{r\in S} (-r, r)\subseteq (-\sqrt{2}, \sqrt{2}).
		      \]

		      Let $x$ be an element of $(-\sqrt{2}, \sqrt{2})$. According to the density of $\mathbb{Q}$ in $\mathbb{R}$, there exist rational numbers $q$ and $q'$ such that $-\sqrt{2} < q < x$ and $x < q' < \sqrt{2}$. Since $-\sqrt{2} < q, q' < \sqrt{2}$, then $\abs{q} < \sqrt{2}, \abs{q'} < \sqrt{2}$. Once again, according to the density of $\mathbb{Q}$ in $\mathbb{R}$, there exist rational numbers $q_{0}$ such that $\max\{ \abs{q}, \abs{q'} \} < q_{0} < \sqrt{2}$. So $q_{0}\in S$ and $x\in (-q_{0}, q_{0})\subseteq \bigcup_{r\in S} (-r, r)$, from which we deduce that $(-\sqrt{2}, \sqrt{2})\subseteq \bigcup_{r\in S}(-r, r)$.

		      Hence $\bigcup_{r\in S} (-r, r) = (-\sqrt{2}, \sqrt{2})$, which is not a member of $\tau_{3}$.

		      Thus \textbf{$\tau_{3}$ is not a topology on $\mathbb{R}$}.
		\item Let $S = \{ r\in\mathbb{Q} \mid -\sqrt{2} < r < 2 \}$. From the definition of $S$, we deduce that
		      \[
			      \bigcup_{r\in S} [-r, r]\subseteq (-\sqrt{2}, \sqrt{2}).
		      \]

		      Let $x$ be an element of $(-\sqrt{2}, \sqrt{2})$. According to the density of $\mathbb{Q}$ in $\mathbb{R}$, there exist rational numbers $q$ and $q'$ such that $-\sqrt{2} < q < x$ and $x < q' < \sqrt{2}$. Since $-\sqrt{2} < q, q' < \sqrt{2}$, then $\abs{q} < \sqrt{2}, \abs{q'} < \sqrt{2}$. Once again, according to the density of $\mathbb{Q}$ in $\mathbb{R}$, there exist rational numbers $q_{0}$ such that $\max\{ \abs{q}, \abs{q'} \} < q_{0} < \sqrt{2}$. So $q_{0}\in S$ and $x\in [-q_{0}, q_{0}]\subseteq \bigcup_{r\in S} [-r, r]$, from which we deduce that $(-\sqrt{2}, \sqrt{2})\subseteq \bigcup_{r\in S}[-r, r]$.

		      Hence $\bigcup_{r\in S} [-r, r] = (-\sqrt{2}, \sqrt{2})$, which is not a member of $\tau_{4}$.

		      Thus \textbf{$\tau_{4}$ is not a topology on $\mathbb{R}$}.
		\item Let $S = \{ s\in\mathbb{R}\setminus\mathbb{Q} \mid 0 < s < 1 \}$. From the definition of $S$, we deduce that
		      \[
			      \bigcup_{s\in S} (-s, s)\subseteq (-1, 1).
		      \]

		      Let $x$ be an element of $(-1, 1)$. According to the density of $\mathbb{R}\setminus\mathbb{Q}$ in $\mathbb{R}$, there exist irrational numbers $a$ and $a'$ such that $-1 < a < x$ and $x < a' < 1$. Since $-1 < a, a' < 1$, then $\abs{a} < 1, \abs{a'} < 1$. Once again, according to the density of $\mathbb{R}\setminus\mathbb{Q}$ in $\mathbb{R}$, there exist irrational number $a_{0}$ such that $\max\{ \abs{a}, \abs{a'} \} < a_{0} < 1$. So $a_{0}\in S$ and $x\in (-a_{0}, a_{0})\subseteq \bigcup_{s\in S} (-s, s)$, from which we deduce that $(-1, 1)\subseteq \bigcup_{s\in S}(-s, s)$.

		      Hence $\bigcup_{s\in S} (-s, s) = (-1, 1)$, which is not a member of $\tau_{5}$.

		      Thus \textbf{$\tau_{5}$ is not a topology on $\mathbb{R}$}.
		\item Let $S = \{ s\in\mathbb{R}\setminus\mathbb{Q} \mid 0 < s < 1 \}$. From the definition of $S$, we deduce that
		      \[
			      A = \bigcup_{s\in S} [-s, s]\subseteq (-1, 1).
		      \]

		      Let $x$ be an element of $(-1, 1)$. According to the density of $\mathbb{R}\setminus\mathbb{Q}$ in $\mathbb{R}$, there exist irrational numbers $a$ and $a'$ such that $-1 < a < x$ and $x < a' < 1$. Since $-1 < a, a' < 1$, then $\abs{a} < 1, \abs{a'} < 1$. Once again, according to the density of $\mathbb{R}\setminus\mathbb{Q}$ in $\mathbb{R}$, there exist irrational number $a_{0}$ such that $\max\{ \abs{a}, \abs{a'} \} < a_{0} < 1$. So $a_{0}\in S$ and $x\in [-a_{0}, a_{0}]\subseteq A$, from which we deduce that $(-1, 1)\subseteq A$.

		      Hence $A = (-1, 1)$, which is not a member of $\tau_{6}$.

		      Thus \textbf{$\tau_{6}$ is not a topology on $\mathbb{R}$}.
		\item Let $S = \{ s\in\mathbb{R} \mid 0 < s < 1 \}$. From the definition of $S$, we deduce that
		      \[
			      A = \bigcup_{s\in S} [-s, s) \subseteq (-1, 1)
		      \]

		      Let $x$ be an element of $(-1, 1)$. According to the density of $\mathbb{R}$, there exist real numbers $a$ and $a'$ such that $-1 < a < x$ and $x < a' < 1$. Since $-1 < a, a' < 1$, then $\abs{a} < 1$, $\abs{a'} < 1$. According to the density of $\mathbb{R}$, there exists a real number $a_{0}$ such that $\max\{ \abs{a}, \abs{a'} \} < a_{0} < 1$. So $a_{0}\in S$ and $x\in [-a_{0}, a_{0})\subseteq A$.

		      Hence $A = (-1, 1)$, which is not a member of $\tau_{7}$.

		      Thus \textbf{$\tau_{7}$ is not a topology on $\mathbb{R}$}.
		\item Let $S = \{ s\in\mathbb{R} \mid 0 < s < 1 \}$. From the definition of $S$, we deduce that
		      \[
			      A = \bigcup_{s\in S} (-s, s] \subseteq (-1, 1)
		      \]

		      Let $x$ be an element of $(-1, 1)$. According to the density of $\mathbb{R}$, there exist real numbers $a$ and $a'$ such that $-1 < a < x$ and $x < a' < 1$. Since $-1 < a, a' < 1$, then $\abs{a} < 1$, $\abs{a'} < 1$. According to the density of $\mathbb{R}$, there exists a real number $a_{0}$ such that $\max\{ \abs{a}, \abs{a'} \} < a_{0} < 1$. So $a_{0}\in S$ and $x\in (-a_{0}, a_{0}]\subseteq A$.

		      Hence $A = (-1, 1)$, which is not a member of $\tau_{8}$.

		      Thus \textbf{$\tau_{8}$ is not a topology on $\mathbb{R}$}.
		\item A union of members of $\tau_{9}$ is of the form
		      \[
			      A\cup B = \bigcup_{s\in S} [-s, s] \cup \bigcup_{r\in S'} (-r, r)
		      \]

		      where $S, S'\subseteq\mathbb{R}$, $A =  \bigcup_{s\in S} [-s, s]$ and $B =  \bigcup_{r\in S'} (-r, r)$.
		      \begin{enumerate}[label={\textbf{Case \arabic*.}},itemindent=1cm]
			      \item $S = \varnothing$.

			            According to (ii), $A\cup B$ is a member of $\tau_{9}$.
			      \item $S$ is bounded from above, $S$ does not have maximum element, $S' = \varnothing$.

			            Let $u = \sup S$, then $u = \sup A$ and $-u = \inf A$. So $A\cup B = A\subseteq (-u, u)$.

			            Let $x$ be an element of $(-u, u)$, then $-u < x < u$. Because $x$ is not the least upper bound nor greatest lower bound of $A$, then there exist $a$, $a'$ in $A$ such that $-u < a < x$ and $x < a' < u$. Because $a$, $a'$ are elements of $A$, then there exist $s_{1}$, $s_{2}$ in $S$ such that $a\in [-s_{1}, s_{1}]$ and $a'\in [-s_{2}, s_{2}]$. Let $s_{0} = \max\{ s_{1}, s_{2} \}$, then $s_{0}\in S$ and $x\in [-s_{0}, s_{0}]$. Therefore $(-u, u)\subseteq A$.

			            Hence $A\cup B = A = (-u, u)$, which is a member of $\tau_{9}$.
			      \item $S$ is bounded from above, $S$ does not have maximum element, $S'$ is bounded from above.

			            Let $u = \sup S$, then $u = \sup A$ and $-u = \inf A$. According to \textbf{Case 2}, $A = (-u, u)$.

			            Let $u' = \sup S'$, then $u' = \sup B$ and $-u' = \inf B$. According to (ii), $B = (-u', u')$.

			            Let $u_{0} = \max\{ u, u' \}$, then $A\cup B = (-u_{0}, u_{0})$, which is a member of $\tau_{9}$.
			      \item $S$ has maximum element, $S' = \varnothing$.

			            Let $u = \max S$, then $u = \max A$ and $-u = \min A$, then $A = [-u, u]$.

			            $A\cup B = A = [-u, u]$, which is a member of $\tau_{9}$.
			      \item $S$ has maximum element, $S'$ is bounded from above.
			            Let $u = \max S$, then $u = \max A$ and $-u = \min A$, then $A = [-u, u]$.

			            Let $u' = \sup S'$, then $u' = \sup B$ and $-u' = \inf B$. According to (ii), $B = (-u', u')$.

			            If $u\geq u'$, then $A\cup B = [-u, u]$, which is a member of $\tau_{9}$.

			            If $u < u'$, then $A\cup B = (-u', u')$, which is a member of $\tau_{9}$.
			      \item $S$ is not bounded from above or $S'$ is not bounded from above.

			            In this case, $A\cup B = \mathbb{R}$, which is a member of $\tau_{9}$.
		      \end{enumerate}

		      So the union of arbitrary members of $\tau_{9}$ is again a member of $\tau_{9}$.

		      Let's consider the intersection of arbitrary two members $X$, $Y$ in $\tau_{9}$.
		      \begin{enumerate}[label={\textbf{Case \arabic*.}},itemindent=1cm]
			      \item $X = \varnothing$ or $Y = \varnothing$.

			            $X\cap Y = \varnothing$, which is a member of $\tau_{9}$.
			      \item $X = \mathbb{R}$ or $Y = \mathbb{R}$.

			            If $X = \mathbb{R}$, then $X\cap Y = Y$, which is a member of $\tau_{9}$.

			            If $Y = \mathbb{R}$, then $X\cap Y = X$, which is a member of $\tau_{9}$.
			      \item $X = (-r, r)$ and $Y = (-r', r')$.

			            Let $r_{0} = \min\{ r, r' \}$, then $X\cap Y = (-r_{0}, r_{0})$, which is a member of $\tau_{9}$.
			      \item $X = [-r, r]$ and $Y = [-r', r']$.

			            Let $r_{0} = \min\{ r, r' \}$, then $X\cap Y = [-r_{0}, r_{0}]$, which is a member of $\tau_{9}$.
			      \item $X = (-r, r)$ and $Y = [-r', r']$.

			            If $r'\geq r$, then $X\cap Y = (-r, r)$, which is a member of $\tau_{9}$.

			            If $r' < r$, then $X\cap Y = [-r', r']$, which is a member of $\tau_{9}$.
		      \end{enumerate}

		      So the intersection of arbitrary two members of $\tau_{9}$ is again a member of $\tau_{9}$.

		      Thus \textbf{$\tau_{9}$ is a topology on $\mathbb{R}$}.
		\item A union of members of $\tau_{10}$ is of the form
		      \[
			      A\cup B = \bigcup_{n\in S} [-n, n] \cup \bigcup_{r\in S'} (-r, r)
		      \]

		      where $S\subseteq\mathbb{Z}_{> 0}$, $S'\subseteq\mathbb{R}$, $A = \bigcup_{n\in S} [-n, n]$ and $B = \bigcup_{r\in S'} (-r, r)$.
		      \begin{enumerate}[label={\textbf{Case \arabic*.}},itemindent=1cm]
			      \item $S = \varnothing$.

			            According to (ii), $A\cup B = B$ is a member of $\tau_{10}$.
			      \item $S$ is bounded from above.
			            $S$ is a subset of $\mathbb{Z}_{> 0}$ and $S$ is bounded from above, so that $S$ has a maximum element --- Let the maximum element be $a$ ($a$ is a positive integer), then $A = [-a, a]$.
			            \begin{itemize}
				            \item if $S' = \varnothing$, then $A\cup B = [-a, a]$.
				            \item if $S'$ is bounded from above, then $B = (-u', u')$ (where $u' = \sup B$, according to (ii)). If $a\geq u'$, then $A\cup B = [-a, a]$. If $a < u'$, then $A\cup B = (-u', u')$.
				            \item if $S'$ is not bounded from above, then $B = \mathbb{R}$, and $A\cup B = \mathbb{R}$.
			            \end{itemize}

			            In any of these three cases, $A\cup B$ is a member of $\tau_{10}$.
			      \item $S$ is not bounded from above.

			            Then $A = \mathbb{R}$, so $A\cup B = \mathbb{R}$, which is a member of $\tau_{10}$.
		      \end{enumerate}

		      So the union of arbitrary members of $\tau_{10}$ is again a member of $\tau_{10}$.

		      Let's consider the intersection of two arbitrary members $X$, $Y$ of $\tau_{10}$.
		      \begin{enumerate}[label={\textbf{Case \arabic*.}},itemindent=1cm]
			      \item $X = \varnothing$ or $Y = \varnothing$.

			            $X\cap Y = \varnothing$, which is a member of $\tau_{10}$.
			      \item $X = [-n, n]$ and $Y = (-r, r)$.

			            If $n\geq r$, then $X\cap Y = (-r, r)$, which is a member of $\tau_{10}$.

			            If $n < r$, then $X\cap Y = [-n, n]$, which is a member of $\tau_{10}$.
			      \item $X = \mathbb{R}$ or $Y = \mathbb{R}$.

			            If $X = \mathbb{R}$, then $X\cap Y = Y$, which is a member of $\tau_{10}$.

			            If $Y = \mathbb{R}$, then $X\cap Y = X$, which is a member of $\tau_{10}$.
		      \end{enumerate}

		      So the intersection of two arbitrary members of $\tau_{10}$ is again a member of $\tau_{10}$.

		      Thus \textbf{$\tau_{10}$ is a topology on $\mathbb{R}$}.
	\end{enumerate}
\end{proof}

\section{Open Sets, Closed Sets, and Clopen Sets}



\section{Finte-Closed Topology}



