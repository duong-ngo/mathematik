\chapter{Vector Spaces}

\section{$\mathbb{R}^{n}$ and $\mathbb{C}^{n}$}

% chapter 1/section A/exercise 1
\begin{exercise}
    Show that $\alpha + \beta = \beta + \alpha$ for all $\alpha, \beta\in \mathbb{C}$.
\end{exercise}

\begin{proof}
    Let $a_{1}, a_{2}$ be real and imaginary part of $\alpha$, $b_{1}, b_{2}$ be real and imaginary part of $\beta$. According to the definition of addition operation on $\mathbb{C}$, the real parts of $\alpha + \beta$ and $\beta + \alpha$ are $a_{1} + b_{1}$ and $b_{1} + a_{1}$, the imaginary parts of $\alpha + \beta$ and $\beta + \alpha$ are $a_{2} + b_{2}$ and $b_{2} + a_{2}$. On the other hand, addition operation on $\mathbb{R}$ is commutative, so $a_{1} + b_{1} = b_{1} + a_{1}$ and $a_{2} + b_{2} = b_{2} + a_{2}$. Hence $\alpha + \beta = \beta + \alpha$ for all $\alpha, \beta\in \mathbb{C}$
\end{proof}

% chapter 1/section A/exercise 2
\begin{exercise}
    Show that $(\alpha + \beta) + \lambda = \alpha + (\beta + \lambda)$ for all $\alpha, \beta, \lambda\in \mathbb{C}$.
\end{exercise}

\begin{proof}
    Let $a_{1}, a_{2}$ be real and imaginary part of $\alpha$, $b_{1}, b_{2}$ be real and imaginary part of $\beta$, $c_{1}, c_{2}$ be real and imaginary part of $\lambda$. According to the definition of addition operation on $\mathbb{C}$, the real parts of $(\alpha + \beta) + \lambda$ and $\alpha + (\beta + \lambda)$ are $(a_{1} + b_{1}) + c_{1}$ and $a_{1} + (b_{1} + c_{1})$, the imaginary parts of $(\alpha + \beta) + \lambda$ and $\alpha + (\beta + \lambda)$ are $(a_{2} + b_{2}) + c_{2}$ and $a_{2} + (b_{2} + c_{2})$. On the other hand, addition operation on $\mathbb{R}$ is associative, so $(a_{1} + b_{1}) + c_{1} = a_{1} + (b_{1} + c_{1})$ and $(a_{2} + b_{2}) + c_{2} = a_{2} + (b_{2} + c_{2})$. Hence $(\alpha + \beta) + \lambda = \alpha + (\beta + \lambda)$ for all $\alpha, \beta, \lambda\in \mathbb{C}$.
\end{proof}

% chapter 1/section A/exercise 3
\begin{exercise}
    Show that $(\alpha\beta)\lambda = \alpha(\beta\lambda)$ for all $\alpha, \beta, \lambda\in \mathbb{C}$.
\end{exercise}

\begin{proof}
    Let $a_{1}, a_{2}$ be real and imaginary part of $\alpha$, $b_{1}, b_{2}$ be real and imaginary part of $\beta$, $c_{1}, c_{2}$ be real and imaginary part of $\lambda$. According to the definition of multiplication operation on $\mathbb{C}$,
    \begin{align*}
        (\alpha\beta)\lambda & = ((a_{1}b_{1} - a_{2}b_{2}) + (a_{1}b_{2} + a_{2}b_{1})\iota) (c_{1} + c_{2}\iota)                                                                       \\
                             & = (a_{1}b_{1}c_{1} - a_{2}b_{2}c_{1} - a_{1}b_{2}c_{2} - a_{2}b_{1}c_{2}) + (a_{1}b_{1}c_{2} + a_{1}b_{2}c_{1} + a_{2}b_{1}c_{1} - a_{2}b_{2}c_{2})\iota, \\
        \alpha(\beta\lambda) & = (a_{1} + a_{2}\iota)((b_{1}c_{1} - b_{2}c_{2}) + (b_{1}c_{2} + b_{2}c_{1})\iota)                                                                        \\
                             & = (a_{1}b_{1}c_{1} - a_{1}b_{2}c_{2} - a_{2}b_{1}c_{2} - a_{2}b_{2}c_{1}) + (a_{1}b_{1}c_{2} + a_{1}b_{2}c_{1} + a_{2}b_{1}c_{1} - a_{2}b_{2}c_{2})\iota.
    \end{align*}

    Hence $(\alpha\beta)\lambda = \alpha(\beta\lambda)$ for all $\alpha, \beta, \lambda\in\mathbb{C}$.
\end{proof}

% chapter 1/section A/exercise 4
\begin{exercise}
    Show that $\lambda (\alpha + \beta) = \lambda\alpha + \lambda\beta$ for all $\lambda, \alpha, \beta\in\mathbb{C}$.
\end{exercise}

\begin{proof}
    Let $a_{1}, a_{2}$ be real and imaginary part of $\alpha$, $b_{1}, b_{2}$ be real and imaginary part of $\beta$, $c_{1}, c_{2}$ be real and imaginary part of $\lambda$. According to the definition of addition and multiplication operation on $\mathbb{C}$,
    \begin{align*}
        \lambda(\alpha + \beta)      & = (c_{1} + c_{2}\iota)((a_{1} + b_{1}) + (a_{2} + b_{2})\iota)                                                                \\
                                     & = (c_{1}a_{1} + c_{1}b_{1} - c_{2}a_{2} - c_{2}b_{2}) + (c_{1}a_{2} + c_{1}b_{2} + c_{2}a_{1} + c_{2}b_{1})\iota,             \\
        \lambda\alpha + \lambda\beta & = (c_{1} + c_{2}\iota)(a_{1} + a_{2}\iota) + (c_{1} + c_{2}\iota)(b_{1} + b_{2}\iota)                                         \\
                                     & = ((c_{1}a_{1} - c_{2}a_{2}) + (c_{2}a_{1} + c_{1}a_{2})\iota) + ((c_{1}b_{1} - b_{2}c_{2}) + (c_{1}b_{2} + c_{2}b_{1})\iota) \\
                                     & = (c_{1}a_{1} + c_{1}b_{1} - c_{2}a_{2} - c_{2}b_{2}) + (c_{1}a_{2} + c_{2}a_{1} + c_{1}b_{2} + c_{2}b_{1})\iota.
    \end{align*}

    Hence $\lambda(\alpha + \beta) = \lambda\alpha + \lambda\beta$ for every $\alpha, \beta, \lambda\in\mathbb{C}$.
\end{proof}

% chapter 1/section A/exercise 5
\begin{exercise}
    Show that for every $\alpha\in\mathbb{C}$, there exists a unique $\beta\in\mathbb{C}$ such that $\alpha + \beta = 0$.
\end{exercise}

\begin{proof}
    Let $a_{1}, a_{2}$ be real and imaginary part of $\alpha$, then $\beta = (-a_{1}) + (-a_{2})\iota$ satisfies $\alpha + \beta = 0$.

    Assume that complex number $\lambda$ satisfies $\alpha + \lambda = 0$, then
    \[
        \lambda = \lambda + 0 = \lambda + (\alpha + \beta) = (\lambda + \alpha) + \beta = 0 + \beta = \beta.
    \]

    Hence for every $\alpha\in \mathbb{C}$, there exists a unique $\beta\in \mathbb{C}$ such that $\alpha + \beta = 0$.
\end{proof}

% chapter 1/section A/exercise 6
\begin{exercise}
    Show that for every $\alpha\in\mathbb{C}$ with $\alpha\ne 0$, there exists a unique $\beta\in\mathbb{C}$ such that $\alpha\beta = 1$.
\end{exercise}

\begin{proof}
    Let $a_{1}, a_{2}$ be real and imaginary part of $\alpha$, then $\beta = \dfrac{a_{1}}{{a_{1}}^{2} + {a_{2}}^{2}} + \frac{-a_{2}}{{a_{1}}^{2} + {a_{2}}^{2}}\iota$ satisfies $\alpha\beta = 1$.

    Assume that complex number $\lambda$ satisfies $\alpha\lambda = 1$, then
    \[
        \lambda = \lambda\cdot 1 = \lambda(\alpha\beta) = (\lambda\alpha)\beta = 1\cdot\beta = \beta.
    \]

    Hence for every nonzero $\alpha\in\mathbb{C}$, there exists a unique $\beta\in\mathbb{C}$ such that $\alpha\beta = 1$.
\end{proof}

% chapter 1/section A/exercise 7
\begin{exercise}
    Show that
    \[
        \frac{-1 + \sqrt{3}\iota}{2}
    \]

    is a cube root of $1$ (meaning that its cube equals $1$).
\end{exercise}

\begin{proof}
    \begin{align*}
        {\left(\frac{-1 + \sqrt{3}\iota}{2}\right)}^{3} & = \frac{{(-1 + \sqrt{3}\iota)}^{3}}{8}                                                                      \\
                                                        & = \frac{(-1) + 3\cdot{(-1)}^{2}\sqrt{3}\iota + 3\cdot (-1){(\sqrt{3}\iota)}^{2} + {(\sqrt{3}\iota)}^{3}}{8} \\
                                                        & = \frac{(-1) + 3\sqrt{3}\iota + 9 + (-3\sqrt{3}\iota)}{8}                                                   \\
                                                        & = \frac{8}{8}                                                                                               \\
                                                        & = 1.
    \end{align*}
\end{proof}

% chapter 1/section A/exercise 8
\begin{exercise}
    Find two distinct square roots of $\iota$.
\end{exercise}

\begin{proof}
    We will find complex roots of the equation $z^{2} = \iota$.

    Let $z = a + b\iota$, where $a$ and $b$ are real numbers. $z^{2} = {(a + b\iota)}^{2} = (a^{2} - b^{2}) + 2ab\iota = \iota$. From this, we deduce that the real and imaginary part of both sides are identical, therefore $a^{2} - b^{2} = 0$ and $2ab = 1$.

    Since $a^{2} - b^{2} = 0$, then $a = b$ or $a = -b$. If $a = b$, then $2a^{2} = 2b^{2} = 1$, equivalently, $a = b = \dfrac{\sqrt{2}}{2}$ or $a = b = \dfrac{-\sqrt{2}}{2}$. Otherwise, $a = -b$, then $1 = 2ab = -2a^{2} < 0$, which is impossible.

    Hence the complex square roots of $\iota$ are $\dfrac{\sqrt{2}(1 + \iota)}{2}$ and $\dfrac{-\sqrt{2}(1 + \iota)}{2}$.
\end{proof}

% chapter 1/section A/exercise 9
\begin{exercise}
    Find $x\in \mathbb{R}^{4}$ such that
    \[
        (4, -3, 1, 7) + 2x = (5, 9, -6, 8).
    \]
\end{exercise}

\begin{proof}
    \begin{align*}
        2x & = (5, 9, -6, 8) - (4, -3, 1, 7) \\
           & = (1, 12, -7, 1)
    \end{align*}

    So $x = \left(\dfrac{1}{2}, 6, \dfrac{-7}{2}, \dfrac{1}{2}\right)$.
\end{proof}

% chapter 1/section A/exercise 10
\begin{exercise}
    Explain why there does not exist $\lambda\in\mathbb{C}$ such that
    \[
        \lambda (2 - 3\iota, 5 + 4\iota, -6 + 7\iota) = (12 - 5\iota, 7 + 22\iota, -32 - 9\iota).
    \]
\end{exercise}

\begin{proof}
    Assume that there does exists such a complex number $\lambda$, then
    \[
        \begin{split}
            \abs{\lambda}\cdot\abs{2 - 3\iota} = \abs{12 - 5\iota} \\
            \abs{\lambda}\cdot\abs{5 + 4\iota} = \abs{7 + 22\iota} \\
            \abs{\lambda}\cdot\abs{-6 + 7\iota} = \abs{-32 - 9\iota}
        \end{split}
    \]

    equivalently,
    \[
        \begin{split}
            \abs{\lambda}\sqrt{13} = 13,         \\
            \abs{\lambda}\sqrt{41} = \sqrt{533}, \\
            \abs{\lambda}\sqrt{85} = \sqrt{1105}.
        \end{split}
    \]

    But there is no complex number $\lambda$ satisfying the above three equations, since $\dfrac{13}{\sqrt{13}}\ne \dfrac{\sqrt{533}}{\sqrt{41}}$. Thus there does not exist $\lambda\in\mathbb{C}$ such that $\lambda (2 - 3\iota, 5 + 4\iota, -6 + 7\iota) = (12 - 5\iota, 7 + 22\iota, -32 - 9\iota)$.
\end{proof}

% chapter 1/section A/exercise 11
\begin{exercise}
    Show that $(x + y) + z = x + (y + z)$ for all $x, y, z\in \mathbb{F}^{n}$.
\end{exercise}

\begin{proof}
    The $i$th component of $(x + y) + z$ and $x + (y + z)$ are $(x_{i} + y_{i}) + z_{i}$ and $x_{i} + (y_{i} + z_{i})$. Since addition operation on $\mathbb{F}$ is associative, then $(x_{i} + y_{i}) + z_{i}$ and $x_{i} + (y_{i} + z_{i})$ are equal. So $(x_{i} + y_{i}) + z_{i} = x_{i} + (y_{i} + z_{i})$ for every $i\in\mathbb{N}$ and $i\leq n$. Thus $(x + y) + z = x + (y + z)$ for all $x, y, z\in \mathbb{F}^{n}$.
\end{proof}

% chapter 1/section A/exercise 12
\begin{exercise}
    Show that $(ab)x = a(bx)$ for all $x\in\mathbb{F}^{n}$ and all $a, b\in\mathbb{F}$.
\end{exercise}

\begin{proof}
    The $i$th component of $(ab)x$ and $a(bx)$ are $(ab)x_{i}$ and $a(bx_{i})$. Since multiplication operation on $\mathbb{F}$ is associative, then $(ab)x_{i} = a(bx_{i})$ for every $a, b, x_{i}\in\mathbb{F}$. So $(ab)x_{i} = a(bx_{i})$ for every $i\in\mathbb{N}$ and $i\leq n$. Thus $(ab)x = a(bx)$ for all $x\in\mathbb{F}^{n}$ and all $a, b\in\mathbb{F}$.
\end{proof}

% chapter 1/section A/exercise 13
\begin{exercise}
    Show that $1x = x$ for all $x\in\mathbb{F}^{n}$.
\end{exercise}

\begin{proof}
    For every $i\in\mathbb{N}$ and $i\leq n$, the $i$th component of $1x$ is $1x_{i} = x_{i}$, which is equal to the $i$th component of $x$. Thus $1x = x$ for all $x\in\mathbb{F}^{n}$.
\end{proof}

% chapter 1/section A/exercise 14
\begin{exercise}
    Show that $\lambda (x + y) = \lambda x + \lambda y$ for all $\lambda\in\mathbb{F}$ and all $x, y\in \mathbb{F}^{n}$.
\end{exercise}

\begin{proof}
    For every $i\in\mathbb{N}$ and $i\leq n$, the $i$th component of $\lambda (x + y)$ is $\lambda (x_{i} + y_{i}) = \lambda x_{i} + \lambda y_{i}$, which is also the $i$th component of $\lambda x + \lambda y$. Thus $\lambda (x + y) = \lambda x + \lambda y$.
\end{proof}

% chapter 1/section A/exercise 15
\begin{exercise}
    Show that $(a + b)x = ax + bx$ for all $a, b\in\mathbb{F}$ and all $x\in \mathbb{F}^{n}$.
\end{exercise}

\begin{proof}
    For every $i\in\mathbb{N}$ and $i\leq n$, the $i$th component of $(a + b)x$ is $(a + b)x_{i} = a x_{i} + b x_{i}$, which is also the $i$th component of $a x + b x$. Thus $(a + b)x = a x + b x$.
\end{proof}

\section{Definition of Vector Space}

% chapter 1/section B/exercise 1
\begin{exercise}
    Prove that $-(-v) = v$ for every $v\in V$.
\end{exercise}

\begin{proof}
    $-v$ is the additive inverse of $v$, and $v$ is the additive inverse of $-v$. On the other hand, $-(-v)$ is the additive inverse of $-v$, then due to the uniqueness of additive inverse, we conclude that $-(-v) = v$ for every $v\in V$.
\end{proof}

% chapter 1/section B/exercise 2
\begin{exercise}
    Suppose $a\in\mathbb{F}, v\in V$, and $av = 0$. Prove that $a = 0$ or $v = 0$.
\end{exercise}

\begin{proof}
    Assume that $a\ne 0$ and $v\ne 0$. Then there exists $a^{-1}$ such that $aa^{-1} = a^{-1}a = 1$. Hence
    \[
        0 = av = a^{-1}(av) = (a^{-1}a)v = 1v = v
    \]

    which contradicts the assumption. Thus $av = 0$ implies $a = 0$ or $v = 0$.
\end{proof}

% chapter 1/section B/exercise 3
\begin{exercise}
    Suppose $v, w\in V$. Explain why there exists a unique $x\in V$ such that $v + 3x = w$.
\end{exercise}

\begin{proof}
    $v + 3x = w$ is equivalent to $3x = w - v$, which is equivalent to $x = \frac{1}{3}w + \frac{-1}{3}v$, which uniquely determines $x$. Thus there exists a unique $x\in V$ such that $v + 3x = w$.
\end{proof}

% chapter 1/section B/exercise 4
\begin{exercise}
    The empty set is not a vector space. The empty set fails to satisfy only one of the requirements listed in the definition of a vector space (1.20). Which one?
\end{exercise}

\begin{proof}
    The empty set is not a vector space because it has no element, therefore no additive identity.
\end{proof}

% chapter 1/section B/exercise 5
\begin{exercise}
    Show that in the definition of a vector space (1.20), the additive inverse condition can be replaced with the condition that
    \[
        0v = 0\text{ for all }v\in V
    \]

    Here the $0$ on the left side is the number $0$, and the $0$ on the right side is the additive identity of $V$.
\end{exercise}

\begin{proof}
    We will show that the additive inverse condition can be derived from the new definition.

    Let $v$ be an arbitrary vector of $V$, then
    \[
        0 = 0v = (1 + (-1)v) = 1v + (-1)v = v + (-1)v.
    \]

    So $(-1)v$ is an additive inverse of $v$. Therefore every vector of $V$ has an additive inverse. Thus the new definition of vector space is equivalent to the original one.
\end{proof}

% chapter 1/section B/exercise 6
\begin{exercise}
    Let $\infty$ and $-\infty$ denote two distinct objects, neither of which is in $\mathbb{R}$. Define an addition and scalar multiplication on $\mathbb{R}\cup \{ \infty, -\infty \}$ as you could guess from the notation. Specifically, the sum and product of two real numbers is as usual, and for $t\in\mathbb{R}$ define
    \[
        t\infty = \begin{cases}
            -\infty & \text{if $t < 0$}, \\
            0       & \text{if $t = 0$}, \\
            \infty  & \text{if $t > 0$},
        \end{cases}
        \qquad
        t(-\infty) = \begin{cases}
            \infty  & \text{if $t < 0$}, \\
            0       & \text{if $t = 0$}, \\
            -\infty & \text{if $t > 0$},
        \end{cases}
    \]

    and
    \begin{align*}
        t + \infty         & = \infty + t = \infty + \infty = \infty,           \\
        t + (-\infty)      & = (-\infty) + t = (-\infty) + (-\infty) = -\infty, \\
        \infty + (-\infty) & = (-\infty) + \infty = 0.
    \end{align*}

    With these operations of addition and scalar multiplication, is $\mathbb{R}\cup \{ \infty, -\infty \}$ a vector space over $\mathbb{R}$? Explain.
\end{exercise}

\begin{proof}
    No, it is not a real vector space. Because the associativity of addition is not satisfied. Let $t$ be a nonzero real number, then
    \begin{align*}
        (\infty + (-\infty)) + t & = 0 + t = t,              \\
        \infty + ((-\infty) + t) & = \infty + (-\infty) = 0.
    \end{align*}
\end{proof}

% chapter 1/section B/exercise 7
\begin{exercise}
    Suppose $S$ is a nonempty set. Let $V^{S}$ denote the set of functions from $S$ to $V$. Define a natural addition and scalar multiplication on $V^{S}$, and show that $V^{S}$ is a vector with these definitions.
\end{exercise}

\begin{proof}
    Let $f$, $g$ be arbitrary functions from $S$ to $V$. We define $f + g$ and $\lambda f$ as the following:
    \begin{align*}
        (f + g)(x)     & = f(x) + g(x)  & \text{for all $x\in S$},                                 \\
        (\lambda f)(x) & = \lambda f(x) & \text{for all $\lambda\in\mathbb{F}$, and all $x\in S$}.
    \end{align*}

    \begin{itemize}
        \item \textbf{commutativity} is satisfied, since
              \[
                  (f + g)(x) = f(x) + g(x) = g(x) + f(x) = (g + f)(x) \quad\text{for all $f, g\in V^{S}$, and all $x\in S$}.
              \]
        \item \textbf{associativity} is satisfied, since
              \begin{align*}
                  ((f + g) + h)(x) & = (f + g)(x) + h(x)                                                           \\
                                   & = (f(x) + g(x)) + h(x)                                                        \\
                                   & = f(x) + (g(x) + h(x))                                                        \\
                                   & = f(x) + (g + h)(x)                                                           \\
                                   & = (f + (g + h))(x) \quad \text{for all $f, g, h\in V^{S}$, and all $x\in S$.}
              \end{align*}
              \begin{align*}
                  ((\lambda_{1}\lambda_{2})f)(x) & = (\lambda_{1}\lambda_{2})f(x)                                                                                                      \\
                                                 & = \lambda_{1}(\lambda_{2}f(x))                                                                                                      \\
                                                 & = \lambda_{1}(\lambda_{2}f)(x)                                                                                                      \\
                                                 & = (\lambda_{1}(\lambda_{2}f))(x) \quad \text{for all $f\in V^{S}$, all $\lambda_{1}, \lambda_{2}\in \mathbb{F}$, and all $x\in S$.}
              \end{align*}
        \item \textbf{additive identity} is satisfied, since with element $0\in V^{S}$ such that $0(x) = 0$ for all $x\in S$, then
              \[
                  (f + 0)(x) = f(x) + 0(x) = f(x) + 0 = f(x) = 0 + f(x) = 0(x) + f(x) = (0 + f)(x)
              \]

              for all $f\in V^{S}$, and all $x\in S$.
        \item \textbf{additive inverse} is satisfied, since with $-f\in V^{S}$ such that $(-f)(x) = -f(x)$ for all $x\in S$, then
              \[
                  (f + (-f))(x) = f(x) + (-f)(x) = f(x) + (-f(x)) = 0
              \]

              for all $x\in S$.
        \item \textbf{multiplicative identity} is satisfied, since for all $f\in V^{S}$ and all $x\in S$,
              \[
                  (1\cdot f)(x) = 1\cdot f(x) = f(x).
              \]
        \item \textbf{distributivity properties} are satisfied, since for all $\lambda_{1}, \lambda_{2}\in \mathbb{F}$, all $f, g\in V^{S}$, and all $x\in S$,
              \[
                  (\lambda_{1}(f + g))(x) = \lambda_{1}(f + g)(x) = \lambda_{1}(f(x) + g(x)) = (\lambda_{1}f)(x) + (\lambda_{1}g)(x) = (\lambda_{1}f + \lambda_{1}g)(x).
              \]
              \[
                  ((\lambda_{1} + \lambda_{2})f)(x) = (\lambda_{1} + \lambda_{2})f(x) = \lambda_{1}f(x) + \lambda_{2}f(x) = (\lambda_{1}f)(x) + (\lambda_{2}f)(x) = (\lambda_{1}f + \lambda_{2}f)(x).
              \]
    \end{itemize}
\end{proof}

% chapter 1/section B/exercise 8
\begin{exercise}
    Suppose $V$ is a real vector space.
    \begin{itemize}
        \item The \textit{complexification} of $V$, denoted by $V_{\mathbb{C}}$, equals $V\times V$. An element of $V_{\mathbb{C}}$ is an ordered pair $(u, v)$, where $u, v\in V$, but we write this as $u + \iota v$.
        \item Addition on $V_{\mathbb{C}}$ is defined by
              \[
                  (u_{1} + \iota v_{1}) + (u_{2} + \iota v_{2}) = (u_{1} + u_{2}) + \iota (v_{1} + v_{2})
              \]

              for all $u_{1}, v_{1}, u_{2}, v_{2}\in V$.
        \item Complex scalar multiplication on $V_{\mathbb{C}}$ is defined by
              \[
                  (a + b\iota) (u + \iota v) = (au - bv) + \iota(av + bu)
              \]

              for all $a, b\in\mathbb{R}$ and all $u, v\in V$.
    \end{itemize}

    Prove that with the definitions of addition and scalar multiplication as above, $V_{\mathbb{C}}$ is a complex vector space.
\end{exercise}

\begin{proof}
    \begin{itemize}
        \item Addition on $V_{\mathbb{C}}$ is commutative, since for all $(u, v), (u', v')\in V_{\mathbb{C}}$,
              \[
                  (u, v) + (u', v') = (u + u', v + v') = (u' + u, v' + v) = (u', v') + (u, v).
              \]
        \item Addition on $V_{\mathbb{C}}$ is associative, since for all $(u, v), (u', v'), (u'', v'')\in V_{\mathbb{C}}$,
              \begin{align*}
                  ((u, v) + (u', v')) + (u'', v'') & = (u + u', v + v') + (u'', v'')     \\
                                                   & = ((u + u') + u'', (v + v') + v'')  \\
                                                   & = (u + (u' + u''), v + (v' + v''))  \\
                                                   & = (u, v) + (u' + u'', v' + v'')     \\
                                                   & = (u, v) + ((u', v') + (u'', v'')).
              \end{align*}
        \item Addition on $V_{\mathbb{C}}$ has identity element, which is $(0, 0)$, since for all $(u, v)\in V_{\mathbb{C}}$,
              \[
                  (u, v) + (0, 0) = (u + 0, v + 0) = (u, v).
              \]
        \item Every element of $V_{\mathbb{C}}$ has an additive inverse. For every $(u, v)\in V_{\mathbb{C}}$,
              \[
                  (u, v) + (-u, -v) = (u + (-u), v + (-v)) = (0, 0).
              \]
        \item For all $(u, v)\in V_{\mathbb{C}}$
              \[
                  1(u, v) = (1 + 0\iota) (u, v) = (1u - 0v, 1v + 0u) = (u, v).
              \]
        \item For all $(u, v)\in V_{\mathbb{C}}$, all $z_{1} = a_{1} + b_{1}\iota, z_{2} = a_{2} + b_{2}\iota\in \mathbb{C}$
              \begin{align*}
                  (z_{1}z_{2})(u, v) & = (a_{1}a_{2} - b_{1}b_{2} + (a_{1}b_{2} + a_{2}b_{1})\iota)(u, v)                                                    \\
                                     & = ((a_{1}a_{2} - b_{1}b_{2})u - (a_{1}b_{2} + a_{2}b_{1})v, (a_{1}a_{2} - b_{1}b_{2})v + (a_{1}b_{2} + a_{2}b_{1})u), \\
                  z_{1}(z_{2}(u, v)) & = (a_{1} + b_{1}\iota)(a_{2}u - b_{2}v, a_{2}v + b_{2}u)                                                              \\
                                     & = (a_{1}(a_{2}u - b_{2}v) - b_{1}(a_{2}v + b_{2}u), a_{1}(a_{2}v + b_{2}u) + b_{1}(a_{2}u - b_{2}v))                  \\
                                     & = ((a_{1}a_{2} - b_{1}b_{2})u - (a_{1}b_{2} + a_{2}b_{1})v, (a_{1}a_{2} - b_{1}b_{2})v + (a_{1}b_{2} + a_{2}b_{1})u).
              \end{align*}

              Hence $(z_{1}z_{2})(u, v) = z_{1}(z_{2}(u, v))$.
        \item Addition and scalar multiplication on $V_{\mathbb{C}}$ are distributive. For all $(u, v), (u', v')\in V_{\mathbb{C}}$, all $z_{1} = a_{1} + b_{1}\iota, z_{2} = a_{2} + b_{2}\iota\in \mathbb{C}$
              \begin{align*}
                  (z_{1} + z_{2})(u, v) & = ((a_{1} + a_{2}) + (b_{1} + b_{2})\iota)(u, v)                             \\
                                        & = ((a_{1} + a_{2})u - (b_{1} + b_{2})v, (a_{1} + a_{2})v + (b_{1} + b_{2})u) \\
                                        & = (a_{1}u - b_{1}v, a_{1}v + b_{1}u) + (a_{2}u - b_{2}v, a_{2}v + b_{2}u)    \\
                                        & = z_{1}(u, v) + z_{2}(u, v).
              \end{align*}
              \begin{align*}
                  z_{1}((u, v) + (u', v')) & = (a_{1} + b_{1}\iota)(u + u', v + v')                                        \\
                                           & = (a_{1}(u + u') - b_{1}(v + v'), a_{1}(v + v') + b_{1}(u + u'))              \\
                                           & = (a_{1}u - b_{1}v, a_{1}v + b_{1}u) + (a_{1}u' - b_{1}v', a_{1}v' + b_{1}u') \\
                                           & = z_{1}(u, v) + z_{2}(u', v').
              \end{align*}
    \end{itemize}
\end{proof}

\section{Subspaces}
