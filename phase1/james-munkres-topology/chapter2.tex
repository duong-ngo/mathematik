\chapter{Topological Spaces and Continuous Functions}

\section{Topological Spaces}

This section has no exercise.

\section{Basis for a Topology}

% chapter2:section13:exercise1
\begin{exercise}\label{chapter2:section13:exercise1}
    Let $X$ be a topological space; let $A$ be a subset of $X$. Suppose that for each $x\in A$ there is an open set $U$ containing $x$ such that $U\subset A$. Show that $A$ is open in $X$.
\end{exercise}

\begin{proof}
    If $A$ is empty then $A$ is open in $X$.

    Suppose $A$ is nonempty. For every $x\in A$, there is an open set $U_{x}$ containing $x$ such that $U_{x}\subset A$. Let's define
    \[
        B = \bigcup_{x\in A}U_{x}
    \]

    then $B\subset A$ because $U_{x}\subset A$ for every $x\in A$. If $a\in A$, then $a\in U_{a}\subset B$, so $A\subset B$. Hence $A = B$, which means $A$ is the union of a collection of open sets. Thus $A$ is open in $X$.
\end{proof}

% chapter2:section13:exercise2
\begin{exercise}\label{chapter2:section13:exercise2}
    Consider the nine topologies on the set $X = \{ a, b, c \}$ indicated in Example 1 of $\S{12}$. Compare them, that is, for each pair of topologies, determine whether they are comparable, and if so, which is the finer.
\end{exercise}

\begin{proof}
    \begin{enumerate}[label={(\roman*)}]
        \item $\{ \varnothing, X \}$.
        \item $\{ \varnothing, X, \{ b \} \}$.
        \item $\{ \varnothing, X, \{ a, b \} \}$.
        \item $\{ \varnothing, X, \{ a \}, \{ a, b \} \}$.
        \item $\{ \varnothing, X, \{ a \}, \{ b, c \} \}$.
        \item $\{ \varnothing, X, \{ a \}, \{ b \}, \{ a, b \} \}$.
        \item $\{ \varnothing, X, \{ b \}, \{ a, b \}, \{ b, c \} \}$.
        \item $\{ \varnothing, X, \{ b \}, \{ c \}, \{ a, b \}, \{ b, c \} \}$.
        \item $\{ \varnothing, X, \{ a \}, \{ b \}, \{ c \}, \{ a, b \}, \{ b, c \}, \{ a, c \} \}$.
    \end{enumerate}

    \begin{itemize}
        \item (i) is not strictly finer than any topology in this list.
        \item (ii) is strictly finer than (i).
        \item (iii) is strictly finer than (i).
        \item (iv) is strictly finer than (iii) and (i).
        \item (v) is strictly finer than (i).
        \item (vi) is strictly finer than (i), (ii), (iii), (iv).
        \item (vii) is strictly finer than (i), (ii), (iii).
        \item (viii) is strictly finer than (i), (ii), (iii), (vii).
        \item (ix) is strictly finer than every other topology in this list.
    \end{itemize}
\end{proof}

% chapter2:section13:exercise3
\begin{exercise}\label{chapter2:section13:exercise3}
    Show that the collection $\mathscr{T}_{c}$ given in Example 4 of $\S{12}$ is a topology on the set $X$. Is the collection
    \[
        \mathscr{T}_{\infty} = \{ U \mid \text{$X - U$ is infinite or empty or all of $X$} \}
    \]

    a topology on $X$?
\end{exercise}

\begin{proof}
    $X - \varnothing$ is all of $X$ and $X - X = \varnothing$ so $\varnothing, X\in\mathscr{T}_{c}$.

    Let ${\{ U_{i} \}}_{i\in I}$ be a collection of sets $U$ such that $X - U$ is either countable or is all of $X$. By De Morgan's law
    \[
        X - \bigcup_{i\in I}U_{i} = \bigcap_{i\in I}(X - U_{i}).
    \]

    Because $X - U_{i}$ is countable or is all of $X$ for $i\in I$, so $X - \bigcup_{i\in I}U_{i}$ is either countable or is all of $X$.

    Let $U_{1}, \ldots, U_{n}$ such that $X - U_{i}$ is either countable or is all of $X$ for $i = 1,\ldots,n$. By De Morgan's law
    \[
        X - \bigcap^{n}_{i=1}U_{i} = \bigcup^{n}_{i=1}(X - U_{i})
    \]

    Because $X - U_{i}$ is countable or is all of $X$ for $i\in X$, so $X - \bigcap^{n}_{i=1}U_{i}$ is either countable or is all of $X$.

    Hence $\mathscr{T}_{c}$ is a topology on $X$.

    \bigskip
    No, $\mathscr{T}_{\infty}$ is not a topology on an infinite set $X$. Assume that it is, then every singleton subset of $X$ belongs to $\mathscr{T}_{\infty}$ because $X - \{ a \}$ is infinite for every $a\in X$. Therefore $\mathscr{T}_{\infty}$ is the powerset of $X$. Let $x\in X$, then $X - \{ x \}\in\mathscr{T}_{\infty}$. However, $X - (X - \{ x \}) = \{ x \}$, which is neither all of $X$, empty, or infinite. Thus $\mathscr{T}_{\infty}$ is not a topology on $X$.
\end{proof}

% chapter2:section13:exercise4
\begin{exercise}\label{chapter2:section13:exercise4}
    \begin{enumerate}[label={(\alph*)}]
        \item If $\{ \mathscr{T}_{\alpha} \}$ is a family of topologies on $X$, show that $\bigcap \mathscr{T}_{\alpha}$ is a topology on $X$. Is $\bigcup \mathscr{T}_{\alpha}$ a topology on $X$?
        \item Let $\{ \mathscr{T}_{\alpha} \}$ be a family of topologies on $X$. Show that there is a unique smallest topology on $X$ containing all the collections $\mathscr{T}_{\alpha}$, and a unique largest topology contained in all $\mathscr{T}_{\alpha}$.
        \item If $X = \{ a, b, c \}$, let
              \[
                  \mathscr{T}_{1} = \{ \varnothing, X, \{ a \}, \{ a, b \} \}\quad\text{and}\quad\mathscr{T}_{2} = \{ \varnothing, X, \{ a \}, \{ b, c \} \}
              \]

              Find the smallest topology containing $\mathscr{T}_{1}$ and $\mathscr{T}_{2}$, and the largest topology contained in $\mathscr{T}_{1}$ and $\mathscr{T}_{2}$.
    \end{enumerate}
\end{exercise}

\begin{proof}
    \begin{enumerate}[label={(\alph*)}]
        \item $\varnothing$ and $X$ belong to $\mathscr{T}_{\alpha}$ for all $\alpha$ so $\varnothing$ and $X$ belong to $\bigcap\mathscr{T}_{\alpha}$.

              If ${\{F_{i}\}}_{i\in I}$ is a collection of sets in $\bigcap\mathscr{T}_{\alpha}$ then $\bigcup_{i\in I}F_{i}\in \mathscr{T}_{\alpha}$ for every $\alpha$. So $\bigcup_{i\in I}F_{i}\in \bigcap\mathscr{T}_{\alpha}$.

              If $F_{1}, \ldots, F_{n}$ are sets in $\bigcap\mathscr{T}_{\alpha}$ then $\bigcap^{n}_{i=1}F_{i}\in \mathscr{T}_{\alpha}$ for every $\alpha$. So $\bigcap^{n}_{i=1}F_{i}\in \bigcap\mathscr{T}_{\alpha}$.

              Hence $\bigcap\mathscr{T}_{\alpha}$ is a topology on $X$.

              \bigskip
              $\bigcup\mathscr{T}_{\alpha}$ is not necessarily a topology on $X$. For example: $X = \{ a, b, c \}$, $\mathscr{T}_{1} = \{ \varnothing, X, \{ a \} \}$ and $\mathscr{T}_{1} = \{ \varnothing, X, \{ b \} \}$ are topologies on $X$. However $\mathscr{T}_{1}\cup\mathscr{T}_{2} = \{ \varnothing, X, \{ a \}, \{ b \} \}$ is not a topology on $X$ because it contains $\{ a \}$ and $\{ b \}$ but doesn't contain $\{ a, b \}$.
        \item Every topology on $X$ is contained in the discrete topology on $X$, so the family of topologies containing $\{ \mathscr{T}_{\alpha} \}$ is not empty. Let $\mathcal{T}_{1}$ be the intersection of the family of topologies containing $\{ \mathscr{T}_{\alpha} \}$. Part (a) implies $\mathcal{T}_{1}$ is a topology on $X$. Moreover, every topology on $X$ containing $\{\mathscr{T}_{\alpha}\}$ contains $\mathcal{T}_{1}$ so $\mathcal{T}_{1}$ is the smallest topology $X$ containing $\{\mathscr{T}_{\alpha}\}$. Thus there is a unique smallest topology on $X$ containing all the collections $\mathscr{T}_{\alpha}$.

              By part (a), $\bigcap \mathscr{T}_{\alpha}$ is a topology on $X$. Moreover, $\bigcap \mathscr{T}_{\alpha}$ is contained in $\mathscr{T}_{\alpha}$ for every $\alpha$. Assume $\mathscr{T}$ is a topology on $X$ contained in $\mathscr{T}_{\alpha}$ for every $\alpha$, then $\mathscr{T}\subset \bigcap \mathscr{T}_{\alpha}$. Thus there is a unique largest topology on $X$ contained in every set in every collection $\mathscr{T}_{\alpha}$.
        \item The smallest topology containing $\mathscr{T}_{1}$ and $\mathscr{T}_{2}$ is $\{ \varnothing, X, \{ a \}, \{ b \}, \{ a, b \}, \{ b, c \} \}$.

              The largest topology contained in $\mathscr{T}_{1}$ and $\mathscr{T}_{2}$ is $\{ \varnothing, X, \{ a \} \}$.
    \end{enumerate}
\end{proof}

% chapter2:section13:exercise5
\begin{exercise}\label{chapter2:section13:exercise5}
    Show that if $\mathscr{A}$ is a basis for a topology on $X$, then the topology generated by $\mathscr{A}$ equals the intersection of all topologies on $X$ that contain $\mathscr{A}$. Prove the same if $\mathscr{A}$ is a subbasis.
\end{exercise}

\begin{proof}[Proof when $\mathscr{A}$ is a basis]
    The intersection of all topologies on $X$ that contains $\mathscr{A}$ is the smallest topology on $X$ that contains $\mathscr{A}$.

    In the topology on $X$ generated by $\mathscr{A}$, every open set is the union of sets in $\mathscr{A}$. Let $A$ be an open set in the topology on $X$ generated by $\mathscr{A}$. Let $\mathscr{T}$ be a topology on $X$ that contains $\mathscr{A}$, then every subset of $X$ which is the union of sets in $\mathscr{A}$ is in $\mathscr{T}$, which means $A$ is in $\mathscr{T}$. Therefore the topology on $X$ generated by $\mathscr{A}$ is contained in any topology on $X$ that contains $\mathscr{A}$.

    Thus the topology generated by the basis $\mathscr{A}$ equals the intersection of all topologies on $X$ that contain $\mathscr{A}$.
\end{proof}

\begin{proof}[Proof when $\mathscr{A}$ is a subbasis]
    The intersection of all topologies on $X$ that contains $\mathscr{A}$ is the smallest topology on $X$ that contains $\mathscr{A}$.

    Let $\mathscr{T}$ be the topology on $X$ generated by the subbbasis $\mathscr{A}$, then every set in $\mathscr{T}$ is the union of finite intersections of elements of $\mathscr{A}$. Let $A$ be an union of finite intersections of elements of $\mathscr{A}$.

    Let $\mathscr{T}'$ be a topology on $X$ containing $\mathscr{A}$. Every finite intersection of elements of $\mathscr{A}$ is open, so the union of finite intersections of elements of $\mathscr{A}$ is open. Therefore $A$ is in $\mathscr{T}'$. Since $A$ and $\mathscr{T}'$ are arbitrary, we conclude that every topology that contains $\mathscr{A}$ also contains $\mathscr{T}$.

    Thus the topology generated by the subbasis $\mathscr{A}$ equals the intersection of all topologies on $X$ that contains $\mathscr{A}$.
\end{proof}

% chapter2:section13:exercise6
\begin{exercise}\label{chapter2:section13:exercise6}
    Show that the topologies of $\mathbb{R}_{\ell}$ and $\mathbb{R}_{K}$ are not comparable.
\end{exercise}

\begin{proof}
    $\halfopenright{0, 1}$ is open in $\mathbb{R}_{\ell}$ by the definition of the lower limit topology. Let $A$ be an open set in $\mathbb{R}_{K}$ that contains $0$, then $A$ also contains negative numbers, which are not in $\halfopenright{0, 1}$. Therefore $A$ is not contained in $\halfopenright{0, 1}$, hence $\halfopenright{0, 1}$ is not open in $\mathbb{R}_{K}$.

    $\openinterval{-1, 1} - K$ is open in $\mathbb{R}_{K}$ by the definition of the $K$-topology. Let $A$ be a half-open interval $\halfopenright{a, b}$ that contains $0$, then $b > 0$, so $\halfopenright{a, b}\cap K\ne \varnothing$. Therefore every half-open interval $\halfopenright{a, b}$ that contains $0$ is not contained in $\openinterval{-1, 1} - K$. Therefore $\openinterval{-1, 1} - K$ is not open in $\mathbb{R}_{\ell}$.

    Thus $\mathbb{R}_{\ell}$ and $\mathbb{R}_{K}$ are not comparable.
\end{proof}

% chapter2:section13:exercise7
\begin{exercise}\label{chapter2:section13:exercise7}
    Consider the following topologies on $\mathbb{R}$
    \begin{align*}
        \mathscr{T}_{1} & = \text{the standard topology}, \\
        \mathscr{T}_{2} & = \text{the topology of $\mathbb{R}_{K}$}, \\
        \mathscr{T}_{3} & = \text{the finite complement topology}, \\
        \mathscr{T}_{4} & = \text{the upper limit topology, having all sets $\halfopenleft{a, b}$ as basis}, \\
        \mathscr{T}_{5} & = \text{the topology having all sets $\openinterval{-\infty, a} = \{ x \mid x < a \}$ as basis}
    \end{align*}

    Determine, for each of these topologies, which of the others it contains.
\end{exercise}

\begin{proof}
    \begin{enumerate}[label={(\roman*)}]
        \item For every $\openinterval{-\infty, a}$
              \[
                  \openinterval{-\infty, a} = \bigcup_{x < a}\openinterval{x, a}
              \]

              so $\mathscr{T}_{1}$ contains $\mathscr{T}_{5}$. On the other hand, $\openinterval{0, 1}$ does not contain any infinite interval so $\mathscr{T}_{5}$ does not contain $\mathscr{T}_{1}$.

              Hence $\mathscr{T}_{5}\subsetneq \mathscr{T}_{1}$.
        \item $\mathscr{T}_{1}\subset \mathscr{T}_{2}$ by definition.

              $\openinterval{-1, 1} - K$ is an open set in $\mathbb{R}_{K}$. However, every open interval $\openinterval{a, b}$ that contains $0\in \openinterval{-1, 1} - K$ is not contain in $\openinterval{-1, 1} - K$ because $\openinterval{a, b}\cap K \ne \varnothing$. So $\openinterval{-1, 1} - K$ is not an open set in the standard topology on $\mathbb{R}$.

              Hence $\mathscr{T}_{1}\subsetneq \mathscr{T}_{2}$.
        \item $\openinterval{0, 1}$ is not open in the finite complement topology, because the complement of $\openinterval{0, 1}$ is not the entire $\mathbb{R}$ or finite.

              Let $A$ be an open set in $\mathscr{T}_{3}$. If $\mathbb{R} - A$ is $\mathbb{R}$ then $A = \varnothing$, an open set in the standard topology. If $\mathbb{R} - A = \varnothing$ then $A = \mathbb{R}$, which is also an open set in the standard topology. If $\mathbb{R} - A$ is a nonempty finite set $\{ a_{1}, \ldots, a_{n} \}$ where $a_{i} < a_{i+1}$ then
              \[
                  A = \openinterval{-\infty, a_{1}} \cup \bigcup^{n-1}_{i=1}\openinterval{a_{i}, a_{i+1}} \cup \openinterval{a_{n}, \infty}
              \]

              and $A$ is indeed an open set in the standard topology.

              Hence $\mathscr{T}_{3}\subsetneq \mathscr{T}_{1}$.
        \item $\halfopenleft{0, 1}$ is open in $\mathscr{T}_{4}$ by definition. However, every basis element of $\mathscr{T}_{2}$ that contains $1$ is not contained in $\halfopenleft{0, 1}$.

              For every open interval $\openinterval{a, b}$
              \[
                  \openinterval{a, b} = \bigcup_{x\in\openinterval{a, b}}\halfopenleft{a, x}
              \]

              so $\openinterval{a, b}$ is open in $\mathscr{T}_{4}$.

              For every $\openinterval{a, b} - K$, let $x$ be an element of $\openinterval{a, b} - K$. If $x\leq 0$ then $x\in\halfopenleft{a, 0}\subset \openinterval{a, b} - K$. If $x > 0$, then there exists a unique positive integer $n$ such that $n\leq \frac{1}{x} < n+1$, which means $\frac{1}{n+1} < x\leq \frac{1}{n}$. Therefore  $\frac{1}{n+1} < x < \frac{1}{n}$ because $x\notin K$. So $x\in\halfopenleft{\frac{1}{n+1}, x}\subset \openinterval{a, b} - K$.

              Hence $\mathscr{T}_{2}\subsetneq \mathscr{T}_{4}$.
        \item Every element of $\{ 1 \}$ is not contained in any set of the form $\openinterval{-\infty, a}$ which is contained in $\{ 1 \}$.

              Every nonempty set other than $\mathbb{R}$ in $\mathscr{T}_{5}$ is not open in $\mathscr{T}_{3}$.

              So $\mathscr{T}_{3}$ and $\mathscr{T}_{5}$ are not comparable.
    \end{enumerate}

    The result are summarized in the following Hasse diagram.
    \begin{center}
        \begin{tikzpicture}
            \node (topology4) at (0, 0) {$\mathscr{T}_{4}$};
            \node [below of=topology4] (topology2) {$\mathscr{T}_{2}$};
            \node [below of=topology2] (topology1) {$\mathscr{T}_{1}$};
            \node [below left of=topology1, node distance=1.5cm] (topology3) {$\mathscr{T}_{3}$};
            \node [below right of=topology1, node distance=1.5cm] (topology5) {$\mathscr{T}_{5}$};
            \draw (topology4) -- (topology2) -- (topology1);
            \draw (topology3) -- (topology1) -- (topology5);
        \end{tikzpicture}
    \end{center}
\end{proof}

% chapter2:section13:exercise8
\begin{exercise}\label{chapter2:section13:exercise8}
    \begin{enumerate}[label={(\alph*)}]
        \item Apply Lemma 13.2 to show that the countable collection
        \[
            \mathscr{B} = \{ \openinterval{a, b} \mid \text{$a < b$, $a$ and $b$ are rational} \}
        \]

        is a basis that generates the standard topology on $\mathbb{R}$.
        \item Show that the collection
        \[
            \mathscr{C} = \{ \halfopenright{a, b} \mid \text{$a < b$, $a$ and $b$ are rational} \}
        \]

        is a basis that generates a topology different from the lower limit topology on $\mathbb{R}$.
    \end{enumerate}
\end{exercise}

\begin{proof}
    \begin{enumerate}[label={(\alph*)}]
        \item Let $U$ be an nonempty open set in $\mathbb{R}$ and $x$ be an element in $U$. $U$ is the union of open intervals, so there exists $\openinterval{a, b}$ such that $x\in \openinterval{a, b}\subset A$. There are rational numbers $r_{1}, r_{2}$ such that $a < r_{1} < x$ and $x < r_{2} < b$, hence $x\in\openinterval{r_{1}, r_{2}}\subset \openinterval{a, b}\subset U$. By Lemma 13.2, $\mathscr{B}$ is a basis that generates the standard topology on $\mathbb{R}$.
        \item $\halfopenright{\sqrt{2}, 2}$ is an open set in the lower limit topology on $\mathbb{R}$. However, every set $\halfopenright{a, b}$ in $\mathscr{C}$ that contains $\sqrt{2}$ satisfies $a < \sqrt{2}$, which implies $\halfopenright{a, b}$ is not contained in $\halfopenright{\sqrt{2}, \sqrt{3}}$. By Lemma 13.2, $\mathscr{C}$ is a basis that generated a topology other than the lower limit topology on $\mathbb{R}$.
    \end{enumerate}
\end{proof}

\section{The Order Topology}

This section has no exercise.

\section{The Product Topology on $X\times Y$}

This section has no exercise.

\section{The Subspace Topology}

% chapter2:section16:exercise1
\begin{exercise}\label{chapter2:section16:exercise1}
    Show that if $Y$ is a subspace of $X$, and $A$ is a subset of $Y$, then the topology $A$ inherits as a subspace of $Y$ is the same as the topology it inherits as a subspace of $X$.
\end{exercise}

\begin{proof}
    Let $U$ be a set in the topology on $A$ which inherits as a subspace of $Y$. By the definition of subspace topology, there is an open set $V$ in $Y$ such that $U = A\cap V$, and there is an open set $W$ in $X$ such that $V = Y\cap W$. Therefore
    \[
        U = A\cap V = A\cap (Y\cap W) = (A\cap Y)\cap W = A\cap W.
    \]

    So $U$ is also in the topology on $A$ which inherits as a subspace of $X$.

    Hence the topology $A$ inherits as a subspace of $Y$ is the same as the topology it inherits as a subspace of $X$.
\end{proof}

% chapter2:section16:exercise2
\begin{exercise}\label{chapter2:section16:exercise2}
    If $\mathscr{T}$ and $\mathscr{T}'$ are topologies on $X$ and $\mathscr{T}'$ is strictly finer than $\mathscr{T}$, what can you say about the corresponding subspace topologies on the subset $Y$ of $X$?
\end{exercise}

\begin{proof}
    Let $U$ be an open set in the subspace topology of $\mathscr{T}$ on $Y$, then there is an open set $A$ in $\mathscr{T}$ such that $U = Y\cap A$. Since $\mathscr{T}\subset\mathscr{T}'$ then $A\in \mathscr{T}'$. We have
    \begin{itemize}[itemsep=0pt]
        \item $Y\cap A$ is in the topology on $Y$ which inherits as a subspace of $(X, \mathscr{T})$.
        \item $Y\cap A$ is in the topology in $Y$ which inherits as a subspace of $(X, \mathscr{T}')$.
    \end{itemize}

    So $U$ is also an open set in the subspace topology of $\mathscr{T}'$ on $Y$.

    Hence the subspace topology of $\mathscr{T}'$ on $Y$ is finer than the subspace topology of $\mathscr{T}$ on $Y$.

    The subspace topology of $\mathscr{T}'$ on $Y$ need not be strictly finer than the subspace topology of $\mathscr{T}$ on $Y$. Here is an example: $X = \{ 0, 1 \}$, $\mathscr{T} = \{ \varnothing, X \}$, $\mathscr{T}' = \{ \varnothing, X, \{ 0 \} \}$, $Y = \{ 1 \}$, then the subspace topology of $\mathscr{T}$ on $Y$ and the subspace topology of $\mathscr{T}'$ on $Y$ are $\{ \varnothing, Y \}$.
\end{proof}

% chapter2:section16:exercise3
\begin{exercise}\label{chapter2:section16:exercise3}
    Consider the set $Y = \closedinterval{-1, 1}$ as a subspace of $\mathbb{R}$. Which of the following sets are open in $Y$? Which are open in $\mathbb{R}$?
    \begin{align*}
        A & = \{ x \mid \frac{1}{2} < \abs{x} < 1 \}, \\
        B & = \{ x \mid \frac{1}{2} < \abs{x} \leq 1 \}, \\
        C & = \{ x \mid \frac{1}{2} \leq \abs{x} < 1 \}, \\
        D & = \{ x \mid \frac{1}{2} \leq \abs{x} \leq 1 \}, \\
        E & = \{ x \mid 0 < \abs{x} < 1 \land 1/x \notin \mathbb{Z}_{+} \}.
    \end{align*}
\end{exercise}

\begin{proof}
    \begin{enumerate}[label={(\alph*)}]
        \item $A = \openinterval{-1, -1/2}\cup \openinterval{1/2, 1}$. So $A$ is open in $\mathbb{R}$ and $Y$.
        \item $B = \halfopenright{-1, -1/2}\cup \halfopenleft{1/2, 1} = Y\cap (\openinterval{-3/2, -1/2}\cup \openinterval{1/2, 3/2})$. So $B$ is open in $Y$.

              $B$ is not open in $\mathbb{R}$ because there is no open set in $\mathbb{R}$ that contains $-1$ contained in $B$.
        \item $C = \halfopenleft{-1, -1/2}\cup\halfopenright{1/2, 1}$ is not open in $Y$, nor open in $\mathbb{R}$.
        \item $D = \closedinterval{-1, -1/2}\cup\closedinterval{1/2, 1}$ is not open in $Y$, nor open in $\mathbb{R}$.
        \item $E = (\openinterval{-1, 0}\cup \openinterval{0, 1}) - K = \openinterval{-1, 0} \cup (\openinterval{0, 1} - K)$, where $K = \{ 1/n \mid n\in\mathbb{Z}_{+} \}$.

              If $x\in \openinterval{-1, 0}$ then there is an open interval $\openinterval{a, b}$ such that $x\in \openinterval{a, b}\in\openinterval{-1, 0}$, since $\openinterval{-1, 0}$ is open. If $x\in \openinterval{0, 1} - K$ then there is a unique positive integer $n$ such that $n < 1/x < n+1$, which means $\frac{1}{n+1} < x < \frac{1}{n}$, hence $x\in \openinterval{\frac{1}{n+1}, \frac{1}{n}}\subset \openinterval{0, 1} - K$.

              Therefore $E$ is open in $\mathbb{R}$. Moreover, $E = Y\cap E$ so $E$ is open in $Y$.
    \end{enumerate}
\end{proof}

% chapter2:section16:exercise4
\begin{exercise}\label{chapter2:section16:exercise4}
    A map $f: X\to Y$ is said to be an \textbf{open map} if for every open set $U$ of $X$, the set $f(U)$ is open in $Y$. Show that $\pi_{1}: X\times Y\to X$ and $\pi_{2}: X\times Y\to Y$ are open maps.
\end{exercise}

\begin{proof}
    Let $U$ be an open set in $X\times Y$. Let $x\in \pi_{1}(U)$ and $y\in \pi_{2}(U)$. Then
    \begin{itemize}[itemsep=0pt]
        \item there is $b\in Y$ such that $\pi_{1}(x\times b) = x$. Because $U$ is open in $X\times Y$, there is a basis element $A_{1}\times B_{1}$ of the product topology on $X\times Y$, where $A_{1}$ is open in $X$ and $B_{1}$ is open in $Y$ such that $x\times b\in A_{1}\times B_{1}\subset U$. $\pi_{1}(A_{1}\times B_{1}) = A_{1}$, which is an open set contained in $\pi_{1}(U)$.
        \item there is $a\in X$ such that $\pi_{2}(a\times y) = y$. Because $U$ is open in $X\times Y$, there is a basis element $A_{2}\times B_{2}$ of the product topology on $X\times Y$, where $A_{2}$ is open in $X$ and $B_{2}$ is open in $Y$ such that $a\times y\in A_{2}\times B_{2}\subset U$. $\pi_{1}(A_{2}\times B_{2}) = B_{2}$, which is an open set contained in $\pi_{2}(U)$.
    \end{itemize}

    Hence $\pi_{1}(U)$ and $\pi_{2}(U)$ are open sets for every open set $U$ in $X\times Y$. Thus $\pi_{1}$ and $\pi_{2}$ are open maps.
\end{proof}

% chapter2:section16:exercise5
\begin{exercise}\label{chapter2:section16:exercise5}
    Let $X$ and $X'$ denote a single set in the topologies $\mathscr{T}$ and $\mathscr{T}'$, respectively; let $Y$ and $Y'$ denote a single set in the topologies $\mathscr{U}$ and $\mathscr{U}'$, respectively. Assume these sets are nonempty.
    \begin{enumerate}[label={(\alph*)}]
        \item Show that if $\mathscr{T}'\supset\mathscr{T}$ and $\mathscr{U}\supset\mathscr{U}'$, then the product topology on $X'\times Y'$ is finer than the product topology on $X\times Y$.
        \item Does the converse of (a) hold? Justify your answer.
    \end{enumerate}
\end{exercise}

\begin{proof}
    \begin{enumerate}[label={(\alph*)}]
        \item Let $A\times B$ be a basis element of the product topology on $X\times Y$, then $A$ is an open set in $X$ and $B$ is an open set in $Y$. Since $\mathscr{T} \subset \mathscr{T}'$ and $\mathscr{U} \subset \mathscr{U}'$ then $A$ is open in $X'$ and $B$ is open in $Y'$. Hence $A\times B$ is a basis element of the product topology on $X'\times Y'$.

              Thus the product topology on $X'\times Y'$ is finer than the product topology on $X\times Y$.
        \item Yes, if $X, Y, X', Y'$ are nonempty.

              Let $A$ be an open set in $X$. $A\times Y$ is a basis element of the product topology on $X\times Y$. Because the product topology on $X\times Y$ is coarser than the product topology on $X'\times Y'$, then $A\times Y = A\times Y'$ is also a basis element of the product topology on $X'\times Y'$. Therefore $A$ is an open set in $X'$. Thus $\mathscr{T}\subset \mathscr{T}'$.

              Let $B$ be an open set in $Y$. $X\times B$ is a basis element of the product topology $X\times Y$. Because the product topology on $X\times Y$ is coarser than the product topology on $X'\times Y'$, then $X\times B = X'\times B$ is also a basis element of the product topology on $X'\times Y'$. Therefore $B$ is an open set in $Y'$. Thus $\mathscr{U}\subset\mathscr{U}'$.
    \end{enumerate}
\end{proof}

% chapter2:section16:exercise6
\begin{exercise}\label{chapter2:section16:exercise6}
    Show that the countable collection
    \[
        \{ \openinterval{a, b}\times \openinterval{c, d} \mid \text{$a < b$ and $c < d$, and $a, b, c, d$ are rational} \}
    \]

    is a basis for $\mathbb{R}^{2}$.
\end{exercise}

\begin{proof}
    Because the collection of open intervals $\openinterval{a, b}$ where $a, b$ are rational is a basis for the standard topology on $\mathbb{R}$, so the given set is a basis for $\mathbb{R}^{2}$.
\end{proof}

% chapter2:section16:exercise7
\begin{exercise}\label{chapter2:section16:exercise7}
    Let $X$ be an ordered set. If $Y$ is a proper subset of $X$ that is convex in $X$, does it follow that $Y$ is an interval or a ray in $X$?
\end{exercise}

\begin{proof}
    No.

    For example: $X = \mathbb{Q}$, $Y = \openinterval{-\sqrt{2}, \sqrt{2}}\cap \mathbb{Q}$. $Y$ is not a ray. Assume there are $a, b\in\mathbb{Q}$ such that $Y$ is an interval whose endpoints are $a, b$, then $a$ is the smallest rational number larger than $-\sqrt{2}$ and $b$ is the largest rational number smaller than $\sqrt{2}$, which is a contradiction because there is no smallest rational number larger than $-\sqrt{2}$ and there is no largest rational number smaller than $\sqrt{2}$. Hence $Y$ is not an interval.
\end{proof}

% chapter2:section16:exercise8
\begin{exercise}\label{chapter2:section16:exercise8}
    If $L$ is a straight line in the plane, describe the topology $L$ inherits as a subspace of $\mathbb{R}_{\ell}\times\mathbb{R}$ and as a subspace of $\mathbb{R}_{\ell}\times\mathbb{R}_{\ell}$. In each case it is a familiar topology.
\end{exercise}

\begin{proof}
\end{proof}

% chapter2:section16:exercise9
\begin{exercise}\label{chapter2:section16:exercise9}
    Show that the dictionary order topology on the set $\mathbb{R}\times\mathbb{R}$ is the same as the product topology $\mathbb{R}_{d}\times\mathbb{R}$, where $\mathbb{R}_{d}$ denotes $\mathbb{R}$ in the discrete topology. Compare this topology with the standard topology on $\mathbb{R}^{2}$.
\end{exercise}

\begin{proof}
\end{proof}

% chapter2:section16:exercise10
\begin{exercise}\label{chapter2:section16:exercise10}
    Let $I = \closedinterval{0, 1}$. Compare the product topology $I\times I$, the dictionary order topology on $I\times I$, and the topology $I\times I$ inherits as a subspace of $\mathbb{R}\times\mathbb{R}$ in the dictionary order topology.
\end{exercise}

\begin{proof}
\end{proof}

\section{Closed Sets and Limit Points}

% chapter2:section17:exercise1
\begin{exercise}\label{chapter2:section17:exercise1}
\end{exercise}

\begin{proof}
\end{proof}

% chapter2:section17:exercise2
\begin{exercise}\label{chapter2:section17:exercise2}
\end{exercise}

\begin{proof}
\end{proof}

% chapter2:section17:exercise3
\begin{exercise}\label{chapter2:section17:exercise3}
\end{exercise}

\begin{proof}
\end{proof}

% chapter2:section17:exercise4
\begin{exercise}\label{chapter2:section17:exercise4}
\end{exercise}

\begin{proof}
\end{proof}

% chapter2:section17:exercise5
\begin{exercise}\label{chapter2:section17:exercise5}
\end{exercise}

\begin{proof}
\end{proof}

% chapter2:section17:exercise6
\begin{exercise}\label{chapter2:section17:exercise6}
\end{exercise}

\begin{proof}
\end{proof}

% chapter2:section17:exercise7
\begin{exercise}\label{chapter2:section17:exercise7}
\end{exercise}

\begin{proof}
\end{proof}

% chapter2:section17:exercise8
\begin{exercise}\label{chapter2:section17:exercise8}
\end{exercise}

\begin{proof}
\end{proof}

% chapter2:section17:exercise9
\begin{exercise}\label{chapter2:section17:exercise9}
\end{exercise}

\begin{proof}
\end{proof}

% chapter2:section17:exercise10
\begin{exercise}\label{chapter2:section17:exercise10}
\end{exercise}

\begin{proof}
\end{proof}

% chapter2:section17:exercise11
\begin{exercise}\label{chapter2:section17:exercise11}
\end{exercise}

\begin{proof}
\end{proof}

% chapter2:section17:exercise12
\begin{exercise}\label{chapter2:section17:exercise12}
\end{exercise}

\begin{proof}
\end{proof}

% chapter2:section17:exercise13
\begin{exercise}\label{chapter2:section17:exercise13}
\end{exercise}

\begin{proof}
\end{proof}

% chapter2:section17:exercise14
\begin{exercise}\label{chapter2:section17:exercise14}
\end{exercise}

\begin{proof}
\end{proof}

% chapter2:section17:exercise15
\begin{exercise}\label{chapter2:section17:exercise15}
\end{exercise}

\begin{proof}
\end{proof}

% chapter2:section17:exercise16
\begin{exercise}\label{chapter2:section17:exercise16}
\end{exercise}

\begin{proof}
\end{proof}

% chapter2:section17:exercise17
\begin{exercise}\label{chapter2:section17:exercise17}
\end{exercise}

\begin{proof}
\end{proof}

% chapter2:section17:exercise18
\begin{exercise}\label{chapter2:section17:exercise18}
\end{exercise}

\begin{proof}
\end{proof}


% chapter2:section17:exercise19
\begin{exercise}\label{chapter2:section17:exercise19}
\end{exercise}

\begin{proof}
\end{proof}

% chapter2:section17:exercise20
\begin{exercise}\label{chapter2:section17:exercise20}
\end{exercise}

\begin{proof}
\end{proof}

% chapter2:section17:exercise21
\begin{exercise}\label{chapter2:section17:exercise21}
\end{exercise}

\begin{proof}
\end{proof}

\section{Continuous Functions}

% chapter2:section18:exercise1
\begin{exercise}\label{chapter2:section18:exercise1}
\end{exercise}

\begin{proof}
\end{proof}


% chapter2:section18:exercise2
\begin{exercise}\label{chapter2:section18:exercise2}
\end{exercise}

\begin{proof}
\end{proof}


% chapter2:section18:exercise3
\begin{exercise}\label{chapter2:section18:exercise3}
\end{exercise}

\begin{proof}
\end{proof}


% chapter2:section18:exercise4
\begin{exercise}\label{chapter2:section18:exercise4}
\end{exercise}

\begin{proof}
\end{proof}


% chapter2:section18:exercise5
\begin{exercise}\label{chapter2:section18:exercise5}
\end{exercise}

\begin{proof}
\end{proof}


% chapter2:section18:exercise6
\begin{exercise}\label{chapter2:section18:exercise6}
\end{exercise}

\begin{proof}
\end{proof}


% chapter2:section18:exercise7
\begin{exercise}\label{chapter2:section18:exercise7}
\end{exercise}

\begin{proof}
\end{proof}


% chapter2:section18:exercise8
\begin{exercise}\label{chapter2:section18:exercise8}
\end{exercise}

\begin{proof}
\end{proof}


% chapter2:section18:exercise9
\begin{exercise}\label{chapter2:section18:exercise9}
\end{exercise}

\begin{proof}
\end{proof}


% chapter2:section18:exercise10
\begin{exercise}\label{chapter2:section18:exercise10}
\end{exercise}

\begin{proof}
\end{proof}


% chapter2:section18:exercise11
\begin{exercise}\label{chapter2:section18:exercise11}
\end{exercise}

\begin{proof}
\end{proof}


% chapter2:section18:exercise12
\begin{exercise}\label{chapter2:section18:exercise12}
\end{exercise}

\begin{proof}
\end{proof}


% chapter2:section18:exercise13
\begin{exercise}\label{chapter2:section18:exercise13}
\end{exercise}

\begin{proof}
\end{proof}


\section{The Product Topology}

% chapter2:section19:exercise1
\begin{exercise}\label{chapter2:section19:exercise1}
\end{exercise}

\begin{proof}
\end{proof}


% chapter2:section19:exercise2
\begin{exercise}\label{chapter2:section19:exercise2}
\end{exercise}

\begin{proof}
\end{proof}


% chapter2:section19:exercise3
\begin{exercise}\label{chapter2:section19:exercise3}
\end{exercise}

\begin{proof}
\end{proof}


% chapter2:section19:exercise4
\begin{exercise}\label{chapter2:section19:exercise4}
\end{exercise}

\begin{proof}
\end{proof}


% chapter2:section19:exercise5
\begin{exercise}\label{chapter2:section19:exercise5}
\end{exercise}

\begin{proof}
\end{proof}


% chapter2:section19:exercise6
\begin{exercise}\label{chapter2:section19:exercise6}
\end{exercise}

\begin{proof}
\end{proof}


% chapter2:section19:exercise7
\begin{exercise}\label{chapter2:section19:exercise7}
\end{exercise}

\begin{proof}
\end{proof}


% chapter2:section19:exercise8
\begin{exercise}\label{chapter2:section19:exercise8}
\end{exercise}

\begin{proof}
\end{proof}


% chapter2:section19:exercise9
\begin{exercise}\label{chapter2:section19:exercise9}
\end{exercise}

\begin{proof}
\end{proof}


% chapter2:section19:exercise10
\begin{exercise}\label{chapter2:section19:exercise10}
\end{exercise}

\begin{proof}
\end{proof}


\section{The Metric Topology}

% chapter2:section20:exercise1
\begin{exercise}\label{chapter2:section20:exercise1}
\end{exercise}

\begin{proof}
\end{proof}


% chapter2:section20:exercise2
\begin{exercise}\label{chapter2:section20:exercise2}
\end{exercise}

\begin{proof}
\end{proof}


% chapter2:section20:exercise3
\begin{exercise}\label{chapter2:section20:exercise3}
\end{exercise}

\begin{proof}
\end{proof}


% chapter2:section20:exercise4
\begin{exercise}\label{chapter2:section20:exercise4}
\end{exercise}

\begin{proof}
\end{proof}


% chapter2:section20:exercise5
\begin{exercise}\label{chapter2:section20:exercise5}
\end{exercise}

\begin{proof}
\end{proof}


% chapter2:section20:exercise6
\begin{exercise}\label{chapter2:section20:exercise6}
\end{exercise}

\begin{proof}
\end{proof}


% chapter2:section20:exercise7
\begin{exercise}\label{chapter2:section20:exercise7}
\end{exercise}

\begin{proof}
\end{proof}


% chapter2:section20:exercise8
\begin{exercise}\label{chapter2:section20:exercise8}
\end{exercise}

\begin{proof}
\end{proof}


% chapter2:section20:exercise9
\begin{exercise}\label{chapter2:section20:exercise9}
\end{exercise}

\begin{proof}
\end{proof}


% chapter2:section20:exercise10
\begin{exercise}\label{chapter2:section20:exercise10}
\end{exercise}

\begin{proof}
\end{proof}


% chapter2:section20:exercise11
\begin{exercise}\label{chapter2:section20:exercise11}
\end{exercise}

\begin{proof}
\end{proof}


\section{The Metric Topology (continued)}

% chapter2:section21:exercise1
\begin{exercise}\label{chapter2:section21:exercise1}
\end{exercise}

\begin{proof}
\end{proof}


% chapter2:section21:exercise2
\begin{exercise}\label{chapter2:section21:exercise2}
\end{exercise}

\begin{proof}
\end{proof}


% chapter2:section21:exercise3
\begin{exercise}\label{chapter2:section21:exercise3}
\end{exercise}

\begin{proof}
\end{proof}


% chapter2:section21:exercise4
\begin{exercise}\label{chapter2:section21:exercise4}
\end{exercise}

\begin{proof}
\end{proof}


% chapter2:section21:exercise5
\begin{exercise}\label{chapter2:section21:exercise5}
\end{exercise}

\begin{proof}
\end{proof}


% chapter2:section21:exercise6
\begin{exercise}\label{chapter2:section21:exercise6}
\end{exercise}

\begin{proof}
\end{proof}


% chapter2:section21:exercise7
\begin{exercise}\label{chapter2:section21:exercise7}
\end{exercise}

\begin{proof}
\end{proof}


% chapter2:section21:exercise8
\begin{exercise}\label{chapter2:section21:exercise8}
\end{exercise}

\begin{proof}
\end{proof}


% chapter2:section21:exercise9
\begin{exercise}\label{chapter2:section21:exercise9}
\end{exercise}

\begin{proof}
\end{proof}


% chapter2:section21:exercise10
\begin{exercise}\label{chapter2:section21:exercise10}
\end{exercise}

\begin{proof}
\end{proof}


% chapter2:section21:exercise11
\begin{exercise}\label{chapter2:section21:exercise11}
\end{exercise}

\begin{proof}
\end{proof}


% chapter2:section21:exercise12
\begin{exercise}\label{chapter2:section21:exercise12}
\end{exercise}

\begin{proof}
\end{proof}


\section{The Quotient Topology}

% chapter2:section22:exercise1
\begin{exercise}\label{chapter2:section22:exercise1}
\end{exercise}

\begin{proof}
\end{proof}


% chapter2:section22:exercise2
\begin{exercise}\label{chapter2:section22:exercise2}
\end{exercise}

\begin{proof}
\end{proof}


% chapter2:section22:exercise3
\begin{exercise}\label{chapter2:section22:exercise3}
\end{exercise}

\begin{proof}
\end{proof}


% chapter2:section22:exercise4
\begin{exercise}\label{chapter2:section22:exercise4}
\end{exercise}

\begin{proof}
\end{proof}


% chapter2:section22:exercise5
\begin{exercise}\label{chapter2:section22:exercise5}
\end{exercise}

\begin{proof}
\end{proof}


% chapter2:section22:exercise6
\begin{exercise}\label{chapter2:section22:exercise6}
\end{exercise}

\begin{proof}
\end{proof}


\begin{section*}{Supplementary Exercises: Topological Groups}

    % chapter2:sectionX:exercise1
    \begin{exercise}\label{chapter2:sectionX:exercise1}
    \end{exercise}

    \begin{proof}
    \end{proof}


    % chapter2:sectionX:exercise2
    \begin{exercise}\label{chapter2:sectionX:exercise2}
    \end{exercise}

    \begin{proof}
    \end{proof}


    % chapter2:sectionX:exercise3
    \begin{exercise}\label{chapter2:sectionX:exercise3}
    \end{exercise}

    \begin{proof}
    \end{proof}


    % chapter2:sectionX:exercise4
    \begin{exercise}\label{chapter2:sectionX:exercise4}
    \end{exercise}

    \begin{proof}
    \end{proof}


    % chapter2:sectionX:exercise5
    \begin{exercise}\label{chapter2:sectionX:exercise5}
    \end{exercise}

    \begin{proof}
    \end{proof}


    % chapter2:sectionX:exercise6
    \begin{exercise}\label{chapter2:sectionX:exercise6}
    \end{exercise}

    \begin{proof}
    \end{proof}


    % chapter2:sectionX:exercise7
    \begin{exercise}\label{chapter2:sectionX:exercise7}
    \end{exercise}

    \begin{proof}
    \end{proof}

\end{section*}
