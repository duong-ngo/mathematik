% chktex-file 44
\chapter{Structure of Groups}

\section{Groups of Permutations}

\subsection*{Computations}

In Exercises 1 through 10 determine whether the given map is a group homomorphism.

% section 1/exercise 1
\begin{exercise}
    Let $\phi: \mathbb{Z}_{10} \to \mathbb{Z}_{2}$ be given by $\phi(x) = $ the remainder when $x$ is divided by $2$.
\end{exercise}

\begin{proof}
    $\phi(x + y) = (x + y) \mod 2 = x \mod 2 + y \mod 2 = \phi(x) + \phi(y)$. Hence $\phi$ is a group homomorphism.
\end{proof}

% section 1/exercise 2
\begin{exercise}
    Let $\phi: \mathbb{Z}_{9} \to \mathbb{Z}_{2}$ be given by $\phi(x) =$ the remainder when $x$ is divided by $2$.
\end{exercise}

\begin{proof}
    $\phi(4 + 5) = \phi(0) = 0\mod 2 = 0$, $\phi(4) + \phi(5) = 4\mod 2 + 5\mod 2 = 0 + 1 = 1$. So $\phi(4 + 5) \ne \phi(4) + \phi(5)$. Hence $\phi$ is not a group homomorphism.
\end{proof}

% section 1/exercise 3
\begin{exercise}
    Let $\phi: \mathbb{Q}^{*} \to \mathbb{Q}^{*}$ be given by $\phi(x) = \abs{x}$.
\end{exercise}

\begin{proof}
    $\phi(x\cdot y) = \abs{x\cdot y} = \abs{x}\cdot\abs{y} = \phi(x)\cdot\phi(y)$. Hence $\phi$ is a group homomorphism.
\end{proof}

% section 1/exercise 4
\begin{exercise}
    Let $\phi: \mathbb{R} \to \mathbb{R}^{+}$ be given by $\phi(x) = 2^{x}$.
\end{exercise}

\begin{proof}
    $\phi(x + y) = {2}^{x+y} = {2}^{x}\cdot {2}^{y} = \phi(x)\cdot\phi(y)$. Hence $\phi$ is a group homomorphism.
\end{proof}

% section 1/exercise 5
\begin{exercise}
    Let $\phi: D_{4} \to \mathbb{Z}_{4}$ be given by $\phi(\rho^{i}) = \phi(\mu\rho^{i}) = i$ for $0\leq i\leq 3$.
\end{exercise}

\begin{proof}
    $\phi(\mu\rho\cdot\mu\rho) = \phi(\iota) = \phi(\rho^{0}) = 0$. $\phi(\mu\rho) + \phi(\mu\rho) = 1 + 1 = 2$. So $\phi(\mu\rho\cdot\mu\rho) \ne \phi(\mu\rho) + \phi(\mu\rho)$. Hence $\phi$ is not a group homomorphism.
\end{proof}

% section 1/exercise 6
\begin{exercise}
    Let $F$ be the additive group of all functions mapping $\mathbb{R}$ to $\mathbb{R}$. Let $\phi: F \to F$ be given by $\phi(f) = g$ where $g(x) = f(x) + x$.
\end{exercise}

\begin{proof}
    $\phi(f + g)(x) = (f + g)(x) + x = f(x) + g(x) + x$. $\phi(f)(x) + \phi(g)(x) = f(x) + g(x) + 2x$. So $\phi(f + g)(x)$ is not equal to $\phi(f)(x) + \phi(g)(x)$ for all real numbers $x$. Hence $\phi$ is not a group homomorphism.
\end{proof}

% section 1/exercise 7
\begin{exercise}
    Let $F$ be as in Exercise 6 and $\phi: F \to F$ be defined by $\phi(f) = 5f$.
\end{exercise}

\begin{proof}
    $\phi(f + g)(x) = (5(f + g))(x) = 5f(x) + 5g(x) = (\phi(f) + \phi(g))(x)$. Hence $\phi$ is a group homomorphism.
\end{proof}

% section 1/exercise 8
\begin{exercise}
    Let $F$ be the additive group of all continuous functions mapping $\mathbb{R}$ to $\mathbb{R}$. Let $\phi: F\to \mathbb{R}$ be defined by $\phi(g) = \int^{1}_{0}g(x)dx$.
\end{exercise}

\begin{proof}
    \begin{align*}
        \phi(f + g) & = \int^{1}_{0}(f + g)(x)dx                \\
                    & = \int^{1}_{0}(f(x) + g(x))dx             \\
                    & = \int^{1}_{0}f(x)dx + \int^{1}_{0}g(x)dx \\
                    & = \phi(f) + \phi(g)
    \end{align*}

    Hence $\phi$ is a group homomorphism.
\end{proof}

% section 1/exercise 9
\begin{exercise}
    Let $M_{n}$ be the additive group of $n\times n$ matrices with real entries. Let $\phi: M_{n}\to \mathbb{R}$ be given by $\phi(A) = \det(A)$, the determinant of $A$.
\end{exercise}

\begin{proof}
    $\phi(I_{n} + I_{n}) = \det(2I_{n}) = 2^{n}$. $\phi(I_{n}) + \phi(I_{n}) = \det(I_{n}) + \det(I_{n}) = 2$. So when $n > 1$, $\phi(I_{n} + I_{n}) \ne \phi(I_{n}) + \phi(I_{n})$. Hence $\phi$ is not a group homomorphism.
\end{proof}

% section 1/exercise 10
\begin{exercise}
    Let $M_{n}$ be as in Exercise 9 and $\phi: M_{n} \to \mathbb{R}$ be defined by $\phi(A) = \tr{A}$ where $\tr{A}$ is the trace of $A$, which is the sum of the entries on the diagonal.
\end{exercise}

\begin{proof}
    Let $A = {(a_{i.j})}_{n\times n}$ and $B = {(b_{i.j})}_{n\times n}$.
    \[
        \phi(A + B) = \tr{A + B} = \sum^{n}_{i=1}(a_{i.i} + b_{i.i}) = \sum^{n}_{i=1}a_{i.i} + \sum^{n}_{i=1}b_{i.i} = \tr{A} + \tr{B}
    \]

    Hence $\phi$ is a group homomorphism.
\end{proof}

In Exercises 11 through 16, compute the kernel for the given homomorphism $\phi$.

% section 1/exercise 11
\begin{exercise}
    $\phi: \mathbb{Z} \to \mathbb{Z}_{8}$ such that $\phi(1) = 6$.
\end{exercise}

\begin{proof}
    $\phi(4) = \phi(1 + 1 + 1 + 1) = (6 + 6 + 6 + 6)\mod 8 = 0$, $\phi(3) = 2$, $\phi(2) = 4$, $\phi(1) = 6$, $\phi(0) = 0$.

    If $n$ leaves remainer $r > 0$ when divided by $4$, then $\phi(n) = \phi(r) \ne 0$. Otherwise, $n$ is divisible by $4$, then $\phi(n) = 4$.

    Hence $\ker\phi = 4\mathbb{Z}$.
\end{proof}

% section 1/exercise 12
\begin{exercise}
    $\phi: \mathbb{Z} \to \mathbb{Z}$ such that $\phi(1) = 12$.
\end{exercise}

\begin{proof}
    For $n\in\mathbb{Z}$, $\phi(n) = 12n$. So $\phi(n) = 0$ iff $n = 0$. Hence $\ker\phi = \{ 0 \}$.
\end{proof}

% section 1/exercise 13
\begin{exercise}
    $\phi: \mathbb{Z}\times\mathbb{Z} \to \mathbb{Z}$ where $\phi(1,0) = 3$ and $\phi(0,1) = -5$.
\end{exercise}

\begin{proof}
    $\phi(m, n) = 3m - 5n$. $\phi(m, n) = 0$ iff $3m = 5n$. $3m = 5n$ iff $m = 5k, n = 3k$ for some integer $k$.

    Hence $\ker\phi = \{ (5k, 3k) \mid k\in\mathbb{Z} \}$.
\end{proof}

% section 1/exercise 14
\begin{exercise}
    $\phi: \mathbb{Z}\times\mathbb{Z} \to \mathbb{Z}$ where $\phi(1,0) = 6$ and $\phi(0,1) = 9$.
\end{exercise}

\begin{proof}
    $\phi(m, n) = 6m + 9n$. $\phi(m, n) = 0$ iff $6m + 9n = 0$, equivalently, $2m + 3n = 0$. $2m + 3n = 0$ iff $m = 3k, n = -2k$ for some integer $k$.

    Hence $\ker\phi = \{ (3k, -2k) \mid k\in\mathbb{Z} \}$.
\end{proof}

% section 1/exercise 15
\begin{exercise}
    $\phi: \mathbb{Z}\times\mathbb{Z} \to \mathbb{Z}\times\mathbb{Z}$ where $\phi(1,0) = (2,5)$ and $\phi(0,1) = (-3,2)$.
\end{exercise}

\begin{proof}
    $\phi(m, n) = (0, 0)$ iff $(2m - 3n, 5m + 2n) = (0, 0)$. From the system of linear equations $2m - 3n = 5m + 2n = 0$, we solve for $m, n$ and obtain $m = n = 0$.

    Hence $\ker\phi = \{ (0,0) \}$.
\end{proof}

% section 1/exercise 16
\begin{exercise}
    Let $D$ be the additive group of all differentiable functions mapping $\mathbb{R}$ to $\mathbb{R}$ and $F$ the additive group of all functions from $\mathbb{R}$ to $\mathbb{R}$. $\phi: D \to F$ is given by $\phi(f) = f'$, the derivative of $f$.
\end{exercise}

\begin{proof}
    Derivative of a function is the zero function if and only if the function is a constant function.

    Hence $\ker\phi =$ the set of constant functions mapping $\mathbb{R}$ to $\mathbb{R}$.
\end{proof}

In Exercises 17 through 22, find all orbits of the given permutation.

% section 1/exercise 17
\begin{exercise}
    $\begin{pmatrix}
            1 & 2 & 3 & 4 & 5 & 6 \\
            5 & 1 & 3 & 6 & 2 & 4
        \end{pmatrix}$
\end{exercise}

\begin{proof}
    The orbits of the given permutation are $\{ 1, 5, 2 \}, \{ 3 \}, \{ 4, 6 \}$.
\end{proof}

% section 1/exercise 18
\begin{exercise}
    $\begin{pmatrix}
            1 & 2 & 3 & 4 & 5 & 6 & 7 & 8 \\
            5 & 6 & 2 & 4 & 8 & 3 & 1 & 7
        \end{pmatrix}$
\end{exercise}

\begin{proof}
    The orbits of the given permutation are $\{ 1, 5, 8, 7 \}, \{ 2, 6, 3 \}, \{ 4 \}$.
\end{proof}

% section 1/exercise 19
\begin{exercise}
    $\begin{pmatrix}
            1 & 2 & 3 & 4 & 5 & 6 & 7 & 8 \\
            2 & 3 & 5 & 1 & 4 & 6 & 8 & 7
        \end{pmatrix}$
\end{exercise}

\begin{proof}
    The orbits of the given permutation are $\{ 1, 2, 3, 5, 4 \}, \{ 6 \}, \{ 7, 8 \}$.
\end{proof}

% section 1/exercise 20
\begin{exercise}
    $\sigma: \mathbb{Z} \to \mathbb{Z}$ where $\sigma(n) = n + 1$
\end{exercise}

\begin{proof}
    $\sigma$ has one orbit, which is $\mathbb{Z}$.
\end{proof}

% section 1/exercise 21
\begin{exercise}
    $\sigma: \mathbb{Z} \to \mathbb{Z}$ where $\sigma(n) = n + 2$
\end{exercise}

\begin{proof}
    The orbits of $\sigma$ are $\{ 2n \mid n\in\mathbb{Z} \}$, $\{ 2n + 1 \mid n\in\mathbb{Z} \}$.
\end{proof}

% section 1/exercise 22
\begin{exercise}
    $\sigma: \mathbb{Z} \to \mathbb{Z}$ where $\sigma(n) = n - 3$
\end{exercise}

\begin{proof}
    The orbits of $\sigma$ are $\{ 3n \mid n\in\mathbb{Z} \}$, $\{ 3n + 1 \mid n\in\mathbb{Z} \}$, $\{ 3n + 2 \mid n\in\mathbb{Z} \}$.
\end{proof}

In Exercises 23 through 25, express the permutation of $\{ 1, 2, 3, 4, 5, 6, 7, 8 \}$ as a product of disjoint cycles, and then as a product of transpositions.

% section 1/exercise 23
\begin{exercise}
    $\begin{pmatrix}
            1 & 2 & 3 & 4 & 5 & 6 & 7 & 8 \\
            8 & 2 & 6 & 3 & 7 & 4 & 5 & 1
        \end{pmatrix}$
\end{exercise}

\begin{proof}
    As product of disjoint cycles: $(1, 8)(3, 6, 4)(5, 7)$.

    As product of transpositions: $(1, 8)(3, 4)(3, 6)(5, 7)$.
\end{proof}

% section 1/exercise 24
\begin{exercise}
    $\begin{pmatrix}
            1 & 2 & 3 & 4 & 5 & 6 & 7 & 8 \\
            3 & 6 & 4 & 1 & 8 & 2 & 5 & 7
        \end{pmatrix}$
\end{exercise}

\begin{proof}
    As product of disjoint cycles: $(1, 3, 4)(2, 6)(5, 8, 7)$.

    As product of transpositions: $(1, 4)(1, 3)(2, 6)(5, 7)(5, 8)$.
\end{proof}

% section 1/exercise 25
\begin{exercise}
    $\begin{pmatrix}
            1 & 2 & 3 & 4 & 5 & 6 & 7 & 8 \\
            5 & 3 & 2 & 8 & 4 & 7 & 6 & 1
        \end{pmatrix}$
\end{exercise}

\begin{proof}
    As product of disjoint cycles: $(1, 5, 4, 8)(2, 3)(6, 7)$.

    As product of transpositions: $(1, 8)(1, 4)(1, 5)(2, 3)(6, 7)$.
\end{proof}

% section 1/exercise 26
\begin{exercise}
    Figure 8.26 shows a Cayley digraph for the alternating group $A_{4}$ using the generating set $S = \{ (1, 2, 3), (1, 2)(3, 4) \}$. Continue labeling the other nine vertices with the elements of $A_{4}$, expressed as a product of disjoint cycles.
\end{exercise}

\begin{proof}
    The Cayley digraph for the alternating group $A_{4}$. \\

    \begin{tikzpicture}[>=Stealth]
        \tikzset{arc type 1/.style={black,postaction={decorate,decoration={markings,mark=at position 0.5 with {\arrow{>}}}}}}

        \coordinate (center1) at (0, 1);
        \coordinate (center2) at ({sqrt(3)/2}, {-1/2});
        \coordinate (center3) at ({-sqrt(3)/2}, {-1/2});
        % (0, 1.5)
        \coordinate (upper1) at (0, 2);
        \coordinate (upper2) at ({-sqrt(3)/2}, {3+1/2});
        \coordinate (upper3) at ({sqrt(3)/2}, {3+1/2});
        % ({0.75*sqrt(3)}, -0.75)
        \coordinate (right1) at ({1.5*sqrt(3)}, -2.5);
        \coordinate (right2) at ({sqrt(3)}, -1);
        \coordinate (right3) at ({2*sqrt(3)}, -1);
        % (-0.75*sqrt(3), -0.75)
        \coordinate (left1) at ({-1.5*sqrt(3)}, -2.5);
        \coordinate (left2) at ({-2*sqrt(3)}, -1);
        \coordinate (left3) at ({-sqrt(3)}, -1);

        \node[label={[font=\small]above:$(1)$}] at (upper2) {};
        \node[label={[font=\small]above:$(1,2,3)$}] at (upper3) {};
        \node[label={[font=\small]right:$(1,3,2)$}] at (upper1) {};

        \node[label={[font=\small]left:$(1,2)(3,4)$}] at (left2) {};
        \node[label={[font=\small]right:$(2,4,3)$}] at (left3) {};
        \node[label={[font=\small]below:$(1,4,3)$}] at (left1) {};

        \node[label={[font=\small]left:$(1,4)(2,3)$}] at (right2) {};
        \node[label={[font=\small]below:$(1,2,4)$}] at (right1) {};
        \node[label={[font=\small]right:$(1,3,4)$}] at (right3) {};

        \node[label={[font=\small]right:$(2,3,4)$}] at (center1) {};
        \node[label={[font=\small]right:$(1,3)(2,4)$}] at (center2) {};
        \node[label={[font=\small]left:$(1,4,2)$}] at (center3) {};

        \draw[dashed] (center1) -- (upper1);
        \draw[dashed] (center2) -- (right2);
        \draw[dashed] (center3) -- (left3);
        \draw[dashed] (upper3) -- (right3);
        \draw[dashed] (left2) -- (upper2);
        \draw[dashed] (right1) -- (left1);

        \draw[arc type 1] (center1) -- (center2);
        \draw[arc type 1] (center2) -- (center3);
        \draw[arc type 1] (center3) -- (center1);
        \draw[arc type 1] (upper1) -- (upper2);
        \draw[arc type 1] (upper2) -- (upper3);
        \draw[arc type 1] (upper3) -- (upper1);
        \draw[arc type 1] (left1) -- (left2);
        \draw[arc type 1] (left2) -- (left3);
        \draw[arc type 1] (left3) -- (left1);
        \draw[arc type 1] (right1) -- (right2);
        \draw[arc type 1] (right2) -- (right3);
        \draw[arc type 1] (right3) -- (right1);
    \end{tikzpicture}
\end{proof}

\subsection*{Concepts}

In Exercises 27 through 29, correct the definition of the italicized term without reference to the text, if correction is needed, so that it is in a form acceptable for publication.

% section 1/exercise 27
\begin{exercise}
    For a permutation $\sigma$ of a set $A$, an \textit{orbit} of $\sigma$ is a nonempty minimal subset of $A$ that is mapped onto itself by $\sigma$.
\end{exercise}

\begin{proof}
    Correct.
\end{proof}

% section 1/exercise 28
\begin{exercise}
    The left regular representation of a group $G$ is the map of $G$ into $S_{G}$ whose value at $g\in G$ is the permutation of $G$ that carries each $x\in G$ into $gx$.
\end{exercise}

\begin{proof}
    Correct.
\end{proof}

% section 1/exercise 29
\begin{exercise}
    The \textit{alternating group} is the group of all even permutations.
\end{exercise}

\begin{proof}
    Correction: The alternating group $A_{n}$ on $\{ 1, 2,\ldots, n \}$ is the subgroup of $S_{n}$ consisting of all even permutations of $\{ 1, 2,\ldots, n \}$.
\end{proof}

% section 1/exercise 30
\begin{exercise}
    Before the proof of Cayley's Theorem, it is shown that $\lambda_{x}$ is one-to-one. In the proof, one-to-one is shown again. Is it necessary to show one-to-one twice? Explain.
\end{exercise}

\begin{proof}
    Yes, it is necessary to show one-to-one twice.

    Because the 1st one-to-one show that each $\lambda_{x}$ is a permutation of the elements of $G$, meanwhile, the 2nd one-to-one mapping is the mapping from $G$ to $S_{G}$ where $\lambda_{x}\in S_{G}$. In other words, the two one-to-one mappings are different in their domain and codomain sets.
\end{proof}

% section 1/exercise 31
\begin{exercise}
    Determine whether each of the following is true or false.
    \begin{enumerate}[label={\textbf{\alph*.}}]
        \item Every permutation is a cycle.
        \item Every cycle is a permutation.
        \item The definition of even and odd permutations could have been given equally well before Theorem 8.19.
        \item Every nontrivial subgroup $H$ of $S_{9}$ containing some odd permutation contains a transposition.
        \item $A_{5}$ has 120 elements.
        \item $S_{n}$ is not cyclic for any $n\geq 1$.
        \item $A_{3}$ is a commutative group.
        \item $S_{7}$ is isomorphic to the subgroup of all those elements of $S_{8}$ that leave the number $8$ fixed.
        \item $S_{7}$ is isomorphic to the subgroup of all those elements of $S_{8}$ that leave the number $5$ fixed.
        \item The odd permutations in $S_{8}$ form a subgroup of $S_{8}$.
        \item Every group $G$ is isomorphic with a subgroup of $S_{G}$.
    \end{enumerate}
\end{exercise}

\begin{proof}
    \begin{enumerate}[label={\textbf{\alph*.}}]
        \item False.
        \item True.
        \item False.
        \item False. For example: $H = \{ (1), (1,2)(3,4) \}$.
        \item False.
        \item False. When $n = 1$ or $2$, $S_{n}$ is cyclic.
        \item True.
        \item True.
        \item True.
        \item False.
        \item True.
    \end{enumerate}
\end{proof}

% section 1/exercise 32
\begin{exercise}
    The dihedral group is defined to be permutations with certain properties. Use the usual notation involving $\mu$ and $\rho$ for elements in $D_{n}$.
    \begin{enumerate}[label={\textbf{\alph*.}}]
        \item Identify which elements in $D_{3}$ are even. Do the even elements form a cyclic group?
        \item Identify which of elements of $D_{4}$ are even. Do the even elements form a cyclic group?
        \item For which values of $n$ do the even permutations of $D_{n}$ form a cyclic group?
    \end{enumerate}
\end{exercise}

\begin{proof}
    \begin{enumerate}[label={\textbf{\alph*.}}]
        \item Even elements in $D_{3}$ are $\rho^{0}, \rho, \rho^{2}$.
        \item Even elements in $D_{4}$ are $\rho^{0}, \rho^{2}, \mu\rho, \mu\rho^{3}$
        \item \textbf{Case 1.} $n$ is odd.

              $\rho = (0,1,\ldots,n-1) = (0,1)(1,2)\cdots(n-2,n-1)$ is a product of an even number of transpositions, so $\rho$ is even. Therefore, $\iota, \rho, \rho^{2}, \ldots, \rho^{n-1}$ are even.

              $\mu = (1,n-1)(2,n-2)\cdots (\frac{n-1}{2},\frac{n+1}{2})$. If $n\equiv 1\pmod{4}$, then $\mu$ is even, $\mu\rho^{k}$ is even for every integer $k$. Otherwise, $n\equiv 3\pmod{3}$, then $\mu$ is odd, $\mu\rho^{k}$ is odd for every integer $k$.

              Hence if $n$ is odd, the even permutations of $D_{n}$ form a cyclic group if and only if $n\equiv 3\pmod{4}$.

              \textbf{Case 2.} $n$ is even.

              $\rho = (0,1\ldots,n-1) = (0,1)(1,2)\cdots (n-2)(n-1)$ is a product of an odd number of transpositions, so $\rho$ is odd. Therefore, $\rho^{k}$ is even if and only if $k$ is even.

              $\mu = (1,n-1)(2,n-2)\cdots(\frac{n-2}{2},\frac{n+2}{2})$. If $n\equiv 0\pmod{4}$, $\mu$ is odd. Otherwise, $n\equiv 2\pmod{4}$, $\mu$ is even.

              If $n\equiv 0\pmod{4}$, then $\rho^{2k}, \mu\rho^{2k+1}$ are even permutations, which does not form a cyclic group. If $n\equiv 2\pmod{4}$, then $\rho^{2k}, \mu\rho\rho^{2k}$ are even permutations, which does not form a cyclic group. In both cases, the even permutations don't form a cyclic group because either $\rho^{2k}$ or $\mu\rho^{k}$ cannot generate the permutation of the other type (the two types are $\rho^{m}$ and $\mu\rho^{m}$).

              Hence the even permutations of $D_{n}$ form a cyclic group if and only if $n\equiv 3\pmod{4}$.
    \end{enumerate}
\end{proof}

\subsection*{Proof Synopsis}

% section 1/exercise 33
\begin{exercise}
    Give a two-sentence synopsis of the proof of Cayley's Theorem
\end{exercise}

\begin{proof}
    Each element of a given group $G$ corresponds to a left regular representation, which is a permutation of $G$, an element of $S_{G}$. Then we show that this mapping is a homomorphism by the associative law.
\end{proof}

% section 1/exercise 34
\begin{exercise}
    Give a two-sentence synopsis of the proof of Theorem 8.19.
\end{exercise}

\begin{proof}
    Consider the two cases: the 1st case is when two elements in the transposition are in different disjoint cycles of the permutation, the 2nd case is when two elements in the transposition are in the same disjoint cycle of the permutation. In each case, one can omit the disjoint cycles which do not contain the two elements in the transposition and then count the number of orbits.
\end{proof}

\subsection*{Theory}

% section 1/exercise 35
\begin{exercise}
    Suppose that $\phi: G\to G'$ is a group homomorphism and $a\in \ker\phi$. Show that for any $g\in G$, $gag^{-1}\in \ker\phi$.
\end{exercise}

\begin{proof}
    Let $e, e'$ be the identity elements of $G, G'$.
    \begin{align*}
        \phi(gag^{-1}) & = \phi(g)\phi(a)\phi(g^{-1}) \\
                       & = \phi(g)e'\phi(g^{-1})      \\
                       & = \phi(g)\phi(g^{-1})        \\
                       & = \phi(gg^{-1})              \\
                       & = \phi(e)                    \\
                       & = e'
    \end{align*}

    Hence if $a\in\ker\phi$, then for any $g\in G$, $gag^{-1}\in\ker\phi$.
\end{proof}

% section 1/exercise 36
\begin{exercise}
    Prove that a homomorphism $\phi: G\to G'$ is one-to-one if and only if $\ker\phi$ is the trivial subgroup of $G$.
\end{exercise}

\begin{proof}
    If $\phi$ is one-to-one, then $\phi(g) = e' = \phi(e)$ implies $g = e$. Hence $\ker\phi$ is the trivial subgroup of $G$.

    If $\ker\phi$ is the trivial subgroup of $G$, then $\phi(a) = \phi(b)$ if and only if $ab^{-1} = e$, equivalently, $a = b$. Hence $\phi$ is one-to-one.

    Thus a group homomorphism is one-to-one if and only if its kernel is the trivial subgroup.
\end{proof}

% section 1/exercise 37
\begin{exercise}
    Let $\phi: G\to G'$ be a group homomorphism. Show that $\phi(a) = \phi(b)$ if and only if $a^{-1}b\in \ker\phi$.
\end{exercise}

\begin{proof}
    Let $e'$ be the identity element of $G'$.

    $\phi(a^{-1}b) = \phi(a^{-1})\phi(b) = {(\phi(a))}^{-1}\phi(b)$. So $\phi(a^{-1}b) = e'$ if and only if $\phi(a) = \phi(b)$. In other words, $a^{-1}b\in\ker\phi$ if and only if $\phi(a) = \phi(b)$.
\end{proof}

% section 1/exercise 38
\begin{exercise}
    Use Exercise 37 to prove that if $\phi: G\to G'$ is a group homomorphism mapping onto $G'$ and $G$ is a finite group, then for any $b, c\in G'$, $\abs{\phi^{-1}[\{b\}]} = \abs{\phi^{-1}[\{c\}]}$. Conclude that if $\abs{G}$ is a prime number, then either $\phi$ is an isomorphism or else $G'$ is the trivial group.
\end{exercise}

\begin{proof}
    Because $\phi$ is onto $G'$, there exist $x, y\in G$ such that $\phi(x) = b$ and $\phi(y) = c$.

    If $\phi(g) = b$, then $\phi(gx^{-1}y) = \phi(g)\phi(x^{-1})\phi(y) = bb^{-1}c = c$. If $\phi(h) = c$, then $\phi(hy^{-1}x) = \phi(h)\phi(y^{-1})\phi(x) = cc^{-1}b = b$. So the mapping $f: \phi^{-1}[\{b\}] \to \phi^{-1}[\{c\}]$ which is defined as $f(g) = gx^{-1}y$ is a bijection. Therefore $\abs{\phi^{-1}[\{b\}]} = \abs{\phi^{-1}[\{c\}]}$.

    Suppose that $\abs{G}$ is a prime number. We define an equivalence relation on $G$ as follows: $x\sim y$ if and only if $\phi(x) = \phi(y)$. According to the 1st part of this proof, all equivalence classes of this equivalence relation have the same number of elements. On the other hand, equivalence classes are disjoint, so
    \[
        \abs{G} = \text{number of elements in each equivalence class} \times \text{number of equivalence classes}
    \]

    Because $\abs{G}$ is a prime number, we conclude that
    \begin{itemize}
        \item the number of elements in each equivalence class is $1$, which means $\phi$ is one-to-one and onto. So $\phi$ is an isomorphism,
        \item or, the number of equivalences classes is $1$. Together with $\phi$ being onto, we conclude that $G'$ has a single element. So $G'$ is a trivial group.
    \end{itemize}
\end{proof}

% section 1/exercise 39
\begin{exercise}
    Show that if $\phi: G\to G'$ and $\gamma: G'\to G''$ are group homomorphisms, then $\gamma\circ\phi: G\to G''$ is also a group homomorphism.
\end{exercise}

\begin{proof}
    Let $x, y$ be elements of $G$.
    \begin{align*}
        (\gamma\circ\phi)(xy) & = \gamma(\phi(xy))               \\
                              & = \gamma(\phi(x)\phi(y))         \\
                              & = \gamma(\phi(x))\gamma(\phi(y))
    \end{align*}

    Hence $\gamma\circ\phi$ is a group homomorphism.
\end{proof}

% section 1/exercise 40
\begin{exercise}
    Let $\phi: G\to G'$ be a group homomorphism. Show that $\phi[G]$ is abelian if and only if $xyx^{-1}y^{-1}\in \ker\phi$ for all $x,y\in G$.
\end{exercise}

\begin{proof}
    Let $e'$ be the identity element of $G'$.

    $\phi[G]$ is abelian if and only if $\phi(x)\phi(y) = \phi(y)\phi(x)$ for every $x,y\in G$. $\phi(x)\phi(y) = \phi(y)\phi(x)$ if and only if $\phi(xyx^{-1}y^{-1}) = e'$, since $\phi(xyx^{-1}y^{-1}) = \phi(x)\phi(y){(\phi(x))}^{-1}{(\phi(y))}^{-1} = \phi(x)\phi(y){(\phi(y)\phi(x))}^{-1}$. Hence $\phi[G]$ is abelian if and only if $xyx^{-1}y^{-1}\in\ker\phi$.
\end{proof}

% section 1/exercise 41
\begin{exercise}
    Prove the following about $S_{n}$ if $n\geq 3$.
    \begin{enumerate}[label={\textbf{\alph*.}}]
        \item Every permutation in $S_{n}$ can be written as a product of at most $n - 1$ transpositions.
        \item Every permutation in $S_{n}$ that is not a cycle can be written as a product of at most $n - 2$ transpositions.
        \item Every odd permutation in $S_{n}$ can be written as a product of $2n + 3$ transpositions, and every even permutation as a product of $2n + 8$ transpositions.
    \end{enumerate}
\end{exercise}

\begin{proof}
    Every permutation in $S_{n}$ can be written as a product of disjoint cycles. Each cycles of length $k$ can be written as the product of $k-1$ tranposition.
    \begin{enumerate}[label={\textbf{\alph*.}}]
        \item A permutation $\sigma\in S_{n}$ can be written as a product of disjoint cycles $\gamma_{1},\ldots,\gamma_{k}$ where $1\leq k < n$ and the sum of lengths of these disjoint cycles is not greater than $n$. $\gamma_{i}$ can be written as a product of as many transpositions as the length of the cycle minus $1$ (of course it can be written as a product of more transpositions). So $\sigma$ can be written as a product of
              \[
                  (\text{order}(\gamma_{1}) - 1) + \cdots + (\text{order}(\sigma_{k}) - 1) \leq n - 1
              \]
              transpositions.

              Particularly, when $\sigma$ is a cycle of length $n$, it can be written as a product of precisely $n - 1$ transpositions.
        \item If a permutation $\sigma\in S_{n}$ is not a cycle, then it can be written as a product of disjoint cycles $\gamma_{1},\gamma_{2},\ldots,\gamma_{k}$ where $2\leq k < n$, their total length is not greater than $n$. Each cycle can be written as a product of as many transpositions as the length of the cycle minus $1$. Therefore $\sigma$ can be written as a product of
              \[
                  (\text{order}(\gamma_{1}) - 1) + (\text{order}(\gamma_{2}) - 1) + \cdots + (\text{order}(\sigma_{k}) - 1) \leq n - 2
              \]
              transpositions.

              Particularly, when $\sigma$ is a product of two disjoint cycles whose total lengths is equal to $n$, it can be written as a product of precisely $n - 2$ transpositions.
        \item Let $\sigma$ be an odd permutation of $S_{n}$, $\tau$ an even permutation of $\sigma$.

              According to part (a), $\sigma$ can be written as a product of $r$ transpositions, where $r \leq n - 1$ and $r$ is odd (because $\sigma$ is an odd permutation). So $\sigma = {(1,2)}^{2n + 3 - r}\sigma$, which means $\sigma$ can be written as a product of $2n + 3$ transpositions.

              According to part (a), $\tau$ can be written as a product of $s$ transpositions, where $s \leq n - 1$ and $s$ is even (because $\tau$ is an even permutation). So $\tau = {(1,2)}^{2n + 8 - s}\tau$, which means $\tau$ can be written as a product of $2n + 8$ transpositions.
    \end{enumerate}
\end{proof}

% section 1/exercise 42
\begin{exercise}
    \begin{enumerate}[label={\textbf{\alph*.}}]
        \item Draw a figure like Fig. 8.20 to illustrate that if $i$ and $j$ are in different orbits of $\sigma$ and $\sigma(i) = i$, then the number of orbits of $(i,j)\sigma$ is one less than the number of orbits of $\sigma$.
        \item Repeat part (a) if $\sigma(j) = j$ also.
    \end{enumerate}
\end{exercise}

\begin{proof}
    \begin{enumerate}[label={\textbf{\alph*.}}]
        \item
              \begin{tikzpicture}[>=Stealth]
                  \coordinate (j) at (-1,-1);
                  \coordinate (a) at (1,-1);
                  \coordinate (b) at (1,1);
                  \coordinate (c) at (-1,1);
                  \coordinate (i) at (-2,0);

                  \node[label={below left:$j$}] at (j) {};
                  \node[label={below right:$a$}] at (a) {};
                  \node[label={above right:$b$}] at (b) {};
                  \node[label={above left:$c$}] at (c) {};
                  \node[label={left:$i$}] at (i) {};

                  \draw [->] (j) -- (a);
                  \draw [->] (a) -- (b);
                  \draw [->] (b) -- (c);
                  \draw [dashed] (c) -- (j);
                  \draw [->] (c) -- (i);
                  \draw [->] (i) -- (j);
              \end{tikzpicture}
        \item
              \begin{tikzpicture}[>=Stealth]
                  \coordinate (i) at (-1,-1);
                  \coordinate (a) at (1,-1);
                  \coordinate (b) at (1,1);
                  \coordinate (c) at (-1,1);
                  \coordinate (j) at (-2,0);

                  \node[label={below left:$i$}] at (i) {};
                  \node[label={below right:$a$}] at (a) {};
                  \node[label={above right:$b$}] at (b) {};
                  \node[label={above left:$c$}] at (c) {};
                  \node[label={left:$j$}] at (j) {};

                  \draw [->] (a) -- (i);
                  \draw [->] (b) -- (a);
                  \draw [->] (c) -- (b);
                  \draw [dashed] (i) -- (c);
                  \draw [->] (j) -- (c);
                  \draw [->] (i) -- (j);
              \end{tikzpicture}
    \end{enumerate}
\end{proof}

% section 1/exercise 43
\begin{exercise}
    Show that for every subgroup $H$ of $S_{n}$, for $n\geq 2$, either all the permutations in $H$ are even or exactly half of them are even.
\end{exercise}

\begin{proof}
    Suppose that a subgroup $H$ of $S_{n}$ contains both even and odd permutations. Let $\sigma_{1}, \ldots, \sigma_{m}$ be all even permutations in $H$. Because $H$ has odd permutation, there exists an odd permutation $\tau$ in $H$. Let $A$ be the set of all even permutations in $H$, $B$ the set of all odd permutations in $H$. Define a mapping $f: A\to B$ as $f(\sigma_{i}) = \tau\sigma_{i}$. $f$ is one-to-one because $\tau\sigma_{i} = \tau\sigma_{j}$ implies $\sigma_{i} = \sigma_{j}$ due to the cancellation law. $f$ is onto because every odd permutation $\pi$ in $H$ can be written as $\pi = (\tau\tau^{-1})\pi = \tau(\tau^{-1}\pi)$ where $\tau^{-1}\pi$ is an even permutation in $H$ (product of two odd permutations is an even permutation and $H$ is closed). Therefore $f$ is a bijection, from which we conclude that $\abs{A} = \abs{B} = \frac{1}{2}\abs{H}$.

    Hence either all permutations in $H$ are even or exactly half of them are even.
\end{proof}

% section 1/exercise 44
\begin{exercise}
    Let $\sigma$ be a permutation of a set $A$. We shall say ``$\sigma$ \textbf{moves} $a\in A$'' if $\sigma(a)\ne a$. If $A$ is a finite set, how many elements are moved by a cycle $\sigma\in S_{A}$ of length $n$?
\end{exercise}

\begin{proof}
    $\sigma = (a_{1}, a_{2}, \ldots, a_{n})$ where $a_{i}$ are pairwise distinct. $\sigma$ leaves the elements other than $a_{i}$ ($i = 1,\ldots, n$) of $A$ unchanged. Hence if $A$ is a finite set, there are exactly $n$ elements of $A$ are moved by a cycle of length $n$.
\end{proof}

% section 1/exercise 45
\begin{exercise}
    Let $A$ be an infinite set. Let $H$ be the set of all $\sigma\in S_{A}$ such that the number of elements moved by $\sigma$ (see Exercise 44) is finite. Show that $H$ is a subgroup of $S_{A}$.
\end{exercise}

\begin{proof}
    The identity permutation, which does not move any elements, is in $H$.

    Let $\sigma,\tau\in H$ and $\sigma$ moves $a_{1}, \ldots, a_{n}$ only, $\tau$ moves $b_{1}, \ldots, b_{m}$ only. Then $\sigma$ permutes on $\{ a_{1}, \ldots, a_{n} \}$ only, $\tau$ permutes on $\{ b_{1}, \ldots, b_{m} \}$ only. Hence $\sigma\circ\tau$ permutes on $\{ a_{1},\ldots, a_{n} \} \cup \{ b_{1}, \ldots, b_{m} \}$ only, which means $\sigma\circ\tau$ moves finitely many elements of $A$. So $H$ is closed under composition. On the other hand, the inverse of $\sigma$ permutes on $\{ a_{1}, \ldots, a_{n} \}$ only, so $\sigma^{-1}$ is in $H$.

    Hence $H$ is a subgroup of $S_{A}$.
\end{proof}

% section 1/exercise 46
\begin{exercise}
    Let $A$ be an infinite set. Let $K$ be the set of all $\sigma\in S_{A}$ that move (see Exercise 44) at most $50$ elements of $A$. Is $K$ a subgroup of $S_{A}$? Why?
\end{exercise}

\begin{proof}
    $K$ is not a subgroup of $S_{A}$. Because $K$ does not contain the identity permutation, which does not move any elements of $A$.
\end{proof}

% section 1/exercise 47
\begin{exercise}
    Consider $S_{n}$ for a fixed $n\geq 2$ and let $\sigma$ be a fixed odd permutation. Show that every odd permutation in $S_{n}$ is a product of $\sigma$ and some permutation in $A_{n}$.
\end{exercise}

\begin{proof}
    Let $\pi$ be an odd permutation in $S_{n}$. $\pi = (\sigma\sigma^{-1})\pi = \sigma(\sigma^{-1}\pi)$. The inverse of an odd permutation is an odd permutation, the product of two odd permutations is an even permutation, so $\sigma^{-1}\pi$ is an even permutation, which is an element of $A_{n}$.

    Thus every odd permutation in $S_{n}$ is a product of $\sigma$ and a permutation in $A_{n}$.
\end{proof}

% section 1/exercise 48
\begin{exercise}
    Show that if $\sigma$ is a cycle of odd length, then $\sigma^{2}$ is a cycle.
\end{exercise}

\begin{proof}
    Let $\sigma = (a_{1}, a_{2}, \ldots, a_{2n-1})$, $\tau = \sigma^{2}$.

    For $1\leq k \leq n-1$, $\tau^{k}(a_{1}) = a_{2k+1}$. For $n\leq k < 2n-1$, $\tau^{k}(a_{1}) = a_{2(k-n+1)}$. For $k = 2n-1$, $\tau^{k}(a_{1}) = a_{1}$. So $\sigma^{2}$ is a cycle of the same length as the cycle $\sigma$.

    \[
        \tau = \sigma^{2} = (a_{1}, a_{3}, \ldots, a_{2n-1}, a_{2}, a_{4}, \ldots, a_{2n-2})
    \]
\end{proof}

% section 1/exercise 49
\begin{exercise}
    Following the line of thought opened by Exercise 48, complete the following with a condition involving $n$ and $r$ so that the resulting statement is a theorem:
    \begin{center}
        If $\sigma$ is a cycle of length $n$, then $\sigma^{r}$ is also a cycle if and only if\ldots
    \end{center}
\end{exercise}

\begin{proof}
    Completion: If $\sigma$ is a cycle of length $n$, then $\sigma^{r}$ is also a cycle if and only if $n$ and $r$ are relatively prime.
    \[
        \sigma = (a_{0}, a_{1}, \ldots, a_{n-1})
    \]

    $\sigma^{k}$ does not move any elements other than $a_{0}, a_{1}, \ldots, a_{n-1}$ for any $k\in\mathbb{Z}$.

    $(\Rightarrow)$ $n$ and $r$ are relatively prime.

    We take modulo $n$ in the indices.
    \begin{align*}
        \sigma^{r}(a_{0})  & = a_{r}          \\
        \sigma^{2r}(a_{0}) & = a_{2r}         \\
        \vdots                                \\
        \sigma^{nr}(a_{0}) & = a_{nr} = a_{0}
    \end{align*}

    For $1\leq i < j \leq n$, $ir - jr = (i - j)r$ is not divisible by $n$, because $\abs{i - j} < n$ and $r, n$ are relatively prime. So $a_{r}, a_{2r}, \ldots, a_{nr} = a_{0}$ are pairwise distinct. Hence $\sigma^{r}$ is a cycle of length $n$.

    $(\Leftarrow)$ $\sigma^{r}$ is a cycle.

    Assume that the greatest common divisor of $n$ and $r$ is $d > 1$. Let $n/d = x$, $r/d = y$. We take modulo $n$ in the indices.
    \begin{align*}
        \sigma^{r}(a_{0})  & = a_{r}                   \\
        \sigma^{2r}(a_{0}) & = a_{2r}                  \\
        \vdots                                         \\
        \sigma^{xr}(a_{0}) & = a_{xr} = a_{ny} = a_{0}
    \end{align*}

    For $1\leq i < j \leq x$, $(i - j)r$ is not divisible by $n$, since $(i - j)r/n = (i - j)y/x$ and $\abs{i - j} < x$, $x, y$ are relatively prime. So $a_{r}, a_{2r}, \ldots, a_{xr} = a_{0}$ are pairwise distinct.
    \[
        \sigma^{r} = (a_{0}, a_{r}, a_{2r}, \ldots, a_{(x-1)r})(a_{1}, a_{1+r}, \ldots, a_{1+(x-1)r})\cdots (a_{r-1}, a_{r-1+r}, \ldots, a_{r-1+r(x-1)})
    \]

    So if $n, r$ are not relatively prime, $\sigma^{r}$ is not a cycle, which is a contradiction. Therefore $n, r$ are relatively prime.

    Thus if $\sigma$ is a cycle of length $n$, $\sigma^{r}$ is a cycle if and only if $n$ and $r$ are relatively prime.
\end{proof}

% section 1/exercise 50
\begin{exercise}
    Show that $S_{n}$ is generated by $\{ (1,2), (1,2,3,\ldots, n) \}$.
\end{exercise}

\begin{proof}
    \textbf{Step 1.} $(a_{1}, \ldots, a_{k})$ and $(a_{k}, b_{1}, \ldots, b_{\ell})$ are cycles where $\{ a_{1}, \ldots, a_{k} \} \cap \{ b_{1}, \ldots, b_{\ell} \} = \varnothing$ then
    \[
        (a_{1}, \ldots, a_{k})(a_{k}, b_{1}, \ldots, b_{\ell}) = (a_{1},\ldots,a_{k},b_{1},\ldots,b_{\ell}).
    \]

    Let $\tau = (a_{1}, \ldots, a_{k})(a_{k}, b_{1}, \ldots, b_{\ell})$. $\tau(a_{m}) = a_{m+1}$ where $1\leq m < k$. $\tau(a_{k}) = b_{1}$. $\tau(b_{m}) = b_{m+1}$ where $1\leq m < \ell$. $\tau(b_{\ell}) = a_{1}$. Therefore, $\tau$ is the following cycle
    \[
        (a_{1}, \ldots, a_{k})(a_{k}, b_{1}, \ldots, b_{\ell}) = (a_{1},\ldots,a_{k},b_{1},\ldots,b_{\ell}).
    \]

    \textbf{Step 2.} $\{ (1,2), (2,3), (3,4), \ldots, (n-1,n), (n,1) \}$ generates $S_{n}$.

    Let $1\leq i < j\leq n$. According to Step 1
    \begin{align*}
        (i, j) & = (i,i+1)(i+1,i+2)\cdots (j-1,j)                         \\
        (j, i) & = (j,j+1)(j+1,j+2)\cdots (n-1,n)(n,1)(1,2)\cdots (i-1,i)
    \end{align*}

    So every transposition can be generated from $\{ (1,2), (2,3), (3,4), \ldots, (n-1,n), (n,1) \}$. On the other hand, every permutation in $S_{n}$ ($n\geq 2$) can be rewritten as a product of transpositions. Hence $\{ (1,2), (2,3), (3,4), \ldots, (n-1,n), (n,1) \}$ generates $S_{n}$.

    \textbf{Step 3.} (Hint from the book) Every transposition in $\{ (1,2), (2,3), (3,4), \ldots, (n-1,n), (n,1) \}$ can be written in the form ${(1,2,3,\ldots,n)}^{r}(1,2){(1,2,3,\ldots,n)}^{n-r}$.

    When $n = 2$, ${(1,2)}^{0}(1,2){(1,2)}^{2} = (1,2)$, ${(1,2)}^{1}(1,2){(1,2)}^{1} = (1,2)$. So it is true for $n = 2$.

    When $n\geq 3$, let's consider the following cases.

    \begin{enumerate}[label={\textbf{Case \arabic*.}},itemindent=1cm]
        \item $r = n-1$

              If $1 < k < n$,
              \begin{align*}
                  ({(1,2,3,\ldots,n)}^{n-1}(1,2)(1,2,3,\ldots, n))(k) & = ({(1,2,3,\ldots,n)}^{n-1}(1,2))(k+1) \\
                                                                      & = {(1,2,3,\ldots,n)}^{n-1}(k+1)        \\
                                                                      & = (k + 1) + (n - 1)                    \\
                                                                      & = k + n                                \\
                                                                      & = k
              \end{align*}

              If $k = 1$
              \begin{align*}
                  ({(1,2,3,\ldots,n)}^{n-1}(1,2)(1,2,3,\ldots,n))(1) & = ({(1,2,3,\ldots,n)}^{n-1}(1,2))(2) \\
                                                                     & = {(1,2,3,\ldots,n)}^{n-1}(1)        \\
                                                                     & = n
              \end{align*}

              If $k = n$
              \begin{align*}
                  ({(1,2,3,\ldots,n)}^{n-1}(1,2)(1,2,3,\ldots,n))(n) & = ({(1,2,3,\ldots,n)}^{n-1}(1,2))(1) \\
                                                                     & = {(1,2,3,\ldots,n)}^{n-1}(2)        \\
                                                                     & = 2 + n - 1                          \\
                                                                     & = n + 1                              \\
                                                                     & = n
              \end{align*}

              Hence ${(1,2,3,\ldots,n)}^{n-1}(1,2)(1,2,3,\ldots,n) = (n,1)$.
        \item $0\leq r < n - 1$

              We prove ${(1,2,3,\ldots,n)}^{r}(1,2){(1,2,3,\ldots,n)}^{n-r} = (r+1,r+2)$ using mathematical induction.

              When $r = 0$, ${(1,2,3,\ldots,n)}^{0}(1,2){(1,2,3,\ldots,n)}^{n} = \text{id}(1,2)\text{id} = (1,2)$. So the statement holds for $r = 0$.

              Assume that the statement holds for $r = k$, where $0\leq k < n - 2$
              \begin{align*}
                    & {(1,2,3,\ldots,n)}^{k+1}(1,2){(1,2,3,\ldots,n)}^{n-k-1}                                                                    \\
                  = & (1,2,3,\ldots,n){(1,2,3,\ldots,n)}^{k}(1,2){(1,2,3,\ldots,n)}^{n-k}{(1,2,3,\ldots,n)}^{-1}                                 \\
                  = & (1,2,3,\ldots,n)(r+1,r+2){(1,2,3,\ldots,n)}^{-1}                                           & \text{(induction hypothesis)}
              \end{align*}

              If $m = 1$
              \begin{align*}
                    & ((1,2,3,\ldots,n)(k+1,k+2){(1,2,3,\ldots,n)}^{-1})(1) \\
                  = & ((1,2,3,\ldots,n)(k+1,k+2))(n)                        \\
                  = & (1,2,3,\ldots,n)(n)                                   \\
                  = & 1
              \end{align*}

              If $1 < m \leq k+1$
              \begin{align*}
                  ((1,2,3,\ldots,n)(k+1,k+2){(1,2,3,\ldots,n)}^{-1})(m) & = ((1,2,3,\ldots,n)(k+1,k+2))(m-1) \\
                                                                        & = (1,2,3,\ldots,n)(m-1)            \\
                                                                        & = m
              \end{align*}

              If $k+3 < m\leq n$
              \begin{align*}
                  ((1,2,3,\ldots,n)(k+1,k+2){(1,2,3,\ldots,n)}^{-1})(m) & = ((1,2,3,\ldots,n)(k+1,k+2))(m-1) \\
                                                                        & = (1,2,3,\ldots,n)(m-1)            \\
                                                                        & = m
              \end{align*}

              If $m = k+2$
              \begin{align*}
                  ((1,2,3,\ldots,n)(k+1,k+2){(1,2,3,\ldots,n)}^{-1})(m) & = ((1,2,3,\ldots,n)(k+1,k+2){(1,2,3,\ldots,n)}^{-1})(k+2) \\
                                                                        & = ((1,2,3,\ldots,n)(k+1,k+2))(k+1)                        \\
                                                                        & = (1,2,3,\ldots,n)(k+2)                                   \\
                                                                        & = k+3
              \end{align*}

              If $m = k+3$
              \begin{align*}
                  ((1,2,3,\ldots,n)(k+1,k+2){(1,2,3,\ldots,n)}^{-1})(m) & = ((1,2,3,\ldots,n)(k+1,k+2){(1,2,3,\ldots,n)}^{-1})(k+3) \\
                                                                        & = ((1,2,3,\ldots,n)(k+1,k+2))(k+2)                        \\
                                                                        & = (1,2,3,\ldots,n)(k+1)                                   \\
                                                                        & = k+2
              \end{align*}

              So ${(1,2,3,\ldots)}^{k+1}(1,2){(1,2,3,\ldots,n)}^{n-k-1} = (1,2,3,\ldots,n)(k+1,k+2){(1,2,3,\ldots,n)}^{-1} = (k+2,k+3)$. Therefore the statement holds for $r = k+1$.

              Due to the principle of mathematical induction, the statement holds for every $0\leq r < n-2$.
    \end{enumerate}

    Hence $(1,2), (2,3), (3,4), \ldots, (n-1, n), (n,1)$ can be written as ${(1,2,3,\ldots,n)}^{r}(1,2){(1,2,3,\ldots,n)}^{n-r}$ for some $0\leq r\leq n-1$.

    \textbf{Step 4.}

    Due to Step 3, $(1,2), (2,3), (3,4), \ldots, (n-1,n), (n,1)$ can be written as ${(1,2,3,\ldots,n)}^{r}(1,2){(1,2,3,\ldots,n)}^{n-r}$ for some $0\leq r\leq n-1$.

    Due to Step 2, $\{ (1,2), (2,3), (3,4), \ldots, (n-1,n), (n,1) \}$ generates $S_{n}$.

    Therefore $\{ (1,2,3,\ldots,n), (1,2) \}$ generates $S_{n}$.
\end{proof}

% section 1/exercise 51
\begin{exercise}
    Let $\sigma\in S_{n}$ and define a relation on $\{ 1, 2, 3, \ldots, n \}$ by $i\sim j$ if and only if $j = \sigma^{k}(i)$ for some $k\in\mathbb{Z}$
    \begin{enumerate}[label={\textbf{\arabic*.}}]
        \item Prove that $\sim$ is an equivalence relation.
        \item Prove that for any $1\leq i\leq n$, the equivalence class of $i$ is the orbit of $i$.
    \end{enumerate}
\end{exercise}

\begin{proof}
    \begin{enumerate}[label={\textbf{\arabic*.}}]
        \item $a = \sigma^{0}(a)$, so $a\sim a$, $\sim$ is reflexive.

              $a\sim b$ implies $a = \sigma^{k}(b)$ for some integer $k$. $a = \sigma^{k}(b)$ iff $b = \sigma^{-k}(a)$. $b = \sigma^{-k}(a)$ implies $b\sim a$. So $a\sim b$ implies $b\sim a$. Similarly, $b\sim a$ implies $a\sim b$. So $\sim$ is symmetric.

              $a\sim b$ and $b\sim c$ imply $a = \sigma^{k}(b)$ and $b = \sigma^{\ell}(c)$ for some integers $k, \ell$. So $a = \sigma^{k+\ell}(c)$. Therefore $a\sim c$, from which we deduce that $\sim$ is transitive.

              Hence $\sim$ is an equivalence relation.
        \item $a$ is in the orbit of $i$ if and only if there exists an integer $k$ such that $\sigma^{k}(i) = a$, equivalently, $a\sim i$.

              Hence the equivalence class of $i$ is precisely the orbit of $i$.
    \end{enumerate}
\end{proof}

% section 1/exercise 52
\begin{exercise}
    The usual definition for the determinant of an $n\times n$ matrix $A = {(a_{i,j})}$ is
    \[
        \det(A) = \sum_{\sigma\in S_{n}}\text{sgn}(\sigma)a_{1,\sigma(1)}a_{2,\sigma(2)}a_{3,\sigma(3)}\cdots a_{n,\sigma(n)}
    \]

    where $\text{sgn}(\sigma)$ is the sign of $\sigma$. Using this definition, prove the following properties of determinants.
    \begin{enumerate}[label={\textbf{\alph*.}}]
        \item If a row of matrix $A$ has all zero entries, then $\det(A) = 0$.
        \item If two different rows of $A$ are switched to obtain $B$, then $\det(B) = -\det(A)$.
        \item If $r$ times one row of $A$ is added to another row of $A$ to obtain a matrix $B$, then $\det(A) = \det(B)$.
        \item If a row of $A$ is multiplied by $r$ to obtain the matrix $B$, then $\det(B) = r\det(A)$.
    \end{enumerate}
\end{exercise}

\begin{proof}
    \begin{enumerate}[label={\textbf{\alph*.}}]
        \item If $A$ is the zero $n\times n$ matrix, then $a_{i,\sigma(i)} = 0$ for every $1\leq i \leq n$ and $\sigma\in S_{n}$. Therefore
              \[
                  \det(A) = \sum_{\sigma\in S_{n}}\text{sgn}(\sigma)a_{1,\sigma(1)}a_{2,\sigma(2)}a_{3,\sigma(3)}\cdots a_{n,\sigma(n)} = \sum_{\sigma\in S_{n}} 0 = 0.
              \]
        \item Suppose that we switch the $i$-row and $j$-row of $A$ to obtain $B$.
              \begingroup
              \allowdisplaybreaks{}
              \begin{align*}
                  \det(B) & = \sum_{\sigma\in S_{n}}\text{sgn}(\sigma)(\cdots b_{i,\sigma(i)}\cdots b_{j,\sigma(j)}\cdots)                 \\
                          & = \sum_{\sigma\in S_{n}}\text{sgn}(\sigma)(\cdots a_{j,\sigma(j)}\cdots a_{i,\sigma(i)}\cdots)                 \\
                          & = \sum_{\sigma\in S_{n}}\text{sgn}((i,j)\sigma)(\cdots a_{i,\sigma(i)}\cdots a_{j,\sigma(j)}\cdots)            \\
                          & = -\sum_{\sigma\in S_{n}}\text{sgn}(\sigma)a_{1,\sigma(1)}a_{2,\sigma(2)}a_{3,\sigma(3)}\cdots a_{n,\sigma(n)} \\
                          & = -\det(A).
              \end{align*}
              \endgroup
        \item Suppose that we added $r$ times the $i$-row of $A$ to the $j$-row of $A$ to obtain $B$. Notice that $n\geq 2$, so $A_{n}$ is defined. Denote the transposition $(i,j)$ by $\tau$.
              \begingroup
              \allowdisplaybreaks{}
              \begin{align*}
                  \det(B) & = \sum_{\sigma\in S_{n}}\text{sgn}(\sigma)(\cdots a_{i,\sigma(i)}\cdots (a_{j,\sigma(j)} + a_{i,\sigma(j)})\cdots)                                                                              \\
                          & = \sum_{\sigma\in S_{n}}\text{sgn}(\sigma)(\cdots a_{i,\sigma(i)}\cdots a_{j,\sigma(j)}\cdots) + \sum_{\sigma\in S_{n}}\text{sgn}(\sigma)(\cdots a_{i,\sigma(j)}\cdots a_{i,\sigma(j)}\cdots)   \\
                          & = \det(A) + \sum_{\sigma\in S_{n}}\text{sgn}(\sigma)(\cdots a_{i,\sigma(i)}\cdots a_{i,\sigma(j)}\cdots)                                                                                        \\
                          & = \det(A) + \sum_{\sigma\in A_{n}}(\text{sgn}(\sigma)(\cdots a_{i,\sigma(i)}\cdots a_{i,\sigma(j)}\cdots) + \text{sgn}(\sigma\tau)(\cdots a_{i,\sigma\tau(i)}\cdots a_{i,\sigma\tau(j)}\cdots)) \\
                          & = \det(A) + \sum_{\sigma\in A_{n}}(\text{sgn}(\sigma)(\cdots a_{i,\sigma(i)}\cdots a_{i,\sigma(j)}\cdots) - \text{sgn}(\sigma)(\cdots a_{i,\sigma(i)}\cdots a_{i,\sigma(j)}\cdots))             \\
                          & = \det(A).
              \end{align*}
              \endgroup
        \item Suppose that we multiply the $i$-th row of $A$ by $r$ to obtain $B$.
              \begingroup
              \allowdisplaybreaks{}
              \begin{align*}
                  \det(B) & = \sum_{\sigma\in S_{n}}\text{sgn}(\sigma)(\cdots (ra_{i,\sigma(i)})\cdots)                                    \\
                          & = r\sum_{\sigma\in S_{n}}\text{sgn}(\sigma)a_{1,\sigma(1)}a_{2,\sigma(2)}a_{3,\sigma(3)}\cdots a_{n,\sigma(n)} \\
                          & = r\det(A).
              \end{align*}
              \endgroup
    \end{enumerate}
\end{proof}

% section 1/exercise 53
\begin{exercise}
    Prove that any finite group $G$ is isomorphic with a subgroup of $\text{GL}(n,\mathbb{R})$ for some $n$.
\end{exercise}

\begin{proof}
    This proof uses a property of matrix multiplication. \\

    \textbf{Lemma.} $\phi: G\to G'$ is a group homomorphism. $H$ is a subgroup of $G'$, then $\phi[H]$ is a subgroup of $G'$.

    \textit{Proof of the Lemma.} Let $e, e'$ be the identity elements of $G, G'$, respectively. Because $\phi(e) = e'$, $\phi[H]$ contains $e'$. $\phi(x), \phi(y)$ are elements of $\phi[H]$ (where $x,y\in H$), then $\phi(x)\phi(y) = \phi(xy) \in \phi[H]$ (because $\phi$ is a group homomorphism and $H$ is a subgroup of $G$). $\phi(x^{-1})\phi(x) = \phi(x)\phi(x^{-1}) = \phi(e) = e'$, so $\phi(x^{-1})$ is the inverse of $\phi(x)$. On the other hand, $x^{-1}\in H$, so $\phi(x^{-1})\in \phi[H]$. Hence $\phi[H]$ is a subgroup of $G'$. \\

    Proof of the Lemma is completed.

    Let $m$ be the number of elements of $G$. According to Cayley's theorem, $G$ is isomorphic to a subgroup of $S_{G}$. $S_{G}$ is isomorphic to $S_{m}$. Let $n = \text{factorial}(m)$.

    Consider the set $E$ of all $n\times n$ matrices whose rows are $e_{1}, e_{2}, \ldots, e_{n}$, where all entries of $e_{i}$ is zero, except the $i$-th entry. Let $\phi: S_{n} \to E$ be defined as $\phi(\sigma) =$ the matrix whose $i$-th row is $e_{\sigma(i)}$. $\phi$ is a one-to-one and onto mapping. On the other hand, every permutation is a product of transpositions, each transposition corresponds to swapping two rows, and swapping two rows of the matrix $A$ on the left of a matrix product $AB$ also swaps two respective rows of the product matrix $AB$, so
    \[
        \phi(\sigma\tau) = \begin{bmatrix}
            (\sigma\tau)(e_{1}) \\
            (\sigma\tau)(e_{2}) \\
            \vdots              \\
            (\sigma\tau)(e_{n})
        \end{bmatrix}
        = \phi(\sigma)\begin{bmatrix}
            \tau(e_{1}) \\
            \tau(e_{2}) \\
            \vdots      \\
            \tau(e_{n})
        \end{bmatrix}
        = \phi(\sigma)\phi(\tau)\begin{bmatrix}
            e_{1}  \\
            e_{2}  \\
            \vdots \\
            e_{n}
        \end{bmatrix}
        = \phi(\sigma)\phi(\tau)I_{n}
        = \phi(\sigma)\phi(\tau).
    \]

    According to the lemma above, $E$ is a subgroup of $\text{GL}(n, \mathbb{R})$.

    To sum up
    \begin{itemize}
        \item $G$ is isomorphic to a subgroup of $S_{G}$
        \item $S_{G}$ is isomorphic to $S_{m}$ where $m = \abs{G}$
        \item $S_{m}$ is isomorphic to a subgroup of $S_{n}$ where $n = \text{factorial}(n)$
        \item $S_{n}$ is isomorphic to the group $E$ of $n\times n$ matrices whose rows are $e_{1},\ldots, e_{n}$
        \item $E$ which is a subgroup of $\text{GL}(n, \mathbb{R})$
    \end{itemize}

    Hence $G$ is isomorphic to a subgroup of $\text{GL}(n, \mathbb{R})$ where $n = \text{factorial}(\abs{G})$.
\end{proof}

% section 1/exercise 54
\begin{exercise}
    Prove Cayley's Theorem using the right regular representation rather than the left regular representation.
\end{exercise}

\begin{proof}
    Let $G$ be a group. For each element $x\in G$, we defined a mapping $\tau_{x}: G\to G$ as $\tau_{x}(g) = gx$ for every $g\in G$. $\tau_{x}$ is an element of $S_{G}$. Let's consider the mapping $\phi: G\to S_{G}$ defined as $x\mapsto \tau_{x}$.

    $\tau_{x} = \tau_{y}$ if and only if $gx = gy$ for every $g\in G$. Due to the cancellation law, $gx = gy$ if and only if $x = y$. So $\phi$ is one-to-one.

    $\tau_{xy}(g) = (xy)g = x(yg) = \tau_{x}(yg) = \tau_{x}(\tau_{y}(g)) = (\tau_{x}\tau_{y})(g)$ for every $g\in G$. So $\phi$ is a homomorphism.

    Therefore $\phi$ is a one-to-one group homomorphism from $G$ into $S_{G}$. Hence $G$ is isomorphic with a subgroup of $S_{G}$.
\end{proof}

% section 1/exercise 55
\begin{exercise}
    Let $\sigma\in S_{n}$. An inversion is a pair $(i, j)$ such that $i < j$ and $\sigma(i) > \sigma(j)$. Prove Theorem 8.19 by showing that multiplying a permutation by a transposition changes the number of inversions by an odd number.
\end{exercise}

\begin{proof}
    I repeat.

    \textbf{Theorem 8.19.} No permutation in $S_{n}$ can be expressed both as a product of an even number of transpositions and as a product of an odd number of transpositions.

    But first, I prove the following statement.

    \textbf{Lemma.} $\tau = (i, j)$ is a transposition in $S_{n}$. $\sigma$ is an element of $S_{n}$. Prove that the difference between the number of inversions of $\sigma$ and $\tau\sigma$ is an odd number.

    \textit{Proof of the Lemma.} First, we count the number of inversions of $\tau$.

    \begin{itemize}
        \item For $(r, s)$ where $\{ r, s \}\cap\{ i, j \} = \emptyset$, $(r, s)$ is not an inversion.
        \item For $(r, s)$ where $s = i, r < i$, $(r, s)$ is not an inversion.
        \item For $(r, s)$ where $r = j, s > r$, $(r, s)$ is not an inversion.
        \item For $(r, s)$ where $r = i, i < s < j$, $\frac{\tau(r) - \tau(s)}{r - s} = \frac{j - s}{i - s} < 0$, so $(r, s)$ is an inversion. There are $j - i - 1$ inversions of this type.
        \item For $(r, s)$ where $i < r < j, s = j$, $\frac{\tau(r) - \tau(s)}{r - s} = \frac{r - i}{r - s} < 0$, so $(r, s)$ is an inversion. There are $j - i - 1$ inversions of this type.
        \item For $(r, s)$ where $r = i, s = j$, $(r, s)$ is an inversion. There is $1$ inversion of this type.
        \item For $(r, s)$ where $r = i, s > j$, $\frac{\tau(r) - \tau(s)}{r - s} = \frac{j - s}{i - s} > 0$, so $(r, s)$ is not an inversion.
        \item For $(r, s)$ where $r < i, s = j$, $\frac{\tau(r) - \tau(s)}{r - s} = \frac{r - i}{r - j} > 0$, so $(r, s)$ is not an inversion.
    \end{itemize}

    To sum up, $\tau$ has $2(j - i) - 1$ inversions, an odd number of inversions.

    Let $f: S_{n}\to \{ -1, 1 \}$ be defined as $f(\sigma) = \prod_{1\leq i < j\leq n}\frac{\sigma(i) - \sigma(j)}{i - j}$. For $\sigma_{1}, \sigma_{2}\in S_{n}$. If $\sigma$ has an odd number of inversions, $f(\sigma) = -1$. Otherwise, $f(\sigma) = 1$.
    \begin{align*}
        f(\sigma_{1}\sigma_{2}) & = \prod_{1\leq i < j \leq n}\frac{(\sigma_{1}\sigma_{2})(i) - (\sigma_{1}\sigma_{2})(j)}{i - j}                                                                                                      \\
                                & = \prod_{1\leq i < j \leq n}\frac{\sigma_{1}(\sigma_{2}(i)) - \sigma_{1}(\sigma_{2}(j))}{\sigma_{2}(i) - \sigma_{2}(j)} \times \prod_{1\leq i < j\leq n} \frac{\sigma_{1}(i) - \sigma_{1}(i)}{i - j} \\
                                & = f(\sigma_{1})f(\sigma_{2})
    \end{align*}

    So $f$ is a group homomorphism. Therefore $f(\tau\sigma) = f(\tau)f(\sigma) = -f(\sigma)$, which indicates the difference between the number of inversions of $\tau\sigma$ and $\sigma$ is an odd number.

    Proof of the lemma is completed.
    \bigskip

    Back to proof of Theorem 8.19. Again, we consider a transposition $\tau$ and a permutation $\sigma$. If $\sigma$ can be rewritten both as a product of an odd and an even number of transpositions, then due to the Lemma, the number of inversions of $\sigma$ can be both odd and even, which is impossible because the number of inversions of a permutation is well-defined.

    Hence a permutation can be rewritten either as a product of an odd number of transpositions, or as a product of an even number of transpositions.
\end{proof}

% section 1/exercise 56
\begin{exercise}
    The sixteen puzzle consists of 15 tiles numbered 1 through 15 arranged in a four-by-four grid with one position left blank. A move is sliding a tile adjacent to the blank position into the blank position. The goal is to arrange the numbers in order by a sequence of moves. Is it possible to start with the configuration pictured in Figure 8.27(a) and solve the puzzle as indicated in Figure 8.27(b)? Prove your answer by finding a sequence of moves to solve the puzzle or by proving that it is impossible to solve.
    \begin{figure}[htp]
        \centering
        \begin{tabular}{|c|c|c|c|}
            \hline
            1  & 2  & 3  & 4  \\
            \hline
            5  & 6  & 7  & 8  \\
            \hline
            9  & 10 & 11 & 12 \\
            \hline
            13 & 15 & 14 &    \\
            \hline
        \end{tabular}
        \hspace{2cm}
        \begin{tabular}{|c|c|c|c|}
            \hline
            1  & 2  & 3  & 4  \\
            \hline
            5  & 6  & 7  & 8  \\
            \hline
            9  & 10 & 11 & 12 \\
            \hline
            13 & 14 & 15 &    \\
            \hline
        \end{tabular}
    \end{figure}
\end{exercise}

\begin{proof}
    Unsolved. See \url{https://en.wikipedia.org/wiki/Sliding_puzzle}
    % unsolved
    % see https://en.wikipedia.org/wiki/Sliding_puzzle
\end{proof}

\section{Finitely Generated Abelian Groups}

\subsection*{Computations}

% section 9/exercise 1
\begin{exercise}
    List the elements of $\mathbb{Z}_{2}\times\mathbb{Z}_{4}$. Find the order of each of the elements. Is this group cyclic?
\end{exercise}

\begin{proof}
    \begin{itemize}
        \item $(0, 0)$. The order of this element is $1$.
        \item $(1, 0)$. The order of this element is $2$.
        \item $(0, 1)$. The order of this element is $4$.
        \item $(0, 2)$. The order of this element is $2$.
        \item $(0, 3)$. The order of this element is $4$.
        \item $(1, 1)$. The order of this element is $4$.
        \item $(1, 2)$. The order of this element is $2$.
        \item $(1, 3)$. The order of this element is $2$.
    \end{itemize}

    This group is not cyclic, since the order of every element is less than the order of the group.
\end{proof}

% section 9/exercise 2
\begin{exercise}
    Repeat Exercise 1 from the group $\mathbb{Z}_{3}\times\mathbb{Z}_{4}$.
\end{exercise}

\begin{proof}
    \begin{itemize}
        \item $(0, 0)$. The order of this element is $1$.
        \item $(1, 0)$. The order of this element is $3$.
        \item $(2, 0)$. The order of this element is $3$.
        \item $(0, 1)$. The order of this element is $4$.
        \item $(1, 1)$. The order of this element is $12$.
        \item $(2, 1)$. The order of this element is $12$.
        \item $(0, 2)$. The order of this element is $2$.
        \item $(1, 2)$. The order of this element is $6$.
        \item $(2, 2)$. The order of this element is $6$.
        \item $(0, 3)$. The order of this element is $4$.
        \item $(1, 3)$. The order of this element is $12$.
        \item $(2, 3)$. The order of this element is $12$.
    \end{itemize}

    This group is cyclic, since $\anglebracket{(1,1)} = \anglebracket{(2,1)} = \anglebracket{(1,3)} = \anglebracket{(2,3)} = \mathbb{Z}_{3}\times\mathbb{Z}_{4}$.
\end{proof}

In Exercise 3 through 7, find the order of the given element of the direct product.

% section 9/exercise 3
\begin{exercise}
    $(2,6)$ in $\mathbb{Z}_{4}\times\mathbb{Z}_{12}$
\end{exercise}

\begin{proof}
    The order of $2$ in $\mathbb{Z}_{4}$ is $2$. The order of $6$ in $\mathbb{Z}_{12}$ is $2$. So the order of $(2,6)$ in $\mathbb{Z}_{4}\times\mathbb{Z}_{12}$ is the least common multiple of $2$ and $2$, which is $2$.
\end{proof}

% section 9/exercise 4
\begin{exercise}
    $(3,4)$ in $\mathbb{Z}_{21}\times\mathbb{Z}_{12}$
\end{exercise}

\begin{proof}
    The order of $3$ in $\mathbb{Z}_{21}$ is $21/\text{gcd}(3,21) = 7$. The order of $4$ in $\mathbb{Z}_{12}$ is $12/\text{gcd}(4,12) = 3$. So the order of $(3,4)$ in $\mathbb{Z}_{21}\times\mathbb{Z}_{12}$ is the least common multiple of $7$ and $3$, which is $21$.
\end{proof}

% section 9/exercise 5
\begin{exercise}
    $(40,12)$ in $\mathbb{Z}_{45}\times\mathbb{Z}_{18}$
\end{exercise}

\begin{proof}
    The order of $40$ in $\mathbb{Z}_{45}$ is $45/\text{gcd}(45,40) = 9$. The order of $12$ in $\mathbb{Z}_{18}$ is $18/\text{gcd}(12,18) = 3$. So the order of $(45,12)$ in $\mathbb{Z}_{21}\times\mathbb{Z}_{18}$ is the least common multiple of $9$ and $3$, which is $9$.
\end{proof}

% section 9/exercise 6
\begin{exercise}
    $(3,10,9)$ in $\mathbb{Z}_{4}\times\mathbb{Z}_{12}\times\mathbb{Z}_{15}$
\end{exercise}

\begin{proof}
    The order of $3$ in $\mathbb{Z}_{4}$ is $4/\text{gcd}(4,3) = 4$. The order of $10$ in $\mathbb{Z}_{12}$ is $12/\text{gcd}(10,12) = 6$. The order of $9$ in $\mathbb{Z}_{15}$ is $15/\text{gcd}(15,9) = 5$. So the order of $(3,10,9)$ in $\mathbb{Z}_{4}\times\mathbb{Z}_{12}\times\mathbb{Z}_{15}$ is the least common multiple of $4, 6$, and $5$, which is $60$.
\end{proof}

% section 9/exercise 7
\begin{exercise}
    $(3,6,12,16)$ in $\mathbb{Z}_{4}\times\mathbb{Z}_{12}\times\mathbb{Z}_{20}\times\mathbb{Z}_{24}$
\end{exercise}

\begin{proof}
    The order of $3$ in $\mathbb{Z}_{4}$ is $4/\text{gcd}(4,3) = 4$. The order of $6$ in $\mathbb{Z}_{12}$ is $12/\text{gcd}(12,6) = 2$. The order of $12$ in $\mathbb{Z}_{20}$ is $20/\text{gcd}(20,12) = 5$. The order of $16$ in $\mathbb{Z}_{24}$ is $24/\text{gcd}(24,16) = 3$. So the order of $(3,6,12,16)$ in $\mathbb{Z}_{4}\times\mathbb{Z}_{12}\times\mathbb{Z}_{20}\times\mathbb{Z}_{24}$ is the least common multiple of $4, 2, 5$, and $3$, which is $60$.
\end{proof}

% section 9/exercise 8
\begin{exercise}
    What is the largest order among the orders of all the cyclic subgroups of $\mathbb{Z}_{6}\times\mathbb{Z}_{8}$? of $\mathbb{Z}_{12}\times\mathbb{Z}_{15}$?
\end{exercise}

\begin{proof}
    For every element $(x,y)$ of $\mathbb{Z}_{6}\times\mathbb{Z}_{8}$, $24(x,y) = (24x,24y) = (0,0)$. On the other hand, the order of $(1,1)$ in $\mathbb{Z}_{6}\times\mathbb{Z}_{8}$ is $24$. So the largest order among the orders of all the cyclic subgroups of $\mathbb{Z}_{6}\times\mathbb{Z}_{8}$ is $24$.

    For every element $(x,y)$ of $\mathbb{Z}_{12}\times\mathbb{Z}_{15}$, $60(x,y) = (60x,60y) = (0,0)$. On the other hand, the order of $(1,1)$ in $\mathbb{Z}_{12}\times\mathbb{Z}_{15}$ is $60$. So the largest order among the orders of all the cyclic subgroups of $\mathbb{Z}_{12}\times\mathbb{Z}_{15}$ is $60$.
\end{proof}

% section 9/exercise 9
\begin{exercise}
    Find all proper nontrivial subgroups of $\mathbb{Z}_{2}\times\mathbb{Z}_{2}$
\end{exercise}

\begin{proof}
    They are $\mathbb{Z}_{2}\times\{ 0 \}$ and $\{ 0 \}\times\mathbb{Z}_{2}$. In fact, these two subgroups are isomorphic.
\end{proof}

% section 9/exercise 10
\begin{exercise}
    Find all proper nontrivial subgroups of $\mathbb{Z}_{2}\times\mathbb{Z}_{2}\times\mathbb{Z}_{2}$
\end{exercise}

\begin{proof}
    They are
    \begin{itemize}
        \item $\mathbb{Z}_{2}\times\{0\}\times\{0\}$, $\{0\}\times\mathbb{Z}_{2}\times\{0\}$, $\{0\}\times\{0\}\times\mathbb{Z}_{2}$
        \item $\mathbb{Z}_{2}\times\mathbb{Z}_{2}\times\{0\}$, $\mathbb{Z}_{2}\times\{0\}\times\mathbb{Z}_{2}$, $\{0\}\times\mathbb{Z}_{2}\times\mathbb{Z}_{2}$
    \end{itemize}
\end{proof}

% section 9/exercise 11
\begin{exercise}
    Find all subgroups of $\mathbb{Z}_{2}\times\mathbb{Z}_{4}$ of order $4$.
\end{exercise}

\begin{proof}
    All subgroups of $\mathbb{Z}_{2}\times\mathbb{Z}_{4}$ are
    \begin{itemize}
        \item $\{0\}\times\{0\}$. This subgroup is of order $1$.
        \item $\mathbb{Z}_{2}\times\{0\}$. This subgroup is of order $2$.
        \item $\{0\}\times\{0,2\}$. This subgroup is of order $2$.
        \item $\{0\}\times\mathbb{Z}_{4}$. This subgroup is of order $4$.
        \item $\mathbb{Z}_{2}\times\{ 0, 2 \}$. This subgroup is of order $4$.
        \item $\mathbb{Z}_{2}\times\mathbb{Z}_{4}$ is of order $8$.
    \end{itemize}

    Hence all subgroups of $\mathbb{Z}_{2}\times\mathbb{Z}_{4}$ of order $4$ are: $\{0\}\times\mathbb{Z}_{4}$, $\mathbb{Z}_{2}\times\{0,2\}$.
\end{proof}

% section 9/exercise 12
\begin{exercise}
    Find all subgroups of $\mathbb{Z}_{2}\times\mathbb{Z}_{2}\times\mathbb{Z}_{4}$ that are isomorphic to the Klein $4$-group.
\end{exercise}

\begin{proof}
    All subgroups of order $4$ of $\mathbb{Z}_{2}\times\mathbb{Z}_{2}\times\mathbb{Z}_{4}$ are
    \begin{itemize}
        \item $\{0\}\times\{0\}\times\mathbb{Z}_{4}$. This subgroup is not isomorphic to the Klein $4$-group.
        \item $\{0\}\times\mathbb{Z}_{2}\times\{0,2\}$. This subgroup is isomorphic to the Klein $4$-group.
        \item $\mathbb{Z}_{2}\times\{0\}\times\{0,2\}$. This subgroup is isomorphic to the Klein $4$-group.
        \item $\mathbb{Z}_{2}\times\mathbb{Z}_{2}\times\{0\}$. This subgroup is isomorphic to the Klein $4$-group.
    \end{itemize}
\end{proof}

% section 9/exercise 13
\begin{exercise}
    Disregarding the order of the factors, write direct products of two or more groups of the form $\mathbb{Z}_{n}$ so that the resulting product is isomorphic to $\mathbb{Z}_{60}$ in as many ways as possible.
\end{exercise}

\begin{proof}
    \begingroup
    \allowdisplaybreaks{}
    \begin{align*}
        \mathbb{Z}_{60} & \simeq \mathbb{Z}_{3}\times\mathbb{Z}_{4}\times\mathbb{Z}_{5} \\
        \mathbb{Z}_{60} & \simeq \mathbb{Z}_{3}\times\mathbb{Z}_{20}                    \\
        \mathbb{Z}_{60} & \simeq \mathbb{Z}_{4}\times\mathbb{Z}_{15}                    \\
        \mathbb{Z}_{60} & \simeq \mathbb{Z}_{5}\times\mathbb{Z}_{12}                    \\
        \mathbb{Z}_{60} & \simeq \mathbb{Z}_{60}
    \end{align*}
    \endgroup
\end{proof}

% section 9/exercise 14
\begin{exercise}
    Fill in the blanks
    \begin{enumerate}[label={\textbf{\alph*.}}]
        \item The cyclic subgroup of $\mathbb{Z}_{24}$ generated by $18$ has order \_\_\_.
        \item $\mathbb{Z}_{3}\times\mathbb{Z}_{4}$ is of order \_\_\_.
        \item The element $(4,2)$ of $\mathbb{Z}_{12}\times\mathbb{Z}_{8}$ has order \_\_\_.
        \item The Klein $4$-group is isomorphic to $\mathbb{Z}\_\_\_\times\mathbb{Z}\_\_\_$.
        \item $\mathbb{Z}_{2}\times\mathbb{Z}\times\mathbb{Z}_{4}$ has \_\_\_ elements of finite order.
    \end{enumerate}
\end{exercise}

\begin{proof}
    \begin{enumerate}[label={\textbf{\alph*.}}]
        \item The cyclic subgroup of $\mathbb{Z}_{24}$ generated by $18$ has order $4$.
        \item $\mathbb{Z}_{3}\times\mathbb{Z}_{4}$ is of order $12$.
        \item The element $(4,2)$ of $\mathbb{Z}_{12}\times\mathbb{Z}_{8}$ has order $12$.
        \item The Klein $4$-group is isomorphic to $\mathbb{Z}_{2}\times\mathbb{Z}_{2}$.
        \item $\mathbb{Z}_{2}\times\mathbb{Z}\times\mathbb{Z}_{4}$ has $8$ elements of finite order.
    \end{enumerate}
\end{proof}

% section 9/exercise 15
\begin{exercise}
    Find the maximum possible order for some element of $\mathbb{Z}_{4}\times\mathbb{Z}_{6}$.
\end{exercise}

\begin{proof}
    The order of $(1,1)$ in $\mathbb{Z}_{4}\times\mathbb{Z}_{6}$ is $12$. For every element $(x,y)$ in $\mathbb{Z}_{4}\times\mathbb{Z}_{6}$, $12(x,y) = (12x,12y) = (0,0)$. So the maximum possible order for some element of $\mathbb{Z}_{4}\times\mathbb{Z}_{6}$ is $12$.
\end{proof}

% section 9/exercise 16
\begin{exercise}
    Are the groups $\mathbb{Z}_{2}\times\mathbb{Z}_{12}$ and $\mathbb{Z}_{4}\times\mathbb{Z}_{6}$ isomorphic? Why or why not?
\end{exercise}

\begin{proof}
    Because $\mathbb{Z}_{mn}\simeq \mathbb{Z}_{m}\times\mathbb{Z}_{n}$ if and only if $m$ and $n$ are relatively prime,
    \begin{align*}
        \mathbb{Z}_{2}\times\mathbb{Z}_{12} & \simeq \mathbb{Z}_{2}\times\mathbb{Z}_{4}\times\mathbb{Z}_{3} \\
        \mathbb{Z}_{4}\times\mathbb{Z}_{6}  & \simeq \mathbb{Z}_{4}\times\mathbb{Z}_{2}\times\mathbb{Z}_{3}
    \end{align*}

    Hence these two groups are isomorphic.
\end{proof}

% section 9/exercise 17
\begin{exercise}
    Find the maximum possible order for some element of $\mathbb{Z}_{8}\times\mathbb{Z}_{28}\times\mathbb{Z}_{24}$.
\end{exercise}

\begin{proof}
    The order of $(1,1,1)$ in $\mathbb{Z}_{8}\times\mathbb{Z}_{28}\times\mathbb{Z}_{24}$ is the least common multiple of $8,24$, and $28$, which is $168$. For every element $(x,y,z)$ in $\mathbb{Z}_{8}\times\mathbb{Z}_{28}\times\mathbb{Z}_{24}$, $168(x,y,z) = (168x,168y,168z) = (0,0,0)$. So the maximum possible order of some element of $\mathbb{Z}_{8}\times\mathbb{Z}_{28}\times\mathbb{Z}_{24}$ is $168$.
\end{proof}

% section 9/exercise 18
\begin{exercise}
    Are the groups $\mathbb{Z}_{8}\times\mathbb{Z}_{10}\times\mathbb{Z}_{24}$ and $\mathbb{Z}_{4}\times\mathbb{Z}_{12}\times\mathbb{Z}_{40}$ isomorphic? Why or why not?
\end{exercise}

\begin{proof}
    Because $\mathbb{Z}_{mn}\simeq \mathbb{Z}_{m}\times\mathbb{Z}_{n}$ if and only if $m$ and $n$ are relatively prime,
    \begin{align*}
        \mathbb{Z}_{8}\times\mathbb{Z}_{10}\times\mathbb{Z}_{24} & \simeq \mathbb{Z}_{8}\times\mathbb{Z}_{2}\times\mathbb{Z}_{5}\times\mathbb{Z}_{8}\times\mathbb{Z}_{3} \simeq \mathbb{Z}_{2}\times\mathbb{Z}_{8}\times\mathbb{Z}_{8}\times\mathbb{Z}_{3}\times\mathbb{Z}_{5} \\
        \mathbb{Z}_{4}\times\mathbb{Z}_{12}\times\mathbb{Z}_{40} & \simeq \mathbb{Z}_{4}\times\mathbb{Z}_{4}\times\mathbb{Z}_{3}\times\mathbb{Z}_{8}\times\mathbb{Z}_{5} \simeq \mathbb{Z}_{4}\times\mathbb{Z}_{4}\times\mathbb{Z}_{8}\times\mathbb{Z}_{3}\times\mathbb{Z}_{5}
    \end{align*}

    According to the primary version of the fundamental theorem of finitely generated abelian group, the two groups are not isomorphic.
\end{proof}

% section 9/exercise 19
\begin{exercise}
    Find the maximum possible order for some element of $\mathbb{Z}_{4}\times\mathbb{Z}_{18}\times\mathbb{Z}_{15}$.
\end{exercise}

\begin{proof}
    The order of $(1,1,1)$ in $\mathbb{Z}_{4}\times\mathbb{Z}_{18}\times\mathbb{Z}_{15}$ is the least common multiple of $4,18$, and $15$, which is $180$. For every element $(x,y,z)$ in $\mathbb{Z}_{4}\times\mathbb{Z}_{18}\times\mathbb{Z}_{15}$, $180(x,y,z) = (180x,180y,180z) = (0,0,0)$. So the maximum possible order of some element of $\mathbb{Z}_{4}\times\mathbb{Z}_{18}\times\mathbb{Z}_{15}$ is $180$.
\end{proof}

% section 9/exercise 20
\begin{exercise}
    Are the groups $\mathbb{Z}_{4}\times\mathbb{Z}_{18}\times\mathbb{Z}_{15}$ and $\mathbb{Z}_{3}\times\mathbb{Z}_{36}\times\mathbb{Z}_{10}$ isomorphic? Why or why not?
\end{exercise}

\begin{proof}
    Because $\mathbb{Z}_{mn}\simeq \mathbb{Z}_{m}\times\mathbb{Z}_{n}$ if and only if $m$ and $n$ are relatively prime,
    \begin{align*}
        \mathbb{Z}_{4}\times\mathbb{Z}_{18}\times\mathbb{Z}_{15} & \simeq \mathbb{Z}_{4}\times\mathbb{Z}_{9}\times\mathbb{Z}_{2}\times\mathbb{Z}_{3}\times\mathbb{Z}_{5} \simeq \mathbb{Z}_{2}\times\mathbb{Z}_{4}\times\mathbb{Z}_{3}\times\mathbb{Z}_{9}\times\mathbb{Z}_{5}  \\
        \mathbb{Z}_{3}\times\mathbb{Z}_{36}\times\mathbb{Z}_{10} & \simeq \mathbb{Z}_{3}\times\mathbb{Z}_{4}\times\mathbb{Z}_{9}\times\mathbb{Z}_{2}\times\mathbb{Z}_{5} \simeq \mathbb{Z}_{2}\times \mathbb{Z}_{4}\times\mathbb{Z}_{3}\times\mathbb{Z}_{9}\times\mathbb{Z}_{5}
    \end{align*}

    So the two groups are isomorphic.
\end{proof}

In Exercises 21 through 25, proceed as in Example 9.13 to find all abelian groups, up to isomorphism, of the given order. For each group, find the invariant factors and find an isomorphic group of the form indicated in Theorem 9.14.

If $n = {(p_{1})}^{k_{1}}{(p_{2})}^{k_{2}}\cdots {(p_{r})}^{k_{r}}$, then there are $p(k_{1})p(k_{2})\cdots p(k_{r})$ abelian groups of order $n$, up to isomorphism, where $p(k)$ is the number of partitions of $k$.

% section 9/exercise 21
\begin{exercise}
    Order 8
\end{exercise}

\begin{proof}
    All abelian groups of order $8 = 2^{3}$, up to isomorphism are
    \begin{enumerate}
        \item $\mathbb{Z}_{8}$
        \item $\mathbb{Z}_{2} \times \mathbb{Z}_{4}$
        \item $\mathbb{Z}_{2} \times \mathbb{Z}_{2} \times \mathbb{Z}_{2}$
    \end{enumerate}
\end{proof}

% section 9/exercise 22
\begin{exercise}
    Order 16
\end{exercise}

\begin{proof}
    All abelian groups of order $16 = 2^{4}$, up to isomorphism are
    \begin{enumerate}
        \item $\mathbb{Z}_{16}$
        \item $\mathbb{Z}_{2} \times \mathbb{Z}_{8}$
        \item $\mathbb{Z}_{4} \times \mathbb{Z}_{4}$
        \item $\mathbb{Z}_{2} \times \mathbb{Z}_{2} \times \mathbb{Z}_{4}$
        \item $\mathbb{Z}_{2} \times \mathbb{Z}_{2} \times \mathbb{Z}_{2} \times \mathbb{Z}_{2}$
    \end{enumerate}
\end{proof}

% section 9/exercise 23
\begin{exercise}
    Order 32
\end{exercise}

\begin{proof}
    All abelian groups of order $32 = 2^{5}$, up to isomorphism are
    \begin{enumerate}
        \item $\mathbb{Z}_{32}$
        \item $\mathbb{Z}_{2} \times \mathbb{Z}_{16}$
        \item $\mathbb{Z}_{4} \times \mathbb{Z}_{8}$
        \item $\mathbb{Z}_{2} \times \mathbb{Z}_{2} \times \mathbb{Z}_{8}$
        \item $\mathbb{Z}_{2} \times \mathbb{Z}_{4} \times \mathbb{Z}_{4}$
        \item $\mathbb{Z}_{2} \times \mathbb{Z}_{2} \times \mathbb{Z}_{2} \times \mathbb{Z}_{4}$
        \item $\mathbb{Z}_{2} \times \mathbb{Z}_{2} \times \mathbb{Z}_{2} \times \mathbb{Z}_{2} \times \mathbb{Z}_{2}$
    \end{enumerate}
\end{proof}

% section 9/exercise 24
\begin{exercise}
    Order 720
\end{exercise}

\begin{proof}
    All abelian groups of order $720 = 2^{4}3^{2}5$, up to isomorphism are
    \begin{enumerate}
        \item $\mathbb{Z}_{720}$
        \item $\mathbb{Z}_{4} \times \mathbb{Z}_{180}$
        \item $\mathbb{Z}_{2} \times \mathbb{Z}_{360}$
        \item $\mathbb{Z}_{2} \times \mathbb{Z}_{2} \times \mathbb{Z}_{180}$
        \item $\mathbb{Z}_{2} \times \mathbb{Z}_{2} \times \mathbb{Z}_{2} \times \mathbb{Z}_{90}$
        \item $\mathbb{Z}_{3} \times \mathbb{Z}_{240}$
        \item $\mathbb{Z}_{12} \times \mathbb{Z}_{60}$
        \item $\mathbb{Z}_{6} \times \mathbb{Z}_{120}$
        \item $\mathbb{Z}_{2} \times \mathbb{Z}_{6} \times \mathbb{Z}_{60}$
        \item $\mathbb{Z}_{2} \times \mathbb{Z}_{2} \times \mathbb{Z}_{6} \times \mathbb{Z}_{30}$
    \end{enumerate}
\end{proof}

% section 9/exercise 25
\begin{exercise}
    Order 1089
\end{exercise}

\begin{proof}
    All abelian groups of order $1089 = 3^{2}11^{2}$, up to isomorphism are
    \begin{enumerate}
        \item $\mathbb{Z}_{1089}$
        \item $\mathbb{Z}_{3} \times \mathbb{Z}_{363}$
        \item $\mathbb{Z}_{11} \times \mathbb{Z}_{99}$
        \item $\mathbb{Z}_{33} \times \mathbb{Z}_{33}$
    \end{enumerate}
\end{proof}

% section 9/exercise 26
\begin{exercise}
    How many abelian groups (up to isomorphism) are there of order 24? of order 25? of order (24)\@(25)?
\end{exercise}

\begin{proof}
    There are 3 abelian groups of order 24 up to isomorphism. There are 2 abelian groups of order 25 up to isomorphism. There are 6 abelian groups of order (24)\@(25) up to isomorphism.
\end{proof}

% section 9/exercise 27
\begin{exercise}
    Following the idea suggested in Exercise 26, let $m$ and $n$ be relatively prime positive integers. Show that if there are (up to isomorphism) $r$ abelian groups of order $m$ and $s$ of order $n$, then there are (up to isomorphism) $rs$ abelian groups of order $mn$.
\end{exercise}

\begin{proof}
    We establish a bijection $f$ from
    \begin{itemize}
        \item the cartesian product $X\times Y$ of the set of abelian groups of order $m$ and the set of abelian groups of order $n$
        \item to the set $Z$ of abelian groups of order $mn$ and
    \end{itemize}

    If $A$ is an abelian group of order $r$, $B$ is an abelian group of order $s$, then the direct product $A\times B$ is an abelian group of order $rs$. Let's define $f: X\times Y \to Z$ as $(A, B) \mapsto A\times B$. According to the primary factor version of the fundamental theorem of finitely generated abelian group (the uniqueness), the fundamental theorem of arithmetics, and $m, n$ are relatively prime, we obtain that $f$ is a one-to-one mapping.

    Let $G$ be an abelian group of order $mn$. According to the primary factor version of the fundamental theorem of finitely generated abelian group, $G$ is isomorphic to a direct product of cyclic groups in the form
    \[
        \mathbb{Z}_{{(p_{1})}^{k_{1}}} \times \mathbb{Z}_{{(p_{2})}^{k_{2}}} \times \cdots \times \mathbb{Z}_{{(p_{e})}^{k_{e}}}
    \]

    where ${(p_{1})}^{k_{1}}{(p_{2})}^{k_{2}}\cdots {(p_{e})}^{k_{e}} = mn$.

    If $m, n$ are relatively prime, then a prime factor of $m$ is not a prime factor of $n$ and vice versa. Without loss of generality, suppose that $p_{1}, p_{2}, \ldots, p_{d}$ are prime factors of $m$ and $p_{d+1}, p_{d+2}, \ldots, p_{e}$ are prime factors of $n$. So
    \[
        \mathbb{Z}_{{(p_{1})}^{k_{1}}} \times \mathbb{Z}_{{(p_{2})}^{k_{2}}} \times \cdots \times \mathbb{Z}_{{(p_{d})}^{k_{d}}}
    \]

    is an abelian group of order $m$, and
    \[
        \mathbb{Z}_{{(p_{d+1})}^{k_{d+1}}} \times \mathbb{Z}_{{(p_{d+2})}^{k_{d+2}}} \times \cdots \times \mathbb{Z}_{{(p_{e})}^{k_{e}}}
    \]

    is an abelian group of order $n$. This implies $G$ can be decomposed into the direct product of an abelian group of order $m$ and an abelian group of order $n$. So $f$ is onto.

    Because $f$ is one-to-one and onto, we conclude that $X\times Y$ and $Z$ have the same number of elements, which means there are $rs$ abelian groups of order $mn$.
\end{proof}

% section 9/exercise 28
\begin{exercise}
    Use Exercise 27 to determine the number of abelian groups (up to isomorphism) of order ${(10)}^{5}$
\end{exercise}

\begin{proof}
    There are $7$ abelian groups of order $2^{5}$, up to isomorphism.
    \begin{enumerate}
        \item $\mathbb{Z}_{2^{5}}$
        \item $\mathbb{Z}_{2} \times \mathbb{Z}_{2^{4}}$
        \item $\mathbb{Z}_{2^{2}} \times \mathbb{Z}_{2^{3}}$
        \item $\mathbb{Z}_{2} \times \mathbb{Z}_{2} \times \mathbb{Z}_{2^{3}}$
        \item $\mathbb{Z}_{2} \times \mathbb{Z}_{2^{2}} \times \mathbb{Z}_{2^{2}}$
        \item $\mathbb{Z}_{2} \times \mathbb{Z}_{2} \times \mathbb{Z}_{2} \times \mathbb{Z}_{2^{2}}$
        \item $\mathbb{Z}_{2} \times \mathbb{Z}_{2} \times \mathbb{Z}_{2} \times \mathbb{Z}_{2} \times \mathbb{Z}_{2}$
    \end{enumerate}

    There are $7$ abelian groups of order $5^{5}$, up to isomorphism.
    \begin{enumerate}
        \item $\mathbb{Z}_{5^{5}}$
        \item $\mathbb{Z}_{5} \times \mathbb{Z}_{5^{4}}$
        \item $\mathbb{Z}_{5^{2}} \times \mathbb{Z}_{5^{3}}$
        \item $\mathbb{Z}_{5} \times \mathbb{Z}_{5} \times \mathbb{Z}_{5^{3}}$
        \item $\mathbb{Z}_{5} \times \mathbb{Z}_{5^{2}} \times \mathbb{Z}_{5^{2}}$
        \item $\mathbb{Z}_{5} \times \mathbb{Z}_{5} \times \mathbb{Z}_{5} \times \mathbb{Z}_{5^{2}}$
        \item $\mathbb{Z}_{5} \times \mathbb{Z}_{5} \times \mathbb{Z}_{5} \times \mathbb{Z}_{5} \times \mathbb{Z}_{5}$
    \end{enumerate}

    According to Exercise 27, there are $49$ abelian groups of order ${(10)}^{5}$, up to isomorphism.
\end{proof}

% section 9/exercise 29
\begin{exercise}
    \begin{enumerate}[label={\textbf{\alph*.}}]
        \item Let $p$ be a prime number. Fill in the second row of the table to give the number of abelian groups of order $p^{n}$ ($n$ is from $2$ to $8$), up to isomorphism.
        \item Let $p, q$, and $r$ be distinct prime numbers. Use the table you created to find the number of abelian groups, up to isomorphism, of the given order.
              \begin{enumerate}[label={\textbf{\roman*.}}]
                  \item $p^{3}q^{4}r^{7}$
                  \item ${(qr)}^{7}$
                  \item $q^{5}r^{4}q^{3}$
              \end{enumerate}
    \end{enumerate}
\end{exercise}

\begin{proof}
    \begin{enumerate}[label={\textbf{\alph*.}}]
        \item
              \[
                  \begin{array}{r|r|r|r|r|r|r|r|}
                      n                       & 2 & 3 & 4 & 5 & 6  & 7  & 8  \\
                      \hline
                      \text{number of groups} & 2 & 3 & 5 & 7 & 11 & 15 & 22
                  \end{array}
              \]
        \item
              \begin{enumerate}[label={\textbf{\roman*.}}]
                  \item There are $3\times 5\times 15 = 225$ abelian groups of order $p^{3}q^{4}r^{7}$.
                  \item There are $15\times 15 = 225$ abelian groups of order ${(qr)}^{7}$.
                  \item There are $22\times 5 = 110$ abelian groups of order $q^{5}r^{4}q^{3}$.
              \end{enumerate}
    \end{enumerate}
\end{proof}

% section 9/exercise 30
\begin{exercise}
    Indicate schematically a Cayley digraph for $\mathbb{Z}_{m}\times \mathbb{Z}_{n}$ for the generating set $S = \{ (1,0), (0,1) \}$.
\end{exercise}

\begin{proof}
    $\mathbb{Z}_{m}\times\mathbb{Z}_{n}$ has $mn$ elements. I draw a Cayley digraph for $\mathbb{Z}_{m}\times\mathbb{Z}_{n}$ as follow
    \begin{itemize}
        \item pick a point as $(0,0)$
        \item draw arcs of the 1st type (corresponding to $(1,0)$) consecutively from $(0,0)$ to obtain $(0,0) -> (1,0) -> \cdots -> (m-1,0) -> (0,0)$
        \item draw arcs of the 2nd type (corresponding to $(0,1)$) consecutively from $(0,0)$ to obtain $(0,0) -> (0,1) -> \cdots -> (0,n-1) -> (0,0)$
        \item draw arcs of the 1st type $(0,k) -> (1,k) -> \cdots -> (m-1,k) -> (0,k)$ for every $k\in\mathbb{Z}_{n}$
        \item draw arcs of the 2nd type $(k,0) -> (k,1) -> \cdots -> (k,n-1) -> (k,0)$ for every $k\in\mathbb{Z}_{m}$
    \end{itemize}

    The resulting Cayley digraph would have rectangular shape.
\end{proof}

% section 9/exercise 31
\begin{exercise}
    Consider Cayley digraphs with two arc types, a solid one with an arrow and a dashed one with no arrow, and consisting of two regular $n$-gons, for $n\geq 3$, with solid arc sides, one inside the other, with dashed arcs joining the vertices of the outer $n$-gon to the inner one. Figure 7.11(b) shows such a Cayley digraph with $n = 3$, and Figure 7.13(b) shows one with $n = 4$. The arrows on the outer $n$-gon may have the same (clockwise or counterclockwise) direction as those on the inner $n$-gon, or they may have the opposite direction. Let $G$ be a group with such a Cayley digraph.
    \begin{enumerate}[label={\textbf{\alph*.}}]
        \item Under what circumstances will $G$ be abelian?
        \item If $G$ is abelian, to what familiar group is it isomorphic?
        \item If $G$ is abelian, under what circumstances is it cyclic?
        \item If $G$ is not abelian, to what group we have discussed is it isomorphic?
    \end{enumerate}
\end{exercise}

\begin{proof}
    \begin{enumerate}[label={\textbf{\alph*.}}]
        \item $G$ is abelian if the two $n$-gons have the same direction.
        \item If $G$ is abelian, $G$ is isomorphic to $\mathbb{Z}_{2}\times\mathbb{Z}_{n}$.
        \item If $G$ is abelian, $G$ is cyclic if and only if $n$ is odd (because $\mathbb{Z}_{2}\times\mathbb{Z}_{n}$ is cyclic if and only if $n$ and $2$ are relatively prime, equivalently, $n$ is odd).
        \item If $G$ is not abelian, $G$ is isomorphic to the dihedreal group $D_{n}$.
    \end{enumerate}
\end{proof}

\subsection*{Concepts}

% section 9/exercise 32
\begin{exercise}
    Determine whether each of the following is true or false.
    \begin{enumerate}[label={\textbf{\alph*.}}]
        \item If $G_{1}$ and $G_{2}$ are any groups, then $G_{1} \times G_{2}$ is always isomorphic to $G_{2} \times G_{1}$.
        \item Computation in an external direct product of groups is easy if you know how to compute in each component group.
        \item Groups of finite order must be used to form an external direct product.
        \item A group of prime order could not be the internal direct product of two proper nontrivial subgroups.
        \item $\mathbb{Z}_{2}\times\mathbb{Z}_{4}$ is isomorphic to $\mathbb{Z}_{8}$.
        \item $\mathbb{Z}_{2}\times\mathbb{Z}_{4}$ is isomorphic to $\mathbb{S}_{8}$.
        \item $\mathbb{Z}_{3}\times\mathbb{Z}_{8}$ is isomorphic to $S_{4}$.
        \item Every element in $\mathbb{Z}_{4}\times\mathbb{Z}_{8}$ has order $8$.
        \item The order of $\mathbb{Z}_{12}\times\mathbb{Z}_{15}$ is $60$.
        \item $\mathbb{Z}_{m} \times \mathbb{Z}_{n}$ has $mn$ elements whether $m$ and $n$ are relatively prime or not.
    \end{enumerate}
\end{exercise}

\begin{proof}
    \begin{enumerate}[label={\textbf{\alph*.}}]
        \item True.
        \item Undeciable.
        \item False.
        \item True.
        \item False.
        \item False.
        \item False.
        \item False.
        \item False.
        \item True.
    \end{enumerate}
\end{proof}

% section 9/exercise 33
\begin{exercise}
    Give an example illustrating that not every non trivial abelian group is the internal direct product of two proper nontrivial subgroups.
\end{exercise}

\begin{proof}
    $\mathbb{Z}_{4}$ is not the internal direct product of two proper nontrivial subgroups.
\end{proof}

% section 9/exercise 34
\begin{exercise}
    \begin{enumerate}
        \item How many subgroups of $\mathbb{Z}_{5} \times \mathbb{Z}_{6}$ are isomorphic to $\mathbb{Z}_{5} \times \mathbb{Z}_{6}$?
        \item How many subgroups of $\mathbb{Z} \times \mathbb{Z}$ are isomorphic to $\mathbb{Z}\times \mathbb{Z}$?
    \end{enumerate}
\end{exercise}

\begin{proof}
    \begin{enumerate}
        \item Only one. It is $\mathbb{Z}_{5}\times\mathbb{Z}_{6}$. Other subgroups of $\mathbb{Z}_{5}\times\mathbb{Z}_{6}$ are proper subgroups, with less elements than $\mathbb{Z}_{5}\times\mathbb{Z}_{6}$.
        \item Infinitely many. Because $\mathbb{Z}\simeq n\mathbb{Z}$ for any nonzero integer $n$, then $m\mathbb{Z} \times n\mathbb{Z} \simeq \mathbb{Z}\times\mathbb{Z}$ for nonzero integers $m, n$.
    \end{enumerate}
\end{proof}

% section 9/exercise 35
\begin{exercise}
    Give an example of a nontrivial group that is not of prime order and is not the internal direct product of two nontrivial subgroups.
\end{exercise}

\begin{proof}
    $\mathbb{Z}_{p^{2}}$ where $p$ is a prime number. Because $\mathbb{Z}_{p^{2}} \not\simeq \mathbb{Z}_{p} \times \mathbb{Z}_{p}$.
\end{proof}

% section 9/exercise 36
\begin{exercise}
    Determine whether each of the following is true or false.
    \begin{enumerate}[label={\textbf{\arabic*.}}]
        \item Every abelian group of prime order is cyclic.
        \item Every abelian group of prime power order is cyclic.
        \item $\mathbb{Z}_{8}$ is generated by $\{ 4, 6 \}$.
        \item $\mathbb{Z}_{8}$ is generated by $\{ 4, 5, 6 \}$.
        \item All finite abelian groups are classified up to isomorphism by Theorem 9.12.
        \item Any two finitely generated abelian groups with the same Betti number are isomorphic.
        \item Every abelian group of order divisible by $5$ contains a cyclic subgroups of order $5$.
        \item Every abelian group of order divisible by $4$ contains a cyclic subgroup of order $4$.
        \item Every abelian group of order divisible by $6$ contains a cyclic subgroup of order $6$.
        \item Every finite abelian group has a Betti number of $0$.
    \end{enumerate}
\end{exercise}

\begin{proof}
    \begin{enumerate}[label={\textbf{\arabic*.}}]
        \item True.
        \item False. Example: Klein $4$-group.
        \item False. $1$ can not be generated by $\{ 4, 6 \}$ in $\mathbb{Z}_{8}$.
        \item True.
        \item True.
        \item False. $\mathbb{Z}_{2}\times\mathbb{Z}_{4} \not\simeq \mathbb{Z}_{8}$ but they have the same Betti number $0$.
        \item True.
        \item False. Example: Klein $4$-group.
        \item True. Because an abelian group of order divisible by $6$ contains a subgroup of order $6$. On the other hand, this subgroup is also abelian. Due to the primary factor version of the fundamental theorem of finitely generated abelian groups, such subgroups is isomorphic to $\mathbb{Z}_{2}\times\mathbb{Z}_{3}$, which is indeed a cyclic subgroup.
        \item True.
    \end{enumerate}
\end{proof}

% section 9/exercise 37
\begin{exercise}
    Let $p$ and $q$ be distinct prime numbers. How does the number (up to isomorphism) of abelian groups of order $p^{r}$ compare with the number (up to isomorphism) of abelian groups of order $q^{r}$?
\end{exercise}

\begin{proof}
    Every abelian group of order $p^{r}$ corresponds to a partition of $r$, in other words, a way to write $r$ as the sum of positive integers. The number of such ways is independent of $p$. So the number (up to isomorphism) of abelian groups of order $p^{r}$ is equal to the number (up to isomorphism) of abelian groups of order $q^{r}$.
\end{proof}

% section 9/exercise 38
\begin{exercise}
    Let $G$ be an abelian group of order $72$.
    \begin{enumerate}[label={\textbf{\alph*.}}]
        \item Can you say how many subgroups of order $8$ $G$ has? Why, or why not?
        \item Can you say how many subgroups of order $4$ $G$ has? Why, or why not?
    \end{enumerate}
\end{exercise}

\begin{proof}
    Up to isomorphism, abelian groups of order $72$ are
    \begin{align*}
         & \mathbb{Z}_{8} \times \mathbb{Z}_{9}                                                                   \\
         & \mathbb{Z}_{2} \times \mathbb{Z}_{4} \times \mathbb{Z}_{9}                                             \\
         & \mathbb{Z}_{2} \times \mathbb{Z}_{2} \times \mathbb{Z}_{2} \times \mathbb{Z}_{9}                       \\
         & \mathbb{Z}_{8} \times \mathbb{Z}_{3} \times \mathbb{Z}_{3}                                             \\
         & \mathbb{Z}_{2} \times \mathbb{Z}_{4} \times \mathbb{Z}_{3} \times \mathbb{Z}_{3}                       \\
         & \mathbb{Z}_{2} \times \mathbb{Z}_{2} \times \mathbb{Z}_{2} \times \mathbb{Z}_{3} \times \mathbb{Z}_{3} \\
    \end{align*}
    \begin{enumerate}[label={\textbf{\alph*.}}]
        \item In each case, there is only one subgroups of order $8$.
        \item No, can't tell. Because,

              $\mathbb{Z}_{2} \times \mathbb{Z}_{2} \times \mathbb{Z}_{2} \times \mathbb{Z}_{9}$ has three subgroups of order $4$; $\mathbb{Z}_{8} \times \mathbb{Z}_{9}$ has only one subgroup of order $4$.
    \end{enumerate}
\end{proof}

% section 9/exercise 39
\begin{exercise}
    Let $G$ be an abelian group. Show that the elements of finite order in $G$ form a subgroup. This subgroup is called the \textbf{torsion subgroup} of $G$
\end{exercise}

\begin{proof}
    Let $H$ be the subset of $G$ which contains precisely every element of finite order in $G$.

    The identity element $e$ of $G$ has order of $1$, so $e$ is in $H$. If $x, y$ is in $H$, then there exists positive integers $m, n$ such that $x^{m}y^{n} = e$, so ${(xy)}^{\text{lcm}(m, n)} = x^{\text{lcm}(m,n)}y^{\text{lcm}(m,n)} = e$, which means $xy$ has finite order. Therefore $H$ is closed under the induced operation. $x^{m} = e$, then ${(x^{-1})}^{m}x^{m} = x^{m}{(x^{-1})}^{m} = e$, so ${(x^{-1})}^{m} = e$. Therfore if $x$ has finite order, then $x^{-1}$ also has finite order, in other words, if $x$ is in $H$, then so is $x^{-1}$.

    Hence $H$ is a subgroup of $G$.
\end{proof}

Exercises 40 through 43 deal with the concept of the torsion subgroup just defined.

% section 9/exercise 40
\begin{exercise}
    Find the order of the torsion subgroup of $\mathbb{Z}_{4} \times \mathbb{Z} \times \mathbb{Z}_{3}$; of $\mathbb{Z}_{12} \times \mathbb{Z} \times \mathbb{Z}_{12}$.
\end{exercise}

\begin{proof}
    An element $(x, y, z)$ in $\mathbb{Z}_{4} \times \mathbb{Z} \times \mathbb{Z}_{3}$ has finite order if and only if $y = 0$. The torsion subgroup of $\mathbb{Z}_{4} \times \mathbb{Z} \times \mathbb{Z}_{3}$ is $\mathbb{Z}_{4} \times \{0\} \times \mathbb{Z}_{3}$. This tursion subgroup has order of $12$.

    An element $(x, y, z)$ in $\mathbb{Z}_{12} \times \mathbb{Z} \times \mathbb{Z}_{12}$ has finite order if and only if $y = 0$. The torsion subgroup of $\mathbb{Z}_{12} \times \mathbb{Z} \times \mathbb{Z}_{12}$ is $\mathbb{Z}_{12} \times \{0\} \times \mathbb{Z}_{12}$. This tursion subgroup has order of $144$.
\end{proof}

% section 9/exercise 41
\begin{exercise}
    Find the torsion subgroup of the multiplicative group $\mathbb{R}^{*}$ of nonzero real numbers.
\end{exercise}

\begin{proof}
    A nonzero real number $x$ has finite order in the multiplicative group $\mathbb{R}^{*}$ if and only if $x^{n} = 1$ for some positive integer $n$. So the torsion subgroup of the multiplicative group $\mathbb{R}^{*}$ is $\{ 1, -1 \}$.
\end{proof}

% section 9/exercise 42
\begin{exercise}
    Find the torsion subgroup $T$ of the multiplicative group $\mathbb{C}^{*}$ of nonzero complex numbers.
\end{exercise}

\begin{proof}
    A nonzero complex number $z$ has finite order in the multiplicative group $\mathbb{C}^{*}$ has finite order if and only if $z^{n} = 1$ for some positive integer $n$. So the torsion subgroup of the multiplicative group $\mathbb{C}^{*}$ is $\{ \cos(2\pi q) + i\sin(2\pi q) \mid q\in\mathbb{Q} \}$.
\end{proof}

% section 9/exercise 43
\begin{exercise}
    An abelian group is \textbf{torsion free} if $e$ is the only element of finite order. Use Theorem 9.12 to show that every finitely generated abelian group is the internal direct product of its torsion subgroup and of a torsion-free subgroup. (Note that $\{e\}$ may be the torsion subgroup, and is also torsion free.)
\end{exercise}

Actually, the proof of Theorem 9.12 uses this result.

\begin{proof}
    According to Theorem 9.12 (primary factor version of the fundamental theorem of finitely generated abelian groups), an abelian group $G$ is isomorphic to the direct product of cyclic groups in the form
    \[
        \mathbb{Z}_{{(p_{1})}^{r_{1}}} \times \mathbb{Z}_{{(p_{2})}^{r_{2}}} \times \cdots \times \mathbb{Z}_{{(p_{k})}^{r_{k}}} \times \mathbb{Z} \times \mathbb{Z} \times \cdots \times \mathbb{Z}
    \]

    Let
    \[
        A = \mathbb{Z}_{{(p_{1})}^{r_{1}}} \times \mathbb{Z}_{{(p_{2})}^{r_{2}}} \times \cdots \times \mathbb{Z}_{{(p_{k})}^{r_{k}}},\qquad B = \mathbb{Z} \times \mathbb{Z} \times \cdots \times \mathbb{Z}
    \]

    If $A = \{e\}$ (when $k = 0$), we consider $A$ a torsion subgroup. If $B = \{e\}$, we consider $B$ torsion free.

    Then $G = A\times B$. $A$ is the torsion subgroup of the internal direct product $A\times B$, $B$ is the torsion-free subgroup of the internal direct product $A\times B$. Thus $G$ is the internal direct product of its torsion subgroup and a torsion-free subgroup.
\end{proof}

% section 9/exercise 44
\begin{exercise}
    Find the torsion coefficients for each of the following groups.
    \begin{enumerate}[label={\textbf{\alph*.}}]
        \item $\mathbb{Z}_{2} \times \mathbb{Z}_{3} \times \mathbb{Z}_{4}$
        \item $\mathbb{Z}_{2} \times \mathbb{Z}_{4} \times \mathbb{Z}_{8} \times \mathbb{Z}_{3} \times \mathbb{Z}_{27}$
        \item $\mathbb{Z}_{8} \times \mathbb{Z}_{2} \times \mathbb{Z}_{49} \times \mathbb{Z}_{7}$
        \item $\mathbb{Z}_{2} \times \mathbb{Z}_{4} \times \mathbb{Z}_{2} \times \mathbb{Z}_{3} \times \mathbb{Z}_{3} \times \mathbb{Z}_{9} \times \mathbb{Z}_{5}$
    \end{enumerate}
\end{exercise}

\begin{proof}
    \begin{enumerate}[label={\textbf{\alph*.}}]
        \item The torsion coefficients of the given group are $1, 1, 2$.
        \item The torsion coefficients of the given group are $1, 2, 3, 1, 3$.
        \item The torsion coefficients of the given group are $3, 1, 2, 1$.
        \item The torsion coefficients of the given group are $1, 2, 1, 1, 1, 2, 1$.
    \end{enumerate}
\end{proof}

\subsection*{Proof Synopsis}

% section 9/exercise 45
\begin{exercise}
    Given a two-sentence synopsis of the proof of Theorem 9.5: The group $\mathbb{Z}_{m}\times \mathbb{Z}_{n}$ is cyclic and is isomorphic to $\mathbb{Z}_{mn}$ if and only if $m$ and $n$ are relatively prime, that is the gcd of $m$ and $n$ is $1$.
\end{exercise}

\begin{proof}
    If $m, n$ are relatively prime, then $mn$ is the least common multiple of $m$ and $n$, from which we can deduce that $(1,1)$ generates $\mathbb{Z}_{m}\times\mathbb{Z}_{n}$.

    Otherwise, if the greatest common divisor of $m$ and $n$ is $d > 1$, then the order of every element in $\mathbb{Z}_{m}\times\mathbb{Z}_{n}$ does not exceed $mn/d$, so $\mathbb{Z}_{m}\times\mathbb{Z}_{n}$ is not cyclic.
\end{proof}

\subsection*{Theory}

% section 9/exercise 46
\begin{exercise}
    Prove that a direct product of abelian groups is abelian.
\end{exercise}

\begin{proof}
    Let $G_{1}, G_{2}$ be abelian group, $x_{1}, y_{1}$ are elements of $G_{1}$, $x_{2}, y_{2}$ are elements of $G_{2}$. Then $(x_{1}, x_{2}), (y_{1}, y_{2})$ are elements of the direct product of $G_{1}$ and $G_{2}$.
    \begin{align*}
        (x_{1}, x_{2})(y_{1}, y_{2}) & = (x_{1}y_{1}, x_{2}y_{2})     \\
                                     & = (y_{1}x_{1}, y_{2}x_{2})     \\
                                     & = (y_{1}, y_{2})(x_{1}, x_{2})
    \end{align*}

    Thus the direct product of $G_{1}$ and $G_{2}$ is abelian.
\end{proof}

% section 9/exercise 47
\begin{exercise}
    Let $G$ be an abelian group. Let $H$ be the subset of $G$ consisting of the identity $e$ together with all elements of $G$ of order $2$. Show that $H$ is a subgroup of $G$.
\end{exercise}

\begin{proof}
    $H$ has at least one element $e$. If the order of $H$ is $1$, then $H$ is the trivial subgroup of $G$.

    If the order of $H$ is not $1$, $H$ has elements other than $e$. Let $x, y$ be elements of $H$ other than $e$. Then $x^{2} = y^{2} = e$.

    Because $G$ is abelian, ${(xy)}^{2} = x^{2}y^{2} = ee = e$. If $x\ne y$, then $e = x^{2} \ne xy$. So if $x\ne y$, the order of $xy$ is $2$; if $x = y$, $xy = e$. Therefore $H$ is closed under the group operation.

    ${(x^{-1})}^{2}x^{2} = x^{2}{(x^{-1})}^{2} = {(xx^{-1})}^{2} = e^{2} = e$ so ${(x^{-1})}^{2}$ is the inverse of $x^{2} = e$, which means ${(x^{-1})}^{2} = e$. Because $x\ne e$, then $x^{-1}\ne e$. Therefore, $x^{-1}$ has order of $2$ and $x^{-1}$ is also in $H$.

    Thus $H$ is a subgroup of $G$.
\end{proof}

% section 9/exercise 48
\begin{exercise}
    Following up the idea of Exercise 47 determine whether $H$ will always be a subgroup for every abelian group $G$ if $H$ consists of the identity $e$ together with all elements of $G$ of order $3$; of order $4$. For what positive integers $n$ will $H$ always be a subgroup for every abelian group $G$, if $H$ consists of the identity $e$ together with all elements of $G$ of order $n$? Compare with Exercise 54 of Section 5.
\end{exercise}

\begin{proof}
    In an abelian group $G$, the order of $x$ is $n$ if and only if the order of $x^{-1}$ is $n$. This is true because $x^{n}{(x^{-1})}^{n} = {(x^{-1})}^{n}x^{n} = e$.

    \textbf{Case 1.} $H$ consists of the identity $e$ together with all elements of $G$ of order $3$.

    Suppose that $H$ is other than the trivial subgroup of $G$, then there exists $x, y\ne H$ other than $e$ ($x, y$ are not necessarily distinct).

    ${(xy)}^{3} = x^{3}y^{3} = ee = e$. If $xy = e$ then $xy\in H$. If $xy\ne e$, then ${(xy)}^{2} \ne e$ (because $xy$ is the inverse of ${(xy)}^{2}$). If ${(xy)}^{2} = e$, then $xy = e\in H$ (because $e = {(xy)}^{3} = (xy){(xy)}^{2}$). If $xy\ne e$ and ${(xy)}^{2}\ne e$, then the order of $xy$ is $3$, which makes $xy$ an element of $H$.

    Together with the statement at the beginning of this proof, we conclude that $H$ is a subgroup of $G$.

    \textbf{Case 2.} $H$ consists of the identity $e$ together with all elements of $G$ of order $4$.

    $H$ is not necessarily a subgroup of $G$. Example: $G = \mathbb{Z}_{4}$, $H = \{ 0, 1, 3 \}$. $1 + 1 = 2\notin H$.

    \textbf{General question.} For what positive integers $n$ will $H$ always be a subgroup for every abelian group $G$?

    If $n = 1$, then $H$ is the trivial subgroup of $G$.

    If $n$ is an even composite. Choose $G = \mathbb{Z}_{n}$, then $0, 1, n - 1$ are elements of $H$ (of course, these are not all elements of $H$). But $1 + 1 = 2\notin H$ because the order of $2$ is less than or equal to $n/2$.

    If $n$ is an odd composite. Let $p$ be a prime divisor of $n$. Choose $G = \mathbb{Z}_{n}$, then $0, 2, n - 1$ are elements of $H$. But $\underbrace{2 + \cdots + 2}_{p}$ has order less than or equal to $n/p$, so $\underbrace{2 + \cdots + 2}_{p} \notin H$.

    If $n$ is a prime number. Suppose that $H$ is other than the trivial subgroup of $G$. Let $x, y$ be elements other than $e$ of $H$. ${(xy)}^{n} = x^{n}y^{n} = ee = e$. Let $k$ be the order of $xy$. Due to the division algorithm, there exists integer $q$ and $0\leq r < k$ such that $n = kq + r$. So $e = {(xy)}^{n} = {(xy)}^{kq + r} = {(xy)}^{r}$. Since $k$ is the smallest positive integer such that ${(xy)}^{k} = e$, then $r$ must be $0$, which means $k$ divides $n$. Because $n$ is a prime number, $k$ is either $1$ or $n$. If $k = n$, $xy\in H$. If $k = 1$, $xy = e\in H$. Therefore $H$ is a subgroup of $G$.

    Hence, for $n = 1$ or any prime number, $H$ is always a subgroup for every abelian group $G$.
\end{proof}

% section 9/exercise 49
\begin{exercise}
    Find a counterexample of Exercise 47 with the hypothesis that G is abelian omitted.
\end{exercise}

\begin{proof}
    Counterexample: The dihedral group $D_{3} = \{ \iota, \rho, \rho^{2}, \mu, \mu\rho, \mu\rho^{2} \}$. $H = \{ \iota, \mu, \mu\rho, \mu\rho^{2} \}$. $H$ is not a subgroup of $D_{3}$ because $H$ is not closed under the induced operation ($\mu\mu\rho = \rho \notin H$).
\end{proof}

Let $H$ and $K$ be subgroups of a group $G$. Exercises 50 and 51 ask you to establish necessary and sufficient criteria for $G$ to appear as the internal direct product of $H$ and $K$.

% section 9/exercise 50
\begin{exercise}
    Let $H$ and $K$ be groups and let $G = H \times K$. Recall that both $H$ and $K$ appear as subgroups of $G$ in a natural way. Show that these subgroups $H$ (actually $H \times \{e\}$) and $K$ (actually $\{e\} \times K$) have the following properties.
    \begin{enumerate}[label={\textbf{\alph*.}}]
        \item Every element of $G$ is of the form $hk$ for some $h\in H$ and $k\in K$.
        \item $hk = kh$ for all $h\in H$ and $k\in K$.
        \item $H\cap K = \{ e \}$
    \end{enumerate}
\end{exercise}

\begin{proof}
    \begin{enumerate}[label={\textbf{\alph*.}}]
        \item $(x, y)$ is an element of $G$, where $x\in H, y\in K$
              \[
                  (x, y) = (xe, ey) = {(x, e)}_{\in H\times\{e\}}{(e, y)}_{\in\{e\}\times K}
              \]

              Hence every element of $G$ is of the form $hk$ where $h\in H\times\{e\}$ and $k\in\{e\}\times K$.
        \item $hk = (h,e)(e,k) = (he,ek) = (h,k) = (eh,ke) = (e,k)(h,e) = kh$. So $hk = kh$ for every $h\in H$ and $k\in K$.
        \item Because $e\in H$ and $e\in K$ so $\{e\} \subseteq H\cap K$.

              If $(x,y)\in H$ and $K$, then $y = e$ (because $H$ is $H\times\{e\}$) and $x = e$ (because $K = \{e\}\times K$). So $(x,y) = e\in \{e\}$. Therefore $H\cap K\subseteq \{e\}$.

              Hence $H\cap K = \{e\}$.
    \end{enumerate}
\end{proof}

% section 9/exercise 51
\begin{exercise}
    Let $H$ and $K$ be subgroups of a group $G$ satisfying the three properties listed in the preceding exercise. Show that for each $g\in G$, the expression $g = hk$ for $h\in H$ and $k\in K$ is unique. Then let each $g$ be renamed $(h, k)$. Show that, under this renaming, $G$ becomes structurally identical (isomorphic) to $H\times K$.
\end{exercise}

\begin{proof}
    Due to property (a), suppose that $g = hk = h'k'$, where $h, h'\in H$ and $k, k'\in K$. Then $h^{-1}(hk)k^{-1} = e$, so $(h^{-1}h')(k'k^{-1}) = e$. Due to property (b), $(k'k^{-1})(h'h^{-1}) = e$. Therefore, $h^{-1}h'$ and $k'k^{-1}$ are inverses. Because $H, K$ are subgroups of $G$, then $h^{-1}h'$ is in $H$ and $K$, $k'k^{-1}$ is in $H$ and $K$. Due to property (c), $h^{-1}h' = k'k^{-1} = e$, so $h = h'$ and $k = k'$. From this, we conclude that the expression $g = hk$ for $h\in H$ and $k\in K$ is unique.

    We define the mapping $\phi: G\to H\times K$ as follows: for each $g\in G$, there exists uniquely $h\in H$, $k\in K$ such that $g = hk$, then we define $\phi(g) = (h, k)$. This mapping is well-defined. $\phi$ is also a one-to-one mapping, because $\phi(g_{1}) = (h_{1},k_{1}) = (h_{2},k_{2}) = \phi(g_{2})$ implies $h_{1} = h_{2}, k_{1} = k_{2}$ and $g_{1} = h_{1}k_{1} = h_{2}k_{2} = g_{2}$. $\phi$ is onto, because $\phi(hk) = (h,k)$ for each $(h,k)\in H\times K$. Let $g_{1} = h_{1}k_{1}$ and $g_{2} = h_{2}k_{2}$ where $h_{1}, h_{2}\in H$ and $k_{1}, k_{2}\in K$, then
    \begin{align*}
        \phi(g_{1}g_{2}) & = \phi(h_{1}k_{1}h_{2}k_{2})                           \\
                         & = \phi(h_{1}(k_{1}h_{2})k_{2})                         \\
                         & = \phi(h_{1}(h_{2}k_{1})k_{2}) & \text{(property (b))} \\
                         & = \phi(h_{1}h_{2} k_{1}k_{2})                          \\
                         & = (h_{1}h_{2}, k_{1}k_{2})                             \\
                         & = (h_{1}, k_{1})(h_{2}, k_{2})                         \\
                         & = \phi(g_{1})\phi(g_{2})
    \end{align*}

    Therefore $\phi$ is an isomorphism. Hence $G$ is isomorphic to $H\times K$.
\end{proof}

% section 9/exercise 52
\begin{exercise}
    Show that a finite abelian group is not cyclic if and only if it contains a subgroup isomorphic to $\mathbb{Z}_{p}\times\mathbb{Z}_{p}$ for some prime $p$.
\end{exercise}

The result in this exercise is the opposite of Theorem 9.20's.

\begin{proof}
    $\mathbb{Z}_{p}\times\mathbb{Z}_{p}$ is not cyclic because the order of each element is less than or equal to $p$, meanwhile the order of $\mathbb{Z}_{p}\times\mathbb{Z}_{p}$ is $p^{2}$. Because subgroups of a cyclic group is cyclic, so if a finite abelian group contains a subgroup isomorphic to $\mathbb{Z}_{p}\times\mathbb{Z}_{p}$ for some prime $p$ then it is not cyclic.

    Let $G$ be a finite abelian group of order $n$ and $G$ is not cyclic. Due to the primary factor version fundamental theorem of finitely generated abelian group, $G$ is isomorphic to a direct product of cyclic groups in the form
    \[
        \mathbb{Z}_{{(p_{1})}^{k_{1}}}
        \times
        \mathbb{Z}_{{(p_{2})}^{k_{2}}}
        \times
        \cdots
        \times
        \mathbb{Z}_{{(p_{r})}^{k_{r}}}
    \]

    where ${(p_{1})}^{k_{1}}{(p_{2})}^{k_{2}}\cdots {(p_{r})}^{k_{r}} = n$. Because $G$ is not cyclic, then $r\geq 2$. If $p_{1}, p_{2}, \ldots, p_{r}$ are pairwise distinct, $G \simeq \mathbb{Z}_{n}$, which means $G$ is cyclic. So there exists $1\leq i < j\leq r$ such that $p_{i} = p_{j}$. If we rearrange the above direct product, we obtain a group which is isomorphic to the original direct product, therefore we can suppose without loss of generality that $p_{1} = p_{2}$. $\anglebracket{{(p_{1})}^{k_{1} - 1}}$ is a subgroup of $\mathbb{Z}_{{(p_{1})}^{k_{1}}}$ and it has order $p_{1}$. $\anglebracket{{(p_{2})}^{k_{2} - 1}}$ is a subgroup of $\mathbb{Z}_{{(p_{2})}^{k_{2}}}$ and it has order $p_{2}$. Hence
    \[
        \mathbb{Z}_{p_{1}} \times \mathbb{Z}_{p_{1}} \simeq
        \anglebracket{{(p_{1})}^{k_{1} - 1}} \times \anglebracket{{(p_{2})}^{k_{2} - 1}}
    \]

    and
    \[
        \anglebracket{{(p_{1})}^{k_{1} - 1}}
        \times
        \anglebracket{{(p_{2})}^{k_{2} - 1}}
        \times
        \{0\}
        \cdots
        \times
        \{0\}
        \leq
        \mathbb{Z}_{{(p_{1})}^{k_{1}}}
        \times
        \mathbb{Z}_{{(p_{2})}^{k_{2}}}
        \times
        \cdots
        \times
        \mathbb{Z}_{{(p_{r})}^{k_{r}}} \simeq G
    \]

    So $G$ has a subgroup isomorphic to $\mathbb{Z}_{p_{1}} \times \mathbb{Z}_{p_{1}}$.

    Hence, a finite abelian group $G$ is not cyclic if and only if it contains a subgroup isomorphic to $\mathbb{Z}_{p}\times\mathbb{Z}_{p}$ for some prime $p$.
\end{proof}

% section 9/exercise 53
\begin{exercise}
    Prove that if a finite abelian group has order a power of a prime $p$, then the order of every element in the group is a power of $p$.
\end{exercise}

\begin{proof}
    Let $G$ be a finite abelian group has order $p^{n}$.

    If the order of $G$ is $1$, then $G$ has a single element $-$ the identity element, of which order is $1 = p^{0}$.

    If the order of $G$ is greater than $1$, then $n\geq 1$. According to the primary factor version of the fundamental theorem of finitely generated abelian group, $G$ is isomorphic to the direct product of cyclic groups in the form
    \[
        \mathbb{Z}_{p^{n_{1}}} \times \cdots \times \mathbb{Z}_{p^{n_{k}}}
    \]

    where $k\geq 1$ and $n_{1} + \cdots + n_{k} = n$. Let $(a_{1}, \ldots, a_{k})$ be an element of $\mathbb{Z}_{p^{n_{1}}} \times \cdots \times \mathbb{Z}_{p^{n_{k}}}$. The order of $a_{i}$ in $\mathbb{Z}_{p^{n_{i}}}$ is a power of $p$ (from Theorem 4.16). So the order of $(a_{1}, \ldots, a_{k})$ in $\mathbb{Z}_{p^{n_{1}}} \times \cdots \times \mathbb{Z}_{p^{n_{k}}}$ is the least common multiple of the order of $a_{i}$ in $\mathbb{Z}_{p^{n_{i}}}$. Therefore the order of $(a_{1}, \ldots, a_{k})$ in $\mathbb{Z}_{p^{n_{1}}} \times \cdots \times \mathbb{Z}_{p^{n_{k}}}$ is a power of $p$.

    Thus the order of every element in $G$ is a power of $p$.
\end{proof}

% section 9/exercise 54
\begin{exercise}
    Let $G, H$, and $K$ be finitely generated abelian groups. Show that if $G\times K$ is isomorphic to $H\times K$, then $G\simeq H$.
\end{exercise}

\begin{proof}
    According to the primary factor version of the fundamental theorem of finitely generated abelian group,
    \begin{itemize}
        \item there exist uniquely primes (not necessarily distinct) $p_{1}, \ldots, p_{a}$ and positive integers $r_{1}, \ldots, r_{a}$ and a nonnegative integer $n_{g}$ such that
              \[
                  G \simeq \mathbb{Z}_{{(p_{1})}^{r_{1}}} \times \cdots \times \mathbb{Z}_{{(p_{a})}^{r_{a}}} \times \mathbb{Z}^{n_{g}}
              \]
        \item there exist uniquely primes (not necessarily distinct) $q_{1}, \ldots, q_{b}$ and positive integers $s_{1}, \ldots, s_{b}$ and a nonnegative integer $n_{h}$ such that
              \[
                  H \simeq \mathbb{Z}_{{(q_{1})}^{s_{1}}} \times \cdots \times \mathbb{Z}_{{(q_{b})}^{s_{b}}} \times \mathbb{Z}^{n_{h}}
              \]
        \item there exist uniquely primes (not necessarily distinct) $x_{1}, \ldots, x_{c}$ and positive integers $t_{1}, \ldots, t_{c}$ and a nonnegative integer $n_{k}$ such that
              \[
                  K \simeq \mathbb{Z}_{{(x_{1})}^{t_{1}}} \times \cdots \times \mathbb{Z}_{{(x_{c})}^{t_{c}}} \times \mathbb{Z}^{n_{k}}
              \]
    \end{itemize}

    Because $G\times K \simeq H\times K$, we obtain that
    \[
        \mathbb{Z}_{{(p_{1})}^{r_{1}}} \times \cdots \times \mathbb{Z}_{{(p_{a})}^{r_{a}}} \times \mathbb{Z}_{{(x_{1})}^{t_{1}}} \times \cdots \times \mathbb{Z}_{{(x_{c})}^{t_{c}}} \times \mathbb{Z}^{n_{g} + n_{k}}
        \simeq
        \mathbb{Z}_{{(q_{1})}^{s_{1}}} \times \cdots \times \mathbb{Z}_{{(q_{b})}^{s_{b}}} \times \mathbb{Z}_{{(x_{1})}^{t_{1}}} \times \cdots \times \mathbb{Z}_{{(x_{c})}^{t_{c}}} \times \mathbb{Z}^{n_{h} + n_{k}}
    \]

    Once again, due to the uniqueness part of the primary factor version of the fundamental theorem of finitely generated abelian group, we conclude that $\mathbb{Z}_{{(p_{1})}^{r_{1}}} \times \cdots \times \mathbb{Z}_{{(p_{a})}^{r_{a}}}$ and $\mathbb{Z}_{{(q_{1})}^{s_{1}}} \times \cdots \times \mathbb{Z}_{{(q_{b})}^{s_{b}}}$ are identical, disregarding the order of the factor. Hence $G\simeq H$.
\end{proof}

% section 9/exercise 55
\begin{exercise}
    Using the notation of Theorem 9.14, prove that for any finite abelian group $G$, every cyclic subgroup of $G$ has order no more than $d_{k}$, the largest invariant factor for $G$.
\end{exercise}

\begin{proof}
    $G$ is a finite abelian group. According to the invariant factor version of the fundamental theorem of finitely generated abelian group, there exist uniquely positive integers $d_{1}, d_{2}, \ldots, d_{k}$ such that $d_{i}$ divides $d_{i+1}$ for $i = 1,\ldots,k-1$ and
    \[
        G \simeq \mathbb{Z}_{d_{1}} \times \mathbb{Z}_{d_{2}} \times \cdots \times \mathbb{Z}_{d_{k}}
    \]

    Let $\anglebracket{(a_{1}, a_{2}, \ldots, a_{k})}$ be a cyclic subgroup of $\mathbb{Z}_{d_{1}} \times \mathbb{Z}_{d_{2}} \times \cdots \times \mathbb{Z}_{d_{k}}$. Due to Theorem 4.16, the order of $a_{i}$ in $\mathbb{Z}_{d_{i}}$ is a divisor of $d_{i}$, for $i = 1,\ldots, k$. The order of $\anglebracket{(a_{1}, a_{2}, \ldots, a_{k})}$ is the least common multiple of the order of $a_{i}$ in $\mathbb{Z}_{d_{i}}$. But $d_{i}$ divides $d_{k}$ for every $i = 1,\ldots, k$, so the order of $\anglebracket{(a_{1}, a_{2}, \ldots, a_{k})}$ divides $d_{k}$.

    Thus the order of every cyclic subgroup of any finite abelian group $G$ is not greater than the largest invariant factor for $G$.
\end{proof}

\section{Cosets and the Theorem of Lagrange}

\subsection*{Computations}

% section 10/exercise 1
\begin{exercise}
    Find all cosets of the subgroup $4\mathbb{Z}$ of $\mathbb{Z}$.
\end{exercise}

\begin{proof}
    All cosets of the subgroup $4\mathbb{Z}$ of $\mathbb{Z}$ are
    \begin{enumerate}[label={(\arabic*)}]
        \item $4\mathbb{Z} = \{ \ldots, -16, -12, -8, -4, 0, 4, 8, 12, 16, \ldots \}$
        \item $1 + 4\mathbb{Z} = \{ \ldots, -15, -11, -7, -3, 1, 5, 9, 13, 17, \ldots \}$
        \item $2 + 4\mathbb{Z} = \{ \ldots, -14, -10, -6, -2, 2, 6, 10, 14, 18, \ldots \}$
        \item $3 + 4\mathbb{Z} = \{ \ldots, -13, -9, -5, -1, 3, 7, 11, 15, 19, \ldots \}$
    \end{enumerate}
\end{proof}

% section 10/exercise 2
\begin{exercise}
    Find all cosets of the subgroup $4\mathbb{Z}$ of $2\mathbb{Z}$.
\end{exercise}

\begin{proof}
    All cosets of the subgroup $4\mathbb{Z}$ of $2\mathbb{Z}$ are
    \begin{enumerate}[label={(\arabic*)}]
        \item $4\mathbb{Z} = \{ \ldots, -8, -4, 0, 4, 8, \ldots \}$
        \item $2 + \mathbb{Z} = \{ \ldots, -6, -2, 2, 6, 10, \ldots \}$
    \end{enumerate}
\end{proof}

% section 10/exercise 3
\begin{exercise}
    Find all cosets of the subgroup $\anglebracket{3}$ in $\mathbb{Z}_{18}$.
\end{exercise}

\begin{proof}
    $\anglebracket{3} = \{ 0, 3, 6, 9, 12, 15 \}$. All cosets of the subgroup $\anglebracket{3}$ in $\mathbb{Z}_{18}$ are
    \begin{enumerate}[label={(\arabic*)}]
        \item $\anglebracket{3} = \{ 0, 3, 6, 9, 12, 15 \}$
        \item $1 + \anglebracket{3} = \{ 1, 4, 7, 10, 13, 16 \}$
        \item $2 + \anglebracket{3} = \{ 2, 5, 8, 11, 14, 17 \}$
    \end{enumerate}
\end{proof}

% section 10/exercise 4
\begin{exercise}
    Find all cosets of the subgroup $\anglebracket{6}$ in $\mathbb{Z}_{18}$.
\end{exercise}

\begin{proof}
    $\anglebracket{6} = \{ 0, 6, 12 \}$. All cosets of the subgroup $\anglebracket{6}$ in $\mathbb{Z}_{18}$ are
    \begin{enumerate}[label={(\arabic*)}]
        \item $\anglebracket{6} = \{ 0, 6, 12 \}$
        \item $1 + \anglebracket{6} = \{ 1, 7, 13 \}$
        \item $2 + \anglebracket{6} = \{ 2, 8, 14 \}$
        \item $3 + \anglebracket{6} = \{ 3, 9, 15 \}$
        \item $4 + \anglebracket{6} = \{ 4, 10, 16 \}$
        \item $5 + \anglebracket{6} = \{ 5, 11, 17 \}$
    \end{enumerate}
\end{proof}

% section 10/exercise 5
\begin{exercise}
    Find all cosets of the subgroup $\anglebracket{18}$ of $\mathbb{Z}_{36}$.
\end{exercise}

\begin{proof}
    $\anglebracket{18} = \{ 0, 18 \}$. All cosets of the subgroup $\anglebracket{18}$ in $\mathbb{Z}_{36}$ are
    \begin{enumerate}[label={(\arabic*)}]
        \item $\anglebracket{18} = \{ 0, 18 \}$
        \item $1 + \anglebracket{18} = \{ 1, 19 \}$
        \item $2 + \anglebracket{18} = \{ 2, 20 \}$
        \item $3 + \anglebracket{18} = \{ 3, 21 \}$
        \item $4 + \anglebracket{18} = \{ 4, 22 \}$
        \item $5 + \anglebracket{18} = \{ 5, 23 \}$
        \item $6 + \anglebracket{18} = \{ 6, 24 \}$
        \item $7 + \anglebracket{18} = \{ 7, 25 \}$
        \item $8 + \anglebracket{18} = \{ 8, 26 \}$
        \item $9 + \anglebracket{18} = \{ 9, 27 \}$
        \item $10 + \anglebracket{18} = \{ 10, 28 \}$
        \item $11 + \anglebracket{18} = \{ 11, 29 \}$
        \item $12 + \anglebracket{18} = \{ 12, 30 \}$
        \item $13 + \anglebracket{18} = \{ 13, 31 \}$
        \item $14 + \anglebracket{18} = \{ 14, 32 \}$
        \item $15 + \anglebracket{18} = \{ 15, 33 \}$
        \item $16 + \anglebracket{18} = \{ 16, 34 \}$
        \item $17 + \anglebracket{18} = \{ 17, 35 \}$
    \end{enumerate}
\end{proof}

% section 10/exercise 6
\begin{exercise}
    Find all left cosets of $\anglebracket{\mu\rho}$ in $D_{4}$.
\end{exercise}

\begin{proof}
    All left cosets of $\anglebracket{\mu\rho}$ in $D_{4}$ are
    \begin{enumerate}[label={(\arabic*)}]
        \item $\iota\anglebracket{\mu\rho} = \{ \iota, \mu\rho \}$
        \item $\rho\anglebracket{\mu\rho} = \{ \rho, \mu \}$
        \item $\rho^{2}\anglebracket{\mu\rho} = \{ \rho^{2}, \mu\rho^{3} \}$
        \item $\rho^{3}\anglebracket{\mu\rho} = \{ \rho^{3}, \mu\rho^{2} \}$
    \end{enumerate}
\end{proof}

% section 10/exercise 7
\begin{exercise}
    Repeating the preceding exercise, but find the right cosets this time. Are they the same as the left cosets?
\end{exercise}

\begin{proof}
    All right cosets of $\anglebracket{\mu\rho}$ in $D_{4}$ are
    \begin{enumerate}[label={(\arabic*)}]
        \item $\anglebracket{\mu\rho}\iota = \{ \iota, \mu\rho \}$
        \item $\anglebracket{\mu\rho}\rho = \{ \rho, \mu\rho^{2} \}$
        \item $\anglebracket{\mu\rho}\rho^{2} = \{ \rho^{2}, \mu\rho^{3} \}$
        \item $\anglebracket{\mu\rho}\rho^{3} = \{ \rho^{3}, \mu \}$
    \end{enumerate}

    The left cosets and the right cosets of $D_{4}$ are not the same. But the left and right cosets of $\anglebracket{\mu\rho}$ containing $\iota$ are identical, the left and right cosets of $\anglebracket{\mu\rho}$ containing $\rho^{2}$ are identical.
\end{proof}

% section 10/exercise 8
\begin{exercise}
    Are the left and right cosets the same for the subgroup $\{ \iota, \rho^{4}, \mu, \mu\rho^{4} \}$ of $D_{8}$? If so, display the cosets. If not, find a left coset that is not the same as any right coset.
\end{exercise}

\begin{proof}
    All left cosets of $\{ \iota, \rho^{4}, \mu, \mu\rho^{4} \}$ in $D_{8}$ are
    \begin{enumerate}[label={(\arabic*)}]
        \item $\iota\{ \iota, \rho^{4}, \mu, \mu\rho^{4} \} = \{ \iota, \rho^{4}, \mu, \mu\rho^{4} \}$
        \item $\rho\{ \iota, \rho^{4}, \mu, \mu\rho^{4} \} = \{ \rho, \rho^{5}, \mu\rho^{7}, \mu\rho^{3} \}$
        \item $\rho^{2}\{ \iota, \rho^{4}, \mu, \mu\rho^{4} \} = \{ \rho^{2}, \rho^{6}, \mu\rho^{6}, \mu\rho^{2} \}$
        \item $\rho^{3}\{ \iota, \rho^{4}, \mu, \mu\rho^{4} \} = \{ \rho^{3}, \rho^{7}, \mu\rho^{5}, \mu\rho \}$
    \end{enumerate}

    All right cosets of $\{ \iota, \rho^{4}, \mu, \mu\rho^{4} \}$ in $D_{8}$ are
    \begin{enumerate}[label={(\arabic*)}]
        \item $\{ \iota, \rho^{4}, \mu, \mu\rho^{4} \}\iota = \{ \iota, \rho^{4}, \mu, \mu\rho^{4} \}$
        \item $\{ \iota, \rho^{4}, \mu, \mu\rho^{4} \}\rho = \{ \rho, \rho^{5}, \mu\rho, \mu\rho^{5} \}$
        \item $\{ \iota, \rho^{4}, \mu, \mu\rho^{4} \}\rho^{2} = \{ \rho^{2}, \rho^{6}, \mu\rho^{2}, \mu\rho^{6} \}$
        \item $\{ \iota, \rho^{4}, \mu, \mu\rho^{4} \}\rho^{3} = \{ \rho^{3}, \rho^{7}, \mu\rho^{3}, \mu\rho^{7} \}$
    \end{enumerate}

    The left and right cosets are not the same for the subgroup $\{ \iota, \rho^{4}, \mu, \mu\rho^{4} \}$ of $D_{8}$. The left coset $\rho\{ \iota, \rho^{4}, \mu, \mu\rho^{4} \}$ is not the same as any right coset.
\end{proof}

% section 10/exercise 9
\begin{exercise}
    Find all left cosets of $\anglebracket{\rho^{2}} \leq D_{4}$.
\end{exercise}

\begin{proof}
    All left cosets of $\anglebracket{\rho^{2}} \leq D_{4}$ are
    \begin{enumerate}[label={(\arabic*)}]
        \item $\iota\anglebracket{\rho^{2}} = \{ \iota, \rho^{2} \}$
        \item $\rho\anglebracket{\rho^{2}} = \{ \rho, \rho^{3} \}$
        \item $\mu\anglebracket{\rho^{2}} = \{ \mu, \mu\rho^{2} \}$
        \item $\mu\rho\anglebracket{\rho^{2}} = \{ \mu\rho, \mu\rho^{3} \}$
    \end{enumerate}
\end{proof}

% section 10/exercise 10
\begin{exercise}
    Repeat the previous exercise, but find the right cosets. Are the left and right cosets the same? If so, make the group table for $D_{4}$, ordering the elements so that the cosets are in blocks, see if the blocks form a group with four elements, and determine what group of order 4 the blocks form.
\end{exercise}

\begin{proof}
    All right cosets of $\anglebracket{\rho^{2}} \leq D_{4}$ are
    \begin{enumerate}[label={(\arabic*)}]
        \item $\anglebracket{\rho^{2}}\iota = \{ \iota, \rho^{2} \}$
        \item $\anglebracket{\rho^{2}}\rho = \{ \rho, \rho^{3} \}$
        \item $\anglebracket{\rho^{2}}\mu = \{ \mu, \mu\rho^{2} \}$
        \item $\anglebracket{\rho^{2}}\mu\rho = \{ \mu\rho, \mu\rho^{3} \}$
    \end{enumerate}

    In this case, the left and right cosets are the same.
    \[
        \begin{array}{c||c|c||c|c||c|c||c|c}
                        & \iota       & \rho^{2}    & \rho        & \rho^{3}    & \mu         & \mu\rho^{2} & \mu\rho     & \mu\rho^{3} \\
            \hline\hline
            \iota       & \iota       & \rho^{2}    & \rho        & \rho^{3}    & \mu         & \mu\rho^{2} & \mu\rho     & \mu\rho^{3} \\
            \hline
            \rho^{2}    & \rho^{2}    & \iota       & \rho^{3}    & \rho        & \mu\rho^{2} & \mu         & \mu\rho^{3} & \mu\rho     \\
            \hline\hline
            \rho        & \rho        & \rho^{3}    & \rho^{2}    & \iota       & \mu\rho^{3} & \mu\rho     & \mu         & \mu\rho^{2} \\
            \hline
            \rho^{3}    & \rho^{3}    & \rho        & \iota       & \rho^{2}    & \mu\rho     & \mu\rho^{3} & \mu\rho^{2} & \mu         \\
            \hline\hline
            \mu         & \mu         & \mu\rho^{2} & \mu\rho     & \mu\rho^{3} & \iota       & \rho^{2}    & \rho        & \rho^{3}    \\
            \hline
            \mu\rho^{2} & \mu\rho^{2} & \mu         & \mu\rho^{3} & \mu\rho     & \rho^{2}    & \iota       & \rho^{3}    & \rho        \\
            \hline\hline
            \mu\rho     & \mu\rho     & \mu\rho^{3} & \mu\rho^{2} & \mu         & \rho^{3}    & \rho        & \iota       & \rho^{2}    \\
            \hline
            \mu\rho^{3} & \mu\rho^{3} & \mu\rho     & \mu         & \mu\rho^{2} & \rho        & \rho^{3}    & \rho^{2}    & \iota       \\
        \end{array}
    \]

    The group of order 4 the blocks form is isomorphic to the Klein 4-group.
\end{proof}

% section 10/exercise 11
\begin{exercise}
    Find the index of $\anglebracket{\rho^{2}}$ in the group $D_{6}$.
\end{exercise}

\begin{proof}
    $\anglebracket{\rho^{2}} = \{ \iota, \rho^{2}, \rho^{4} \}$ has 3 elements. $D_{6}$ has 12 elements. According to the definition of index in the case of finite groups, the index of $\anglebracket{\rho^{2}}$ in $D_{6}$ is 4.
\end{proof}

% section 10/exercise 12
\begin{exercise}
    Find the index of $\anglebracket{3}$ in the group $\mathbb{Z}_{24}$.
\end{exercise}

\begin{proof}
    $\anglebracket{3} = \{ 0, 3, 6, 9, 12, 15, 18, 21 \}$ has 8 elements. $\mathbb{Z}_{24}$ has 24 elements. According to the definition of index in the case of finite groups, the index of $\anglebracket{3}$ in $\mathbb{Z}_{24}$ is 3.
\end{proof}

% section 10/exercise 13
\begin{exercise}
    Find the index of $12\mathbb{Z}$ in $\mathbb{Z}$.
\end{exercise}

\begin{proof}
    All left cosets of $12\mathbb{Z}$ in $\mathbb{Z}$ are
    \begin{enumerate}[label={(\arabic*)}]
        \item $12\mathbb{Z}$
        \item $1 + 12\mathbb{Z}$
        \item $2 + 12\mathbb{Z}$
        \item $3 + 12\mathbb{Z}$
        \item $4 + 12\mathbb{Z}$
        \item $5 + 12\mathbb{Z}$
        \item $6 + 12\mathbb{Z}$
        \item $7 + 12\mathbb{Z}$
        \item $8 + 12\mathbb{Z}$
        \item $9 + 12\mathbb{Z}$
        \item $10 + 12\mathbb{Z}$
        \item $11 + 12\mathbb{Z}$
    \end{enumerate}

    Hence the index of $12\mathbb{Z}$ in $\mathbb{Z}$ is 12.
\end{proof}

% section 10/exercise 14
\begin{exercise}
    Find the index of $12\mathbb{Z}$ in $3\mathbb{Z}$.
\end{exercise}

\begin{proof}
    All left cosets of $12\mathbb{Z}$ in $3\mathbb{Z}$ are
    \begin{enumerate}[label={(\arabic*)}]
        \item $12\mathbb{Z}$
        \item $3 + 12\mathbb{Z}$
        \item $6 + 12\mathbb{Z}$
        \item $9 + 12\mathbb{Z}$
    \end{enumerate}

    Thus the index of $12\mathbb{Z}$ in $3\mathbb{Z}$ is 4.
\end{proof}

% section 10/exercise 15
\begin{exercise}
    Let $\sigma = (1,2,5,4)(2,3)$ in $S_{5}$. Find the index of $\anglebracket{\sigma}$ in $S_{5}$.
\end{exercise}

\begin{proof}
    $\sigma = (1,2,5,4)(2,3) = (1,2,3,5,4)$. The order of $\sigma$ in $S_{5}$ is 5. So the index of $\anglebracket{\sigma}$ in $S_{5}$ is $5!/5 = 24$.
\end{proof}

% section 10/exercise 16
\begin{exercise}
    Let $\mu = (1,2,4,5)(3,6)$ in $S_{6}$. Find the index of $\anglebracket{\mu}$ in $S_{6}$.
\end{exercise}

\begin{proof}
    The order of $(1,2,4,5)$ in $S_{6}$ is 4. The order of $(3,6)$ in $S_{6}$ is 2. So the order of $\mu$ in $S_{6}$ is the least common multiple of 4 and 2, which is 4. So the index of $\anglebracket{\mu}$ in $S_{6}$ is $6!/4 = 180$.
\end{proof}

\subsection*{Concepts}

In Exercises 17 through 19, correct the definition of the italicized term without reference to the text, if correction is needed, so that it is in a form acceptable for publication.

% section 10/exercise 17
\begin{exercise}
    Let $G$ be a group and let $H\subseteq G$. The \textit{left coset of $H$ containing $a$} is $aH = \{ ah \mid h\in H \}$.
\end{exercise}

\begin{proof}
    Correction: Let $G$ be a group and let $H\leq G$. The \textit{left coset of $H$ containing $a$} is $aH = \{ ah \mid h\in H \}$.
\end{proof}

% section 10/exercise 18
\begin{exercise}
    Let $G$ be a group and let $H\leq G$. The \textit{index of $H$ in $G$} is the number of right cosets of $H$ in $G$.
\end{exercise}

\begin{proof}
    The definition is correct.
\end{proof}

% section 10/exercise 19
\begin{exercise}
    Let $\phi: G\to G'$. Then the \textit{kernel} of $\phi$ is $\ker\phi = \{ g \in G \mid \phi(g) = e \}$
\end{exercise}

\begin{proof}
    Correction: Let $\phi: G\to G'$ be a group homomorphism. Then the \textit{kernel} of $\phi$ is $\ker\phi = \{ g \in G \mid \phi(g) = e' \}$ where $e'$ is the identity element of $G'$.
\end{proof}

% section 10/exercise 20
\begin{exercise}
    Determine whether each of the following is true or false.
    \begin{enumerate}[label={\textbf{\alph*.}}]
        \item Every subgroup of every group has left cosets.
        \item The number of left cosets of a subgroup of a finite group divides the order of the group.
        \item Every group of prime order is abelian.
        \item One cannot have left cosets of a finite subgroup of an infinite group.
        \item A subgroup of a group is a left coset of itself.
        \item Only subgroups of finite groups can have left cosets.
        \item $A_{n}$ is of index 2 in $S_{n}$ for $n > 1$.
        \item The theorem of Lagrange is a nice result.
        \item Every finite group contains an element of every order that divides the order of the group.
        \item Every finite cyclic group contains an element of every order that divides the order of the group.
        \item The kernel of a homomorphism is a subgroup of the range of the homomorphism.
        \item Left cosets and right cosets of the kernel of a homomorphism are the same.
    \end{enumerate}
\end{exercise}

\begin{proof}
    \begin{enumerate}[label={\textbf{\alph*.}}]
        \item True.
        \item True.
        \item True. Every group of prime order is cyclic, therefore abelian.
        \item False.
        \item True.
        \item False.
        \item True. Because $\abs{S_{n}} = n!{}$ and $\abs{A_{n}} = n!/2$ for $n > 1$.
        \item Undeciable, because this statement is subjective. But in my opinion, it is a really nice result.
        \item False. Example: In $D_{3}$, none of its element has order of 6.
        \item True. Because cyclic group is abelian, and a finite abelian group $G$ has subgroup of any order which divides the order of $G$, and every subgroup of a cyclic group is cyclic.
        \item False.
        \item True.
    \end{enumerate}
\end{proof}

In Exercises 21 through 26, give an example of the desired subgroup and group if possible. If impossible, say why
it is impossible.

% section 10/exercise 21
\begin{exercise}
    A subgroup $H\leq G$ with $G$ infinite and $H$ having only a finite number of left cosets in $G$
\end{exercise}

\begin{proof}
    Example: $2\mathbb{Z} \leq \mathbb{Z}$. $2\mathbb{Z}$ has exactly 2 left cosets in $\mathbb{Z}$.
\end{proof}

% section 10/exercise 22
\begin{exercise}
    A subgroup of an abelian group $G$ whose left cosets and right cosets give different partition of $G$.
\end{exercise}

\begin{proof}
    Impossible. Since $G$ is abelian, $aH = Ha$ for every subgroup $H\leq G$ and $a\in G$.
\end{proof}

% section 10/exercise 23
\begin{exercise}
    A subgroup of a group $G$ whose left cosets give a partition of $G$ into just one cell
\end{exercise}

\begin{proof}
    For any group $G$, $G$ is its improper subgroup. $G$ is the only left coset of $G$, within $G$.
\end{proof}

% section 10/exercise 24
\begin{exercise}
    A subgroup of a group of order 6 whose left cosets give a partition of the group into 6 cells.
\end{exercise}

\begin{proof}
    Example: $\mathbb{Z}_{6}$ and its trivial subgroup $\{0\}$. $\{0\}$ has 6 cosets, which are $\{0\}$, $\{1\}$, $\{2\}$, $\{3\}$, $\{4\}$, $\{5\}$.
\end{proof}

% section 10/exercise 25
\begin{exercise}
    A subgroup of a group of order 6 whose left cosets give a partition of the group into 12 cells
\end{exercise}

\begin{proof}
    Impossible. Because 12 is larger than 6. A subgroup of a group of order 6 can have at most 6 left cosets, due to Lagrange's theorem.
\end{proof}

% section 10/exercise 26
\begin{exercise}
    A subgroup of a group of order 6 whose left cosets give a partition of the group into 4 cells
\end{exercise}

\begin{proof}
    Impossible. Because 4 does not divide 6. According to the Lagrange's theorem, the number of left cosets of a subgroup of a finite group $G$ must divide the order of $G$.
\end{proof}

\subsection*{Proof Synopsis}

% section 10/exercise 27
\begin{exercise}
    Give a one-sentence synopsis of the proof of the Theorem of Lagrange.
\end{exercise}

\begin{proof}
    All left cosets of a subgroup of a finite group constitute a partion of the given finite group and those left cosets have the same number of elements, so the order of the subgroup divides the order of the group.
\end{proof}

\subsection*{Theory}

% section 10/exercise 28
\begin{exercise}
    Prove that the relation ${\sim}_{R}$ that is used to define right cosets is an equivalence relation.
\end{exercise}

\begin{proof}
    $H$ is a subgroup of $G$, $e$ is the identity element of $G$.

    For $a\in G$, $aa^{-1} = e\in H$ (because $H$ is a subgroup), so $a {\sim}_{R} a$.

    If $a {\sim}_{R} b$, then $ab^{-1}\in H$. It follows that its inverse ${(ab^{-1})} = ba^{-1}$ is also in $H$. So $a {\sim}_{R} b$ implies $b {\sim}_{R} a$. Analogously, $b {\sim}_{R} a$ implies $a {\sim}_{R} b$.

    If $a {\sim}_{R} b$ and $b {\sim}_{R} c$ implies $ab^{-1}, bc^{-1}\in H$. So $(ab^{-1})(bc^{-1}) = ac^{-1}\in H$. Therefore $a {\sim}_{R} b$.

    Hence ${\sim}_{R}$ is reflexive, symmetric, and transitive, which implies ${\sim}_{R}$ is an equivalence relation.
\end{proof}

% section 10/exercise 29
\begin{exercise}
    Let $H$ be a subgroup of a group $G$ and let $g\in G$. Define a one-to-one map of $H$ onto $Hg$. Prove that your map is one-to-one and is onto $Hg$.
\end{exercise}

\begin{proof}
    Let $\phi: H\to Hg$ be the map defined as $\phi(x) = xg$ for every $x\in G$. If $ag = bg$ ($ag, bg$ are elements of $Hg$), then $a = b$ due to the right cancellation law. So $\phi$ is one-to-one. On the other hand, every element in $Hg$ is of the form $xg$ where $x\in H$, so $\phi(x) = xg$, which means $\phi$ is onto $Hg$. Hence the defined map $\phi$ is a one-to-one map from $H$ onto $Hg$.
\end{proof}

% section 10/exercise 30
\begin{exercise}
    Let $H$ be a subgroup of a group $G$ such that $g^{-1}hg\in H$ for all $g\in G$ and all $h\in H$. Show that every left coset $gH$ is the same as the right coset $Hg$.
\end{exercise}

\begin{proof}
    An element of the left coset $gH$ is of the form $gh$, where $h\in H$. Because $ghg^{-1}\in H$, then $gh = gh(g^{-1}g) = (ghg^{-1})g \in Hg$. Therefore every element of $gH$ is also an element of $Hg$, so $gH \subseteq Hg$.

    An element of the right coset $Hg$ is of the form $hg$, where $h\in H$. Since $g^{-1}hg\in H$, then $hg = (gg^{-1})hg = g(g^{-1}hg)\in gH$. It follows that every element of $Hg$ is also an element of $gH$, so $Hg \subseteq gH$.

    Hence $gH = Hg$ if $g^{-1}hg\in H$ for all $g\in G$ and $h\in H$.
\end{proof}

% section 10/exercise 31
\begin{exercise}
    Let $H$ be a subgroup of a group $G$. Prove that if the partition of $G$ into left cosets of $H$ is the same as the partition into right cosets of $H$, then $g^{-1}hg$ for all $g\in G$ and all $h\in H$. (Note that this is the converse of Exercise 30.)
\end{exercise}

\begin{proof}
    Because the left cosets of $H$ partition $G$ the same as the right cosets of $H$, each left coset of $H$ is some right coset of $H$.

    Suppose that the left coset $aH$ and the right coset $Hg$ are identical. Let $h$ be an element of $H$. Because $g\in Hg$, so $g\in aH$, and there exists $x\in H$ such that $g = ax$, so $g^{-1} = x^{-1}a^{-1}$. On the other hand, $hg\in Hg$, so $hg\in aH$, and there exists $y\in H$ such that $hg = ay$. So $g^{-1}hg = (g^{-1}h)g = (x^{-1}a^{-1})(ay) = x^{-1}y$. Since $x^{-1}y\in H$, it follows that $g^{-1}hg\in H$, for every $g\in G$, $h\in H$.
\end{proof}

Let $H$ be a subgroup of a group $G$ and let $a, b\in G$. In Exercises 32 through 35 prove the statement or give a counterexample.

% section 10/exercise 32
\begin{exercise}
    If $aH = bH$, then $Ha = Hb$.
\end{exercise}

\begin{proof}
    The statement is false.

    Counterexample: $\{ \iota, \mu\rho \}$ is a subgroup of $D_{4}$.

    The following left cosets are identical: $\mu\{ \iota, \mu\rho \} = \{ \mu, \rho \}$ and $\rho \{ \iota, \mu\rho \} = \{ \rho, \mu \}$.

    But the corresponding right cosets are different: $\{ \iota, \mu\rho \}\mu = \{ \mu, \rho^{3} \}$ and $\{ \iota, \mu\rho \}\rho = \{ \rho, \mu\rho^{2} \}$.
\end{proof}

% section 10/exercise 33
\begin{exercise}
    If $Ha = Hb$, then $b\in Ha$.
\end{exercise}

\begin{proof}
    The statement is true.

    Let $e$ be the identity element of $G$, then $e$ is also the identity element of $H$. Since $b = eb\in Hb$ and $Hb = Ha$, we conclude that $b\in Ha$.
\end{proof}

% section 10/exercise 34
\begin{exercise}
    If $aH = bH$, then $Ha^{-1} = Hb^{-1}$.
\end{exercise}

\begin{proof}
    The statement is true.

    If $aH = bH$, then $a\in bH, b\in aH$, and there exist $x, y\in H$ such that $a = bx, b = ay$. Let $h$ be an element in $H$.
    \[
        \begin{split}
            ha^{-1} = ha^{-1}(bb^{-1}) = (ha^{-1}b)b^{-1} = (hy)b^{-1} \in Hb^{-1}, \\
            hb^{-1} = hb^{-1}(aa^{-1}) = (hb^{-1}a)a^{-1} = (hx)a^{-1} \in Ha^{-1}.
        \end{split}
    \]

    So $Ha^{-1} \subseteq Hb^{-1}$ and $Hb^{-1} \subseteq Ha^{-1}$, which means $Ha^{-1} = Hb^{-1}$.
\end{proof}

% section 10/exercise 35
\begin{exercise}
    If $aH = bH$, then $a^{2}H = b^{2}H$.
\end{exercise}

\begin{proof}
    The statement is not true in general.

    Counterexample: $\{ \iota, \mu \}$ is a subgroup of $D_{3}$. The following left cosets are identical
    \begin{align*}
        \rho\{ \iota, \mu \}        & = \{ \rho, \mu\rho^{2} \}, \\
        \mu\rho^{2}\{ \iota, \mu \} & = \{ \mu\rho^{2}, \rho \}.
    \end{align*}

    but
    \begin{align*}
        {(\rho)}^{2}\{ \iota, \mu \}        & = \{ \rho^{2}, \mu\rho \}, \\
        {(\mu\rho^{2})}^{2}\{ \iota, \mu \} & = \{ \iota, \mu \}.
    \end{align*}
\end{proof}

% section 10/exercise 36
\begin{exercise}
    Let $G$ be a group of order $pq$, where $p$ and $q$ are prime numbers. Show that every proper subgroup of $G$ is cyclic.
\end{exercise}

\begin{proof}
    According to Lagrange's theorem, every proper subgroup of $G$ has order of 1, $p$, or $q$. If the order of the subgroup is 1, then it is cyclic. If the order of the subgroup is $p$ or $q$, then the order of nonidentity element in the subgroup must be $p$ or $q$ (respectively) due to Lagrange's theorem, which means the subgroup is cyclic.

    Hence every proper subgroup of $G$ is cyclic.
\end{proof}

% section 10/exercise 37
\begin{exercise}
    Show that there are the same number of left as right cosets of a subgroup $H$ of a group $G$; that is, exhibit a one-to-one map of the collection of left cosets onto the collection of right cosets. (Note that this result is obvious by counting for finite groups. Your proof must hold for any group.)
\end{exercise}

\begin{proof}
    \textbf{Lemma.} $aH = bH$ if and only if $Ha^{-1} = Hb^{-1}$.

    \textit{Proof of the Lemma.} Let $h$ be an element of $H$.

    If $aH = bH$, then $a\in bH, b\in aH$. So there exist $x, y$ such that $a = bx$ and $b = ay$.
    \[
        \begin{split}
            ha^{-1} = ha^{-1}(bb^{-1}) = h(a^{-1}b)b^{-1} = (hy)b^{-1} \in Hb^{-1} \\
            hb^{-1} = hb^{-1}(aa^{-1}) = h(b^{-1}a)a^{-1} = (hx)a^{-1} \in Ha^{-1}
        \end{split}
    \]

    So $Ha^{-1}\subseteq Hb^{-1}$ and $Hb^{-1}\subseteq Ha^{-1}$. Hence $Ha^{-1} = Hb^{-1}$.

    If $Ha^{-1} = Hb^{-1}$, then $a^{-1}\in Hb^{-1}, b^{-1}\in Ha^{-1}$. So there exist $u, v$ such that $a^{-1} = ub^{-1}$ and $b^{-1} = va^{-1}$.
    \[
        \begin{split}
            ah = (bb^{-1})ah = b(b^{-1}a)h = b(vh) \in bH \\
            bh = (aa^{-1})bh = a(a^{-1}b)h = a(uh) \in aH
        \end{split}
    \]

    So $aH\subseteq bH$ and $bH\subseteq aH$. Hence $aH = bH$.

    Proof of the Lemma is completed.

    Back to main proof. Let $L_{H}$ be the collection of left cosets of $H$, $R_{H}$ the collection of right cosets of $H$. We define a mapping $f: L_{H}\to R_{H}$ as follows
    \[
        f: aH \mapsto Ha^{-1}
    \]

    Due to the Lemma, $f(aH) = Ha^{-1} = Hb^{-1} = f(bH)$ implies $aH = bH$, so $f$ is one-to-one. On the other hand, for every right coset $Hx$, $f(x^{-1}H) = Hx$, so $f$ is onto $R_{H}$. Therefore $f$ is a bijection from $L_{H}$ onto $R_{H}$. Hence the cardinality of the left cosets of $H$ and the cardinality of the right cosets of $H$ are equal.
\end{proof}

% section 10/exercise 38
\begin{exercise}
    Exercise 29 of Section 2 showed that every finite group of even order $2n$ contains an element of order $2$. Using the theorem of Lagrange, show that if $n$ is odd, then an abelian group of order $2n$ contains precisely one element of order $2$.
\end{exercise}

\begin{proof}
    Let $a$ be an element of order $2$ in the given abelian group $G$ of order $2n$ where $n$ is odd. $\anglebracket{a}$ is a subgroup of $G$.

    Assume (for proof by contradiction) that there is another element $b$ of order $2$ in $G$. Then $ab = ba$ is also an element of order $2$. Moreover, $\{ e, a, b, ab \}$ is a subgroup of $G$ ($e, a, b, ab$ are pairwise distinct). Due to the theorem of Lagrange, $4$ divides $2n$, but this contrary to $n$ being an odd number.

    Thus in an abelian group of order $2n$, there is precisely one element of order $2$.
\end{proof}

% section 10/exercise 39
\begin{exercise}
    Show that a group with at least two elements but with no proper nontrivial subgroups must be finite and of prime order.
\end{exercise}

\begin{proof}
    Let $G$ be a group with at least two elements and $G$ has no proper nontrivial subgroups. Let $e, a$ be the identity element and a nonidentity element of $G$.

    Assume that $G$ is infinite. Because $G$ has no proper nontrivial subgroups, so $\anglebracket{a}$ is infinite and $\anglebracket{a} = G$. On the other hand, $\anglebracket{a^{2}}$ is a proper nontrivial subgroup of $\anglebracket{a} = G$, which contrary to $G$ having no proper nontrivial subgroups. Hence $G$ is finite.

    Because $G$ does not have any proper nontrivial subgroup, then $\anglebracket{a} = G$. Let $n$ be the number of elements in $G$. Because $n\geq 2$, $n$ is divisible by a prime number $p$. If $n > p$, then $\anglebracket{a^{n/p}}$ is a proper nontrivial subgroup of $\anglebracket{a} = G$. Therefore $n = p$.

    Thus $G$ is finite and $G$ is of prime order.
\end{proof}

% section 10/exercise 40
\begin{exercise}
    Prove Theorem 10.11: Suppose $H$ and $K$ are subgroups of a group $G$ such that $K\leq H\leq G$, and suppose $(H:K)$ and $(G:H)$ are both finite. Then $(G:K)$ is finite, and $(G:K) = (G:H)(H:K)$.
\end{exercise}

\begin{proof}
    Let $a_{1}H, \ldots, a_{r}H$ be all distinct left cosets of $H$ in $G$. Let $b_{1}K, \ldots, b_{s}K$ be all distinct left cosets of $K$ in $H$.

    For every $1\leq i\leq r, 1\leq j\leq s$, $(a_{i}b_{j})K$ is a left coset of $K$ in $G$. We will prove that for every left coset $xK$ of $K$ in $G$, there exist uniquely $1\leq i\leq r, 1\leq j\leq s$ such that $xK = a_{i}b_{j}K$.

    Let $xK$ be a left coset of $K$ in $G$. $xK$ is a subset of $xH$. $xH$ is one of the $r$ left cosets of $H$ in $G$. So there exists uniquely $1\leq i\leq r$ such that $xH = a_{i}H$, equivalently, ${a_{i}}^{-1}x\in H$. ${a_{i}}^{-1}xK$ is a left coset of $K$ in $H$, so there exists uniquely $1\leq j\leq s$ such that ${a_{i}}^{-1}xK = b_{j}K$, equivalently, ${b_{j}}^{-1}{a_{i}}^{-1}xK = K$. Hence there exist uniquely $1\leq i\leq r, 1\leq j\leq s$ such that $xK = (a_{i}b_{j})K$.

    So $(a_{i}b_{j})K$ for $1\leq i\leq r, 1\leq j\leq s$ are precisely all distinct left cosets of $K$ in $G$. Hence
    \[
        (G:K) = rs = (G:H)(H:K).\qedhere
    \]
\end{proof}

% section 10/exercise 41
\begin{exercise}
    Show that if $H$ is a subgroup of index $2$ in a finite group $G$, then every left coset of $H$ is also a right coset of $H$.
\end{exercise}

\begin{proof}
    $G$ being finite group is redundant.

    The subgroup $H$ in the group $G$ has index of $2$ means $H$ has precisely $2$ left cosets in $G$, and precisely $2$ right cosets in $G$. There are two left cosets: one is $H$, the other is the complement of $H$ in $G$.

    Let $aH$ be a left coset of $H$ in $G$.

    If $aH = H$, then $a\in H$. So $Ha = H$ (right regular representation). Hence $Ha = aH = H$.

    If $aH\ne H$, then $a\notin H$ (if $a\in H$, then $aH = H$). So $Ha\ne H$. Hence $Ha = aH$, which is the left coset other than $H$.

    Hence every left coset of $H$ is also a right coset of $H$.
\end{proof}

% section 10/exercise 42
\begin{exercise}
    Show that if a group $G$ with identity $e$ has finite order $n$, then $a^{n} = e$ for all $a\in G$.
\end{exercise}

\begin{proof}
    Due to the theorem of Lagrange, the order of $\anglebracket{a}$ divides $n$. Let the order of $\anglebracket{a}$ be $d$. Therefore $a^{n} = {(a^{d})}^{n/d} = e^{n/d} = e$. Thus $a^{n} = e$ for every $a\in G$.
\end{proof}

% section 10/exercise 43
\begin{exercise}
    Show that every left coset of the subgroup $\mathbb{Z}$ of the additive group of real numbers contains exactly one element $x$ such that $0\leq x < 1$.
\end{exercise}

\begin{proof}
    Let $a + \mathbb{Z}$ be a left coset of $\mathbb{Z}$ in $\mathbb{R}$.

    Assume that $a$ is not less than any integer, then $a$ is an upper bound of $\mathbb{Z}$. Then due to the completeness axiom, $\mathbb{Z}$ has a least upper bound $u$. Because $u$ is the least upper bound of $\mathbb{Z}$, there exists an integer $m$ such that $m > u - 1$, from which we obtain that $u < 1 + m$, and this contrary to the definition of least upper bound. Hence there exist integers which are greater than $a$.

    Due to the well-ordering principle, in the set of integers greater than $a$ (this set has a lower bound due to its definition, and it is nonempty), there must be a smallest element, let that element be $n$. Then $n - 1\leq a < n$. $n$ is the only integer with such property. Therefore $0\leq a + (1 - n) < 1$, which means there exists an element $x$ in $a + \mathbb{Z}$ such that $0\leq x < 1$.

    Assume that $x, y\in a + \mathbb{Z}$ and $0\leq x, y < 1$. From this, we deduce that $x < 1 \leq 1 + y$ and $x\ge 0 > y - 1$, so $-1 < x - y < 1$. On the other hand, due to the definition of $a + \mathbb{Z}$, $x - y\in\mathbb{Z}$. Therefore $x - y = 0$, which means $x = y$.

    Thus, every left coset of $\mathbb{Z}$ in the addition group $\mathbb{Z}$ contains exactly one element $x$ such that $0\leq x < 1$.
\end{proof}

% section 10/exercise 44
\begin{exercise}
    Show that the function \textit{sine} assigns the same value to each element of any fixed left coset of the subgroup $\anglebracket{2\pi}$ of the additive group lR of real numbers.
\end{exercise}

\begin{proof}
    The result follows from $\sin(x + k2\pi) = \sin(x)$ where $k\in\mathbb{Z}$.
\end{proof}

% section 10/exercise 45
\begin{exercise}
    Let $H$ and $K$ be subgroups of a group $G$. Define $\sim$ on $G$ by $a\sim b$ if and only if $a = hbk$ for some $h\in H$ and some $k\in K$.
    \begin{enumerate}[label={\textbf{\alph*.}}]
        \item Prove that $\sim$ is an equivalence relation on $G$.
        \item Describe the elements in the equivalence class containing $a\in G$. (These equivalence classes are called \textbf{double cosets}.)
    \end{enumerate}
\end{exercise}

\begin{proof}
    Let $e$ be the identity element of $G$.
    \begin{enumerate}[label={\textbf{\alph*.}}]
        \item Because $H$ and $K$ are subgroups of $G$, then $e$ is the identity element of $H$ and $K$. $a = eae$, so $\sim$ is reflexive.

              If $a\sim b$, then $a = hbk$ for some $h\in H$ and some $k\in K$, it follows that $b = h^{-1}ak^{-1}$. $h^{-1}\in H$ and $k^{-1}\in K$ because $H$ and $K$ are subgroups of $G$. So $b\sim a$. Therefore $\sim$ is symmetric.

              If $a\sim b$, $b\sim c$, then $a = h_{1}bk_{1}$ and $b = h_{2}ck_{2}$ for some $h_{1}, h_{2}\in H$ and some $k_{1}, k_{2}\in K$. Then $a = (h_{1}h_{2})c(k_{2}k_{1})$, where $h_{1}h_{2}\in H$ and $k_{2}k_{1}\in K$ because $H$, $K$ are closed under the group operation of $G$. So $a\sim c$, which means $\sim$ is transitive.

              Hence $\sim$ is an equivalence relation on $G$.
        \item The equivalence class containing $a\in G$ is $\{ hak \mid h\in H, k\in K \}$.
    \end{enumerate}
\end{proof}

% section 10/exercise 46
\begin{exercise}
    Let $S_{A}$ be the group of all permutations of the set $A$, and let $c$ be one particular element of $A$.
    \begin{enumerate}[label={\textbf{\alph*.}}]
        \item Show that $\{ \sigma\in S_{A} \mid \sigma(c) = c \}$ is a subgroup $S_{c,c}$ of $S_{A}$.
        \item Let $d\ne c$ be another particular element of $A$. Is $S_{c,d} = \{ \sigma\in S_{A} \mid \sigma(c) = d \}$ a subgroup of $S_{A}$? Why or why not?
        \item Characterize the set $S_{c,d}$ of part (b) in terms of the subgroup $S_{c,c}$ of part (a).
    \end{enumerate}
\end{exercise}

\begin{proof}
    \begin{enumerate}[label={\textbf{\alph*.}}]
        \item The identity permutation is in $S_{c,c}$.

              Let $\sigma,\tau$ be permutations in $S_{c,c}$. $(\sigma\tau)(c) = \sigma(\tau(c)) = \sigma(c) = c$. So $S_{c,c}$ is closed under mapping composition. $\sigma(c) = c$ implies $\sigma^{-1}(c) = c$, so $\sigma^{-1}$ is in $S_{c,c}$. Hence $S_{c,c}$ is a subgroup of $S_{A}$.
        \item $S_{c,d}$ is not a subgroup of $S_{A}$ because the identity permutation isn't in $S_{c,d}$.
        \item Let $\pi\in S_{c,d}$. Then $((c,d)\pi)(c) = (c,d)(d) = c$. So the product of the transposition $(c,d)$ and $\pi$ is an element in $S_{c,c}$.

            Let $\sigma\in S_{c,c}$. Then $((c,d)\sigma)(c) = (c,d)(\sigma(c)) = (c,d)(c) = d$. So the product of the transposition $(c,d)$ and $\sigma$ is an element in $S_{c,d}$.

            Hence $S_{c,d} = (c,d)S_{c,c}$.
    \end{enumerate}
\end{proof}

% section 10/exercise 47
\begin{exercise}
    Show that a finite cyclic group of order $n$ has exactly one subgroup of each order $d$ dividing $n$, and that these are all the subgroups it has.
\end{exercise}

\begin{proof}
    Let the given finite cyclic group be $G$. Let $a$ be a generator of $G$, $e$ be the identity element of $G$.

    (In this paragraph, the finiteness of $G$ is not used) Let $H$ be a subgroup of $G$. Let $k$ be the least positive integer such that $a^{k}\in H$. Let $h$ be an element in $H$, then there exists an integer $m$ such that $a^{m} = h$. By the division algorithm, there exists integer $q$ and $0\leq r < k$ such that $m = kq + r$, so $a^{r} = a^{m-kq} = e$. According to the definition of $k$, we deduce that $r = 0$, which means $h = a^{m} = {(a^{k})}^{q}$. So $\anglebracket{a^{k}} = H$. Hence $H$ is cyclic, it follows that every subgroup of a cyclic group is cyclic.

    \hrulefill{}

    Let's consider the cyclic subgroup $\anglebracket{a^{k}}$. Let $c = \text{gcd}(n,k)$. $a^{k} = a^{c(k/c)}$. Because $n/c$ and $k/c$ are relatively prime, $n/c$ is the smallest positive integer such that $c(k/c)(n/c)$ is divisible by $n$. In other words, $n/c$ is the order of $\anglebracket{a^{k}}$.

    If $\anglebracket{a^{k}} = \anglebracket{a^{\ell}}$, then $n/\text{gcd}(n,k) = n/\text{gcd}(n,\ell)$, which implies $\text{gcd}(n,k) = \text{gcd}(n,\ell)$. If $\text{gcd}(n,k) = \text{gcd}(n,\ell) = d$, then $\anglebracket{a^{k}} = \anglebracket{a^{d}} = \anglebracket{a^{\ell}}$. Hence $\anglebracket{a^{k}} = \anglebracket{a^{\ell}}$ if and only if $\text{gcd}(n,k) = \text{gcd}(n,\ell)$.

    Let $d$ be a divisor of $n$. The order of $a$ is $n$, so the order of $a^{n/d}$ is $d$. The order of the cyclic subgroup $\anglebracket{a^{n/d}}$ is $d$.

    Thus, a finite cyclic group of order $n$ has exactly one subgroup of each order $d$ dividing $n$ and these are all subgroups it has.
\end{proof}

% section 10/exercise 48
\begin{exercise}
    The \textbf{Euler phi-function} is defined for positive integers $n$ by $\varphi(n) = s$, where $s$ is the number of positive integers less than or equal to $n$ that are are relatively prime to $n$. Use Exercise 47 to show that
    \[
        n = \sum_{d \mid n}\varphi(d),
    \]

    the sum being taken over all positive integers $d$ dividing $n$.
\end{exercise}

\begin{proof}
    Due to Exercise 43, in the cyclic group $\mathbb{Z}_{n}$, the order of any element is a divisor of $n$, and for every divisor $d$ of $n$, there is exactly one cyclic subgroup in $\mathbb{Z}_{n}$ of order $d$. Therefore, $n$ is equal to the sum of the number of elements whose order is $d$, where $d$ varies in the set of divisors of $n$. In other words
    \[
        n = \sum_{d \mid n}\#\text{(elements of order $d$)}
    \]

    $\anglebracket{n/d}$ is the only cyclic subgroup of order $d$ in $\mathbb{Z}_{n}$. Every element $x$ of order $d$ is a generator of $\anglebracket{n/d}$ and vice versa (due to Exercise 43). $\anglebracket{n/d}$ is isomorphic to $\mathbb{Z}_{d}$. $\mathbb{Z}_{d}$ has $\varphi(d)$ generators (there are $\varphi(d)$ positive integers not exceeding $d$ and relatively prime with $d$). Therefore $\anglebracket{n/d}$ has $\varphi(d)$ generators, from which we deduce that within $\mathbb{Z}_{n}$, there are $\varphi(d)$ elements of order $d$. Thus
    \[
        n = \sum_{d \mid n}\varphi(d).
    \]
\end{proof}

% section 10/exercise 49
\begin{exercise}
    Let $G$ be a finite group. Show that if for each positive integer $m$ the number of solutions $x$ of the equation $x^{m} = e$ in $G$ is at most $m$, then $G$ is cyclic.
\end{exercise}

\begin{proof}
    Denote by $n$ the number of elements in $G$. Let $a$ be an arbitrary element in $G$. $\anglebracket{a}$ is a subgroup of $G$. According to Lagrange's theorem, the order of $a$ is a divisor of $n$, let this divisor be $d$. Then $a^{n} = a^{d(n/d)} = e^{n/d} = e$. So the equation $x^{n} = e$ has exactly $n$ solutions.

    If $n = 1$, $G$ is cyclic.

    If $n$ is a prime number, then the order of a nonidentity element must be $n$ (due to Lagrange's theorem and $n$ being a prime), so $G$ is cyclic.

    If $n$ is a composite, let $h$ be a maximal divisor of $n$ (the only multiple of $h$ and not exceeding $n$ is $n$). Suppose that the equation $x^{h} = e$ has $k$ solutions $e, a_{1}, \ldots, a_{k-1}$, where $k\leq h < n$. $x^{n} = x^{h(n/h)} = e$ has exacly $n$ solutions, they are $e, a_{1}, \ldots, a_{k-1}, a_{k}, \ldots, a_{n-1}$. So the orders of $a_{k}, \ldots, a_{n-1}$ are greater than $h$. But the orders of $a_{k}, \ldots, a_{n-1}$ are at most $n$ and are the multiple of $h$. Due to the definition of $h$, the orders of $a_{k}, \ldots, a_{n-1}$ are $n$, which is equal to the order of the finite group $G$. So $G$ is cyclic.

    Thus $G$ is cyclic.
\end{proof}

% section 10/exercise 50
\begin{exercise}
    Show that a finite group cannot be written as the union of two of its proper subgroups. Does the statement remain true if ``two'' is replaced by ``three''?
\end{exercise}

\begin{proof}
    Let $G$ be a finite group. If $G$ has exactly one element, then it can't be written as the union of two of its proper subgroups, because it doesn't have any proper subgroups.

    If $G$ has $n$ elements ($n > 1$). Assume that $G = A\cup B$ where $A, B$ are proper subgroups of $G$. $A$ is not a subset of $B$, $B$ is not a subset of $A$ (because if so, then $A = G$ or $B = G$). $A\cap B$ is also a subgroup of $G$. Let the number of elements in $A\cap B$ be $d$. Due to Lagrange's theorem, the order of $A\cap B$ divides the orders of $A$ and $B$. Let the orders of $A$, $B$ be $ad, bd$, respectively. According to the principle of inclusion and exclusion, $ad + bd - d = n$. $ad + (b-1)d = n$, $(a-1)d + bd = n$, and $ad, bd$ divide $n$ (due to Lagrange's theorem). So $ad$ divides $(b - 1)d$ and $bd$ divides $(a - 1)d$. Therefore $a\leq b - 1$ and $b\leq a - 1$, which implies $a\leq a - 2$ and $b\leq b - 2$, this is impossible because $a > a - 2$ and $b > b - 2$. So the assumption is false. Hence $G$ cannot be written as the union of two of its proper subgroups.

    The statement doesn't remain true if ``two'' is replaced by ``three''. Here is a counter example: the Klein 4-group $V$ has 4 elements $e, a, b, c$, where $c = ab = ba$, $b = ca = ac$, $a = bc = cb$. $\{e, a\}, \{e, b\}, \{e, c\}$ are proper subgroups of $V$ and the union of them is the whole set under $V$.
\end{proof}

\section{Plane Isometries}
