\chapter{Groups and Subgroups}
\section{Binary Operations}

\subsection*{Computations}

Exercises 1 through 4 concern the binary operation $*$ defined on $S = \{a, b, c, d, e\}$ by means of Table 1.31 (see the book)

% section 1/exercise 1
\begin{exercise}
    Compute $b * d, c * c$, and $((a * c) * e) * a$.
\end{exercise}

\begin{proof}
    \[
        b * d = e.
    \]
    \[
        ((a * c) * e) * a = (c * e) * a = a * a = a.
    \]
\end{proof}

% section 1/exercise 2
\begin{exercise}
    Compute $(a * b) * c$ and $a * (b * c)$. Can you say on the basis of this computation whether $*$ is associative?
\end{exercise}

\begin{proof}
    \[
        \begin{split}
            (a * b) * c = b * c = a, \\
            a * (b * c) = a * a = a.
        \end{split}
    \]

    On the basis of this computation, it is not sufficient to say $*$ is associative.
\end{proof}

% section 1/exercise 3
\begin{exercise}
    Compute $(b * d) * c$ and $b * (d * c)$. Can you say on the basis of this computation whether $*$ is associative?
\end{exercise}

\begin{proof}
    \[
        \begin{split}
            (b * d) * c = e * c = a, \\
            b * (d * c) = b * b = c.
        \end{split}
    \]

    On the basis of this computation, I can say $*$ is not associative.
\end{proof}

% section 1/exercise 4
\begin{exercise}
    Is $*$ commutative? Why?
\end{exercise}

\begin{proof}
    $*$ is not commutative. Because $e * b = b \ne c = b * e$.
\end{proof}

% section 1/exercise 5
\begin{exercise}
    Complete Table 1.32 so as to define a commutative binary operation $*$ on $S = \{ a, b, c, d \}$.
\end{exercise}

\begin{proof}
    % chktex-file 44
    \begin{tabular}{c|c|c|c|c}
        * & a & b & c & d \\
        \midrule
        a & a & b & c & d \\
        b & b & d & a & c \\
        c & c & a & d & b \\
        d & d & c & b & a
    \end{tabular}
\end{proof}

% section 1/exercise 6
\begin{exercise}
    Table 1.33 can be completed to define an associative binary operation $*$ on $S = \{ a, b, c, d \}$. Assume this is possible and compute the missing entries. Does $S$ have an identity element?
\end{exercise}

\begin{proof}
    % chktex-file 44
    \begin{tabular}{c|c|c|c|c}
        * & a          & b          & c          & d          \\
        \hline
        a & a          & b          & c          & d          \\
        b & b          & c          & a          & d          \\
        c & c          & d          & c          & d          \\
        d & \textbf{d} & \textbf{c} & \textbf{c} & \textbf{d}
    \end{tabular}

    $S$ has an identity element, which is $a$.
\end{proof}

In Exercises 7 through 11, determine whether the binary operation $*$ defined is commutative and whether $*$ is associative.

% section 1/exercise 7
\begin{exercise}
    $*$ defined on $\mathbb{Z}$ by letting $a * b = a - b$
\end{exercise}

\begin{proof}
    $*$ is noncommutative. Because $1 * 2 = 1 - 2 = -1 \ne 1 = 2 - 1 = 2 * 1$.
\end{proof}

% section 1/exercise 8
\begin{exercise}
    $*$ defined on $\mathbb{Q}$ by letting $a * b = 2ab + 3$
\end{exercise}

\begin{proof}
    $*$ is commutative. Because for every two rational numbers $a$ and $b$, $ab + 1\in\mathbb{Q}$ and
    \[
        a * b = 2ab + 3 = 2ba + 3 = b * a.
    \]
\end{proof}

% section 1/exercise 9
\begin{exercise}
    $*$ defined on $\mathbb{Z}$ by letting $a * b = ab + a + b$
\end{exercise}

\begin{proof}
    $*$ is commutative. Because for every two rational numbers $a$ and $b$, $ab + a + b\in\mathbb{Z}$ and
    \[
        a * b = ab + a + b = ba + b + a = b * a
    \]
\end{proof}

% section 1/exercise 10
\begin{exercise}
    $*$ defined on $\mathbb{Z}^{*}$ by letting $a * b = 2^{ab}$
\end{exercise}

\begin{proof}
    $*$ is commutative. Because for every two positive integers $a, b$, $2^{ab}\in\mathbb{Z}^{+}$ and
    \[
        a * b = 2^{ab} = 2^{ba} = b * a.
    \]
\end{proof}

% section 1/exercise 11
\begin{exercise}
    $*$ defined on $\mathbb{Z}^{+}$ by letting $a * b = a^{b}$
\end{exercise}

\begin{proof}
    $*$ is not commutative. Because $1 * 2 = 1^{2} = 1 \ne 2 = 2^{1} = 2 * 1$.
\end{proof}

% section 1/exercise 12
\begin{exercise}
    Let $S$ be a set having exactly one element. How many different binary operations can be defined on $S$? Answer the question if $S$ has exactly $2$ elements; exactly $3$ elements; exactly $n$ elements.
\end{exercise}

\begin{proof}
    If $\card{S} = 1$, there is $1$ binary operations can be defined on $S$.

    If $\card{S} = 2$, $\card{S\times S} = 4$, there are $2^{4}$ binary operations can be defined on $S$.

    If $\card{S} = 3$, $\card{S\times S} = 9$, there are $3^{9}$ binary operations can be defined on $S$.

    If $\card{S} = n$, $\card{S\times S} = n^{2}$, there are $n^{(n^{2})}$ binary operations can be defined on $S$.
\end{proof}

% section 1/exercise 13
\begin{exercise}
    How many different commutative binary operations can be defined on a set of $2$ elements? on a set of $3$ elements? on a set of $n$ elements?
\end{exercise}

\begin{proof}
    If $\card{S} = 1$, there is $1$ binary operations can be defined on $S$ and it is also commutative.

    If $\card{S} = 2$, there are $2^{2(2+1)/2} = 8$ commutative binary operations can be defined on $S$.

    If $\card{S} = 3$, there are $3^{3(3+1)/2} = 729$ commutative binary operations can be defined on $S$.

    If $\card{S} = n$, there are $n^{n(n+1)/2}$ commutative binary operations can be defined on $S$.
\end{proof}

% section 1/exercise 14
\begin{exercise}
    How many different binary operations on a set $S$ with $n$ elements have the property that for all $x\in S, x * x = x$?
\end{exercise}

\begin{proof}
    There are $n^{n(n-1)}$ such binary operations.
\end{proof}

% section 1/exercise 15
\begin{exercise}
    How many different binary operations on a set $S$ with $n$ elements have an identity element?
\end{exercise}

\begin{proof}
    $S = \{ a_{1}, \ldots, a_{n} \}$.

    If $a_{i}$ is the identity element, then there are $n^{{(n-1)}^{2}}$ such binary operations.

    Hence there are $n^{1 + {(n-1)}^{2}}$ such binary operations.
\end{proof}

\subsection*{Concepts}

In Exercises 16 through 19, correct the definition of the italicized term without reference to the text, if correction is needed, so that it is in a form acceptable for publication.

% section 1/exercise 16
\begin{exercise}
    A binary operation $*$ is \textit{commutative} if and only if $a * b = b * a$.
\end{exercise}

\begin{proof}
    Correction: A binary operation $*$ on a set $S$ is \textit{commutative} if and only if $a * b = b * a$ for every two elements $a, b$.
\end{proof}

% section 1/exercise 17
\begin{exercise}
    A binary operation $*$ on a set $S$ is \textit{associative} if and only if, for all $a, b, c\in S$, we have $(b * c) * a = b * (c * a)$.
\end{exercise}

\begin{proof}
    This definition doesn't need correction.
\end{proof}

% section 1/exercise 18
\begin{exercise}
    A subset $H$ of a set $S$ is \textit{closed} under a binary operation $*$ on $S$ if and only if $(a * b)\in H$ for all $a, b\in S$.
\end{exercise}

\begin{proof}
    Correction: A subset $H$ of a set $S$ is \textit{closed} under a binary operation $*$ on $S$ if and only if $(a * b)\in H$ for all $a, b\in H$.
\end{proof}

% section 1/exercise 19
\begin{exercise}
    An identity in the set $S$ with operation $*$ is an element $e\in S$ such that $a * e = e * a = a$.
\end{exercise}

\begin{proof}
    Correction: An identity in the set $S$ with operation $*$ is an element $e\in S$ such that $a * e = e * a = a$ for all $a\in S$.
\end{proof}

% section 1/exercise 20
\begin{exercise}
    Is there an example of a set $S$, a binary operation on $S$, and two different elements $e_{1}, e_{2}\in S$ such that for all $a\in S$, $e_{1} * a = a$ and $a * e_{2} = a$? If so, give an example and if not, prove there is not one.
\end{exercise}

\begin{proof}
    There is no such binary operation.

    $e_{1} * a = a$ for all $a\in S$ so $e_{1} * e_{2} = e_{2}$.

    $a * e_{2} = a$ for all $a\in S$ so $e_{1} * e_{2} = e_{1}$.

    Thus $e_{1} = e_{1} * e_{2} = e_{2}$.
\end{proof}

In Exercises 21 through 26, determine whether the definition of $*$ does give a binary operation on the set. In the event that $*$ is not a binary operation, state whether Condition 1, Condition 2, or both conditions regarding defining binary operations are violated.

% section 1/exercise 21
\begin{exercise}
    On $\mathbb{Z}^{+}$, define $*$ by letting $a * b = a^{b}$.
\end{exercise}

\begin{proof}
    This is a binary operation.
\end{proof}

% section 1/exercise 22
\begin{exercise}
    On $\mathbb{R}^{+}$, define $*$ by letting $a * b = 2a - b$.
\end{exercise}

\begin{proof}
    This is not a binary operation. Condition 2 is violated, since $1 * 2 = 2\cdot 1 - 2 = 2 - 2 = 0\notin\mathbb{R}^{+}$.
\end{proof}

% section 1/exercise 23
\begin{exercise}
    On $\mathbb{R}^{+}$, define $*$ by $a * b$ to be the minimum of $a$ and $b - 1$ if they are different and their common value if they are the same.
\end{exercise}

\begin{proof}
    This is not a binary operation. Condition 2 is violated, since $1 * 1 = 1 - 1 = 0\notin\mathbb{R}^{+}$.
\end{proof}

% section 1/exercise 24
\begin{exercise}
    On $\mathbb{R}$, define $a * b$ to be the number $c$ so that $c^{b} = a$.
\end{exercise}

\begin{proof}
    This is not a binary operation. Both conditions are violated, since
    \begin{itemize}
        \item $c^{0} = 1$ for every $c\ne 0$,
        \item there is no $c\in\mathbb{R}$ such that $c^{1/2} = -1$.
    \end{itemize}
\end{proof}

% section 1/exercise 25
\begin{exercise}
    On $\mathbb{Z}^{+}$, define $*$ letting $a * b = c$, where $c$ is at least $5$ more than $a + b$.
\end{exercise}

\begin{proof}
    This is not a binary operation. Because Condition 1 is violated. $a * b$ can be assigned to $a + b + 5, a + b + 6, \ldots$
\end{proof}

% section 1/exercise 26
\begin{exercise}
    On $\mathbb{Z}^{+}$, define $*$ by letting $a * b = c$, where $c$ is the largest integer less than the product of $a$ and $b$.
\end{exercise}

\begin{proof}
    This is not a binary operation. Because Condition 2 is violated. $1 * 1 = 0\notin\mathbb{Z}^{+}$.
\end{proof}

% section 1/exercise 27
\begin{exercise}
    Let $H$ be the subset of $M_{2}(\mathbb{R})$ consisting of all matrices of the form $\begin{bmatrix}a & -b \\ b & a\end{bmatrix}$ for $a, b\in\mathbb{R}$. Is $H$ closed under
    \begin{enumerate}[label={\textbf{\alph*}}]
        \item matrix addition?
        \item matrix multiplication?
    \end{enumerate}
\end{exercise}

\begin{proof}
    \begin{enumerate}[label={\textbf{\alph*}}]
        \item For any two matrices within $H$
              \[
                  \begin{bmatrix}
                      a & -b \\
                      b & a
                  \end{bmatrix}
                  +
                  \begin{bmatrix}
                      c & -d \\
                      d & c
                  \end{bmatrix}
                  =
                  \begin{bmatrix}
                      a + c & -(b + d) \\
                      b + d & a + c
                  \end{bmatrix}
                  \in H
              \]

              so $H$ is closed under matrix addition.
        \item For any two matrices within $H$
              \[
                  \begin{bmatrix}
                      a & -b \\
                      b & a
                  \end{bmatrix}
                  \cdot
                  \begin{bmatrix}
                      c & -d \\
                      d & c
                  \end{bmatrix}
                  =
                  \begin{bmatrix}
                      ac - bd & -(ad + bc) \\
                      ad + bc & ac - bd
                  \end{bmatrix}
                  \in H
              \]

              so $H$ is closed under matrix multiplication.
    \end{enumerate}
\end{proof}

% section 1/exercise 28
\begin{exercise}
    Mark each of the following true or false.
    \begin{enumerate}[label={\textbf{\alph*.}},itemsep=0pt,topsep=0pt]
        \item If $*$ is any binary operation on any set $S$, then $a * a = a$ for all $a\in S$.
        \item If $*$ is any commutative binary operation on any set $S$, then $a * (b * c) = (b * c) * a$ for all $a, b, c \in S$.
        \item If $*$ is any associative binary operation on any set $S$, then $a * (b * c) = (b * c) * a$ for all $a, b, c \in S$.
        \item The only binary operations of any importance are those defined on sets of numbers.
        \item A binary operation $*$ on a set $S$ is commutative if there exist $a, b \in S$ such that $a * b = b * a$.
        \item Every binary operation defined on a set having exactly one element is both commutative and associative.
        \item A binary operation on a set $S$ assigns at least one element of $S$ to each ordered pair of elements of $S$.
        \item A binary operation on a set $S$ assigns at most one element of $S$ to each ordered pair of elements of $S$.
        \item A binary operation on a set $S$ assigns exactly one element of $S$ to each ordered pair of elements of $S$.
        \item A binary operation on a set $S$ may assign more than one element of $S$ to some ordered pair of elements of $S$.
        \item For any binary operation $*$ on the set $S$, if $a, b, c\in S$ and $a * b = a * c$, then $b = c$.
        \item For any binary operation $*$ on the set $S$, there is an element $e\in S$ such that for all $x\in S, x * e = x$.
        \item There is an operation on the set $S = \{ e_{1}, e_{2}, a \}$ so that for all $x\in S$, $e_{1} * x = e_{2} * x = x$.
        \item Identity elements are always called $e$.
    \end{enumerate}
\end{exercise}

\begin{proof}
    \begin{enumerate}[label={\textbf{\alph*.}},itemsep=0pt,topsep=0pt]
        \item False. Example: $\mathbb{Z}$ with addition.
        \item True.
        \item False. Example: $M(2,\mathbb{R})$ with multiplication.
              \[
                  \begin{bmatrix}
                      1 & 1 \\
                      0 & 0
                  \end{bmatrix}
                  \cdot
                  \left(
                  \begin{bmatrix}
                          1 & 0 \\
                          0 & 1
                      \end{bmatrix}
                  \cdot
                  \begin{bmatrix}
                          1 & 1 \\
                          1 & 0
                      \end{bmatrix}
                  \right)
                  =
                  \begin{bmatrix}
                      2 & 1 \\
                      0 & 0
                  \end{bmatrix}
                  \ne
                  \begin{bmatrix}
                      1 & 1 \\
                      1 & 1
                  \end{bmatrix}
                  =
                  \left(
                  \begin{bmatrix}
                          1 & 0 \\
                          0 & 1
                      \end{bmatrix}
                  \cdot
                  \begin{bmatrix}
                          1 & 1 \\
                          1 & 0
                      \end{bmatrix}
                  \right)
                  \cdot
                  \begin{bmatrix}
                      1 & 1 \\
                      0 & 0
                  \end{bmatrix}.
              \]
        \item Undeciable, since ``important'' is undefined.
        \item False. Example: $M_{2}(\mathbb{R})$ with multiplication, any $2\times 2$ matrix is commutative with the $2\times 2$ identity matrix, but multiplication of any two $2\times 2$ matrices are not necessarily commutative.
        \item True.
        \item False. Must be exactly one, not at least one.
        \item False. Must be exactly one, not at most one.
        \item True.
        \item False.
        \item False. Counterexample: $\mathbb{Z}$ with usual multiplication, $0 \cdot 1 = 0 \cdot 2$ but $1 \ne 2$.
        \item False. Counterexample: $S = \{ a, b \}$, $a * a = b, a * b = b, b * b = a, b * a = a$.
        \item True. Example:
              \[
                  \begin{array}{c|ccc}
                      *     & e_{1} & e_{2} & a \\
                      \hline
                      e_{1} & e_{1} & e_{2} & a \\
                      e_{2} & e_{1} & e_{2} & a \\
                      a     & a     & a     & a
                  \end{array}
              \]
        \item False. We can called it anything.
    \end{enumerate}
\end{proof}

% section 1/exercise 29
\begin{exercise}
    Give a set different from any of those described in the examples of the text and not a set of numbers. Define two different binary operations $*$ and $*'$ on this set. Be sure that your set is well defined.
\end{exercise}

\begin{proof}
    Let $S$ be a set of two elements $a$ and $b$. Define two binary operations $*$ and $*'$ as follows
    % chktex-file 44
    \begin{tabular}{c|cc}
        * & a & b \\
        \hline
        a & a & a \\
        b & a & a
    \end{tabular}
    % chktex-file 44
    \begin{tabular}{c|cc}
        * & a & b \\
        \hline
        a & a & b \\
        b & b & a
    \end{tabular}
\end{proof}

\subsection*{Theory}

% section 1/exercise 30
\begin{exercise}
    Prove that if $*$ is an associative and commutative binary operation on a set $S$, then
    \[
        (a * b) * (c * d) = ((d * c) * a) * b
    \]

    for all $a, b, c, d\in S$. Assume the associative law only for triples as in the definition, that is, assume only
    \[
        (x * y) * z = x * (y * z)
    \]

    for all $x, y, z\in S$.
\end{exercise}

\begin{proof}
    $d * c = e\in S$.

    \begin{align*}
        (a * b) * (c * d) & = (c * d) * (a * b)  & \text{(commutative)} \\
                          & = (d * c) * (a * b)  & \text{(commutative)} \\
                          & = e * (a * b)                               \\
                          & = (e * a) * b        & \text{(associative)} \\
                          & = ((d * c) * a) * b.
    \end{align*}
\end{proof}

In Exercises 31 and 32, either prove the statement or give a counterexample.

% section 1/exercise 31
\begin{exercise}
    Every binary operation on a set consisting of a single element is both commutative and associative.
\end{exercise}

\begin{proof}
    Let $S = \{ a \}$. There is only one binary operation $*$ can be define on $S$, where $a * a = a$.

    Since $a * a = a * a$ and $(a * a) * a = a * a = a * (a * a)$, $S$ with $*$ is commutative and associative.
\end{proof}

% section 1/exercise 32
\begin{exercise}
    Every commutative binary operation on a set having just two elements is associative.
\end{exercise}

\begin{proof}
    False.

    Counterexample: Let $S = \{ a, b \}$, and $*$ be the commutative binary operation on $S$ as follows
    % chktex-file 44
    \begin{tabular}{c|cc}
        * & a & b \\
        \midrule
        a & b & b \\
        b & b & a
    \end{tabular}

    \[
        (a * a) * b = b * b = a \ne b = a * b = a * (a * b).
    \]
\end{proof}

Let $F$ be the set of all real-valued functions having as domain the set $\mathbb{R}$ of all real numbers. Example 2.7 defined the binary operations $+, -, \cdot$, and $\circ$ on $F$. In Exercises 29 through 35, either prove the given statement or give a
counterexample.

% section 1/exercise 33
\begin{exercise}
    Function addition $+$ on $F$ is associative.
\end{exercise}

\begin{proof}
    For every real number $x$ and every three functions $f, g, h$ in $F$
    \begin{align*}
        ((f + g) + h)(x) & = (f + g)(x) + h(x)    \\
                         & = (f(x) + g(x)) + h(x) \\
                         & = f(x) + (g(x) + h(x)) \\
                         & = f(x) + (g + h)(x)    \\
                         & = (f + (g + h))(x)
    \end{align*}

    Thus function addition $+$ on $F$ is associative.
\end{proof}

% section 1/exercise 34
\begin{exercise}
    Function subtraction $-$ on $F$ is commutative.
\end{exercise}

\begin{proof}
    This is false.

    Counterexample: $f(x) = x, g(x) = x + 1$, then $(f - g)(x) = -1\ne 1 = (g - f)(x)$.
\end{proof}

% section 1/exercise 35
\begin{exercise}
    Function substraction $-$ on $F$ is associative.
\end{exercise}

\begin{proof}
    This is false.

    Counterexample: $f(x) = g(x) = h(x) = x$, then for $x\ne 0$, $((f - g) - h)(x) = -x \ne x = (f - (g - h))(x)$.
\end{proof}

% section 1/exercise 36
\begin{exercise}
    Under function substraction $-$ $F$ has an identity.
\end{exercise}

\begin{proof}
    This is false.

    Assume that under function substraction $-$ $F$ has an identity $\iota$. Then for every $f\in F$ and for every real number $x$
    \[
        f(x) - \iota(x) = \iota(x) - f(x) = f(x)
    \]

    We deduce that $\iota(x) = 0$ and $\iota(x) = \frac{1}{2}f(x)$ for every real number $x$. This is a contradiction, because $\frac{1}{2}f(x)$ is not necessarily equal to $0$ for every real number $x$.

    Thus Under function substraction $-$ $F$ does not have an identity.
\end{proof}

% section 1/exercise 37
\begin{exercise}
    Under function multiplication $\cdot$ $F$ has an identity.
\end{exercise}

\begin{proof}
    This is true.

    Let $\iota(x) = 1$ for every real number $x$. Then for every $f\in F$ and every real number $x$
    \[
        (f\cdot \iota)(x) = f(x)\cdot \iota(x) = f(x) = \iota(x)\cdot f(x) = (\iota\cdot f)(x).
    \]
\end{proof}

% section 1/exercise 38
\begin{exercise}
    Function multiplication $\cdot$ on $F$ is commutative.
\end{exercise}

\begin{proof}
    For every real number $x$ and every two functions $f, g$ in $F$
    \begin{align*}
        (f\cdot g)(x) & = f(x)g(x)       \\
                      & = g(x)f(x)       \\
                      & = (g\cdot f)(x).
    \end{align*}

    Thus function multiplication $\cdot$ on $F$ is commutative.
\end{proof}

% section 1/exercise 39
\begin{exercise}
    Function multiplication $\cdot$ on $F$ is associative.
\end{exercise}

\begin{proof}
    For every real number $x$ and every three functions $f, g, h$ in $F$
    \begin{align*}
        ((f \cdot g) \cdot h)(x) & = (f \cdot g)(x) \cdot h(x)    \\
                                 & = (f(x) \cdot g(x)) \cdot h(x) \\
                                 & = f(x) \cdot (g(x) \cdot h(x)) \\
                                 & = f(x) \cdot (g \cdot h)(x)    \\
                                 & = (f \cdot (g \cdot h))(x)
    \end{align*}

    Thus function multiplication $\cdot$ on $F$ is associative.
\end{proof}

% section 1/exercise 40
\begin{exercise}
    Function composition $\circ$ on $F$ is commutative.
\end{exercise}

\begin{proof}
    This is false.

    Counterexample: $f(x) = x + 1, g(x) = 2x$. $(f\circ g)(x) = f(g(x)) = 2x + 1\ne 2x + 2 = g(x + 1) = g(f(x))$.
\end{proof}

% section 1/exercise 41
\begin{exercise}
    If $*$ and $*'$ are any two binary operations on a set $S$, then
    \[
        a * (b *' c) = (a * b) *' (a * c)\quad\text{for all $a, b, c\in S$.}
    \]
\end{exercise}

\begin{proof}
    This is false. We give a counterexample.

    Let $S = \{ 0, 1 \}$. We define $*$ and $*'$ as follows
    % chktex-file 44
    \begin{tabular}{c|cc}
        * & 0 & 1 \\
        \midrule
        0 & 0 & 0 \\
        1 & 0 & 1
    \end{tabular}
    \begin{tabular}{c|cc}
        \midrule
        *' & 0 & 1 \\
        0  & 1 & 0 \\
        1  & 0 & 1
    \end{tabular}
    \[
        \begin{split}
            0 * (0 *' 1) = 0 * 0 = 0 \ne 1,  \\
            (0 * 0) *' (0 * 1) = 0 *' 0 = 1.
        \end{split}
    \]
\end{proof}

% section 1/exercise 42
\begin{exercise}
    Suppose that $*$ is an \textit{associative binary} operation on a set $S$. Let $H = \{ a \in S \vert a * x = x * a \text{ for all } x\in S \}$. Show that $H$ is closed under $*$. (We think of $H$ as consisting of all elements of $S$ that \textit{commute} with every element in $S$.)
\end{exercise}

\begin{proof}
    Let $a, b$ be elements of $H$. For every element $x$ of $S$
    \begin{align*}
        (a * b) * x & = a * (b * x) & \text{(associative)} \\
                    & = a * (x * b) & (b\in H)             \\
                    & = (a * x) * b & \text{(associative)} \\
                    & = (x * a) * b & (a\in H)             \\
                    & = x * (a * b) & \text{(associative)}
    \end{align*}

    According to the definition of $H$, $a * b\in H$. Hence $H$ is closed under the binary operation $*$ on $S$.
\end{proof}

% section 1/exercise 43
\begin{exercise}
    Suppose that $*$ is an associative and commutative operation on a set $S$. Show that $H = \{ a\in S \vert a * a = a \}$ is closed under $*$. (The element of $H$ are \textbf{idempotents} of the binary operation $*$.)
\end{exercise}

\begin{proof}
    Let $a, b$ be elements of $H$.
    \begin{align*}
        (a * b) * (a * b) & = (a * b) * (b * a) & \text{(commutative)} \\
                          & = ((a * b) * b) * a & \text{(associative)} \\
                          & = (a * (b * b)) * a & \text{(associative)} \\
                          & = (a * b) * a       & (b\in H)             \\
                          & = (b * a) * a       & \text{(commutative)} \\
                          & = b * (a * a)       & \text{(associative)} \\
                          & = b * a             & (a\in H)             \\
                          & = a * b             & \text{(commutative)}
    \end{align*}

    According to the definition of $H$, $a * b\in H$. Hence $H$ is closed under $*$.
\end{proof}

% section 1/exercise 44
\begin{exercise}
    Let $S$ be a set and let $*$ be a binary operation on $S$ satisfying the two laws
    \begin{itemize}
        \item $x * x = x$ for all $x\in S$, and
        \item $(x * y) * z = (y * z) * x$ for all $x, y, z\in S$.
    \end{itemize}

    Show that $*$ is associative and commutative.
\end{exercise}

\begin{proof}
    For all $x, y\in S$,

    \begin{align*}
        x * y & = (x * y) * (x * y) & \text{(Law 1)} \\
              & = (y * (x * y)) * x & \text{(Law 2)} \\
              & = ((x * y) * x) * y & \text{(Law 2)} \\
              & = ((y * x) * x) * y & \text{(Law 2)} \\
              & = ((x * x) * y) * y & \text{(Law 2)} \\
              & = (x * y) * y       & \text{(Law 1)} \\
              & = (y * y) * x       & \text{(Law 2)} \\
              & = y * x             & \text{(Law 1)}
    \end{align*}

    So $*$ is commutative.

    \begin{align*}
        (x * y) * z & = (y * z) * x & \text{(Law 2)}           \\
                    & = x * (y * z) & \text{(Commutative law)}
    \end{align*}

    So $*$ is associative.
\end{proof}


\section{Groups}


\section{Abelian Examples}


\section{Nonabelian Examples}

\section{Introduction and Examples}
\setcounter{exercise}{0}

In Exercises 1 through 9 compute the given arithmetic expression and give the answer in the form $a + bi$ for $a, b\in \mathbb{R}$.

\begin{exercise}
    $i^{3}$
\end{exercise}

\begin{proof}
    $i^{3} = {i}^{2}i = -i = 0 + (-1)i$.
\end{proof}

\begin{exercise}
    $i^{4}$
\end{exercise}

\begin{proof}
    $i^{4} = {i}^{2}{i}^{2} = (-1)\cdot (-1) = 1 = 1 + 0i$.
\end{proof}

\begin{exercise}
    $i^{23}$
\end{exercise}

\begin{proof}
    $i^{23} = {i}^{3}{i}^{20} = -i = 0 + (-1)i$.
\end{proof}

\begin{exercise}
    ${(-i)}^{35}$
\end{exercise}

\begin{proof}
    ${(-i)}^{35} = {(-i)}^{3}{(-i)}^{32} = {(-i)}^{3} = i = 0 + 1i$.
\end{proof}

\begin{exercise}
    $(4 - i)(5 + 3i)$
\end{exercise}

\begin{proof}
    $(4 - i)(5 + 3i) = 23 + 7i$.
\end{proof}

\begin{exercise}
    $(8 + 2i)(3 - i)$
\end{exercise}

\begin{proof}
    $(8 + 2i)(3 - i) = 26 + (-2)i$.
\end{proof}

\begin{exercise}
    $(2 - 3i)(4 + i) + (6 - 5i)$
\end{exercise}

\begin{proof}
    $(2 - 3i)(4 + i) + (6 - 5i) = (11 - 10i) + (6 - 5i) = 17 - 15i = 17 + (-15)i$.
\end{proof}

\begin{exercise}
    ${(1+i)}^{3}$
\end{exercise}

\begin{proof}
    ${(1+i)}^{3} = 1^{3} + 3\cdot 1^{2}i + 3\cdot 1\cdot i^{2} + i^{3} = 1 + 3i - 3 - i = (-2) + 2i$.
\end{proof}

\begin{exercise}
    ${(1-i)}^{5}$
\end{exercise}

\begin{proof}
    \begin{align*}
        {(1-i)}^{5} & = 1^{5} - 5\cdot 1^{4}i + 10\cdot 1^{3}i^{2} - 10\cdot 1^{2}i^{3} + 5\cdot 1\cdot i^{4} - i^{5} \\
                    & = 1 - 5i - 10 + 10i + 5 - i                                                                     \\
                    & = (-4) + 4i.
    \end{align*}
\end{proof}

\begin{exercise}
    Find $\abs{3 - 4i}$.
\end{exercise}

\begin{proof}
    $\abs{3 - 4i} = \sqrt{3^{2} + 4^{2}} = 5$.
\end{proof}

\begin{exercise}
    Find $\abs{6 + 4i}$.
\end{exercise}

\begin{proof}
    $\abs{6 + 4i} = \sqrt{6^{2} + 4^{2}} = \sqrt{52} = 2\sqrt{13}$.
\end{proof}

In Exercises 12 through 15 write the given complex number $z$ in the polar form $\abs{z}(p + qi)$ where $\abs{p + qi} = 1$.

\begin{exercise}
    $3 - 4i$
\end{exercise}

\begin{proof}
    $\abs{3 - 4i} = \sqrt{3^{2} + {(-4)}^{2}} = 5$.

    $3 - 4i = 5\left(\frac{3}{5} + \frac{-4}{5}i\right)$.
\end{proof}

\begin{exercise}
    $-1 + i$
\end{exercise}

\begin{proof}
    $\abs{-1 + i} = \sqrt{{(-1)}^{2} + 1^{2}} = \sqrt{2}$.

    $-1 + i = \sqrt{2}\left(\frac{-\sqrt{2}}{2} + \frac{\sqrt{2}}{2}i\right)$.
\end{proof}

\begin{exercise}
    $12 + 5i$
\end{exercise}

\begin{proof}
    $\abs{12 + 5i} = \sqrt{12^{2} + 5^{2}} = 13$.

    $12 + 5i = 13\left( \frac{12}{13} + \frac{5}{13}i \right)$.
\end{proof}

\begin{exercise}
    $-3 + 5i$
\end{exercise}

\begin{proof}
    $\abs{-3 + 5i} = \sqrt{{(-3)}^{2} + 5^{2}} = \sqrt{34}$.

    $-3 + 5i = \sqrt{34}\left( \frac{-3\sqrt{34}}{34} + \frac{5\sqrt{34}}{34}i \right)$.
\end{proof}

In Exercises 16 through 21, find all solutions in $\mathbb{C}$ of the given equation.

\begin{exercise}
    $z^{4} = 1$
\end{exercise}

\begin{proof}
    In polar form, the equation is $z^{4} = \abs{z}^{4}(\cos (4\phi) + i\sin (4\phi))$.

    $z^{4} = 1$ so $\abs{z}^{4} = 1$, $\cos(4\phi) = 1$, and $\sin(4\phi) = 0$. Different values of $\phi$ in $0\le \phi < 2\pi$ are $0, \frac{\pi}{2}, \pi, \frac{3\pi}{2}$. So the roots are
    \[
        1,\quad i,\quad -1,\quad -i.
    \]
\end{proof}

\begin{exercise}
    $z^{4} = -1$
\end{exercise}

\begin{proof}
    In polar form, the equation is $\abs{z}^{4}(\cos (4\phi) + i\sin (4\phi)) = 1(\cos\pi + i\sin\pi)$.

    The roots of the equation are
    \[
        \frac{\sqrt{2}}{2} + \frac{\sqrt{2}}{2}i,\quad \frac{-\sqrt{2}}{2} + \frac{\sqrt{2}}{2}i,\quad \frac{-\sqrt{2}}{2} + \frac{-\sqrt{2}}{2}i,\quad \frac{\sqrt{2}}{2} + \frac{-\sqrt{2}}{2}i.
    \]
\end{proof}

\begin{exercise}
    $z^{3} = -8$
\end{exercise}

\begin{proof}
    In polar form, the equation is $\abs{z}^{3}(\cos (3\phi) + i\sin (3\phi)) = 2^{3}(\cos\pi + i\sin\pi)$.

    The roots of the equation are
    \[
        1 + \sqrt{3}i,\quad -2,\quad 1 - \sqrt{3}i.
    \]
\end{proof}

\begin{exercise}
    $z^{3} = -27i$
\end{exercise}

\begin{proof}
    In polar form, the equation is $\abs{z}^{3}(\cos (3\phi) + i\sin (3\phi)) = 3^{3}(\cos\frac{3\pi}{2} + \sin\frac{3\pi}{2}i)$.

    The roots of the equation are
    \[
        3i,\quad \frac{-3\sqrt{3}}{2} + \frac{-3}{2}i,\quad\frac{3\sqrt{3}}{2} + \frac{-3}{2}i.
    \]
\end{proof}

\begin{exercise}
    $z^{6} = 1$
\end{exercise}

\begin{proof}
    The roots of the equation are
    \[
        1,\quad \frac{1}{2} + \frac{\sqrt{3}}{2}i,\quad \frac{-1}{2} + \frac{\sqrt{3}}{2}i, -1,\quad \frac{-1}{2} + \frac{-\sqrt{3}}{2}i,\quad \frac{1}{2} + \frac{-\sqrt{3}}{2}i.
    \]
\end{proof}

\begin{exercise}
    $z^{6} = -64$
\end{exercise}

\begin{proof}
    In polar form, the equation is $\abs{z}^{6}(\cos(6\phi) + i\sin(6\phi)) = 2^{6}(\cos\pi + i\sin\pi)$.

    pi/6 + 2pi/6 * 0 = pi/6
    pi/6 + 2pi/6 * 1 = 3pi/6 = pi/2
    pi/6 + 2pi/6 * 2 = 5pi/6
    pi/6 + 2pi/6 * 3 = 7pi/6
    pi/6 + 2pi/6 * 4 = 9pi/6
    pi/6 + 2pi/6 * 5 = 11pi/6

    The roots of the equation are
    \[
        \sqrt{3} + i,\quad 2i,\quad -\sqrt{3} + i,\quad -\sqrt{3} - i,\quad -2i,\quad \sqrt{3} - i.
    \]
\end{proof}

In Exercises 22 through 27, compute the given expression using the indicated modular addition.

\begin{exercise}
    $10 {+}_{17} 16$
\end{exercise}

\begin{proof}
    $10 {+}_{17} 16 = 10 + 16 - 17 = 9$.
\end{proof}

\begin{exercise}
    $8 {+}_{10} 6$
\end{exercise}

\begin{proof}
    $8 {+}_{10} 6 = 8 + 6 - 10 = 4$.
\end{proof}

\begin{exercise}
    $20.5 {+}_{25} 19.3$
\end{exercise}

\begin{proof}
    $20.5 {+}_{25} 19.3 = 20.5 + 19.3 - 25 = 14.8$.
\end{proof}

\begin{exercise}
    $\frac{1}{2} {+}_{1} \frac{7}{8}$
\end{exercise}

\begin{proof}
    $\frac{1}{2} {+}_{1} \frac{7}{8} = \frac{1}{2} + \frac{7}{8} - 1 = \frac{3}{8}$.
\end{proof}

\begin{exercise}
    $\frac{3\pi}{4} {+}_{2\pi} \frac{3\pi}{2}$
\end{exercise}

\begin{proof}
    $\frac{3\pi}{4} {+}_{2\pi} \frac{3\pi}{2} = \frac{3\pi}{4} + \frac{3\pi}{2} - 2\pi = \frac{\pi}{4}$.
\end{proof}

\begin{exercise}
    $2\sqrt{2} {+}_{\sqrt{32}} 3\sqrt{2}$
\end{exercise}

\begin{proof}
    $2\sqrt{2} {+}_{\sqrt{32}} 3\sqrt{2} = 2\sqrt{2} + 3\sqrt{2} - \sqrt{32} = 5\sqrt{2} - 4\sqrt{2} = \sqrt{2}$.
\end{proof}

\begin{exercise}
    Explain why the expression $5 {+}_{6} 8$ in $\mathbb{R}_{6}$ makes no sense.
\end{exercise}

\begin{proof}
    The expression makes no sense because $8\notin \mathbb{R}_{6}$.
\end{proof}

In Exercises 29 through 34, find \textit{all} solutions $x$ of the given equation.

\begin{exercise}
    $x {+}_{15} 7 = 3$ in $\mathbb{Z}_{15}$
\end{exercise}

\begin{proof}
    $x = 3 {+}_{15} 8 = 3 + 8 = 11$.
\end{proof}

\begin{exercise}
    $x {+}_{2\pi} \frac{3\pi}{2} = \frac{3\pi}{4}$ in $\mathbb{R}_{2\pi}$
\end{exercise}

\begin{proof}
    $x = \frac{3\pi}{4} {+}_{2\pi} \frac{\pi}{2} = \frac{3\pi}{4} + \frac{\pi}{2} = \frac{5\pi}{4}$.
\end{proof}

\begin{exercise}
    $x {+}_{7} x = 3$ in $\mathbb{Z}_{7}$
\end{exercise}

\begin{proof}
    $x = 5$.
\end{proof}

\begin{exercise}
    $x {+}_{7} x {+}_{7} x = 5$ in $\mathbb{Z}_{7}$
\end{exercise}

\begin{proof}
    $x = 4$.
\end{proof}

\begin{exercise}
    $x {+}_{12} x = 2$ in $\mathbb{Z}_{12}$
\end{exercise}

\begin{proof}
    $x = 1$, or $x = 7$.
\end{proof}

\begin{exercise}
    $x {+}_{4} x {+}_{4} x {+}_{4} x = 0$ in $\mathbb{Z}_{4}$
\end{exercise}

\begin{proof}
    $x = 0$, or $x = 1$, or $x = 2$, or $x = 3$.
\end{proof}

\begin{exercise}
    Example 1.15 asserts that there is an isomorphism of $U_{8}$ with $\mathbb{Z}_{8}$ in which $\zeta = e^{i(\pi/4)}\leftrightarrow 5$ and $\zeta^{2}\leftrightarrow 2$. Find the element of $\mathbb{Z}_{8}$ that corresponds to each of the remaining six elements $\zeta^{m}$ in $U_{8}$ for $m = 0, 3, 4, 5, 6$, and $7$.
\end{exercise}

\begin{proof}
    $\zeta^{0} \leftrightarrow 0$, $\zeta^{3} \leftrightarrow 7$, $\zeta^{4} \leftrightarrow 4$, $\zeta^{5}\leftrightarrow 1$, $\zeta^{6}\leftrightarrow 6$, and $\zeta^{7}\leftrightarrow 3$.
\end{proof}

\begin{exercise}
    There is an isomorphism of $U_{7}$ with $\mathbb{Z}_{7}$ in which $\zeta = e^{i(2\pi/7)}\leftrightarrow 4$. Find the element in $\mathbb{Z}_{7}$ to which $\zeta^{m}$ must correspond for $m = 0, 2, 3, 4, 5$, and $6$.
\end{exercise}

\begin{proof}
    $\zeta^{0} \leftrightarrow 0$, $\zeta^{2} \leftrightarrow 1$, $\zeta^{3} \leftrightarrow 5$, $\zeta^{4} \leftrightarrow 2$, $\zeta^{5} \leftrightarrow 6$, and $\zeta^{6} \leftrightarrow 3$.
\end{proof}

\begin{exercise}
    Why can there be no isomorphism of $U_{6}$ with $\mathbb{Z}_{6}$ in which $\zeta = e^{i(\pi/3)}$ corresponds to $4$?
\end{exercise}

\begin{proof}
    Assume that there is such an isomorphism.

    Then $\zeta^{3}$ corresponds to $4 {+}_{12} 4 {+}_{12} 4 = 0$. On the other hand, $\zeta^{0}$ corresponds to $0$, which is a contradiction.

    Hence there can be no isophism of $U_{6}$ with $\mathbb{Z}_{6}$ in which $\zeta = e^{i(\pi/3)}\leftrightarrow 4$.
\end{proof}

\begin{exercise}
    Derive the formulas
    \[
        \sin(a + b) = \sin a\cos b + \cos a\sin b
    \]

    and
    \[
        \cos(a + b) = \cos a\cos b - \sin a\sin b
    \]

    by using Euler's formula and computing $e^{ia}e^{ib}$.
\end{exercise}

\begin{proof}
    $e^{ia}e^{ib} = e^{i(a+b)} = \cos(a+b) + i\sin(a+b)$.

    $e^{ia}e^{ib} = (\cos a + i\sin a)(\cos b + i\sin b) = (\cos a\cos b - \sin a\sin b) + i(\sin a\cos b + \cos a\sin b)$.

    Thus $\sin(a + b) = \sin a\cos b + \cos a\sin b$ and $\cos(a + b) = \cos a\cos b - \sin a\sin b$.
\end{proof}

\begin{exercise}
    Let $z_{1} = \abs{z_{1}}(\cos{\theta_{1}} + i\sin{\theta_{1}})$ and $z_{2} = \abs{z_{2}}(\cos{\theta_{2}} + i\sin{\theta_{2}})$. Use the trigonometric identities in Exercise 38 to derive $z_{1}z_{2} = \abs{z_{1}}\abs{z_{2}}(\cos{(\theta_{1} + \theta_{2})} + i\sin{(\theta_{1} + \theta_{2})})$.
\end{exercise}

\begin{proof}
    \begin{align*}
        z_{1}z_{2} & = \abs{z_{1}}\abs{z_{2}}(\cos a + i\sin a)(\cos b + i\sin b)                             \\
                   & = \abs{z_{1}}\abs{z_{2}}((\cos a\cos b - \sin a\sin b) + i(\sin a\cos b + \cos a\sin b)) \\
                   & = \abs{z_{1}}\abs{z_{2}}(\cos{(a+b)} + i\sin{(a+b)}).
    \end{align*}
\end{proof}

\begin{exercise}
    \begin{enumerate}[topsep=0pt,itemsep=0pt,label={\textbf{\alph*.}}]
        \item Derive a formula for $\cos{3\theta}$ in terms of $\sin{\theta}$ and $\cos{\theta}$ using Euler's formula.
        \item Derive the formula $\cos{3\theta} = 4\cos^{3}{\theta} - 3\cos{\theta}$ from part (a) and the identity $\sin^{2}{\theta} + \cos^{2}{\theta} = 1$.
    \end{enumerate}
\end{exercise}

\begin{proof}
    \begin{enumerate}[topsep=0pt,itemsep=0pt,label={\textbf{\alph*.}}]
        \item \begin{align*}
                  \cos{3\theta} & = \cos{2\theta}\cos{\theta} - \sin{\theta}\sin{2\theta}                             \\
                                & = \cos{\theta}(\cos^{2}{\theta} - \sin^{2}{\theta}) - 2\sin^{2}{\theta}\cos{\theta} \\
                                & = \cos^{3}{\theta} - \cos{\theta}\sin^{2}{\theta} - 2\sin^{2}{\theta}\cos{\theta}.
              \end{align*}
        \item \begin{align*}
                  \cos{3\theta} & = \cos^{3}{\theta} - \cos{\theta}\sin^{2}{\theta} - 2\sin^{2}{\theta}\cos{\theta}             \\
                                & = \cos^{3}{\theta} - \cos{\theta}(1 - \cos^{2}{\theta}) - 2(1 - \cos^{2}{\theta})\cos{\theta} \\
                                & = \cos^{3}{\theta} - \cos{\theta} + \cos^{3}{\theta} + 2\cos^{3}{\theta} - 2\cos{\theta}      \\
                                & = 4\cos^{3}{\theta} - 3\cos{\theta}.
              \end{align*}
    \end{enumerate}
\end{proof}

\begin{exercise}
    Recall the power series expansions
    \begin{align*}
        e^{x}  & = 1 + x + \frac{x^{2}}{2!} + \frac{x^{3}}{3!} + \frac{x^{4}}{4!} + \cdots + \frac{x^{n}}{n!} + \cdots,                            \\
        \sin x & = x - \frac{x^{3}}{3!} + \frac{x^{5}}{5!} - \frac{x^{7}}{7!} + \cdots + {(-1)}^{n-1}\frac{x^{2n-1}}{(2n-1)!} + \cdots, \text{and} \\
        \cos x & = 1 - \frac{x^{2}}{2!} + \frac{x^{4}}{4!} - \frac{x^{6}}{6!} + \cdots + {(-1)}^{n}\frac{x^{2n}}{(2n)!} + \cdots
    \end{align*}

    from calculus. Derive Euler's formula $e^{i\theta} = \cos{\theta} + i\sin{\theta}$ formally from these three series expansions.
\end{exercise}

\begin{proof}
    \begin{align*}
        e^{i\theta} & = 1 + i\theta + \frac{{(i\theta)}^{2}}{2!} + \frac{{(i\theta)}^{3}}{3!} + \frac{{(i\theta)}^{4}}{4!} + \cdots                                                                 \\
                    & = \left(1 + \frac{{(i\theta)}^{2}}{2!} + \frac{{(i\theta)}^{4}}{4!} + \cdots\right) + \left(i\theta + \frac{{(i\theta)}^{3}}{3!} + \frac{{(i\theta)}^{5}}{5!} + \cdots\right) \\
                    & = \left(1 - \frac{x^{2}}{2!} + \frac{x^{4}}{4!} + \cdots \right) + \left(i\theta - i\frac{\theta^{3}}{3!} + i\frac{\theta^{5}}{5!} - \cdots \right)                           \\
                    & = \cos{\theta} + i\sin{\theta}.\qedhere
    \end{align*}
\end{proof}

\section{Isomorphic Binary Structures}
\setcounter{exercise}{0}

In all the exercises, $+$ is the usual addition on the set where it is specified, and $\cdot$ is the usual multiplication.

\textbf{Computations}

\begin{exercise}
    What three things must we check to determine whether a function $\phi: S\to S'$ is an isomorphism of a binary structure $\anglebracket{S, *}$ with $\anglebracket{S', *'}$.
\end{exercise}

\begin{proof}
    Those three things are: $\phi$ is an one-to-one function, $\phi$ is onto $S'$, and $\phi(x * y) = \phi(x) *' \phi(y)$ for every $x, y\in S$.
\end{proof}

In Exercises 2 through 10, determine whether the given map $\phi$ is an isomorphism of the first binary structure with the second. (See Exercise 1.) If it is not an isomorphism, why not?

\begin{exercise}
    $\anglebracket{\mathbb{Z}, +}$ with $\anglebracket{\mathbb{Z}, +}$ where $\phi(n) = -n$ for $n\in\mathbb{Z}$
\end{exercise}

\begin{proof}
    $\phi(n) = \phi(m)$ if and only if $-n = -m$. $-n = -m$ if and only if $n = m$. So $\phi$ is an one-to-one function.

    For each $n\in\mathbb{Z}$, $\phi(-n) = n$. So $\phi$ is onto $\mathbb{Z}$.

    For every $m, n\in\mathbb{Z}$, $\phi(m + n) = -(m + n) = (-m) + (-n) = \phi(m) + \phi(n)$.

    Hence $\phi$ is an isomorphism.
\end{proof}

\begin{exercise}
    $\anglebracket{\mathbb{Z}, +}$ with $\anglebracket{\mathbb{Z}, +}$ where $\phi(n) = 2n$ for $n\in\mathbb{Z}$
\end{exercise}

\begin{proof}
    Since $\phi(n) = 2n$, there is no integer $x$ such that $\phi(x)$ is an odd integer. So $\phi$ is not onto $\mathbb{Z}$.

    Hence $\phi$ is not an isomorphism.
\end{proof}

\begin{exercise}
    $\anglebracket{\mathbb{Z}, +}$ with $\anglebracket{\mathbb{Z}, +}$ where $\phi(n) = n + 1$ for $n\in\mathbb{Z}$
\end{exercise}

\begin{proof}
    $\phi(n) = \phi(m)$ if and only if $n + 1 = m + 1$. $n + 1 = m + 1$ if and only if $n = m$. So $\phi$ is an one-to-one function.

    For each $n\in\mathbb{Z}$, $\phi(n - 1) = n$. So $\phi$ is onto $\mathbb{Z}$.

    For every $m, n\in\mathbb{Z}$, $\phi(m + n) = m + n + 1 = (m + 1) + (n + 1) - 1 = \phi(m) + \phi(n) - 1 \ne \phi(m) + \phi(n)$.

    Hence $\phi$ is not an isomorphism.
\end{proof}

\begin{exercise}
    $\anglebracket{\mathbb{Q}, +}$ with $\anglebracket{\mathbb{Q}, +}$ where $\phi(x) = x/2$ for $x\in\mathbb{Q}$
\end{exercise}

\begin{proof}
    $\phi(x) = \phi(y)$ if and only if $x/2 = y/2$. $x/2 = y/2$ if and only if $x = y$. So $\phi$ is an one-to-one function.

    For each $x\in\mathbb{Q}$, $\phi(2x) = x$. So $\phi$ is onto $\mathbb{Q}$.

    For every $x, y\in\mathbb{Q}$, $\phi(x\cdot y) = x\cdot y/2 = 2\cdot(x/2)\cdot(y/2) = 2\cdot\phi(x)\cdot\phi(y)$. If $x, y\ne 0$, $\phi(x\cdot y) \ne \phi(x)\cdot\phi(y)$.

    Hence $\phi$ is not an isomorphism.
\end{proof}

\begin{exercise}
    $\anglebracket{\mathbb{Q},\cdot}$ with $\anglebracket{\mathbb{Q},\cdot}$ where $\phi(x) = x^{2}$ for $x\in\mathbb{Q}$.
\end{exercise}

\begin{proof}
    There is no rational number $q$ such that $\phi(q) = q^{2} = 2$. So $\phi$ is not onto $\mathbb{Q}$.

    Hence $\phi$ is not an isomorphism.
\end{proof}

\begin{exercise}
    $\anglebracket{\mathbb{R},\cdot}$ with $\anglebracket{\mathbb{R},\cdot}$ where $\phi(x) = x^{3}$
\end{exercise}

\begin{proof}
    $\phi(x) = \phi(y)$ if and only if $x^{3} = y^{3}$. $x^{3} = y^{3}$ if and only if $x = y$ ($x, y\in\mathbb{R}$). So $\phi$ is an one-to-one function.

    For each $x\in\mathbb{R}$, there exists a real number $a$ such that $a^{3} = x$. Equivalently, $\phi(a) = a^{3} = x$. So $\phi$ is onto $\mathbb{R}$.

    For every $x, y\in\mathbb{R}$, $\phi(x\cdot y) = {(x\cdot y)}^{3} = x^{3}\cdot y^{3} = \phi(x)\cdot\phi(y)$.

    Hence $\phi$ is an isomorphism.
\end{proof}

\begin{exercise}
    $\anglebracket{M_{2}(\mathbb{R}), \cdot}$ with $\anglebracket{\mathbb{R},\cdot}$ where $\phi(A)$ is the determinant of matrix $A$
\end{exercise}

\begin{proof}
    There are different $2\times 2$ matrices with the same determinant. For examples:
    \[
        \begin{vmatrix}
            1 & 0 \\
            0 & 1
        \end{vmatrix}
        =
        \begin{vmatrix}
            1 & 1 \\
            0 & 1
        \end{vmatrix}.
    \]

    So $\phi$ is not an one-to-one function.

    Hence $\phi$ is not an isomorphism.
\end{proof}

\begin{exercise}
    $\anglebracket{M_{1}(\mathbb{R}),\cdot}$ with $\anglebracket{\mathbb{R},\cdot}$ where $\phi(A)$ is the determinant of matrix $A$
\end{exercise}

\begin{proof}
    $\phi(A) = \phi(B)$ if and only if $\det(A) = \det(B)$. Since $A, B$ are $1\times 1$ matrices, $\det(A) = \det(B)$ if and only if $A = B$. So $\phi$ is an one-to-one function.

    For every $x\in\mathbb{R}$, $\phi(\begin{bmatrix}x\end{bmatrix})\det\begin{bmatrix}x\end{bmatrix} = x$. So $\phi$ is onto $\mathbb{R}$.

    For every $A, B\in M_{1}(\mathbb{R})$, $\phi(A\cdot B) = \det(A\cdot B) = \det(A)\cdot\det(B) = \phi(A)\cdot\phi(B)$.

    Hence $\phi$ is an isomorphism.
\end{proof}

\begin{exercise}
    $\anglebracket{\mathbb{R}, +}$ with $\anglebracket{\mathbb{R}^{+},\cdot}$ where $\phi(r) = {0.5}^{r}$ for $r\in\mathbb{R}$
\end{exercise}

\begin{proof}
    $\phi(x) = \phi(y)$ if and only if ${0.5}^{x} = {0.5}^{y}$. ${0.5}^{x} = {0.5}^{y}$ if and only if $\log_{0.5}{0.5}^{x} = \log_{0.5}{0.5}^{y}$, in other words, $x = y$. So $\phi$ is an one-to-one function.

    For each $m\in\mathbb{R}^{+}$, $\phi(\log_{0.5}m) = {0.5}^{\log_{0.5}m} = m$. So $\phi$ is onto $\mathbb{R}^{+}$.

    For every $x, y\in\mathbb{R}$, $\phi(x + y) = {0.5}^{x+y} = {0.5}^{x}\cdot{0.5}^{y} = \phi(x)\cdot\phi(y)$.

    Hence $\phi$ is an isomorphism.
\end{proof}

In Exercises 11 through 15, let $F$ be the set of all functions $f$ mapping $\mathbb{R}$ into $\mathbb{R}$ that have deravatives of all orders. Follow the instructions for Exercies 2 through 10.

\begin{exercise}
    $\anglebracket{F,+}$ with $\anglebracket{F,+}$ where $\phi(f) = f'$, the derivative of $f$
\end{exercise}

\begin{proof}
    Let $f(x) = x$ and $g(x) = x + 1$. $f'(x) = g'(x) = 1$. So $\phi$ is not an one-to-one function.

    Hence $\phi$ is not an isomorphism.
\end{proof}

\begin{exercise}
    $\anglebracket{F, +}$ with $\anglebracket{\mathbb{R}, +}$ where $\phi(f) = f'(0)$
\end{exercise}

\begin{proof}
    Let $f(x) = x$ and $g(x) = x + 1$. $f'(0) = g'(0) = 1$. So $\phi$ is not an one-to-one function.

    Hence $\phi$ is not an isomorphism.
\end{proof}

\begin{exercise}
    $\anglebracket{F,+}$ with $\anglebracket{F,+}$ where $\phi(f)(x) = \int^{x}_{0}f(t)dt$.
\end{exercise}

\begin{proof}
    If $f\in F$, $f$ has antiderivatives. If $g$ is an antiderivative of $f$, then every antiderivative $h$ of $f$ satisfies $h(x) = g(x) + C$, where $C$ is a constant. But there is a unique antiderivative that has zero value at $x = 0$, which is $\int^{x}_{0}f(t)dt$. So $\phi$ is an one-to-one function.

    Since $\phi(f)$ has zero value at $x = 0$, so for every $g\in F$ such that $g(0)\ne 0$, there is no $h\in F$ such that $\phi(h) = g$. So $\phi$ is not onto $\anglebracket{F,+}$.

    Hence $\phi$ is not an isomorphism.
\end{proof}

\begin{exercise}
    $\anglebracket{F,+}$ with $\anglebracket{F,+}$ where $\phi(f)(x) = \frac{d}{dx}\left(\int^{x}_{0} f(t)dt\right)$
\end{exercise}

\begin{proof}
    $f\in F$, then $f$ has antiderivate. According to \textit{the fundamental theorem of calculus}
    \[
        \frac{d}{dx}\left(\int^{x}_{0}f(t)dt\right) = f(x) - f(0)
    \]

    $\phi(f) = \phi(g)$ if and only if $\phi(f)(x) = \phi(g)(x)$ for every $x\in\mathbb{R}$. $\phi(f)(x) = \phi(g)(x)$ for every $x\in\mathbb{R}$ if and only if $f(x) - f(0) = g(x) - g(0)$ for every $x\in\mathbb{R}$, in other words, $f = g$. So $\phi$ is an one-to-one function.

    Let $f(x) = x + 1$, then $f(0) = 1\ne 0$. There is no function $g\in F$ such that $\phi(g)(x) = f(x)$, since $\phi(g)(0) = g(0) - g(0) = 0 \ne 1 = f(0)$.

    Hence $\phi$ is not an isomorphism.
\end{proof}

\begin{exercise}
    $\anglebracket{F,\cdot}$ with $\anglebracket{F,\cdot}$ where $\phi(f)(x) = x\cdot f(x)$
\end{exercise}

\begin{proof}
    $\phi(f) = \phi(g)$ if and only if $\phi(f)(x) = \phi(g)(x)$ for every $x\in\mathbb{R}$. $\phi(f)(x) = \phi(g)(x)$ for every $x\in\mathbb{R}$ if and only if $x\cdot f(x) = x\cdot g(x)$ for every $x\in\mathbb{R}$. If $x\ne 0$, then $f(x) = g(x)$. Otherwise, since $f$ and $g$ are continuous at $x = 0$, for every $\varepsilon > 0$, there exists $\delta_{f}(\varepsilon)$ and $\delta_{f}(\varepsilon)$ such that $\abs{f(x) - f(0)} < \frac{\varepsilon}{2}$ if $0 < \abs{x} < \delta_{f}(\varepsilon)$ and $\abs{g(x) - g(0)} < \frac{\varepsilon}{2}$ if $0 < \abs{x} < \delta_{g}(\varepsilon)$. For $0 < \abs{x} < \min\{ \delta_{f}(\varepsilon), \delta_{f}(\varepsilon) \}$, $\abs{f(0) - g(0)}\le \abs{f(0) - f(x)} + \abs{f(x) - g(x)} + \abs{g(x) - g(0)} = \varepsilon/2 + 0 + \varepsilon/2 = \varepsilon$, which implies $f(0) = g(0)$. So $f(x) = g(x)$ for every $x\in\mathbb{R}$. So $\phi$ is an one-to-one function.

    There is no function $f\in F$ such that $x\cdot f(x) = 1$ for every $x\in\mathbb{R}$. So $\phi$ is not onto $F$.

    For $x\ne 0$ and $x\ne 1$, $f(x) = 1, g(x) = 1$, we have
    \[
        \phi(f\cdot g)(x) = x\cdot (f\cdot g)(x) = x \ne x^{2} = (x\cdot f(x))\cdot (y\cdot g(x)) = \phi(f)(x) \cdot \phi(g)(x) = (\phi(f)\cdot\phi(g))(x)
    \]

    Hence $\phi$ is not an isomorphism.
\end{proof}

\begin{exercise}
    The map $\phi: \mathbb{Z} \to \mathbb{Z}$ define by $\phi(n) = n + 1$ for $n\in\mathbb{Z}$ is one-to-one and onto $\mathbb{Z}$. Give the definition of a binary operation $*$ on $\mathbb{Z}$ such that $\phi$ is an isomorphism mapping
    \begin{enumerate}[label={\textbf{\alph*}},itemsep=0pt,topsep=0pt]
        \item $\anglebracket{\mathbb{Z}, +}$ onto $\anglebracket{\mathbb{Z}, *}$
        \item $\anglebracket{\mathbb{Z}, *}$ onto $\anglebracket{\mathbb{Z}, +}$
    \end{enumerate}

    In each case, give the identity element for $*$ on $\mathbb{Z}$.
\end{exercise}

\begin{proof}
    \begin{enumerate}[label={\textbf{\alph*}},itemsep=0pt,topsep=0pt]
        \item Define $*: \mathbb{Z}\times\mathbb{Z}\to\mathbb{Z}$ as follows: $m * n = m + n - 1$ for every $m, n\in\mathbb{Z}$. Then
              \[
                  \phi(m + n) = m + n + 1 = (m + 1) + (n + 1) - 1 = \phi(m) + \phi(n) - 1 = \phi(m) * \phi(n).
              \]

              The identity element of $*$ on $\mathbb{Z}$ is $1$.
        \item Define $*: \mathbb{Z}\times\mathbb{Z}\to\mathbb{Z}$ as follows: $m * n = m + n + 1$ for every $m, n\in\mathbb{Z}$. Then
              \[
                  \phi(m * n) = \phi(m + n + 1) = m + n + 2 = (m + 1) + (n + 1) = \phi(m) + \phi(n).
              \]

              The identity element of $*$ on $\mathbb{Z}$ is $-1$.
    \end{enumerate}
\end{proof}

\begin{exercise}
    The map $\phi: \mathbb{Z} \to \mathbb{Z}$ define by $\phi(n) = n + 1$ for $n\in\mathbb{Z}$ is one-to-one and onto $\mathbb{Z}$. Give the definition of a binary operation $*$ on $\mathbb{Z}$ such that $\phi$ is an isomorphism mapping
    \begin{enumerate}[label={\textbf{\alph*}},itemsep=0pt,topsep=0pt]
        \item $\anglebracket{\mathbb{Z}, \cdot}$ onto $\anglebracket{\mathbb{Z}, *}$
        \item $\anglebracket{\mathbb{Z}, *}$ onto $\anglebracket{\mathbb{Z}, \cdot}$
    \end{enumerate}

    In each case, give the identity element for $*$ on $\mathbb{Z}$.
\end{exercise}

\begin{proof}
    \begin{enumerate}[label={\textbf{\alph*}},itemsep=0pt,topsep=0pt]
        \item Define $*: \mathbb{Z}\times\mathbb{Z}\to\mathbb{Z}$ as follows: $m * n = mn - m - n + 2$ for every $m, n\in\mathbb{Z}$. Then
              \begin{align*}
                  \phi(m\cdot n) & = m\cdot n + 1                                 \\
                                 & = m\cdot n + m + n + 1 - (m + 1) - (n + 1) + 2 \\
                                 & = (m + 1)(n + 1) - (m + 1) - (n + 1) + 2       \\
                                 & = \phi(m)\cdot\phi(n) - \phi(m) - \phi(n) + 2  \\
                                 & = \phi(m) * \phi(n)
              \end{align*}

              The identity element of $*$ in $\mathbb{Z}$ is $2$.
        \item Define $*: \mathbb{Z}\times\mathbb{Z}\to\mathbb{Z}$ as follows: $m * n = mn + m + n$ for every $m, n\in\mathbb{Z}$. Then
              \begin{align*}
                  \phi(m * n) & = m * n + 1            \\
                              & = mn + m + n + 1       \\
                              & = (m + 1)\cdot (n + 1) \\
                              & = \phi(m)\cdot\phi(n)
              \end{align*}

              The identity element of $*$ in $\mathbb{Z}$ is $0$.
    \end{enumerate}
\end{proof}

\begin{exercise}
    The map $\phi: \mathbb{Q} \to \mathbb{Q}$ defined by $\phi(x) = 3x - 1$ for $x\in\mathbb{Q}$ is one-to-one and onto $\mathbb{Z}$. Give the definition of a binary operation $*$ on $\mathbb{Q}$ such that $\phi$ is an isomorphism mapping
    \begin{enumerate}[label={\textbf{\alph*}},itemsep=0pt,topsep=0pt]
        \item $\anglebracket{\mathbb{Q}, +}$ onto $\anglebracket{\mathbb{Q}, *}$
        \item $\anglebracket{\mathbb{Q}, *}$ onto $\anglebracket{\mathbb{Q}, +}$
    \end{enumerate}

    In each case, give the identity element for $*$ on $\mathbb{Q}$.
\end{exercise}

\begin{proof}
    \begin{enumerate}[label={\textbf{\alph*}},itemsep=0pt,topsep=0pt]
        \item Define $*: \mathbb{Q}\times\mathbb{Q} \to \mathbb{Q}$ as follows: $x * y = x + y + 1$ for every $x, y\in\mathbb{Q}$. Then
              \[
                  \phi(x + y) = 3(x + y) - 1 = (3x - 1) + (3y - 1) + 1 = \phi(x) + \phi(y) + 1 = \phi(x) * \phi(y).
              \]

              The identity element of $*$ on $\mathbb{Q}$ is $-1$.
        \item Define $*: \mathbb{Q}\times\mathbb{Q} \to \mathbb{Q}$ as follows: $x * y = x + y - \frac{1}{3}$ for every $x, y\in\mathbb{Q}$. Then
              \[
                  \phi(x * y) = 3\cdot(x * y) - 1 = 3\cdot\left(x + y - \frac{1}{3}\right) - 1 = 3x + 3y - 2 = (3x - 1) + (3y - 1) = \phi(x) + \phi(y).
              \]

              The identity element of $*$ on $\mathbb{Q}$ is $\frac{1}{3}$.
    \end{enumerate}
\end{proof}

\begin{exercise}
    The map $\phi: \mathbb{Q} \to \mathbb{Q}$ defined by $\phi(x) = 3x - 1$ for $x\in\mathbb{Q}$ is one-to-one and onto $\mathbb{Z}$. Give the definition of a binary operation $*$ on $\mathbb{Q}$ such that $\phi$ is an isomorphism mapping
    \begin{enumerate}[label={\textbf{\alph*}},itemsep=0pt,topsep=0pt]
        \item $\anglebracket{\mathbb{Q}, \cdot}$ onto $\anglebracket{\mathbb{Q}, *}$
        \item $\anglebracket{\mathbb{Q}, *}$ onto $\anglebracket{\mathbb{Q}, \cdot}$
    \end{enumerate}

    In each case, give the identity element for $*$ on $\mathbb{Q}$.
\end{exercise}

\begin{proof}
    \item Define $*: \mathbb{Q}\times\mathbb{Q} \to \mathbb{Q}$ as follows: $x * y = \frac{1}{3}xy + \frac{1}{3}x + \frac{1}{3}y - \frac{2}{3}$ for every $x, y\in\mathbb{Q}$. Then
    \begin{align*}
        \phi(x\cdot y) & = 3xy - 1 = \frac{1}{3}(9xy - 3x - 3y + 1) + x + y - \frac{4}{3}                        \\
                       & = \frac{1}{3}(3x - 1)(3y - 1) + \frac{1}{3}(3x - 1) + \frac{1}{3}(3y - 1) - \frac{2}{3} \\
                       & = \frac{1}{3}\phi(x)\phi(y) + \frac{1}{3}x + \frac{1}{3}y - \frac{2}{3}                 \\
                       & = \phi(x) * \phi(y)
    \end{align*}

    The identity element of $*$ on $\mathbb{Q}$ is $2$.
    \item Define $*: \mathbb{Q}\times\mathbb{Q} \to \mathbb{Q}$ as follows: $x * y = 3xy - x - y + \frac{2}{3}$ for every $x, y\in\mathbb{Q}$. Then
    \begin{align*}
        \phi(x * y) & = 3\cdot (x * y) - 1    \\
                    & = 9xy - 3x - 3y + 2 - 1 \\
                    & = 9xy - 3x - 3y + 1     \\
                    & = (3x - 1)(3y - 1)
    \end{align*}

    The identity element of $*$ on $\mathbb{Q}$ is $\frac{2}{3}$.
\end{proof}

\textbf{Concepts}

\begin{exercise}
    The displayed homomorphism condition for an isomorphism $\phi$ in Definition 3.7 is sometimes summarized by saying ``$\phi$ must commute with the binary operation\@(s).\@'' Explain how that condition can be viewed in this manner.
\end{exercise}

\begin{proof}
    $\phi: S\to S'$, let's call $\phi(x)$ the image of $x$, where $x\in S$.

    The image of the combination of $x, y\in S$ (with respect to $*$) is the combination of the images of $x$ and $y$ (with respect to $*'$).
\end{proof}

In Exercises 21 and 22, correct the definition of the italicized term without reference to the text, if correction is needed, so that it is in a form acceptable for publication.

\begin{exercise}
    A function $\phi: S\to S'$ is an \textit{isomorphism} if and only if $\phi(a * b) = \phi(a) *' \phi(b)$.
\end{exercise}

\begin{proof}
    Correction: A function $\phi: S\to S'$ is an \textit{isomorphism} if and only if $\phi$ is an one-to-one function from $S$ onto $S'$ and $\phi(a * b) = \phi(a) *' \phi(b)$.
\end{proof}

\begin{exercise}
    Let $*$ be a binary operation on a set $S$. An element $e$ of $S$ with the property $s * e = s = e * s$ is an \textit{identity element for $*$}  for all $s\in S$.
\end{exercise}

\begin{proof}
    No correction is needed.
\end{proof}

\textbf{Proof Synopnis}

\begin{exercise}
    Give a proof synopsis of Theorem 3.13.
\end{exercise}

\begin{proof}
    We suppose that $e$ and $e'$ are identity elements of the binary operation $*$. We combine the two identity elements using the binary operation $*$ to deduce that $e = e'$.
\end{proof}

\textbf{Theory}

\begin{exercise}
    An identity element for a binary operation $*$ as described by Definition 3.12 is sometimes referred to as ``a two-sided identity element.\@'' Using complete sentences, give analogous definition for
    \begin{enumerate}[label={\textbf{\alph*.}},itemsep=0pt]
        \item a \textit{left identity element $e_{L}$ for $*$},
        \item a \textit{right identity element $e_{R}$ for $*$}.
    \end{enumerate}

    Theorem 3.13 shows that if a two-sided identity element for $*$ exists, it is unique. Is the same true for a one-sided identity element you just defined? If so, prove it. If not, give a counterexample $\anglebracket{S,*}$ for a finite set $S$ and find the first place where the proof of Theorem 3.13 breaks down.
\end{exercise}

\begin{proof}
    \begin{enumerate}[label={\textbf{\alph*.}},itemsep=0pt]
        \item A \textit{left identity element $e_{L}$ for $*$} is an element that yields $x$ when combined from the left with $x$ for every element $x$ by binary operation $*$.
        \item A \textit{right identity element $e_{R}$ for $*$} is an element that yields $x$ when combined from the right with $x$ for every element $x$ by binary operation $*$.
    \end{enumerate}

    Left identity element, if exists, is not necessarily unique. Example: $S = \{ 0, 1 \}$, $*: S\times S\to S$ is defined as $0 * 0 = 0, 0 * 1 = 1, 1 * 0 = 0, 1 * 1 = 1$. In this example, $0$ and $1$ are left identity elements.

    Right identity element, if exists, is not necessarily unique. Example: $S = \{ 0, 1 \}$, $*: S\times S\to S$ is defined as $0 * 0 = 0, 0 * 1 = 0, 1 * 0 = 1, 1 * 1 = 1$. In this example, $0$ and $1$ are right identity elements.
\end{proof}

\begin{exercise}
    Continuing the ideas of Exercise 24, can a binary structure have a left identity $e_{L}$ and a right identity $e_{R}$ where $e_{L}\ne e_{R}$? If so, give an example, using an operation on a finite set $S$. If not, prove that it is impossible.
\end{exercise}

\begin{proof}
    Such binary structure does not exist.

    $e_{L} * x = x$ for every $x\in S$, so $e_{L} * e_{R} = e_{R}$. On the other hand, $x * e_{R} = x$ for every $x\in S$, so $e_{L} * e_{R} = e_{L}$. Hence $e_{L} = e_{R}$.

    Thus, if a binary structure have a left identity and a right identity, they must be identical.
\end{proof}

\begin{exercise}
    Recall that if $f: A \to B$ is an one-to-one function mapping $A$ onto $B$, then $\phi^{-1}(b)$ is the unique $a\in A$ such that $f(a) = b$. Prove that if $\phi: S \to S'$ is an isomorphism of $\anglebracket{S, *}$ with $\anglebracket{S', *'}$, then $\phi^{-1}$ is an isomorphism of $\anglebracket{S', *'}$ with $\anglebracket{S, *}$.
\end{exercise}

\begin{proof}
    $\phi$ is an one-to-one function mapping $A$ onto $B$, then $\phi^{-1}$ is an one-to-one function mapping $B$ onto $A$.

    For every $x', y'\in S$, there exist uniquely $x, y\in S$ such that $\phi(x) = x'$ and $\phi(y) = y'$.
    \begin{align*}
        \phi^{-1}(x' *' y') & = \phi^{-1}(\phi(x) *' \phi(y)) \\
                            & = \phi^{-1}(\phi(x * y))        \\
                            & = x * y                         \\
                            & = \phi^{-1}(x') * \phi^{-1}(y')
    \end{align*}

    Hence $\phi^{-1}$ is an isomorphism of $\anglebracket{S', *'}$ with $\anglebracket{S, *}$.
\end{proof}

\begin{exercise}
    Prove that if $\phi: S \to S'$ is an isomorphism of $\anglebracket{S, *}$ with $\anglebracket{S', *}$ and $\psi: S' \to S''$ is an isomorphism of $\anglebracket{S', *'}$ with $\anglebracket{S'', *''}$, then the composite function $\psi\circ\phi$ is an isomorphism of $\anglebracket{S, *}$ with $\anglebracket{S'', *''}$.
\end{exercise}

\begin{proof}
    Because $\psi$ is one-to-one function, $(\psi\circ\phi)(x) = (\psi\circ\phi)(y)$ if and only if $\phi(x) = \phi(y)$. Because $\phi$ is one-to-one function, $\phi(x) = \phi(y)$ if and only if $x = y$. Therefore, $(\psi\circ\phi)(x) = (\psi\circ\phi)(y)$ if and only if $x = y$. So $\psi\circ\phi$ is an one-to-one function from $S$ into $S''$.

    Let $x''$ be an element of $S''$. Because $\psi$ is one-to-one function, there exists uniquely $x'\in S'$ such that $\psi(x') = x''$. Because $\phi$ is one to one function, there exists uniquely $x\in S$ such that $\phi(x) = x'$. So for every $x''\in S''$, there exists uniquely $x\in S$ such that $(\psi\circ\phi)(x) = x''$, equivalently, $\psi\circ\phi$ is onto $S''$.

    For every $x, y\in S$
    \begin{align*}
        (\psi\circ\phi)(x * y) & = \psi(\phi(x * y))                         \\
                               & = \psi(\phi(x) *' \phi(y))                  \\
                               & = \psi(\phi(x)) *'' \psi(\phi(y))           \\
                               & = (\psi\circ\phi)(x) *'' (\psi\circ\phi)(y)
    \end{align*}

    Hence $\psi\circ\phi$ is an isomorphism of $\anglebracket{S, *}$ with $\anglebracket{S'', *''}$.
\end{proof}

\begin{exercise}
    Prove that the relation $\simeq$ of being isomorphic, described in Definition 3.7 is an equivalence relation on any set of binary structures. You may simply quote the results you were asked to prove in the preceding two exercises at appropriate places in your proof.
\end{exercise}

\begin{proof}
    A binary structure $\anglebracket{S, *}$ is isomorphic to itself (follows from the identity mapping from $S$ to $S$, which is an isomorphism). So $\simeq$ is reflexive.

    If $\anglebracket{S, *}$ is isomorphic to $\anglebracket{S', *'}$, then $\anglebracket{S', *'}$ is isomorphic to $\anglebracket{S, *}$ (Exercise 26). So $\simeq$ is symmetric.

    If $\anglebracket{S, *}$ is isomorphic to $\anglebracket{S', *'}$, and $\anglebracket{S', *'}$ is isomorphic to $\anglebracket{S'', *''}$, then $\anglebracket{S, *}$ is isomorphic to $\anglebracket{S'', *''}$ (Exercise 27). So $\simeq$ is transitive.

    Hence the relation $\simeq$ is an equivalence relation on any set of binary structures.
\end{proof}

In Exercises 29 through 32, give a careful proof for a skeptic that the indicated property of a binary structure $\anglebracket{S, *}$ is indeed a structural property. (In Theorem 3.14, we did this for the property, ``There is an identity element for *.\@'')

\begin{exercise}
    The operation $*$ is commutative.
\end{exercise}

\begin{proof}
    Let $\phi: S\to S'$ be an isomorphism of $\anglebracket{S, *}$ with $\anglebracket{S', *'}$.

    For every $x', y'\in S$, there exist uniquely $x, y\in S$ such that $\phi(x) = x'$ and $\phi(y) = y'$.
    \begin{align*}
        x' *' y' & = \phi(x) *' \phi(y) \\
                 & = \phi(x * y)        \\
                 & = \phi(y * x)        \\
                 & = \phi(y) *' \phi(x) \\
                 & = y' * x'
    \end{align*}

    Hence $*'$ is also commutative.
\end{proof}

\begin{exercise}
    The operation $*$ is associative.
\end{exercise}

\begin{proof}
    Let $\phi: S\to S'$ be an isomorphism of $\anglebracket{S, *}$ with $\anglebracket{S', *'}$.

    For every $x', y', z'\in S$, there exist uniquely $x, y, z\in S$ such that $\phi(x) = x', \phi(y) = y'$ and $\phi(z) = z'$.
    \begin{align*}
        (x' *' y') *' z' & = (\phi(x) *' \phi(y)) *' \phi(z) \\
                         & = \phi(x * y) *' \phi(z)          \\
                         & = \phi((x * y) * z)               \\
                         & = \phi(x * (y * z))               \\
                         & = \phi(x) *' \phi(y * z)          \\
                         & = \phi(x) *' (\phi(y) *' \phi(z)) \\
                         & = x' *' (y' *' z')
    \end{align*}

    Hence $*'$ is also commutative.
\end{proof}

\begin{exercise}
    For each $c\in S$, the equation $x * x = c$ has a solution $x$ in $S$.
\end{exercise}

\begin{proof}
    Let $\phi: S\to S'$ be an isomorphism of $\anglebracket{S, *}$ with $\anglebracket{S', *'}$.

    For every $c'\in S'$, there exists uniquely $c\in S$ such that $\phi(c) = c'$.

    The equation $x * x = c$ has a solution $x$ in $S$, so $\phi(x) *' \phi(x) = \phi(x * x) = \phi(c) = c'$. Hence the equation $y * y = c'$ has a solution $y$ in $S'$.
\end{proof}

\begin{exercise}
    There exists an element $b$ in $S$ such that $b * b = b$.
\end{exercise}

\begin{proof}
    Let $\phi: S\to S'$ be an isomorphism of $\anglebracket{S, *}$ with $\anglebracket{S', *'}$.

    For every $b'\in S'$, there exists uniquely $b\in S$ such that $\phi(b) = b'$.

    There exists an element $b$ in $S$ such that $b * b = b$. So $b' *' b' = \phi(b) *' \phi(b) = \phi(b * b) = \phi(b) = b'$.

    Hence there exists an element $b'$ in $S'$ such that $b' * b' = b'$.
\end{proof}

\begin{exercise}
    Let $H$ be the subset $M_{2}(\mathbb{R})$ consisting of all matrices of the form $\begin{bmatrix}a & -b \\ b & a\end{bmatrix}$ for $a, b\in\mathbb{R}$. Exercise 23 of Section 2 shows that $H$ is closed under both matrix addition and matrix multiplication.
    \begin{enumerate}[label={\textbf{\alph*.}},topsep=0pt,itemsep=0pt]
        \item Show that $\anglebracket{\mathbb{C},+}$ is isomorphic to $\anglebracket{H,+}$.
        \item Show that $\anglebracket{\mathbb{C},\cdot}$ is isomorphic to $\anglebracket{H,\cdot}$.
    \end{enumerate}

    (We say that $H$ is a \textit{matrix representation} of the complex numbers $\mathbb{C}$.)
\end{exercise}

\begin{proof}
    Define the following one-to-one function $\phi$ from $\mathbb{C}$ onto $H$
    \[
        \phi(a + bi) = \begin{bmatrix}
            a & -b \\
            b & a
        \end{bmatrix}
    \]

    \begin{enumerate}[label={\textbf{\alph*.}},topsep=0pt,itemsep=0pt]
        \item According to Exercise 23 of Section 2
              \begin{align*}
                  \phi((a + bi) + (c + di)) & = \phi((a + c) + (b + d)i)    \\
                                            & = \begin{bmatrix}
                                                    a + c & -(b + d) \\
                                                    b + d & a + c
                                                \end{bmatrix}            \\
                                            & = \begin{bmatrix}
                                                    a & -b \\
                                                    b & a
                                                \end{bmatrix} +
                  \begin{bmatrix}
                      c & -d \\
                      d & c
                  \end{bmatrix}                                            \\
                                            & = \phi(a + bi) + \phi(c + di)
              \end{align*}

              So $\phi$ is an isomorphism of $\anglebracket{\mathbb{C},+}$ and $\anglebracket{H,+}$.

              Hence $\anglebracket{\mathbb{C},+}$ is isomorphic to $\anglebracket{H,+}$.
        \item According to Exercise 23 of Section 2
              \begin{align*}
                  \phi((a + bi) \cdot (c + di)) & = \phi((ac - bd) + (ad + bc)i)    \\
                                                & = \begin{bmatrix}
                                                        ac - bd & -(ad + bc) \\
                                                        ad + bc & ac - bd
                                                    \end{bmatrix}            \\
                                                & = \begin{bmatrix}
                                                        a & -b \\
                                                        b & a
                                                    \end{bmatrix} \cdot
                  \begin{bmatrix}
                      c & -d \\
                      d & c
                  \end{bmatrix}                                                    \\
                                                & = \phi(a + bi) \cdot \phi(c + di)
              \end{align*}

              So $\phi$ is an isomorphism of $\anglebracket{\mathbb{C},\cdot}$ and $\anglebracket{H,\cdot}$.

              Hence $\anglebracket{\mathbb{C},\cdot}$ is isomorphic to $\anglebracket{H,\cdot}$.
    \end{enumerate}

\end{proof}

\begin{exercise}
    There are 16 possible binary structures on the set $\{ a, b \}$ of two elements. How many nonisomorphic (that is, structurally different) structures are there among these 16? Phrased more precisely in terms of the isomorphism equivalence relation $\simeq$ on this set of 16 structures, how many equivalence classes are there? Write down one structure from each equivalence class.
\end{exercise}

\begin{proof}
\end{proof}

\section{Subgroups}

\section{Cyclic Groups}

\section{Generating Sets and Cayley Digraphs}
