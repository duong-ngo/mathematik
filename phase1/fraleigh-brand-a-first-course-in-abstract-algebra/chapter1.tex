\newpage
\chapter{Groups and Subgroups}

\newpage
\section{Binary Operations}

\subsection*{Computations}

Exercises 1 through 4 concern the binary operation $*$ defined on $S = \{a, b, c, d, e\}$ by means of Table 1.31 (see the book)

\newpage
% section 1/exercise 1
\begin{exercise}
    Compute $b * d, c * c$, and $((a * c) * e) * a$.
\end{exercise}

\begin{proof}
    \[
        b * d = e.
    \]
    \[
        ((a * c) * e) * a = (c * e) * a = a * a = a.
    \]
\end{proof}

\newpage
% section 1/exercise 2
\begin{exercise}
    Compute $(a * b) * c$ and $a * (b * c)$. Can you say on the basis of this computation whether $*$ is associative?
\end{exercise}

\begin{proof}
    \[
        \begin{split}
            (a * b) * c = b * c = a, \\
            a * (b * c) = a * a = a.
        \end{split}
    \]

    On the basis of this computation, it is not sufficient to say $*$ is associative.
\end{proof}

\newpage
% section 1/exercise 3
\begin{exercise}
    Compute $(b * d) * c$ and $b * (d * c)$. Can you say on the basis of this computation whether $*$ is associative?
\end{exercise}

\begin{proof}
    \[
        \begin{split}
            (b * d) * c = e * c = a, \\
            b * (d * c) = b * b = c.
        \end{split}
    \]

    On the basis of this computation, I can say $*$ is not associative.
\end{proof}

\newpage
% section 1/exercise 4
\begin{exercise}
    Is $*$ commutative? Why?
\end{exercise}

\begin{proof}
    $*$ is not commutative. Because $e * b = b \ne c = b * e$.
\end{proof}

\newpage
% section 1/exercise 5
\begin{exercise}
    Complete Table 1.32 so as to define a commutative binary operation $*$ on $S = \{ a, b, c, d \}$.
\end{exercise}

\begin{proof}
    % chktex-file 44
    \begin{tabular}{c|c|c|c|c}
        * & a & b & c & d \\
        \midrule
        a & a & b & c & d \\
        b & b & d & a & c \\
        c & c & a & d & b \\
        d & d & c & b & a
    \end{tabular}
\end{proof}

\newpage
% section 1/exercise 6
\begin{exercise}
    Table 1.33 can be completed to define an associative binary operation $*$ on $S = \{ a, b, c, d \}$. Assume this is possible and compute the missing entries. Does $S$ have an identity element?
\end{exercise}

\begin{proof}
    % chktex-file 44
    \begin{tabular}{c|c|c|c|c}
        * & a          & b          & c          & d          \\
        \hline
        a & a          & b          & c          & d          \\
        b & b          & c          & a          & d          \\
        c & c          & d          & c          & d          \\
        d & \textbf{d} & \textbf{c} & \textbf{c} & \textbf{d}
    \end{tabular}

    $S$ has an identity element, which is $a$.
\end{proof}

In Exercises 7 through 11, determine whether the binary operation $*$ defined is commutative and whether $*$ is associative.

\newpage
% section 1/exercise 7
\begin{exercise}
    $*$ defined on $\mathbb{Z}$ by letting $a * b = a - b$
\end{exercise}

\begin{proof}
    $*$ is noncommutative. Because $1 * 2 = 1 - 2 = -1 \ne 1 = 2 - 1 = 2 * 1$.
\end{proof}

\newpage
% section 1/exercise 8
\begin{exercise}
    $*$ defined on $\mathbb{Q}$ by letting $a * b = 2ab + 3$
\end{exercise}

\begin{proof}
    $*$ is commutative. Because for every two rational numbers $a$ and $b$, $ab + 1\in\mathbb{Q}$ and
    \[
        a * b = 2ab + 3 = 2ba + 3 = b * a.
    \]
\end{proof}

\newpage
% section 1/exercise 9
\begin{exercise}
    $*$ defined on $\mathbb{Z}$ by letting $a * b = ab + a + b$
\end{exercise}

\begin{proof}
    $*$ is commutative. Because for every two rational numbers $a$ and $b$, $ab + a + b\in\mathbb{Z}$ and
    \[
        a * b = ab + a + b = ba + b + a = b * a
    \]
\end{proof}

\newpage
% section 1/exercise 10
\begin{exercise}
    $*$ defined on $\mathbb{Z}^{*}$ by letting $a * b = 2^{ab}$
\end{exercise}

\begin{proof}
    $*$ is commutative. Because for every two positive integers $a, b$, $2^{ab}\in\mathbb{Z}^{+}$ and
    \[
        a * b = 2^{ab} = 2^{ba} = b * a.
    \]
\end{proof}

\newpage
% section 1/exercise 11
\begin{exercise}
    $*$ defined on $\mathbb{Z}^{+}$ by letting $a * b = a^{b}$
\end{exercise}

\begin{proof}
    $*$ is not commutative. Because $1 * 2 = 1^{2} = 1 \ne 2 = 2^{1} = 2 * 1$.
\end{proof}

\newpage
% section 1/exercise 12
\begin{exercise}
    Let $S$ be a set having exactly one element. How many different binary operations can be defined on $S$? Answer the question if $S$ has exactly $2$ elements; exactly $3$ elements; exactly $n$ elements.
\end{exercise}

\begin{proof}
    If $\card{S} = 1$, there is $1$ binary operations can be defined on $S$.

    If $\card{S} = 2$, $\card{S\times S} = 4$, there are $2^{4}$ binary operations can be defined on $S$.

    If $\card{S} = 3$, $\card{S\times S} = 9$, there are $3^{9}$ binary operations can be defined on $S$.

    If $\card{S} = n$, $\card{S\times S} = n^{2}$, there are $n^{(n^{2})}$ binary operations can be defined on $S$.
\end{proof}

\newpage
% section 1/exercise 13
\begin{exercise}
    How many different commutative binary operations can be defined on a set of $2$ elements? on a set of $3$ elements? on a set of $n$ elements?
\end{exercise}

\begin{proof}
    If $\card{S} = 1$, there is $1$ binary operations can be defined on $S$ and it is also commutative.

    If $\card{S} = 2$, there are $2^{2(2+1)/2} = 8$ commutative binary operations can be defined on $S$.

    If $\card{S} = 3$, there are $3^{3(3+1)/2} = 729$ commutative binary operations can be defined on $S$.

    If $\card{S} = n$, there are $n^{n(n+1)/2}$ commutative binary operations can be defined on $S$.
\end{proof}

\newpage
% section 1/exercise 14
\begin{exercise}
    How many different binary operations on a set $S$ with $n$ elements have the property that for all $x\in S, x * x = x$?
\end{exercise}

\begin{proof}
    There are $n^{n(n-1)}$ such binary operations.
\end{proof}

\newpage
% section 1/exercise 15
\begin{exercise}
    How many different binary operations on a set $S$ with $n$ elements have an identity element?
\end{exercise}

\begin{proof}
    $S = \{ a_{1}, \ldots, a_{n} \}$.

    If $a_{i}$ is the identity element, then there are $n^{{(n-1)}^{2}}$ such binary operations.

    Hence there are $n^{1 + {(n-1)}^{2}}$ such binary operations.
\end{proof}

\subsection*{Concepts}

In Exercises 16 through 19, correct the definition of the italicized term without reference to the text, if correction is needed, so that it is in a form acceptable for publication.

\newpage
% section 1/exercise 16
\begin{exercise}
    A binary operation $*$ is \textit{commutative} if and only if $a * b = b * a$.
\end{exercise}

\begin{proof}
    Correction: A binary operation $*$ on a set $S$ is \textit{commutative} if and only if $a * b = b * a$ for every two elements $a, b$.
\end{proof}

\newpage
% section 1/exercise 17
\begin{exercise}
    A binary operation $*$ on a set $S$ is \textit{associative} if and only if, for all $a, b, c\in S$, we have $(b * c) * a = b * (c * a)$.
\end{exercise}

\begin{proof}
    This definition doesn't need correction.
\end{proof}

\newpage
% section 1/exercise 18
\begin{exercise}
    A subset $H$ of a set $S$ is \textit{closed} under a binary operation $*$ on $S$ if and only if $(a * b)\in H$ for all $a, b\in S$.
\end{exercise}

\begin{proof}
    Correction: A subset $H$ of a set $S$ is \textit{closed} under a binary operation $*$ on $S$ if and only if $(a * b)\in H$ for all $a, b\in H$.
\end{proof}

\newpage
% section 1/exercise 19
\begin{exercise}
    An identity in the set $S$ with operation $*$ is an element $e\in S$ such that $a * e = e * a = a$.
\end{exercise}

\begin{proof}
    Correction: An identity in the set $S$ with operation $*$ is an element $e\in S$ such that $a * e = e * a = a$ for all $a\in S$.
\end{proof}

\newpage
% section 1/exercise 20
\begin{exercise}
    Is there an example of a set $S$, a binary operation on $S$, and two different elements $e_{1}, e_{2}\in S$ such that for all $a\in S$, $e_{1} * a = a$ and $a * e_{2} = a$? If so, give an example and if not, prove there is not one.
\end{exercise}

\begin{proof}
    There is no such binary operation.

    $e_{1} * a = a$ for all $a\in S$ so $e_{1} * e_{2} = e_{2}$.

    $a * e_{2} = a$ for all $a\in S$ so $e_{1} * e_{2} = e_{1}$.

    Thus $e_{1} = e_{1} * e_{2} = e_{2}$.
\end{proof}

In Exercises 21 through 26, determine whether the definition of $*$ does give a binary operation on the set. In the event that $*$ is not a binary operation, state whether Condition 1, Condition 2, or both conditions regarding defining binary operations are violated.

\newpage
% section 1/exercise 21
\begin{exercise}
    On $\mathbb{Z}^{+}$, define $*$ by letting $a * b = a^{b}$.
\end{exercise}

\begin{proof}
    This is a binary operation.
\end{proof}

\newpage
% section 1/exercise 22
\begin{exercise}
    On $\mathbb{R}^{+}$, define $*$ by letting $a * b = 2a - b$.
\end{exercise}

\begin{proof}
    This is not a binary operation. Condition 2 is violated, since $1 * 2 = 2\cdot 1 - 2 = 2 - 2 = 0\notin\mathbb{R}^{+}$.
\end{proof}

\newpage
% section 1/exercise 23
\begin{exercise}
    On $\mathbb{R}^{+}$, define $*$ by $a * b$ to be the minimum of $a$ and $b - 1$ if they are different and their common value if they are the same.
\end{exercise}

\begin{proof}
    This is not a binary operation. Condition 2 is violated, since $1 * 1 = 1 - 1 = 0\notin\mathbb{R}^{+}$.
\end{proof}

\newpage
% section 1/exercise 24
\begin{exercise}
    On $\mathbb{R}$, define $a * b$ to be the number $c$ so that $c^{b} = a$.
\end{exercise}

\begin{proof}
    This is not a binary operation. Both conditions are violated, since
    \begin{itemize}
        \item $c^{0} = 1$ for every $c\ne 0$,
        \item there is no $c\in\mathbb{R}$ such that $c^{1/2} = -1$.
    \end{itemize}
\end{proof}

\newpage
% section 1/exercise 25
\begin{exercise}
    On $\mathbb{Z}^{+}$, define $*$ letting $a * b = c$, where $c$ is at least $5$ more than $a + b$.
\end{exercise}

\begin{proof}
    This is not a binary operation. Because Condition 1 is violated. $a * b$ can be assigned to $a + b + 5, a + b + 6, \ldots$
\end{proof}

\newpage
% section 1/exercise 26
\begin{exercise}
    On $\mathbb{Z}^{+}$, define $*$ by letting $a * b = c$, where $c$ is the largest integer less than the product of $a$ and $b$.
\end{exercise}

\begin{proof}
    This is not a binary operation. Because Condition 2 is violated. $1 * 1 = 0\notin\mathbb{Z}^{+}$.
\end{proof}

\newpage
% section 1/exercise 27
\begin{exercise}
    Let $H$ be the subset of $M_{2}(\mathbb{R})$ consisting of all matrices of the form $\begin{bmatrix}a & -b \\ b & a\end{bmatrix}$ for $a, b\in\mathbb{R}$. Is $H$ closed under
    \begin{enumerate}[label={\textbf{\alph*}}]
        \item matrix addition?
        \item matrix multiplication?
    \end{enumerate}
\end{exercise}

\begin{proof}
    \begin{enumerate}[label={\textbf{\alph*}}]
        \item For any two matrices within $H$
              \[
                  \begin{bmatrix}
                      a & -b \\
                      b & a
                  \end{bmatrix}
                  +
                  \begin{bmatrix}
                      c & -d \\
                      d & c
                  \end{bmatrix}
                  =
                  \begin{bmatrix}
                      a + c & -(b + d) \\
                      b + d & a + c
                  \end{bmatrix}
                  \in H
              \]

              so $H$ is closed under matrix addition.
        \item For any two matrices within $H$
              \[
                  \begin{bmatrix}
                      a & -b \\
                      b & a
                  \end{bmatrix}
                  \cdot
                  \begin{bmatrix}
                      c & -d \\
                      d & c
                  \end{bmatrix}
                  =
                  \begin{bmatrix}
                      ac - bd & -(ad + bc) \\
                      ad + bc & ac - bd
                  \end{bmatrix}
                  \in H
              \]

              so $H$ is closed under matrix multiplication.
    \end{enumerate}
\end{proof}

\newpage
% section 1/exercise 28
\begin{exercise}
    Mark each of the following true or false.
    \begin{enumerate}[label={\textbf{\alph*.}},itemsep=0pt,topsep=0pt]
        \item If $*$ is any binary operation on any set $S$, then $a * a = a$ for all $a\in S$.
        \item If $*$ is any commutative binary operation on any set $S$, then $a * (b * c) = (b * c) * a$ for all $a, b, c \in S$.
        \item If $*$ is any associative binary operation on any set $S$, then $a * (b * c) = (b * c) * a$ for all $a, b, c \in S$.
        \item The only binary operations of any importance are those defined on sets of numbers.
        \item A binary operation $*$ on a set $S$ is commutative if there exist $a, b \in S$ such that $a * b = b * a$.
        \item Every binary operation defined on a set having exactly one element is both commutative and associative.
        \item A binary operation on a set $S$ assigns at least one element of $S$ to each ordered pair of elements of $S$.
        \item A binary operation on a set $S$ assigns at most one element of $S$ to each ordered pair of elements of $S$.
        \item A binary operation on a set $S$ assigns exactly one element of $S$ to each ordered pair of elements of $S$.
        \item A binary operation on a set $S$ may assign more tha one element of $S$ to some ordered pair of elements of $S$.
        \item For any binary operation $*$ on the set $S$, if $a, b, c\in S$ and $a * b = a * c$, then $b = c$.
        \item For any binary operation $*$ on the set $S$, there is an element $e\in S$ such that for all $x\in S, x * e = x$.
        \item There is an operation on the set $S = \{ e_{1}, e_{2}, a \}$ so that for all $x\in S$, $e_{1} * x = e_{2} * x = x$.
        \item Identity elements are always called $e$.
    \end{enumerate}
\end{exercise}

\begin{proof}
    \begin{enumerate}[label={\textbf{\alph*.}},itemsep=0pt,topsep=0pt]
        \item False. Example: $\mathbb{Z}$ with addition.
        \item True.
        \item False. Example: $M(2,\mathbb{R})$ with multiplication.
              \[
                  \begin{bmatrix}
                      1 & 1 \\
                      0 & 0
                  \end{bmatrix}
                  \cdot
                  \left(
                  \begin{bmatrix}
                          1 & 0 \\
                          0 & 1
                      \end{bmatrix}
                  \cdot
                  \begin{bmatrix}
                          1 & 1 \\
                          1 & 0
                      \end{bmatrix}
                  \right)
                  =
                  \begin{bmatrix}
                      2 & 1 \\
                      0 & 0
                  \end{bmatrix}
                  \ne
                  \begin{bmatrix}
                      1 & 1 \\
                      1 & 1
                  \end{bmatrix}
                  =
                  \left(
                  \begin{bmatrix}
                          1 & 0 \\
                          0 & 1
                      \end{bmatrix}
                  \cdot
                  \begin{bmatrix}
                          1 & 1 \\
                          1 & 0
                      \end{bmatrix}
                  \right)
                  \cdot
                  \begin{bmatrix}
                      1 & 1 \\
                      0 & 0
                  \end{bmatrix}.
              \]
        \item Undeciable, since ``important'' is undefined.
        \item False. Example: $M_{2}(\mathbb{R})$ with multiplication, any $2\times 2$ matrix is commutative with the $2\times 2$ identity matrix, but multiplication of any two $2\times 2$ matrices are not necessarily commutative.
        \item True.
        \item False. Must be exactly one, not at least one.
        \item False. Must be exactly one, not at most one.
        \item True.
        \item False.
        \item False. Counterexample: $\mathbb{Z}$ with usual multiplication, $0 \cdot 1 = 0 \cdot 2$ but $1 \ne 2$.
        \item False. Counterexample: $S = \{ a, b \}$, $a * a = b, a * b = b, b * b = a, b * a = a$.
        \item True. Example:
              \[
                  \begin{array}{c|ccc}
                      *     & e_{1} & e_{2} & a \\
                      \hline
                      e_{1} & e_{1} & e_{2} & a \\
                      e_{2} & e_{1} & e_{2} & a \\
                      a     & a     & a     & a
                  \end{array}
              \]
        \item False. We can called it anything.
    \end{enumerate}
\end{proof}

\newpage
% section 1/exercise 29
\begin{exercise}
    Give a set different from any of those described in the examples of the text and not a set of numbers. Define two different binary operations $*$ and $*'$ on this set. Be sure that your set is well defined.
\end{exercise}

\begin{proof}
    Let $S$ be a set of two elements $a$ and $b$. Define two binary operations $*$ and $*'$ as follows
    % chktex-file 44
    \begin{tabular}{c|cc}
        * & a & b \\
        \hline
        a & a & a \\
        b & a & a
    \end{tabular}
    % chktex-file 44
    \begin{tabular}{c|cc}
        * & a & b \\
        \hline
        a & a & b \\
        b & b & a
    \end{tabular}
\end{proof}

\subsection*{Theory}

\newpage
% section 1/exercise 30
\begin{exercise}
    Prove that if $*$ is an associative and commutative binary operation on a set $S$, then
    \[
        (a * b) * (c * d) = ((d * c) * a) * b
    \]

    for all $a, b, c, d\in S$. Assume the associative law only for triples as in the definition, that is, assume only
    \[
        (x * y) * z = x * (y * z)
    \]

    for all $x, y, z\in S$.
\end{exercise}

\begin{proof}
    $d * c = e\in S$.

    \begin{align*}
        (a * b) * (c * d) & = (c * d) * (a * b)  & \text{(commutative)} \\
                          & = (d * c) * (a * b)  & \text{(commutative)} \\
                          & = e * (a * b)                               \\
                          & = (e * a) * b        & \text{(associative)} \\
                          & = ((d * c) * a) * b.
    \end{align*}
\end{proof}

In Exercises 31 and 32, either prove the statement or give a counterexample.

\newpage
% section 1/exercise 31
\begin{exercise}
    Every binary operation on a set consisting of a single element is both commutative and associative.
\end{exercise}

\begin{proof}
    Let $S = \{ a \}$. There is only one binary operation $*$ can be define on $S$, where $a * a = a$.

    Since $a * a = a * a$ and $(a * a) * a = a * a = a * (a * a)$, $S$ with $*$ is commutative and associative.
\end{proof}

\newpage
% section 1/exercise 32
\begin{exercise}
    Every commutative binary operation on a set having just two elements is associative.
\end{exercise}

\begin{proof}
    False.

    Counterexample: Let $S = \{ a, b \}$, and $*$ be the commutative binary operation on $S$ as follows
    % chktex-file 44
    \begin{tabular}{c|cc}
        * & a & b \\
        \midrule
        a & b & b \\
        b & b & a
    \end{tabular}

    \[
        (a * a) * b = b * b = a \ne b = a * b = a * (a * b).
    \]
\end{proof}

Let $F$ be the set of all real-valued functions having as domain the set $\mathbb{R}$ of all real numbers. Example 2.7 defined the binary operations $+, -, \cdot$, and $\circ$ on $F$. In Exercises 29 through 35, either prove the given statement or give a
counterexample.

\newpage
% section 1/exercise 33
\begin{exercise}
    Function addition $+$ on $F$ is associative.
\end{exercise}

\begin{proof}
    For every real number $x$ and every three functions $f, g, h$ in $F$
    \begin{align*}
        ((f + g) + h)(x) & = (f + g)(x) + h(x)    \\
                         & = (f(x) + g(x)) + h(x) \\
                         & = f(x) + (g(x) + h(x)) \\
                         & = f(x) + (g + h)(x)    \\
                         & = (f + (g + h))(x)
    \end{align*}

    Thus function addition $+$ on $F$ is associative.
\end{proof}

\newpage
% section 1/exercise 34
\begin{exercise}
    Function subtraction $-$ on $F$ is commutative.
\end{exercise}

\begin{proof}
    This is false.

    Counterexample: $f(x) = x, g(x) = x + 1$, then $(f - g)(x) = -1\ne 1 = (g - f)(x)$.
\end{proof}

\newpage
% section 1/exercise 35
\begin{exercise}
    Function substraction $-$ on $F$ is associative.
\end{exercise}

\begin{proof}
    This is false.

    Counterexample: $f(x) = g(x) = h(x) = x$, then for $x\ne 0$, $((f - g) - h)(x) = -x \ne x = (f - (g - h))(x)$.
\end{proof}

\newpage
% section 1/exercise 36
\begin{exercise}
    Under function substraction $-$ $F$ has an identity.
\end{exercise}

\begin{proof}
    This is false.

    Assume that under function substraction $-$ $F$ has an identity $\iota$. Then for every $f\in F$ and for every real number $x$
    \[
        f(x) - \iota(x) = \iota(x) - f(x) = f(x)
    \]

    We deduce that $\iota(x) = 0$ and $\iota(x) = \frac{1}{2}f(x)$ for every real number $x$. This is a contradiction, because $\frac{1}{2}f(x)$ is not necessarily equal to $0$ for every real number $x$.

    Thus Under function substraction $-$ $F$ does not have an identity.
\end{proof}

\newpage
% section 1/exercise 37
\begin{exercise}
    Under function multiplication $\cdot$ $F$ has an identity.
\end{exercise}

\begin{proof}
    This is true.

    Let $\iota(x) = 1$ for every real number $x$. Then for every $f\in F$ and every real number $x$
    \[
        (f\cdot \iota)(x) = f(x)\cdot \iota(x) = f(x) = \iota(x)\cdot f(x) = (\iota\cdot f)(x).
    \]
\end{proof}

\newpage
% section 1/exercise 38
\begin{exercise}
    Function multiplication $\cdot$ on $F$ is commutative.
\end{exercise}

\begin{proof}
    For every real number $x$ and every two functions $f, g$ in $F$
    \begin{align*}
        (f\cdot g)(x) & = f(x)g(x)       \\
                      & = g(x)f(x)       \\
                      & = (g\cdot f)(x).
    \end{align*}

    Thus function multiplication $\cdot$ on $F$ is commutative.
\end{proof}

\newpage
% section 1/exercise 39
\begin{exercise}
    Function multiplication $\cdot$ on $F$ is associative.
\end{exercise}

\begin{proof}
    For every real number $x$ and every three functions $f, g, h$ in $F$
    \begin{align*}
        ((f \cdot g) \cdot h)(x) & = (f \cdot g)(x) \cdot h(x)    \\
                                 & = (f(x) \cdot g(x)) \cdot h(x) \\
                                 & = f(x) \cdot (g(x) \cdot h(x)) \\
                                 & = f(x) \cdot (g \cdot h)(x)    \\
                                 & = (f \cdot (g \cdot h))(x)
    \end{align*}

    Thus function multiplication $\cdot$ on $F$ is associative.
\end{proof}

\newpage
% section 1/exercise 40
\begin{exercise}
    Function composition $\circ$ on $F$ is commutative.
\end{exercise}

\begin{proof}
    This is false.

    Counterexample: $f(x) = x + 1, g(x) = 2x$. $(f\circ g)(x) = f(g(x)) = 2x + 1\ne 2x + 2 = g(x + 1) = g(f(x))$.
\end{proof}

\newpage
% section 1/exercise 41
\begin{exercise}
    If $*$ and $*'$ are any two binary operations on a set $S$, then
    \[
        a * (b *' c) = (a * b) *' (a * c)\quad\text{for all $a, b, c\in S$.}
    \]
\end{exercise}

\begin{proof}
    This is false. We give a counterexample.

    Let $S = \{ 0, 1 \}$. We define $*$ and $*'$ as follows
    % chktex-file 44
    \begin{tabular}{c|cc}
        * & 0 & 1 \\
        \midrule
        0 & 0 & 0 \\
        1 & 0 & 1
    \end{tabular}
    \begin{tabular}{c|cc}
        \midrule
        *' & 0 & 1 \\
        0  & 1 & 0 \\
        1  & 0 & 1
    \end{tabular}
    \[
        \begin{split}
            0 * (0 *' 1) = 0 * 0 = 0 \ne 1,  \\
            (0 * 0) *' (0 * 1) = 0 *' 0 = 1.
        \end{split}
    \]
\end{proof}

\newpage
% section 1/exercise 42
\begin{exercise}
    Suppose that $*$ is an \textit{associative binary} operation on a set $S$. Let $H = \{ a \in S \mid a * x = x * a \text{ for all } x\in S \}$. Show that $H$ is closed under $*$. (We think of $H$ as consisting of all elements of $S$ that \textit{commute} with every element in $S$.)
\end{exercise}

\begin{proof}
    Let $a, b$ be elements of $H$. For every element $x$ of $S$
    \begin{align*}
        (a * b) * x & = a * (b * x) & \text{(associative)} \\
                    & = a * (x * b) & (b\in H)             \\
                    & = (a * x) * b & \text{(associative)} \\
                    & = (x * a) * b & (a\in H)             \\
                    & = x * (a * b) & \text{(associative)}
    \end{align*}

    According to the definition of $H$, $a * b\in H$. Hence $H$ is closed under the binary operation $*$ on $S$.
\end{proof}

\newpage
% section 1/exercise 43
\begin{exercise}
    Suppose that $*$ is an associative and commutative operation on a set $S$. Show that $H = \{ a\in S \mid a * a = a \}$ is closed under $*$. (The element of $H$ are \textbf{idempotents} of the binary operation $*$.)
\end{exercise}

\begin{proof}
    Let $a, b$ be elements of $H$.
    \begin{align*}
        (a * b) * (a * b) & = (a * b) * (b * a) & \text{(commutative)} \\
                          & = ((a * b) * b) * a & \text{(associative)} \\
                          & = (a * (b * b)) * a & \text{(associative)} \\
                          & = (a * b) * a       & (b\in H)             \\
                          & = (b * a) * a       & \text{(commutative)} \\
                          & = b * (a * a)       & \text{(associative)} \\
                          & = b * a             & (a\in H)             \\
                          & = a * b             & \text{(commutative)}
    \end{align*}

    According to the definition of $H$, $a * b\in H$. Hence $H$ is closed under $*$.
\end{proof}

\newpage
% section 1/exercise 44
\begin{exercise}
    Let $S$ be a set and let $*$ be a binary operation on $S$ satisfying the two laws
    \begin{itemize}
        \item $x * x = x$ for all $x\in S$, and
        \item $(x * y) * z = (y * z) * x$ for all $x, y, z\in S$.
    \end{itemize}

    Show that $*$ is associative and commutative.
\end{exercise}

\begin{proof}
    For all $x, y\in S$,

    \begin{align*}
        x * y & = (x * y) * (x * y) & \text{(Law 1)} \\
              & = (y * (x * y)) * x & \text{(Law 2)} \\
              & = ((x * y) * x) * y & \text{(Law 2)} \\
              & = ((y * x) * x) * y & \text{(Law 2)} \\
              & = ((x * x) * y) * y & \text{(Law 2)} \\
              & = (x * y) * y       & \text{(Law 1)} \\
              & = (y * y) * x       & \text{(Law 2)} \\
              & = y * x             & \text{(Law 1)}
    \end{align*}

    So $*$ is commutative.

    \begin{align*}
        (x * y) * z & = (y * z) * x & \text{(Law 2)}           \\
                    & = x * (y * z) & \text{(Commutative law)}
    \end{align*}

    So $*$ is associative.
\end{proof}

\newpage
\section{Groups}

\subsection*{Computations}

In Exercises 1 through 9, determine whether the binary operation $*$ gives a group structure on the given set. If no group results, give the first axiom in order $\mathcal{G}_{1}, \mathcal{G}_{2}, \mathcal{G}_{3}$ from Definition 2.1 that does not hold.

\newpage
% section 2/exercise 1
\begin{exercise}
    Let $*$ be defined on $\mathbb{Z}$ by letting $a * b = ab$.
\end{exercise}

\begin{proof}
    This is not a group structure. $\mathcal{G}_{3}$ does not hold.
\end{proof}

\newpage
% section 2/exercise 2
\begin{exercise}
    Let $*$ be defined on $2\mathbb{Z} = \{ 2n \mid n\in\mathbb{Z} \}$ by letting $a * b = a + b$.
\end{exercise}

\begin{proof}
    This is a group structure.
\end{proof}

\newpage
% section 2/exercise 3
\begin{exercise}
    Let $*$ be defined on $\mathbb{R}^{+}$ by letting $a * b = \sqrt{ab}$.
\end{exercise}

\begin{proof}
    This is not a group structure. $\mathcal{G}_{1}$ does not hold.
\end{proof}

\newpage
% section 2/exercise 4
\begin{exercise}
    Let $*$ be defined on $\mathbb{Q}$ by letting $a * b = ab$.
\end{exercise}

\begin{proof}
    This is not a group structure. $\mathcal{G}_{3}$ does not hold.
\end{proof}

\newpage
% section 2/exercise 5
\begin{exercise}
    Let $*$ be defined on the set $\mathbb{R}^{*}$ of nonzero real numbers by letting $a * b = a/b$.
\end{exercise}

\begin{proof}
    This is not a group structure. $\mathcal{G}_{1}$ does not hold.
\end{proof}

\newpage
% section 2/exercise 6
\begin{exercise}
    Let $*$ be defined on $\mathbb{C}$ by letting $a * b = \abs{ab}$.
\end{exercise}

\begin{proof}
    This is not a group structure. $\mathcal{G}_{2}$ does not hold.
\end{proof}

\newpage
% section 2/exercise 7
\begin{exercise}
    Let $*$ be defined on the set $\{ a, b \}$ by Table 2.26.
    \begin{tabular}{c|cc}
        * & a & b \\
        \hline
        a & a & b \\
        b & b & b
    \end{tabular}
\end{exercise}

\begin{proof}
    $(a * a) * a = a = a * (a * a)$

    $(b * b) * b = b = b * (b * b)$

    $a * (a * b) = a * b = (a * a) * b$

    $a * (b * a) = a * b = b = b * a = (a * b) * a$

    $b * (a * a) = b * a = (b * a) * a$

    $b * (a * b) = b * b = (b * a) * b$

    $a * (b * b) = a * b = b = b * b = (a * b) * b$

    $b * (b * a) = b * b = b = b * a = (b * b) * a$

    So this structure is associative.

    $a * b = b * a = b$

    $a * a = a$

    So this structure has an identity element, which is $a$.

    This is not a group structure. $\mathcal{G}_{3}$ does not hold, since $b$ does not have an inverse.
\end{proof}

\newpage
% section 2/exercise 8
\begin{exercise}
    Let $*$ be defined on the set $\{ a, b \}$ by Table 2.27.
    \begin{tabular}{c|cc}
        * & a & b \\
        \hline
        a & a & b \\
        b & a & b
    \end{tabular}
\end{exercise}

\begin{proof}
    $(a * a) * a = a = a * (a * a)$

    $(b * b) * b = b = b * (b * b)$

    $(a * a) * b = a * b = a * (a * b)$

    $(a * b) * a = b * a = a = a * a = a * (b * a)$

    $(b * a) * a = a * a = a = b * a = b * (a * a)$

    $(a * b) * b = b * b = b = a * b = a * (b * b)$

    $(b * b) * a = b * a = b * (b * a)$

    $(b * a) * b = a * b = b = b * b = b * (a * b)$

    So this structure is associative.

    This is not a group structure. $\mathcal{G}_{2}$ does not hold, since if there are an identity element, that must be commutative with the other, but in fact,
    \[
        a * b = b \ne a = b * a.
    \]
\end{proof}

\newpage
% section 2/exercise 9
\begin{exercise}
    Let $*$ be defined on the set $\{ e, a, b \}$ by Table 2.28.
    \begin{tabular}{c|ccc}
        * & e & a & b \\
        \hline
        e & e & a & b \\
        a & a & e & b \\
        b & b & b & a
    \end{tabular}
\end{exercise}

\begin{proof}
    This structure is not a group. $\mathcal{G}_{1}$ does not hold, since
    \[
        (a * b) * b = b * b = e \ne a = a * e = a * (b * b).
    \]
\end{proof}

\newpage
% section 2/exercise 10
\begin{exercise}
    Let $n$ be a positive integer and let $n\mathbb{Z} = \{ nm \mid m\in\mathbb{Z} \}$.
    \begin{enumerate}[label={\textbf{\alph*.}}]
        \item Show that $\anglebracket{n\mathbb{Z}, +}$ is a group.
        \item Show that $\anglebracket{n\mathbb{Z}, +} \simeq \anglebracket{\mathbb{Z}, +}$
    \end{enumerate}
\end{exercise}

\begin{proof}
    \begin{enumerate}[label={\textbf{\alph*.}}]
        \item Let $x, y, z$ be elements of $n\mathbb{Z}$. According to the definition of $n\mathbb{Z}$, there exist integers $a, b, c$ such that $x = na, y = nb, z = nc$.

              $x + y = na + nb = n(a + b)\in n\mathbb{Z}$. So $n\mathbb{Z}$ is closed under the operation $+$.

              $(x + y) + z = (na + nb) + nc = n(a + b) + nc = n((a + b) + c) = n(a + (b + c)) = na + n(b + c) = na + (nb + nc)$. So $\anglebracket{n\mathbb{Z}, +}$ is associative.

              $x + 0 = x = 0 + x$. So $\anglebracket{n\mathbb{Z}, +}$ has an identity element, which is $0$.

              $x + (-na) = na + (-na) = n (a + (-a)) = 0 = n ((-a) + a) = (-na) + na = (-na) + x$. So each element of $\anglebracket{n\mathbb{Z}, +}$ has an inverse.

              Hence $\anglebracket{n\mathbb{Z}, +}$ is a group.
        \item Let's define mappings $f: n\mathbb{Z} \to \mathbb{Z}$ as $f(x) = x/n$ and $g: \mathbb{Z} \to n\mathbb{Z}$ as $g(y) = n\cdot y$.

              For every $x\in n\mathbb{Z}, y\in\mathbb{Z}$
              \[
                  \begin{split}
                      f(g(y)) = f(ny) = (n\cdot y)/n = y, \\
                      g(f(x)) = g(x/n) = n\cdot (x/n) = x.
                  \end{split}
              \]

              So $f$ is a one-to-one function from $n\mathbb{Z}$ onto $\mathbb{Z}$.

              For every $x, y\in n\mathbb{Z}$, $x/n$ and $y/n$ are elements of $\mathbb{Z}$
              \[
                  f(x + y) = \frac{x + y}{n} = \frac{x}{n} + \frac{y}{n} = f(x) + f(y)
              \]

              Hence $\anglebracket{n\mathbb{Z}, +}$ and $\anglebracket{\mathbb{Z}, +}$ are isomorphic.
    \end{enumerate}
\end{proof}

In Exercises 11 through 18, determine whether the given set of matrices under the specified operation, matrix addition or multiplication, is a group.

\newpage
% section 2/exercise 11
\begin{exercise}
    All $n\times n$ diagonal matrices under matrix addition.
\end{exercise}

\begin{proof}
    This is a group.
\end{proof}

\newpage
% section 2/exercise 12
\begin{exercise}
    All $n\times n$ diagonal matrices under matrix multiplication.
\end{exercise}

\begin{proof}
    This is not a group, since the zero $n\times n$ matrix (which is also an $n\times n$ diagonal matrix) does not have a multiplicative inverse.
\end{proof}

\newpage
% section 2/exercise 13
\begin{exercise}
    All $n\times n$ diagonal matrices with no zero diagonal entry under matrix multiplication.
\end{exercise}

\begin{proof}
    This is a group.
\end{proof}

\newpage
% section 2/exercise 14
\begin{exercise}
    All $n\times n$ diagonal matrices with all diagonal entries $1$ or $-1$ under matrix multiplication.
\end{exercise}

\begin{proof}
    This is a group.
\end{proof}

\newpage
% section 2/exercise 15
\begin{exercise}
    All $n\times n$ upper-triangular matrices under matrix multiplication.
\end{exercise}

\begin{proof}
    $A = {(a_{i.j})}_{n\times n}$, $B = {(b_{i.j})}_{n\times n}$ such that $a_{i.j} = 0$ if $i\geq j$, $b_{i.j} = 0$ if $i\geq j$. Let $C = {(c_{i.j})}_{n\times n} = AB$.

    Suppose that $1\le s\le r\le n$.
    \begin{align*}
        c_{r.s} & = \sum^{n}_{k=1}a_{r.k}b_{k.s}                                  \\
                & = \sum^{s}_{k=1}a_{r.k}b_{k.s} + \sum^{n}_{k=s+1}a_{r.k}b_{k.s} \\
                & = 0 + 0                                                         \\
                & = 0.
    \end{align*}

    $\displaystyle\sum^{s}_{k=1}a_{r.k}b_{k.s} = 0$ because $r\ge s\ge k$. $\displaystyle\sum^{n}_{k=s+1}a_{r.k}b_{k.s} = 0$ because $k \geq s + 1 > s$.

    So the set of $n\times n$ upper-triangular matrices is closed under matrix multiplication.

    However, the $n\times n$ zero matrix (which is also an $n\times n$ upper-triangular matrix) is not invertible. Hence this is not a group.
\end{proof}

\newpage
% section 2/exercise 16
\begin{exercise}
    All $n\times n$ upper-triangular matrices under matrix addition.
\end{exercise}

\begin{proof}
    This is a group.
\end{proof}

\newpage
% section 2/exercise 17
\begin{exercise}
    All $n\times n$ upper-triangular matrices with determinant $1$ under multiplication.
\end{exercise}

\begin{proof}
    According to the proof of Exercise 15 and $\det(AB) = \det(A)\det(B)$, the set of $n\times n$ upper-triangular matrices is closed under matrix multiplication.

    Let $A, B, C$ be $n\times n$ matrices whose determinants is $1$.

    Matrix multiplication is associative, so the matrix multiplication on this set of matrices is associative.

    The $n\times n$ identity matrix $I_{n}$, $AI = IA = A$. So the matrix multiplication on this set of matrices has an identity matrix.

    Let
    \[
        A = \begin{bmatrix}
            a_{1.1} & a_{1.2} & \cdots & a_{1.n} \\
            0       & a_{2.2} & \cdots & a_{2.n} \\
            \vdots  & \vdots  &        & \vdots  \\
            0       & 0       & \cdots & a_{n.n}
        \end{bmatrix}.
    \]

    Since $A$ is a diagonal matrix, then $\det(A) = a_{1.1}a_{2.2}\cdots a_{n.n}$. On the other hand, $\det(A) = 1$, so $a_{1.1}, a_{2.2}, \ldots, a_{n.n}$ are non-zero. To define the multiplicative inverse of $A$, we perform elementary row operations on the following marix
    \[
        \begin{array}{cccc|cccc}
            a_{1.1} & a_{1.2} & \cdots & a_{1.n} & 1      & 0      & \cdots & 0      \\
            0       & a_{2.2} & \cdots & a_{2.n} & 0      & 1      & \cdots & 0      \\
            \vdots  & \vdots  &        & \vdots  & \vdots & \vdots &        & \vdots \\
            0       & 0       & \cdots & a_{n.n} & 0      & 0      & \cdots & 1
        \end{array}
    \]

    Multiply $i$-th row by ${a_{i.i}}^{-1}$.
    \[
        \begin{array}{cccc|cccc}
            1      & {a_{1.1}}^{-1}a_{1.2} & \cdots & {a_{1.1}}^{-1}a_{1.n} & {a_{1.1}}^{-1} & 0              & \cdots & 0              \\
            0      & 1                     & \cdots & {a_{2.2}}^{-1}a_{2.n} & 0              & {a_{2.2}}^{-1} & \cdots & 0              \\
            \vdots & \vdots                &        & \vdots                & \vdots         & \vdots         &        & \vdots         \\
            0      & 0                     & \cdots & 1                     & 0              & 0              & \cdots & {a_{n.n}}^{-1}
        \end{array}
    \]
    \begin{itemize}
        \item Add $-{a_{i.i}}^{-1}a_{i.n}$ times the $n$-th row to the $i$-th row ($1\le i < n$). After these operations, the $n\times n$ matrix on the right is still an $n\times n$ upper-triangular matrix.
        \item Add $-{a_{i.i}}^{-1}a_{i.(n-1)}$ times the $(n-1)$-th row to the $i$-th row ($1\le i < n-1$). After these operations, the $n\times n$ matrix on the right is still an $n\times n$ upper-triangular matrix.
        \item \ldots
        \item Add $-{a_{i.i}}^{-1}a_{i.2}$ times the $2$-nd row to the $i$-th row ($1\le i < 2$). After these operations, the $n\times n$ matrix on the right is still an $n\times n$ upper-triangular matrix.
    \end{itemize}

    After all these operations, the $n\times n$ matrix on the left is the $n\times n$ identity matrix, and the $n\times n$ matrix on the right is still an $n\times n$ upper-triangular matrix, which is also the multiplicative inverse of $A$. So there exists an $n\times n$ upper-triangular matrix $A'$ such that $AA' = A'A = I_{n}$.

    Thus, all $n\times n$ upper-triangular matrices with determinant $1$ under multiplication is a group.
\end{proof}

\newpage
% section 2/exercise 18
\begin{exercise}
    The set of $2\times 2$ matrices $G = \{ e, a, b \}$ where $e = \begin{bmatrix}1 & 0 \\ 0 & 1\end{bmatrix}$, $a = \begin{bmatrix}-\frac{1}{2} & -\frac{\sqrt{3}}{2} \\ \frac{\sqrt{3}}{2} & -\frac{1}{2}\end{bmatrix}$, and $b = \begin{bmatrix}-\frac{1}{2} & \frac{\sqrt{3}}{2} \\ -\frac{\sqrt{3}}{2} & -\frac{1}{2}\end{bmatrix}$ under matrix multiplication.
\end{exercise}

\begin{proof}
    \[
        ee = e,\qquad ea = ae = a,\qquad eb = be = b.
    \]
    \[
        aa = b,\qquad bb = a,\qquad ab = ba = e.
    \]

    So $G$ is closed under matrix multiplication. Matrix multiplication within $G$ is associative. According to the calculation above, matrix multiplication within $G$ has an identity element, which is $e$. Each element of $G$ has an inverse: $ab = ba = e$, $ee = e$.

    Hence $G$ with the matrix multiplication is a group.
\end{proof}

\newpage
% section 2/exercise 19
\begin{exercise}
    Let $S$ be the set of all real numbers except $-1$. Define $*$ on $S$ by
    \[
        a * b = a + b + ab.
    \]
    \begin{enumerate}[label={\textbf{\alph*.}}]
        \item Show that $*$ gives a binary operation on $S$.
        \item Show that $\anglebracket{S, *}$ is a group.
        \item Find the solution of the equation $2 * x * 3 = 7$ in $S$.
    \end{enumerate}
\end{exercise}

\begin{proof}
    \begin{enumerate}[label={\textbf{\alph*.}}]
        \item Let $a, b$ be elements of $S$
              \[
                  a * b + 1 = 1 + a + b + ab = (1 + a)(1 + b)\ne 0.
              \]

              So $a * b\ne -1$, which means $*$ is a binary operation on $S$.
        \item Let $a, b, c$ be elements of $S$
              \begin{align*}
                  (a * b) * c & = (a + b + ab) * c                   \\
                              & = a + b + c + ab + ca + bc + abc     \\
                              & = a + (b + c + bc) + (ab + ac + abc) \\
                              & = a + b * c + a\cdot (b * c)         \\
                              & = a * (b * c)
              \end{align*}

              So $\anglebracket{S, *}$ is associative.
              \[
                  a * 0 = a + 0 + a\cdot 0 = a = 0 + a + 0\cdot a = 0 * a.
              \]

              So $\anglebracket{S, *}$ has an identity element.
              \[
                  \begin{split}
                      a * \frac{-a}{a+1} = a + \frac{-a}{a+1} + \frac{-a^{2}}{a+1} = \frac{a^{2}}{a+1} + \frac{-a^{2}}{a+1} = 0, \\
                      \frac{-a}{a+1} * a = \frac{-a}{a+1} + a + \frac{-a^{2}}{a+1} = \frac{a^{2}}{a+1} + \frac{-a^{2}}{a+1} = 0.
                  \end{split}
              \]

              Hence $\anglebracket{S, *}$ is a group.
        \item
              \begin{align*}
                  2 * x * 3                                          & = 7                          \\
                  \left(\left(\frac{-2}{3} * 2\right) * x\right) * 3 & = \frac{-2}{3} * 7           \\
                  (0 * x) * 3                                        & = \frac{5}{3}                \\
                  x * 3                                              & = \frac{5}{3}                \\
                  x * \left(3 * \frac{-3}{4}\right)                  & = \frac{5}{3} * \frac{-3}{4} \\
                  x * 0                                              & = \frac{-1}{3}               \\
                  x                                                  & = \frac{-1}{3}
              \end{align*}

              Hence $x = \frac{-1}{3}$.
    \end{enumerate}
\end{proof}

\newpage
% section 2/exercise 20
\begin{exercise}
    This exercise shows that there are two nonisomorphic group structures on a set of $4$ elements.

    Let the set be $\{ e, a, b, c \}$, with $e$ the identity element for the group operation. A group table would then have to start in the manner shown in Table 2.29. The square indicated by the question mark cannot be filled in with $a$. It must be filled in either with the identity element $e$ or with an element different from both $e$ and $a$. In this latter case, it is no loss of generality to assume that this element is $b$. If this square is filled in with $e$, the table can then be completed in two ways to give a group. Find these two tables. (You need not check the associative law.) If this square is filled in with $b$, then the table can only be completed in one way to give a group. Find this table. (Again, you need not check the associative law.) Of the three tables you now have, two give isomorphic groups. Determine which two tables these are, and give the one-to-one onto relabeling function which is an isomorphism.

    \begin{enumerate}[label={\textbf{\alph*}}]
        \item Are all groups of $4$ elements commutative?
        \item Find a way to relabel the four matrices
              \[
                  \left\{\begin{bmatrix}
                      1 & 0 \\
                      0 & 1
                  \end{bmatrix},
                  \begin{bmatrix}
                      0 & -1 \\
                      1 & 0
                  \end{bmatrix},
                  \begin{bmatrix}
                      -1 & 0  \\
                      0  & -1
                  \end{bmatrix},
                  \begin{bmatrix}
                      0  & 1 \\
                      -1 & 0
                  \end{bmatrix}\right\}
              \]

              so the atrix multiplication table is identical to one you constructed. This shows that the table you constructed defines an associative operation and therefore gives a group.
        \item So that for a particular value of $n$, the group elements given in Exercise 14 can be relabeled so their group table is identitcal to one you constructed. This implies the operation in the table is also associate.
    \end{enumerate}
\end{exercise}

\begin{proof}
    Possible group tables are
    \[
        \begin{array}{c|cccc}
            * & e & a & b & c \\
            \hline
            e & e & a & b & c \\
            a & a & b & c & e \\
            b & b & c & e & a \\
            c & c & e & a & b
        \end{array},\qquad
        \begin{array}{c|cccc}
            * & e & a & b & c \\
            \hline
            e & e & a & b & c \\
            a & a & e & c & b \\
            b & b & c & e & a \\
            c & c & b & a & e
        \end{array},\qquad
        \begin{array}{c|cccc}
            * & e & a & b & c \\
            \hline
            e & e & a & b & c \\
            a & a & e & c & b \\
            b & b & c & a & e \\
            c & c & b & e & a
        \end{array}
    \]

    The 1st and the 3rd tables give isomorphic groups. Relabel: $(e, a, b, c)\mapsto (e, b, c, a)$.

    \begin{enumerate}[label={\textbf{\alph*.}}]
        \item These groups of 4 elements are commutative.
        \item One way to make the matrix multiplication table identical to the 3rd table
              \[
                  e = \begin{bmatrix}
                      1 & 0 \\
                      0 & 1
                  \end{bmatrix},
                  b = \begin{bmatrix}
                      0 & -1 \\
                      1 & 0
                  \end{bmatrix},
                  a = \begin{bmatrix}
                      -1 & 0  \\
                      0  & -1
                  \end{bmatrix},
                  c = \begin{bmatrix}
                      0  & 1 \\
                      -1 & 0
                  \end{bmatrix}
              \]
        \item When $n = 2$, we label
              \[
                  e = \begin{bmatrix}
                      1 & 0 \\
                      0 & 1
                  \end{bmatrix},
                  a = \begin{bmatrix}
                      0 & -1 \\
                      1 & 0
                  \end{bmatrix},
                  b = \begin{bmatrix}
                      -1 & 0  \\
                      0  & -1
                  \end{bmatrix},
                  c = \begin{bmatrix}
                      0  & 1 \\
                      -1 & 0
                  \end{bmatrix}
              \]
    \end{enumerate}
\end{proof}

\newpage
% section 2/exercise 21
\begin{exercise}
    According to Exercise $12$ of Section $1$, there are $16$ possible binary operations on a set of $2$ elements. How many of these give a structure of a group? How many of the $19,683$ possible binary operations on a set of $3$ elements give a group structure.
\end{exercise}

\begin{proof}
    Let $X = \{ a, b \}$ be a set of $2$ elements.

    A binary operation on $X$ give a structure of a group if in its group table:
    \begin{itemize}
        \item each column and each row contains distinct elements,
        \item there is exactly one row which is identical to the top header, there is exactly one column which is identical to the left header.
    \end{itemize}

    So there are two possible binary operations which can give group structure.
    \[
        \begin{array}{c|cc}
            * & a & b \\
            \hline
            a & a & b \\
            b & b & a
        \end{array},\qquad
        \begin{array}{c|cc}
            * & a & b \\
            \hline
            a & b & a \\
            b & a & b
        \end{array}
    \]

    After checking for associativity, we conclude that there are $2$ binary operations that give group structure. However, they are isomorphic.

    Let $Y = \{ a, b, c \}$ be a set of $3$ elements.

    The following tables correspond to binary operations which possibly give group structure.
    \[
        \begin{array}{c|ccc}
            * & a & b & c \\
            \hline
            a & a & b & c \\
            b & b & c & a \\
            c & c & a & b \\
        \end{array},\qquad
        \begin{array}{c|ccc}
            * & a & b & c \\
            \hline
            a & c & a & b \\
            b & a & b & c \\
            c & b & c & a \\
        \end{array},\qquad
        \begin{array}{c|ccc}
            * & a & b & c \\
            \hline
            a & b & c & a \\
            b & c & a & b \\
            c & a & b & c \\
        \end{array}
    \]

    After checking for associativity, we conclude that there are $3$ binary operations that give group structure. However, they are isomorphic.
\end{proof}

\subsection*{Concepts}

\newpage
% section 2/exercise 22
\begin{exercise}
    Consider our axioms $\mathscr{G}_{1}, \mathscr{G}_{2}$, and $\mathscr{G}_{3}$ for a group. We gave them in the order $\mathscr{G}_{1}\mathscr{G}_{2}\mathscr{G}_{3}$. Conceivable other orders to state the axioms are  $\mathscr{G}_{1}\mathscr{G}_{3}\mathscr{G}_{2}$, $\mathscr{G}_{2}\mathscr{G}_{1}\mathscr{G}_{3}$, $\mathscr{G}_{2}\mathscr{G}_{3}\mathscr{G}_{1}$, $\mathscr{G}_{3}\mathscr{G}_{1}\mathscr{G}_{2}$, and $\mathscr{G}_{3}\mathscr{G}_{2}\mathscr{G}_{1}$. Of these six possible orders, exactly three are acceptable for a definition. Which orders are not acceptable, any why?
\end{exercise}

\begin{proof}
    Inacceptable orders are $\mathscr{G}_{1}\mathscr{G}_{3}\mathscr{G}_{2}$, $\mathscr{G}_{3}\mathscr{G}_{1}\mathscr{G}_{2}$, $\mathscr{G}_{3}\mathscr{G}_{2}\mathscr{G}_{1}$. Because $\mathcal{G}_{3}$ depends on the definition of identity element, which is defined in $\mathcal{G}_{2}$.
\end{proof}

\newpage
% section 2/exercise 23
\begin{exercise}
    The following ``definitions'' of a group are taken verbatim, including spelling and punctuation, from papers of students who wrote a bit too quickly and carelessly. Criticize them.
    \begin{enumerate}
        \item A group $G$ is a set of elements together with a binary operation $*$ such that the following conditions are satisfied

              $*$ is associative

              There exists $e\in G$ such that
              \[
                  e * x = x * e = x = \text{identity}
              \]

              For every $a\in G$ there exists an $a'$ (inverse) such that
              \[
                  a\cdot a' = a'\cdot a = e
              \]
        \item A group is a set $G$ such that

              The operation on $G$ is associative

              there is an identity element ($e$) in $G$.

              for every $a\in G$, there is an $a'$ (inverse for each element)
        \item A group is a set with a binary operation such

              the binary operation is defined

              an inverse exists

              an identity element exists
        \item A set $G$ is called a group over the binery operation $*$ such that for all $a, b\in G$

              Binary operation $*$ is associative under addition

              there exist an element $\{e\}$ such that
              \[
                  a * e = e * a = e
              \]

              Fore every element $a$ there exists an element $a'$ such that
              \[
                  a * a' = a' * a = e
              \]
    \end{enumerate}
\end{exercise}

\begin{proof}
    \begin{enumerate}
        \item $x$ is not neccessarily equal to the identity element. The notations for the binary opertion are inconsistent ($*$ and $\cdot$).
        \item First letter of each sentence should be in uppercase. A group is a set $G$ with a binary operation. The meaning of identity element and inverse must be clarified.
        \item The definition of idenity element, inverse must be given. Sentences are not clear.
        \item $b$ isn't used anywhere. $e$ must be used instead of $\{e\}$. There are typos.
    \end{enumerate}
\end{proof}

\newpage
% section 2/exercise 24
\begin{exercise}
    Give a table defining an operation satisfying axioms $\mathscr{G}_{2}$ and $\mathscr{G}_{3}$ in the definition of a group, but not satisfying axiom $\mathscr{G}_{1}$ for the set
    \begin{enumerate}[label={\textbf{\alph*.}}]
        \item $\{ e, a, b \}$
        \item $\{ e, a, b, c \}$
    \end{enumerate}
\end{exercise}

\begin{proof}
    \begin{enumerate}[label={\textbf{\alph*.}}]
        \item
              \[
                  \begin{array}{c|ccc}
                      * & e & a & b \\
                      \hline
                      e & e & a & b \\
                      a & a & e & e \\
                      b & b & e & a \\
                  \end{array}
              \]

              $(a * a) * b = e * b = b \ne a = a * e = a * (a * b)$. So $*$ is not associative.
        \item $\{ e, a, b, c \}$
              \[
                  \begin{array}{c|cccc}
                      * & e & a & b & c \\
                      \hline
                      e & e & a & b & c \\
                      a & a & c & e & b \\
                      b & b & e & a & a \\
                      c & c & b & a & e
                  \end{array}
              \]

              $(b * a) * c = e * c = c \ne a = b * b = b * (a * c)$. So $*$ is not associative.
    \end{enumerate}
\end{proof}

\newpage
% section 2/exercise 25
\begin{exercise}
    Mark each of the following true or false.
    \begin{enumerate}[label={\textbf{\alph*.}}]
        \item A group may have more tha one identity element.
        \item Any two groups of three elements are isomorphic.
        \item In a group, each linear equation has a solution.
        \item The proper attitude toward a definition is to memorize it so that you can reproduce it word for word as in the text.
        \item Any definition a person gives for a group is correct provided that everything that is a group by that person's defintion is also a group by the definition in the text.
        \item Any definition a person gives for a group is correct provided he or she can show that everything that satisfies the definition satisfies the one in the text and conversely.
        \item Every finite group of at most three elements is abelian.
        \item An equation of the form $a * x * b = c$ always has a unique solution in a group.
        \item The empty set can be considered a group.
        \item Every group is a binary algebraic structure.
    \end{enumerate}
\end{exercise}

\begin{proof}
    \begin{enumerate}[label={\textbf{\alph*.}}]
        \item False.
        \item True.
        \item True.
        \item False.
        \item False.
        \item True.
        \item True.
        \item True.
        \item False.
        \item True.
    \end{enumerate}
\end{proof}

\subsection*{Proof synopsis}

\newpage
% section 2/exercise 26
\begin{exercise}
    Give a one-sentence synopsis of the proof of the left cancellation law in Theorem 2.16.
\end{exercise}

\begin{proof}
    Multiply both sides with the inverse of the left element and apply the associative law, it follows that the two left elements are equal.
\end{proof}

\newpage
% section 2/exercise 27
\begin{exercise}
    Give at most a two-sentence synopsis of the proof in Theorem 2.17 that an equation $ax = b$ has a unique solution in a group.
\end{exercise}

\begin{proof}
    Multiply both sides with the inverse of the coefficient and apply the associative law, we obtain a solution. To prove that this solution is unique, we suppose that there are two solutions and apply the left cancellation law to show that the two solutions are identical.
\end{proof}

\subsection*{Theory}

\newpage
% section 2/exercise 28
\begin{exercise}
    An element $a\ne e$ in a group is said to have order $2$ if $a * a = e$. Prove that if $G$ is a group and $a\in G$ has order $2$, then for any $b\in G$, $b' * a * b$ also has order $2$.
\end{exercise}

\begin{proof}
    \begin{align*}
        (b' * a * b) * (b' * a * b) & = (b' * a) * (b * b') * (a * b) & \text{associative law}   \\
                                    & = (b' * a) * e * (a * b)                                   \\
                                    & = (b' * a) * (a * b)                                       \\
                                    & = b' * (a * a) * b              & \text{associative law}   \\
                                    & = b' * a * b                    & \text{$a$ has order $2$}
    \end{align*}

    Hence, $b' * a * b$ also has order $2$.
\end{proof}

\newpage
% section 2/exercise 29
\begin{exercise}
    Show that if $G$ is a finite group with identity $e$ and with an even number of elements, then there is $a\ne e$ in $G$ such that $a * a = e$.
\end{exercise}

\begin{proof}
    Assume that there is no element other than $e$ that has order $2$.

    Let the elements of $G$ be $e, a_{1}, a_{2}, \ldots, a_{2n-1}$. Since $G$ is a group, each element $a_{i}$ has an unique inverse. So we can pair each $a_{i}$ with its inverse and none of them are $e$. Therefore, $\{ a_{1}, a_{2}, \ldots, a_{2n-1} \}$ contains an even number of elements, which contradicts the fact that $2n - 1$ is an odd number.

    So the initial assumption is false, and there is $a\ne e$ in $G$ such that $a * a = e$.
\end{proof}

\newpage
% section 2/exercise 30
\begin{exercise}
    Let $\mathbb{R}^{*}$ be the set of all real numbers except $0$. Define $*$ on $\mathbb{R}^{*}$ by letting $a * b = \abs{a}b$.
    \begin{enumerate}[label={\textbf{\alph*.}}]
        \item Show that $*$ gives an associative binary operation on $\mathbb{R}^{*}$.
        \item Show that there is a left identity for $*$ and a right inverse for each element in $\mathbb{R}^{*}$.
        \item Is $\mathbb{R}^{*}$ with this binary operation a group?
        \item Explain the significance of this exercise.
    \end{enumerate}
\end{exercise}

\begin{proof}
    \begin{enumerate}[label={\textbf{\alph*.}}]
        \item Let $a, b, c\in\mathbb{R}^{*}$
              \begin{align*}
                  (a * b) * c & = \abs{a}b * c    \\
                              & = \abs{\abs{a}b}c \\
                              & = \abs{ab}c       \\
                              & = \abs{a}\abs{b}c \\
                              & = \abs{a}(b * c)  \\
                              & = a * (b * c)
              \end{align*}

              So $*$ gives an associative binary operation on $\mathbb{R}^{*}$.
        \item $1 * b = b$, $-1 * b = b$ for all $b\in\mathbb{R}^{*}$.

              $a * \frac{1}{\abs{a}} = 1$, $a * \frac{-1}{\abs{a}} = -1$ for all $a\in\mathbb{R}^{*}$.
        \item $\mathbb{R}^{*}$ with this binary operation is not a group, because $\mathbb{R}^{*}$ does not have an identity element under the operation $*$.
        \item When checking the Axiom 2 of group structure, one must check if the identity element for both side.
    \end{enumerate}
\end{proof}

\newpage
% section 2/exercise 31
\begin{exercise}
    If $*$ is a binary operation on a set $S$, an element $x$ of $S$ is an \textbf{idempotent for $*$} if $x * x = x$. Prove hat a group has exactly one idempotent element.
\end{exercise}

\begin{proof}
    Let $e$ be the identity element of the group $\anglebracket{S, *}$. Since $e * e = e$, $e$ is an idempotent element.

    Suppose that $x\in G$ is an idempotent element, then $x * x = x$. Let $x'$ be the inverse of $x$, then $x' * (x * x) = x' * x$. According to Axiom 1, we obtain that $(x' * x) * x = x' * x$, and by Axiom 2, 3, it follows that $x = e$.

    Therefore, a group has exactly one idempotent element, which is also the identity element.
\end{proof}

\newpage
% section 2/exercise 32
\begin{exercise}
    Show that every group $G$ with identity $e$ and such that $x * x = e$ for all $x\in G$ is abelian.
\end{exercise}

\begin{proof}
    In a group, every element has a unique inverse.

    Let $a, b$ be two elements of $G$, and $a', b'$ be the inverses of $a, b$, respectively.
    \[
        \begin{split}
            (a * b) * (b' * a') = a * (b * b') * a' = a * e * a' = a * a' = e, \\
            (b' * a') * (a * b) = b' * (a' * a) * b = b' * e * b = b' * b = e.
        \end{split}
    \]

    So $b' * a'$ is the inverse element of $a * b$.

    On the one hand, $(a * b) * (a * b) = e$, which means $a * b$ is also an inverse of $a * b$. So $b' * a' = a * b$. On the other hand, $a = a'$ and $b' = b$ because $a * a = b * b = e$. Therefore $a * b = b * a$.

    Thus $G$ is abelian.
\end{proof}

\newpage
% section 2/exercise 33
\begin{exercise}
    Let $G$ be an abelian group and let $c^{n} = c * c * \cdots * c$ for $n$ factors $c$, where $c\in G$ and $n\in\mathbb{Z}^{+}$. Give a mathematical induction proof that ${(a * b)}^{n} = (a^{n}) * (b^{n})$ for all $a, b\in G$.
\end{exercise}

\begin{proof}
    The statement holds for $n = 1$, since $a * b = a * b$.

    Assume that the statement holds for $n = k$.
    \begin{align*}
        {(a * b)}^{k+1} & = {(a * b)}^{k} * (a * b)                                             \\
                        & = (a^{k}) * (b^{k}) * (a * b) & \text{induction hypothesis}           \\
                        & = (a^{k} * a) * (b^{k} * b)   & \text{associative and $G$ is abelian} \\
                        & = (a^{k+1}) * (b^{k+1})
    \end{align*}

    So the statement also holds for $n = k + 1$.

    According to the principle of mathematical induction, ${(a * b)}^{n} = (a^{n}) * (b^{n})$ for all $a, b\in G$.
\end{proof}

\newpage
% section 2/exercise 34
\begin{exercise}
    Suppose that $G$ is a group and $a, b\in G$ satisfy $a * b = b * a'$ where as usual, $a'$ is the inverse for $a$. Prove that $b * a = a' * b$.
\end{exercise}

\begin{proof}
    Let $e$ be the identity element of $G$.

    \begin{align*}
        b * a & = (e * b) * a         \\
              & = ((a' * a) * b) * a  \\
              & = (a' * (a * b)) * a  \\
              & = (a' * (b * a')) * a \\
              & = a' * ((b * a') * a) \\
              & = a' * (b * (a' * a)) \\
              & = a' * (b * e)        \\
              & = a' * b
    \end{align*}
\end{proof}

\newpage
% section 2/exercise 35
\begin{exercise}
    Suppose that $G$ is a group and $a$ and $b$ are elements of $G$ that satisfy $a * b = b * a^{3}$. Rewrite the element ${(a * b)}^{2}$ in the form $b^{k}a^{r}$ (See Exercise 33 for power notaion.)
\end{exercise}

\begin{proof}
    \begin{align*}
        {(a * b)}^{2} & = (a * b) * (a * b)               \\
                      & = (b * a^{3}) * (b * a^{3})       \\
                      & = b * a^{2} * (a * b) * a^{3}     \\
                      & = b * a^{2} * (b * a^{3}) * a^{3} \\
                      & = b * a * (a * b) * a^{6}         \\
                      & = b * a * (b * a^{3}) * a^{6}     \\
                      & = b * (a * b) * a^{3} * a^{6}     \\
                      & = b * (b * a^{3}) * a^{9}         \\
                      & = b^{2} * a^{12}.
    \end{align*}
\end{proof}

\newpage
% section 2/exercise 36
\begin{exercise}
    Let $G$ be a group with a finite number of elements. Show that for any $a\in G$, there exists an $n\in\mathbb{Z}^{+}$ such that $a^{n} = e$.
\end{exercise}

\begin{proof}
    Suppose that $G$ has $m$ elements.

    Consider $e = a^{0}, a, a^{2}, \ldots a^{m}$. These are $(m + 1)$ elements in $G$. So there exists two distinct non-negative integers $0\le p < q\le m$ such that $a^{p} = a^{q}$. Therefore, $a^{q-p} = e$.

    Hence, there exists an $n\in\mathbb{Z}^{+}$ such that $a^{n} = e$.
\end{proof}

\newpage
% section 2/exercise 37
\begin{exercise}
    Show that if ${(a * b)}^{2} = a^{2} * b^{2}$ for $a$ and $b$ in a group $G$, then $a * b = b * a$.
\end{exercise}

\begin{proof}
    Since ${(a + b)}^{2} = a^{2} * b^{2}$, then $a' * (a * b) * (a * b) * b' = (a' * a^{2}) * (b^{2} * b')$.

    It follows that $(a' * a) * (b * a) * (b * b') = (a' * a) * (a * b) * (b * b')$, so $b * a = a * b$.

    Thus, $G$ is abelian.
\end{proof}

\newpage
% section 2/exercise 38
\begin{exercise}
    Let $G$ be a group and let $a, b\in G$. Show that ${(a * b)}' = a' * b'$ if and only if $a * b = b * a$.
\end{exercise}

\begin{proof}
    For any $a, b\in G$, $(b * a)' = a' * b'$.

    $(\Rightarrow)$ If $a * b = b * a$, then $(a * b)' = (b * a)' = a' * b'$.

    $(\Leftarrow)$ If ${(a * b)}' = a' * b'$, then the inverse of ${(a * b)}'$ is equal to the inverse of $a' * b'$. The inverse of $a' * b'$ is $b * a$, the inverse of ${(a * b)}'$ is $a * b$. Therefore $a * b = b * a$.

    Hence ${(a * b)}' = a' * b'$ if and only if $a * b = b * a$.
\end{proof}

\newpage
% section 2/exercise 39
\begin{exercise}
    Let $G$ be a group and suppose that $a * b * c = e$ for $a, b, c\in G$. Show that $b * c * a = e$ also.
\end{exercise}

\begin{proof}
    $a' = a' * e = a' * (a * b * c) = (a' * a) * (b * c) = e * (b * c) = b * c$. So $b * c$ is the inverse of $a$. Therefore, $b * c * a = e$.
\end{proof}

\newpage
% section 2/exercise 40
\begin{exercise}
    Prove that a set $G$, together with a binary operation $*$ on $G$ satisfying the left axioms 1, 2, and 3 after Corollary 2.19, is a group.
\end{exercise}

\begin{proof}
    \begin{enumerate}[label={\textbf{Axiom \arabic*.}},itemindent=1cm]
        \item The binary operation $*$ on $G$ is associative.
        \item There exists a \textbf{left identity element} $e$ in $G$ such that $e * x = x$ for all $x\in G$.
        \item For each $a\in G$, there exists \textbf{a left inverse} $a'$ in $G$ such that $a' * a = e$.
    \end{enumerate}

    For every $x\in G$, there exists $x'\in G$ such that $x' * x = e$. Furthermore, $x' * (x * e) = (x' * x) * e = e * e = e$.

    There exists $y$ such that $y * x' = e$, then
    \[
        y * (x' * x) = y * (x' * (x * e)) = y
    \]

    Since $*$ on $G$ is associative, and $y * x' = e$, we obtain that $x = x * e = y$. Therefore $x * x' = e$. So that $e * x = x * e = x$ and $x' * x = e = x * x$, we conclude that $G$ with the binary operation $*$ that satisfies the left axioms is a group.
\end{proof}

\newpage
% section 2/exercise 41
\begin{exercise}
    Prove that a nonempty set $G$, together with an associate binary operation $*$ on $G$ such that
    \begin{center}
        $a * x = b$ and $y * a = b$ have solutions in $G$ for all $a, b\in G$
    \end{center}

    is a group.
\end{exercise}

\begin{proof}
    According to the hypothesis
    \begin{itemize}
        \item there exists $e\in G$ such that $e * a = a$
        \item for each $m\in G$, there exists $y\in G$ such that $a * y = m$
    \end{itemize}

    For each $m\in G$
    \begin{itemize}
        \item $e * m = e * (a * y) = (e * a) * y = a * y = m$. So $G$ has a \textbf{left identity element}.
        \item according to the hypothesis, there exists $y'$ such that $y' * y = e$, there exists $a'$ such that $a' * a = e$, so
              \begin{align*}
                  (y' * a') * m & = (y' * a') * (a * y)                                         \\
                                & = y' * ((a' * a) * y) & \text{associativity}                  \\
                                & = y' * (e * y)                                                \\
                                & = y' * y              & \text{$e$ is a left identity element} \\
                                & = e
              \end{align*}

              which implies $m$ has a \textbf{left inverse}.
    \end{itemize}

    Hence $G$ with the binary operation $*$ is associative, has a left identity element, and each element has a left inverse. According to Exercise 40 of Section 2, we conclude that $G$ with the binary operation is a group.
\end{proof}

\newpage
% section 2/exercise 42
\begin{exercise}
    Let $G$ be a group. Prove that $(a')' = a$.
\end{exercise}

\begin{proof}
    \begin{align*}
        (a')' & = (a')' * e        \\
              & = (a')' * (a' * a) \\
              & = ((a')' * a') * a \\
              & = e * a            \\
              & = a.
    \end{align*}
\end{proof}

\newpage
% section 2/exercise 43
\begin{exercise}
    Let $\phi$ be an isometry of the plane.
    \begin{enumerate}[label={\textbf{\alph*.}}]
        \item Prove that $\phi$ is a one-to-one function.
        \item Prove that $\phi$ maps onto $\mathbb{R}^{2}$.
    \end{enumerate}
\end{exercise}

\begin{proof}
    \begin{enumerate}[label={\textbf{\alph*.}}]
        \item Let $A, B$ be two points in the plane. Since $\phi$ is an isometry, then $\abs{\phi(A)\phi(B)} = \abs{AB}$. Therefore, $A\equiv B$ if and only if $\phi(A)\equiv\phi(B)$. Hence $\phi$ is a one-to-one function.
        \item Let $A, B$ be two points in the plane. Let $C = \phi(B)$. Consider two circles whose centers are $B, C$ and their radii equal $\abs{AC}$. Denote these circles by $\odot(B, \abs{AC})$ and $\odot(C, \abs{AC})$.

              $\phi$ restricted on $\odot(B, \abs{AC})$, into $\odot(C, \abs{AC})$ is still a one-to-one mapping.

              Let $M$ be a point on $\odot(B, \abs{AC})$. Since $\phi$ is an isometry, $\abs{BM} = \abs{C\phi(M)}$, so $\phi(M)$ lies on $\odot(C, \abs{AC})$. There are at most two points $P, Q$ on $\odot(B, \abs{AC})$ such that $\abs{MP} = \abs{MQ} = \abs{A\phi(M)}$. There are at most two points on $\odot(C, \abs{AC})$ such that their distances to $\phi(M)$ are equal to $\abs{A\phi(M)}$, and one of them is $A$. Therefore, either $\phi(P) = A$ or $\phi(Q) = A$. Hence, there exists a point on $\odot(B, \abs{AC})$ such that its image under $\phi$ is $A$.

              Hence, $\phi$ is onto $\mathbb{R}^{2}$.
    \end{enumerate}
\end{proof}

\newpage
% section 2/exercise 44
\begin{exercise}
    Prove that if $f: G_{1} \to G_{2}$ is a group isomorphism from the group $\anglebracket{G_{1}, {*}_{1}}$ to the group $\anglebracket{G_{2}, {*}_{2}}$, then $f^{-1}: G_{2} \to G_{1}$ is also a group isomorphism.
\end{exercise}

\begin{proof}
    $f: G_{1} \to G_{2}$ is one-to-one and onto, so that $f^{-1}: G_{2} \to G_{1}$ is also one-to-one and onto.

    Let $a', b'$ be elements of $G_{2}$, and $a = f^{-1}(a'), b = f^{-1}(b')$.
    \begin{align*}
        f^{-1}(a'\ {*}_{2}\ b') & = f^{-1}(f(a)\ {*}_{2}\ f(b))     \\
                                & = f^{-1}(f(a\ {*}_{1}\ b))        \\
                                & = a\ {*}_{1}\ b                   \\
                                & = f^{-1}(a')\ {*}_{1}\ f^{-1}(b')
    \end{align*}

    Thus $f^{-1}$ is also a group isomorphism.
\end{proof}

\newpage
% section 2/exercise 45
\begin{exercise}
    Suppose that $G$ is a group with $n$ elements and $A\subseteq G$ has more than $\frac{n}{2}$ elements. Prove that for every $g\in G$, there exists $a, b\in A$ such that $a * b = g$.
\end{exercise}

\begin{proof}
    Let $B = \{ x' * g \mid x\in A \}$. $\phi: A\to B$ is defined as $\phi(x) = x' * g$, then $\phi$ is a bijection. Therefore, $A$ and $B$ have the same number of elements. According to the principle of inclusion and exclusion, $\abs{A\cap B} = \abs{A} + \abs{B} - \abs{A\cup B} > \frac{n}{2} + \frac{n}{2} - n = 0$, so $A\cap B$ is not empty. Let $b$ be an element of $A\cap B$. Since $\phi$ is a bijection, there exists $a\in A$ such that $a' * g = b$, equivalently, $g = (a * a') * g = a * (a' * g) = a * b$.

    Thus, there exist $a, b\in A$ such that $a * b = g$.
\end{proof}

\newpage
\section{Abelian Examples}

\setcounter{exercise}{0}

In Exercises 1 through 9 compute the given arithmetic expression and give the answer in the form $a + bi$ for $a, b\in \mathbb{R}$.

\newpage
% section 3/exercise 1
\begin{exercise}
    $i^{3}$
\end{exercise}

\begin{proof}
    $i^{3} = {i}^{2}i = -i = 0 + (-1)i$.
\end{proof}

\newpage
% section 3/exercise 2
\begin{exercise}
    $i^{4}$
\end{exercise}

\begin{proof}
    $i^{4} = {i}^{2}{i}^{2} = (-1)\cdot (-1) = 1 = 1 + 0i$.
\end{proof}

\newpage
% section 3/exercise 3
\begin{exercise}
    $i^{26}$
\end{exercise}

\begin{proof}
    $i^{26} = {i}^{2}{i}^{24} = -1 = -1 + 0i$.
\end{proof}

\newpage
% section 3/exercise 4
\begin{exercise}
    ${(-i)}^{39}$
\end{exercise}

\begin{proof}
    ${(-i)}^{39} = {(-i)}^{3}{(-i)}^{36} = {(-i)}^{3} = i = 0 + 1i$.
\end{proof}

\newpage
% section 3/exercise 5
\begin{exercise}
    $(3 - 2i)(6 + i)$
\end{exercise}

\begin{proof}
    $(3 - 2i)(6 + i) = 20 - 9i$.
\end{proof}

\newpage
% section 3/exercise 6
\begin{exercise}
    $(8 + 2i)(3 - i)$
\end{exercise}

\begin{proof}
    $(8 + 2i)(3 - i) = 26 + (-2)i$.
\end{proof}

\newpage
% section 3/exercise 7
\begin{exercise}
    $(2 - 3i)(4 + i) + (6 - 5i)$
\end{exercise}

\begin{proof}
    $(2 - 3i)(4 + i) + (6 - 5i) = (11 - 10i) + (6 - 5i) = 17 - 15i = 17 + (-15)i$.
\end{proof}

\newpage
% section 3/exercise 8
\begin{exercise}
    ${(1+i)}^{3}$
\end{exercise}

\begin{proof}
    ${(1+i)}^{3} = 1^{3} + 3\cdot 1^{2}i + 3\cdot 1\cdot i^{2} + i^{3} = 1 + 3i - 3 - i = (-2) + 2i$.
\end{proof}

\newpage
% section 3/exercise 9
\begin{exercise}
    ${(1-i)}^{5}$
\end{exercise}

\begin{proof}
    \begin{align*}
        {(1-i)}^{5} & = 1^{5} - 5\cdot 1^{4}i + 10\cdot 1^{3}i^{2} - 10\cdot 1^{2}i^{3} + 5\cdot 1\cdot i^{4} - i^{5} \\
                    & = 1 - 5i - 10 + 10i + 5 - i                                                                     \\
                    & = (-4) + 4i.
    \end{align*}
\end{proof}

\newpage
% section 3/exercise 10
\begin{exercise}
    Find $\abs{5 - 12i}$.
\end{exercise}

\begin{proof}
    $\abs{5 - 12i} = \sqrt{5^{2} + 12^{2}} = 13$.
\end{proof}

\newpage
% section 3/exercise 11
\begin{exercise}
    Find $\abs{\pi + ei}$.
\end{exercise}

\begin{proof}
    $\abs{\pi + ei} = \sqrt{\pi^{2} + e^{2}}$.
\end{proof}

In Exercises 12 through 15 write the given complex number $z$ in the polar form $\abs{z}(p + qi)$ where $\abs{p + qi} = 1$.

\newpage
% section 3/exercise 12
\begin{exercise}
    $3 - 4i$
\end{exercise}

\begin{proof}
    $\abs{3 - 4i} = \sqrt{3^{2} + {(-4)}^{2}} = 5$.

    $3 - 4i = 5\left(\frac{3}{5} + \frac{-4}{5}i\right)$.
\end{proof}

\newpage
% section 3/exercise 13
\begin{exercise}
    $-1 - i$
\end{exercise}

\begin{proof}
    $\abs{-1 - i} = \sqrt{{(-1)}^{2} + {(-1)}^{2}} = \sqrt{2}$.

    $-1 - i = \sqrt{2}\left(\frac{-\sqrt{2}}{2} + \frac{-\sqrt{2}}{2}i\right)$.
\end{proof}

\newpage
% section 3/exercise 14
\begin{exercise}
    $12 + 5i$
\end{exercise}

\begin{proof}
    $\abs{12 + 5i} = \sqrt{12^{2} + 5^{2}} = 13$.

    $12 + 5i = 13\left( \frac{12}{13} + \frac{5}{13}i \right)$.
\end{proof}

\newpage
% section 3/exercise 15
\begin{exercise}
    $-3 + 5i$
\end{exercise}

\begin{proof}
    $\abs{-3 + 5i} = \sqrt{{(-3)}^{2} + 5^{2}} = \sqrt{34}$.

    $-3 + 5i = \sqrt{34}\left( \frac{-3\sqrt{34}}{34} + \frac{5\sqrt{34}}{34}i \right)$.
\end{proof}

In Exercises 16 through 21, find all solutions in $\mathbb{C}$ of the given equation.

\newpage
% section 3/exercise 16
\begin{exercise}
    $z^{4} = 1$
\end{exercise}

\begin{proof}
    In polar form, the equation is $z^{4} = \abs{z}^{4}(\cos (4\phi) + i\sin (4\phi))$.

    $z^{4} = 1$ so $\abs{z}^{4} = 1$, $\cos(4\phi) = 1$, and $\sin(4\phi) = 0$. Different values of $\phi$ in $0\le \phi < 2\pi$ are $0, \frac{\pi}{2}, \pi, \frac{3\pi}{2}$. So the roots are
    \[
        1,\quad i,\quad -1,\quad -i.
    \]
\end{proof}

\newpage
% section 3/exercise 17
\begin{exercise}
    $z^{4} = -1$
\end{exercise}

\begin{proof}
    In polar form, the equation is $\abs{z}^{4}(\cos (4\phi) + i\sin (4\phi)) = 1(\cos\pi + i\sin\pi)$.

    The roots of the equation are
    \[
        \frac{\sqrt{2}}{2} + \frac{\sqrt{2}}{2}i,\quad \frac{-\sqrt{2}}{2} + \frac{\sqrt{2}}{2}i,\quad \frac{-\sqrt{2}}{2} + \frac{-\sqrt{2}}{2}i,\quad \frac{\sqrt{2}}{2} + \frac{-\sqrt{2}}{2}i.
    \]
\end{proof}

\newpage
% section 3/exercise 18
\begin{exercise}
    $z^{3} = -125$
\end{exercise}

\begin{proof}
    In polar form, the equation is $\abs{z}^{3}(\cos (3\phi) + i\sin (3\phi)) = 5^{3}(\cos\pi + i\sin\pi)$.

    The roots of the equation are
    \[
        \frac{5}{2} + \frac{5\sqrt{3}}{2}i,\quad -5,\quad \frac{5}{2} + \frac{-5\sqrt{3}}{2}i.
    \]
\end{proof}

\newpage
% section 3/exercise 19
\begin{exercise}
    $z^{3} = -27i$
\end{exercise}

\begin{proof}
    In polar form, the equation is $\abs{z}^{3}(\cos (3\phi) + i\sin (3\phi)) = 3^{3}(\cos\frac{3\pi}{2} + \sin\frac{3\pi}{2}i)$.

    The roots of the equation are
    \[
        3i,\quad \frac{-3\sqrt{3}}{2} + \frac{-3}{2}i,\quad\frac{3\sqrt{3}}{2} + \frac{-3}{2}i.
    \]
\end{proof}

\newpage
% section 3/exercise 20
\begin{exercise}
    $z^{6} = 1$
\end{exercise}

\begin{proof}
    The roots of the equation are
    \[
        1,\quad \frac{1}{2} + \frac{\sqrt{3}}{2}i,\quad \frac{-1}{2} + \frac{\sqrt{3}}{2}i, -1,\quad \frac{-1}{2} + \frac{-\sqrt{3}}{2}i,\quad \frac{1}{2} + \frac{-\sqrt{3}}{2}i.
    \]
\end{proof}

\newpage
% section 3/exercise 21
\begin{exercise}
    $z^{6} = -64$
\end{exercise}

\begin{proof}
    In polar form, the equation is $\abs{z}^{6}(\cos(6\phi) + i\sin(6\phi)) = 2^{6}(\cos\pi + i\sin\pi)$.

    % pi/6 + 2pi/6 * 0 = pi/6
    % pi/6 + 2pi/6 * 1 = 3pi/6 = pi/2
    % pi/6 + 2pi/6 * 2 = 5pi/6
    % pi/6 + 2pi/6 * 3 = 7pi/6
    % pi/6 + 2pi/6 * 4 = 9pi/6
    % pi/6 + 2pi/6 * 5 = 11pi/6

    The roots of the equation are
    \[
        \sqrt{3} + i,\quad 2i,\quad -\sqrt{3} + i,\quad -\sqrt{3} - i,\quad -2i,\quad \sqrt{3} - i.
    \]
\end{proof}

In Exercises 22 through 27, compute the given expression using the indicated modular addition.

\newpage
% section 3/exercise 22
\begin{exercise}
    $10 {+}_{17} 16$
\end{exercise}

\begin{proof}
    $10 {+}_{17} 16 = 10 + 16 - 17 = 9$.
\end{proof}

\newpage
% section 3/exercise 23
\begin{exercise}
    $14 {+}_{99} 92$
\end{exercise}

\begin{proof}
    $14 {+}_{99} 92 = 14 + 92 - 99 = 7$.
\end{proof}

\newpage
% section 3/exercise 24
\begin{exercise}
    $3.141 {+}_{4} 2.718$
\end{exercise}

\begin{proof}
    $3.141 {+}_{4} 2.718 = 3.141 + 2.718 - 4 = 1.859$.
\end{proof}

\newpage
% section 3/exercise 25
\begin{exercise}
    $\frac{1}{2} {+}_{1} \frac{7}{8}$
\end{exercise}

\begin{proof}
    $\frac{1}{2} {+}_{1} \frac{7}{8} = \frac{1}{2} + \frac{7}{8} - 1 = \frac{3}{8}$.
\end{proof}

\newpage
% section 3/exercise 26
\begin{exercise}
    $\frac{3\pi}{4} {+}_{2\pi} \frac{3\pi}{2}$
\end{exercise}

\begin{proof}
    $\frac{3\pi}{4} {+}_{2\pi} \frac{3\pi}{2} = \frac{3\pi}{4} + \frac{3\pi}{2} - 2\pi = \frac{\pi}{4}$.
\end{proof}

\newpage
% section 3/exercise 27
\begin{exercise}
    $2\sqrt{2} {+}_{\sqrt{32}} 3\sqrt{2}$
\end{exercise}

\begin{proof}
    $2\sqrt{2} {+}_{\sqrt{32}} 3\sqrt{2} = 2\sqrt{2} + 3\sqrt{2} - \sqrt{32} = 5\sqrt{2} - 4\sqrt{2} = \sqrt{2}$.
\end{proof}

\newpage
% section 3/exercise 28
\begin{exercise}
    Explain why the expression $5 {+}_{6} 8$ in $\mathbb{R}_{6}$ makes no sense.
\end{exercise}

\begin{proof}
    The expression makes no sense because $8\notin \mathbb{R}_{6}$.
\end{proof}

In Exercises 29 through 34, find \textit{all} solutions $x$ of the given equation.

\newpage
% section 3/exercise 29
\begin{exercise}
    $x {+}_{10} 7 = 3$ in $\mathbb{Z}_{10}$
\end{exercise}

\begin{proof}
    $x = 3 {+}_{10} 3 = 3 + 3 = 6$.
\end{proof}

\newpage
% section 3/exercise 30
\begin{exercise}
    $x {+}_{2\pi} \pi = \frac{\pi}{2}$ in $\mathbb{R}_{2\pi}$
\end{exercise}

\begin{proof}
    $x = \frac{\pi}{2} {+}_{2\pi} \pi = \frac{\pi}{2} + \pi = \frac{3\pi}{2}$.
\end{proof}

\newpage
% section 3/exercise 31
\begin{exercise}
    $x {+}_{7} x = 3$ in $\mathbb{Z}_{7}$
\end{exercise}

\begin{proof}
    $x = 5$.
\end{proof}

\newpage
% section 3/exercise 32
\begin{exercise}
    $x {+}_{13} x {+}_{13} x = 5$ in $\mathbb{Z}_{13}$
\end{exercise}

\begin{proof}
    $x = 6$.
\end{proof}

\newpage
% section 3/exercise 33
\begin{exercise}
    $x {+}_{12} x = 2$ in $\mathbb{Z}_{12}$
\end{exercise}

\begin{proof}
    $x = 1$, or $x = 7$.
\end{proof}

\newpage
% section 3/exercise 34
\begin{exercise}
    $x {+}_{8} x {+}_{8} x {+}_{8} x = 4$ in $\mathbb{Z}_{8}$
\end{exercise}

\begin{proof}
    $x = 1$, or $x = 3$, or $x = 5$, or $x = 7$.
\end{proof}

\newpage
% section 3/exercise 35
\begin{exercise}
    Prove or give a counterexample to the statement that for any $n\in\mathbb{Z}^{+}$ and $a\in\mathbb{Z}_{n}$, the equation $x {+}_{n} x = a$ has at most two solutions in $\mathbb{Z}_{n}$.
\end{exercise}

\begin{proof}
    When $n$ is even and $a$ is odd. For every $x\in\mathbb{Z}_{n}$, $x {+}_{n} x$ is even. So $x {+}_{n} x = a$ has no solutions in this case.

    When $n$ is even and $a$ is even, $x = \frac{a}{2}$ and $x = \frac{a + n}{2}$ are two solutions. Suppose that $x = y$ is a solution (1).
    \begin{enumerate}[label={\textbf{Case \arabic*.}},itemindent=1cm]
        \item $0\le y + y < n$.

              If $y < \frac{a}{2}$, then $0\le y + y < a$, which is a contradiction to (1).

              If $y > \frac{a}{2}$, then $0\le a < y + y < n$, which is also a contradiction to (1).

              So $y = \frac{a}{2}$.
        \item $y + y \ge n$.

              If $y < \frac{a + n}{2}$, then $y + y - n < a$, which is a contradiction to (1).

              If $y > \frac{a + n}{2}$, then $y + y - n > a$, which is a contradiction to (1).

              So $y = \frac{a + n}{2}$.
    \end{enumerate}

    So when $n$ is even and $a$ is even, the equation has two solutions.

    When $n$ is odd, and $a$ is even, $x = \frac{a}{2}$ is a solution. Suppose that $x = y$ is a solution (2).
    \begin{enumerate}[label={\textbf{Case \arabic*}},itemindent=1cm]
        \item $0\le y + y < n$.

              If $y < \frac{a}{2}$, then $0\le y + y < a$, which is a contradiction to (2).

              If $y > \frac{a}{2}$, then $0\le a < y + y < n$, which is also a contradiction to (2).

              So $y = \frac{a}{2}$.
        \item $y + y\ge n$.

              $y + y - n$ is odd. Meanwhile, $a$ is even. So $x = y$ cannot be a solution.
    \end{enumerate}

    So when $n$ is odd and $a$ is even, the equation has one solution.

    When $n$ is odd, and $a$ is odd, $x = \frac{a+n}{2}$ is a solution. Suppose that $x = y$ is a solution (3).
    \begin{enumerate}[label={\textbf{Case \arabic*}},itemindent=1cm]
        \item $0\le y + y < n$.

              $y + y$ is even. Meanwhile, $a$ is odd. So $x = y$ cannot be a solution.

        \item $y + y\ge n$.

              If $y < \frac{a+n}{2}$, then $y + y - n < a$, which is a contradiction to (3).

              If $y > \frac{a+n}{2}$, then $y + y - n > a$, which is also a contradiction to (3).

              So $y = \frac{a+n}{2}$.
    \end{enumerate}

    So when $n$ is odd and $a$ is odd, the equation has one solution.

    In conclusion, the equation has at most two solution.
\end{proof}

\newpage
% section 3/exercise 36
\begin{exercise}
    Prove or give a counterexample to the statement that for any $n\in\mathbb{Z}^{+}$ and $a\in\mathbb{Z}_{n}$, if $n$ is not a multiple of $3$, then the equation $x {+}_{n} x {+}_{n} x = a$ has exactly one solution in $\mathbb{Z}_{m}$.
\end{exercise}

\begin{proof}
    If $n$ is not a multiple of $3$ then $n$ and $3$ are relatively prime. According to Bezout's theorem, there exist integer $s, t$ such that $3s + nt = 1$. If $3s + nt = 1$, then $3(as) + nat = a$. So $3(as)\equiv a\pmod{n}$. From this, we deduce two things: the equation has a solution which is $x = x_{0}$ such that $x_{0}\equiv as\pmod{n}$, and if $x = x_{1}$ is a solution, $x_{1}\equiv as \equiv x_{0} \pmod{n}$. Therefore, the equation has exactly one solution.
\end{proof}

\newpage
% section 3/exercise 37
\begin{exercise}
    There is an isomorphism of $U_{8}$ with $\mathbb{Z}_{8}$ in which $\zeta = e^{i(\pi/4)}\leftrightarrow 5$ and $\zeta^{2}\leftrightarrow 2$. Find the element of $\mathbb{Z}_{8}$ that corresponds to each of the remaining six elements $\zeta^{m}$ in $U_{8}$ for $m = 0, 3, 4, 5, 6$, and $7$.
\end{exercise}

\begin{proof}
    $\zeta^{0} \leftrightarrow 0$, $\zeta^{3} \leftrightarrow 7$, $\zeta^{4} \leftrightarrow 4$, $\zeta^{5}\leftrightarrow 1$, $\zeta^{6}\leftrightarrow 6$, and $\zeta^{7}\leftrightarrow 3$.
\end{proof}

\newpage
% section 3/exercise 38
\begin{exercise}
    There is an isomorphism of $U_{7}$ with $\mathbb{Z}_{7}$ in which $\zeta = e^{i(2\pi/7)}\leftrightarrow 4$. Find the element in $\mathbb{Z}_{7}$ to which $\zeta^{m}$ must correspond for $m = 0, 2, 3, 4, 5$, and $6$.
\end{exercise}

\begin{proof}
    $\zeta^{0} \leftrightarrow 0$, $\zeta^{2} \leftrightarrow 1$, $\zeta^{3} \leftrightarrow 5$, $\zeta^{4} \leftrightarrow 2$, $\zeta^{5} \leftrightarrow 6$, and $\zeta^{6} \leftrightarrow 3$.
\end{proof}

\newpage
% section 3/exercise 39
\begin{exercise}
    Why can there be no isomorphism of $U_{6}$ with $\mathbb{Z}_{6}$ in which $\zeta = e^{i(\pi/3)}$ corresponds to $4$?
\end{exercise}

\begin{proof}
    Assume that there is such an isomorphism.

    Then $\zeta^{3}$ corresponds to $4 {+}_{12} 4 {+}_{12} 4 = 0$. On the other hand, $\zeta^{0}$ corresponds to $0$, which is a contradiction.

    Hence there can be no isomorphism of $U_{6}$ with $\mathbb{Z}_{6}$ in which $\zeta = e^{i(\pi/3)}\leftrightarrow 4$.
\end{proof}

\newpage
% section 3/exercise 40
\begin{exercise}
    Derive the formulas
    \[
        \sin(a + b) = \sin a\cos b + \cos a\sin b
    \]

    and
    \[
        \cos(a + b) = \cos a\cos b - \sin a\sin b
    \]

    by using Euler's formula and computing $e^{ia}e^{ib}$.
\end{exercise}

\begin{proof}
    $e^{ia}e^{ib} = e^{i(a+b)} = \cos(a+b) + i\sin(a+b)$.

    $e^{ia}e^{ib} = (\cos a + i\sin a)(\cos b + i\sin b) = (\cos a\cos b - \sin a\sin b) + i(\sin a\cos b + \cos a\sin b)$.

    Thus $\sin(a + b) = \sin a\cos b + \cos a\sin b$ and $\cos(a + b) = \cos a\cos b - \sin a\sin b$.
\end{proof}

\newpage
% section 3/exercise 41
\begin{exercise}
    Let $z_{1} = \abs{z_{1}}(\cos{\theta_{1}} + i\sin{\theta_{1}})$ and $z_{2} = \abs{z_{2}}(\cos{\theta_{2}} + i\sin{\theta_{2}})$. Use the trigonometric identities in Exercise 38 to derive $z_{1}z_{2} = \abs{z_{1}}\abs{z_{2}}(\cos{(\theta_{1} + \theta_{2})} + i\sin{(\theta_{1} + \theta_{2})})$.
\end{exercise}

\begin{proof}
    \begin{align*}
        z_{1}z_{2} & = \abs{z_{1}}\abs{z_{2}}(\cos a + i\sin a)(\cos b + i\sin b)                             \\
                   & = \abs{z_{1}}\abs{z_{2}}((\cos a\cos b - \sin a\sin b) + i(\sin a\cos b + \cos a\sin b)) \\
                   & = \abs{z_{1}}\abs{z_{2}}(\cos{(a+b)} + i\sin{(a+b)}).
    \end{align*}
\end{proof}

\newpage
% section 3/exercise 42
\begin{exercise}
    \begin{enumerate}[topsep=0pt,itemsep=0pt,label={\textbf{\alph*.}}]
        \item Derive a formula for $\cos{3\theta}$ in terms of $\sin{\theta}$ and $\cos{\theta}$ using Euler's formula.
        \item Derive the formula $\cos{3\theta} = 4\cos^{3}{\theta} - 3\cos{\theta}$ from part (a) and the identity $\sin^{2}{\theta} + \cos^{2}{\theta} = 1$.
    \end{enumerate}
\end{exercise}

\begin{proof}
    \begin{enumerate}[topsep=0pt,itemsep=0pt,label={\textbf{\alph*.}}]
        \item \begin{align*}
                  \cos{3\theta} & = \cos{2\theta}\cos{\theta} - \sin{\theta}\sin{2\theta}                             \\
                                & = \cos{\theta}(\cos^{2}{\theta} - \sin^{2}{\theta}) - 2\sin^{2}{\theta}\cos{\theta} \\
                                & = \cos^{3}{\theta} - \cos{\theta}\sin^{2}{\theta} - 2\sin^{2}{\theta}\cos{\theta}.
              \end{align*}
        \item \begin{align*}
                  \cos{3\theta} & = \cos^{3}{\theta} - \cos{\theta}\sin^{2}{\theta} - 2\sin^{2}{\theta}\cos{\theta}             \\
                                & = \cos^{3}{\theta} - \cos{\theta}(1 - \cos^{2}{\theta}) - 2(1 - \cos^{2}{\theta})\cos{\theta} \\
                                & = \cos^{3}{\theta} - \cos{\theta} + \cos^{3}{\theta} + 2\cos^{3}{\theta} - 2\cos{\theta}      \\
                                & = 4\cos^{3}{\theta} - 3\cos{\theta}.
              \end{align*}
    \end{enumerate}
\end{proof}

\newpage
% section 3/exercise 43
\begin{exercise}
    Recall the power series expansions
    \begin{align*}
        e^{x}  & = 1 + x + \frac{x^{2}}{2!} + \frac{x^{3}}{3!} + \frac{x^{4}}{4!} + \cdots + \frac{x^{n}}{n!} + \cdots,                            \\
        \sin x & = x - \frac{x^{3}}{3!} + \frac{x^{5}}{5!} - \frac{x^{7}}{7!} + \cdots + {(-1)}^{n-1}\frac{x^{2n-1}}{(2n-1)!} + \cdots, \text{and} \\
        \cos x & = 1 - \frac{x^{2}}{2!} + \frac{x^{4}}{4!} - \frac{x^{6}}{6!} + \cdots + {(-1)}^{n}\frac{x^{2n}}{(2n)!} + \cdots
    \end{align*}

    from calculus. Derive Euler's formula $e^{i\theta} = \cos{\theta} + i\sin{\theta}$ formally from these three series expansions.
\end{exercise}

\begin{proof}
    \begin{align*}
        e^{i\theta} & = 1 + i\theta + \frac{{(i\theta)}^{2}}{2!} + \frac{{(i\theta)}^{3}}{3!} + \frac{{(i\theta)}^{4}}{4!} + \cdots                                                                 \\
                    & = \left(1 + \frac{{(i\theta)}^{2}}{2!} + \frac{{(i\theta)}^{4}}{4!} + \cdots\right) + \left(i\theta + \frac{{(i\theta)}^{3}}{3!} + \frac{{(i\theta)}^{5}}{5!} + \cdots\right) \\
                    & = \left(1 - \frac{x^{2}}{2!} + \frac{x^{4}}{4!} + \cdots \right) + \left(i\theta - i\frac{\theta^{3}}{3!} + i\frac{\theta^{5}}{5!} - \cdots \right)                           \\
                    & = \cos{\theta} + i\sin{\theta}.\qedhere
    \end{align*}
\end{proof}

\newpage
% section 3/exercise 44
\begin{exercise}
    Prove that for any $n\in\mathbb{Z}^{+}$, $\anglebracket{\mathbb{Z}_{n}, {+}_{n}}$ is associative without using the fact that $U_{n}$ is associative.
\end{exercise}

\begin{proof}
    Let $a, b, c$ be elements of $\mathbb{Z}_{n}$.

    $a {+}_{n} b$ and $a + b$ are congruent modulo $n$. So $(a {+}_{n} b) {+}_{n} c$ and $(a + b) + c$ are congruent modulo $n$.

    $b {+}_{n} c$ and $b + c$ are congruent modulo $n$. So $a {+}_{n} (b {+}_{n} c)$ and $a + (b + c)$ are congruent modulo $n$.

    Since $(a + b) + c$ and $a + (b + c)$ are congruent modulo $n$, we conclude that $(a {+}_{n} b) {+}_{n} c$ and $a {+}_{n} (b {+}_{n} c)$ are congruent modulo $n$. On the other hand, $0\le (a {+}_{n} b) {+}_{n} c, a {+}_{n} (b {+}_{n} c) < n$, so $(a {+}_{n} b) {+}_{n} c + a {+}_{n} (b {+}_{n} c)$.

    Thus $\anglebracket{\mathbb{Z}_{n}, {+}_{n}}$ is associative.
\end{proof}

\newpage
% section 3/exercise 45
\begin{exercise}
    Let $b, c\in\mathbb{R}^{+}$. Find a one-to-one and onto function $f: \mathbb{R}_{b} \to \mathbb{R}_{c}$ that has the homomorphism property. Conclude that $\mathbb{R}_{c}$ is an abelian group that is isomorphic with $U$.
\end{exercise}

\begin{proof}
    We define $f(x) = \frac{cx}{b}$. Let $x, y$ be elements of $\mathbb{R}_{b}$. $f$ is a one-to-one and onto function.

    If $0\le x + y < b$,
    \[
        f(x {+}_{b} y) = f(x + y) = \frac{c(x+y)}{b} = \frac{cx}{b} + \frac{cy}{b} = f(x) {+}_{c} f(y).
    \]

    If $x + y \ge b$,
    \[
        f(x {+}_{b} y) = f(x + y - b) = \frac{c(x + y - b)}{b} = \frac{cx}{b} + \frac{cy}{b} - c = f(x) + f(y) - c = f(x) {+}_{c} f(y).
    \]

    Therefore, $\mathbb{R}_{b}$ and $\mathbb{R}_{c}$ are isomorphic, for any positive real numbers $b, c$. Since $U\simeq\mathbb{R}_{2\pi}$ and $\mathbb{R}_{b}$ is an abelian group, we conclude that $\mathbb{R}_{c}$ is an abelian group that is isomorphic with $U$.
\end{proof}

\newpage
% section 3/exercise 46
\begin{exercise}
    Prove that for any $n\geq 1$, $U_{n}$ is a group.
\end{exercise}

\begin{proof}
    $U_{n}$ has $n$ elements $\zeta^{0}, \zeta^{1}, \zeta^{2}, \ldots, \zeta^{n-1}$, where $\zeta = e^{i(2\pi/n)}$.

    $U_{n}$ is closed under multiplication. $U_{n}$ is associative because complex number multiplication is associative. $U_{n}$ has an identity element, which is $1 = \zeta^{0}$. Each element of $U_{n}$ has an inverse, $\zeta^{0}\zeta^{0} = 1$ and $\zeta^{m}\zeta^{n-m} = \zeta^{n} = 1$ (if $0 < m < n$).

    Hence $U_{n}$ is a group.
\end{proof}

\newpage
\section{Nonabelian Examples}
\setcounter{exercise}{0}

\subsection*{Computation}

In Exercises l through 5, compute the indicated product involving the following permutations in $S_{6}$:

\[
    \sigma = \begin{pmatrix}
        1 & 2 & 3 & 4 & 5 & 6 \\
        3 & 1 & 4 & 5 & 6 & 2
    \end{pmatrix},
    \qquad
    \tau = \begin{pmatrix}
        1 & 2 & 3 & 4 & 5 & 6 \\
        2 & 4 & 1 & 3 & 6 & 5
    \end{pmatrix},
    \qquad
    \mu = \begin{bmatrix}
        1 & 2 & 3 & 4 & 5 & 6 \\
        5 & 2 & 4 & 3 & 1 & 6
    \end{bmatrix}
\]

\newpage
% section 4/exercise 1
\begin{exercise}
    $\tau\sigma$
\end{exercise}

\begin{proof}
    \[
        \tau\sigma =
        \begin{pmatrix}
            3 & 1 & 4 & 5 & 6 & 2 \\
            1 & 2 & 3 & 6 & 5 & 4
        \end{pmatrix},
        \begin{pmatrix}
            1 & 2 & 3 & 4 & 5 & 6 \\
            3 & 1 & 4 & 5 & 6 & 2
        \end{pmatrix} =
        \begin{pmatrix}
            1 & 2 & 3 & 4 & 5 & 6 \\
            1 & 2 & 3 & 6 & 5 & 4
        \end{pmatrix}
    \]
\end{proof}

\newpage
% section 4/exercise 2
\begin{exercise}
    ${\tau}^{2}\sigma$
\end{exercise}

\begin{proof}
    \[
        {\tau}^{2} =
        \begin{pmatrix}
            2 & 4 & 1 & 3 & 6 & 5 \\
            4 & 3 & 2 & 1 & 5 & 6
        \end{pmatrix}
        \begin{pmatrix}
            1 & 2 & 3 & 4 & 5 & 6 \\
            2 & 4 & 1 & 3 & 6 & 5
        \end{pmatrix} =
        \begin{pmatrix}
            1 & 2 & 3 & 4 & 5 & 6 \\
            4 & 3 & 2 & 1 & 5 & 6
        \end{pmatrix}
    \]
    \[
        {\tau}^{2}\sigma =
        \begin{pmatrix}
            3 & 1 & 4 & 5 & 6 & 2 \\
            2 & 4 & 1 & 5 & 6 & 3
        \end{pmatrix}
        \begin{pmatrix}
            1 & 2 & 3 & 4 & 5 & 6 \\
            3 & 1 & 4 & 5 & 6 & 2
        \end{pmatrix} =
        \begin{pmatrix}
            1 & 2 & 3 & 4 & 5 & 6 \\
            2 & 4 & 1 & 5 & 6 & 3
        \end{pmatrix}
    \]
\end{proof}

\newpage
% section 4/exercise 3
\begin{exercise}
    $\mu{\sigma}^{2}$
\end{exercise}

\begin{proof}
    \[
        {\sigma}^{2} =
        \begin{pmatrix}
            3 & 1 & 4 & 5 & 6 & 2 \\
            4 & 3 & 5 & 6 & 2 & 1
        \end{pmatrix}
        \begin{pmatrix}
            1 & 2 & 3 & 4 & 5 & 6 \\
            3 & 1 & 4 & 5 & 6 & 2
        \end{pmatrix} =
        \begin{pmatrix}
            1 & 2 & 3 & 4 & 5 & 6 \\
            4 & 3 & 5 & 6 & 2 & 1
        \end{pmatrix}
    \]
    \[
        \mu{\sigma}^{2} =
        \begin{bmatrix}
            4 & 3 & 5 & 6 & 2 & 1 \\
            3 & 4 & 1 & 6 & 2 & 5
        \end{bmatrix}
        \begin{pmatrix}
            1 & 2 & 3 & 4 & 5 & 6 \\
            4 & 3 & 5 & 6 & 2 & 1
        \end{pmatrix} =
        \begin{pmatrix}
            1 & 2 & 3 & 4 & 5 & 6 \\
            3 & 4 & 1 & 6 & 2 & 5
        \end{pmatrix}
    \]
\end{proof}

\newpage
% section 4/exercise 4
\begin{exercise}
    ${\sigma}^{-2}\tau$
\end{exercise}

\begin{proof}
    \[
        {\sigma}^{-1} =
        \begin{pmatrix}
            1 & 2 & 3 & 4 & 5 & 6 \\
            2 & 6 & 1 & 3 & 4 & 5
        \end{pmatrix}
    \]
    \[
        {\sigma}^{-2} =
        \begin{pmatrix}
            2 & 6 & 1 & 3 & 4 & 5 \\
            6 & 5 & 2 & 1 & 3 & 4
        \end{pmatrix} =
        \begin{pmatrix}
            1 & 2 & 3 & 4 & 5 & 6 \\
            2 & 6 & 1 & 3 & 4 & 5
        \end{pmatrix} =
        \begin{pmatrix}
            1 & 2 & 3 & 4 & 5 & 6 \\
            6 & 5 & 2 & 1 & 3 & 4
        \end{pmatrix}
    \]
    \[
        {\sigma}^{-2}\tau =
        \begin{pmatrix}
            2 & 4 & 1 & 3 & 6 & 5 \\
            5 & 1 & 6 & 2 & 4 & 3
        \end{pmatrix}
        \begin{pmatrix}
            1 & 2 & 3 & 4 & 5 & 6 \\
            2 & 4 & 1 & 3 & 6 & 5
        \end{pmatrix} =
        \begin{pmatrix}
            1 & 2 & 3 & 4 & 5 & 6 \\
            5 & 1 & 6 & 2 & 4 & 3
        \end{pmatrix}
    \]
\end{proof}

\newpage
% section 4/exercise 5
\begin{exercise}
    ${\sigma}^{-1}\tau\sigma$
\end{exercise}

\begin{proof}
    \[
        {\sigma}^{-1} =
        \begin{pmatrix}
            1 & 2 & 3 & 4 & 5 & 6 \\
            2 & 6 & 1 & 3 & 4 & 5
        \end{pmatrix}
    \]
    \[
        {\sigma}^{-1}\tau =
        \begin{pmatrix}
            2 & 4 & 1 & 3 & 6 & 5 \\
            6 & 3 & 2 & 1 & 5 & 4
        \end{pmatrix}
        \begin{pmatrix}
            1 & 2 & 3 & 4 & 5 & 6 \\
            2 & 4 & 1 & 3 & 6 & 5
        \end{pmatrix} =
        \begin{pmatrix}
            1 & 2 & 3 & 4 & 5 & 6 \\
            6 & 3 & 2 & 1 & 5 & 4
        \end{pmatrix}
    \]
    \[
        {\sigma}^{-1}\tau{\sigma} =
        \begin{pmatrix}
            3 & 1 & 4 & 5 & 6 & 2 \\
            2 & 6 & 1 & 5 & 4 & 3
        \end{pmatrix}
        \begin{pmatrix}
            1 & 2 & 3 & 4 & 5 & 6 \\
            3 & 1 & 4 & 5 & 6 & 2
        \end{pmatrix} =
        \begin{pmatrix}
            1 & 2 & 3 & 4 & 5 & 6 \\
            2 & 6 & 1 & 5 & 4 & 3
        \end{pmatrix}
    \]
\end{proof}

In Exercises 6 through 9, compute the expressions shown for the permutations $\sigma$, $\tau$, and $\mu$, defined prior to Exercise 1.

\newpage
% section 4/exercise 6
\begin{exercise}
    $\sigma^{6}$
\end{exercise}

\begin{proof}
    $\sigma^{5}(1) = \sigma^{4}(3) = \sigma^{3}(4) = \sigma^{2}(5) = \sigma(6) = 1$.

    Therefore, $\sigma^{5} = \iota$ and $\sigma^{6} = \sigma$.
    \[
        \sigma^{6} = \sigma =
        \begin{pmatrix}
            1 & 2 & 3 & 4 & 5 & 6 \\
            3 & 1 & 4 & 5 & 6 & 2
        \end{pmatrix}
    \]
\end{proof}

\newpage
% section 4/exercise 7
\begin{exercise}
    $\mu^{2}$
\end{exercise}

\begin{proof}
    \[
        \mu^{2} =
        \begin{pmatrix}
            5 & 2 & 4 & 3 & 1 & 6 \\
            1 & 2 & 3 & 4 & 5 & 6
        \end{pmatrix}
        \begin{pmatrix}
            1 & 2 & 3 & 4 & 5 & 6 \\
            5 & 2 & 4 & 3 & 1 & 6
        \end{pmatrix} =
        \begin{pmatrix}
            1 & 2 & 3 & 4 & 5 & 6 \\
            1 & 2 & 3 & 4 & 5 & 6
        \end{pmatrix} =
        \iota
    \]
\end{proof}

\newpage
% section 4/exercise 8
\begin{exercise}
    $\sigma^{100}$
\end{exercise}

\begin{proof}
    According to Exercise 6, $\sigma^{5} = \iota$, so $\sigma^{100} = {(\sigma^{5})}^{20} = \iota^{20} = \iota$.
\end{proof}

\newpage
% section 4/exercise 9
\begin{exercise}
    $\mu^{100}$
\end{exercise}

\begin{proof}
    According to Exercise 7, $\mu^{2} = \iota$, so $\mu^{100} = {(\mu^{2})}^{50} = \iota^{50} = \iota$.
\end{proof}

\newpage
% section 4/exercise 10
\begin{exercise}
    Convert the permutations $\sigma$, $\tau$, and $\mu$, defined prior to Exercise 1 to disjoint cycle notation.
\end{exercise}

\begin{proof}
    \[
        \begin{split}
            \sigma & = (1, 3, 4, 5, 6, 2) \\
            \tau   & = (1, 2, 4, 3)(5, 6) \\
            \mu    & = (1, 5)(3, 4)
        \end{split}
    \]
\end{proof}

\newpage
% section 4/exercise 11
\begin{exercise}
    Convert the following permutations in $S_{8}$ from disjoint cycle notation to two-row notation.
    \begin{enumerate}[label={\textbf{\arabic*.}}]
        \item $(1, 4, 5)(2, 3)$
        \item $(1, 8, 5)(2, 6, 7, 3, 4)$
        \item $(1, 2, 3)(4, 5)(6, 7, 8)$
    \end{enumerate}
\end{exercise}

\begin{proof}
    \begin{enumerate}[label={\textbf{\arabic*.}}]
        \item $(1, 4, 5)(2, 3) = \begin{pmatrix}
                      1 & 2 & 3 & 4 & 5 & 6 & 7 & 8 \\
                      4 & 3 & 2 & 5 & 1 & 6 & 7 & 8
                  \end{pmatrix}$.
        \item $(1, 8, 5)(2, 6, 7, 3, 4) = \begin{pmatrix}
                      1 & 2 & 3 & 4 & 5 & 6 & 7 & 8 \\
                      8 & 6 & 4 & 2 & 1 & 7 & 3 & 5
                  \end{pmatrix}$.
        \item $(1, 2, 3)(4, 5)(6, 7, 8) = \begin{pmatrix}
                      1 & 2 & 3 & 4 & 5 & 6 & 7 & 8 \\
                      2 & 3 & 1 & 5 & 4 & 7 & 8 & 6
                  \end{pmatrix}$
    \end{enumerate}
\end{proof}

\newpage
% section 4/exercise 12
\begin{exercise}
    Compute the permutation products.
    \begin{enumerate}[label={\textbf{\alph*.}}]
        \item $(1, 5, 2, 4)(1, 5, 2, 3)$
        \item $(1, 5, 3)(1, 2, 3, 4, 5, 6){(1, 5, 3)}^{-1}$
        \item ${({(1, 6, 7, 2)}^{2}{(4, 5, 2, 6)}^{-1}(1, 7, 3))}^{-1}$
        \item $(1, 6)(1, 5)(1, 4)(1, 3)(1, 2)$
    \end{enumerate}
\end{exercise}

\begin{proof}
    \begin{enumerate}[label={\textbf{\alph*.}}]
        \item $(1, 5, 2, 4)(1, 5, 2, 3) = (1, 2, 3, 5, 4)$
        \item \begin{align*}
                    & (1, 5, 3)(1, 2, 3, 4, 5, 6){(1, 5, 3)}^{-1} \\
                  = & (1, 5, 3)(1, 2, 3, 4, 5, 6)(1, 3, 5)        \\
                  = & (1, 4, 3, 6, 5, 2)
              \end{align*}
        \item \begin{align*}
                    & {({(1, 6, 7, 2)}^{2}{(4, 5, 2, 6)}^{-1}(1, 7, 3))}^{-1} \\
                  = & (1, 3, 7)(4, 5, 2, 6){(1, 2, 7, 6)}^{2}                 \\
                  = & (1, 3, 7)(4, 5, 2, 6)(1, 7)(2, 6)                       \\
                  = & (2, 4, 5)(3, 7)
              \end{align*}
        \item \begin{align*}
                    & (1, 6)(1, 5)(1, 4)(1, 3)(1, 2) \\
                  = & (1, 2)(3, 4)(5, 6)
              \end{align*}
    \end{enumerate}
\end{proof}

\newpage
% section 4/exercise 13
\begin{exercise}
    Compute the following elements of $D_{12}$. Write your answer in standard form.
    \begin{enumerate}[label={\textbf{\alph*.}}]
        \item $\mu{\rho}^{2}\mu{\rho}^{8}$
        \item $\mu{\rho}^{10}\mu{\rho}^{-1}$
        \item $\rho\mu{\rho}^{-1}$
        \item ${(\mu{\rho}^{3}{\mu}^{-1}{\rho}^{-1})}^{-1}$
    \end{enumerate}
\end{exercise}

\begin{proof}
    \begin{enumerate}[label={\textbf{\alph*.}}]
        \item $\mu{\rho}^{2}\mu{\rho}^{8} = \mu{\rho}^{2}(\mu{\rho}^{8}) = \mu{\rho}^{2}({\rho}^{4}\mu) = \mu{\rho}^{6}\mu = {(\mu{\rho}^{6})}\mu = {({\rho}^{6}\mu)}\mu = \rho^{6}$
        \item $\mu{\rho}^{10}\mu{\rho}^{-1} = (\mu{\rho}^{10})\mu{\rho}^{-1} = ({\rho}^{2}\mu)\mu{\rho}^{11} = {\rho}^{2}(\mu\mu){\rho}^{-1} = \rho$
        \item $\rho\mu{\rho}^{-1} = (\rho\mu){\rho}^{-1} = \mu{\rho}^{11}{\rho}^{-1} = \mu{\rho}^{10}$
        \item ${(\mu{\rho}^{3}{\mu}^{-1}{\rho}^{-1})}^{-1} = \rho\mu\rho^{9}\mu = (\rho\mu)\rho^{9}\mu = (\mu\rho^{11})\rho^{9}\mu = \mu(\rho^{11}\rho^{9})\mu = \mu{\rho}^{8}\mu = ({\rho}^{4}\mu)\mu = \rho^{4}$
    \end{enumerate}
\end{proof}

\newpage
% section 4/exercise 14
\begin{exercise}
    Write the group table for $D_{3}$. Compare the group tables for $D_{3}$ and $S_{3}$. Are the groups isomorphic?
\end{exercise}

\begin{proof}
    Group table for $S_{3}$
    \[
        \begin{array}{c|cccccc}
                      & \iota     & (1, 2, 3) & (1, 3, 2) & (1, 2)    & (2, 3)    & (1, 3)    \\
            \hline
            \iota     & \iota     & (1, 2, 3) & (3, 1, 2) & (1, 2)    & (2, 3)    & (1, 3)    \\
            (1, 2, 3) & (1, 2, 3) & (1, 3, 2) & \iota     & (1, 3)    & (1, 2)    & (2, 3)    \\
            (1, 3, 2) & (1, 3, 2) & \iota     & (1, 2, 3) & (2, 3)    & (1, 3)    & (1, 2)    \\
            (1, 2)    & (1, 2)    & (2, 3)    & (1, 3)    & \iota     & (1, 2, 3) & (1, 3, 2) \\
            (2, 3)    & (2, 3)    & (1, 3)    & (1, 2)    & (1, 3, 2) & \iota     & (1, 2, 3) \\
            (1, 3)    & (1, 3)    & (1, 2)    & (2, 3)    & (1, 2, 3) & (1, 3, 2) & \iota
        \end{array}
    \]

    Group table for $D_{3}$
    \[
        \begin{array}{c|cccccc}
                        & \iota       & \rho        & \rho^{2}    & \mu         & \mu\rho     & \mu\rho^{2} \\
            \hline
            \iota       & \iota       & \rho        & \rho^{2}    & \mu         & \mu\rho     & \mu\rho^{2} \\
            \rho        & \rho        & \rho^{2}    & \iota       & \mu\rho^{2} & \mu         & \mu\rho     \\
            \rho^{2}    & \rho^{2}    & \iota       & \rho        & \mu\rho     & \mu\rho^{2} & \mu         \\
            \mu         & \mu         & \mu\rho     & \mu\rho^{2} & \iota       & \rho        & \rho^{2}    \\
            \mu\rho     & \mu\rho     & \mu\rho^{2} & \mu         & \rho^{2}    & \iota       & \rho        \\
            \mu\rho^{2} & \mu\rho^{2} & \mu         & \mu\rho     & \rho        & \rho^{2}    & \iota
        \end{array}
    \]

    $D_{3}$ and $S_{3}$ are isomorphic.
\end{proof}

Let $A$ be a set and let $\sigma\in S_{A}$. For a fixed $a\in A$, the set
\[
    \mathcal{O}_{a,\sigma} = \{ \sigma^{n}(a) \mid n\in\mathbb{Z} \}
\]

is the \textbf{orbit} of $a$ \textbf{under} $\sigma$. In Exercise 15 through 17, find the orbit of $1$ under the permutation defined prior to Exercise 1.

\newpage
% section 4/exercise 15
\begin{exercise}
    $\sigma$
\end{exercise}

\begin{proof}
    $\mathcal{O}_{1,\sigma} = \{ 1, 2, 3, 4, 5, 6 \}$
\end{proof}

\newpage
% section 4/exercise 16
\begin{exercise}
    $\tau$
\end{exercise}

\begin{proof}
    $\mathcal{O}_{1,\tau} = \{ 1, 2, 3, 4 \}$
\end{proof}

\newpage
% section 4/exercise 17
\begin{exercise}
    $\mu$
\end{exercise}

\begin{proof}
    $\mathcal{P}_{1,\mu} = \{ 1, 5 \}$
\end{proof}

\newpage
% section 4/exercise 18
\begin{exercise}
    Verify that $H = \{ \iota, \mu, \rho^{2}, \mu\rho^{2} \}\subseteq D_{4}$ is a group using the operation function composition.
\end{exercise}

\begin{proof}
    $\iota\circ\mu = \mu\circ\iota = \mu, \iota\circ\rho^{2} = \rho^{2}\circ\iota = \rho^{2}, \iota\circ\mu\rho^{2} = \mu\rho^{2}\circ\iota = \mu\rho^{2}$.

    $\rho^{2}\circ\rho^{2} = \mu\circ\mu = \mu\rho^{2}\circ\mu\rho^{2} = \iota$.

    $\mu\circ\rho^{2} = \mu\rho^{2}$, $\rho^{2}\circ\mu = \mu\rho^{2}$.

    $\mu\circ\mu\rho^{2} = \rho^{2}$, $\mu\rho^{2}\circ\mu = \mu\mu\rho^{2} = \rho^{2}$.

    $\rho^{2}\circ\mu\rho^{2} = \mu\rho^{2}\rho^{2} = \mu$.

    So, $H$ is closed under function composition, $H$ is associative, $H$ has an identity element ($\iota$), and each element has an inverse. Therefore, $H$ is a group.
\end{proof}

\newpage
% section 4/exercise 19
\begin{exercise}
    \begin{enumerate}[label={\textbf{\alph*.}}]
        \item Verify that the six matrices
              \[
                  \begin{bmatrix}
                      1 & 0 & 0 \\
                      0 & 1 & 0 \\
                      0 & 0 & 1
                  \end{bmatrix},
                  \begin{bmatrix}
                      0 & 1 & 0 \\
                      0 & 0 & 1 \\
                      1 & 0 & 0
                  \end{bmatrix},
                  \begin{bmatrix}
                      0 & 0 & 1 \\
                      1 & 0 & 0 \\
                      0 & 1 & 0
                  \end{bmatrix},
                  \begin{bmatrix}
                      1 & 0 & 0 \\
                      0 & 0 & 1 \\
                      0 & 1 & 0
                  \end{bmatrix},
                  \begin{bmatrix}
                      0 & 0 & 1 \\
                      0 & 1 & 0 \\
                      1 & 0 & 0
                  \end{bmatrix},
                  \begin{bmatrix}
                      0 & 1 & 0 \\
                      1 & 0 & 0 \\
                      0 & 0 & 1
                  \end{bmatrix}
              \]

              form a group under matrix multiplication.
        \item What group discussed in this section is isomorphic to this group of six matrices?
    \end{enumerate}
\end{exercise}

\begin{proof}
    \[
        \begin{split}
            \begin{bmatrix}
                1 & 0 & 0 \\
                0 & 1 & 0 \\
                0 & 0 & 1
            \end{bmatrix}
            \begin{bmatrix}
                1 \\
                2 \\
                3
            \end{bmatrix} =
            \begin{bmatrix}
                1 \\
                2 \\
                3
            \end{bmatrix},\quad
            \begin{bmatrix}
                0 & 1 & 0 \\
                0 & 0 & 1 \\
                1 & 0 & 0
            \end{bmatrix}
            \begin{bmatrix}
                1 \\
                2 \\
                3
            \end{bmatrix} =
            \begin{bmatrix}
                2 \\
                3 \\
                1
            \end{bmatrix},\quad
            \begin{bmatrix}
                0 & 0 & 1 \\
                1 & 0 & 0 \\
                0 & 1 & 0
            \end{bmatrix}
            \begin{bmatrix}
                1 \\
                2 \\
                3
            \end{bmatrix} =
            \begin{bmatrix}
                3 \\
                1 \\
                2
            \end{bmatrix}, \\
            \begin{bmatrix}
                1 & 0 & 0 \\
                0 & 0 & 1 \\
                0 & 1 & 0
            \end{bmatrix}
            \begin{bmatrix}
                1 \\
                2 \\
                3
            \end{bmatrix} =
            \begin{bmatrix}
                1 \\
                3 \\
                2
            \end{bmatrix},\quad
            \begin{bmatrix}
                0 & 0 & 1 \\
                0 & 1 & 0 \\
                1 & 0 & 0
            \end{bmatrix}
            \begin{bmatrix}
                1 \\
                2 \\
                3
            \end{bmatrix} =
            \begin{bmatrix}
                3 \\
                2 \\
                1
            \end{bmatrix},\quad
            \begin{bmatrix}
                0 & 1 & 0 \\
                1 & 0 & 0 \\
                0 & 0 & 1
            \end{bmatrix}
            \begin{bmatrix}
                1 \\
                2 \\
                3
            \end{bmatrix} =
            \begin{bmatrix}
                2 \\
                1 \\
                3
            \end{bmatrix}.
        \end{split}
    \]

    We can relabel the six matrices to six permutations $\iota, (1, 2, 3), (1, 3, 2), (2, 3), (1, 3), (1, 2)$. So the six matrices form a group under matrix multiplication. This group of six matrices is isomorphic to $S_{3}$.
\end{proof}

\newpage
% section 4/exercise 20
\begin{exercise}
    After working Exercise 18, write down eight matrices that form a group under matrix multiplication that is isomorphic to $D_{4}$.
\end{exercise}

\begin{proof}
    \[
        \begin{split}
            \begin{bmatrix}
                1 & 0 & 0 & 0 \\
                0 & 1 & 0 & 0 \\
                0 & 0 & 1 & 0 \\
                0 & 0 & 0 & 1
            \end{bmatrix}\leftrightarrow\iota,\quad
            \begin{bmatrix}
                0 & 1 & 0 & 0 \\
                0 & 0 & 1 & 0 \\
                0 & 0 & 0 & 1 \\
                1 & 0 & 0 & 0
            \end{bmatrix}\leftrightarrow\rho,\quad
            \begin{bmatrix}
                0 & 0 & 1 & 0 \\
                0 & 0 & 0 & 1 \\
                1 & 0 & 0 & 0 \\
                0 & 1 & 0 & 0
            \end{bmatrix}\leftrightarrow\rho^{2},\quad
            \begin{bmatrix}
                0 & 0 & 0 & 1 \\
                1 & 0 & 0 & 0 \\
                0 & 1 & 0 & 0 \\
                0 & 0 & 1 & 0
            \end{bmatrix}\leftrightarrow\rho^{3}, \\
            \begin{bmatrix}
                0 & 0 & 0 & 1 \\
                0 & 0 & 1 & 0 \\
                0 & 1 & 0 & 0 \\
                1 & 0 & 0 & 0
            \end{bmatrix}\leftrightarrow\mu,\quad
            \begin{bmatrix}
                1 & 0 & 0 & 0 \\
                0 & 0 & 0 & 1 \\
                0 & 0 & 1 & 0 \\
                0 & 1 & 0 & 0
            \end{bmatrix}\leftrightarrow\mu\rho,\quad
            \begin{bmatrix}
                0 & 1 & 0 & 0 \\
                1 & 0 & 0 & 0 \\
                0 & 0 & 0 & 1 \\
                0 & 0 & 1 & 0
            \end{bmatrix}\leftrightarrow\mu\rho^{2},\quad
            \begin{bmatrix}
                0 & 0 & 1 & 0 \\
                0 & 1 & 0 & 0 \\
                1 & 0 & 0 & 0 \\
                0 & 0 & 0 & 1
            \end{bmatrix}\leftrightarrow\mu\rho^{3}.
        \end{split}
    \]
\end{proof}

\subsection*{Concepts}

In Exercises 21 through 23, correct the definition of the italicized term without reference to the text, if correction is needed, so that it is in a form acceptable for publication.

\newpage
% section 4/exercise 21
\begin{exercise}
    The \textit{dihedral group $D_{n}$} is the set of all functions $\phi: \mathbb{Z}_{n} \to \mathbb{Z}_{n}$ such that the line segment between vertex $i$ and $j$ of $U_{n}$ is an edge of $P_{n}$ if and only if the line segment between vertices $\phi(i)$ and $\phi(j)$ in $U_{n}$ is an edge of $P_{n}$.
\end{exercise}

\begin{proof}
    No correction is needed.
\end{proof}

\newpage
% section 4/exercise 22
\begin{exercise}
    A \textit{permutation} of a set $S$ is a one-to-one map from $S$ to $S$.
\end{exercise}

\begin{proof}
    Correction:  A \textit{permutation} of a set $S$ is a one-to-one map from $S$ onto $S$.
\end{proof}

\newpage
% section 4/exercise 23
\begin{exercise}
    The \textit{order} of a group is the number of elements in the group.
\end{exercise}

\begin{proof}
    Correction: The \textit{order} of a group is the number of elements in the group or the cardinality of the group.
\end{proof}

In Exercises 24 through 28, determine whether the given function is a permutation of $\mathbb{R}$.

\newpage
% section 4/exercise 24
\begin{exercise}
    $f_{1}: \mathbb{R} \to \mathbb{R}$ defined by $f_{1}(x) = x + 1$
\end{exercise}

\begin{proof}
    $f_{1}$ is one-to-one, since $x\ne y$ implies $x+1\ne y+1$.

    $f_{1}$ is onto, since $f_{1}(x - 1) = x$.

    Hence $f_{1}$ is a permutation of $\mathbb{R}$.
\end{proof}

\newpage
% section 4/exercise 25
\begin{exercise}
    $f_{2}: \mathbb{R} \to \mathbb{R}$ defined by $f_{2}(x) = x^{2}$
\end{exercise}

\begin{proof}
    $f_{2}$ is not one-to-one, since $f_{2}(1) = f_{2}(-1) = 1$.

    Hence $f_{2}$ is not a permutation of $\mathbb{R}$.
\end{proof}

\newpage
% section 4/exercise 26
\begin{exercise}
    $f_{3}: \mathbb{R} \to \mathbb{R}$ defined by $f_{3}(x) = -x^{3}$
\end{exercise}

\begin{proof}
    $f_{3}$ is one-to-one, since $x\ne y$ implies $-x^{3} \ne -y^{3}$ (because $x^{3} - y^{3} = (x - y)(x^{2} + xy + y^{2})$).

    $f_{3}$ is onto, since $f_{1}(-\sqrt[3]{x}) = x$.

    Hence $f_{3}$ is a permutation of $\mathbb{R}$.
\end{proof}

\newpage
% section 4/exercise 27
\begin{exercise}
    $f_{4}: \mathbb{R} \to \mathbb{R}$ defined by $f_{4}(x) = e^{x}$
\end{exercise}

\begin{proof}
    $f_{4}$ is not onto, since there is no real number $x$ such that $e^{x} = 0$.

    Hence $f_{4}$ is not a permutation of $\mathbb{R}$.
\end{proof}

\newpage
% section 4/exercise 28
\begin{exercise}
    $f_{5}: \mathbb{R} \to \mathbb{R}$ defined by $f_{5}(x) = x^{3} - x^{2} - 2x$
\end{exercise}

\begin{proof}
    $f_{5}$ is not one-to-one, since $f_{5}(0) = f_{5}(-1) = 0$.

    Hence $f_{5}$ is not a permutation of $\mathbb{R}$.
\end{proof}

\newpage
% section 4/exercise 29
\begin{exercise}
    Determine whether each of the following is true or false.
    \begin{enumerate}[label={\textbf{\alph*.}}]
        \item Every permutation is a one-to-one function.
        \item Every function is a permutation if and only if it is one-to-one.
        \item Every function from a finite set onto itself must be one-to-one.
        \item Every subset of an abelian group $G$ that is also a group using the same operation as $G$ is abelian.
        \item The symmetric group $S_{10}$ has $10$ elements.
        \item If $\phi\in D_{n}$, then $\phi$ is a permutation on the set $\mathbb{Z}_{n}$.
        \item The group $D_{n}$ has exactly $n$ elements.
        \item $D_{3}$ is a subset of $D_{4}$.
    \end{enumerate}
\end{exercise}

\begin{proof}
    \begin{enumerate}[label={\textbf{\alph*.}}]
        \item True.
        \item False.
        \item False.
        \item False.
        \item False.
        \item True.
        \item False.
        \item False.
    \end{enumerate}
\end{proof}

\subsection*{Theory}

\newpage
% section 4/exercise 30
\begin{exercise}
    Let $n\geq 3$ and $k\in\mathbb{Z}_{n}$. Prove that in $D_{n}$, $\rho^{k}\mu = \mu\rho^{n-k}$.
\end{exercise}

\begin{proof}
    Let $x$ be an element of $\mathbb{Z}_{n}$.
    \[
        \begin{split}
            (\rho^{k}\mu)(x) = n - x + k \mod n = k - x\mod n, \\
            (\mu\rho^{n-k})(x) = n - (x + n - k) \mod n = k - x \mod n
        \end{split}
    \]

    Hence $\rho^{k}\mu = \mu\rho^{n-k}$.
\end{proof}

\newpage
% section 4/exercise 31
\begin{exercise}
    Show that $S_{n}$ is nonabelian group for $n\geq 3$.
\end{exercise}

\begin{proof}
    $\sigma = (1, 2)$ is a permutation in $S_{n}$, which swaps $1$ and $2$. $\tau = (1, 3)$ is a permutation in $S_{n}$, which swaps $1$ and $3$.
    \[
        \sigma\tau = (1, 3, 2) \ne (1, 2, 3) = \tau\sigma
    \]

    So $S_{n}$ is nonabelian group for $n\geq 3$.
\end{proof}

\newpage
% section 4/exercise 32
\begin{exercise}
    Strengthening Exercise 31, show that if $n\geq 3$, then the only element of $\sigma$ of $S_{n}$ satisfying $\sigma\gamma = \gamma\sigma$ for all $\gamma\in S_{n}$ is $\sigma = \iota$, the identity permutation.
\end{exercise}

\begin{proof}
    Let $\alpha\in S_{n}$ be the transposition that swaps $1$ and $2$, let $\beta\in S_{n}$ be the transposition that swaps $1$ and $k$ ($k\geq 3$).

    $(\sigma\alpha)(1) = \sigma(\alpha(1)) = \sigma(2), (\alpha\sigma)(1) = \alpha(\sigma(1))$, so $\alpha$ swaps $\sigma(1)$ and $\sigma(2)$. So either $\sigma(1) = 1, \sigma(2) = 2$ or $\sigma(1) = 2$, $\sigma(2) = 1$.

    $(\sigma\beta)(1) = \sigma(\beta(1)) = \sigma(k), (\beta\sigma)(1) = \beta(\sigma(1))$, so $\beta$ swaps $\sigma(1)$ and $\sigma(k)$. So either $\sigma(1) = 1, \sigma(k) = k$ or $\sigma(1) = k$, $\sigma(k) = 1$.

    Therefore, $\sigma(1) = 1, \sigma(2) = 2, \sigma(k) = k$ (other cases lead to contradiction). Thus $\sigma = \iota$.
\end{proof}

\newpage
% section 4/exercise 33
\begin{exercise}
    Orbits were defined before Exercise 15. Let $a, b\in A$ and $\sigma\in S_{A}$. Show that if $\mathcal{O}_{a,\sigma}$ and $\mathcal{O}_{b,\sigma}$ have an element in common, then $\mathcal{O}_{a,\sigma} = \mathcal{O}_{b,\sigma}$.
\end{exercise}

\begin{proof}
    Let $c$ be the common element of $\mathcal{O}_{a,\sigma}$ and $\mathcal{O}_{b,\sigma}$. According to the definition of orbits, there exists integers $p, q$ such that $c = \sigma^{p}(a)$ and $c = \sigma^{q}(b)$. Therefore, $\sigma^{p}(a) = \sigma^{q}(b)$, so $\sigma^{p-q}(a) = b$ and $\sigma^{q-p}(b) = a$.

    Let $x = \sigma^{n}(a)$ be an element of $\mathcal{O}_{a,\sigma}$, then $x = \sigma^{n}(\sigma^{q-p}(b)) = \sigma^{n+q-p}(b)$, so $x$ is also an element of $\mathcal{O}_{b,\sigma}$. Therefore, $\mathcal{O}_{a,\sigma} \subseteq \mathcal{O}_{b,\sigma}$.

    Let $y = \sigma^{m}(b)$ be an element of $\mathcal{O}_{b,\sigma}$, then $y = \sigma^{m}(\sigma^{p-q}(a)) = \sigma^{m+p-q}(a)$, so $y$ is also an element of $\mathcal{O}_{a,\sigma}$. Therefore, $\mathcal{P}_{b,\sigma} \subseteq \mathcal{P}_{a,\sigma}$.

    Thus $\mathcal{O}_{a,\sigma} = \mathcal{O}_{b,\sigma}$.
\end{proof}

\newpage
% section 4/exercise 34
\begin{exercise}
    (See the warning following Theorem 4.8.) Let $G$ be a group with binary operation $*$. Let $G'$ be the same set as $G$, and define a binary operation $*'$ on $G'$ by $x *' y = y * x$ for all $x, y\in G'$.
    \begin{enumerate}[label={\textbf{\alph*.}}]
        \item (Intuitive argument that $G'$ under $*'$ is a group.) Suppose the front wall of your classroom were made of transparent glass, and that all possible products $a * b = c$ and all possible instances $a * (b * c) = (a * b) * c$ of the associative property for $G$ under $*$ were written on the wall with a magic marker. What would a person see when looking at the other side of the wall from the next room in front of yours?
        \item Show from the mathematical definition of $*'$ that $G'$ is a group under $*'$.
    \end{enumerate}
\end{exercise}

\begin{proof}
    $G'$ is closed under $*'$ since $G$ is closed under $*$.

    For all $x, y, z\in G'$, $(x *' y) *' z = (y * x) *' z = z * (y * x) = (z * y) * x = (y *' z) * x = x *' (y' * z')$. So $*'$ is associative.

    Let $e$ be the identity element in $G$. $x *' e = e * x = x = x * e = e *' x$. So $*'$ has an identity element, the same as $G$.

    Let $x'$ be the inverse of $x$ in $G$, $x *' x' = x' * x = e = x * x' = x' *' x$. So each element of $G'$ has an inverse with $*'$, the same as $*$.

    Hence $G'$ is a group under $*'$.
\end{proof}

\newpage
% section 4/exercise 35
\begin{exercise}
    Give a careful proof using the definition of isomorphism that if $G$ and $G'$ are both groups with $G$ abelian and $G'$ not abelian, then $G$ and $G'$ are not isomorphic.
\end{exercise}

\begin{proof}
    Assume that $G$ and $G'$ are isomorphic. According to the definition of isomorphism, there exists a one-to-one function $\phi$ from $G$ onto $G'$ such that $\phi(x * y) = \phi(x) *' \phi(y)$ for all $x, y\in G$.

    Since $G'$ is not abelian, there exist two elements $a, b$ in $G'$ such that $a *' b \ne b *' a$. $\phi$ is onto $G'$ so there exist $x, y$ in $G$ such that $\phi(x) = a$ and $\phi(y) = b$. $a *' b = \phi(x) *' \phi(y) = \phi(x * y) = \phi(y * x) = \phi(y) *' \phi(x) = b *' a$, this contradicts $a *'b \ne b *' a$. So the initial assumption is false.

    Thus $G$ and $G'$ are not isomorphic.
\end{proof}

\newpage
% section 4/exercise 36
\begin{exercise}
    Prove that for any integer $n\geq 2$, there are at least two nonisomorphic groups with exactly $2n$ elements.
\end{exercise}

\begin{proof}
    $D_{n}$ and $\mathbb{Z}_{2n}$, each has $2n$ elements. But $D_{n}$ is not abelian, $\mathbb{Z}_{2n}$ is abelian. Thus, there are at least two nonisomorphic groups with exactly $2n$ elements.
\end{proof}

\newpage
% section 4/exercise 37
\begin{exercise}
    Let $n\geq 3$ and $0\leq k \leq n-1$. Prove that the map $\mu\rho^{k} \in D_{n}$ is the reflection about the line through the origin that makes an angle of $-\frac{\pi k}{n}$ with the $x$-axis.
\end{exercise}

\begin{proof}
    On the unit circle, $(\cos\frac{2m\pi}{n}, \sin\frac{2m\pi}{n})$ where $0\leq k \leq n$ are geometric representations of $n$-th roots of unity.

    The line through the origin that makes an angle of $-\frac{\pi k}{n}$ with the $x$-axis has equation
    \[
        \sin\frac{k\pi}{n} x + \cos\frac{k\pi}{n}y = 0
    \]

    Remind that the reflection of $(x_{0}, y_{0})$ about the line $ax + by + c = 0$ has coordinates
    \[
        \left(x_{0} - \frac{2a(ax_{0} + by_{0} + c)}{a^{2} + b^{2}}, y_{0} - \frac{2b(ax_{0} + by_{0} + c)}{a^{2} + b^{2}}\right)
    \]

    The reflection of $(\cos\frac{2m\pi}{n}, \sin\frac{2m\pi}{n})$ about this line has coordinates
    \begin{align*}
          & \left(\cos\frac{2m\pi}{n} - 2\sin\frac{k\pi}{n}\left(\sin\frac{k\pi}{n}\cos\frac{2m\pi}{n} + \cos\frac{k\pi}{n}\sin\frac{2m\pi}{n}\right), \sin\frac{2m\pi}{n} - 2\cos\frac{k\pi}{n}\left(\sin\frac{k\pi}{n}\cos\frac{2m\pi}{n} + \cos\frac{k\pi}{n}\sin\frac{2m\pi}{n}\right)\right) \\
        = & \left( \cos\frac{2m\pi}{n} - 2\sin\frac{k\pi}{n}\sin\frac{(2m+k)\pi}{n}, \sin\frac{2m\pi}{n} - 2\cos\frac{k\pi}{n}\sin\frac{(2m+k)\pi}{n} \right)                                                                                                                                     \\
        = & \left( \cos\frac{(2m+2k)\pi}{n}, \sin\frac{2m\pi}{n} - 2\cos\frac{-k\pi}{n}\sin\frac{(2m+k)\pi}{n} \right)                                                                                                                                                                            \\
        = & \left( \cos\frac{-(2m+2k)\pi}{n}, \sin\frac{-(2m+2k)\pi}{n} \right)
    \end{align*}

    Image of $(\cos\frac{2m\pi}{n}, \sin\frac{2m\pi}{n})$ under $\rho^{k}$ and $\mu$ is
    \[
        \mu\left(\rho^{k}\left(\left( \cos\frac{2m\pi}{n}, \frac{2m\pi}{n} \right)\right)\right) = \mu\left(\left( \cos\frac{(2m+2k)\pi}{n}, \sin\frac{(2m+2k)\pi}{n} \right)\right) = \left( \cos\frac{-(2m+2k)\pi}{n}, \sin\frac{-(2m+2k)\pi}{n} \right).
    \]

    Thus, $\mu\rho^{k} \in D_{n}$ is the reflection about the line through the origin and make an angle of $-\frac{\pi k}{n}$ with the $x$-axis.
\end{proof}

\newpage
% section 4/exercise 38
\begin{exercise}
    Let $n\geq 3$ and $k, r\in\mathbb{Z}_{n}$. Based on Exercise 37, determine the element of $D_{n}$ that corresponds to first reflecting across the line through the origin at an angle of $-\frac{2\pi k}{n}$ and then reflection across the line through the origin making an angle of $-\frac{2\pi r}{n}$. Prove your answer.
\end{exercise}

\begin{proof}
    According to Exercise 37
    \begin{itemize}
        \item the element that corresponds to the reflection about the line through the origin at an angle of $-\frac{2\pi k}{n}$ is $\mu\rho^{2k}$,
        \item the element that corresponds to the reflection about the line through the origin at an angle of $-\frac{2\pi r}{n}$ is $\mu\rho^{2r}$.
    \end{itemize}

    Their composition is $\mu\rho^{2r}\mu\rho^{2k}$. According to Exercise 32
    \[
        \mu\rho^{2r}\mu\rho^{2k} = \mu(\mu\rho^{n-2r})\rho^{2k} = \rho^{n-2(r-k)}.
    \]

    Hence the desired element is $\rho^{n-2(r-k)}$.
\end{proof}

\newpage
\section{Subgroups}

\subsection*{Computations}

In Exercises 1 through 6, determine whether the given subset of the complex numbers is a subgroup of the group $\mathbb{C}$ of complex numbers under addition.

\newpage
% section 5/exercise 1
\begin{exercise}
    $\mathbb{R}$
\end{exercise}

\begin{proof}
    $\mathbb{R}$ is a subset of $\mathbb{C}$.

    $\mathbb{R}$ is closed under addition. $\mathbb{R}$ contains the identity element $0$. For each element $x$ of $\mathbb{R}$, $-x$ is also in $\mathbb{R}$.

    Hence $\mathbb{R}$ is a subgroup of $\mathbb{C}$ under addition.
\end{proof}

\newpage
% section 5/exercise 2
\begin{exercise}
    $\mathbb{Q}^{+}$
\end{exercise}

\begin{proof}
    $\mathbb{Q}^{+}$ is a subset of $\mathbb{C}$.

    $\mathbb{Q}^{+}$ is closed under addition. But $\mathbb{Q}^{+}$ does not contain the identity element $0$.

    Hence $\mathbb{Q}^{+}$ is not a subgroup of $\mathbb{C}$ under addition.
\end{proof}

\newpage
% section 5/exercise 3
\begin{exercise}
    $7\mathbb{Z}$
\end{exercise}

\begin{proof}
    $7\mathbb{Z}$ is a subset of $\mathbb{C}$.

    $7\mathbb{Z}$ is closed under addition. $7\mathbb{Z}$ contains the identity element $0 = 0\cdot 7$. For each element $7n$ of $7\mathbb{Z}$ (where $n\mathbb{Z}$), $-7n$ is also in $\mathbb{C}$.

    Hence $7\mathbb{Z}$ is a subgroup of $\mathbb{C}$ under addition.
\end{proof}

\newpage
% section 5/exercise 4
\begin{exercise}
    $i\mathbb{R}$
\end{exercise}

\begin{proof}
    $i\mathbb{R}$ is a subset of $\mathbb{C}$.

    $i\mathbb{R}$ is closed under addition. $i\mathbb{R}$ contains the identity element $0 = 0i$. For each element $iy$ of $i\mathbb{R}$ (where $y\in\mathbb{R}$), $-iy$ is also in $\mathbb{C}$.

    Hence $i\mathbb{R}$ is a subgroup of $\mathbb{C}$ under addition.
\end{proof}

\newpage
% section 5/exercise 5
\begin{exercise}
    $\pi\mathbb{Q}$
\end{exercise}

\begin{proof}
    $\pi\mathbb{Q}$ is a subset of $\mathbb{C}$.

    $\pi\mathbb{Q}$ is closed under addition, since $\pi a + \pi b = \pi (a + b)$, where $a, b\in\mathbb{Q}$. $\pi\mathbb{Q}$ contains the additive identity $0 = 0\pi$. For each element $\pi q\in\mathbb{Q}$ (where $q\in\mathbb{Q}$), $-\pi q$ is also an element of $\pi\mathbb{Q}$.

    Hence $\pi\mathbb{Q}$ is a subgroup of $\mathbb{C}$ under addition.
\end{proof}

\newpage
% section 5/exercise 6
\begin{exercise}
    The set $\{ \pi^{n} \mid n\in\mathbb{Z} \}$
\end{exercise}

\begin{proof}
    Since $\pi$ is transcendental, $\pi + \pi$ is not an integer power of $\pi$, so this set is not closed under addition.

    Hence $\{ \pi^{n} \mid n\in\mathbb{Z} \}$ is not a subgroup of $\mathbb{C}$ under addition.
\end{proof}

\newpage
% section 5/exercise 7
\begin{exercise}
    Which of the sets in Exercises 1 through 6 are subgroups of the group $\mathbb{C}^{*}$ of nonzero complex numbers under multiplication?
\end{exercise}

\begin{proof}
    $\mathbb{Q}^{+}$ and $\{ \pi^{n} \mid n\in\mathbb{Z} \}$ are subgroups of $\mathbb{C}^{*}$ under multiplication.
\end{proof}

In Exercises 8 through 13, determine whether the given set of invertible $n\times n$ matrices with real number entries is a subgroup of $GL(n, \mathbb{R})$.

\newpage
% section 5/exercise 8
\begin{exercise}
    The $n\times n$ matrices with determinant greater than or equal to $1$
\end{exercise}

\begin{proof}
    If an $n\times n$ matrix $A$ has determinant greater than $1$, then $A$ is invertible and its inverse $A^{-1}$ has determinant smaller than $1$. So $A^{-1}$ is not in the given set.

    Hence the given set is not a subgroup of $GL(n, \mathbb{R})$.
\end{proof}

\newpage
% section 5/exercise 9
\begin{exercise}
    The diagonal $n\times n$ matrices with no zeros on the diagonal
\end{exercise}

\begin{proof}
    Product of two diagonal $n\times n$ matrices $A = \text{diag}(a_{1}, \ldots, a_{n})$ and $B = \text{diag}(b_{1}, \ldots, b_{n})$ is the matrix $AB = \text{diag}(a_{1}b_{1}, \ldots, a_{n}b_{n})$ whose entries on the diagonal are nonzero.

    The identity $n\times n$ matrix is in the given set.

    For each $A = \text{diag}(a_{1}, \ldots, a_{n})$ in the given set, its inverse $A^{-1} = \text{diag}({a_{1}}^{-1}, \ldots, {a_{n}}^{-1})$ is also in the given set.

    Hence the given set is a subgroup of $GL(n, \mathbb{R})$.
\end{proof}

\newpage
% section 5/exercise 10
\begin{exercise}
    The $n\times n$ matrices with determinant $2^{k}$ for some integer $k$
\end{exercise}

\begin{proof}
    If $A, B$ are in the given set, $\det(AB) = \det(A)\det(B)$ is an integer power of $2$. So the given set is closed under matrix multiplication.

    The determinant of the $n\times n$ identity matrix is $1 = 2^{0}$, so the given set contains the identity $n\times n$ matrix.

    If $A$ is in the given set, then $\det(A) = 2^{k}$ for some integer $k$. Therefore $A$ is invertible (since $\det(A)\ne 0$) and $\det(A^{-1}) = 2^{-k}$. Therefore, if $A$ is in the given set, then so does $A^{-1}$.

    Hence the given set is a subgroup of $GL(n, \mathbb{R})$.
\end{proof}

\newpage
% section 5/exercise 11
\begin{exercise}
    The $n\times n$ matrices with determinant $-1$
\end{exercise}

\begin{proof}
    If $A, B$ are in the given set, $\det(AB) = \det(A)\cdot\det(B) = (-1)\cdot (-1) = 1$. So the given set is not closed under matrix multiplication.

    Hence the given set is not a subgroup of $GL(n, \mathbb{R})$.
\end{proof}

\newpage
% section 5/exercise 12
\begin{exercise}
    The $n\times n$ matrices with determinant $-1$ or $1$
\end{exercise}

\begin{proof}
    If $A, B$ are in the given set, $\det(AB) = \det(A)\cdot\det(B)$ and this product is equal to either $1$ or $-1$. So the given set is closed under matrix multiplication.

    The identity $n\times n$ matrix has determinant $1$ so it is in the given set.

    If $A$ is in the given set, then $A$ is invertible. $\det(A) = 1$ if and only if $\det(A^{-1}) = 1$, $\det(A) = -1$ if and only if $\det(A^{-1}) = -1$. Therefore, if $A$ is in the given set, then so does the inverse of $A$.

    Hence the given set is a subgroup of $GL(n, \mathbb{R})$.
\end{proof}

\newpage
% section 5/exercise 13

\begin{exercise}
    The set of all $n\times n$ matrices $A$ such that $(A^{T})A = I_{n}$.
\end{exercise}

\begin{proof}
    If $A, B$ are in the given set, ${(AB)}^{T}AB = B^{T}A^{T}AB = B^{T}(A^{T}A)B = B^{T}B = I_{n}$. So the given set is closed under matrix multiplication.

    The identity $n\times n$ matrix is in the given set because ${I_{n}}^{T}I_{n} = I_{n}I_{n} = I_{n}$.

    If $A$ is in the given set, ${(\det(A))}^{2} = 1$, so $\det(A) = \pm 1$, which implies $A$ is invertible. Since $A^{T}A = I_{n}$, we obtain that $A^{T}$ is also the inverse of $A$, in other words, $A^{T}A = AA^{T} = I_{n}$ and $A^{T} = A^{-1}$. So if $A$ is in the given set, then so does $A^{-1}$.

    Hence the given set is a subgroup of $GL(n, \mathbb{R})$.
\end{proof}

Let $F$ be the set of all real-valued functions with domain $\mathbb{R}$ and let $\tilde{F}$ be the subset of $F$ consisting of those functions that have a nonzero value at every point in $\mathbb{R}$. In Exercises 14 through 19, determine whether the given subset of $F$ with the induced operation is (a) a subgroup of the group $F$ under addition, (b) a subgroup of the group $\tilde{F}$ under multiplication.

\newpage
% section 5/exercise 14
\begin{exercise}
    The subset $\tilde{F}$
\end{exercise}

\begin{proof}
    \begin{enumerate}[label={(\alph*)}]
        \item Addition.

              $\tilde{F}$ does not contain the additive identity (the constant function $0$) so $\tilde{F}$ is not a subgroup of $F$ under addition.
        \item Multiplication.

              $\tilde{F}$ is the improper subgroup of $\tilde{F}$ under mulitplication.
    \end{enumerate}
\end{proof}

\newpage
% section 5/exercise 15
\begin{exercise}
    The subset of all $f\in F$ such that $f(1) = 0$
\end{exercise}

\begin{proof}
    \begin{enumerate}[label={(\alph*)}]
        \item Addition.

              If $f, g$ is in the given subset, then $(f + g)(1) = f(1) + g(1) = 0 + 0 = 0$. So the given subset is closed under addition.

              The constant function $0$ is in the give subset, which is also the additive identity.

              If $f$ is in the given subset, then $g = -f$ satisfies $g(1) = -f(1) = 0$.

              Hence the given subset is a subgroup of $F$ under addition.
        \item Multiplication.

              The multiplicative identity, which is the constant function $1$ is not in the given subset.

              Hence the given subset is not a subgroup of $\tilde{F}$ under multiplication.
    \end{enumerate}
\end{proof}

\newpage
% section 5/exercise 16
\begin{exercise}
    The subset of all $f\in\tilde{F}$ such that $f(1) = 1$
\end{exercise}

\begin{proof}
    \begin{enumerate}[label={(\alph*)}]
        \item Addition.

              If $f, g$ is in the given subset, then $(f + g)(1) = f(1) + g(1) = 1 + 1 = 2\ne 1$. So the given subset is not closed under addition.

              Hence the given subset is not a subgroup of $F$ under addition.
        \item Multiplication.

              If $f, g$ is in the given subset, then $(f\cdot g)(1) = f(1)\cdot g(1) = 1$. On the other hand, $(f\cdot g)(x) = f(x)g(x) \ne 0$ for all $x\in\mathbb{R}$. So the given subset is closed under multiplication.

              The multiplicative identity, which is the constant function $1$ is in the given subset.

              If $f$ is in the given subset, then $g$ defined as $g(x) = 1/f(x)$ satisfies $g(1) = 1/f(1) = 1$. So $g$ (the multiplicative inverse of $f$) is also in the given subset.

              Hence the give subset is a subgroup of $\tilde{F}$ under multiplication.
    \end{enumerate}
\end{proof}

\newpage
% section 5/exercise 17
\begin{exercise}
    The subset of all $f\in\tilde{F}$ such that $f(0) = 1$
\end{exercise}

\begin{proof}
    \begin{enumerate}[label={(\alph*)}]
        \item Addition.

              If $f, g$ is in the given subset, then $(f + g)(0) = f(0) + g(0) = 1 + 1 = 2\ne 1$. So the given subset is not closed under addition.

              Hence the given subset is not a subgroup of $F$ under addition.
        \item Multiplication.

              If $f, g$ is in the given subset, then $(f\cdot g)(0) = f(0)\cdot g(0) = 1$. On the other hand, $(f\cdot g)(x) \ne 0$ for all $x\in\mathbb{R}$. So the given subset is closed under multiplication.

              The multiplicative identity, which is the constant function $1$ is in the given subset.

              If $f$ is in the given subset, then $g$ defined as $g(x) = 1/f(x)$ satisfies $g(0) = 1/f(0) = 1$. So $g$ (the multiplicative inverse of $f$) is also in the given subset.

              Hence the given subset is a subgroup of $\tilde{F}$ under multiplication.
    \end{enumerate}
\end{proof}

\newpage
% section 5/exercise 18
\begin{exercise}
    The subset of all $f\in\tilde{F}$ such that $f(0) = -1$
\end{exercise}

\begin{proof}
    \begin{enumerate}[label={(\alph*)}]
        \item Addition.

              The additive identity, which is the constant function $0$ is not in the given subset. Hence the given subset is not a subgroup of $F$ under addition.
        \item Multiplication.

              The multiplicative identity, which is the constant function $1$ is not in the given subset. Hence the give subset is not a subgroup of $\tilde{F}$ under multiplication.
    \end{enumerate}
\end{proof}

\newpage
% section 5/exercise 19
\begin{exercise}
    The subset of all constant functions in $F$.
\end{exercise}

\begin{proof}
    \begin{enumerate}[label={(\alph*)}]
        \item Addition.

              If $f, g$ are constant functions, then so does $f + g$. So the given subset is closed under addition.

              The additive identity, which is the constant function $0$ is in the given subset.

              If $f$ is a constant function, then so does $-f$.

              Hence the given subset is a subgroup of $F$ under addition.
        \item Multiplication.

              The given subset contains the constant function $0$, which is not in $\tilde{F}$. Hence the given subset is not a subgroup of $\tilde{F}$ under multiplication.
    \end{enumerate}
\end{proof}

\newpage
% section 5/exercise 20
\begin{exercise}
    Nine groups are given below. Give a \textit{complete} list of all subgroup relations, of the form $G_{i} \leq G_{j}$, that exist between these given group $G_{1}, G_{2}, \ldots, G_{9}$. \\
    $G_{1} = \mathbb{Z}$ under addition \\
    $G_{2} = 12\mathbb{Z}$ under addition \\
    $G_{3} = \mathbb{Q}^{+}$ under multiplication \\
    $G_{4} = \mathbb{R}$ under addition \\
    $G_{5} = \mathbb{R}^{+}$ under multiplication \\
    $G_{6} = \{ \pi^{n} \mid n\in\mathbb{Z} \}$ under multiplication \\
    $G_{7} = 3\mathbb{Z}$ under addition \\
    $G_{8} =$ the set of all integral multiple of $6$ under addition \\
    $G_{9} = \{ 6^{n} \mid n\in\mathbb{Z} \}$ under multiplication
\end{exercise}

\begin{proof}
    \begin{enumerate}
        \item $G_{1} \leq G_{1}$, $G_{1} \leq G_{4}$
        \item $G_{2} \leq G_{1}$, $G_{2} \leq G_{2}$, $G_{2} \leq G_{4}$, $G_{2} \leq G_{7}$, $G_{2} \leq G_{8}$
        \item $G_{3} \leq G_{3}$, $G_{3} \leq G_{5}$
        \item $G_{4} \leq G_{4}$
        \item $G_{5} \leq G_{5}$
        \item $G_{6} \leq G_{5}$, $G_{6} \leq G_{6}$
        \item $G_{7} \leq G_{1}$, $G_{7} \leq G_{4}$, $G_{7} \leq G_{7}$
        \item $G_{8} \leq G_{1}$, $G_{8} \leq G_{4}$, $G_{8} \leq G_{7}$, $G_{8} \leq G_{8}$
        \item $G_{9} \leq G_{3}$, $G_{9} \leq G_{5}$, $G_{9} \leq G_{9}$
    \end{enumerate}
\end{proof}

\newpage
% section 5/exercise 21
\begin{exercise}
    Write at least $5$ elements of each of the following cyclic groups.
    \begin{enumerate}[label={\textbf{\alph*.}}]
        \item $25\mathbb{Z}$ under addition
        \item $\left\{ {\left(\frac{1}{2}\right)}^{n} \mid n\in\mathbb{Z} \right\}$ under multiplication
        \item $\{ \pi^{m} \mid n\in\mathbb{Z} \}$ under multiplication
        \item $\anglebracket{\rho^{3}}$ in the group $D_{18}$
        \item $\anglebracket{(1, 2, 3)(5, 6)}$ in the group $S_{6}$
    \end{enumerate}
\end{exercise}

\begin{proof}
    \begin{enumerate}[label={\textbf{\alph*}}]
        \item $\{ \ldots, -50, -25, 0, 25, 50, \ldots \}$
        \item $\{ \ldots, 4, 2, 1, \frac{1}{2}, \frac{1}{4}, \ldots \}$
        \item $\{ \ldots, \frac{1}{\pi^{2}}, \frac{1}{\pi}, 1, \pi, \pi^{2}, \ldots \}$
        \item $\{ \iota, \rho^{3}, \rho^{6}, \rho^{9}, \rho^{12}, \rho^{15} \}$
        \item $(1, 2, 3)(5, 6)$, $(1, 3, 2)$, $(5, 6)$, $(1, 2, 3)$, $(1, 3, 2)(5, 6)$, $\iota$
    \end{enumerate}
\end{proof}

In Exercises 22 through 25, describe all the elements in the cyclic subgroup of $GL(2, \mathbb{R})$ generated by the given $2\times 2$ matrix.

\newpage
% section 5/exercise 22
\begin{exercise}
    $\begin{bmatrix}
            0  & -1 \\
            -1 & 0
        \end{bmatrix}$
\end{exercise}

\begin{proof}
    \[
        \anglebracket{\begin{bmatrix}
                0  & -1 \\
                -1 & 0
            \end{bmatrix}} = \left\{
        \begin{bmatrix}
            1 & 0 \\
            0 & 1
        \end{bmatrix},
        \begin{bmatrix}
            0  & -1 \\
            -1 & 0
        \end{bmatrix}
        \right\}
    \]
\end{proof}

\newpage
% section 5/exercise 23
\begin{exercise}
    $\begin{bmatrix}
            1 & 1 \\
            0 & 1
        \end{bmatrix}$
\end{exercise}

\begin{proof}
    \[
        \anglebracket{\begin{bmatrix}
                1 & 1 \\
                0 & 1
            \end{bmatrix}} = \left\{
        \begin{bmatrix}
            1 & n \\
            0 & 1
        \end{bmatrix} \mid n\in\mathbb{Z}
        \right\}
    \]
\end{proof}

\newpage
% section 5/exercise 24
\begin{exercise}
    $\begin{bmatrix}
            3 & 0 \\
            0 & 2
        \end{bmatrix}$
\end{exercise}

\begin{proof}
    \[
        \anglebracket{\begin{bmatrix}
                3 & 0 \\
                0 & 2
            \end{bmatrix}} = \left\{
        \begin{bmatrix}
            3^{n} & 0     \\
            0     & 2^{n}
        \end{bmatrix} \mid n\in\mathbb{Z}
        \right\}
    \]
\end{proof}

\newpage
% section 5/exercise 25
\begin{exercise}
    $\begin{bmatrix}
            0  & -2 \\
            -2 & 0
        \end{bmatrix}$
\end{exercise}

\begin{proof}
    \[
        \anglebracket{\begin{bmatrix}
                0  & -2 \\
                -2 & 0
            \end{bmatrix}} = \left\{
        \begin{bmatrix}
            2^{2n} & 0      \\
            0      & 2^{2n}
        \end{bmatrix},
        \begin{bmatrix}
            0         & -2^{2n+1} \\
            -2^{2n+1} & 0
        \end{bmatrix} \mid n\in\mathbb{Z}
        \right\}
    \]
\end{proof}

\newpage
% section 5/exercise 26
\begin{exercise}
    Which of the following groups are cyclic? For each cyclic, list all the generators of the group
    \begin{align*}
         & G_{1} = \anglebracket{\mathbb{Z}, +} \quad G_{2} = \anglebracket{\mathbb{Q}, +} \quad G_{3} = \anglebracket{\mathbb{Q}^{+}, \cdot} \quad G_{4} = \anglebracket{6\mathbb{Z}, +} \\
         & G_{5} = \{ 6^{n} \mid n\in\mathbb{Z} \}\ \text{under multiplication}                                                                                                           \\
         & G_{6} = \{ a = b\sqrt{2} \mid a, b\in\mathbb{Z} \}\ \text{under addition}
    \end{align*}
\end{exercise}

\begin{proof}
    \begin{itemize}
        \item $G_{1}$ is a cyclic group. $G_{1} = \anglebracket{1} = \anglebracket{-1}$
        \item $G_{2}$ is not a cyclic group.
        \item $G_{3}$ is not a cyclic group.
        \item $G_{4}$ is a cyclic group. $G_{4} = \anglebracket{6} = \anglebracket{-6}$
        \item $G_{5}$ is a cyclic group. $G_{5} = \anglebracket{6} = \anglebracket{\frac{1}{6}}$
        \item $G_{6}$ is not a cyclic group.
    \end{itemize}
\end{proof}

In Exercises 27 through 35, find the order of the cyclic subgroup of the given group generated by the indicated element.

\newpage
% section 5/exercise 27
\begin{exercise}
    The subgroup of $\mathbb{Z}_{4}$ generated by $3$
\end{exercise}

\begin{proof}
    $\anglebracket{3} = \{ 3, 2, 1, 0 \} = \mathbb{Z}_{4}$. The order of this subgroup is $4$.
\end{proof}

\newpage
% section 5/exercise 28
\begin{exercise}
    The subgroup of $V$ generated by $c$ (see Table 5.9)
    \begin{tabular}{c|cccc}
          & e & a & b & c \\
        \hline
        e & e & a & b & c \\
        a & a & e & c & b \\
        b & b & c & e & a \\
        c & c & b & a & e
    \end{tabular}
\end{exercise}

\begin{proof}
    $\anglebracket{c} = \{ c, e \}$. The order of this subgroup is $2$.
\end{proof}

\newpage
% section 5/exercise 29
\begin{exercise}
    The subgroup of $U_{6}$ generated by $\cos\frac{2\pi}{3} + i\sin\frac{2\pi}{3}$
\end{exercise}

\begin{proof}
    $\anglebracket{e^{i(2\pi/3)}} = \{ e^{i(2\pi/3)}, e^{i(4\pi/3)}, 1 \} = U_{3}$. The order of this subgroup is $3$.
\end{proof}

\newpage
% section 5/exercise 30
\begin{exercise}
    The subgroup of $\mathbb{Z}_{10}$ generated by $8$
\end{exercise}

\begin{proof}
    $\anglebracket{8} = \{ 8, 6, 4, 2, 0 \}$. The order of this subgroup is $5$.
\end{proof}

\newpage
% section 5/exercise 31
\begin{exercise}
    The subgroup of $\mathbb{Z}_{16}$ generated by $12$
\end{exercise}

\begin{proof}
    $\anglebracket{12} = \{ 12, 8, 4, 0 \}$. The order of this subgroup is $4$.
\end{proof}

\newpage
% section 5/exercise 32
\begin{exercise}
    The subgroup of the symmetric group $S_{8}$ generated by $(2, 4, 6, 9)(3, 5, 7)$
\end{exercise}

\begin{proof}
    \begin{multline*}
        \anglebracket{(2, 4, 6, 9)(3, 5, 7)} = \{ (2, 4, 6, 9)(3, 5, 7), (2, 6)(4, 9)(3, 7, 5), (2, 9, 6, 4), (3, 5, 7), \\
        (2, 4, 6, 9)(3, 7, 5), (2, 6)(4, 9), (2, 9, 6, 4)(3, 5, 7), (3, 7, 5), \\
        (2, 4, 6, 9), (2, 6)(4, 9)(3, 5, 7), (2, 9, 6, 4)(3, 7, 5), \iota \}
    \end{multline*}

    The order of this subgroup is $12$.
\end{proof}

\newpage
% section 5/exercise 33
\begin{exercise}
    The subgroup of the symmetric group $S_{10}$ generated by $(1, 10)(2, 9)(3, 8)(4, 7)(5, 6)$
\end{exercise}

\begin{proof}
    $\anglebracket{(1, 10)(2, 9)(3, 8)(4, 7)(5, 6)} = \{ (1, 10)(2, 9)(3, 8)(4, 7)(5, 6), \iota \}$. The order of this subgroup is $2$.
\end{proof}

\newpage
% section 5/exercise 34
\begin{exercise}
    The subgroup of the multiplicative group $G$ of invertible $4\times 4$ generated by
    \[
        \begin{bmatrix}
            0 & 0 & 0 & 1 \\
            0 & 0 & 1 & 0 \\
            1 & 0 & 0 & 0 \\
            0 & 1 & 0 & 0
        \end{bmatrix}
    \]
\end{exercise}

\begin{proof}
    \[
        {\begin{bmatrix}
                    0 & 0 & 0 & 1 \\
                    0 & 0 & 1 & 0 \\
                    1 & 0 & 0 & 0 \\
                    0 & 1 & 0 & 0
                \end{bmatrix}}^{4} =
        \begin{bmatrix}
            1 & 0 & 0 & 0 \\
            0 & 1 & 0 & 0 \\
            0 & 0 & 1 & 0 \\
            0 & 0 & 0 & 1
        \end{bmatrix} = I_{4}
    \]

    The order of this subgroup is $4$.
\end{proof}

\newpage
% section 5/exercise 35
\begin{exercise}
    The subgroup of the multiplicative group $G$ of invertible $4\times 4$ generated by
    \[
        \begin{bmatrix}
            0 & 1 & 0 & 0 \\
            0 & 0 & 0 & 1 \\
            0 & 0 & 1 & 0 \\
            1 & 0 & 0 & 0
        \end{bmatrix}
    \]
\end{exercise}

\begin{proof}
    \[
        {\begin{bmatrix}
                    0 & 1 & 0 & 0 \\
                    0 & 0 & 0 & 1 \\
                    0 & 0 & 1 & 0 \\
                    1 & 0 & 0 & 0
                \end{bmatrix}}^{3} =
        \begin{bmatrix}
            1 & 0 & 0 & 0 \\
            0 & 1 & 0 & 0 \\
            0 & 0 & 1 & 0 \\
            0 & 0 & 0 & 1
        \end{bmatrix} = I_{4}
    \]

    The order of this subgroup is $3$.
\end{proof}

\newpage
% section 5/exercise 36
\begin{exercise}
    \begin{enumerate}[label={\textbf{\alph*.}}]
        \item Complete the group table of the group $\mathbb{Z}_{6}$.
        \item Compute the subgroups $\anglebracket{1}, \anglebracket{2}, \anglebracket{3}, \anglebracket{4}$, and $\anglebracket{5}$ of the group $\mathbb{Z}_{6}$.
        \item Which elements are generators for the group $\mathbb{Z}_{6}$?
        \item Give the subgroup diagram for part (b) subgroups of $\mathbb{Z}_{6}$
    \end{enumerate}
\end{exercise}

\begin{proof}
    \begin{enumerate}[label={\textbf{\alph*.}}]
        \item
              \begin{tabular}{c|cccccc}
                  + & 0 & 1 & 2 & 3 & 4 & 5 \\
                  0 & 0 & 1 & 2 & 3 & 4 & 5 \\
                  1 & 1 & 2 & 3 & 4 & 5 & 0 \\
                  2 & 2 & 3 & 4 & 5 & 0 & 1 \\
                  3 & 3 & 4 & 5 & 0 & 1 & 2 \\
                  4 & 4 & 5 & 0 & 1 & 2 & 3 \\
                  5 & 5 & 0 & 1 & 2 & 3 & 4
              \end{tabular}
        \item \begin{align*}
                  \anglebracket{0} & = \{ 0 \}                \\
                  \anglebracket{1} & = \{ 0, 1, 2, 3, 4, 5 \} \\
                  \anglebracket{2} & = \{ 0, 2, 4 \}          \\
                  \anglebracket{3} & = \{ 0, 3 \}             \\
                  \anglebracket{4} & = \{ 0, 2, 4 \}          \\
                  \anglebracket{5} & = \{ 0, 1, 2, 3, 4, 5 \}
              \end{align*}
        \item $1$ and $5$ are generators for $\mathbb{Z}_{6}$.
        \item
              \begin{center}
                  \begin{tikzpicture}
                      \node (top) at (0, 0) {$\{ 0, 1, 2, 3, 4, 5 \}$};
                      \node [xshift=-40pt, yshift=-40pt, at=(top)] (left) {$\{ 0, 2, 4 \}$};
                      \node [xshift=40pt, yshift=-40pt, at=(top)] (right) {$\{ 0, 3 \}$};
                      \node [below=60pt of top] (bottom) {$\{ 0 \}$};
                      \draw (top) -- (left);
                      \draw (top) -- (right);
                      \draw (top) -- (bottom);
                      \draw (left) -- (bottom);
                      \draw (right) -- (bottom);
                  \end{tikzpicture}
              \end{center}
    \end{enumerate}
\end{proof}

\subsection*{Concepts}

In Exercises 37 and 38, correct the definition of the italicized term without reference to the text, if correction is needed, so that it is in a form acceptable for publication.

\newpage
% section 5/exercise 37
\begin{exercise}
    A \textit{subgroup} of a group $G$ is a subset $H$ of $G$ that contains the identity element $e$ of $G$ and also contains the inverse of each of its elements.
\end{exercise}

\begin{proof}
    Correction: A \textit{subgroup} of a group $G$ is a subset $H$ of $G$ that is closed under the operation in $G$, contains the identity element $e$ of $G$, the inverse of each of its elements.
\end{proof}

\newpage
% section 5/exercise 38
\begin{exercise}
    A group $G$ is cyclic if and only if there exists $a\in G$ such that $G = \{ a^{n} \mid n\in\mathbb{Z} \}$.
\end{exercise}

\begin{proof}
    The definition is correct.
\end{proof}

\newpage
% section 5/exercise 39
\begin{exercise}
    Determine whether each of the following is true or false.
    \begin{enumerate}[label={\textbf{\alph*}}]
        \item The associative law holds in every group.
        \item There may be a group in which the cancellation law fails.
        \item Every group is a subgroup of itself.
        \item Every group has exactly two improper subgroups.
        \item In every cyclic group, every element is a generator.
        \item A cyclic group has a unique generator.
        \item Every set of numbers that is a group under addition is also a group under multiplication.
        \item A subgroup may be defined as a subset of a group.
        \item $\mathbb{Z}_{4}$ is a cyclic group.
        \item Every subset of every group is a subgroup under the induced operation.
        \item For any $n \geq 3$, the dihedral group $D_{n}$ has at least $n + 2$ cyclic subgroups.
    \end{enumerate}
\end{exercise}

\begin{proof}
    \begin{enumerate}[label={\textbf{\alph*}}]
        \item True.
        \item False.
        \item True.
        \item False.
        \item False.
        \item False.
        \item False.
        \item False.
        \item True.
        \item False.
        \item True. $(n+2)$ subgroups are $\anglebracket{\iota}, \anglebracket{\rho}, \anglebracket{\mu\rho^{k}}$, where $0\leq k \leq n-1$.
    \end{enumerate}
\end{proof}

\newpage
% section 5/exercise 40
\begin{exercise}
    Show by means of an example that it is possible for the quadratic equation $x^{2} = e$ to have more than two solutions in some group $G$ with identity $e$.
\end{exercise}

\begin{proof}
    In Klein-4 group $V = \{ e, a, b, c \}$, $e^{2} = a^{2} = b^{2} = c^{2} = e$.

    In Dihedral group $D_{n}$ where $n\geq 3$, $\iota^{2} = {\mu}^{2} = {\mu\rho^{k}}^{2} = \iota$.
\end{proof}

In Exercises 41 through 44 let $B$ be a subset of $A$, and let $b$ be a particular element of $B$. Determine whether the given set is a subgroup of the symmetric group $S_{A}$ under the induced operation. Here $\sigma[B] = \{ \sigma(x) \mid x\in B \}$.

\newpage
% section 5/exercise 41
\begin{exercise}
    $\{ \sigma\in S_{A} \mid \sigma(b) = b \}$
\end{exercise}

\begin{proof}
    If $\sigma, \tau$ are in the given subset, then $(\sigma\tau)(b) = \sigma(\tau(b)) = \sigma(b) = b$. So the given subset is closed under composition.

    The identity mapping is an element of the given subset, because $\iota(b) = b$.

    If $\sigma$ is in the given subset, $\sigma^{-1}(b) = \sigma^{-1}(\sigma(b)) = (\sigma^{-1}\sigma)(b) = \iota(b) = b$. So the given subset contains the inverse of each element.

    Hence the given subset is a subgroup of $S_{A}$.
\end{proof}

\newpage
% section 5/exercise 42
\begin{exercise}
    $\{ \sigma \in S_{A} \mid \sigma(b) \in B \}$
\end{exercise}

\begin{proof}
    This subset is not necessarily a subgroup of $S_{A}$.

    For example, choose $B = \{ 1, 2 \} \subset A = \{ 1, 2, 3 \}$, $b = 1$, then $(1, 2, 3)$ is in $B$. But $(1, 2, 3)(1, 2, 3) = (1, 3, 2)$, which maps $1$ to $3$ and $3$ is not in $B$.
\end{proof}

\newpage
% section 5/exercise 43
\begin{exercise}
    $\{ \sigma \in S_{A} \mid \sigma[B] \subseteq B \}$
\end{exercise}

\begin{proof}
    When $A$ is finite, $\sigma[B] \subseteq B$ implies $\sigma[B] = B$. See Exercise 44 (the next one).

    This subset is not necessarily a subgroup of $S_{A}$ when $A$ is infinite.

    For example, let $A = \mathbb{Z}$, $B = \mathbb{N}$, and $\sigma: x \mapsto x + 1$. $\sigma[B] = \{ 2, 3, 4, \ldots \}$ is a proper subset of $\mathbb{N}$, but $\sigma^{-1}[B] = \{ 0, 1, 2, \ldots \}$ is a proper superset of $\mathbb{N}$.
\end{proof}

\newpage
% section 5/exercise 44
\begin{exercise}
    $\{ \sigma \in S_{A} \mid \sigma[B] = B \}$
\end{exercise}

\begin{proof}
    If $\sigma, \tau$ are in the given subset, $(\sigma\tau)[B] = \sigma[\tau[B]] = \sigma[B] = B$. So the given subset is closed under composition.

    The identity mapping is an element of the given subset.

    If $\sigma$ is in the given subset, then $\sigma\vert_{B}: B \to B$ is a bijection. Therefore, $\sigma^{-1}\vert_{B}: B \to B$, so $\sigma^{-1}$ is also an element of the given subset.

    Hence the given subset is a subgroup of $S_{A}$.
\end{proof}

\subsection*{Theory}

In Exercises 45 and 46, let $\phi: G \to G'$ be an isomorphism of a group $\anglebracket{G, *}$ with a group $\anglebracket{G', *'}$. Write out a proof to convince a skeptic of the intuitively clear statement.

\newpage
% section 5/exercise 45
\begin{exercise}
    If $H$ is a subgroup of $G$, then $\phi[H] = \{ \phi(h) \mid h\in H \}$ is a subgroup of $G'$. That is, an isomorphism carries subgroups into subgroups.
\end{exercise}

\begin{proof}
    $\phi[H]$ is a subset of $G'$.

    If $x', y'$ are in $\phi[H]$, then there exist $x, y\in H$ such that $\phi(x) = x', \phi(y) = y'$. $x' *' y' = \phi(x) *' \phi(y) = \phi(x * y)$. Because $x * y \in H$, we obtain that $\phi[x * y]$ is in $\phi[H]$. So $\phi[H]$ is closed under the operation $*'$.

    A group isomorphism maps the identity element of $G$ to the identity element of $G'$. So $\phi[H]$ contains the identity element of $G'$ with the operation $*'$.

    If $x'$ is in $\phi[X]$, then $\phi(x^{-1}) *' \phi(x) = \phi(x^{-1}x) = \phi(e) = \phi(xx^{-1}) = \phi(x) *' \phi(x^{-1})$. So $\phi(x^{-1})$ is the inverse of $x' = \phi(x)$, which means $\phi[X]$ contains the inverse of each element.

    Hence $\phi[H]$ is a subgroup of $G'$.
\end{proof}

\newpage
% section 5/exercise 46
\begin{exercise}
    If there is an $a\in G$ such that $\anglebracket{a} = G$, then $G'$ is cyclic.
\end{exercise}

\begin{proof}
    Let $x'$ be an element of $G'$. Because $\phi: G \to G'$ is an isomorphism, there exists $x\in G$ such that $x' = \phi(x)$. On the other hand, $a$ generates $G$, so there exists $n\in\mathbb{Z}$ such that $a^{n} = x$. $x' = \phi(x) = \phi(a^{n}) = {(\phi(a))}^{n}$. Therefore, $G' = \anglebracket{\phi(a)}$.

    Hence $G'$ is cyclic.
\end{proof}

\newpage
% section 5/exercise 47
\begin{exercise}
    Show that if $H$ and $K$ are subgroups of an abelian group $G$, then
    \[
        \{ hk \mid h\in H \text{ and } k\in K \}
    \]

    is a subgroup of $G$.
\end{exercise}

\begin{proof}
    Because $G$ is abelian, then every subgroup of $G$ is abelian.

    If $x_{1}, x_{2}$ are elements of the given subset, then there exist $h_{1}, h_{2}\in H$ and $k_{1}, k_{2}\in K$ such that $h_{1}k_{1} = x_{1}$ and $h_{2}k_{2} = x_{2}$.
    \[
        x_{1}x_{2} = (h_{1}k_{1})(h_{2}k_{2}) = (h_{1}(k_{1}h_{2}))k_{2} = ((h_{1}h_{2})k_{1})k_{2} = (h_{1}h_{2})(k_{1}k_{2})
    \]

    Because $h_{1}h_{2}\in H$ and $k_{1}k_{2}\in K$, then $x_{1}x_{2} = (h_{1}h_{2})(k_{1}k_{2})$ is in the given subset. So the given subset is closed under the induced operation.

    $H, K$ are subgroups of $G$ so the identity element $e$ of $G$ is in $H$ and $K$. So $e = ee$ is in the given subset.

    If $x$ is an element of the given subset, then there exist $h\in H, k\in K$ such that $x = hk$. $G$ is abelian, so $h^{-1}k^{-1} = k^{-1}h^{-1}$
    \[
        (h^{-1}k^{-1})x = x(h^{-1}k^{-1}) = (hk)(k^{-1}h^{-1}) = (h(kk^{-1}))h^{-1} = hh^{-1} = e
    \]

    $h^{-1}\in H$, $k^{-1}\in K$ because $H, K$ are subgroups of $G$, so $h^{-1}k^{-1}$ is in the given subset. Therefore, the inverse of $x$ is in the given subset.

    Hence $\{ hk \mid h\in H \text{ and } k\in K \}$ is a subgroup of $K$.
\end{proof}

\newpage
% section 5/exercise 48
\begin{exercise}
    Find an example of a group $G$ and two subgroups $H$ and $K$ such that the set in Exercise 47 is not a subgroup of $G$.
\end{exercise}

\begin{proof}
    Example: $G = D_{3} = \{ \iota, \rho, \rho^{2}, \mu, \mu\rho, \mu\rho^{2} \}$, $H = \{ 1, \mu \}$, $K = \{ 1, \mu\rho \}$.

    The set in Exercise 47 is $\{ 1, \mu, \mu\rho, \rho \}$, this is not a subgroup of $G$ because $\rho\mu = \mu\rho^{2}$ is not its element.
\end{proof}

\newpage
% section 5/exercise 49
\begin{exercise}
    Prove that for any integer $n\geq 3$, $S_{n}$ has a subgroup isomorphic with $D_{n}$.
\end{exercise}

\begin{proof}
    $S_{n}$ is the group of all permutations on $\{ 1, 2, \ldots, n \}$. We says $i, j$ are next to each other if $\abs{i - j} = 1$ or $\abs{i = j} = n - 1$. Let $T_{n}$ be the set of permutations $\phi$ on $\{ 1, 2, \ldots, n \}$ such that $i, j$ are next to each other if and only if $\phi(i), \phi(j)$ are next to each other.

    If $\sigma, \tau$ are in $T_{n}$, then $i, j$ are next to each other iff $\tau(i), \tau(j)$ are next to each other iff $\sigma(\tau(i)), \sigma(\tau(j))$ are next to each other. So $T_{n}$ is closed under composition. The identity mapping is an element of $T_{n}$. If $\sigma$ is in $T_{n}$, then $i, j$ are next to each other iff $\sigma(i), \sigma(j)$ are next to each other. Therefore, $\phi^{-1}(j), \phi^{-1}(j)$ are next to each other iff $i, j$ are next to each other. So $T_{n}$ contains the inverse of each element. Hence $T_{n}$ is a subgroup of $S_{n}$.

    Let $P_{0}P_{1}\ldots P_{n-1}$ be a regular $n$-gon. $D_{n}$ is the group of bijection $\phi$ from $\mathbb{Z}_{n} \to \mathbb{Z}_{n}$ such that $P_{i}P_{j}$ is an edge of the polygon if and only if $P_{\phi(i)}P_{\phi(j)}$ is an edge of the polygon. $P_{i}P_{j}$ is an edge of the polygon iff $\abs{i - j} = 1$ or $\abs{i - j} = n - 1$.

    Let $\sigma$ be an element of $T_{n}$. We define a map $\phi: T_{n} \to D_{n}$ as follows:
    \begin{align*}
        \phi:          & \quad\sigma \mapsto \phi(\sigma)                                  \\
        \phi(\tau)(x)= & \quad\sigma(x + 1) - 1\quad \text{for every $x\in\mathbb{Z}_{n}$}
    \end{align*}

    $\phi$ is a bijection from $T_{n}$ onto $D_{n}$. Let's check for the homomorphism property.
    \begin{align*}
        (\phi(\sigma)\circ\phi(\tau))(x) & = \phi(\sigma)(\phi(\tau)(x))     \\
                                         & = \phi(\sigma)(\tau(x + 1) - 1)   \\
                                         & = \sigma(\tau(x + 1) - 1 + 1) - 1 \\
                                         & = \sigma(\tau(x + 1)) - 1         \\
        \phi(\sigma\circ\tau)(x)         & = (\sigma\circ\tau)(x + 1) - 1    \\
                                         & = \sigma(\tau(x + 1)) - 1
    \end{align*}

    So $\phi(\sigma)\circ\phi(\tau) = \phi(\sigma\circ\tau)$, and $\phi$ is an isomorphism.

    Thus the subgroup $T_{n}$ of $S_{n}$ is isomorphic with $D_{n}$.
\end{proof}

\newpage
% section 5/exercise 50
\begin{exercise}
    Find the flaw in the following argument: ``Condition $2$ of Theorem 5.12 is redundant, since it can be derived from $1$ and $3$, for let $a\in H$. Then $a^{-1}\in H$ by 3, and by 1, $aa^{-1} = e$ is an element of $H$, proving $2$''
\end{exercise}

\begin{proof}
    When the subset is the empty set, then condition 1 and 3 hold. But the empty set is not a subgroup of any group.
\end{proof}

\newpage
% section 5/exercise 51
\begin{exercise}
    Prove Theorem 5.15: A nonempty subset $H$ of the group $G$ is a subgroup of $G$ if and only if for all $a, b\in H$, $ab^{-1}\in H$.
\end{exercise}

\begin{proof}
    Since $a, b\in H$ implies $ab^{-1}\in H$, then $a\in H$ implies $e = aa^{-1}\in H$. So $H$ contains the identity element. If $a\in H$, then $e, a\in H$, it follows that $a^{-1} = ea^{-1}\in H$. So $H$ contains the inverse of each element.

    If $a, b\in H$, then $a, b^{-1}\in H$, it follows that $ab = a{(b^{-1})}^{-1} \in H$. So $H$ is closed under the induced operation.

    Thus $H$ is a subgroup of $G$.
\end{proof}

\newpage
% section 5/exercise 52
\begin{exercise}
    Prove that if $G$ is a cyclic group and $\abs{G}$, then $G$ has at least $2$ generators.
\end{exercise}

\begin{proof}
    Let $x$ be a generator of $G$, then $G = \{ x^{n} \mid n\in\mathbb{Z} \}$.

    If $x^{2} = e$, then $\abs{G} = 2$. So $x^{2} \ne e$, which means the inverse of $x$ is not equal to $x$. On the other hand, $\{ x^{n} \mid n\in\mathbb{Z} \} = \{ {(x^{-1})}^{n} \mid n\in\mathbb{Z} \}$, so $x^{-1}$ generates $G$. Thus $G$ has at least $2$ generators.
\end{proof}

\newpage
% section 5/exercise 53
\begin{exercise}
    Prove that if $G$ is an abelian group, written multiplicatively, with identity element $e$, then all elements $x$ of $G$ satisfying the equation $x^{2} = e$ form a subgroup $H$ of $G$.
\end{exercise}

\begin{proof}
    $e^{2} = e$, so $e\in H$. So $H$ contains the identity element.

    If $x, y\in H$, then ${(xy)}^{2} = xyxy = xxyy = ee = e$ and $x^{-1} = x$, it follows that ${(x^{-1})}^{2} = x^{2} = e$. So $H$ is also closed under the induced operation and $H$ contains the inverse of each element.

    Thus $H$ is a subgroup of $G$.
\end{proof}

\newpage
% section 5/exercise 54
\begin{exercise}
    Repeat Exercise 53 for the general situation of the set $H$ of all solutions $x$ of the equation $x^{n} = e$ for a fixed integer $n \geq 1$ in an abelian group $G$ with identity $e$.
\end{exercise}

\begin{proof}
    $e^{n} = e$, so $H$ contains the identity element.

    We prove by mathematical induction that ${(xy)}^{n} = x^{n}y^{n}$. The statement holds for $n = 1$. Assume that the statment holds for $n = k$, then ${(xy)}^{k+1} = {(xy)}^{k}(xy) = (x^{k}y^{k})(xy) = (x^{k}(xy^{k}))y = x^{k+1}y^{k+1}$. Due to the principle of mathematical induction, ${(xy)}^{n} = x^{n}y^{n}$. So, if $x, y\in H$, ${(xy)}^{n} = x^{n}y^{n} = ee = e$, it follows that $H$ is closed under the induced operation.

    If $x\in H$, then $xx^{n-1} = x^{n-1}x = e$, which means $x^{n-1}$ is the inverse of $x$. ${(x^{n-1})}^{n} = x^{n(n-1)} = {(x^{n})}^{n-1} = e^{n-1} = e$. So $H$ contains the inverse of each element.

    Thus $H$ is a subgroup of $G$.
\end{proof}

\newpage
% section 5/exercise 55
\begin{exercise}
    Find a counterexample to Exercise 53 if the assumption of abelian is dropped.
\end{exercise}

\begin{proof}
    Example: $D_{3} = \{ \iota, \rho, \rho^{2}, \mu, \mu\rho, \mu\rho^{2} \}$. The subset of $D_{3}$ of all solutions of the equation $x^{2} = \iota$ is $\{ \iota, \mu, \mu\rho, \mu\rho^{2} \}$. But this is not a subgroup because $\mu\mu\rho = \rho$, which is not in the subset. In the other words, this subset is not closed under the induced operation.
\end{proof}

\newpage
% section 5/exercise 56
\begin{exercise}
    Show that if $a\in G$, where $G$ is a finite group with identity $e$, then there exists $n\in\mathbb{Z}$ such that $a^{n} = e$.
\end{exercise}

\begin{proof}
    Let $m$ be the order of $G$. $a, a^{2}, \ldots, a^{m}, a^{m+1}$ are elements of $G$, so there exists $1\leq i < j \leq m$ such that $a^{i} = a^{j}$. Therefore $a^{j-i} = e$. Pick $n = j - i$, we obtain that $a^{n} = e$.

    Thus, there exists a positive integer $n$ such that $a^{n} = e$.
\end{proof}

\newpage
% section 5/exercise 57
\begin{exercise}
    Prove Theorem 5.16: Let $H$ be a finite nonempty subset of the group $G$. Then $H$ is a subgroup of $G$ if and only if $H$ is closed under the operation of $G$.
\end{exercise}

\begin{proof}
    $(\Rightarrow)$ If $H$ is a subgroup of $G$, then $H$ is closed under the operation of $G$, according to the definition of subgroups.

    $(\Leftrightarrow)$ $H$ is closed under the operation of $G$.

    Let $x$ be an element of $H$. Let $m$ be the order of $H$. $x, x^{2}, \ldots, x^{m+1}$ are $(m+1)$ elements of $H$, so there exist $1\leq i < j \leq m$ such that $x^{i} = x^{j}$. Let $n = j - i$, then $x^{n} = e$.

    Since $H$ is closed under the operation of $G$, $x^{n}$ and $x^{n-1}$ are elements of $H$. On the other hand, $x^{n} = e$, $x^{n-1}$ is the inverse of $x$. It follows that $H$ contains the identity element and the inverse of each element. Hence $H$ is a subgroup of $G$.

    In conclusion, for finite nonempty subset $H$ of the group $G$, $H$ is a subgroup of $G$ if and only if $H$ is closed under the operation of $G$.
\end{proof}

\newpage
% section 5/exercise 58
\begin{exercise}
    Let $G$ be a group and let $a$ be one fixed element of $G$. Show that
    \[
        H_{a} = \{ x\in G \mid xa = ax \}
    \]

    is a subgroup of $G$.
\end{exercise}

\begin{proof}
    Since $ea = ae$, then $H_{a}$ contains the identity element.

    Assume that $x, y\in H_{a}$, due to associative law
    \begin{align*}
        (xy)a & = x(ya) = x(ay) & \text{$y\in H_{a}$} \\
              & = (xa)y = (ax)y & \text{$x\in H_{a}$} \\
              & = a(xy)
    \end{align*}

    it follows that $H_{a}$ is closed under the induced operation. On the other hand, $xa = ax$ implies $x^{-1}xax^{-1} = x^{-1}axx^{-1}$. Therefore, $ax^{-1} = x^{-1}a$, $H_{a}$ contains the inverse of each element.

    Thus $H_{a}$ is a subgroup of $G$.
\end{proof}

\newpage
% section 5/exercise 59
\begin{exercise}
    Generalizing Exercise 58, let $S$ be any subset of a group $G$.
    \begin{enumerate}[label={\textbf{\alph*.}}]
        \item Show that $H_{S} = \{ x\in G \mid xs = sx \text{ for all } s\in S \}$ is a subgroup of $G$.
        \item In reference to part (a), the subgroup $H_{G}$ is the \textbf{center of $G$}. Show that $H_{G}$ is an abelian group.
    \end{enumerate}
\end{exercise}

\begin{proof}
    \begin{enumerate}[label={\textbf{\alph*.}}]
        \item $es = se = s$ for all $s\in S$, so $H_{S}$ contains the identity element.

              If $x, y\in H_{S}$, then $(xy)s = x(ys) = x(sy) = (xs)y = (sx)y = s(xy)$. So $xy\in H_{S}$, which means $H_{S}$ is closed under the induced operation.

              If $x\in H_{S}$, then $xs = sx$ for all $s\in S$. So $x^{-1}xsx^{-1} = x^{-1}sxx^{-1}$ for all $s\in S$. Therefore, $sx^{-1} = x^{-1}s$ for all $s\in S$, which means $x^{-1}\in H_{S}$.

              Hence $H_{S}$ is a subgroup of $G$.
        \item Let $x, y\in H_{G}$, then $y\in G$. According to the definition of $H_{G}$, $xy = yx$. Therefore, $xy = yx$ for all $x, y\in H_{G}$.

              Hence $H_{G}$ is an abelian group.
    \end{enumerate}
\end{proof}

\newpage
% section 5/exercise 60
\begin{exercise}
    Let $H$ be a subgroup of a group $G$. For $a, b\in G$, let $a\sim b$ if and only if $ab^{-1}\in H$. Show that $\sim$ is an equivalent relation on $G$.
\end{exercise}

\begin{proof}
    $aa^{-1} = e \in H$ because $H$ is a subgroup of $G$. So $\sim$ is reflexive.

    $a\sim b$ iff $ab^{-1}\in H$. The inverse of $ab^{-1}$ is $ba^{-1}$. $ab^{-1}\in H$ iff $ba^{-1}\in H$. $ba^{-1}\in H$ iff $b\sim a$. So $\sim$ is symmetric.

    $a\sim b$ and $b\sim c$ iff $ab^{-1}\in H, bc^{-1}\in H$. It follows that $ac^{-1} = ab^{-1}bc^{-1} \in H$ (since $H$ is a subgroup of $G$). Therefore $a\sim c$. So $\sim$ is transitive.

    Thus $\sim$ is an equivalence relation on $G$.
\end{proof}

\newpage
% section 5/exercise 61
\begin{exercise}
    For sets $H$ and $K$, we define the \textbf{intersection} $H\cap K$ by
    \[
        H\cap K = \{ x \mid x\in H \text{ and } x\in K \}
    \]

    Show that if $H\leq G$ and $K\leq G$, then $H\cap K\leq G$.
\end{exercise}

\begin{proof}
    $H\cap K$ is a subset of $H$, $H$ is a subset of $G$, so $H\cap K$ is a subset of $G$.

    $H, K$ contain the identity element, so $H\cap K$ contains the identity element.

    If $x, y\in H\cap K$, then $x, y\in H, x, y\in K$. Because $H, K$ are subgroups of $G$, it follows that $xy\in H$ and $xy\in K$. So $xy\in HK$ for all $x, y\in H\cap K$. Therefore $H\cap K$ is closed under the induced operation.

    If $x\in H\cap K$, then $x\in H, x\in K$. Because $H, K$ are subgroups of $G$, it follows that $x^{-1}\in H, x^{-1}\in K$. So $x^{-1}\in H\cap K$ for all $x\in H\cap K$. Therefore $H\cap K$ contains the inverse of each element.

    Thus, if $H\leq G$ and $K\leq G$, then $H\cap K \leq G$.
\end{proof}

\newpage
% section 5/exercise 62
\begin{exercise}
    Prove that every cyclic group is abelian.
\end{exercise}

\begin{proof}
    Let $G$ be a cyclic group. According to the definition of cyclic groups, there exists an element $a$ of $G$ such that $a$ generates $G$. Let $x, y$ be elements of $G$. Because $a$ generates $G$, there exist integers $m, n$ such that $a^{m} = x$ and $a^{n} = y$. $xy = a^{m}a^{n} = a^{m+n} = a^{n+m} = a^{n}a^{m} = yx$. Therefore $G$ is abelian.

    Thus every cyclic group is abelian.
\end{proof}

\newpage
% section 5/exercise 63
\begin{exercise}
    Let $G$ be a group and let $G_{n} = \{ g^{n} \mid g\in G \}$. Under what hypothesis about $G$ can we show that $G_{n}$ is a subgroup of $G$.
\end{exercise}

\begin{proof}
    If $G$ is abelian.

    When $G$ is abelian, $x^{n}y^{n} = {(xy)}^{n}$ for all $x, y\in G$. This has been proved in Exercise 54 section 5 by using mathematical induction. With this condition, $G_{n}$ is closed under the induced operation. On the other hand, $G_{n}$ contains the identity element $e = e^{n}$ and the inverse of each of its elements (the inverse of $x = g^{n}$ is ${(g^{-1})}^{n}$). Thus if $G$ is abelian, then $G_{n}$ is a subgroup of $G$.
\end{proof}

\newpage
% section 5/exercise 64
\begin{exercise}
    Show that a group with no proper nontrivial subgroups is cyclic. Is the converse true?
\end{exercise}

\begin{proof}
    Let $G$ be such a group.

    If $G$ is a trivial group, then $G$ is cyclic.

    Otherwise, $G$ has an element $a$ other than the identity element $e$. The set $\{ a^{n} \mid n\in\mathbb{Z} \}$ is a subgroup of $G$. This subgroup is nontrivial since $a\ne e$. On the other hand, $G$ does not have a proper nontrivial subgroups, so $\{ a^{n} \mid n\in\mathbb{Z} \}$ is $G$. Therefore $a$ generates $G$. Due to the definition of cyclic group, $G$ is cyclic.

    Thus, a group with no proper nontrivial subgroups is cyclic.

    The converse is not necessarily true. For example, $\mathbb{Z}_{4}$ is generated by $1$, but it has the proper nontrivial subgroup $\{ 0, 2 \}$.
\end{proof}

\newpage
% section 5/exercise 65
\begin{exercise}
    Cracker Barrel Restaurants place a puzzle called ``Jump All But One Game'' at each table. The puzzle starts with golf tees arranged in a triangle as in Figure 5.29a where the presence of a tee is noted with a solid dot and the absence is noted with a hollow dot. A move can be made if a tee can jump over one adjacent tee and land on an empty space. When a move is made, the tee that is jumped over is removed. A possible first move is shown in Figure 5.29b. The goal is to have just one remaining tee. Use the Klein 4-group to show that no matter what sequence of (legal) moves you make, the last remaining tee cannot be in a bottom comer position.
\end{exercise}

\begin{proof}
    % unsolved
\end{proof}

\newpage
\section{Cyclic Groups}

\subsection*{Computations}

In Exercises 1 through 4, find the quotient and remainder, according to the division algorithm, when $n$ is divided by $m$.

\newpage
% section 6/exercise 1
\begin{exercise}
    $n = 42, m = 9$
\end{exercise}

\begin{proof}
    $42 = 4\cdot 9 + 6$. The quotient is $4$, the remainder is $6$ when $42$ is divided by $9$.
\end{proof}

\newpage
% section 6/exercise 2
\begin{exercise}
    $n = -42, m = 9$
\end{exercise}

\begin{proof}
    $-42 = -5 \cdot 9 + 3$. The quotient is $-5$, the remainder is $3$ when $-42$ is divided by $9$.
\end{proof}

\newpage
% section 6/exercise 3
\begin{exercise}
    $n = -37, m = 8$
\end{exercise}

\begin{proof}
    $-37 = -5\cdot 8 + 3$. The quotient is $-5$, the remainder is $3$ when $-37$ is divided by $8$.
\end{proof}

\newpage
% section 6/exercise 4
\begin{exercise}
    $n = 37, m = 8$
\end{exercise}

\begin{proof}
    $37 = 4\cdot 8 + 5$. The quotient is $4$, the remainder is $5$ when $37$ is divided by $8$.
\end{proof}

In Exercises 5 through 7, find the greatest common divisor of the two integers.

\newpage
% section 6/exercise 5
\begin{exercise}
    $32$ and $24$
\end{exercise}

\begin{proof}
    \begin{align*}
        32 & = 1\cdot 24 + 8 \\
        24 & = 3\cdot 8
    \end{align*}

    Hence $\text{gcd}(32, 24) = 8$.
\end{proof}

\newpage
% section 6/exercise 6
\begin{exercise}
    $48$ and $88$
\end{exercise}

\begin{proof}
    \begin{align*}
        88 & = 1\cdot 48 + 40 \\
        48 & = 1\cdot 40 + 8  \\
        40 & = 5\cdot 8
    \end{align*}

    Hence $\text{gcd}(88, 48) = 8$.
\end{proof}

\newpage
% section 6/exercise 7
\begin{exercise}
    $360$ and $420$
\end{exercise}

\begin{proof}
    \begin{align*}
        420 & = 1\cdot 360 + 60 \\
        360 & = 6\cdot 60
    \end{align*}

    Hence $\text{gcd}(360, 420) = 60$.
\end{proof}

In Exercises 8 through 11, find the number of generators of a cyclic group having the given order.

\newpage
% section 6/exercise 8
\begin{exercise}
    $5$
\end{exercise}

\begin{proof}
    $1, 2, 3, 4$ are precisely the positive integers less than $5$ and relatively prime to $5$.

    Hence a $5$-order cyclic group has $4$ generators.
\end{proof}

\newpage
% section 6/exercise 9
\begin{exercise}
    $8$
\end{exercise}

\begin{proof}
    $1, 3, 5, 7$ are precisely the positive integers less than $8$ and relatively prime to $8$.

    Hence an $8$-order cyclic group has $4$ generators.
\end{proof}

\newpage
% section 6/exercise 10
\begin{exercise}
    $24$
\end{exercise}

\begin{proof}
    $1, 5, 7, 11, 13, 17, 19, 23$ are precisely the positive integers less than $24$ and relatively prime to $24$.

    Hence an $24$-order cyclic group has $8$ generators.
\end{proof}

\newpage
% section 6/exercise 11
\begin{exercise}
    $84$
\end{exercise}

\begin{proof}
    $1, 5, 11, 13, 17, 19, 23, 25, 29, 31, 37, 41, 43, 47, 53, 55, 59, 61, 65, 67, 71, 73, 79, 83$ are precisely the positive integers less than $84$ and relatively prime to $84$.

    Hence an $84$-order cyclic group has $24$ generators.
\end{proof}

An isomorphism of a group with itself is an \textbf{automorphism of the group}. In Exercises 12 through 16, find the number of automorphisms of the given group. (You may use Exercise 53. What must be the image of a generator under an automorphism?)

Within a cyclic group, an automorphism maps a generator to a generator.

\newpage
% section 6/exercise 12
\begin{exercise}
    $\mathbb{Z}_{2}$
\end{exercise}

\begin{proof}
    $\mathbb{Z}_{2}$ has $1$ generators. So there is $1$ automorphism.
\end{proof}

\newpage
% section 6/exercise 13
\begin{exercise}
    $\mathbb{Z}_{6}$
\end{exercise}

\begin{proof}
    $\mathbb{Z}_{6}$ has $2$ generators. So there is $2$ automorphisms.
\end{proof}

\newpage
% section 6/exercise 14
\begin{exercise}
    $\mathbb{Z}_{8}$
\end{exercise}

\begin{proof}
    $\mathbb{Z}_{8}$ has $4$ generators. So there is $4$ automorphisms.
\end{proof}

\newpage
% section 6/exercise 15
\begin{exercise}
    $\mathbb{Z}$
\end{exercise}

\begin{proof}
    $\mathbb{Z}$ has $2$ generators. So there is $2$ automorphisms.
\end{proof}

\newpage
% section 6/exercise 16
\begin{exercise}
    $\mathbb{Z}_{84}$
\end{exercise}

\begin{proof}
    $\mathbb{Z}_{84}$ has $24$ generators. So there is $24$ automorphisms.
\end{proof}

In Exercises 17 through 23, find the number of elements in the indicated cyclic group.

\newpage
% section 6/exercise 17
\begin{exercise}
    The cyclic subgroup of $\mathbb{Z}_{30}$ generated by $25$.
\end{exercise}

\begin{proof}
    $\text{gcd}(25, 30) = 5$.

    Thus the number of elements in the indicated cyclic group is $30/5 = 6$.
\end{proof}

\newpage
% section 6/exercise 18
\begin{exercise}
    The cyclic subgroup of $\mathbb{Z}_{42}$ generated by $30$.
\end{exercise}

\begin{proof}
    $\text{gcd}(30, 42) = 6$.

    Thus the number of elements in the indicated cyclic group is $42/6 = 7$.
\end{proof}

\newpage
% section 6/exercise 19
\begin{exercise}
    The cyclic subgroup $\anglebracket{i}$ of the group $\mathbb{C}^{*}$ of nonzero complex numbers under multiplication
\end{exercise}

\begin{proof}
    \begin{align*}
        i^{2} & = -1 \\
        i^{3} & = -i \\
        i^{4} & = 1
    \end{align*}

    So $4$ is the smallest positive integer $n$ such that $i^{n} = 1$. Hence the number of elements in the indicated cyclic group is $4$.
\end{proof}

\newpage
% section 6/exercise 20
\begin{exercise}
    The cyclic subgroup of the group $\mathbb{C}^{*}$ of Exercise 19 generated by $(1 + i)/\sqrt{2}$
\end{exercise}

\begin{proof}
    \begin{align*}
        {((1 + i)/\sqrt{2})}^{2} & = i                   \\
        {((1 + i)/\sqrt{2})}^{3} & = {(1 + i)/\sqrt{2}}  \\
        {((1 + i)/\sqrt{2})}^{4} & = -1                  \\
        {((1 + i)/\sqrt{2})}^{5} & = {(-1 - i)/\sqrt{2}} \\
        {((1 + i)/\sqrt{2})}^{6} & = -i                  \\
        {((1 + i)/\sqrt{2})}^{7} & = {(1 - i)/\sqrt{2}}  \\
        {((1 + i)/\sqrt{2})}^{8} & = 1
    \end{align*}

    So $8$ is the smallest positive integer $n$ such that $i^{n} = 1$. Hence the number of elements in the indicated cyclic group is $8$.
\end{proof}

\newpage
% section 6/exercise 21
\begin{exercise}
    The cyclic subgroup of the group $\mathbb{C}^{*}$ of Exercise 19 generated by $(1 + i)$.
\end{exercise}

\begin{proof}
    \[
        {(1 + i)}^{n} = {\sqrt{2}(\cos\frac{\pi}{4} + i\sin\frac{\pi}{4})}^{n} = \sqrt{2^{n}}(\cos\frac{n\pi}{4} + i\sin\frac{n\pi}{4})
    \]

    So ${(1 + i)}^{n} = 1$ if and only if $n = 0$. Hence the number of elements in the indicated cyclic group is infinite.
\end{proof}

\newpage
% section 6/exercise 22
\begin{exercise}
    The cyclic subgroup $\anglebracket{\rho^{10}}$ of $D_{24}$
\end{exercise}

\begin{proof}
    $R_{24} = \{ \iota, \rho, \ldots, \rho^{23} \}$ is a cyclic subgroup of $D_{24}$, and the indicated subgroup is a subgroup of $R_{24}$. $R_{24}$ has $24$ elements.

    $\text{gcd}(10, 24) = 2$. Hence the number of elements in the indicated cyclic group is $24/2 = 12$.
\end{proof}

\newpage
% section 6/exercise 23
\begin{exercise}
    The cyclic subgroup $\anglebracket{\rho^{35}}$ of $D_{375}$
\end{exercise}

\begin{proof}
    $R_{375} = \{ \iota, \rho, \ldots, \rho^{374} \}$ is a cyclic subgroup of $D_{375}$. The indicated subgroup is a subgroup of $R_{375}$. $R_{375}$ has $375$ elements.

    $\text{gcd}(35, 375) = 5$. Hence the number of elements in the indicated cyclic group is $375/5 = 75$.
\end{proof}

\newpage
% section 6/exercise 24
\begin{exercise}
    Consider the group $S_{10}$
    \begin{enumerate}[label={\textbf{\alph*.}}]
        \item What is the order of the cycle $(2, 4, 6, 7)$?
        \item What is the order of $(1, 4)(2, 3, 5)$? Of $(1, 3)(2, 4, 6, 7, 8)$?
        \item What is the order of $(1, 5, 9)(2, 6, 7)$? Of $(1, 3)(2, 5, 6, 8)$?
        \item What is the order of $(1, 2)(3, 4, 5, 6, 7, 8)$? Of $(1, 2, 3)(4, 5, 6, 7, 8, 9)$?
        \item State a theorem suggested by parts (c) and (d).
    \end{enumerate}
\end{exercise}

\begin{proof}
    If $\sigma, \tau$ are disjoint cycles, then ${(\sigma\tau)}^{n} = \sigma^{n}\tau^{n}$ for every integer $n$ (it follows from mathematical induction and $\sigma\tau = \tau\sigma$).

    \begin{enumerate}[label={\textbf{\alph*.}}]
        \item The order of the cycle $(2, 4, 6, 7)$ is $4$ (the length of the cycle).
        \item The orders of the cycles $(1, 4)$ and $(2, 3, 5)$ are $2$ and $3$. If $n$ is the smallest positive integer such that ${[(1, 4)(2, 3, 5)]}^{n} = \iota$, then $2$ and $3$ divides $n$. Such smallest positive number is $6$. So the order of $(1, 4)(2, 3, 5)$ is $6$.

              The order of $(1, 3)(2, 4, 6, 7, 8)$ is $10$.
        \item The order of $(1, 5, 9)(2, 6, 7)$ is $3$. The order of $(1, 3)(2, 5, 6, 8)$ is $8$.
        \item The order of $(1, 2)(3, 4, 5, 6, 7, 8)$ is $6$. The order of $(1, 2, 3)(4, 5, 6, 7, 8, 9)$ is $6$.
        \item Theorem: if a permutation on $S_{n}$ is the product of $r$ disjoint cycles with length $k_{1}$, $k_{2}$, \ldots, $k_{r}$, then the order of the permutation is the least common multiple of $k_{1}$, $k_{2}$, \ldots, $k_{r}$.\qedhere
    \end{enumerate}
\end{proof}

In Exercises 25 through 30, find the maximum possibe order for an element of $S_{n}$ for a given value of $n$.

Every permutation in $S_{n}$ is a cycle or a product of disjoint cycles. The order of a permutation in $S_{n}$ is the least common multiple of the orders of its disjoint cycles. So a permutation with the maximum possible order corresponds to a partition of $n = k_{1} + k_{2} + \cdots + k_{r}$ such that the least common multiple of $k_{1}$, $k_{2}$, \ldots, $k_{r}$ is the greatest. See Landau's function.

\newpage
% section 6/exercise 25
\begin{exercise}
    $n = 5$
\end{exercise}

\begin{proof}
    \begin{align*}
        5 & = 5                           \\
        5 & = 1 + 4                       \\
        5 & = 1 + 1 + 3                   \\
        5 & = 1 + 1 + 1 + 2               \\
        5 & = 1 + 1 + 1 + 1 + 1           \\
        5 & = 2 + 3             & (\star) \\
        5 & = 1+ 2 + 2
    \end{align*}

    Hence the maximum possible order of an element of $S_{5}$ is $6$.
\end{proof}

\newpage
% section 6/exercise 26
\begin{exercise}
    $n = 6$
\end{exercise}

\begin{proof}
    \begin{align*}
        6 & = 6                     & (\star) \\
        6 & = 1 + 5                           \\
        6 & = 1 + 1 + 4                       \\
        6 & = 1 + 1 + 1 + 3                   \\
        6 & = 1 + 1 + 1 + 1 + 2               \\
        6 & = 1 + 1 + 1 + 1 + 1 + 1           \\
        6 & = 1 + 1 + 2 + 2                   \\
        6 & = 1 + 2 + 3                       \\
        6 & = 2 + 2 + 2                       \\
        6 & = 2 + 4                           \\
        6 & = 3 + 3
    \end{align*}

    Hence the maximum possible order of an element of $S_{6}$ is $6$.
\end{proof}

\newpage
% section 6/exercise 27
\begin{exercise}
    $n = 7$
\end{exercise}

\begin{proof}
    \begin{align*}
        7 & = 7                                   \\
        7 & = 1 + 6                               \\
        7 & = 1 + 1 + 5                           \\
        7 & = 1 + 1 + 1 + 4                       \\
        7 & = 1 + 1 + 1 + 1 + 3                   \\
        7 & = 1 + 1 + 1 + 1 + 1 + 2               \\
        7 & = 1 + 1 + 1 + 1 + 1 + 1 + 1           \\
        7 & = 1 + 1 + 1 + 2 + 2                   \\
        7 & = 1 + 1 + 2 + 3                       \\
        7 & = 1 + 2 + 2 + 2                       \\
        7 & = 1 + 2 + 4                           \\
        7 & = 1 + 3 + 3                           \\
        7 & = 2 + 2 + 3                           \\
        7 & = 2 + 5                               \\
        7 & = 3 + 4                     & (\star)
    \end{align*}

    Hence the maximum possible order of an element of $S_{7}$ is $12$.
\end{proof}

\newpage
% section 6/exercise 28
\begin{exercise}
    $n = 8$
\end{exercise}

\begin{proof}
    \begin{align*}
        8 & = 8                                       \\
        8 & = 1 + 7                                   \\
        8 & = 1 + 1 + 6                               \\
        8 & = 1 + 1 + 1 + 5                           \\
        8 & = 1 + 1 + 1 + 1 + 4                       \\
        8 & = 1 + 1 + 1 + 1 + 1 + 3                   \\
        8 & = 1 + 1 + 1 + 1 + 1 + 1 + 2               \\
        8 & = 1 + 1 + 1 + 1 + 1 + 1 + 1 + 1           \\
        8 & = 1 + 1 + 1 + 1 + 2 + 2                   \\
        8 & = 1 + 1 + 1 + 2 + 3                       \\
        8 & = 1 + 1 + 2 + 2 + 2                       \\
        8 & = 1 + 1 + 2 + 4                           \\
        8 & = 1 + 1 + 3 + 3                           \\
        8 & = 1 + 2 + 5                               \\
        8 & = 1 + 2 + 2 + 3                           \\
        8 & = 1 + 3 + 4                               \\
        8 & = 2 + 6                                   \\
        8 & = 2 + 2 + 2 + 2                           \\
        8 & = 2 + 2 + 4                               \\
        8 & = 2 + 3 + 3                               \\
        8 & = 3 + 5                         & (\star) \\
        8 & = 4 + 4
    \end{align*}

    Hence the maximum possible order of an element of $S_{8}$ is $15$.
\end{proof}

\newpage
% section 6/exercise 29
\begin{exercise}
    $n = 10$
\end{exercise}

\begin{proof}
    \begin{align*}
        10 & = 10                                              \\
        10 & = 1 + 9                                           \\
        10 & = 1 + 1 + 8                                       \\
        10 & = 1 + 1 + 1 + 7                                   \\
        10 & = 1 + 1 + 1 + 1 + 6                               \\
        10 & = 1 + 1 + 1 + 1 + 1 + 5                           \\
        10 & = 1 + 1 + 1 + 1 + 1 + 1 + 4                       \\
        10 & = 1 + 1 + 1 + 1 + 1 + 1 + 1 + 3                   \\
        10 & = 1 + 1 + 1 + 1 + 1 + 1 + 1 + 1 + 2               \\
        10 & = 1 + 1 + 1 + 1 + 1 + 1 + 1 + 1 + 1 + 1           \\
        10 & = 1 + 1 + 1 + 1 + 1 + 1 + 2 + 2                   \\
        10 & = 1 + 1 + 1 + 1 + 1 + 2 + 3                       \\
        10 & = 1 + 1 + 1 + 1 + 2 + 2 + 2                       \\
        10 & = 1 + 1 + 1 + 1 + 2 + 4                           \\
        10 & = 1 + 1 + 1 + 1 + 3 + 3                           \\
        10 & = 1 + 1 + 1 + 2 + 2 + 3                           \\
        10 & = 1 + 1 + 1 + 2 + 5                               \\
        10 & = 1 + 1 + 1 + 3 + 4                               \\
        10 & = 1 + 1 + 2 + 2 + 2 + 2                           \\
        10 & = 1 + 1 + 2 + 2 + 4                               \\
        10 & = 1 + 1 + 2 + 6                                   \\
        10 & = 1 + 1 + 2 + 3 + 3                               \\
        10 & = 1 + 1 + 3 + 5                                   \\
        10 & = 1 + 1 + 4 + 4                                   \\
        10 & = 1 + 2 + 2 + 2 + 3                               \\
        10 & = 1 + 2 + 2 + 5                                   \\
        10 & = 1 + 2 + 3 + 4                                   \\
        10 & = 1 + 2 + 7                                       \\
        10 & = 1 + 3 + 3 + 3                                   \\
        10 & = 1 + 3 + 6                                       \\
        10 & = 1 + 4 + 5                                       \\
        10 & = 2 + 8                                           \\
        10 & = 2 + 2 + 2 + 2 + 2                               \\
        10 & = 2 + 2 + 2 + 4                                   \\
        10 & = 2 + 2 + 6                                       \\
        10 & = 2 + 2 + 3 + 3                                   \\
        10 & = 2 + 3 + 5                             & (\star) \\
        10 & = 2 + 4 + 4                                       \\
        10 & = 3 + 7                                           \\
        10 & = 3 + 3 + 4                                       \\
        10 & = 4 + 6                                           \\
        10 & = 5 + 5
    \end{align*}

    Hence the maximum possible order of an element of $S_{10}$ is $30$.
\end{proof}

\newpage
% section 6/exercise 30
\begin{exercise}
    $n = 15$
\end{exercise}

\begin{proof}
    \begingroup
    \allowdisplaybreaks{}
    \begin{align*}
        15 & = 15                                                                  \\
           & = 1 + 14                                                              \\
           & = 1 + 1 + 13                                                          \\
           & = 1 + 1 + 1 + 12                                                      \\
           & = 1 + 1 + 1 + 1 + 11                                                  \\
           & = 1 + 1 + 1 + 1 + 1 + 10                                              \\
           & = 1 + 1 + 1 + 1 + 1 + 1 + 9                                           \\
           & = 1 + 1 + 1 + 1 + 1 + 1 + 1 + 8                                       \\
           & = 1 + 1 + 1 + 1 + 1 + 1 + 1 + 1 + 7                                   \\
           & = 1 + 1 + 1 + 1 + 1 + 1 + 1 + 1 + 1 + 6                               \\
           & = 1 + 1 + 1 + 1 + 1 + 1 + 1 + 1 + 1 + 1 + 5                           \\
           & = 1 + 1 + 1 + 1 + 1 + 1 + 1 + 1 + 1 + 1 + 1 + 4                       \\
           & = 1 + 1 + 1 + 1 + 1 + 1 + 1 + 1 + 1 + 1 + 1 + 1 + 3                   \\
           & = 1 + 1 + 1 + 1 + 1 + 1 + 1 + 1 + 1 + 1 + 1 + 1 + 1 + 2               \\
           & = 1 + 1 + 1 + 1 + 1 + 1 + 1 + 1 + 1 + 1 + 1 + 1 + 1 + 1 + 1           \\
           & = 1 + 1 + 1 + 1 + 1 + 1 + 1 + 1 + 1 + 1 + 1 + 2 + 2                   \\
           & = 1 + 1 + 1 + 1 + 1 + 1 + 1 + 1 + 1 + 1 + 2 + 3                       \\
           & = 1 + 1 + 1 + 1 + 1 + 1 + 1 + 1 + 1 + 2 + 2 + 2                       \\
           & = 1 + 1 + 1 + 1 + 1 + 1 + 1 + 1 + 1 + 2 + 4                           \\
           & = 1 + 1 + 1 + 1 + 1 + 1 + 1 + 1 + 1 + 3 + 3                           \\
           & = 1 + 1 + 1 + 1 + 1 + 1 + 1 + 1 + 2 + 2 + 3                           \\
           & = 1 + 1 + 1 + 1 + 1 + 1 + 1 + 1 + 2 + 5                               \\
           & = 1 + 1 + 1 + 1 + 1 + 1 + 1 + 1 + 3 + 4                               \\
           & = 1 + 1 + 1 + 1 + 1 + 1 + 1 + 2 + 2 + 2 + 2                           \\
           & = 1 + 1 + 1 + 1 + 1 + 1 + 1 + 2 + 2 + 4                               \\
           & = 1 + 1 + 1 + 1 + 1 + 1 + 1 + 2 + 3 + 3                               \\
           & = 1 + 1 + 1 + 1 + 1 + 1 + 1 + 2 + 6                                   \\
           & = 1 + 1 + 1 + 1 + 1 + 1 + 1 + 3 + 5                                   \\
           & = 1 + 1 + 1 + 1 + 1 + 1 + 1 + 4 + 4                                   \\
           & = 1 + 1 + 1 + 1 + 1 + 1 + 2 + 2 + 2 + 3                               \\
           & = 1 + 1 + 1 + 1 + 1 + 1 + 2 + 2 + 5                                   \\
           & = 1 + 1 + 1 + 1 + 1 + 1 + 2 + 3 + 4                                   \\
           & = 1 + 1 + 1 + 1 + 1 + 1 + 2 + 7                                       \\
           & = 1 + 1 + 1 + 1 + 1 + 1 + 3 + 3 + 3                                   \\
           & = 1 + 1 + 1 + 1 + 1 + 1 + 3 + 6                                       \\
           & = 1 + 1 + 1 + 1 + 1 + 1 + 4 + 5                                       \\
           & = 1 + 1 + 1 + 1 + 1 + 2 + 2 + 2 + 2 + 2                               \\
           & = 1 + 1 + 1 + 1 + 1 + 2 + 2 + 2 + 4                                   \\
           & = 1 + 1 + 1 + 1 + 1 + 2 + 2 + 6                                       \\
           & = 1 + 1 + 1 + 1 + 1 + 2 + 2 + 3 + 3                                   \\
           & = 1 + 1 + 1 + 1 + 1 + 2 + 3 + 5                                       \\
           & = 1 + 1 + 1 + 1 + 1 + 2 + 8                                           \\
           & = 1 + 1 + 1 + 1 + 1 + 2 + 4 + 4                                       \\
           & = 1 + 1 + 1 + 1 + 1 + 3 + 3 + 4                                       \\
           & = 1 + 1 + 1 + 1 + 1 + 3 + 7                                           \\
           & = 1 + 1 + 1 + 1 + 1 + 4 + 6                                           \\
           & = 1 + 1 + 1 + 1 + 1 + 5 + 5                                           \\
           & = 1 + 1 + 1 + 1 + 2 + 2 + 2 + 2 + 3                                   \\
           & = 1 + 1 + 1 + 1 + 2 + 2 + 2 + 5                                       \\
           & = 1 + 1 + 1 + 1 + 2 + 2 + 3 + 4                                       \\
           & = 1 + 1 + 1 + 1 + 2 + 2 + 7                                           \\
           & = 1 + 1 + 1 + 1 + 2 + 3 + 3 + 3                                       \\
           & = 1 + 1 + 1 + 1 + 2 + 3 + 6                                           \\
           & = 1 + 1 + 1 + 1 + 2 + 4 + 5                                           \\
           & = 1 + 1 + 1 + 1 + 2 + 9                                               \\
           & = 1 + 1 + 1 + 1 + 3 + 3 + 5                                           \\
           & = 1 + 1 + 1 + 1 + 3 + 4 + 4                                           \\
           & = 1 + 1 + 1 + 1 + 3 + 8                                               \\
           & = 1 + 1 + 1 + 1 + 4 + 7                                               \\
           & = 1 + 1 + 1 + 1 + 5 + 6                                               \\
           & = 1 + 1 + 1 + 2 + 10                                                  \\
           & = 1 + 1 + 1 + 2 + 2 + 8                                               \\
           & = 1 + 1 + 1 + 2 + 2 + 2 + 6                                           \\
           & = 1 + 1 + 1 + 2 + 2 + 2 + 2 + 4                                       \\
           & = 1 + 1 + 1 + 2 + 2 + 2 + 3 + 3                                       \\
           & = 1 + 1 + 1 + 2 + 2 + 2 + 2 + 2 + 2                                   \\
           & = 1 + 1 + 1 + 2 + 2 + 3 + 5                                           \\
           & = 1 + 1 + 1 + 2 + 2 + 4 + 4                                           \\
           & = 1 + 1 + 1 + 2 + 3 + 3 + 4                                           \\
           & = 1 + 1 + 1 + 2 + 3 + 7                                               \\
           & = 1 + 1 + 1 + 2 + 4 + 6                                               \\
           & = 1 + 1 + 1 + 2 + 5 + 5                                               \\
           & = 1 + 1 + 1 + 3 + 9                                                   \\
           & = 1 + 1 + 1 + 3 + 3 + 6                                               \\
           & = 1 + 1 + 1 + 3 + 3 + 3 + 3                                           \\
           & = 1 + 1 + 1 + 3 + 4 + 5                                               \\
           & = 1 + 1 + 1 + 4 + 8                                                   \\
           & = 1 + 1 + 1 + 4 + 4 + 4                                               \\
           & = 1 + 1 + 1 + 5 + 7                                                   \\
           & = 1 + 1 + 1 + 6 + 6                                                   \\
           & = 1 + 1 + 2 + 11                                                      \\
           & = 1 + 1 + 2 + 2 + 9                                                   \\
           & = 1 + 1 + 2 + 2 + 2 + 7                                               \\
           & = 1 + 1 + 2 + 2 + 2 + 2 + 5                                           \\
           & = 1 + 1 + 2 + 2 + 2 + 2 + 2 + 3                                       \\
           & = 1 + 1 + 2 + 2 + 2 + 3 + 4
           & = 1 + 1 + 2 + 2 + 3 + 3 + 3                                           \\
           & = 1 + 1 + 2 + 2 + 3 + 6                                               \\
           & = 1 + 1 + 2 + 2 + 4 + 5                                               \\
           & = 1 + 1 + 2 + 3 + 3 + 5                                               \\
           & = 1 + 1 + 2 + 3 + 4 + 4                                               \\
           & = 1 + 1 + 2 + 3 + 8                                                   \\
           & = 1 + 1 + 2 + 4 + 7                                                   \\
           & = 1 + 1 + 2 + 5 + 6                                                   \\
           & = 1 + 1 + 3 + 3 + 7                                                   \\
           & = 1 + 1 + 3 + 3 + 3 + 4                                               \\
           & = 1 + 1 + 3 + 4 + 6                                                   \\
           & = 1 + 1 + 3 + 5 + 5                                                   \\
           & = 1 + 1 + 3 + 10                                                      \\
           & = 1 + 1 + 4 + 9                                                       \\
           & = 1 + 1 + 4 + 4 + 5                                                   \\
           & = 1 + 1 + 5 + 8                                                       \\
           & = 1 + 1 + 6 + 7                                                       \\
           & = 1 + 2 + 12                                                          \\
           & = 1 + 2 + 2 + 10                                                      \\
           & = 1 + 2 + 2 + 2 + 8                                                   \\
           & = 1 + 2 + 2 + 2 + 2 + 6                                               \\
           & = 1 + 2 + 2 + 2 + 2 + 2 + 4                                           \\
           & = 1 + 2 + 2 + 2 + 2 + 2 + 2 + 2                                       \\
           & = 1 + 2 + 2 + 2 + 2 + 3 + 3                                           \\
           & = 1 + 2 + 2 + 2 + 3 + 5                                               \\
           & = 1 + 2 + 2 + 2 + 4 + 4                                               \\
           & = 1 + 2 + 2 + 3 + 3 + 4                                               \\
           & = 1 + 2 + 2 + 3 + 7                                                   \\
           & = 1 + 2 + 2 + 4 + 6                                                   \\
           & = 1 + 2 + 2 + 5 + 5                                                   \\
           & = 1 + 2 + 3 + 3 + 3 + 3                                               \\
           & = 1 + 2 + 3 + 3 + 6                                                   \\
           & = 1 + 2 + 3 + 4 + 5                                                   \\
           & = 1 + 2 + 3 + 9                                                       \\
           & = 1 + 2 + 4 + 4 + 4                                                   \\
           & = 1 + 2 + 4 + 8                                                       \\
           & = 1 + 2 + 5 + 7                                                       \\
           & = 1 + 2 + 6 + 6                                                       \\
           & = 1 + 3 + 11                                                          \\
           & = 1 + 3 + 3 + 3 + 5                                                   \\
           & = 1 + 3 + 3 + 4 + 4                                                   \\
           & = 1 + 3 + 3 + 8                                                       \\
           & = 1 + 3 + 4 + 7                                                       \\
           & = 1 + 3 + 5 + 6                                                       \\
           & = 1 + 4 + 4 + 6                                                       \\
           & = 1 + 4 + 5 + 5                                                       \\
           & = 1 + 4 + 10                                                          \\
           & = 1 + 5 + 9                                                           \\
           & = 1 + 6 + 8                                                           \\
           & = 1 + 7 + 7                                                           \\
           & = 2 + 2 + 2 + 2 + 2 + 2 + 3                                           \\
           & = 2 + 2 + 2 + 2 + 2 + 5                                               \\
           & = 2 + 2 + 2 + 2 + 3 + 4                                               \\
           & = 2 + 2 + 2 + 2 + 7                                                   \\
           & = 2 + 2 + 2 + 3 + 3 + 3                                               \\
           & = 2 + 2 + 2 + 3 + 6                                                   \\
           & = 2 + 2 + 2 + 4 + 5                                                   \\
           & = 2 + 2 + 2 + 9                                                       \\
           & = 2 + 2 + 3 + 3 + 5                                                   \\
           & = 2 + 2 + 3 + 4 + 4                                                   \\
           & = 2 + 2 + 3 + 8                                                       \\
           & = 2 + 2 + 4 + 7                                                       \\
           & = 2 + 2 + 5 + 6                                                       \\
           & = 2 + 2 + 11                                                          \\
           & = 2 + 3 + 3 + 3 + 4                                                   \\
           & = 2 + 3 + 3 + 7                                                       \\
           & = 2 + 3 + 4 + 6                                                       \\
           & = 2 + 3 + 5 + 5                                                       \\
           & = 2 + 3 + 10                                                          \\
           & = 2 + 4 + 4 + 5                                                       \\
           & = 2 + 4 + 9                                                           \\
           & = 2 + 5 + 8                                                           \\
           & = 2 + 6 + 7                                                           \\
           & = 2 + 13                                                              \\
           & = 3 + 3 + 3 + 3 + 3                                                   \\
           & = 3 + 3 + 3 + 6                                                       \\
           & = 3 + 3 + 4 + 5                                                       \\
           & = 3 + 3 + 9                                                           \\
           & = 3 + 4 + 4 + 4                                                       \\
           & = 3 + 4 + 8                                                           \\
           & = 3 + 5 + 7                                                 & (\star) \\
           & = 3 + 6 + 6                                                           \\
           & = 3 + 12                                                              \\
           & = 4 + 4 + 7                                                           \\
           & = 4 + 5 + 6                                                           \\
           & = 4 + 11                                                              \\
           & = 5 + 5 + 5                                                           \\
           & = 5 + 10                                                              \\
           & = 6 + 9                                                               \\
           & = 7 + 8
    \end{align*}
    \endgroup

    Hence the maximum possible order of an element of $S_{15}$ is $105$.
\end{proof}

In Exercises 31 through 33, find all subgroups of the given group, and draw the subgroup diagram for the sub-groups.

\newpage
% section 6/exercise 31
\begin{exercise}
    $\mathbb{Z}_{12}$
\end{exercise}

\begin{proof}
    \begin{tikzpicture}[every node/.style={node distance=2cm}]
        \node (g1) at (0, 0) {$\anglebracket{1}$};
        \node [below left of=g1] (g2) {$\anglebracket{2}$};
        \node [below right of=g1] (g3) {$\anglebracket{3}$};
        \node [below left of=g2] (g4) {$\anglebracket{4}$};
        \node [below right of=g2] (g6) {$\anglebracket{6}$};
        \node [below of=g6] (g0) {$\anglebracket{0}$};
        \draw (g1) -- (g2);
        \draw (g1) -- (g3);
        \draw (g2) -- (g4);
        \draw (g2) -- (g6);
        \draw (g3) -- (g6);
        \draw (g4) -- (g0);
        \draw (g6) -- (g0);
    \end{tikzpicture}
\end{proof}

\newpage
% section 6/exercise 32
\begin{exercise}
    $\mathbb{Z}_{36}$
\end{exercise}

\begin{proof}
    \begin{tikzpicture}[every node/.style={node distance=2cm}]
        \node (g1) at (0, 0) {$\anglebracket{1}$};
        \node [below left of=g1] (g2) {$\anglebracket{2}$};
        \node [below right of=g1] (g3) {$\anglebracket{3}$};
        \node [below left of=g2] (g4) {$\anglebracket{4}$};
        \node [below right of=g2] (g6) {$\anglebracket{6}$};
        \node [below right of=g3] (g9) {$\anglebracket{9}$};
        \node [below left of=g6] (g12) {$\anglebracket{12}$};
        \node [below left of=g9] (g18) {$\anglebracket{18}$};
        \node [below left of=g18] (g0) {$\anglebracket{0}$};
        \draw (g1) -- (g2);
        \draw (g1) -- (g3);
        \draw (g2) -- (g4);
        \draw (g2) -- (g6);
        \draw (g3) -- (g6);
        \draw (g3) -- (g9);
        \draw (g4) -- (g12);
        \draw (g6) -- (g12);
        \draw (g6) -- (g18);
        \draw (g9) -- (g18);
        \draw (g12) -- (g0);
        \draw (g18) -- (g0);
    \end{tikzpicture}
\end{proof}

\newpage
% section 6/exercise 33
\begin{exercise}
    $\mathbb{Z}_{8}$
\end{exercise}

\begin{proof}
    \begin{tikzpicture}[every node/.style={node distance=1cm}]
        \node (g1) at (0, 0) {$\anglebracket{1}$};
        \node [below of=g1] (g2) {$\anglebracket{2}$};
        \node [below of=g2] (g4) {$\anglebracket{4}$};
        \node [below of=g4] (g0) {$\anglebracket{0}$};
        \draw (g1) -- (g2) -- (g4) -- (g0);
    \end{tikzpicture}
\end{proof}

In Exercises 34 through 38, find all orders of subgroups of the given group.

Every subgroups of a finite cyclic group with $n$ elements is another subgroup with $n/d$ elements, where $d$ is a divisor of $n$.

\newpage
% section 6/exercise 34
\begin{exercise}
    $\mathbb{Z}_{6}$
\end{exercise}

\begin{proof}
    The orders of all subgroups of $\mathbb{Z}_{6}$ are $1, 2, 3, 6$.
\end{proof}

\newpage
% section 6/exercise 35
\begin{exercise}
    $\mathbb{Z}_{8}$
\end{exercise}

\begin{proof}
    The orders of all subgroups of $\mathbb{Z}_{8}$ are $1, 2, 4, 8$.
\end{proof}

\newpage
% section 6/exercise 36
\begin{exercise}
    $\mathbb{Z}_{12}$
\end{exercise}

\begin{proof}
    The orders of all subgroups of $\mathbb{Z}_{12}$ are $1, 2, 3, 4, 6, 12$.
\end{proof}

\newpage
% section 6/exercise 37
\begin{exercise}
    $\mathbb{Z}_{20}$
\end{exercise}

\begin{proof}
    The orders of all subgroups of $\mathbb{Z}_{20}$ are $1, 2, 4, 5, 10, 20$.
\end{proof}

\newpage
% section 6/exercise 38
\begin{exercise}
    $\mathbb{Z}_{17}$
\end{exercise}

\begin{proof}
    The orders of all subgroups of $\mathbb{Z}_{17}$ are $1, 17$.
\end{proof}

\subsection*{Concepts}

In Exercises 39 and 40, correct the definition of the italicized term without reference to the text, if correction is needed, so that it is in a form acceptable for publication.

\newpage
% section 6/exercise 39
\begin{exercise}
    An element a of a group $G$ has order $n\in\mathbb{Z}^{+}$ if and only if $a^{n} = e$.
\end{exercise}

\begin{proof}
    Correction:  An element a of a group $G$ has order $n\in\mathbb{Z}^{+}$ if and only if $n$ is the smallest positive integer such that $a^{n} = e$.
\end{proof}

\newpage
% section 6/exercise 40
\begin{exercise}
    The \textit{greatest common divisor} of two positive integers is the largest positive integer that divides both of them.
\end{exercise}

\begin{proof}
    The definition is correct.
\end{proof}

\newpage
% section 6/exercise 41
\begin{exercise}
    Determine whether each of the following is true or false.
    \begin{enumerate}[label={\textbf{\alph*.}}]
        \item Every cyclic group is abelian.
        \item Every abelian group is cyclic.
        \item $\mathbb{Q}$ under addition is a cyclic group.
        \item Every element of every cyclic group generates the group.
        \item There is at least one abelian group of every finite order $> 0$.
        \item Every group of order $\leq 4$ is cyclic.
        \item All generators of $\mathbb{Z}_{20}$ are prime numbers.
        \item If $G$ and $G'$ are groups, then $G\cap G'$ is a group.
        \item If $H$ and $K$ are subgroups of a group $G$, then $H\cap K$ is a group.
        \item Every cyclic group of order $> 2$ has at least two distinct generators.
    \end{enumerate}
\end{exercise}

\begin{proof}
    \begin{enumerate}[label={\textbf{\alph*.}}]
        \item True.
        \item False. For example: Klein 4-group.
        \item False. Because there is no generator.
        \item False.
        \item True. The order of $\mathbb{Z}_{n}$ is $n$.
        \item False. For example: Klein 4-group.
        \item False. For example: $\anglebracket{9} = \mathbb{Z}_{20}$.
        \item False. For example: $S_{n} \cap \mathbb{Z} = \varnothing$.
        \item True.
        \item True.
    \end{enumerate}
\end{proof}

In Exercises 42 through 46, either give an example of a group with the property described, or explain why no example exists.

\newpage
% section 6/exercise 42
\begin{exercise}
    A finite abelian group that is not cyclic
\end{exercise}

\begin{proof}
    The Klein 4-group is abelian but not cyclic.
\end{proof}

\newpage
% section 6/exercise 43
\begin{exercise}
    An infinite group that is not cyclic
\end{exercise}

\begin{proof}
    $\mathbb{Q}$ with addition is not cyclic.
\end{proof}

\newpage
% section 6/exercise 44
\begin{exercise}
    A cyclic group having only one generator
\end{exercise}

\begin{proof}
    $\mathbb{Z}_{2}$ has only one generator, which is $1$.
\end{proof}

\newpage
% section 6/exercise 45
\begin{exercise}
    An infinite cyclic group having four generators
\end{exercise}

\begin{proof}
    An infinite cyclic group is isomorphic to $\mathbb{Z}$ with addition. But $\mathbb{Z}$ has only $2$ generators, which are $1$ and $-1$. Therefore, an infinite cyclic group cannot have four generators.
\end{proof}

\newpage
% section 6/exercise 46
\begin{exercise}
    A finite cyclic group having four generators
\end{exercise}

\begin{proof}
    $\mathbb{Z}_{5}$ under addition has exactly four generators. $\anglebracket{1} = \anglebracket{2} = \anglebracket{3} = \anglebracket{4} = \mathbb{Z}_{5}$.
\end{proof}

The generators of the cyclic multiplicative group $U_{n}$ of all $n$th roots of unity in $\mathbb{C}$ are the \textbf{primitive $n$th roots of unity}. In Exercises 47 through 50, find the primitive nth roots of unity for the given value of $n$.

$U_{n}$ and $\mathbb{Z}_{n}$ are isomorphic. An isomorphism $\phi: U_{n} \to \mathbb{Z}_{n}$ is
\[
    \phi: e^{i(2k\pi/n)} \mapsto k
\]

where $0\leq k < n$.

$k$ is a generator of $\mathbb{Z}_{n}$ iff $\text{gcd}(n, k) = 1$. $e^{i(2k\pi/n)}$ is a generator of $U_{n}$ iff $k$ is a generator of $\mathbb{Z}_{n}$.

\newpage
% section 6/exercise 47
\begin{exercise}
    $n = 4$
\end{exercise}

\begin{proof}
    The primitive $4$th roots of unity are
    \[
        i = e^{i(2\pi/4)},\qquad -i = e^{i(6\pi/4)}
    \]
\end{proof}

\newpage
% section 6/exercise 48
\begin{exercise}
    $n = 6$
\end{exercise}

\begin{proof}
    The primitive $6$th roots of unity are
    \[
        \frac{1}{2} + \frac{\sqrt{3}}{2}i = e^{i(2\pi/6)},\qquad \frac{1}{2} + \frac{-\sqrt{3}}{2}i = e^{i(10\pi/6)}
    \]
\end{proof}

\newpage
% section 6/exercise 49
\begin{exercise}
    $n = 8$
\end{exercise}

\begin{proof}
    The primitive $8$th roots of unity are
    \[
        e^{i(2\pi/8)} = \frac{1 + i}{\sqrt{2}},\qquad e^{i(6\pi/8)} = \frac{-1+i}{\sqrt{2}},\qquad e^{i(10\pi/8)} = \frac{-1-i}{\sqrt{2}},\qquad e^{i(14\pi/8)} = \frac{1-i}{\sqrt{2}}
    \]
\end{proof}

\newpage
% section 6/exercise 50
\begin{exercise}
    $n = 12$
\end{exercise}

\begin{proof}
    The primitive $12$th roots of unity are
    \[
        \begin{split}
            e^{i(2\pi/12)} = \frac{\sqrt{3} + i}{2} \qquad e^{i(10\pi/12)} = \frac{-\sqrt{3} + i}{2}\qquad e^{i(14\pi/12)} = \frac{-\sqrt{3} - i}{2} \qquad e^{i(22\pi/12)} = \frac{\sqrt{3} - i}{2}
        \end{split}
    \]
\end{proof}

\subsection*{Proof Synopsis}

\newpage
% section 6/exercise 51
\begin{exercise}
    Give a one-sentence synopsis of the proof of Theorem 6.1: Every cyclic group is abelian.
\end{exercise}

\begin{proof}
    Every cyclic group is abelian because every two powers of the generator are commutative.
\end{proof}

\newpage
% section 6/exercise 52
\begin{exercise}
    Give at most a three-sentence synopsis of the proof of Theorem 6.6: A subgroup of a cyclic group is cyclic.
\end{exercise}

\begin{proof}
    If the subgroup is the trivial group, then it is cyclic. Otherwise, let $m$ be the smallest positive integer such that the generator to the power of $m$ is in the subgroup. We prove this power generates the subgroup by using the division algorithm.
\end{proof}

\subsection*{Theory}

\newpage
% section 6/exercise 53
\begin{exercise}
    Let $G$ be a cyclic group with generator $a$, and let $G'$ be a group isomorphic to $G$. If $\phi: G \to G'$ is an isomorphism, show that, for every $x\in G$, $\phi(x)$ is completely determined by the value $\phi(a)$. That is, if $\phi: G \to G'$ and $\psi: G \to G'$ are two isomorphisms such that $\phi(a) = \psi(a)$, then $\phi(x) = \psi(x)$ for all $x\in G$.
\end{exercise}

\begin{proof}
    First, we prove that $G'$ is also a cyclic group. Let $x'$ be an element of $G'$. Because $\phi: G \to G'$ is a one-to-one and onto mapping, there exists $x\in G$ such that $x' = \phi(x)$. $G$ is cyclic, it follows that there exists $n\in\mathbb{Z}$ such that $a^{n} = x$. $x' = \phi(a^{n}) = {[\phi(a)]}^{n}$. So $\phi(a)$ generates $G'$, which means $G'$ is cyclic.

    Let $x$ be an element of $G$. Because $G$ is cyclic, there exists an integer $n$ such that $x = a^{n}$. $\phi(a) = \psi(a)$ implies that ${[\phi(a)]}^{n} = {[\psi(a)]}^{n}$, that is, $\phi(x) = \phi(a^{n}) = {[\phi(a)]}^{n} = {[\psi(a)]}^{n} = \psi(a^{n}) = \psi(x)$. So $\phi(x) = \psi(x)$ for all $x\in G$. Hence $\phi$ and $\psi$ are identical.

    Thus $\phi$ is completely determined by the value $\phi(a)$.
\end{proof}

\newpage
% section 6/exercise 54
\begin{exercise}
    Let $r$ and $s$ be integers. Show that $\{ nr + ms \mid n, m \in \mathbb{Z} \}$ is a subgroup of $\mathbb{Z}$.
\end{exercise}

\begin{proof}
    Let $x_{1}, y_{1}$ be elements of $\{ nr + ms \mid n, m \in \mathbb{Z} \}$. So there exist integers $n_{1}, m_{1}, n_{2}, m_{2}$ such that $x_{1} = n_{1}r + m_{1}s$, $x_{2} = n_{2}r + m_{2}s$. $x_{1} + x_{2} = (n_{1} + n_{2})r + (m_{1} + m_{2})s$. Therefore the given set is closed under addition.

    $0 = 0\cdot r + 0\cdot s$, so the given set contains $0$.

    $-x_{1} = (-n_{1})r + (-m_{1})s$ is also an element of the given set. Therefore, the given set contains the inverse of each of its elements.

    Thus $\{ nr + ms \mid n, m \in \mathbb{Z} \}$ is a subgroup of $\mathbb{Z}$.
\end{proof}

\newpage
% section 6/exercise 55
\begin{exercise}
    Prove that if $G$ is a finite cyclic group, $H$ and $K$ are subgroups of $G$, and $H\ne K$, then $\abs{H} \ne \abs{K}$.
\end{exercise}

\begin{proof}
    I re-prove Theorem 6.15.

    Let $n$ be the number of elements of $G$ and $a, h, k$ be generators of $G, H, K$, respectively. Let $s$ be a positive integer such that $a^{s}\in H$, $t$ a positive integer such that $a^{t} = k$. $s \ne t$, since if $s = t$, then $H = K$, which contrary to the hypothesis.

    Let $m$ be the smallest positive integer such that $h^{m} = e$, then $a^{sm} = e$. Since the order of $a$ is $n$, it follows that $n$ divides $sm$. Let $d_{s}$ be the greatest common divisor of $n$ and $s$, then $n/d_{s}, s/d_{s}$ are relatively prime and $n/d_{s}$ divides $m(s/d_{s})$. Together with $n/d_{s}, s/d_{s}$ being relatively prime, we conclude that $n/d_{s}$ divides $m$. So the smallest positive integer $m$ such that $h^{m} = e$ is $n/d_{s}$. In other words, the order of $H$ is $n/d_{s}$.
    \[
        H = \{ e, \mathbf{a^{s}}, a^{2s}, \ldots, a^{(n/d_{s} - 1)s} \}
    \]

    Similarly, the order of $K$ is $n/d_{t}$, where $d_{t}$ is the greatest common divisor of $n$ and $t$.
    \[
        K = \{ e, \mathbf{a^{t}}, a^{2t}, \ldots, a^{(n/d_{t} - 1)t} \}
    \]

    Assume that $\abs{H} = \abs{K}$, then $n/d_{s} = n/d_{t}$, from which we obtain that $d_{s} = d_{t} = d$. Every element of $H$ is of the form $a^{xs} = a^{xs + yn}$. On the other hand, due to Bezout's theorem, there exist integers $x_{0}, y_{0}$ such that $x_{0}s + y_{0}n = d$, $d$ is also the smallest positive integer in the set $\{ xs + yn \mid x, y\in\mathbb{Z} \}$ (also the set of multiple of $d$). Therefore, $a^{d}$ generates $H$. Similarly, $a^{d}$ generates $K$. So $H = K$, which contrary to the hypothesis.

    Thus $\abs{H} \ne \abs{K}$.
\end{proof}

\newpage
% section 6/exercise 56
\begin{exercise}
    Let $a$ and $b$ be elements of a group $G$. Show that if $ab$ has finite order $n$, then $ba$ also has order $n$.
\end{exercise}

\begin{proof}
    $ba = bae = babb^{-1}$. We prove that ${(ba)}^{m} = b{(ab)}^{m}b^{-1}$ for positive integer $m$.

    The statement holds for $m = 1$. Assume the statement holds for $m = k$.
    \begin{align*}
        {(ba)}^{k+1} & = (ba){(ba)}^{k}                 \\
                     & = (babb^{-1})(b{(ab)}^{k}b^{-1}) \\
                     & = b(ab)(b^{-1}b){(ab)}^{k}b^{-1} \\
                     & = b(ab){(ab)}^{k}b^{-1}          \\
                     & = b{(ab)}^{k+1}b^{-1}
    \end{align*}

    So the statement also holds for $m = k + 1$. According to the mathematical induction, the statement holds for all positive integers.

    Therefore, ${(ba)}^{n} = b{(ab)}^{n}b^{-1}$, so ${(ba)}^{n} = e$ if and only if ${(ab)}^{n} = e$. Thus, if the order of $ab$ is $n$, then the order of $ba$ is also $n$.
\end{proof}

\newpage
% section 6/exercise 57
\begin{exercise}
    Let $r$ and $s$ be positive integers.
    \begin{enumerate}[label={\textbf{\alph*.}}]
        \item Define the \textbf{least common multiple} of $r$ and $s$ as a generator of a certain cyclic group.
        \item Under what condition is the least common multiple of $r$ and $s$ their product, $rs$?
        \item Generalizing part (b), show that the product of the greatest common divisor and of the least common multiple of $r$ and $s$ is $rs$.
    \end{enumerate}
\end{exercise}

\begin{proof}
    \begin{enumerate}[label={\textbf{\alph*.}}]
        \item The least common multiple of $r$ and $s$ is the positive generator of $r\mathbb{Z}\cap s\mathbb{Z}$.
        \item $rs$ is the common multiple of $r$ and $s$ if and only if $r\mathbb{Z}\cap s\mathbb{Z} = rs\mathbb{Z}$. Therefore, $r \nmid s, 2s, \ldots, (r-1)s$. So for different $a, b$ $1\leq a, b \leq r$, $r$ does not divide $(b - a)s$. So $s, 2s, \ldots, (r-1)s, rs$ leave different remainers when divided by $r$. So there exists $1 \leq h < r$ such that $hs$ leaves remainder $1$ when divided by $r$. Therefore, there exists an integer $k$ such that $hs = 1 + kr$. From this, we deduce that $r$ and $s$ are relatively prime.

              Conversely, $r$ and $s$ are relatively prime. Let $x$ be a common multiple of $r$ and $s$. On the one hand, $x = r\cdot (x/r)$. On the other hand, $s$ divides $x = r\cdot (x/r)$ and $r, s$ are relatively prime, from which we conclude that $s$ divides $x/r$. It follows that $rs$ divides $x$, which means every common multiple of $r$ and $s$ is divisible by $rs$. So $rs$ is the least common multiple of $r$ and $s$.

              Thus the least common multiple of $r$ and $s$ is $rs$ iff $r$ and $s$ are relatively prime.
        \item Let $d$ be the greatest common divisor of $r$ and $s$, then $r/d$ and $s/d$ are relatively prime. Let $x$ be a common multiple of $r$ and $s$. On the one hand $s = (s/d)d$ divides $x = (x/r)(r/d)d$, so $(s/d)$ divides $(x/r)(r/d)$. On the other hand, $s/d$ and $r/d$ are relatively prime, from which we obtain $s/d$ divides $x/r$, it follows that $rs/d$ divides $x$. Hence every common multiple of $r$ and $s$ is also a multiple of $rs/d$, which means $rs/d$ is the least common multiple of $r$ and $s$.

              Thus the product of the greatest common divisor and the least common multiple of $r$ and $s$ is $d\cdot (rs/d) = rs$.
    \end{enumerate}
\end{proof}

\newpage
% section 6/exercise 58
\begin{exercise}
    Show that a group that has only a finite number of subgroups must be a finite group.
\end{exercise}

\begin{proof}
    Let $G$ be a group that has only a finite number of subgroups. $G$ is the union of all of its cyclic subgroups.
    \[
        G = \bigcup_{a\in G} \{ a \} = \bigcup_{a\in G}\anglebracket{a}
    \]

    Together with $G$ has only a finite number of subgroups, it follows that $G$ is the union of a finite number of cyclic subgroups.

    Assume that $G$ has infinite elements, then there exists a cyclic subgroup of $G$ which has infinite element. Let $b$ be a generator of this infinite cyclic subgroup, then $\anglebracket{b}$ has an infinite number of subgroup, which are $\anglebracket{b^{n}}$, where $n$ is any positive integer. This contrary to the hypothesis that $G$ has only a finite number of subgroups, so the assumption is false. Hence $G$ is a finite group.

    Thus a group that has only a finite number of subgroups is a finite group.
\end{proof}

\newpage
% section 6/exercise 59
\begin{exercise}
    Show by a counterexample that the following ``converse'' of Theorem 6.6 is not a theorem: ``If a group $G$ is such that every proper subgroup is cyclic, then $G$ is cyclic.\@''
\end{exercise}

\begin{proof}
    The Klein 4-group $V = \{ e, a, b, c \}$ where $e = a^{2} = b^{2} = c^{2}$, $ab = ba = c, bc = cb = a, ca = ac = b$ has four proper subgroups, which are $\{ e \}$, $\{ e, a \}$, $\{ e, b \}$, $\{ e, c \}$. $V$ is not cyclic because the orders of $e, a, b, c$ are $1, 2, 2, 2$, respectively, which are less than the order of $V$.
\end{proof}

\newpage
% section 6/exercise 60
\begin{exercise}
    Let $G$ be a group and suppose $a\in G$ generates a cyclic subgroup of order $2$ and is the \textit{unique} such element. Show that $ax = xa$ for all $x\in G$.
\end{exercise}

\begin{proof}
    $a$ generates a cyclic subgroup of order $2$ means $a\ne e$ and $a^{2} = e$. Let $x$ be an element of $G$.
    \[
        {(xax^{-1})}^{2} = (xax^{-1})(xax^{-1}) = x(a(x^{-1}x)a)x^{-1} = xa^{2}x^{-1} = xx^{-1} = e
    \]

    Due to the hypothesis, $a$ is the unique element such that $a$ generates a cyclic subgroup of order $2$, so $xax^{-1} = a$. It follows that $ax = xax^{-1}x = xa$.

    Thus $ax = xa$ for all $x\in G$.
\end{proof}

\newpage
% section 6/exercise 61
\begin{exercise}
    Prove that if $G$ is a cyclic group with an odd number of generators, then $G$ has at most two elements.
\end{exercise}

\begin{proof}
    If $G$ is a trivial group, then $G$ has one element, which is only the only generator.

    Otherwise, let $a_{1}, a_{2}, \ldots, a_{2n-1}$ be all generators of $G$. If an element is a generator, then so is its inverse. We map each element in $\{ a_{1}, \ldots, a_{2n-1} \}$ to its inverse, which is also in this set. Since this set has an odd number of generators, then there is at least one element which is mapped to itself. Let such element be $a_{k}$, then $G = \anglebracket{a_{k}} = \{ e, a_{k} \}$. Therefore $G$ has two elements.

    Thus if $G$ is a cyclic group with an odd number of generators, then $G$ has at most two elements.
\end{proof}

\newpage
% section 6/exercise 62
\begin{exercise}
    Let $p$ and $q$ be distinct prime numbers. Find the number of generators of the cyclic group $\mathbb{Z}_{pq}$.
\end{exercise}

\begin{proof}
    An element $x$ is a generator of $\mathbb{Z}_{pq}$ if and only if $x$ and $pq$ are relatively prime. Equivalently $x$ is not divisible by $p$ and $q$. So the number of generators of the cyclic group $\mathbb{Z}_{pq}$ is the same as the number of positive integers less than $pq$ and relatively prime with $pq$.

    Let's consider the set $A = \{ 0, 1, 2, \ldots, pq - 1 \}$. There are $p$ elements of $A$ which are divisible by $q$. There are $q$ elements of $B$ which are divisible by $p$. There is one element which is divisible by $p$ and $q$. So there are $p + q - 1$ elements which are divisible by $p$ or $q$. Therefore, there are $pq - p - q + 1$ elements which are not divisible by $p$ and $q$.

    Hence $\mathbb{Z}_{pq}$ has $pq - p - q + 1$ generators.
\end{proof}

\newpage
% section 6/exercise 63
\begin{exercise}
    Let $p$ be a prime number. Find the number of generators of the cyclic group $\mathbb{Z}_{{p}^{r}}$, where $r$ is an integer $\geq 1$.
\end{exercise}

\begin{proof}
    An element $x$ of $\mathbb{Z}_{{p}^{r}}$ is a generator of $\mathbb{Z}_{{p}^{r}}$ if and only if $x$ and $p^{r}$ are relatively prime. $x$ and $p^{r}$ are relatively prime if and only if $x$ is not divisible by $p$.

    There are $p^{r}/p = p^{r-1}$ elements of $\mathbb{Z}_{{p}^{r}}$ which are divisible by $p$. Thus, the number of generators of $\mathbb{Z}_{{p}^{r}}$ is $p^{r} - p^{r-1}$.
\end{proof}

\newpage
% section 6/exercise 64
\begin{exercise}
    Show that in a finite cyclic group $G$ of order $n$, written multiplicatively, the equation $x^{m} = e$ has exactly $m$ solutions $x$ in $G$ for each positive integers $m$ that divides $n$.
\end{exercise}

\begin{proof}
    Let $a$ be a generator of $G$. If $m$ divides $n$, then $a^{0}, a^{n/m}, a^{2n/m}, \ldots, a^{(m-1)n/m}$ are distinct $m$ solutions to the equation $x^{m} = e$.

    Suppose that $b$ is a solution to the equation $x^{m} = e$. Since $a$ generates $G$, there exists a positive integer $r$ such that $a^{r} = b$. Because $b^{m} = e$, it follows that $a^{rm} = e$. Because the order of $a$ is $n$ ($G$ is a cyclic group of order $n$ and $a$ is one of its generator), we deduce that $n$ divides $rm$. Because $n$ also divides $m$, it follows that $n/m$ divides $r$. In other words, $r$ is a multiple of $n/m$. Therefore, $b$ is one of the listed $m$ solutions above.

    Thus, if $m$ divides $n$, the equation $x^{m} = e$ has exactly $m$ solutions $x$ in $G$.
\end{proof}

\newpage
% section 6/exercise 65
\begin{exercise}
    With reference to Exercise 64, what is the situation if $1 < m < n$ and $m$ does not divide $n$?
\end{exercise}

\begin{proof}
    The result does not hold if $1 < m < n$ and $m$ does no divide $n$.

    Example: In cyclic group $\mathbb{Z}_{3}$, the equation $2x = 1$ has only one solution $x = 2$.
\end{proof}

\newpage
% section 6/exercise 66
\begin{exercise}
    Show that $\mathbb{Z}_{p}$ has no proper nontrivial subgroups if $p$ is a prime number.
\end{exercise}

\begin{proof}
    Let $G$ be a subgroup of $\mathbb{Z}_{p}$ other than the trivial subgroup. Since subgroup of a cyclic group is also cyclic, we deduce that $G$ is cyclic. Let $x$ be a generator of $G$. Since $x < p$, then $x$ and $p$ are relatively prime, so there exists integers $a, b$ such that $ax + bp = 1$. Therefore, $ax = ax + bp = 1$, so $1\in G$. Because $1$ generates $\mathbb{Z}_{p}$, we conclude that $G = \mathbb{Z}_{p}$. Hence the only subgroups of $\mathbb{Z}_{p}$ are itself and $\anglebracket{0}$.

    Thus $\mathbb{Z}_{p}$ has no proper nontrival subgroups if $p$ is a prime number.
\end{proof}

\newpage
% section 6/exercise 67
\begin{exercise}
    Let $G$ be an abelian group and let $H$ and $K$ be finite cyclic subgroups with $\abs{H} = r$ and $\abs{K} = s$.
    \begin{enumerate}[label={\textbf{\alph*.}}]
        \item Show that if $r$ and $s$ are relatively prime, then $G$ contains a cyclic group of order $rs$.
        \item Generalizing part (a), show that $G$ contains a cyclic subgroup of order the least common multiple of $r$ and $s$.
    \end{enumerate}
\end{exercise}

\begin{proof}
    Let $a$ be a generator of $H$, and $b$ a generator of $K$. $H = \{ e, a, a^{2}, \ldots, a^{r-1} \}$, $K = \{ e, b, b^{2}, \ldots, b^{s-1} \}$.

    Since the order of $H$ and $K$ are $r$, $s$, the following definition of $J$ is equivalent: $J = \{ a^{m}b^{n} \mid 0\leq m < r, 0\leq n < s \}$.

    Notice that, since $G$ is abelian, we can prove by mathematical induction that ${(ab)}^{n} = a^{n}b^{n}$ for all integers $n$.

    \begin{enumerate}[label={\textbf{\alph*.}}]
        \item We prove that $\anglebracket{ab}$ is a cyclic subgroup of order $rs$.

              ${(ab)}^{rs} = a^{rs}b^{rs} = ee = e$. So the order of $ab$ is at most $rs$. Let $k$ be the order of $ab$, then $k$ is the smallest positive integer such that ${(ab)}^{k} = e$.

              $a^{k}b^{k} = {(ab)}^{k} = e$. It follows that $a^{k} = b^{-k}$. $\anglebracket{a^{k}}$ is a cyclic subgroup of $H$, and $\anglebracket{b^{-k}}$ is a cyclic subgroup of $K$. On the other hand, $\anglebracket{a^{k}} = \anglebracket{b^{-k}}$ (because the two generators are identical). The order of these cyclic subgroups is a common divisor of $r$ and $s$ (due to Theorem 6.15). Because $r$ and $s$ are relatively prime (the common divisors are $\pm 1$), it follows that $\anglebracket{a^{k}}$ and $\anglebracket{b^{-k}}$ are trivial groups. So $a^{k} = b^{-k} = e$, from which we deduce that $k$ is divisible by $r$ and $s$. Hence the order of $ab$ is at least $rs$.

              Thus $\anglebracket{ab}$ is a cyclic subgroup of order $rs$.
        \item Let the prime factorization of $r$ and $s$ be
              \[
                  \begin{split}
                      r = {p}_{1}^{r_{1}}\cdots {p}_{m}^{r_{m}} \\
                      s = {p}_{1}^{s_{1}}\cdots {p}_{m}^{s_{m}}
                  \end{split}
              \]

              where we reorder the primes such that there exists a positive integer $t$ where $r_{i}\geq s_{i}$ for $i \leq t$, and $r_{i} < s_{i}$ for $i > t$. The least common multiple of $r$ and $s$ is
              \[
                  ({p}_{1}^{r_{1}}\cdots {p}_{t}^{r_{t}})({p}_{t+1}^{s_{t+1}}\cdots {p}_{m}^{s_{m}})
              \]

              We choose $x = {p}_{t+1}^{r_{t+1}}\cdots {p}_{m}^{r_{m}}$ and $y = {p}_{1}^{s_{1}}\cdots {p}_{t}^{s_{t}}$. The order of $a^{x}$ is ${p}_{1}^{r_{1}}\cdots {p}_{t}^{r_{t}}$, the order of $b^{y}$ is ${p}_{t+1}^{s_{t+1}}\cdots {p}_{m}^{s_{m}}$. These two orders are relatively prime.

              According to part (a), there exists a cyclic subgroup of $G$ of order the least common multiple of $r$ and $s$.
    \end{enumerate}
\end{proof}

\newpage
\section{Generating Sets and Cayley Digraphs}

\subsection*{Computations}

In Exercises 1 through 8, list the elements of the subgroup generated by the given subset.

\newpage
% section 7/exercise 1
\begin{exercise}
    The subset $\{ 2, 3 \}$ of $\mathbb{Z}_{12}$
\end{exercise}

\begin{proof}
    $\{ 0, 1, 2, 3, 4, 5, 6, 7, 8, 9, 10, 11 \}$
\end{proof}

\newpage
% section 7/exercise 2
\begin{exercise}
    The subset $\{ 4, 6 \}$ of $\mathbb{Z}_{12}$
\end{exercise}

\begin{proof}
    $\{ 0, 2, 4, 6, 8, 10 \}$
\end{proof}

\newpage
% section 7/exercise 3
\begin{exercise}
    The subset $\{ 4, 6 \}$ of $\mathbb{Z}_{25}$
\end{exercise}

\begin{proof}
    $\{ 0, 1, 2, 3, 4, 5, 6, 7, 8, 9, 10, 11, 12, 13, 14, 15, 16, 17, 18, 19, 20, 21, 22, 23, 24 \}$
\end{proof}

\newpage
% section 7/exercise 4
\begin{exercise}
    The subset $\{ 12, 30 \}$ of $\mathbb{Z}_{36}$
\end{exercise}

\begin{proof}
    $\{ 0, 6, 12, 18, 24, 30 \}$
\end{proof}

\newpage
% section 7/exercise 5
\begin{exercise}
    The subset $\{ 12, 42 \}$ of $\mathbb{Z}$
\end{exercise}

\begin{proof}
    $6\mathbb{Z}$
\end{proof}

\newpage
% section 7/exercise 6
\begin{exercise}
    The subset $\{ 18, 24, 39 \}$ of $\mathbb{Z}$
\end{exercise}

\begin{proof}
    $3\mathbb{Z}$
\end{proof}

\newpage
% section 7/exercise 7
\begin{exercise}
    The subset $\{ \mu, \mu\rho^{2} \}$ of $D_{8}$
\end{exercise}

\begin{proof}
    $\{ \iota, \rho^{2}, \rho^{4}, \rho^{6}, \mu, \mu\rho^{2}, \mu\rho^{4}, \mu\rho^{6} \}$
\end{proof}

\newpage
% section 7/exercise 8
\begin{exercise}
    The subset $\{ \rho^{8}, \rho^{10} \}$ of $D_{18}$
\end{exercise}

\begin{proof}
    $\{ \iota, \rho^{2}, \rho^{4}, \rho^{6}, \rho^{8}, \rho^{10}, \rho^{12}, \rho^{14}, \rho^{16} \}$
\end{proof}

\newpage
% section 7/exercise 9
\begin{exercise}
    Use the Cayley digraph in Figure 7.15 to compute these products. Note that the solid edge represent the generator $a$ and the dashed lines represent $b$.
    \begin{enumerate}[label={\textbf{\alph*.}}]
        \item ${(ba^{2})}a^{3}$
        \item $(ba)(ba^{3})$
        \item $b(a^{2}b)$
    \end{enumerate}
    \begin{center}
        \begin{tikzpicture}
            \tikzset{generator a path/.style={postaction={decorate,decoration={markings,mark=at position 0.5 with {\arrow{>}}}}}}
            \tikzset{generator b path/.style={dashed}}

            \tikzmath{%
                \outerordinate = 2;
                \innerordinate = 1;
            }

            \coordinate (e) at (-\outerordinate, \outerordinate);
            \coordinate (a) at (\outerordinate, \outerordinate);
            \coordinate (d) at (\outerordinate, -\outerordinate);
            \coordinate (f) at (-\outerordinate, -\outerordinate);
            \coordinate (b) at (-\innerordinate, \innerordinate);
            \coordinate (c) at (\innerordinate, \innerordinate);
            \coordinate (g) at (\innerordinate, -\innerordinate);
            \coordinate (h) at (-\innerordinate, -\innerordinate);

            \node [above left] at (e) {$e$};
            \node [above right] at (a) {$a$};
            \node [below right] at (d) {$d$};
            \node [below left] at (f) {$f$};
            \node [below right] at (b) {$b$};
            \node [below left] at (c) {$c$};
            \node [above left] at (g) {$g$};
            \node [above right] at (h) {$h$};

            \draw [generator a path] (e) -- (a);
            \draw [generator a path] (a) -- (d);
            \draw [generator a path] (d) -- (f);
            \draw [generator a path] (f) -- (e);
            \draw [generator a path] (b) -- (c);
            \draw [generator a path] (c) -- (g);
            \draw [generator a path] (g) -- (h);
            \draw [generator a path] (h) -- (b);
            \draw [generator b path] (e) -- (b);
            \draw [generator b path] (a) -- (c);
            \draw [generator b path] (d) -- (g);
            \draw [generator b path] (f) -- (h);
        \end{tikzpicture}
    \end{center}
\end{exercise}

\begin{proof}
    \begin{enumerate}[label={\textbf{\alph*.}}]
        \item $(ba^{2})a^{3} = (ca)a^{3} = ga^{3} = ha^{2} = ba = c$.
        \item $(ba)(ba^{3}) = cba^{3} = a^{4} = e$.
        \item $b(a^{2}b) = (ba)(ab) = cab = gb = d$.
    \end{enumerate}
\end{proof}

In Exercises 10 through 12, give the table for the group having the indicated digraph. In each digraph, take $e$ as identity element. List the identity $e$ first in your table, and list the remaining elements alphabetically, so that your answers will be easy to check.
\begin{center}
    \begin{tikzpicture}[dot/.style={circle,fill,minimum size=3pt,inner sep=0pt,outer sep=0pt}]
        \begin{scope}[local bounding box=digraph a]
            \coordinate (b) at (0, 0);
            \coordinate (c) at (1.5, 0);
            \coordinate (a) at (1.5, 1.5);
            \coordinate (e) at (0, 1.5);

            \node [dot,label={left:$b$}] at (b) {};
            \node [dot,label={right:$c$}] at (c) {};
            \node [dot,label={left:$e$}] at (e) {};
            \node [dot,label={right:$a$}] at (a) {};

            \draw [solid] (e) -- (a);
            \draw [solid] (b) -- (c);
            \draw [dashed] (e) -- (b);
            \draw [dashed] (a) -- (c);

            \node at (0.75, -0.75) {(a)};
        \end{scope}
        \begin{scope}[xshift=120pt]
            \coordinate (d) at (0, 0);
            \coordinate (f) at (1, 0);
            \coordinate (c) at (1.5, {sqrt(3)/2});
            \coordinate (a) at (1, {sqrt(3)});
            \coordinate (e) at (0, {sqrt(3)});
            \coordinate (b) at (-0.5, {sqrt(3)/2});

            \node [dot,label={below:$d$}] at (d) {};
            \node [dot,label={below:$f$}] at (f) {};
            \node [dot,label={right:$c$}] at (c) {};
            \node [dot,label={above:$a$}] at (a) {};
            \node [dot,label={above:$e$}] at (e) {};
            \node [dot,label={left:$b$}] at (b) {};

            \draw [solid] (e) -- (a);
            \draw [solid] (c) -- (f);
            \draw [solid] (b) -- (d);
            \draw [dashed] (e) -- (b);
            \draw [dashed] (c) -- (a);
            \draw [dashed] (f) -- (d);

            \node at (0.5, -0.75) {(b)};
        \end{scope}
        \begin{scope}[xshift=200pt]
            \tikzset{arrow path/.style={postaction={decorate,decoration={markings,mark=at position 0.5 with {\arrow{>}}}}}}
            \coordinate (c) at (0, 0);
            \coordinate (a) at (4, 0);
            \coordinate (e) at (2, {2*sqrt(3)});
            \coordinate (d) at (1, {1/sqrt(3)});
            \coordinate (f) at (3, {1/sqrt(3)});
            \coordinate (b) at (2, {4/sqrt(3)});

            \node [dot,label={left:$c$}] at (c) {};
            \node [dot,label={right:$a$}] at (a) {};
            \node [dot,label={above:$e$}] at (e) {};
            \node [dot,label={right:$b$}] at (b) {};
            \node [dot,label={left:$d$}] at (d) {};
            \node [dot,label={right:$f$}] at (f) {};

            \draw [dashed] (e) -- (b);
            \draw [dashed] (c) -- (d);
            \draw [dashed] (a) -- (f);
            \draw [arrow path] (a) -- (c);
            \draw [arrow path] (e) -- (a);
            \draw [arrow path] (c) -- (e);
            \draw [arrow path] (d) -- (f);
            \draw [arrow path] (f) -- (b);
            \draw [arrow path] (b) -- (d);

            \node at (2, -0.75) {(c)};
        \end{scope}
    \end{tikzpicture}
\end{center}

\newpage
% section 7/exercise 10
\begin{exercise}
    The digraph in Fig. 7.16 (a)
\end{exercise}

\begin{proof}
    \[
        \begin{array}{c|cccc}
              & e & a & b & c \\
            \hline
            e & e & a & b & c \\
            a & a & e & c & b \\
            b & b & c & e & a \\
            c & c & b & a & e
        \end{array}
    \]
\end{proof}

\newpage
% section 7/exercise 11
\begin{exercise}
    The digraph in Fig. 7.16 (b)
\end{exercise}

\begin{proof}
    \[
        \begin{array}{c|cccccc}
              & e & a & b & c & d & f \\
            \hline
            e & e & a & b & c & d & f \\
            a & a & e & c & b & f & d \\
            b & b & d & e & f & a & c \\
            c & c & f & a & d & e & b \\
            d & d & b & f & e & c & a \\
            f & f & c & d & a & b & e
        \end{array}
    \]
\end{proof}

\newpage
% section 7/exercise 12
\begin{exercise}
    The digraph in Fig. 7.16 (c)
\end{exercise}

\begin{proof}
    \[
        \begin{array}{c|cccccc}
              & e & a & b & c & d & f \\
            \hline
            e & e & a & b & c & d & f \\
            a & a & c & f & e & b & d \\
            b & b & d & e & f & a & c \\
            c & c & e & d & a & f & b \\
            d & d & f & c & b & e & a \\
            f & f & b & a & d & c & e
        \end{array}
    \]
\end{proof}

\subsection*{Concepts}

\newpage
% section 7/exercise 13
\begin{exercise}
    How can we tell from a Cayley digraph whether or not the corresponding group is commutative.
\end{exercise}

\begin{proof}
    A group is commutative if and only if the generators are commutative.

    A Cayley digraph corresponds to a commutative group if and only if from one fixed vertex, for every pair of arc types $x, y$, after passing through arc of type $x$ then $y$ or type $x$ then $x$, we end up at the same vertex.
\end{proof}

\newpage
% section 7/exercise 14
\begin{exercise}
    Using the condition found in Exercise 13, show that the group corresponding to the Cayley digraph in Figure 7.13 is not commutative. \\
    \begin{center}
        \begin{tikzpicture}
            \tikzset{dot/.style={circle,minimum size=3pt,inner sep=0pt,outer sep=0pt}}
            \tikzset{arrow path/.style={postaction={decorate,decoration={markings,mark=at position 0.5 with {\arrow{>}}}}}}
            \begin{scope}
                \coordinate (e) at (0, 0);
                \coordinate (a) at (4, 0);
                \coordinate (aa) at (4, -4);
                \coordinate (aaa) at (0, -4);
                \coordinate (b) at (1, -1);
                \coordinate (ba) at (1, -3);
                \coordinate (baa) at (3, -3);
                \coordinate (baaa) at (3, -1);

                \node [label={above:$e$},dot] at (e) {};
                \node [label={above:$a$},dot] at (a) {};
                \node [label={right:$a^{2}$},dot] at (aa) {};
                \node [label={left:$a^{3}$},dot] at (aaa) {};
                \node [label={below left:$b$},dot] at (b) {};
                \node [label={above left:$ba$},dot] at (ba) {};
                \node [label={above right:$ba^{2}$},dot] at (baa) {};
                \node [label={below right:$ba^{3}$},dot] at (baaa) {};

                \draw [arrow path] (e) -- (a);
                \draw [arrow path] (a) -- (aa);
                \draw [arrow path] (aa) -- (aaa);
                \draw [arrow path] (aaa) -- (e);
                \draw [arrow path] (b) -- (ba);
                \draw [arrow path] (ba) -- (baa);
                \draw [arrow path] (baa) -- (baaa);
                \draw [arrow path] (baaa) -- (b);
                \draw [dashed] (e) -- (b);
                \draw [dashed] (a) -- (baaa);
                \draw [dashed] (aa) -- (baa);
                \draw [dashed] (aaa) -- (ba);
            \end{scope}
        \end{tikzpicture}
    \end{center}
\end{exercise}

\begin{proof}
    This group is not commutative because $ab = ba^{3} \ne ba$
\end{proof}

\newpage
% section 7/exercise 15
\begin{exercise}
    Is it obvious from a Cayley digraph of a group whether or not the group is cyclic?
\end{exercise}

\begin{proof}
    Yes. If there is an arc type $x$ such that every vertice has an arc of type $x$ maps to it, then the group corresponding to the Cayley digraph is cyclic.
\end{proof}

\newpage
% section 7/exercise 16
\begin{exercise}
    The large outside triangle in Fig. 7.11 (b) exhibits the cyclic subgroup $\{ 0, 2, 4\}$ of $\mathbb{Z}_{6}$. Does the smaller inside triangle similarly exhibit a cyclic subgroup of $\mathbb{Z}_{6}$ Why or why not? \\
    \begin{center}
        \begin{tikzpicture}
            \tikzset{dot/.style={circle,minimum size=3pt,inner sep=0pt,outer sep=0pt}}
            \tikzset{arrow path/.style={postaction={decorate,decoration={markings,mark=at position 0.5 with {\arrow{>}}}}}}
            \coordinate (v4) at (0, 0);
            \coordinate (v2) at (4, 0);
            \coordinate (v0) at (2, {2*sqrt(3)});
            \coordinate (v1) at (1, {1/sqrt(3)});
            \coordinate (v5) at (3, {1/sqrt(3)});
            \coordinate (v3) at (2, {4/sqrt(3)});

            \node [dot,label={left:$4$}] at (v4) {};
            \node [dot,label={right:$2$}] at (v2) {};
            \node [dot,label={above:$0$}] at (v0) {};
            \node [dot,label={below:$5$}] at (v5) {};
            \node [dot,label={below:$1$}] at (v1) {};
            \node [dot,label={left:$3$}] at (v3) {};

            \draw [dashed] (v0) -- (v3);
            \draw [dashed] (v1) -- (v4);
            \draw [dashed] (v2) -- (v5);
            \draw [arrow path] (v0) -- (v2);
            \draw [arrow path] (v2) -- (v4);
            \draw [arrow path] (v4) -- (v0);
            \draw [arrow path] (v1) -- (v3);
            \draw [arrow path] (v3) -- (v5);
            \draw [arrow path] (v5) -- (v1);
        \end{tikzpicture}
    \end{center}
\end{exercise}

\begin{proof}
    The smaller inside triangle does not exhibit a cyclic subgroup of $\mathbb{Z}_{6}$. Because it does not contain the vertice corresponding to the identity element.
\end{proof}

\newpage
% section 7/exercise 17
\begin{exercise}
    The generating set $S = \{ 1, 2 \}$ for $\mathbb{Z}_{6}$ contains more generators than necessary, since $1$ is a generator for the group. Nevertheless, we can draw a Cayley digraph for $\mathbb{Z}_{6}$ with this generating set $S$. Draw such a Cayley digraph.
\end{exercise}

\begin{proof}
    \begin{tikzpicture}
        \tikzset{dot/.style={circle,minimum size=3pt,inner sep=0pt,outer sep=0pt}}
        \tikzset{arc type 1/.style={postaction={decorate,decoration={markings,mark=at position 0.5 with {\arrow{>}}}}}}
        \tikzset{arc type 2/.style={postaction={decorate,decoration={markings,mark=at position 0.5 with {\arrow{>>}}}}}}

        \coordinate (v0) at (2, 0);
        \coordinate (v1) at (1, {sqrt(3)});
        \coordinate (v2) at (-1, {sqrt(3)});
        \coordinate (v3) at (-2, 0);
        \coordinate (v4) at (-1, {-sqrt(3)});
        \coordinate (v5) at (1, {-sqrt(3)});

        \node [dot,label={0:$0$}] at (v0) {};
        \node [dot,label={60:$1$}] at (v1) {};
        \node [dot,label={120:$2$}] at (v2) {};
        \node [dot,label={180:$3$}] at (v3) {};
        \node [dot,label={240:$4$}] at (v4) {};
        \node [dot,label={300:$5$}] at (v5) {};

        \draw [arc type 1] (v0) -- (v1);
        \draw [arc type 1] (v1) -- (v2);
        \draw [arc type 1] (v2) -- (v3);
        \draw [arc type 1] (v3) -- (v4);
        \draw [arc type 1] (v4) -- (v5);
        \draw [arc type 1] (v5) -- (v0);
        \draw [arc type 2] (v0) -- (v2);
        \draw [arc type 2] (v2) -- (v4);
        \draw [arc type 2] (v4) -- (v0);
        \draw [arc type 2] (v1) -- (v3);
        \draw [arc type 2] (v3) -- (v5);
        \draw [arc type 2] (v5) -- (v1);
    \end{tikzpicture}
\end{proof}

\newpage
% section 7/exercise 18
\begin{exercise}
    Draw a Cayley digraph for $\mathbb{Z}_{8}$ with generating set $S = \{ 2, 5 \}$
\end{exercise}

\begin{proof}
    \begin{tikzpicture}
        \tikzset{dot/.style={circle,minimum size=3pt,inner sep=0pt,outer sep=0pt}}
        \tikzset{arc type 1/.style={red,postaction={decorate,decoration={markings,mark=at position 0.5 with {\arrow{>}}}}}}
        \tikzset{arc type 2/.style={blue, postaction={decorate,decoration={markings,mark=at position 0.5 with {\arrow{>>}}}}}}
        \coordinate (v0) at (2, 0);
        \coordinate (v1) at ({sqrt(2)}, {sqrt(2)});
        \coordinate (v2) at (0, 2);
        \coordinate (v3) at ({-sqrt(2)}, {sqrt(2)});
        \coordinate (v4) at (-2, 0);
        \coordinate (v5) at ({-sqrt(2)}, {-sqrt(2)});
        \coordinate (v6) at (0, -2);
        \coordinate (v7) at ({sqrt(2)}, {-sqrt(2)});

        \node [dot,label={0:$0$}] at (v0) {};
        \node [dot,label={45:$1$}] at (v1) {};
        \node [dot,label={90:$2$}] at (v2) {};
        \node [dot,label={135:$3$}] at (v3) {};
        \node [dot,label={180:$4$}] at (v4) {};
        \node [dot,label={225:$5$}] at (v5) {};
        \node [dot,label={270:$6$}] at (v6) {};
        \node [dot,label={315:$7$}] at (v7) {};

        \draw [arc type 1] (v0) -- (v2);
        \draw [arc type 1] (v2) -- (v4);
        \draw [arc type 1] (v4) -- (v6);
        \draw [arc type 1] (v6) -- (v0);
        \draw [arc type 1] (v1) -- (v3);
        \draw [arc type 1] (v3) -- (v5);
        \draw [arc type 1] (v5) -- (v7);
        \draw [arc type 1] (v7) -- (v1);

        \draw [arc type 2] (v0) -- (v5);
        \draw [arc type 2] (v5) -- (v2);
        \draw [arc type 2] (v2) -- (v7);
        \draw [arc type 2] (v7) -- (v4);
        \draw [arc type 2] (v4) -- (v1);
        \draw [arc type 2] (v1) -- (v6);
        \draw [arc type 2] (v6) -- (v3);
        \draw [arc type 2] (v3) -- (v0);
    \end{tikzpicture}
\end{proof}

\newpage
% section 7/exercise 19
\begin{exercise}
    A \textbf{relation} on a set $S$ of generators of a group $G$ is an equation that equates some product of generators and their inverses to the identity $e$ of $G$. For example, if $S = \{ a, b \}$ and $G$ is commutative so that $ab = ba$, then one relation is $aba^{-1}b^{-1} = e$. If, moreover, $b$ is its own inverse, then another relation is $b^{2} = e$.
    \begin{enumerate}[label={\textbf{\alph*.}}]
        \item Explain how we can find some relations on $S$ from a Cayley graph of $S$.
        \item Find three relations on the set $S = \{ a, b \}$ of generators for the group described by Fig. 7.13(b).
    \end{enumerate}
    \begin{center}
        \begin{tikzpicture}
            \tikzset{dot/.style={circle,minimum size=3pt,inner sep=0pt,outer sep=0pt}}
            \tikzset{arrow path/.style={postaction={decorate,decoration={markings,mark=at position 0.5 with {\arrow{>}}}}}}
            \begin{scope}
                \coordinate (e) at (0, 0);
                \coordinate (a) at (4, 0);
                \coordinate (aa) at (4, -4);
                \coordinate (aaa) at (0, -4);
                \coordinate (b) at (1, -1);
                \coordinate (ba) at (1, -3);
                \coordinate (baa) at (3, -3);
                \coordinate (baaa) at (3, -1);

                \node [label={above:$e$},dot] at (e) {};
                \node [label={above:$a$},dot] at (a) {};
                \node [label={right:$a^{2}$},dot] at (aa) {};
                \node [label={left:$a^{3}$},dot] at (aaa) {};
                \node [label={below left:$b$},dot] at (b) {};
                \node [label={above left:$ba$},dot] at (ba) {};
                \node [label={above right:$ba^{2}$},dot] at (baa) {};
                \node [label={below right:$ba^{3}$},dot] at (baaa) {};

                \draw [arrow path] (e) -- (a);
                \draw [arrow path] (a) -- (aa);
                \draw [arrow path] (aa) -- (aaa);
                \draw [arrow path] (aaa) -- (e);
                \draw [arrow path] (b) -- (ba);
                \draw [arrow path] (ba) -- (baa);
                \draw [arrow path] (baa) -- (baaa);
                \draw [arrow path] (baaa) -- (b);
                \draw [dashed] (e) -- (b);
                \draw [dashed] (a) -- (baaa);
                \draw [dashed] (aa) -- (baa);
                \draw [dashed] (aaa) -- (ba);
            \end{scope}
        \end{tikzpicture}
    \end{center}
\end{exercise}

\begin{proof}
    \begin{enumerate}[label={\textbf{\alph*.}}]
        \item From the vertex corresponding to $e$, we move along an arc to get to vertex $a$. Because the Cayley digraph is connected, we can go from $a$ to $e$ via consecutive arcs. This path is a cycle, so it corresponds to an equation that equates some product of generators and to the identity.
        \item $a^{4} = e$, $b^{2} = e$, $abab = e$, $a^{2}ba^{2}b = e$, $a^{3}ba^{3}b = e$, $baba = e$.
    \end{enumerate}
\end{proof}

\newpage
% section 7/exercise 20
\begin{exercise}
    Draw digraphs of the two possible structurally different groups of order 4, taking as small a generating set as possible in each case. You need not label vertices.
\end{exercise}

\begin{proof}
    Belows are the Cayley digraphs of $\mathbb{Z}_{4}$ and the Klein 4-group.
    \begin{center}
        \begin{tikzpicture}
            \begin{scope}
                \tikzset{arc type 1/.style={postaction={decorate,decoration={markings,mark=at position 0.5 with {\arrow{>}}}}}}
                \draw [arc type 1] (0, 0) -- (2, 0);
                \draw [arc type 1] (2, 0) -- (2, 2);
                \draw [arc type 1] (2, 2) -- (0, 2);
                \draw [arc type 1] (0, 2) -- (0, 0);
            \end{scope}
            \begin{scope}[xshift=120pt]
                \draw (0, 0) -- (2, 0);
                \draw [dashed] (2, 0) -- (2, 2);
                \draw (2, 2) -- (0, 2);
                \draw [dashed] (0, 2) -- (0, 0);
            \end{scope}
        \end{tikzpicture}
    \end{center}
\end{proof}

\subsection*{Theory}

\newpage
% section 7/exercise 21
\begin{exercise}
    Use Cayley digraphs to show that for $n \geq 3$, there exists a nonabelian group with $2n$ elements that is generated by two elements of order $2$.
\end{exercise}

\begin{proof}
    We construct such Cayley digraph and label each vertex by the corresponding element of the group.

    Let $G = (V, E)$ be a graph that has $2n$ vertices and $2n$ edges: The vertices are $V_{i}$ where $i = \overline{1,2n}$, the edges are $V_{i}V_{i+1}$ $i = \overline{1,2n}$ (addition modulo $2n$). We mark the edge $V_{2k}V_{2k+1}$ (where $1\le k \le n$) by $a$, and mark the edge $V_{2k-1}V_{2k}$ (where $1\leq k \leq n$) by $b$. This graph is a Cayley digraph.

    $ab$ corresponds to the consecutive arcs $V_{2}V_{3}, V_{3}V_{4}$, $ba$ corresponds to the consecutive arcs $V_{2}V_{1}, V_{1}V_{2n}$. So $ab\ne ba$, from which we conclude that the corresponding group is nonabelian.
\end{proof}

\newpage
% section 7/exercise 22
\begin{exercise}
    Prove that there are at least three different abelian groups of order 8.
\end{exercise}

\begin{proof}
    Belows are Cayley digraphs of three different abelian groups of order 8.

    \begin{center}
        \begin{tikzpicture}[>=stealth,scale=1.5]
            \begin{scope}
                \tikzset{arc type 1/.style={red,postaction={decorate,decoration={markings,mark=at position 0.5 with {\arrow{>}}}}}}
                \coordinate (v0) at (1, 0);
                \coordinate (v1) at ({sqrt(2)/2}, {sqrt(2)/2});
                \coordinate (v2) at (0, 1);
                \coordinate (v3) at ({-sqrt(2)/2}, {sqrt(2)/2});
                \coordinate (v4) at (-1, 0);
                \coordinate (v5) at ({-sqrt(2)/2}, {-sqrt(2)/2});
                \coordinate (v6) at (0, -1);
                \coordinate (v7) at ({sqrt(2)/2}, {-sqrt(2)/2});

                \draw [arc type 1] (v0) -- (v1);
                \draw [arc type 1] (v1) -- (v2);
                \draw [arc type 1] (v2) -- (v3);
                \draw [arc type 1] (v3) -- (v4);
                \draw [arc type 1] (v4) -- (v5);
                \draw [arc type 1] (v5) -- (v6);
                \draw [arc type 1] (v6) -- (v7);
                \draw [arc type 1] (v7) -- (v0);
            \end{scope}
            \begin{scope}[xshift=80pt]
                \tikzset{arc type 1/.style={red,postaction={decorate,decoration={markings,mark=at position 0.5 with {\arrow{>}}}}}}
                \tikzset{arc type 2/.style={blue}}
                \coordinate (v0) at (1, 0);
                \coordinate (v1) at ({sqrt(2)/2}, {sqrt(2)/2});
                \coordinate (v2) at (0, 1);
                \coordinate (v3) at ({-sqrt(2)/2}, {sqrt(2)/2});
                \coordinate (v4) at (-1, 0);
                \coordinate (v5) at ({-sqrt(2)/2}, {-sqrt(2)/2});
                \coordinate (v6) at (0, -1);
                \coordinate (v7) at ({sqrt(2)/2}, {-sqrt(2)/2});

                \draw [arc type 1] (v0) -- (v1);
                \draw [arc type 1] (v1) -- (v4);
                \draw [arc type 1] (v4) -- (v5);
                \draw [arc type 1] (v5) -- (v0);
                \draw [arc type 1] (v2) -- (v3);
                \draw [arc type 1] (v3) -- (v6);
                \draw [arc type 1] (v6) -- (v7);
                \draw [arc type 1] (v7) -- (v2);
                \draw [arc type 2] (v1) -- (v2);
                \draw [arc type 2] (v3) -- (v4);
                \draw [arc type 2] (v5) -- (v6);
                \draw [arc type 2] (v7) -- (v0);
            \end{scope}
            \begin{scope}[xshift=160pt]
                \tikzset{arc type 1/.style={red}}
                \tikzset{arc type 2/.style={blue, dashed}}
                \tikzset{arc type 3/.style={black, dotted, very thick}}
                \coordinate (v0) at (1, 0);
                \coordinate (v1) at ({sqrt(2)/2}, {sqrt(2)/2});
                \coordinate (v2) at (0, 1);
                \coordinate (v3) at ({-sqrt(2)/2}, {sqrt(2)/2});
                \coordinate (v4) at (-1, 0);
                \coordinate (v5) at ({-sqrt(2)/2}, {-sqrt(2)/2});
                \coordinate (v6) at (0, -1);
                \coordinate (v7) at ({sqrt(2)/2}, {-sqrt(2)/2});

                \draw [arc type 1] (v0) -- (v1);
                \draw [arc type 2] (v1) -- (v2);
                \draw [arc type 3] (v2) -- (v3);
                \draw [arc type 1] (v2) -- (v7);
                \draw [arc type 1] (v4) -- (v5);
                \draw [arc type 2] (v3) -- (v4);
                \draw [arc type 3] (v0) -- (v5);
                \draw [arc type 2] (v0) -- (v7);
                \draw [arc type 3] (v6) -- (v7);
                \draw [arc type 2] (v5) -- (v6);
                \draw [arc type 3] (v1) -- (v4);
                \draw [arc type 1] (v3) -- (v6);
            \end{scope}
        \end{tikzpicture}
    \end{center}
\end{proof}
