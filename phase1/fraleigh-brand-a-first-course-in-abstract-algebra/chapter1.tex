\chapter{Groups and Subgroups}
\section{Binary Operations}

\subsection*{Computations}

Exercises 1 through 4 concern the binary operation $*$ defined on $S = \{a, b, c, d, e\}$ by means of Table 1.31 (see the book)

% section 1/exercise 1
\begin{exercise}
    Compute $b * d, c * c$, and $((a * c) * e) * a$.
\end{exercise}

\begin{proof}
    \[
        b * d = e.
    \]
    \[
        ((a * c) * e) * a = (c * e) * a = a * a = a.
    \]
\end{proof}

% section 1/exercise 2
\begin{exercise}
    Compute $(a * b) * c$ and $a * (b * c)$. Can you say on the basis of this computation whether $*$ is associative?
\end{exercise}

\begin{proof}
    \[
        \begin{split}
            (a * b) * c = b * c = a, \\
            a * (b * c) = a * a = a.
        \end{split}
    \]

    On the basis of this computation, it is not sufficient to say $*$ is associative.
\end{proof}

% section 1/exercise 3
\begin{exercise}
    Compute $(b * d) * c$ and $b * (d * c)$. Can you say on the basis of this computation whether $*$ is associative?
\end{exercise}

\begin{proof}
    \[
        \begin{split}
            (b * d) * c = e * c = a, \\
            b * (d * c) = b * b = c.
        \end{split}
    \]

    On the basis of this computation, I can say $*$ is not associative.
\end{proof}

% section 1/exercise 4
\begin{exercise}
    Is $*$ commutative? Why?
\end{exercise}

\begin{proof}
    $*$ is not commutative. Because $e * b = b \ne c = b * e$.
\end{proof}

% section 1/exercise 5
\begin{exercise}
    Complete Table 1.32 so as to define a commutative binary operation $*$ on $S = \{ a, b, c, d \}$.
\end{exercise}

\begin{proof}
    % chktex-file 44
    \begin{tabular}{c|c|c|c|c}
        * & a & b & c & d \\
        \midrule
        a & a & b & c & d \\
        b & b & d & a & c \\
        c & c & a & d & b \\
        d & d & c & b & a
    \end{tabular}
\end{proof}

% section 1/exercise 6
\begin{exercise}
    Table 1.33 can be completed to define an associative binary operation $*$ on $S = \{ a, b, c, d \}$. Assume this is possible and compute the missing entries. Does $S$ have an identity element?
\end{exercise}

\begin{proof}
    % chktex-file 44
    \begin{tabular}{c|c|c|c|c}
        * & a          & b          & c          & d          \\
        \hline
        a & a          & b          & c          & d          \\
        b & b          & c          & a          & d          \\
        c & c          & d          & c          & d          \\
        d & \textbf{d} & \textbf{c} & \textbf{c} & \textbf{d}
    \end{tabular}

    $S$ has an identity element, which is $a$.
\end{proof}

In Exercises 7 through 11, determine whether the binary operation $*$ defined is commutative and whether $*$ is associative.

% section 1/exercise 7
\begin{exercise}
    $*$ defined on $\mathbb{Z}$ by letting $a * b = a - b$
\end{exercise}

\begin{proof}
    $*$ is noncommutative. Because $1 * 2 = 1 - 2 = -1 \ne 1 = 2 - 1 = 2 * 1$.
\end{proof}

% section 1/exercise 8
\begin{exercise}
    $*$ defined on $\mathbb{Q}$ by letting $a * b = 2ab + 3$
\end{exercise}

\begin{proof}
    $*$ is commutative. Because for every two rational numbers $a$ and $b$, $ab + 1\in\mathbb{Q}$ and
    \[
        a * b = 2ab + 3 = 2ba + 3 = b * a.
    \]
\end{proof}

% section 1/exercise 9
\begin{exercise}
    $*$ defined on $\mathbb{Z}$ by letting $a * b = ab + a + b$
\end{exercise}

\begin{proof}
    $*$ is commutative. Because for every two rational numbers $a$ and $b$, $ab + a + b\in\mathbb{Z}$ and
    \[
        a * b = ab + a + b = ba + b + a = b * a
    \]
\end{proof}

% section 1/exercise 10
\begin{exercise}
    $*$ defined on $\mathbb{Z}^{*}$ by letting $a * b = 2^{ab}$
\end{exercise}

\begin{proof}
    $*$ is commutative. Because for every two positive integers $a, b$, $2^{ab}\in\mathbb{Z}^{+}$ and
    \[
        a * b = 2^{ab} = 2^{ba} = b * a.
    \]
\end{proof}

% section 1/exercise 11
\begin{exercise}
    $*$ defined on $\mathbb{Z}^{+}$ by letting $a * b = a^{b}$
\end{exercise}

\begin{proof}
    $*$ is not commutative. Because $1 * 2 = 1^{2} = 1 \ne 2 = 2^{1} = 2 * 1$.
\end{proof}

% section 1/exercise 12
\begin{exercise}
    Let $S$ be a set having exactly one element. How many different binary operations can be defined on $S$? Answer the question if $S$ has exactly $2$ elements; exactly $3$ elements; exactly $n$ elements.
\end{exercise}

\begin{proof}
    If $\card{S} = 1$, there is $1$ binary operations can be defined on $S$.

    If $\card{S} = 2$, $\card{S\times S} = 4$, there are $2^{4}$ binary operations can be defined on $S$.

    If $\card{S} = 3$, $\card{S\times S} = 9$, there are $3^{9}$ binary operations can be defined on $S$.

    If $\card{S} = n$, $\card{S\times S} = n^{2}$, there are $n^{(n^{2})}$ binary operations can be defined on $S$.
\end{proof}

% section 1/exercise 13
\begin{exercise}
    How many different commutative binary operations can be defined on a set of $2$ elements? on a set of $3$ elements? on a set of $n$ elements?
\end{exercise}

\begin{proof}
    If $\card{S} = 1$, there is $1$ binary operations can be defined on $S$ and it is also commutative.

    If $\card{S} = 2$, there are $2^{2(2+1)/2} = 8$ commutative binary operations can be defined on $S$.

    If $\card{S} = 3$, there are $3^{3(3+1)/2} = 729$ commutative binary operations can be defined on $S$.

    If $\card{S} = n$, there are $n^{n(n+1)/2}$ commutative binary operations can be defined on $S$.
\end{proof}

% section 1/exercise 14
\begin{exercise}
    How many different binary operations on a set $S$ with $n$ elements have the property that for all $x\in S, x * x = x$?
\end{exercise}

\begin{proof}
    There are $n^{n(n-1)}$ such binary operations.
\end{proof}

% section 1/exercise 15
\begin{exercise}
    How many different binary operations on a set $S$ with $n$ elements have an identity element?
\end{exercise}

\begin{proof}
    $S = \{ a_{1}, \ldots, a_{n} \}$.

    If $a_{i}$ is the identity element, then there are $n^{{(n-1)}^{2}}$ such binary operations.

    Hence there are $n^{1 + {(n-1)}^{2}}$ such binary operations.
\end{proof}

\subsection*{Concepts}

In Exercises 16 through 19, correct the definition of the italicized term without reference to the text, if correction is needed, so that it is in a form acceptable for publication.

% section 1/exercise 16
\begin{exercise}
    A binary operation $*$ is \textit{commutative} if and only if $a * b = b * a$.
\end{exercise}

\begin{proof}
    Correction: A binary operation $*$ on a set $S$ is \textit{commutative} if and only if $a * b = b * a$ for every two elements $a, b$.
\end{proof}

% section 1/exercise 17
\begin{exercise}
    A binary operation $*$ on a set $S$ is \textit{associative} if and only if, for all $a, b, c\in S$, we have $(b * c) * a = b * (c * a)$.
\end{exercise}

\begin{proof}
    This definition doesn't need correction.
\end{proof}

% section 1/exercise 18
\begin{exercise}
    A subset $H$ of a set $S$ is \textit{closed} under a binary operation $*$ on $S$ if and only if $(a * b)\in H$ for all $a, b\in S$.
\end{exercise}

\begin{proof}
    Correction: A subset $H$ of a set $S$ is \textit{closed} under a binary operation $*$ on $S$ if and only if $(a * b)\in H$ for all $a, b\in H$.
\end{proof}

% section 1/exercise 19
\begin{exercise}
    An identity in the set $S$ with operation $*$ is an element $e\in S$ such that $a * e = e * a = a$.
\end{exercise}

\begin{proof}
    Correction: An identity in the set $S$ with operation $*$ is an element $e\in S$ such that $a * e = e * a = a$ for all $a\in S$.
\end{proof}

% section 1/exercise 20
\begin{exercise}
    Is there an example of a set $S$, a binary operation on $S$, and two different elements $e_{1}, e_{2}\in S$ such that for all $a\in S$, $e_{1} * a = a$ and $a * e_{2} = a$? If so, give an example and if not, prove there is not one.
\end{exercise}

\begin{proof}
    There is no such binary operation.

    $e_{1} * a = a$ for all $a\in S$ so $e_{1} * e_{2} = e_{2}$.

    $a * e_{2} = a$ for all $a\in S$ so $e_{1} * e_{2} = e_{1}$.

    Thus $e_{1} = e_{1} * e_{2} = e_{2}$.
\end{proof}

In Exercises 21 through 26, determine whether the definition of $*$ does give a binary operation on the set. In the event that $*$ is not a binary operation, state whether Condition 1, Condition 2, or both conditions regarding defining binary operations are violated.

% section 1/exercise 21
\begin{exercise}
    On $\mathbb{Z}^{+}$, define $*$ by letting $a * b = a^{b}$.
\end{exercise}

\begin{proof}
    This is a binary operation.
\end{proof}

% section 1/exercise 22
\begin{exercise}
    On $\mathbb{R}^{+}$, define $*$ by letting $a * b = 2a - b$.
\end{exercise}

\begin{proof}
    This is not a binary operation. Condition 2 is violated, since $1 * 2 = 2\cdot 1 - 2 = 2 - 2 = 0\notin\mathbb{R}^{+}$.
\end{proof}

% section 1/exercise 23
\begin{exercise}
    On $\mathbb{R}^{+}$, define $*$ by $a * b$ to be the minimum of $a$ and $b - 1$ if they are different and their common value if they are the same.
\end{exercise}

\begin{proof}
    This is not a binary operation. Condition 2 is violated, since $1 * 1 = 1 - 1 = 0\notin\mathbb{R}^{+}$.
\end{proof}

% section 1/exercise 24
\begin{exercise}
    On $\mathbb{R}$, define $a * b$ to be the number $c$ so that $c^{b} = a$.
\end{exercise}

\begin{proof}
    This is not a binary operation. Both conditions are violated, since
    \begin{itemize}
        \item $c^{0} = 1$ for every $c\ne 0$,
        \item there is no $c\in\mathbb{R}$ such that $c^{1/2} = -1$.
    \end{itemize}
\end{proof}

% section 1/exercise 25
\begin{exercise}
    On $\mathbb{Z}^{+}$, define $*$ letting $a * b = c$, where $c$ is at least $5$ more than $a + b$.
\end{exercise}

\begin{proof}
    This is not a binary operation. Because Condition 1 is violated. $a * b$ can be assigned to $a + b + 5, a + b + 6, \ldots$
\end{proof}

% section 1/exercise 26
\begin{exercise}
    On $\mathbb{Z}^{+}$, define $*$ by letting $a * b = c$, where $c$ is the largest integer less than the product of $a$ and $b$.
\end{exercise}

\begin{proof}
    This is not a binary operation. Because Condition 2 is violated. $1 * 1 = 0\notin\mathbb{Z}^{+}$.
\end{proof}

% section 1/exercise 27
\begin{exercise}
    Let $H$ be the subset of $M_{2}(\mathbb{R})$ consisting of all matrices of the form $\begin{bmatrix}a & -b \\ b & a\end{bmatrix}$ for $a, b\in\mathbb{R}$. Is $H$ closed under
    \begin{enumerate}[label={\textbf{\alph*}}]
        \item matrix addition?
        \item matrix multiplication?
    \end{enumerate}
\end{exercise}

\begin{proof}
    \begin{enumerate}[label={\textbf{\alph*}}]
        \item For any two matrices within $H$
              \[
                  \begin{bmatrix}
                      a & -b \\
                      b & a
                  \end{bmatrix}
                  +
                  \begin{bmatrix}
                      c & -d \\
                      d & c
                  \end{bmatrix}
                  =
                  \begin{bmatrix}
                      a + c & -(b + d) \\
                      b + d & a + c
                  \end{bmatrix}
                  \in H
              \]

              so $H$ is closed under matrix addition.
        \item For any two matrices within $H$
              \[
                  \begin{bmatrix}
                      a & -b \\
                      b & a
                  \end{bmatrix}
                  \cdot
                  \begin{bmatrix}
                      c & -d \\
                      d & c
                  \end{bmatrix}
                  =
                  \begin{bmatrix}
                      ac - bd & -(ad + bc) \\
                      ad + bc & ac - bd
                  \end{bmatrix}
                  \in H
              \]

              so $H$ is closed under matrix multiplication.
    \end{enumerate}
\end{proof}

% section 1/exercise 28
\begin{exercise}
    Mark each of the following true or false.
    \begin{enumerate}[label={\textbf{\alph*.}},itemsep=0pt,topsep=0pt]
        \item If $*$ is any binary operation on any set $S$, then $a * a = a$ for all $a\in S$.
        \item If $*$ is any commutative binary operation on any set $S$, then $a * (b * c) = (b * c) * a$ for all $a, b, c \in S$.
        \item If $*$ is any associative binary operation on any set $S$, then $a * (b * c) = (b * c) * a$ for all $a, b, c \in S$.
        \item The only binary operations of any importance are those defined on sets of numbers.
        \item A binary operation $*$ on a set $S$ is commutative if there exist $a, b \in S$ such that $a * b = b * a$.
        \item Every binary operation defined on a set having exactly one element is both commutative and associative.
        \item A binary operation on a set $S$ assigns at least one element of $S$ to each ordered pair of elements of $S$.
        \item A binary operation on a set $S$ assigns at most one element of $S$ to each ordered pair of elements of $S$.
        \item A binary operation on a set $S$ assigns exactly one element of $S$ to each ordered pair of elements of $S$.
        \item A binary operation on a set $S$ may assign more tha one element of $S$ to some ordered pair of elements of $S$.
        \item For any binary operation $*$ on the set $S$, if $a, b, c\in S$ and $a * b = a * c$, then $b = c$.
        \item For any binary operation $*$ on the set $S$, there is an element $e\in S$ such that for all $x\in S, x * e = x$.
        \item There is an operation on the set $S = \{ e_{1}, e_{2}, a \}$ so that for all $x\in S$, $e_{1} * x = e_{2} * x = x$.
        \item Identity elements are always called $e$.
    \end{enumerate}
\end{exercise}

\begin{proof}
    \begin{enumerate}[label={\textbf{\alph*.}},itemsep=0pt,topsep=0pt]
        \item False. Example: $\mathbb{Z}$ with addition.
        \item True.
        \item False. Example: $M(2,\mathbb{R})$ with multiplication.
              \[
                  \begin{bmatrix}
                      1 & 1 \\
                      0 & 0
                  \end{bmatrix}
                  \cdot
                  \left(
                  \begin{bmatrix}
                          1 & 0 \\
                          0 & 1
                      \end{bmatrix}
                  \cdot
                  \begin{bmatrix}
                          1 & 1 \\
                          1 & 0
                      \end{bmatrix}
                  \right)
                  =
                  \begin{bmatrix}
                      2 & 1 \\
                      0 & 0
                  \end{bmatrix}
                  \ne
                  \begin{bmatrix}
                      1 & 1 \\
                      1 & 1
                  \end{bmatrix}
                  =
                  \left(
                  \begin{bmatrix}
                          1 & 0 \\
                          0 & 1
                      \end{bmatrix}
                  \cdot
                  \begin{bmatrix}
                          1 & 1 \\
                          1 & 0
                      \end{bmatrix}
                  \right)
                  \cdot
                  \begin{bmatrix}
                      1 & 1 \\
                      0 & 0
                  \end{bmatrix}.
              \]
        \item Undeciable, since ``important'' is undefined.
        \item False. Example: $M_{2}(\mathbb{R})$ with multiplication, any $2\times 2$ matrix is commutative with the $2\times 2$ identity matrix, but multiplication of any two $2\times 2$ matrices are not necessarily commutative.
        \item True.
        \item False. Must be exactly one, not at least one.
        \item False. Must be exactly one, not at most one.
        \item True.
        \item False.
        \item False. Counterexample: $\mathbb{Z}$ with usual multiplication, $0 \cdot 1 = 0 \cdot 2$ but $1 \ne 2$.
        \item False. Counterexample: $S = \{ a, b \}$, $a * a = b, a * b = b, b * b = a, b * a = a$.
        \item True. Example:
              \[
                  \begin{array}{c|ccc}
                      *     & e_{1} & e_{2} & a \\
                      \hline
                      e_{1} & e_{1} & e_{2} & a \\
                      e_{2} & e_{1} & e_{2} & a \\
                      a     & a     & a     & a
                  \end{array}
              \]
        \item False. We can called it anything.
    \end{enumerate}
\end{proof}

% section 1/exercise 29
\begin{exercise}
    Give a set different from any of those described in the examples of the text and not a set of numbers. Define two different binary operations $*$ and $*'$ on this set. Be sure that your set is well defined.
\end{exercise}

\begin{proof}
    Let $S$ be a set of two elements $a$ and $b$. Define two binary operations $*$ and $*'$ as follows
    % chktex-file 44
    \begin{tabular}{c|cc}
        * & a & b \\
        \hline
        a & a & a \\
        b & a & a
    \end{tabular}
    % chktex-file 44
    \begin{tabular}{c|cc}
        * & a & b \\
        \hline
        a & a & b \\
        b & b & a
    \end{tabular}
\end{proof}

\subsection*{Theory}

% section 1/exercise 30
\begin{exercise}
    Prove that if $*$ is an associative and commutative binary operation on a set $S$, then
    \[
        (a * b) * (c * d) = ((d * c) * a) * b
    \]

    for all $a, b, c, d\in S$. Assume the associative law only for triples as in the definition, that is, assume only
    \[
        (x * y) * z = x * (y * z)
    \]

    for all $x, y, z\in S$.
\end{exercise}

\begin{proof}
    $d * c = e\in S$.

    \begin{align*}
        (a * b) * (c * d) & = (c * d) * (a * b)  & \text{(commutative)} \\
                          & = (d * c) * (a * b)  & \text{(commutative)} \\
                          & = e * (a * b)                               \\
                          & = (e * a) * b        & \text{(associative)} \\
                          & = ((d * c) * a) * b.
    \end{align*}
\end{proof}

In Exercises 31 and 32, either prove the statement or give a counterexample.

% section 1/exercise 31
\begin{exercise}
    Every binary operation on a set consisting of a single element is both commutative and associative.
\end{exercise}

\begin{proof}
    Let $S = \{ a \}$. There is only one binary operation $*$ can be define on $S$, where $a * a = a$.

    Since $a * a = a * a$ and $(a * a) * a = a * a = a * (a * a)$, $S$ with $*$ is commutative and associative.
\end{proof}

% section 1/exercise 32
\begin{exercise}
    Every commutative binary operation on a set having just two elements is associative.
\end{exercise}

\begin{proof}
    False.

    Counterexample: Let $S = \{ a, b \}$, and $*$ be the commutative binary operation on $S$ as follows
    % chktex-file 44
    \begin{tabular}{c|cc}
        * & a & b \\
        \midrule
        a & b & b \\
        b & b & a
    \end{tabular}

    \[
        (a * a) * b = b * b = a \ne b = a * b = a * (a * b).
    \]
\end{proof}

Let $F$ be the set of all real-valued functions having as domain the set $\mathbb{R}$ of all real numbers. Example 2.7 defined the binary operations $+, -, \cdot$, and $\circ$ on $F$. In Exercises 29 through 35, either prove the given statement or give a
counterexample.

% section 1/exercise 33
\begin{exercise}
    Function addition $+$ on $F$ is associative.
\end{exercise}

\begin{proof}
    For every real number $x$ and every three functions $f, g, h$ in $F$
    \begin{align*}
        ((f + g) + h)(x) & = (f + g)(x) + h(x)    \\
                         & = (f(x) + g(x)) + h(x) \\
                         & = f(x) + (g(x) + h(x)) \\
                         & = f(x) + (g + h)(x)    \\
                         & = (f + (g + h))(x)
    \end{align*}

    Thus function addition $+$ on $F$ is associative.
\end{proof}

% section 1/exercise 34
\begin{exercise}
    Function subtraction $-$ on $F$ is commutative.
\end{exercise}

\begin{proof}
    This is false.

    Counterexample: $f(x) = x, g(x) = x + 1$, then $(f - g)(x) = -1\ne 1 = (g - f)(x)$.
\end{proof}

% section 1/exercise 35
\begin{exercise}
    Function substraction $-$ on $F$ is associative.
\end{exercise}

\begin{proof}
    This is false.

    Counterexample: $f(x) = g(x) = h(x) = x$, then for $x\ne 0$, $((f - g) - h)(x) = -x \ne x = (f - (g - h))(x)$.
\end{proof}

% section 1/exercise 36
\begin{exercise}
    Under function substraction $-$ $F$ has an identity.
\end{exercise}

\begin{proof}
    This is false.

    Assume that under function substraction $-$ $F$ has an identity $\iota$. Then for every $f\in F$ and for every real number $x$
    \[
        f(x) - \iota(x) = \iota(x) - f(x) = f(x)
    \]

    We deduce that $\iota(x) = 0$ and $\iota(x) = \frac{1}{2}f(x)$ for every real number $x$. This is a contradiction, because $\frac{1}{2}f(x)$ is not necessarily equal to $0$ for every real number $x$.

    Thus Under function substraction $-$ $F$ does not have an identity.
\end{proof}

% section 1/exercise 37
\begin{exercise}
    Under function multiplication $\cdot$ $F$ has an identity.
\end{exercise}

\begin{proof}
    This is true.

    Let $\iota(x) = 1$ for every real number $x$. Then for every $f\in F$ and every real number $x$
    \[
        (f\cdot \iota)(x) = f(x)\cdot \iota(x) = f(x) = \iota(x)\cdot f(x) = (\iota\cdot f)(x).
    \]
\end{proof}

% section 1/exercise 38
\begin{exercise}
    Function multiplication $\cdot$ on $F$ is commutative.
\end{exercise}

\begin{proof}
    For every real number $x$ and every two functions $f, g$ in $F$
    \begin{align*}
        (f\cdot g)(x) & = f(x)g(x)       \\
                      & = g(x)f(x)       \\
                      & = (g\cdot f)(x).
    \end{align*}

    Thus function multiplication $\cdot$ on $F$ is commutative.
\end{proof}

% section 1/exercise 39
\begin{exercise}
    Function multiplication $\cdot$ on $F$ is associative.
\end{exercise}

\begin{proof}
    For every real number $x$ and every three functions $f, g, h$ in $F$
    \begin{align*}
        ((f \cdot g) \cdot h)(x) & = (f \cdot g)(x) \cdot h(x)    \\
                                 & = (f(x) \cdot g(x)) \cdot h(x) \\
                                 & = f(x) \cdot (g(x) \cdot h(x)) \\
                                 & = f(x) \cdot (g \cdot h)(x)    \\
                                 & = (f \cdot (g \cdot h))(x)
    \end{align*}

    Thus function multiplication $\cdot$ on $F$ is associative.
\end{proof}

% section 1/exercise 40
\begin{exercise}
    Function composition $\circ$ on $F$ is commutative.
\end{exercise}

\begin{proof}
    This is false.

    Counterexample: $f(x) = x + 1, g(x) = 2x$. $(f\circ g)(x) = f(g(x)) = 2x + 1\ne 2x + 2 = g(x + 1) = g(f(x))$.
\end{proof}

% section 1/exercise 41
\begin{exercise}
    If $*$ and $*'$ are any two binary operations on a set $S$, then
    \[
        a * (b *' c) = (a * b) *' (a * c)\quad\text{for all $a, b, c\in S$.}
    \]
\end{exercise}

\begin{proof}
    This is false. We give a counterexample.

    Let $S = \{ 0, 1 \}$. We define $*$ and $*'$ as follows
    % chktex-file 44
    \begin{tabular}{c|cc}
        * & 0 & 1 \\
        \midrule
        0 & 0 & 0 \\
        1 & 0 & 1
    \end{tabular}
    \begin{tabular}{c|cc}
        \midrule
        *' & 0 & 1 \\
        0  & 1 & 0 \\
        1  & 0 & 1
    \end{tabular}
    \[
        \begin{split}
            0 * (0 *' 1) = 0 * 0 = 0 \ne 1,  \\
            (0 * 0) *' (0 * 1) = 0 *' 0 = 1.
        \end{split}
    \]
\end{proof}

% section 1/exercise 42
\begin{exercise}
    Suppose that $*$ is an \textit{associative binary} operation on a set $S$. Let $H = \{ a \in S \vert a * x = x * a \text{ for all } x\in S \}$. Show that $H$ is closed under $*$. (We think of $H$ as consisting of all elements of $S$ that \textit{commute} with every element in $S$.)
\end{exercise}

\begin{proof}
    Let $a, b$ be elements of $H$. For every element $x$ of $S$
    \begin{align*}
        (a * b) * x & = a * (b * x) & \text{(associative)} \\
                    & = a * (x * b) & (b\in H)             \\
                    & = (a * x) * b & \text{(associative)} \\
                    & = (x * a) * b & (a\in H)             \\
                    & = x * (a * b) & \text{(associative)}
    \end{align*}

    According to the definition of $H$, $a * b\in H$. Hence $H$ is closed under the binary operation $*$ on $S$.
\end{proof}

% section 1/exercise 43
\begin{exercise}
    Suppose that $*$ is an associative and commutative operation on a set $S$. Show that $H = \{ a\in S \vert a * a = a \}$ is closed under $*$. (The element of $H$ are \textbf{idempotents} of the binary operation $*$.)
\end{exercise}

\begin{proof}
    Let $a, b$ be elements of $H$.
    \begin{align*}
        (a * b) * (a * b) & = (a * b) * (b * a) & \text{(commutative)} \\
                          & = ((a * b) * b) * a & \text{(associative)} \\
                          & = (a * (b * b)) * a & \text{(associative)} \\
                          & = (a * b) * a       & (b\in H)             \\
                          & = (b * a) * a       & \text{(commutative)} \\
                          & = b * (a * a)       & \text{(associative)} \\
                          & = b * a             & (a\in H)             \\
                          & = a * b             & \text{(commutative)}
    \end{align*}

    According to the definition of $H$, $a * b\in H$. Hence $H$ is closed under $*$.
\end{proof}

% section 1/exercise 44
\begin{exercise}
    Let $S$ be a set and let $*$ be a binary operation on $S$ satisfying the two laws
    \begin{itemize}
        \item $x * x = x$ for all $x\in S$, and
        \item $(x * y) * z = (y * z) * x$ for all $x, y, z\in S$.
    \end{itemize}

    Show that $*$ is associative and commutative.
\end{exercise}

\begin{proof}
    For all $x, y\in S$,

    \begin{align*}
        x * y & = (x * y) * (x * y) & \text{(Law 1)} \\
              & = (y * (x * y)) * x & \text{(Law 2)} \\
              & = ((x * y) * x) * y & \text{(Law 2)} \\
              & = ((y * x) * x) * y & \text{(Law 2)} \\
              & = ((x * x) * y) * y & \text{(Law 2)} \\
              & = (x * y) * y       & \text{(Law 1)} \\
              & = (y * y) * x       & \text{(Law 2)} \\
              & = y * x             & \text{(Law 1)}
    \end{align*}

    So $*$ is commutative.

    \begin{align*}
        (x * y) * z & = (y * z) * x & \text{(Law 2)}           \\
                    & = x * (y * z) & \text{(Commutative law)}
    \end{align*}

    So $*$ is associative.
\end{proof}


\section{Groups}

\subsection*{Computations}

In Exercises 1 through 9, determine whether the binary operation $*$ gives a group structure on the given set. If no group results, give the first axiom in order $\mathcal{G}_{1}, \mathcal{G}_{2}, \mathcal{G}_{3}$ from Definition 2.1 that does not hold.

% section 2/exercise 1
\begin{exercise}
    Let $*$ be defined on $\mathbb{Z}$ by letting $a * b = ab$.
\end{exercise}

\begin{proof}
    This is not a group structure. $\mathcal{G}_{3}$ does not hold.
\end{proof}

% section 2/exercise 2
\begin{exercise}
    Let $*$ be defined on $2\mathbb{Z} = \{ 2n \vert n\in\mathbb{Z} \}$ by letting $a * b = a + b$.
\end{exercise}

\begin{proof}
    This is a group structure.
\end{proof}

% section 2/exercise 3
\begin{exercise}
    Let $*$ be defined on $\mathbb{R}^{+}$ by letting $a * b = \sqrt{ab}$.
\end{exercise}

\begin{proof}
    This is not a group structure. $\mathcal{G}_{1}$ does not hold.
\end{proof}

% section 2/exercise 4
\begin{exercise}
    Let $*$ be defined on $\mathbb{Q}$ by letting $a * b = ab$.
\end{exercise}

\begin{proof}
    This is not a group structure. $\mathcal{G}_{3}$ does not hold.
\end{proof}

% section 2/exercise 5
\begin{exercise}
    Let $*$ be defined on the set $\mathbb{R}^{*}$ of nonzero real numbers by letting $a * b = a/b$.
\end{exercise}

\begin{proof}
    This is not a group structure. $\mathcal{G}_{1}$ does not hold.
\end{proof}

% section 2/exercise 6
\begin{exercise}
    Let $*$ be defined on $\mathbb{C}$ by letting $a * b = \abs{ab}$.
\end{exercise}

\begin{proof}
    This is not a group structure. $\mathcal{G}_{2}$ does not hold.
\end{proof}

% section 2/exercise 7
\begin{exercise}
    Let $*$ be defined on the set $\{ a, b \}$ by Table 2.26.
    \begin{tabular}{c|cc}
        * & a & b \\
        \hline
        a & a & b \\
        b & b & b
    \end{tabular}
\end{exercise}

\begin{proof}
    $(a * a) * a = a = a * (a * a)$

    $(b * b) * b = b = b * (b * b)$

    $a * (a * b) = a * b = (a * a) * b$

    $a * (b * a) = a * b = b = b * a = (a * b) * a$

    $b * (a * a) = b * a = (b * a) * a$

    $b * (a * b) = b * b = (b * a) * b$

    $a * (b * b) = a * b = b = b * b = (a * b) * b$

    $b * (b * a) = b * b = b = b * a = (b * b) * a$

    So this structure is associative.

    $a * b = b * a = b$

    $a * a = a$

    So this structure has an identity element, which is $a$.

    This is not a group structure. $\mathcal{G}_{3}$ does not hold, since $b$ does not have an inverse.
\end{proof}

% section 2/exercise 8
\begin{exercise}
    Let $*$ be defined on the set $\{ a, b \}$ by Table 2.27.
    \begin{tabular}{c|cc}
        * & a & b \\
        \hline
        a & a & b \\
        b & a & b
    \end{tabular}
\end{exercise}

\begin{proof}
    $(a * a) * a = a = a * (a * a)$

    $(b * b) * b = b = b * (b * b)$

    $(a * a) * b = a * b = a * (a * b)$

    $(a * b) * a = b * a = a = a * a = a * (b * a)$

    $(b * a) * a = a * a = a = b * a = b * (a * a)$

    $(a * b) * b = b * b = b = a * b = a * (b * b)$

    $(b * b) * a = b * a = b * (b * a)$

    $(b * a) * b = a * b = b = b * b = b * (a * b)$

    So this structure is associative.

    This is not a group structure. $\mathcal{G}_{2}$ does not hold, since if there are an identity element, that must be commutative with the other, but in fact,
    \[
        a * b = b \ne a = b * a.
    \]
\end{proof}

% section 2/exercise 9
\begin{exercise}
    Let $*$ be defined on the set $\{ e, a, b \}$ by Table 2.28.
    \begin{tabular}{c|ccc}
        * & e & a & b \\
        \hline
        e & e & a & b \\
        a & a & e & b \\
        b & b & b & a
    \end{tabular}
\end{exercise}

\begin{proof}
    This structure is not a group. $\mathcal{G}_{1}$ does not hold, since
    \[
        (a * b) * b = b * b = e \ne a = a * e = a * (b * b).
    \]
\end{proof}

% section 2/exercise 10
\begin{exercise}
    Let $n$ be a positive integer and let $n\mathbb{Z} = \{ nm \vert m\in\mathbb{Z} \}$.
    \begin{enumerate}[label={\textbf{\alph*.}}]
        \item Show that $\anglebracket{n\mathbb{Z}, +}$ is a group.
        \item Show that $\anglebracket{n\mathbb{Z}, +} \simeq \anglebracket{\mathbb{Z}, +}$
    \end{enumerate}
\end{exercise}

\begin{proof}
    \begin{enumerate}[label={\textbf{\alph*.}}]
        \item Let $x, y, z$ be elements of $n\mathbb{Z}$. According to the definition of $n\mathbb{Z}$, there exist integers $a, b, c$ such that $x = na, y = nb, z = nc$.

              $x + y = na + nb = n(a + b)\in n\mathbb{Z}$. So $n\mathbb{Z}$ is closed under the operation $+$.

              $(x + y) + z = (na + nb) + nc = n(a + b) + nc = n((a + b) + c) = n(a + (b + c)) = na + n(b + c) = na + (nb + nc)$. So $\anglebracket{n\mathbb{Z}, +}$ is associative.

              $x + 0 = x = 0 + x$. So $\anglebracket{n\mathbb{Z}, +}$ has an identity element, which is $0$.

              $x + (-na) = na + (-na) = n (a + (-a)) = 0 = n ((-a) + a) = (-na) + na = (-na) + x$. So each element of $\anglebracket{n\mathbb{Z}, +}$ has an inverse.

              Hence $\anglebracket{n\mathbb{Z}, +}$ is a group.
        \item Let's define mappings $f: n\mathbb{Z} \to \mathbb{Z}$ as $f(x) = x/n$ and $g: \mathbb{Z} \to n\mathbb{Z}$ as $g(y) = n\cdot y$.

              For every $x\in n\mathbb{Z}, y\in\mathbb{Z}$
              \[
                  \begin{split}
                      f(g(y)) = f(ny) = (n\cdot y)/n = y, \\
                      g(f(x)) = g(x/n) = n\cdot (x/n) = x.
                  \end{split}
              \]

              So $f$ is a one-to-one function from $n\mathbb{Z}$ onto $\mathbb{Z}$.

              For every $x, y\in n\mathbb{Z}$, $x/n$ and $y/n$ are elements of $\mathbb{Z}$
              \[
                  f(x + y) = \frac{x + y}{n} = \frac{x}{n} + \frac{y}{n} = f(x) + f(y)
              \]

              Hence $\anglebracket{n\mathbb{Z}, +}$ and $\anglebracket{\mathbb{Z}, +}$ are isomorphic.
    \end{enumerate}
\end{proof}

In Exercises 11 through 18, determine whether the given set of matrices under the specified operation, matrix addition or multiplication, is a group.

% section 2/exercise 11
\begin{exercise}
    All $n\times n$ diagonal matrices under matrix addition.
\end{exercise}

\begin{proof}
    This is a group.
\end{proof}

% section 2/exercise 12
\begin{exercise}
    All $n\times n$ diagonal matrices under matrix multiplication.
\end{exercise}

\begin{proof}
    This is not a group, since the zero $n\times n$ matrix (which is also an $n\times n$ diagonal matrix) does not have a multiplicative inverse.
\end{proof}

% section 2/exercise 13
\begin{exercise}
    All $n\times n$ diagonal matrices with no zero diagonal entry under matrix multiplication.
\end{exercise}

\begin{proof}
    This is a group.
\end{proof}

% section 2/exercise 14
\begin{exercise}
    All $n\times n$ diagonal matrices with all diagonal entries $1$ or $-1$ under matrix multiplication.
\end{exercise}

\begin{proof}
    This is a group.
\end{proof}

% section 2/exercise 15
\begin{exercise}
    All $n\times n$ upper-triangular matrices under matrix multiplication.
\end{exercise}

\begin{proof}
    $A = {(a_{i.j})}_{n\times n}$, $B = {(b_{i.j})}_{n\times n}$ such that $a_{i.j} = 0$ if $i\geq j$, $b_{i.j} = 0$ if $i\geq j$. Let $C = {(c_{i.j})}_{n\times n} = AB$.

    Suppose that $1\le s\le r\le n$.
    \begin{align*}
        c_{r.s} & = \sum^{n}_{k=1}a_{r.k}b_{k.s}                                  \\
                & = \sum^{s}_{k=1}a_{r.k}b_{k.s} + \sum^{n}_{k=s+1}a_{r.k}b_{k.s} \\
                & = 0 + 0                                                         \\
                & = 0.
    \end{align*}

    $\displaystyle\sum^{s}_{k=1}a_{r.k}b_{k.s} = 0$ because $r\ge s\ge k$. $\displaystyle\sum^{n}_{k=s+1}a_{r.k}b_{k.s} = 0$ because $k \geq s + 1 > s$.

    So the set of $n\times n$ upper-triangular matrices is closed under matrix multiplication.

    However, the $n\times n$ zero matrix (which is also an $n\times n$ upper-triangular matrix) is not invertible. Hence this is not a group.
\end{proof}

% section 2/exercise 16
\begin{exercise}
    All $n\times n$ upper-triangular matrices under matrix addition.
\end{exercise}

\begin{proof}
    This is a group.
\end{proof}

% section 2/exercise 17
\begin{exercise}
    All $n\times n$ upper-triangular matrices with determinant $1$ under multiplication.
\end{exercise}

\begin{proof}
    According to the proof of Exercise 15 and $\det(AB) = \det(A)\det(B)$, the set of $n\times n$ upper-triangular matrices is closed under matrix multiplication.

    Let $A, B, C$ be $n\times n$ matrices whose determinants is $1$.

    Matrix multiplication is associative, so the matrix multiplication on this set of matrices is associative.

    The $n\times n$ identity matrix $I_{n}$, $AI = IA = A$. So the matrix multiplication on this set of matrices has an identity matrix.

    Let
    \[
        A = \begin{bmatrix}
            a_{1.1} & a_{1.2} & \cdots & a_{1.n} \\
            0       & a_{2.2} & \cdots & a_{2.n} \\
            \vdots  & \vdots  &        & \vdots  \\
            0       & 0       & \cdots & a_{n.n}
        \end{bmatrix}.
    \]

    Since $A$ is a diagonal matrix, then $\det(A) = a_{1.1}a_{2.2}\cdots a_{n.n}$. On the other hand, $\det(A) = 1$, so $a_{1.1}, a_{2.2}, \ldots, a_{n.n}$ are non-zero. To define the multiplicative inverse of $A$, we perform elementary row operations on the following marix
    \[
        \begin{array}{cccc|cccc}
            a_{1.1} & a_{1.2} & \cdots & a_{1.n} & 1      & 0      & \cdots & 0      \\
            0       & a_{2.2} & \cdots & a_{2.n} & 0      & 1      & \cdots & 0      \\
            \vdots  & \vdots  &        & \vdots  & \vdots & \vdots &        & \vdots \\
            0       & 0       & \cdots & a_{n.n} & 0      & 0      & \cdots & 1
        \end{array}
    \]

    Multiply $i$-th row by ${a_{i.i}}^{-1}$.
    \[
        \begin{array}{cccc|cccc}
            1      & {a_{1.1}}^{-1}a_{1.2} & \cdots & {a_{1.1}}^{-1}a_{1.n} & {a_{1.1}}^{-1} & 0              & \cdots & 0              \\
            0      & 1                     & \cdots & {a_{2.2}}^{-1}a_{2.n} & 0              & {a_{2.2}}^{-1} & \cdots & 0              \\
            \vdots & \vdots                &        & \vdots                & \vdots         & \vdots         &        & \vdots         \\
            0      & 0                     & \cdots & 1                     & 0              & 0              & \cdots & {a_{n.n}}^{-1}
        \end{array}
    \]
    \begin{itemize}
        \item Add $-{a_{i.i}}^{-1}a_{i.n}$ times the $n$-th row to the $i$-th row ($1\le i < n$). After these operations, the $n\times n$ matrix on the right is still an $n\times n$ upper-triangular matrix.
        \item Add $-{a_{i.i}}^{-1}a_{i.(n-1)}$ times the $(n-1)$-th row to the $i$-th row ($1\le i < n-1$). After these operations, the $n\times n$ matrix on the right is still an $n\times n$ upper-triangular matrix.
        \item \ldots
        \item Add $-{a_{i.i}}^{-1}a_{i.2}$ times the $2$-nd row to the $i$-th row ($1\le i < 2$). After these operations, the $n\times n$ matrix on the right is still an $n\times n$ upper-triangular matrix.
    \end{itemize}

    After all these operations, the $n\times n$ matrix on the left is the $n\times n$ identity matrix, and the $n\times n$ matrix on the right is still an $n\times n$ upper-triangular matrix, which is also the multiplicative inverse of $A$. So there exists an $n\times n$ upper-triangular matrix $A'$ such that $AA' = A'A = I_{n}$.

    Thus, all $n\times n$ upper-triangular matrices with determinant $1$ under multiplication is a group.
\end{proof}

% section 2/exercise 18
\begin{exercise}
    The set of $2\times 2$ matrices $G = \{ e, a, b \}$ where $e = \begin{bmatrix}1 & 0 \\ 0 & 1\end{bmatrix}$, $a = \begin{bmatrix}-\frac{1}{2} & -\frac{\sqrt{3}}{2} \\ \frac{\sqrt{3}}{2} & -\frac{1}{2}\end{bmatrix}$, and $b = \begin{bmatrix}-\frac{1}{2} & \frac{\sqrt{3}}{2} \\ -\frac{\sqrt{3}}{2} & -\frac{1}{2}\end{bmatrix}$ under matrix multiplication.
\end{exercise}

\begin{proof}
    \[
        ee = e,\qquad ea = ae = a,\qquad eb = be = b.
    \]
    \[
        aa = b,\qquad bb = a,\qquad ab = ba = e.
    \]

    So $G$ is closed under matrix multiplication. Matrix multiplication within $G$ is associative. According to the calculation above, matrix multiplication within $G$ has an identity element, which is $e$. Each element of $G$ has an inverse: $ab = ba = e$, $ee = e$.

    Hence $G$ with the matrix multiplication is a group.
\end{proof}

% section 2/exercise 19
\begin{exercise}
    Let $S$ be the set of all real numbers except $-1$. Define $*$ on $S$ by
    \[
        a * b = a + b + ab.
    \]
    \begin{enumerate}[label={\textbf{\alph*.}}]
        \item Show that $*$ gives a binary operation on $S$.
        \item Show that $\anglebracket{S, *}$ is a group.
        \item Find the solution of the equation $2 * x * 3 = 7$ in $S$.
    \end{enumerate}
\end{exercise}

\begin{proof}
    \begin{enumerate}[label={\textbf{\alph*.}}]
        \item Let $a, b$ be elements of $S$
              \[
                  a * b + 1 = 1 + a + b + ab = (1 + a)(1 + b)\ne 0.
              \]

              So $a * b\ne -1$, which means $*$ is a binary operation on $S$.
        \item Let $a, b, c$ be elements of $S$
              \begin{align*}
                  (a * b) * c & = (a + b + ab) * c                   \\
                              & = a + b + c + ab + ca + bc + abc     \\
                              & = a + (b + c + bc) + (ab + ac + abc) \\
                              & = a + b * c + a\cdot (b * c)         \\
                              & = a * (b * c)
              \end{align*}

              So $\anglebracket{S, *}$ is associative.
              \[
                  a * 0 = a + 0 + a\cdot 0 = a = 0 + a + 0\cdot a = 0 * a.
              \]

              So $\anglebracket{S, *}$ has an identity element.
              \[
                  \begin{split}
                      a * \frac{-a}{a+1} = a + \frac{-a}{a+1} + \frac{-a^{2}}{a+1} = \frac{a^{2}}{a+1} + \frac{-a^{2}}{a+1} = 0, \\
                      \frac{-a}{a+1} * a = \frac{-a}{a+1} + a + \frac{-a^{2}}{a+1} = \frac{a^{2}}{a+1} + \frac{-a^{2}}{a+1} = 0.
                  \end{split}
              \]

              Hence $\anglebracket{S, *}$ is a group.
        \item
              \begin{align*}
                  2 * x * 3                                          & = 7                          \\
                  \left(\left(\frac{-2}{3} * 2\right) * x\right) * 3 & = \frac{-2}{3} * 7           \\
                  (0 * x) * 3                                        & = \frac{5}{3}                \\
                  x * 3                                              & = \frac{5}{3}                \\
                  x * \left(3 * \frac{-3}{4}\right)                  & = \frac{5}{3} * \frac{-3}{4} \\
                  x * 0                                              & = \frac{-1}{3}               \\
                  x                                                  & = \frac{-1}{3}
              \end{align*}

              Hence $x = \frac{-1}{3}$.
    \end{enumerate}
\end{proof}

% section 2/exercise 20
\begin{exercise}
    This section 2/exercise shows that there are two nonisomorphic group structures on a set of $4$ elements.

    Let the set be $\{ e, a, b, c \}$, with $e$ the identity element for the group operation. A group table would then have to start in the manner shown in Table 2.29. The square indicated by the question mark cannot be filled in with $a$. It must be filled in either with the identity element $e$ or with an element different from both $e$ and $a$. In this latter case, it is no loss of generality to assume that this element is $b$. If this square is filled in with $e$, the table can then be completed in two ways to give a group. Find these two tables. (You need not check the associative law.) If this square is filled in with $b$, then the table can only be completed in one way to give a group. Find this table. (Again, you need not check the associative law.) Of the three tables you now have, two give isomorphic groups. Determine which two tables these are, and give the one-to-one onto relabeling function which is an isomorphism.

    \begin{enumerate}[label={\textbf{\alph*}}]
        \item Are all groups of $4$ elements commutative?
        \item Find a way to relabel the four matrices
              \[
                  \left\{\begin{bmatrix}
                      1 & 0 \\
                      0 & 1
                  \end{bmatrix},
                  \begin{bmatrix}
                      0 & -1 \\
                      1 & 0
                  \end{bmatrix},
                  \begin{bmatrix}
                      -1 & 0  \\
                      0  & -1
                  \end{bmatrix},
                  \begin{bmatrix}
                      0  & 1 \\
                      -1 & 0
                  \end{bmatrix}\right\}
              \]

              so the atrix multiplication table is identical to one you constructed. This shows that the table you constructed defines an associative operation and therefore gives a group.
        \item So that for a particular value of $n$, the group elements given in Exercise 14 can be relabeled so their group table is identitcal to one you constructed. This implies the operation in the table is also associate.
    \end{enumerate}
\end{exercise}

\begin{proof}
    Possible group tables are
    \[
        \begin{array}{c|cccc}
            * & e & a & b & c \\
            \hline
            e & e & a & b & c \\
            a & a & b & c & e \\
            b & b & c & e & a \\
            c & c & e & a & b
        \end{array},\qquad
        \begin{array}{c|cccc}
            * & e & a & b & c \\
            \hline
            e & e & a & b & c \\
            a & a & e & c & b \\
            b & b & c & e & a \\
            c & c & b & a & e
        \end{array},\qquad
        \begin{array}{c|cccc}
            * & e & a & b & c \\
            \hline
            e & e & a & b & c \\
            a & a & e & c & b \\
            b & b & c & a & e \\
            c & c & b & e & a
        \end{array}
    \]

    The 1st and the 3rd tables give isomorphic groups. Relabel: $(e, a, b, c)\mapsto (e, b, c, a)$.

    \begin{enumerate}[label={\textbf{\alph*.}}]
        \item These groups of 4 elements are commutative.
        \item One way to make the matrix multiplication table identical to the 3rd table
              \[
                  e = \begin{bmatrix}
                      1 & 0 \\
                      0 & 1
                  \end{bmatrix},
                  b = \begin{bmatrix}
                      0 & -1 \\
                      1 & 0
                  \end{bmatrix},
                  a = \begin{bmatrix}
                      -1 & 0  \\
                      0  & -1
                  \end{bmatrix},
                  c = \begin{bmatrix}
                      0  & 1 \\
                      -1 & 0
                  \end{bmatrix}
              \]
        \item When $n = 2$, we label
              \[
                  e = \begin{bmatrix}
                      1 & 0 \\
                      0 & 1
                  \end{bmatrix},
                  a = \begin{bmatrix}
                      0 & -1 \\
                      1 & 0
                  \end{bmatrix},
                  b = \begin{bmatrix}
                      -1 & 0  \\
                      0  & -1
                  \end{bmatrix},
                  c = \begin{bmatrix}
                      0  & 1 \\
                      -1 & 0
                  \end{bmatrix}
              \]
    \end{enumerate}
\end{proof}

% section 2/exercise 21
\begin{exercise}
    According to Exercise $12$ of Section $1$, there are $16$ possible binary operations on a set of $2$ elements. How many of these give a structure of a group? How many of the $19,683$ possible binary operations on a set of $3$ elements give a group structure.
\end{exercise}

\begin{proof}
    Let $X = \{ a, b \}$ be a set of $2$ elements.

    A binary operation on $X$ give a structure of a group if in its group table:
    \begin{itemize}
        \item each column and each row contains distinct elements,
        \item there is exactly one row which is identical to the top header, there is exactly one column which is identical to the left header.
    \end{itemize}

    So there are two possible binary operations which can give group structure.
    \[
        \begin{array}{c|cc}
            * & a & b \\
            \hline
            a & a & b \\
            b & b & a
        \end{array},\qquad
        \begin{array}{c|cc}
            * & a & b \\
            \hline
            a & b & a \\
            b & a & b
        \end{array}
    \]

    After checking for associativity, we conclude that there are $2$ binary operations that give group structure. However, they are isomorphic.

    Let $Y = \{ a, b, c \}$ be a set of $3$ elements.

    The following tables correspond to binary operations which possibly give group structure.
    \[
        \begin{array}{c|ccc}
            * & a & b & c \\
            \hline
            a & a & b & c \\
            b & b & c & a \\
            c & c & a & b \\
        \end{array},\qquad
        \begin{array}{c|ccc}
            * & a & b & c \\
            \hline
            a & c & a & b \\
            b & a & b & c \\
            c & b & c & a \\
        \end{array},\qquad
        \begin{array}{c|ccc}
            * & a & b & c \\
            \hline
            a & b & c & a \\
            b & c & a & b \\
            c & a & b & c \\
        \end{array}
    \]

    After checking for associativity, we conclude that there are $3$ binary operations that give group structure. However, they are isomorphic.
\end{proof}

\subsection*{Concepts}

% section 2/exercise 22
\begin{exercise}
    Consider our axioms $\mathscr{G}_{1}, \mathscr{G}_{2}$, and $\mathscr{G}_{3}$ for a group. We gave them in the order $\mathscr{G}_{1}\mathscr{G}_{2}\mathscr{G}_{3}$. Conceivable other orders to state the axioms are  $\mathscr{G}_{1}\mathscr{G}_{3}\mathscr{G}_{2}$, $\mathscr{G}_{2}\mathscr{G}_{1}\mathscr{G}_{3}$, $\mathscr{G}_{2}\mathscr{G}_{3}\mathscr{G}_{1}$, $\mathscr{G}_{3}\mathscr{G}_{1}\mathscr{G}_{2}$, and $\mathscr{G}_{3}\mathscr{G}_{2}\mathscr{G}_{1}$. Of these six possible orders, exactly three are acceptable for a definition. Which orders are not acceptable, any why?
\end{exercise}

\begin{proof}
    Inacceptable orders are $\mathscr{G}_{1}\mathscr{G}_{3}\mathscr{G}_{2}$, $\mathscr{G}_{3}\mathscr{G}_{1}\mathscr{G}_{2}$, $\mathscr{G}_{3}\mathscr{G}_{2}\mathscr{G}_{1}$. Because $\mathcal{G}_{3}$ depends on the definition of identity element, which is defined in $\mathcal{G}_{2}$.
\end{proof}

% section 2/exercise 23
\begin{exercise}
    The following ``definitions'' of a group are taken verbatim, including spelling and punctuation, from papers of students who wrote a bit too quickly and carelessly. Criticize them.
    \begin{enumerate}
        \item A group $G$ is a set of elements together with a binary operation $*$ such that the following conditions are satisfied

              $*$ is associative

              There exists $e\in G$ such that
              \[
                  e * x = x * e = x = \text{identity}
              \]

              For every $a\in G$ there exists an $a'$ (inverse) such that
              \[
                  a\cdot a' = a'\cdot a = e
              \]
        \item A group is a set $G$ such that

              The operation on $G$ is associative

              there is an identity element ($e$) in $G$.

              for every $a\in G$, there is an $a'$ (inverse for each element)
        \item A group is a set with a binary operation such

              the binary operation is defined

              an inverse exists

              an identity element exists
        \item A set $G$ is called a group over the binery operation $*$ such that for all $a, b\in G$

              Binary operation $*$ is associative under addition

              there exist an element $\{e\}$ such that
              \[
                  a * e = e * a = e
              \]

              Fore every element $a$ there exists an element $a'$ such that
              \[
                  a * a' = a' * a = e
              \]
    \end{enumerate}
\end{exercise}

\begin{proof}
    \begin{enumerate}
        \item $x$ is not neccessarily equal to the identity element. The notations for the binary opertion are inconsistent ($*$ and $\cdot$).
        \item First letter of each sentence should be in uppercase. A group is a set $G$ with a binary operation. The meaning of identity element and inverse must be clarified.
        \item The definition of idenity element, inverse must be given. Sentences are not clear.
        \item $b$ isn't used anywhere. $e$ must be used instead of $\{e\}$. There are typos.
    \end{enumerate}
\end{proof}

% section 2/exercise 24
\begin{exercise}
    Give a table defining an operation satisfying axioms $\mathscr{G}_{2}$ and $\mathscr{G}_{3}$ in the definition of a group, but not satisfying axiom $\mathscr{G}_{1}$ for the set
    \begin{enumerate}[label={\textbf{\alph*.}}]
        \item $\{ e, a, b \}$
        \item $\{ e, a, b, c \}$
    \end{enumerate}
\end{exercise}

\begin{proof}
    \begin{enumerate}[label={\textbf{\alph*.}}]
        \item
              \[
                  \begin{array}{c|ccc}
                      * & e & a & b \\
                      \hline
                      e & e & a & b \\
                      a & a & e & e \\
                      b & b & e & a \\
                  \end{array}
              \]

              $(a * a) * b = e * b = b \ne a = a * e = a * (a * b)$. So $*$ is not associative.
        \item $\{ e, a, b, c \}$
              \[
                  \begin{array}{c|cccc}
                      * & e & a & b & c \\
                      \hline
                      e & e & a & b & c \\
                      a & a & c & e & b \\
                      b & b & e & a & a \\
                      c & c & b & a & e
                  \end{array}
              \]

              $(b * a) * c = e * c = c \ne a = b * b = b * (a * c)$. So $*$ is not associative.
    \end{enumerate}
\end{proof}

% section 2/exercise 25
\begin{exercise}
    Mark each of the following true or false.
    \begin{enumerate}[label={\textbf{\alph*.}}]
        \item A group may have more tha one identity element.
        \item Any two groups of three elements are isomorphic.
        \item In a group, each linear equation has a solution.
        \item The proper attitude toward a definition is to memorize it so that you can reproduce it word for word as in the text.
        \item Any definition a person gives for a group is correct provided that everything that is a group by that person's defintion is also a group by the definition in the text.
        \item Any definition a person gives for a group is correct provided he or she can show that everything that satisfies the definition satisfies the one in the text and conversely.
        \item Every finite group of at most three elements is abelian.
        \item An equation of the form $a * x * b = c$ always has a unique solution in a group.
        \item The empty set can be considered a group.
        \item Every group is a binary algebraic structure.
    \end{enumerate}
\end{exercise}

\begin{proof}
    \begin{enumerate}[label={\textbf{\alph*.}}]
        \item False.
        \item True.
        \item True.
        \item False.
        \item False.
        \item True.
        \item True.
        \item True.
        \item False.
        \item True.
    \end{enumerate}
\end{proof}

\subsection*{Proof synopsis}

% section 2/exercise 26
\begin{exercise}
    Give a one-sentence synopsis of the proof of the left cancellation law in Theorem 2.16.
\end{exercise}

\begin{proof}
    Multiply both sides with the inverse of the left element and apply the associative law, it follows that the two left elements are equal.
\end{proof}

% section 2/exercise 27
\begin{exercise}
    Give at most a two-sentence synopsis of the proof in Theorem 2.17 that an equation $ax = b$ has a unique solution in a group.
\end{exercise}

\begin{proof}
    Multiply both sides with the inverse of the coefficient and apply the associative law, we obtain a solution. To prove that this solution is unique, we suppose that there are two solutions and apply the left cancellation law to show that the two solutions are identical.
\end{proof}

\subsection*{Theory}

% section 2/exercise 28
\begin{exercise}
    An element $a\ne e$ in a group is said to have order $2$ if $a * a = e$. Prove that if $G$ is a group and $a\in G$ has order $2$, then for any $b\in G$, $b' * a * b$ also has order $2$.
\end{exercise}

\begin{proof}
    \begin{align*}
        (b' * a * b) * (b' * a * b) & = (b' * a) * (b * b') * (a * b) & \text{associative law}   \\
                                    & = (b' * a) * e * (a * b)                                   \\
                                    & = (b' * a) * (a * b)                                       \\
                                    & = b' * (a * a) * b              & \text{associative law}   \\
                                    & = b' * a * b                    & \text{$a$ has order $2$}
    \end{align*}

    Hence, $b' * a * b$ also has order $2$.
\end{proof}

% section 2/exercise 29
\begin{exercise}
    Show that if $G$ is a finite group with identity $e$ and with an even number of elements, then there is $a\ne e$ in $G$ such that $a * a = e$.
\end{exercise}

\begin{proof}
    Assume that there is no element other than $e$ that has order $2$.

    Let the elements of $G$ be $e, a_{1}, a_{2}, \ldots, a_{2n-1}$. Since $G$ is a group, each element $a_{i}$ has an unique inverse. So we can pair each $a_{i}$ with its inverse and none of them are $e$. Therefore, $\{ a_{1}, a_{2}, \ldots, a_{2n-1} \}$ contains an even number of elements, which contradicts the fact that $2n - 1$ is an odd number.

    So the initial assumption is false, and there is $a\ne e$ in $G$ such that $a * a = e$.
\end{proof}

% section 2/exercise 30
\begin{exercise}
    Let $\mathbb{R}^{*}$ be the set of all real numbers except $0$. Define $*$ on $\mathbb{R}^{*}$ by letting $a * b = \abs{a}b$.
    \begin{enumerate}[label={\textbf{\alph*.}}]
        \item Show that $*$ gives an associative binary operation on $\mathbb{R}^{*}$.
        \item Show that there is a left identity for $*$ and a right inverse for each element in $\mathbb{R}^{*}$.
        \item Is $\mathbb{R}^{*}$ with this binary operation a group?
        \item Explain the significance of this exercise.
    \end{enumerate}
\end{exercise}

\begin{proof}
    \begin{enumerate}[label={\textbf{\alph*.}}]
        \item Let $a, b, c\in\mathbb{R}^{*}$
              \begin{align*}
                  (a * b) * c & = \abs{a}b * c    \\
                              & = \abs{\abs{a}b}c \\
                              & = \abs{ab}c       \\
                              & = \abs{a}\abs{b}c \\
                              & = \abs{a}(b * c)  \\
                              & = a * (b * c)
              \end{align*}

              So $*$ gives an associative binary operation on $\mathbb{R}^{*}$.
        \item $1 * b = b$, $-1 * b = b$ for all $b\in\mathbb{R}^{*}$.

              $a * \frac{1}{\abs{a}} = 1$, $a * \frac{-1}{\abs{a}} = -1$ for all $a\in\mathbb{R}^{*}$.
        \item $\mathbb{R}^{*}$ with this binary operation is not a group, because $\mathbb{R}^{*}$ does not have an identity element under the operation $*$.
        \item When checking the Axiom 2 of group structure, one must check if the identity element for both side.
    \end{enumerate}
\end{proof}

% section 2/exercise 31
\begin{exercise}
    If $*$ is a binary operation on a set $S$, an element $x$ of $S$ is an \textbf{idempotent for $*$} if $x * x = x$. Prove hat a group has exactly one idempotent element.
\end{exercise}

\begin{proof}
    Let $e$ be the identity element of the group $\anglebracket{S, *}$. Since $e * e = e$, $e$ is an idempotent element.

    Suppose that $x\in G$ is an idempotent element, then $x * x = x$. Let $x'$ be the inverse of $x$, then $x' * (x * x) = x' * x$. According to Axiom 1, we obtain that $(x' * x) * x = x' * x$, and by Axiom 2, 3, it follows that $x = e$.

    Therefore, a group has exactly one idempotent element, which is also the identity element.
\end{proof}

% section 2/exercise 32
\begin{exercise}
    Show that every group $G$ with identity $e$ and such that $x * x = e$ for all $x\in G$ is abelian.
\end{exercise}

\begin{proof}
    In a group, every element has a unique inverse.

    Let $a, b$ be two elements of $G$, and $a', b'$ be the inverses of $a, b$, respectively.
    \[
        \begin{split}
            (a * b) * (b' * a') = a * (b * b') * a' = a * e * a' = a * a' = e, \\
            (b' * a') * (a * b) = b' * (a' * a) * b = b' * e * b = b' * b = e.
        \end{split}
    \]

    So $b' * a'$ is the inverse element of $a * b$.

    On the one hand, $(a * b) * (a * b) = e$, which means $a * b$ is also an inverse of $a * b$. So $b' * a' = a * b$. On the other hand, $a = a'$ and $b' = b$ because $a * a = b * b = e$. Therefore $a * b = b * a$.

    Thus $G$ is abelian.
\end{proof}

% section 2/exercise 33
\begin{exercise}
    Let $G$ be an abelian group and let $c^{n} = c * c * \cdots * c$ for $n$ factors $c$, where $c\in G$ and $n\in\mathbb{Z}^{+}$. Give a mathematical induction proof that ${(a * b)}^{n} = (a^{n}) * (b^{n})$ for all $a, b\in G$.
\end{exercise}

\begin{proof}
    The statement holds for $n = 1$, since $a * b = a * b$.

    Assume that the statement holds for $n = k$.
    \begin{align*}
        {(a * b)}^{k+1} & = {(a * b)}^{k} * (a * b)                                             \\
                        & = (a^{k}) * (b^{k}) * (a * b) & \text{induction hypothesis}           \\
                        & = (a^{k} * a) * (b^{k} * b)   & \text{associative and $G$ is abelian} \\
                        & = (a^{k+1}) * (b^{k+1})
    \end{align*}

    So the statement also holds for $n = k + 1$.

    According to the principle of mathematical induction, ${(a * b)}^{n} = (a^{n}) * (b^{n})$ for all $a, b\in G$.
\end{proof}

% section 2/exercise 34
\begin{exercise}
    Suppose that $G$ is a group and $a, b\in G$ satisfy $a * b = b * a'$ where as usual, $a'$ is the inverse for $a$. Prove that $b * a = a' * b$.
\end{exercise}

\begin{proof}
    Let $e$ be the identity element of $G$.

    \begin{align*}
        b * a & = (e * b) * a         \\
              & = ((a' * a) * b) * a  \\
              & = (a' * (a * b)) * a  \\
              & = (a' * (b * a')) * a \\
              & = a' * ((b * a') * a) \\
              & = a' * (b * (a' * a)) \\
              & = a' * (b * e)        \\
              & = a' * b
    \end{align*}
\end{proof}

% section 2/exercise 35
\begin{exercise}
    Suppose that $G$ is a group and $a$ and $b$ are elements of $G$ that satisfy $a * b = b * a^{3}$. Rewrite the element ${(a * b)}^{2}$ in the form $b^{k}a^{r}$ (See Exercise 33 for power notaion.)
\end{exercise}

\begin{proof}
    \begin{align*}
        {(a * b)}^{2} & = (a * b) * (a * b)               \\
                      & = (b * a^{3}) * (b * a^{3})       \\
                      & = b * a^{2} * (a * b) * a^{3}     \\
                      & = b * a^{2} * (b * a^{3}) * a^{3} \\
                      & = b * a * (a * b) * a^{6}         \\
                      & = b * a * (b * a^{3}) * a^{6}     \\
                      & = b * (a * b) * a^{3} * a^{6}     \\
                      & = b * (b * a^{3}) * a^{9}         \\
                      & = b^{2} * a^{12}.
    \end{align*}
\end{proof}

% section 2/exercise 36
\begin{exercise}
    Let $G$ be a group with a finite number of elements. Show that for any $a\in G$, there exists an $n\in\mathbb{Z}^{+}$ such that $a^{n} = e$.
\end{exercise}

\begin{proof}
    Suppose that $G$ has $m$ elements.

    Consider $e = a^{0}, a, a^{2}, \ldots a^{m}$. These are $(m + 1)$ elements in $G$. So there exists two distinct non-negative integers $0\le p < q\le m$ such that $a^{p} = a^{q}$. Therefore, $a^{q-p} = e$.

    Hence, there exists an $n\in\mathbb{Z}^{+}$ such that $a^{n} = e$.
\end{proof}

% section 2/exercise 37
\begin{exercise}
    Show that if ${(a * b)}^{2} = a^{2} * b^{2}$ for $a$ and $b$ in a group $G$, then $a * b = b * a$.
\end{exercise}

\begin{proof}
    Since ${(a + b)}^{2} = a^{2} * b^{2}$, then $a' * (a * b) * (a * b) * b' = (a' * a^{2}) * (b^{2} * b')$.

    It follows that $(a' * a) * (b * a) * (b * b') = (a' * a) * (a * b) * (b * b')$, so $b * a = a * b$.

    Thus, $G$ is abelian.
\end{proof}

% section 2/exercise 38
\begin{exercise}
    Let $G$ be a group and let $a, b\in G$. Show that ${(a * b)}' = a' * b'$ if and only if $a * b = b * a$.
\end{exercise}

\begin{proof}
    For any $a, b\in G$, $(b * a)' = a' * b'$.

    $(\Rightarrow)$ If $a * b = b * a$, then $(a * b)' = (b * a)' = a' * b'$.

    $(\Leftarrow)$ If ${(a * b)}' = a' * b'$, then the inverse of ${(a * b)}'$ is equal to the inverse of $a' * b'$. The inverse of $a' * b'$ is $b * a$, the inverse of ${(a * b)}'$ is $a * b$. Therefore $a * b = b * a$.

    Hence ${(a * b)}' = a' * b'$ if and only if $a * b = b * a$.
\end{proof}

% section 2/exercise 39
\begin{exercise}
    Let $G$ be a group and suppose that $a * b * c = e$ for $a, b, c\in G$. Show that $b * c * a = e$ also.
\end{exercise}

\begin{proof}
    $a' = a' * e = a' * (a * b * c) = (a' * a) * (b * c) = e * (b * c) = b * c$. So $b * c$ is the inverse of $a$. Therefore, $b * c * a = e$.
\end{proof}

% section 2/exercise 40
\begin{exercise}
    Prove that a set $G$, together with a binary operation $*$ on $G$ satisfying the left axioms 1, 2, and 3 after Corollary 2.19, is a group.
\end{exercise}

\begin{proof}
    \begin{enumerate}[label={\textbf{Axiom \arabic*.}},itemindent=1cm]
        \item The binary operation $*$ on $G$ is associative.
        \item There exists a \textbf{left identity element} $e$ in $G$ such that $e * x = x$ for all $x\in G$.
        \item For each $a\in G$, there exists \textbf{a left inverse} $a'$ in $G$ such that $a' * a = e$.
    \end{enumerate}

    For every $x\in G$, there exists $x'\in G$ such that $x' * x = e$. Furthermore, $x' * (x * e) = (x' * x) * e = e * e = e$.

    There exists $y$ such that $y * x' = e$, then
    \[
        y * (x' * x) = y * (x' * (x * e)) = y
    \]

    Since $*$ on $G$ is associative, and $y * x' = e$, we obtain that $x = x * e = y$. Therefore $x * x' = e$. So that $e * x = x * e = x$ and $x' * x = e = x * x$, we conclude that $G$ with the binary operation $*$ that satisfies the left axioms is a group.
\end{proof}

% section 2/exercise 41
\begin{exercise}
    Prove that a nonempty set $G$, together with an associate binary operation $*$ on $G$ such that
    \begin{center}
        $a * x = b$ and $y * a = b$ have solutions in $G$ for all $a, b\in G$
    \end{center}

    is a group.
\end{exercise}

\begin{proof}
    According to the hypothesis
    \begin{itemize}
        \item there exists $e\in G$ such that $e * a = a$
        \item for each $m\in G$, there exists $y\in G$ such that $a * y = m$
    \end{itemize}

    For each $m\in G$
    \begin{itemize}
        \item $e * m = e * (a * y) = (e * a) * y = a * y = m$. So $G$ has a \textbf{left identity element}.
        \item according to the hypothesis, there exists $y'$ such that $y' * y = e$, there exists $a'$ such that $a' * a = e$, so
              \begin{align*}
                  (y' * a') * m & = (y' * a') * (a * y)                                         \\
                                & = y' * ((a' * a) * y) & \text{associativity}                  \\
                                & = y' * (e * y)                                                \\
                                & = y' * y              & \text{$e$ is a left identity element} \\
                                & = e
              \end{align*}

              which implies $m$ has a \textbf{left inverse}.
    \end{itemize}

    Hence $G$ with the binary operation $*$ is associative, has a left identity element, and each element has a left inverse. According to Exercise 40 of Section 2, we conclude that $G$ with the binary operation is a group.
\end{proof}

% section 2/exercise 42
\begin{exercise}
    Let $G$ be a group. Prove that $(a')' = a$.
\end{exercise}

\begin{proof}
    \begin{align*}
        (a')' & = (a')' * e        \\
              & = (a')' * (a' * a) \\
              & = ((a')' * a') * a \\
              & = e * a            \\
              & = a.
    \end{align*}
\end{proof}

% section 2/exercise 43
\begin{exercise}
    Let $\phi$ be an isometry of the plane.
    \begin{enumerate}[label={\textbf{\alph*.}}]
        \item Prove that $\phi$ is a one-to-one function.
        \item Prove that $\phi$ maps onto $\mathbb{R}^{2}$.
    \end{enumerate}
\end{exercise}

\begin{proof}
    \begin{enumerate}[label={\textbf{\alph*.}}]
        \item Let $A, B$ be two points in the plane. Since $\phi$ is an isometry, then $\abs{\phi(A)\phi(B)} = \abs{AB}$. Therefore, $A\equiv B$ if and only if $\phi(A)\equiv\phi(B)$. Hence $\phi$ is a one-to-one function.
        \item Let $A, B$ be two points in the plane. Let $C = \phi(B)$. Consider two circles whose centers are $B, C$ and their radii equal $\abs{AC}$. Denote these circles by $\odot(B, \abs{AC})$ and $\odot(C, \abs{AC})$.

              $\phi$ restricted on $\odot(B, \abs{AC})$, into $\odot(C, \abs{AC})$ is still a one-to-one mapping.

              Let $M$ be a point on $\odot(B, \abs{AC})$. Since $\phi$ is an isometry, $\abs{BM} = \abs{C\phi(M)}$, so $\phi(M)$ lies on $\odot(C, \abs{AC})$. There are at most two points $P, Q$ on $\odot(B, \abs{AC})$ such that $\abs{MP} = \abs{MQ} = \abs{A\phi(M)}$. There are at most two points on $\odot(C, \abs{AC})$ such that their distances to $\phi(M)$ are equal to $\abs{A\phi(M)}$, and one of them is $A$. Therefore, either $\phi(P) = A$ or $\phi(Q) = A$. Hence, there exists a point on $\odot(B, \abs{AC})$ such that its image under $\phi$ is $A$.

              Hence, $\phi$ is onto $\mathbb{R}^{2}$.
    \end{enumerate}
\end{proof}

% section 2/exercise 44
\begin{exercise}
    Prove that if $f: G_{1} \to G_{2}$ is a group isomorphism from the group $\anglebracket{G_{1}, {*}_{1}}$ to the group $\anglebracket{G_{2}, {*}_{2}}$, then $f^{-1}: G_{2} \to G_{1}$ is also a group isomorphism.
\end{exercise}

\begin{proof}
    $f: G_{1} \to G_{2}$ is one-to-one and onto, so that $f^{-1}: G_{2} \to G_{1}$ is also one-to-one and onto.

    Let $a', b'$ be elements of $G_{2}$, and $a = f^{-1}(a'), b = f^{-1}(b')$.
    \begin{align*}
        f^{-1}(a'\ {*}_{2}\ b') & = f^{-1}(f(a)\ {*}_{2}\ f(b))     \\
                                & = f^{-1}(f(a\ {*}_{1}\ b))        \\
                                & = a\ {*}_{1}\ b                   \\
                                & = f^{-1}(a')\ {*}_{1}\ f^{-1}(b')
    \end{align*}

    Thus $f^{-1}$ is also a group isomorphism.
\end{proof}

% section 2/exercise 45
\begin{exercise}
    Suppose that $G$ is a group with $n$ elements and $A\subseteq G$ has more than $\frac{n}{2}$ elements. Prove that for every $g\in G$, there exists $a, b\in A$ such that $a * b = g$.
\end{exercise}

\begin{proof}
    Let $B = \{ x' * g \vert x\in A \}$. $\phi: A\to B$ is defined as $\phi(x) = x' * g$, then $\phi$ is a bijection. Therefore, $A$ and $B$ have the same number of elements. According to the principle of inclusion and exclusion, $\abs{A\cap B} = \abs{A} + \abs{B} - \abs{A\cup B} > \frac{n}{2} + \frac{n}{2} - n = 0$, so $A\cap B$ is not empty. Let $b$ be an element of $A\cap B$. Since $\phi$ is a bijection, there exists $a\in A$ such that $a' * g = b$, equivalently, $g = (a * a') * g = a * (a' * g) = a * b$.

    Thus, there exist $a, b\in A$ such that $a * b = g$.
\end{proof}

\section{Abelian Examples}

\setcounter{exercise}{0}

In Exercises 1 through 9 compute the given arithmetic expression and give the answer in the form $a + bi$ for $a, b\in \mathbb{R}$.

% section 3/exercise 1
\begin{exercise}
    $i^{3}$
\end{exercise}

\begin{proof}
    $i^{3} = {i}^{2}i = -i = 0 + (-1)i$.
\end{proof}

% section 3/exercise 2
\begin{exercise}
    $i^{4}$
\end{exercise}

\begin{proof}
    $i^{4} = {i}^{2}{i}^{2} = (-1)\cdot (-1) = 1 = 1 + 0i$.
\end{proof}

% section 3/exercise 3
\begin{exercise}
    $i^{26}$
\end{exercise}

\begin{proof}
    $i^{26} = {i}^{2}{i}^{24} = -1 = -1 + 0i$.
\end{proof}

% section 3/exercise 4
\begin{exercise}
    ${(-i)}^{39}$
\end{exercise}

\begin{proof}
    ${(-i)}^{39} = {(-i)}^{3}{(-i)}^{36} = {(-i)}^{3} = i = 0 + 1i$.
\end{proof}

% section 3/exercise 5
\begin{exercise}
    $(3 - 2i)(6 + i)$
\end{exercise}

\begin{proof}
    $(3 - 2i)(6 + i) = 20 - 9i$.
\end{proof}

% section 3/exercise 6
\begin{exercise}
    $(8 + 2i)(3 - i)$
\end{exercise}

\begin{proof}
    $(8 + 2i)(3 - i) = 26 + (-2)i$.
\end{proof}

% section 3/exercise 7
\begin{exercise}
    $(2 - 3i)(4 + i) + (6 - 5i)$
\end{exercise}

\begin{proof}
    $(2 - 3i)(4 + i) + (6 - 5i) = (11 - 10i) + (6 - 5i) = 17 - 15i = 17 + (-15)i$.
\end{proof}

% section 3/exercise 8
\begin{exercise}
    ${(1+i)}^{3}$
\end{exercise}

\begin{proof}
    ${(1+i)}^{3} = 1^{3} + 3\cdot 1^{2}i + 3\cdot 1\cdot i^{2} + i^{3} = 1 + 3i - 3 - i = (-2) + 2i$.
\end{proof}

% section 3/exercise 9
\begin{exercise}
    ${(1-i)}^{5}$
\end{exercise}

\begin{proof}
    \begin{align*}
        {(1-i)}^{5} & = 1^{5} - 5\cdot 1^{4}i + 10\cdot 1^{3}i^{2} - 10\cdot 1^{2}i^{3} + 5\cdot 1\cdot i^{4} - i^{5} \\
                    & = 1 - 5i - 10 + 10i + 5 - i                                                                     \\
                    & = (-4) + 4i.
    \end{align*}
\end{proof}

% section 3/exercise 10
\begin{exercise}
    Find $\abs{5 - 12i}$.
\end{exercise}

\begin{proof}
    $\abs{5 - 12i} = \sqrt{5^{2} + 12^{2}} = 13$.
\end{proof}

% section 3/exercise 11
\begin{exercise}
    Find $\abs{\pi + ei}$.
\end{exercise}

\begin{proof}
    $\abs{\pi + ei} = \sqrt{\pi^{2} + e^{2}}$.
\end{proof}

In Exercises 12 through 15 write the given complex number $z$ in the polar form $\abs{z}(p + qi)$ where $\abs{p + qi} = 1$.

% section 3/exercise 12
\begin{exercise}
    $3 - 4i$
\end{exercise}

\begin{proof}
    $\abs{3 - 4i} = \sqrt{3^{2} + {(-4)}^{2}} = 5$.

    $3 - 4i = 5\left(\frac{3}{5} + \frac{-4}{5}i\right)$.
\end{proof}

% section 3/exercise 13
\begin{exercise}
    $-1 - i$
\end{exercise}

\begin{proof}
    $\abs{-1 - i} = \sqrt{{(-1)}^{2} + {(-1)}^{2}} = \sqrt{2}$.

    $-1 - i = \sqrt{2}\left(\frac{-\sqrt{2}}{2} + \frac{-\sqrt{2}}{2}i\right)$.
\end{proof}

% section 3/exercise 14
\begin{exercise}
    $12 + 5i$
\end{exercise}

\begin{proof}
    $\abs{12 + 5i} = \sqrt{12^{2} + 5^{2}} = 13$.

    $12 + 5i = 13\left( \frac{12}{13} + \frac{5}{13}i \right)$.
\end{proof}

% section 3/exercise 15
\begin{exercise}
    $-3 + 5i$
\end{exercise}

\begin{proof}
    $\abs{-3 + 5i} = \sqrt{{(-3)}^{2} + 5^{2}} = \sqrt{34}$.

    $-3 + 5i = \sqrt{34}\left( \frac{-3\sqrt{34}}{34} + \frac{5\sqrt{34}}{34}i \right)$.
\end{proof}

In Exercises 16 through 21, find all solutions in $\mathbb{C}$ of the given equation.

% section 3/exercise 16
\begin{exercise}
    $z^{4} = 1$
\end{exercise}

\begin{proof}
    In polar form, the equation is $z^{4} = \abs{z}^{4}(\cos (4\phi) + i\sin (4\phi))$.

    $z^{4} = 1$ so $\abs{z}^{4} = 1$, $\cos(4\phi) = 1$, and $\sin(4\phi) = 0$. Different values of $\phi$ in $0\le \phi < 2\pi$ are $0, \frac{\pi}{2}, \pi, \frac{3\pi}{2}$. So the roots are
    \[
        1,\quad i,\quad -1,\quad -i.
    \]
\end{proof}

% section 3/exercise 17
\begin{exercise}
    $z^{4} = -1$
\end{exercise}

\begin{proof}
    In polar form, the equation is $\abs{z}^{4}(\cos (4\phi) + i\sin (4\phi)) = 1(\cos\pi + i\sin\pi)$.

    The roots of the equation are
    \[
        \frac{\sqrt{2}}{2} + \frac{\sqrt{2}}{2}i,\quad \frac{-\sqrt{2}}{2} + \frac{\sqrt{2}}{2}i,\quad \frac{-\sqrt{2}}{2} + \frac{-\sqrt{2}}{2}i,\quad \frac{\sqrt{2}}{2} + \frac{-\sqrt{2}}{2}i.
    \]
\end{proof}

% section 3/exercise 18
\begin{exercise}
    $z^{3} = -125$
\end{exercise}

\begin{proof}
    In polar form, the equation is $\abs{z}^{3}(\cos (3\phi) + i\sin (3\phi)) = 5^{3}(\cos\pi + i\sin\pi)$.

    The roots of the equation are
    \[
        \frac{5}{2} + \frac{5\sqrt{3}}{2}i,\quad -5,\quad \frac{5}{2} + \frac{-5\sqrt{3}}{2}i.
    \]
\end{proof}

% section 3/exercise 19
\begin{exercise}
    $z^{3} = -27i$
\end{exercise}

\begin{proof}
    In polar form, the equation is $\abs{z}^{3}(\cos (3\phi) + i\sin (3\phi)) = 3^{3}(\cos\frac{3\pi}{2} + \sin\frac{3\pi}{2}i)$.

    The roots of the equation are
    \[
        3i,\quad \frac{-3\sqrt{3}}{2} + \frac{-3}{2}i,\quad\frac{3\sqrt{3}}{2} + \frac{-3}{2}i.
    \]
\end{proof}

% section 3/exercise 20
\begin{exercise}
    $z^{6} = 1$
\end{exercise}

\begin{proof}
    The roots of the equation are
    \[
        1,\quad \frac{1}{2} + \frac{\sqrt{3}}{2}i,\quad \frac{-1}{2} + \frac{\sqrt{3}}{2}i, -1,\quad \frac{-1}{2} + \frac{-\sqrt{3}}{2}i,\quad \frac{1}{2} + \frac{-\sqrt{3}}{2}i.
    \]
\end{proof}

% section 3/exercise 21
\begin{exercise}
    $z^{6} = -64$
\end{exercise}

\begin{proof}
    In polar form, the equation is $\abs{z}^{6}(\cos(6\phi) + i\sin(6\phi)) = 2^{6}(\cos\pi + i\sin\pi)$.

    % pi/6 + 2pi/6 * 0 = pi/6
    % pi/6 + 2pi/6 * 1 = 3pi/6 = pi/2
    % pi/6 + 2pi/6 * 2 = 5pi/6
    % pi/6 + 2pi/6 * 3 = 7pi/6
    % pi/6 + 2pi/6 * 4 = 9pi/6
    % pi/6 + 2pi/6 * 5 = 11pi/6

    The roots of the equation are
    \[
        \sqrt{3} + i,\quad 2i,\quad -\sqrt{3} + i,\quad -\sqrt{3} - i,\quad -2i,\quad \sqrt{3} - i.
    \]
\end{proof}

In Exercises 22 through 27, compute the given expression using the indicated modular addition.

% section 3/exercise 22
\begin{exercise}
    $10 {+}_{17} 16$
\end{exercise}

\begin{proof}
    $10 {+}_{17} 16 = 10 + 16 - 17 = 9$.
\end{proof}

% section 3/exercise 23
\begin{exercise}
    $14 {+}_{99} 92$
\end{exercise}

\begin{proof}
    $14 {+}_{99} 92 = 14 + 92 - 99 = 7$.
\end{proof}

% section 3/exercise 24
\begin{exercise}
    $3.141 {+}_{4} 2.718$
\end{exercise}

\begin{proof}
    $3.141 {+}_{4} 2.718 = 3.141 + 2.718 - 4 = 1.859$.
\end{proof}

% section 3/exercise 25
\begin{exercise}
    $\frac{1}{2} {+}_{1} \frac{7}{8}$
\end{exercise}

\begin{proof}
    $\frac{1}{2} {+}_{1} \frac{7}{8} = \frac{1}{2} + \frac{7}{8} - 1 = \frac{3}{8}$.
\end{proof}

% section 3/exercise 26
\begin{exercise}
    $\frac{3\pi}{4} {+}_{2\pi} \frac{3\pi}{2}$
\end{exercise}

\begin{proof}
    $\frac{3\pi}{4} {+}_{2\pi} \frac{3\pi}{2} = \frac{3\pi}{4} + \frac{3\pi}{2} - 2\pi = \frac{\pi}{4}$.
\end{proof}

% section 3/exercise 27
\begin{exercise}
    $2\sqrt{2} {+}_{\sqrt{32}} 3\sqrt{2}$
\end{exercise}

\begin{proof}
    $2\sqrt{2} {+}_{\sqrt{32}} 3\sqrt{2} = 2\sqrt{2} + 3\sqrt{2} - \sqrt{32} = 5\sqrt{2} - 4\sqrt{2} = \sqrt{2}$.
\end{proof}

% section 3/exercise 28
\begin{exercise}
    Explain why the expression $5 {+}_{6} 8$ in $\mathbb{R}_{6}$ makes no sense.
\end{exercise}

\begin{proof}
    The expression makes no sense because $8\notin \mathbb{R}_{6}$.
\end{proof}

In Exercises 29 through 34, find \textit{all} solutions $x$ of the given equation.

% section 3/exercise 29
\begin{exercise}
    $x {+}_{10} 7 = 3$ in $\mathbb{Z}_{10}$
\end{exercise}

\begin{proof}
    $x = 3 {+}_{10} 3 = 3 + 3 = 6$.
\end{proof}

% section 3/exercise 30
\begin{exercise}
    $x {+}_{2\pi} \pi = \frac{\pi}{2}$ in $\mathbb{R}_{2\pi}$
\end{exercise}

\begin{proof}
    $x = \frac{\pi}{2} {+}_{2\pi} \pi = \frac{\pi}{2} + \pi = \frac{3\pi}{2}$.
\end{proof}

% section 3/exercise 31
\begin{exercise}
    $x {+}_{7} x = 3$ in $\mathbb{Z}_{7}$
\end{exercise}

\begin{proof}
    $x = 5$.
\end{proof}

% section 3/exercise 32
\begin{exercise}
    $x {+}_{13} x {+}_{13} x = 5$ in $\mathbb{Z}_{13}$
\end{exercise}

\begin{proof}
    $x = 6$.
\end{proof}

% section 3/exercise 33
\begin{exercise}
    $x {+}_{12} x = 2$ in $\mathbb{Z}_{12}$
\end{exercise}

\begin{proof}
    $x = 1$, or $x = 7$.
\end{proof}

% section 3/exercise 34
\begin{exercise}
    $x {+}_{8} x {+}_{8} x {+}_{8} x = 4$ in $\mathbb{Z}_{8}$
\end{exercise}

\begin{proof}
    $x = 1$, or $x = 3$, or $x = 5$, or $x = 7$.
\end{proof}

% section 3/exercise 35
\begin{exercise}
    Prove or give a counterexample to the statement that for any $n\in\mathbb{Z}^{+}$ and $a\in\mathbb{Z}_{n}$, the equation $x {+}_{n} x = a$ has at most two solutions in $\mathbb{Z}_{n}$.
\end{exercise}

\begin{proof}
    When $n$ is even and $a$ is odd. For every $x\in\mathbb{Z}_{n}$, $x {+}_{n} x$ is even. So $x {+}_{n} x = a$ has no solutions in this case.

    When $n$ is even and $a$ is even, $x = \frac{a}{2}$ and $x = \frac{a + n}{2}$ are two solutions. Suppose that $x = y$ is a solution (1).
    \begin{enumerate}[label={\textbf{Case \arabic*.}},itemindent=1cm]
        \item $0\le y + y < n$.

              If $y < \frac{a}{2}$, then $0\le y + y < a$, which is a contradiction to (1).

              If $y > \frac{a}{2}$, then $0\le a < y + y < n$, which is also a contradiction to (1).

              So $y = \frac{a}{2}$.
        \item $y + y \ge n$.

              If $y < \frac{a + n}{2}$, then $y + y - n < a$, which is a contradiction to (1).

              If $y > \frac{a + n}{2}$, then $y + y - n > a$, which is a contradiction to (1).

              So $y = \frac{a + n}{2}$.
    \end{enumerate}

    So when $n$ is even and $a$ is even, the equation has two solutions.

    When $n$ is odd, and $a$ is even, $x = \frac{a}{2}$ is a solution. Suppose that $x = y$ is a solution (2).
    \begin{enumerate}[label={\textbf{Case \arabic*}},itemindent=1cm]
        \item $0\le y + y < n$.

              If $y < \frac{a}{2}$, then $0\le y + y < a$, which is a contradiction to (2).

              If $y > \frac{a}{2}$, then $0\le a < y + y < n$, which is also a contradiction to (2).

              So $y = \frac{a}{2}$.
        \item $y + y\ge n$.

              $y + y - n$ is odd. Meanwhile, $a$ is even. So $x = y$ cannot be a solution.
    \end{enumerate}

    So when $n$ is odd and $a$ is even, the equation has one solution.

    When $n$ is odd, and $a$ is odd, $x = \frac{a+n}{2}$ is a solution. Suppose that $x = y$ is a solution (3).
    \begin{enumerate}[label={\textbf{Case \arabic*}},itemindent=1cm]
        \item $0\le y + y < n$.

              $y + y$ is even. Meanwhile, $a$ is odd. So $x = y$ cannot be a solution.

        \item $y + y\ge n$.

              If $y < \frac{a+n}{2}$, then $y + y - n < a$, which is a contradiction to (3).

              If $y > \frac{a+n}{2}$, then $y + y - n > a$, which is also a contradiction to (3).

              So $y = \frac{a+n}{2}$.
    \end{enumerate}

    So when $n$ is odd and $a$ is odd, the equation has one solution.

    In conclusion, the equation has at most two solution.
\end{proof}

% section 3/exercise 36
\begin{exercise}
    Prove or give a counterexample to the statement that for any $n\in\mathbb{Z}^{+}$ and $a\in\mathbb{Z}_{n}$, if $n$ is not a multiple of $3$, then the equation $x {+}_{n} x {+}_{n} x = a$ has exactly one solution in $\mathbb{Z}_{m}$.
\end{exercise}

\begin{proof}
    If $n$ is not a multiple of $3$ then $n$ and $3$ are relatively prime. According to Bezout's theorem, there exist integer $s, t$ such that $3s + nt = 1$. If $3s + nt = 1$, then $3(as) + nat = a$. So $3(as)\equiv a\pmod{n}$. From this, we deduce two things: the equation has a solution which is $x = x_{0}$ such that $x_{0}\equiv as\pmod{n}$, and if $x = x_{1}$ is a solution, $x_{1}\equiv as \equiv x_{0} \pmod{n}$. Therefore, the equation has exactly one solution.
\end{proof}

% section 3/exercise 37
\begin{exercise}
    There is an isomorphism of $U_{8}$ with $\mathbb{Z}_{8}$ in which $\zeta = e^{i(\pi/4)}\leftrightarrow 5$ and $\zeta^{2}\leftrightarrow 2$. Find the element of $\mathbb{Z}_{8}$ that corresponds to each of the remaining six elements $\zeta^{m}$ in $U_{8}$ for $m = 0, 3, 4, 5, 6$, and $7$.
\end{exercise}

\begin{proof}
    $\zeta^{0} \leftrightarrow 0$, $\zeta^{3} \leftrightarrow 7$, $\zeta^{4} \leftrightarrow 4$, $\zeta^{5}\leftrightarrow 1$, $\zeta^{6}\leftrightarrow 6$, and $\zeta^{7}\leftrightarrow 3$.
\end{proof}

% section 3/exercise 38
\begin{exercise}
    There is an isomorphism of $U_{7}$ with $\mathbb{Z}_{7}$ in which $\zeta = e^{i(2\pi/7)}\leftrightarrow 4$. Find the element in $\mathbb{Z}_{7}$ to which $\zeta^{m}$ must correspond for $m = 0, 2, 3, 4, 5$, and $6$.
\end{exercise}

\begin{proof}
    $\zeta^{0} \leftrightarrow 0$, $\zeta^{2} \leftrightarrow 1$, $\zeta^{3} \leftrightarrow 5$, $\zeta^{4} \leftrightarrow 2$, $\zeta^{5} \leftrightarrow 6$, and $\zeta^{6} \leftrightarrow 3$.
\end{proof}

% section 3/exercise 39
\begin{exercise}
    Why can there be no isomorphism of $U_{6}$ with $\mathbb{Z}_{6}$ in which $\zeta = e^{i(\pi/3)}$ corresponds to $4$?
\end{exercise}

\begin{proof}
    Assume that there is such an isomorphism.

    Then $\zeta^{3}$ corresponds to $4 {+}_{12} 4 {+}_{12} 4 = 0$. On the other hand, $\zeta^{0}$ corresponds to $0$, which is a contradiction.

    Hence there can be no isomorphism of $U_{6}$ with $\mathbb{Z}_{6}$ in which $\zeta = e^{i(\pi/3)}\leftrightarrow 4$.
\end{proof}

% section 3/exercise 40
\begin{exercise}
    Derive the formulas
    \[
        \sin(a + b) = \sin a\cos b + \cos a\sin b
    \]

    and
    \[
        \cos(a + b) = \cos a\cos b - \sin a\sin b
    \]

    by using Euler's formula and computing $e^{ia}e^{ib}$.
\end{exercise}

\begin{proof}
    $e^{ia}e^{ib} = e^{i(a+b)} = \cos(a+b) + i\sin(a+b)$.

    $e^{ia}e^{ib} = (\cos a + i\sin a)(\cos b + i\sin b) = (\cos a\cos b - \sin a\sin b) + i(\sin a\cos b + \cos a\sin b)$.

    Thus $\sin(a + b) = \sin a\cos b + \cos a\sin b$ and $\cos(a + b) = \cos a\cos b - \sin a\sin b$.
\end{proof}

% section 3/exercise 41
\begin{exercise}
    Let $z_{1} = \abs{z_{1}}(\cos{\theta_{1}} + i\sin{\theta_{1}})$ and $z_{2} = \abs{z_{2}}(\cos{\theta_{2}} + i\sin{\theta_{2}})$. Use the trigonometric identities in Exercise 38 to derive $z_{1}z_{2} = \abs{z_{1}}\abs{z_{2}}(\cos{(\theta_{1} + \theta_{2})} + i\sin{(\theta_{1} + \theta_{2})})$.
\end{exercise}

\begin{proof}
    \begin{align*}
        z_{1}z_{2} & = \abs{z_{1}}\abs{z_{2}}(\cos a + i\sin a)(\cos b + i\sin b)                             \\
                   & = \abs{z_{1}}\abs{z_{2}}((\cos a\cos b - \sin a\sin b) + i(\sin a\cos b + \cos a\sin b)) \\
                   & = \abs{z_{1}}\abs{z_{2}}(\cos{(a+b)} + i\sin{(a+b)}).
    \end{align*}
\end{proof}

% section 3/exercise 42
\begin{exercise}
    \begin{enumerate}[topsep=0pt,itemsep=0pt,label={\textbf{\alph*.}}]
        \item Derive a formula for $\cos{3\theta}$ in terms of $\sin{\theta}$ and $\cos{\theta}$ using Euler's formula.
        \item Derive the formula $\cos{3\theta} = 4\cos^{3}{\theta} - 3\cos{\theta}$ from part (a) and the identity $\sin^{2}{\theta} + \cos^{2}{\theta} = 1$.
    \end{enumerate}
\end{exercise}

\begin{proof}
    \begin{enumerate}[topsep=0pt,itemsep=0pt,label={\textbf{\alph*.}}]
        \item \begin{align*}
                  \cos{3\theta} & = \cos{2\theta}\cos{\theta} - \sin{\theta}\sin{2\theta}                             \\
                                & = \cos{\theta}(\cos^{2}{\theta} - \sin^{2}{\theta}) - 2\sin^{2}{\theta}\cos{\theta} \\
                                & = \cos^{3}{\theta} - \cos{\theta}\sin^{2}{\theta} - 2\sin^{2}{\theta}\cos{\theta}.
              \end{align*}
        \item \begin{align*}
                  \cos{3\theta} & = \cos^{3}{\theta} - \cos{\theta}\sin^{2}{\theta} - 2\sin^{2}{\theta}\cos{\theta}             \\
                                & = \cos^{3}{\theta} - \cos{\theta}(1 - \cos^{2}{\theta}) - 2(1 - \cos^{2}{\theta})\cos{\theta} \\
                                & = \cos^{3}{\theta} - \cos{\theta} + \cos^{3}{\theta} + 2\cos^{3}{\theta} - 2\cos{\theta}      \\
                                & = 4\cos^{3}{\theta} - 3\cos{\theta}.
              \end{align*}
    \end{enumerate}
\end{proof}

% section 3/exercise 43
\begin{exercise}
    Recall the power series expansions
    \begin{align*}
        e^{x}  & = 1 + x + \frac{x^{2}}{2!} + \frac{x^{3}}{3!} + \frac{x^{4}}{4!} + \cdots + \frac{x^{n}}{n!} + \cdots,                            \\
        \sin x & = x - \frac{x^{3}}{3!} + \frac{x^{5}}{5!} - \frac{x^{7}}{7!} + \cdots + {(-1)}^{n-1}\frac{x^{2n-1}}{(2n-1)!} + \cdots, \text{and} \\
        \cos x & = 1 - \frac{x^{2}}{2!} + \frac{x^{4}}{4!} - \frac{x^{6}}{6!} + \cdots + {(-1)}^{n}\frac{x^{2n}}{(2n)!} + \cdots
    \end{align*}

    from calculus. Derive Euler's formula $e^{i\theta} = \cos{\theta} + i\sin{\theta}$ formally from these three series expansions.
\end{exercise}

\begin{proof}
    \begin{align*}
        e^{i\theta} & = 1 + i\theta + \frac{{(i\theta)}^{2}}{2!} + \frac{{(i\theta)}^{3}}{3!} + \frac{{(i\theta)}^{4}}{4!} + \cdots                                                                 \\
                    & = \left(1 + \frac{{(i\theta)}^{2}}{2!} + \frac{{(i\theta)}^{4}}{4!} + \cdots\right) + \left(i\theta + \frac{{(i\theta)}^{3}}{3!} + \frac{{(i\theta)}^{5}}{5!} + \cdots\right) \\
                    & = \left(1 - \frac{x^{2}}{2!} + \frac{x^{4}}{4!} + \cdots \right) + \left(i\theta - i\frac{\theta^{3}}{3!} + i\frac{\theta^{5}}{5!} - \cdots \right)                           \\
                    & = \cos{\theta} + i\sin{\theta}.\qedhere
    \end{align*}
\end{proof}

% section 3/exercise 44
\begin{exercise}
    Prove that for any $n\in\mathbb{Z}^{+}$, $\anglebracket{\mathbb{Z}_{n}, {+}_{n}}$ is associative without using the fact that $U_{n}$ is associative.
\end{exercise}

\begin{proof}
    Let $a, b, c$ be elements of $\mathbb{Z}_{n}$.

    $a {+}_{n} b$ and $a + b$ are congruent modulo $n$. So $(a {+}_{n} b) {+}_{n} c$ and $(a + b) + c$ are congruent modulo $n$.

    $b {+}_{n} c$ and $b + c$ are congruent modulo $n$. So $a {+}_{n} (b {+}_{n} c)$ and $a + (b + c)$ are congruent modulo $n$.

    Since $(a + b) + c$ and $a + (b + c)$ are congruent modulo $n$, we conclude that $(a {+}_{n} b) {+}_{n} c$ and $a {+}_{n} (b {+}_{n} c)$ are congruent modulo $n$. On the other hand, $0\le (a {+}_{n} b) {+}_{n} c, a {+}_{n} (b {+}_{n} c) < n$, so $(a {+}_{n} b) {+}_{n} c + a {+}_{n} (b {+}_{n} c)$.

    Thus $\anglebracket{\mathbb{Z}_{n}, {+}_{n}}$ is associative.
\end{proof}

% section 3/exercise 45
\begin{exercise}
    Let $b, c\in\mathbb{R}^{+}$. Find a one-to-one and onto function $f: \mathbb{R}_{b} \to \mathbb{R}_{c}$ that has the homomorphism property. Conclude that $\mathbb{R}_{c}$ is an abelian group that is isomorphic with $U$.
\end{exercise}

\begin{proof}
    We define $f(x) = \frac{cx}{b}$. Let $x, y$ be elements of $\mathbb{R}_{b}$. $f$ is a one-to-one and onto function.

    If $0\le x + y < b$,
    \[
        f(x {+}_{b} y) = f(x + y) = \frac{c(x+y)}{b} = \frac{cx}{b} + \frac{cy}{b} = f(x) {+}_{c} f(y).
    \]

    If $x + y \ge b$,
    \[
        f(x {+}_{b} y) = f(x + y - b) = \frac{c(x + y - b)}{b} = \frac{cx}{b} + \frac{cy}{b} - c = f(x) + f(y) - c = f(x) {+}_{c} f(y).
    \]

    Therefore, $\mathbb{R}_{b}$ and $\mathbb{R}_{c}$ are isomorphic, for any positive real numbers $b, c$. Since $U\simeq\mathbb{R}_{2\pi}$ and $\mathbb{R}_{b}$ is an abelian group, we conclude that $\mathbb{R}_{c}$ is an abelian group that is isomorphic with $U$.
\end{proof}

% section 3/exercise 46
\begin{exercise}
    Prove that for any $n\geq 1$, $U_{n}$ is a group.
\end{exercise}

\begin{proof}
    $U_{n}$ has $n$ elements $\zeta^{0}, \zeta^{1}, \zeta^{2}, \ldots, \zeta^{n-1}$, where $\zeta = e^{i(2\pi/n)}$.

    $U_{n}$ is closed under multiplication. $U_{n}$ is associative because complex number multiplication is associative. $U_{n}$ has an identity element, which is $1 = \zeta^{0}$. Each element of $U_{n}$ has an inverse, $\zeta^{0}\zeta^{0} = 1$ and $\zeta^{m}\zeta^{n-m} = \zeta^{n} = 1$ (if $0 < m < n$).

    Hence $U_{n}$ is a group.
\end{proof}

\section{Nonabelian Examples}
\setcounter{exercise}{0}

\subsection*{Computation}

In Exercises l through 5, compute the indicated product involving the following permutations in $S_{6}$:

\[
    \sigma = \begin{pmatrix}
        1 & 2 & 3 & 4 & 5 & 6 \\
        3 & 1 & 4 & 5 & 6 & 2
    \end{pmatrix},
    \qquad
    \tau = \begin{pmatrix}
        1 & 2 & 3 & 4 & 5 & 6 \\
        2 & 4 & 1 & 3 & 6 & 5
    \end{pmatrix},
    \qquad
    \mu = \begin{bmatrix}
        1 & 2 & 3 & 4 & 5 & 6 \\
        5 & 2 & 4 & 3 & 1 & 6
    \end{bmatrix}
\]

% section 4/exercise 1
\begin{exercise}
    $\tau\sigma$
\end{exercise}

\begin{proof}
    \[
        \tau\sigma =
        \begin{pmatrix}
            3 & 1 & 4 & 5 & 6 & 2 \\
            1 & 2 & 3 & 6 & 5 & 4
        \end{pmatrix},
        \begin{pmatrix}
            1 & 2 & 3 & 4 & 5 & 6 \\
            3 & 1 & 4 & 5 & 6 & 2
        \end{pmatrix} =
        \begin{pmatrix}
            1 & 2 & 3 & 4 & 5 & 6 \\
            1 & 2 & 3 & 6 & 5 & 4
        \end{pmatrix}
    \]
\end{proof}

% section 4/exercise 2
\begin{exercise}
    ${\tau}^{2}\sigma$
\end{exercise}

\begin{proof}
    \[
        {\tau}^{2} =
        \begin{pmatrix}
            2 & 4 & 1 & 3 & 6 & 5 \\
            4 & 3 & 2 & 1 & 5 & 6
        \end{pmatrix}
        \begin{pmatrix}
            1 & 2 & 3 & 4 & 5 & 6 \\
            2 & 4 & 1 & 3 & 6 & 5
        \end{pmatrix} =
        \begin{pmatrix}
            1 & 2 & 3 & 4 & 5 & 6 \\
            4 & 3 & 2 & 1 & 5 & 6
        \end{pmatrix}
    \]
    \[
        {\tau}^{2}\sigma =
        \begin{pmatrix}
            3 & 1 & 4 & 5 & 6 & 2 \\
            2 & 4 & 1 & 5 & 6 & 3
        \end{pmatrix}
        \begin{pmatrix}
            1 & 2 & 3 & 4 & 5 & 6 \\
            3 & 1 & 4 & 5 & 6 & 2
        \end{pmatrix} =
        \begin{pmatrix}
            1 & 2 & 3 & 4 & 5 & 6 \\
            2 & 4 & 1 & 5 & 6 & 3
        \end{pmatrix}
    \]
\end{proof}

% section 4/exercise 3
\begin{exercise}
    $\mu{\sigma}^{2}$
\end{exercise}

\begin{proof}
    \[
        {\sigma}^{2} =
        \begin{pmatrix}
            3 & 1 & 4 & 5 & 6 & 2 \\
            4 & 3 & 5 & 6 & 2 & 1
        \end{pmatrix}
        \begin{pmatrix}
            1 & 2 & 3 & 4 & 5 & 6 \\
            3 & 1 & 4 & 5 & 6 & 2
        \end{pmatrix} =
        \begin{pmatrix}
            1 & 2 & 3 & 4 & 5 & 6 \\
            4 & 3 & 5 & 6 & 2 & 1
        \end{pmatrix}
    \]
    \[
        \mu{\sigma}^{2} =
        \begin{bmatrix}
            4 & 3 & 5 & 6 & 2 & 1 \\
            3 & 4 & 1 & 6 & 2 & 5
        \end{bmatrix}
        \begin{pmatrix}
            1 & 2 & 3 & 4 & 5 & 6 \\
            4 & 3 & 5 & 6 & 2 & 1
        \end{pmatrix} =
        \begin{pmatrix}
            1 & 2 & 3 & 4 & 5 & 6 \\
            3 & 4 & 1 & 6 & 2 & 5
        \end{pmatrix}
    \]
\end{proof}

% section 4/exercise 4
\begin{exercise}
    ${\sigma}^{-2}\tau$
\end{exercise}

\begin{proof}
    \[
        {\sigma}^{-1} =
        \begin{pmatrix}
            1 & 2 & 3 & 4 & 5 & 6 \\
            2 & 6 & 1 & 3 & 4 & 5
        \end{pmatrix}
    \]
    \[
        {\sigma}^{-2} =
        \begin{pmatrix}
            2 & 6 & 1 & 3 & 4 & 5 \\
            6 & 5 & 2 & 1 & 3 & 4
        \end{pmatrix} =
        \begin{pmatrix}
            1 & 2 & 3 & 4 & 5 & 6 \\
            2 & 6 & 1 & 3 & 4 & 5
        \end{pmatrix} =
        \begin{pmatrix}
            1 & 2 & 3 & 4 & 5 & 6 \\
            6 & 5 & 2 & 1 & 3 & 4
        \end{pmatrix}
    \]
    \[
        {\sigma}^{-2}\tau =
        \begin{pmatrix}
            2 & 4 & 1 & 3 & 6 & 5 \\
            5 & 1 & 6 & 2 & 4 & 3
        \end{pmatrix}
        \begin{pmatrix}
            1 & 2 & 3 & 4 & 5 & 6 \\
            2 & 4 & 1 & 3 & 6 & 5
        \end{pmatrix} =
        \begin{pmatrix}
            1 & 2 & 3 & 4 & 5 & 6 \\
            5 & 1 & 6 & 2 & 4 & 3
        \end{pmatrix}
    \]
\end{proof}

% section 4/exercise 5
\begin{exercise}
    ${\sigma}^{-1}\tau\sigma$
\end{exercise}

\begin{proof}
    \[
        {\sigma}^{-1} =
        \begin{pmatrix}
            1 & 2 & 3 & 4 & 5 & 6 \\
            2 & 6 & 1 & 3 & 4 & 5
        \end{pmatrix}
    \]
    \[
        {\sigma}^{-1}\tau =
        \begin{pmatrix}
            2 & 4 & 1 & 3 & 6 & 5 \\
            6 & 3 & 2 & 1 & 5 & 4
        \end{pmatrix}
        \begin{pmatrix}
            1 & 2 & 3 & 4 & 5 & 6 \\
            2 & 4 & 1 & 3 & 6 & 5
        \end{pmatrix} =
        \begin{pmatrix}
            1 & 2 & 3 & 4 & 5 & 6 \\
            6 & 3 & 2 & 1 & 5 & 4
        \end{pmatrix}
    \]
    \[
        {\sigma}^{-1}\tau{\sigma} =
        \begin{pmatrix}
            3 & 1 & 4 & 5 & 6 & 2 \\
            2 & 6 & 1 & 5 & 4 & 3
        \end{pmatrix}
        \begin{pmatrix}
            1 & 2 & 3 & 4 & 5 & 6 \\
            3 & 1 & 4 & 5 & 6 & 2
        \end{pmatrix} =
        \begin{pmatrix}
            1 & 2 & 3 & 4 & 5 & 6 \\
            2 & 6 & 1 & 5 & 4 & 3
        \end{pmatrix}
    \]
\end{proof}

In Exercises 6 through 9, compute the expressions shown for the permutations $\sigma$, $\tau$, and $\mu$, defined prior to Exercise 1.

% section 4/exercise 6
\begin{exercise}
    $\sigma^{6}$
\end{exercise}

\begin{proof}
    $\sigma^{5}(1) = \sigma^{4}(3) = \sigma^{3}(4) = \sigma^{2}(5) = \sigma(6) = 1$.

    Therefore, $\sigma^{5} = \iota$ and $\sigma^{6} = \sigma$.
    \[
        \sigma^{6} = \sigma =
        \begin{pmatrix}
            1 & 2 & 3 & 4 & 5 & 6 \\
            3 & 1 & 4 & 5 & 6 & 2
        \end{pmatrix}
    \]
\end{proof}

% section 4/exercise 7
\begin{exercise}
    $\mu^{2}$
\end{exercise}

\begin{proof}
    \[
        \mu^{2} =
        \begin{pmatrix}
            5 & 2 & 4 & 3 & 1 & 6 \\
            1 & 2 & 3 & 4 & 5 & 6
        \end{pmatrix}
        \begin{pmatrix}
            1 & 2 & 3 & 4 & 5 & 6 \\
            5 & 2 & 4 & 3 & 1 & 6
        \end{pmatrix} =
        \begin{pmatrix}
            1 & 2 & 3 & 4 & 5 & 6 \\
            1 & 2 & 3 & 4 & 5 & 6
        \end{pmatrix} =
        \iota
    \]
\end{proof}

% section 4/exercise 8
\begin{exercise}
    $\sigma^{100}$
\end{exercise}

\begin{proof}
    According to Exercise 6, $\sigma^{5} = \iota$, so $\sigma^{100} = {(\sigma^{5})}^{20} = \iota^{20} = \iota$.
\end{proof}

% section 4/exercise 9
\begin{exercise}
    $\mu^{100}$
\end{exercise}

\begin{proof}
    According to Exercise 7, $\mu^{2} = \iota$, so $\mu^{100} = {(\mu^{2})}^{50} = \iota^{50} = \iota$.
\end{proof}

% section 4/exercise 10
\begin{exercise}
    Convert the permutations $\sigma$, $\tau$, and $\mu$, defined prior to Exercise 1 to disjoint cycle notation.
\end{exercise}

\begin{proof}
    \[
        \begin{split}
            \sigma & = (1, 3, 4, 5, 6, 2) \\
            \tau   & = (1, 2, 4, 3)(5, 6) \\
            \mu    & = (1, 5)(3, 4)
        \end{split}
    \]
\end{proof}

% section 4/exercise 11
\begin{exercise}
    Convert the following permutations in $S_{8}$ from disjoint cycle notation to two-row notation.
    \begin{enumerate}[label={\textbf{\arabic*.}}]
        \item $(1, 4, 5)(2, 3)$
        \item $(1, 8, 5)(2, 6, 7, 3, 4)$
        \item $(1, 2, 3)(4, 5)(6, 7, 8)$
    \end{enumerate}
\end{exercise}

\begin{proof}
    \begin{enumerate}[label={\textbf{\arabic*.}}]
        \item $(1, 4, 5)(2, 3) = \begin{pmatrix}
                      1 & 2 & 3 & 4 & 5 & 6 & 7 & 8 \\
                      4 & 3 & 2 & 5 & 1 & 6 & 7 & 8
                  \end{pmatrix}$.
        \item $(1, 8, 5)(2, 6, 7, 3, 4) = \begin{pmatrix}
                      1 & 2 & 3 & 4 & 5 & 6 & 7 & 8 \\
                      8 & 6 & 4 & 2 & 1 & 7 & 3 & 5
                  \end{pmatrix}$.
        \item $(1, 2, 3)(4, 5)(6, 7, 8) = \begin{pmatrix}
                      1 & 2 & 3 & 4 & 5 & 6 & 7 & 8 \\
                      2 & 3 & 1 & 5 & 4 & 7 & 8 & 6
                  \end{pmatrix}$
    \end{enumerate}
\end{proof}

% section 4/exercise 12
\begin{exercise}
    Compute the permutation products.
    \begin{enumerate}[label={\textbf{\alph*.}}]
        \item $(1, 5, 2, 4)(1, 5, 2, 3)$
        \item $(1, 5, 3)(1, 2, 3, 4, 5, 6){(1, 5, 3)}^{-1}$
        \item ${({(1, 6, 7, 2)}^{2}{(4, 5, 2, 6)}^{-1}(1, 7, 3))}^{-1}$
        \item $(1, 6)(1, 5)(1, 4)(1, 3)(1, 2)$
    \end{enumerate}
\end{exercise}

\begin{proof}
    \begin{enumerate}[label={\textbf{\alph*.}}]
        \item $(1, 5, 2, 4)(1, 5, 2, 3) = (1, 2, 3, 5, 4)$
        \item \begin{align*}
                    & (1, 5, 3)(1, 2, 3, 4, 5, 6){(1, 5, 3)}^{-1} \\
                  = & (1, 5, 3)(1, 2, 3, 4, 5, 6)(1, 3, 5)        \\
                  = & (1, 4, 3, 6, 5, 2)
              \end{align*}
        \item \begin{align*}
                    & {({(1, 6, 7, 2)}^{2}{(4, 5, 2, 6)}^{-1}(1, 7, 3))}^{-1} \\
                  = & (1, 3, 7)(4, 5, 2, 6){(1, 2, 7, 6)}^{2}                 \\
                  = & (1, 3, 7)(4, 5, 2, 6)(1, 7)(2, 6)                       \\
                  = & (2, 4, 5)(3, 7)
              \end{align*}
        \item \begin{align*}
                    & (1, 6)(1, 5)(1, 4)(1, 3)(1, 2) \\
                  = & (1, 2)(3, 4)(5, 6)
              \end{align*}
    \end{enumerate}
\end{proof}

% section 4/exercise 13
\begin{exercise}
    Compute the following elements of $D_{12}$. Write your answer in standard form.
    \begin{enumerate}[label={\textbf{\alph*.}}]
        \item $\mu{\rho}^{2}\mu{\rho}^{8}$
        \item $\mu{\rho}^{10}\mu{\rho}^{-1}$
        \item $\rho\mu{\rho}^{-1}$
        \item ${(\mu{\rho}^{3}{\mu}^{-1}{\rho}^{-1})}^{-1}$
    \end{enumerate}
\end{exercise}

\begin{proof}
    \begin{enumerate}[label={\textbf{\alph*.}}]
        \item $\mu{\rho}^{2}\mu{\rho}^{8} = \mu{\rho}^{2}(\mu{\rho}^{8}) = \mu{\rho}^{2}({\rho}^{4}\mu) = \mu{\rho}^{6}\mu = {(\mu{\rho}^{6})}\mu = {({\rho}^{6}\mu)}\mu = \rho^{6}$
        \item $\mu{\rho}^{10}\mu{\rho}^{-1} = (\mu{\rho}^{10})\mu{\rho}^{-1} = ({\rho}^{2}\mu)\mu{\rho}^{11} = {\rho}^{2}(\mu\mu){\rho}^{-1} = \rho$
        \item $\rho\mu{\rho}^{-1} = (\rho\mu){\rho}^{-1} = \mu{\rho}^{11}{\rho}^{-1} = \mu{\rho}^{10}$
        \item ${(\mu{\rho}^{3}{\mu}^{-1}{\rho}^{-1})}^{-1} = \rho\mu\rho^{9}\mu = (\rho\mu)\rho^{9}\mu = (\mu\rho^{11})\rho^{9}\mu = \mu(\rho^{11}\rho^{9})\mu = \mu{\rho}^{8}\mu = ({\rho}^{4}\mu)\mu = \rho^{4}$
    \end{enumerate}
\end{proof}

% section 4/exercise 14
\begin{exercise}
    Write the group table for $D_{3}$. Compare the group tables for $D_{3}$ and $S_{3}$. Are the groups isomorphic?
\end{exercise}

\begin{proof}
    Group table for $S_{3}$
    \[
        \begin{array}{c|cccccc}
                      & \iota     & (1, 2, 3) & (1, 3, 2) & (1, 2)    & (2, 3)    & (1, 3)    \\
            \hline
            \iota     & \iota     & (1, 2, 3) & (3, 1, 2) & (1, 2)    & (2, 3)    & (1, 3)    \\
            (1, 2, 3) & (1, 2, 3) & (1, 3, 2) & \iota     & (1, 3)    & (1, 2)    & (2, 3)    \\
            (1, 3, 2) & (1, 3, 2) & \iota     & (1, 2, 3) & (2, 3)    & (1, 3)    & (1, 2)    \\
            (1, 2)    & (1, 2)    & (2, 3)    & (1, 3)    & \iota     & (1, 2, 3) & (1, 3, 2) \\
            (2, 3)    & (2, 3)    & (1, 3)    & (1, 2)    & (1, 3, 2) & \iota     & (1, 2, 3) \\
            (1, 3)    & (1, 3)    & (1, 2)    & (2, 3)    & (1, 2, 3) & (1, 3, 2) & \iota
        \end{array}
    \]

    Group table for $D_{3}$
    \[
        \begin{array}{c|cccccc}
                        & \iota       & \rho        & \rho^{2}    & \mu         & \mu\rho     & \mu\rho^{2} \\
            \hline
            \iota       & \iota       & \rho        & \rho^{2}    & \mu         & \mu\rho     & \mu\rho^{2} \\
            \rho        & \rho        & \rho^{2}    & \iota       & \mu\rho^{2} & \mu         & \mu\rho     \\
            \rho^{2}    & \rho^{2}    & \iota       & \rho        & \mu\rho     & \mu\rho^{2} & \mu         \\
            \mu         & \mu         & \mu\rho     & \mu\rho^{2} & \iota       & \rho        & \rho^{2}    \\
            \mu\rho     & \mu\rho     & \mu\rho^{2} & \mu         & \rho^{2}    & \iota       & \rho        \\
            \mu\rho^{2} & \mu\rho^{2} & \mu         & \mu\rho     & \rho        & \rho^{2}    & \iota
        \end{array}
    \]

    $D_{3}$ and $S_{3}$ are isomorphic.
\end{proof}

Let $A$ be a set and let $\sigma\in S_{A}$. For a fixed $a\in A$, the set
\[
    \mathcal{O}_{a,\sigma} = \{ \sigma^{n}(a) \vert n\in\mathbb{Z} \}
\]

is the \textbf{orbit} of $a$ \textbf{under} $\sigma$. In Exercise 15 through 17, find the orbit of $1$ under the permutation defined prior to Exercise 1.

% section 4/exercise 15
\begin{exercise}
    $\sigma$
\end{exercise}

\begin{proof}
    $\mathcal{O}_{1,\sigma} = \{ 1, 2, 3, 4, 5, 6 \}$
\end{proof}

% section 4/exercise 16
\begin{exercise}
    $\tau$
\end{exercise}

\begin{proof}
    $\mathcal{O}_{1,\tau} = \{ 1, 2, 3, 4 \}$
\end{proof}

% section 4/exercise 17
\begin{exercise}
    $\mu$
\end{exercise}

\begin{proof}
    $\mathcal{P}_{1,\mu} = \{ 1, 5 \}$
\end{proof}

% section 4/exercise 18
\begin{exercise}
    Verify that $H = \{ \iota, \mu, \rho^{2}, \mu\rho^{2} \}\subseteq D_{4}$ is a group using the operation function composition.
\end{exercise}

\begin{proof}
    $\iota\circ\mu = \mu\circ\iota = \mu, \iota\circ\rho^{2} = \rho^{2}\circ\iota = \rho^{2}, \iota\circ\mu\rho^{2} = \mu\rho^{2}\circ\iota = \mu\rho^{2}$.

    $\rho^{2}\circ\rho^{2} = \mu\circ\mu = \mu\rho^{2}\circ\mu\rho^{2} = \iota$.

    $\mu\circ\rho^{2} = \mu\rho^{2}$, $\rho^{2}\circ\mu = \mu\rho^{2}$.

    $\mu\circ\mu\rho^{2} = \rho^{2}$, $\mu\rho^{2}\circ\mu = \mu\mu\rho^{2} = \rho^{2}$.

    $\rho^{2}\circ\mu\rho^{2} = \mu\rho^{2}\rho^{2} = \mu$.

    So, $H$ is closed under function composition, $H$ is associative, $H$ has an identity element ($\iota$), and each element has an inverse. Therefore, $H$ is a group.
\end{proof}

% section 4/exercise 19
\begin{exercise}
    \begin{enumerate}[label={\textbf{\alph*.}}]
        \item Verify that the six matrices
              \[
                  \begin{bmatrix}
                      1 & 0 & 0 \\
                      0 & 1 & 0 \\
                      0 & 0 & 1
                  \end{bmatrix},
                  \begin{bmatrix}
                      0 & 1 & 0 \\
                      0 & 0 & 1 \\
                      1 & 0 & 0
                  \end{bmatrix},
                  \begin{bmatrix}
                      0 & 0 & 1 \\
                      1 & 0 & 0 \\
                      0 & 1 & 0
                  \end{bmatrix},
                  \begin{bmatrix}
                      1 & 0 & 0 \\
                      0 & 0 & 1 \\
                      0 & 1 & 0
                  \end{bmatrix},
                  \begin{bmatrix}
                      0 & 0 & 1 \\
                      0 & 1 & 0 \\
                      1 & 0 & 0
                  \end{bmatrix},
                  \begin{bmatrix}
                      0 & 1 & 0 \\
                      1 & 0 & 0 \\
                      0 & 0 & 1
                  \end{bmatrix}
              \]

              form a group under matrix multiplication.
        \item What group discussed in this section is isomorphic to this group of six matrices?
    \end{enumerate}
\end{exercise}

\begin{proof}
    \[
        \begin{split}
            \begin{bmatrix}
                1 & 0 & 0 \\
                0 & 1 & 0 \\
                0 & 0 & 1
            \end{bmatrix}
            \begin{bmatrix}
                1 \\
                2 \\
                3
            \end{bmatrix} =
            \begin{bmatrix}
                1 \\
                2 \\
                3
            \end{bmatrix},\quad
            \begin{bmatrix}
                0 & 1 & 0 \\
                0 & 0 & 1 \\
                1 & 0 & 0
            \end{bmatrix}
            \begin{bmatrix}
                1 \\
                2 \\
                3
            \end{bmatrix} =
            \begin{bmatrix}
                2 \\
                3 \\
                1
            \end{bmatrix},\quad
            \begin{bmatrix}
                0 & 0 & 1 \\
                1 & 0 & 0 \\
                0 & 1 & 0
            \end{bmatrix}
            \begin{bmatrix}
                1 \\
                2 \\
                3
            \end{bmatrix} =
            \begin{bmatrix}
                3 \\
                1 \\
                2
            \end{bmatrix}, \\
            \begin{bmatrix}
                1 & 0 & 0 \\
                0 & 0 & 1 \\
                0 & 1 & 0
            \end{bmatrix}
            \begin{bmatrix}
                1 \\
                2 \\
                3
            \end{bmatrix} =
            \begin{bmatrix}
                1 \\
                3 \\
                2
            \end{bmatrix},\quad
            \begin{bmatrix}
                0 & 0 & 1 \\
                0 & 1 & 0 \\
                1 & 0 & 0
            \end{bmatrix}
            \begin{bmatrix}
                1 \\
                2 \\
                3
            \end{bmatrix} =
            \begin{bmatrix}
                3 \\
                2 \\
                1
            \end{bmatrix},\quad
            \begin{bmatrix}
                0 & 1 & 0 \\
                1 & 0 & 0 \\
                0 & 0 & 1
            \end{bmatrix}
            \begin{bmatrix}
                1 \\
                2 \\
                3
            \end{bmatrix} =
            \begin{bmatrix}
                2 \\
                1 \\
                3
            \end{bmatrix}.
        \end{split}
    \]

    We can relabel the six matrices to six permutations $\iota, (1, 2, 3), (1, 3, 2), (2, 3), (1, 3), (1, 2)$. So the six matrices form a group under matrix multiplication. This group of six matrices is isomorphic to $S_{3}$.
\end{proof}

% section 4/exercise 20
\begin{exercise}
    After working Exercise 18, write down eight matrices that form a group under matrix multiplication that is isomorphic to $D_{4}$.
\end{exercise}

\begin{proof}
    \[
        \begin{split}
            \begin{bmatrix}
                1 & 0 & 0 & 0 \\
                0 & 1 & 0 & 0 \\
                0 & 0 & 1 & 0 \\
                0 & 0 & 0 & 1
            \end{bmatrix}\leftrightarrow\iota,\quad
            \begin{bmatrix}
                0 & 1 & 0 & 0 \\
                0 & 0 & 1 & 0 \\
                0 & 0 & 0 & 1 \\
                1 & 0 & 0 & 0
            \end{bmatrix}\leftrightarrow\rho,\quad
            \begin{bmatrix}
                0 & 0 & 1 & 0 \\
                0 & 0 & 0 & 1 \\
                1 & 0 & 0 & 0 \\
                0 & 1 & 0 & 0
            \end{bmatrix}\leftrightarrow\rho^{2},\quad
            \begin{bmatrix}
                0 & 0 & 0 & 1 \\
                1 & 0 & 0 & 0 \\
                0 & 1 & 0 & 0 \\
                0 & 0 & 1 & 0
            \end{bmatrix}\leftrightarrow\rho^{3}, \\
            \begin{bmatrix}
                0 & 0 & 0 & 1 \\
                0 & 0 & 1 & 0 \\
                0 & 1 & 0 & 0 \\
                1 & 0 & 0 & 0
            \end{bmatrix}\leftrightarrow\mu,\quad
            \begin{bmatrix}
                1 & 0 & 0 & 0 \\
                0 & 0 & 0 & 1 \\
                0 & 0 & 1 & 0 \\
                0 & 1 & 0 & 0
            \end{bmatrix}\leftrightarrow\mu\rho,\quad
            \begin{bmatrix}
                0 & 1 & 0 & 0 \\
                1 & 0 & 0 & 0 \\
                0 & 0 & 0 & 1 \\
                0 & 0 & 1 & 0
            \end{bmatrix}\leftrightarrow\mu\rho^{2},\quad
            \begin{bmatrix}
                0 & 0 & 1 & 0 \\
                0 & 1 & 0 & 0 \\
                1 & 0 & 0 & 0 \\
                0 & 0 & 0 & 1
            \end{bmatrix}\leftrightarrow\mu\rho^{3}.
        \end{split}
    \]
\end{proof}

\subsection*{Concepts}

In Exercises 21 through 23, correct the definition of the italicized term without reference to the text, if correction is needed, so that it is in a form acceptable for publication.

% section 4/exercise 21
\begin{exercise}
    The \textit{dihedral group $D_{n}$} is the set of all functions $\phi: \mathbb{Z}_{n} \to \mathbb{Z}_{n}$ such that the line segment between vertex $i$ and $j$ of $U_{n}$ is an edge of $P_{n}$ if and only if the line segment between vertices $\phi(i)$ and $\phi(j)$ in $U_{n}$ is an edge of $P_{n}$.
\end{exercise}

\begin{proof}
    No correction is needed.
\end{proof}

% section 4/exercise 22
\begin{exercise}
    A \textit{permutation} of a set $S$ is a one-to-one map from $S$ to $S$.
\end{exercise}

\begin{proof}
    Correction:  A \textit{permutation} of a set $S$ is a one-to-one map from $S$ onto $S$.
\end{proof}

% section 4/exercise 23
\begin{exercise}
    The \textit{order} of a group is the number of elements in the group.
\end{exercise}

\begin{proof}
    Correction: The \textit{order} of a group is the number of elements in the group or the cardinality of the group.
\end{proof}

In Exercises 24 through 28, determine whether the given function is a permutation of $\mathbb{R}$.

% section 4/exercise 24
\begin{exercise}
    $f_{1}: \mathbb{R} \to \mathbb{R}$ defined by $f_{1}(x) = x + 1$
\end{exercise}

\begin{proof}
    $f_{1}$ is one-to-one, since $x\ne y$ implies $x+1\ne y+1$.

    $f_{1}$ is onto, since $f_{1}(x - 1) = x$.

    Hence $f_{1}$ is a permutation of $\mathbb{R}$.
\end{proof}

% section 4/exercise 25
\begin{exercise}
    $f_{2}: \mathbb{R} \to \mathbb{R}$ defined by $f_{2}(x) = x^{2}$
\end{exercise}

\begin{proof}
    $f_{2}$ is not one-to-one, since $f_{2}(1) = f_{2}(-1) = 1$.

    Hence $f_{2}$ is not a permutation of $\mathbb{R}$.
\end{proof}

% section 4/exercise 26
\begin{exercise}
    $f_{3}: \mathbb{R} \to \mathbb{R}$ defined by $f_{3}(x) = -x^{3}$
\end{exercise}

\begin{proof}
    $f_{3}$ is one-to-one, since $x\ne y$ implies $-x^{3} \ne -y^{3}$ (because $x^{3} - y^{3} = (x - y)(x^{2} + xy + y^{2})$).

    $f_{3}$ is onto, since $f_{1}(-\sqrt[3]{x}) = x$.

    Hence $f_{3}$ is a permutation of $\mathbb{R}$.
\end{proof}

% section 4/exercise 27
\begin{exercise}
    $f_{4}: \mathbb{R} \to \mathbb{R}$ defined by $f_{4}(x) = e^{x}$
\end{exercise}

\begin{proof}
    $f_{4}$ is not onto, since there is no real number $x$ such that $e^{x} = 0$.

    Hence $f_{4}$ is not a permutation of $\mathbb{R}$.
\end{proof}

% section 4/exercise 28
\begin{exercise}
    $f_{5}: \mathbb{R} \to \mathbb{R}$ defined by $f_{5}(x) = x^{3} - x^{2} - 2x$
\end{exercise}

\begin{proof}
    $f_{5}$ is not one-to-one, since $f_{5}(0) = f_{5}(-1) = 0$.

    Hence $f_{5}$ is not a permutation of $\mathbb{R}$.
\end{proof}

% section 4/exercise 29
\begin{exercise}
    Determine whether each of the following is true or false.
    \begin{enumerate}[label={\textbf{\alph*.}}]
        \item Every permutation is a one-to-one function.
        \item Every function is a permutation if and only if it is one-to-one.
        \item Every function from a finite set onto itself must be one-to-one.
        \item Every subset of an abelian group $G$ that is also a group using the same operation as $G$ is abelian.
        \item The symmetric group $S_{10}$ has $10$ elements.
        \item If $\phi\in D_{n}$, then $\phi$ is a permutation on the set $\mathbb{Z}_{n}$.
        \item The group $D_{n}$ has exactly $n$ elements.
        \item $D_{3}$ is a subset of $D_{4}$.
    \end{enumerate}
\end{exercise}

\begin{proof}
    \begin{enumerate}[label={\textbf{\alph*.}}]
        \item True.
        \item False.
        \item False.
        \item False.
        \item False.
        \item True.
        \item False.
        \item False.
    \end{enumerate}
\end{proof}

\subsection*{Theory}

% section 4/exercise 30
\begin{exercise}
    Let $n\geq 3$ and $k\in\mathbb{Z}_{n}$. Prove that in $D_{n}$, $\rho^{k}\mu = \mu\rho^{n-k}$.
\end{exercise}

\begin{proof}
    Let $x$ be an element of $\mathbb{Z}_{n}$.
    \[
        \begin{split}
            (\rho^{k}\mu)(x) = n - x + k \mod n = k - x\mod n, \\
            (\mu\rho^{n-k})(x) = n - (x + n - k) \mod n = k - x \mod n
        \end{split}
    \]

    Hence $\rho^{k}\mu = \mu\rho^{n-k}$.
\end{proof}

% section 4/exercise 31
\begin{exercise}
    Show that $S_{n}$ is nonabelian group for $n\geq 3$.
\end{exercise}

\begin{proof}
    $\sigma = (1, 2)$ is a permutation in $S_{n}$, which swaps $1$ and $2$. $\tau = (1, 3)$ is a permutation in $S_{n}$, which swaps $1$ and $3$.
    \[
        \sigma\tau = (1, 3, 2) \ne (1, 2, 3) = \tau\sigma
    \]

    So $S_{n}$ is nonabelian group for $n\geq 3$.
\end{proof}

% section 4/exercise 32
\begin{exercise}
    Strengthening Exercise 31, show that if $n\geq 3$, then the only element of $\sigma$ of $S_{n}$ satisfying $\sigma\gamma = \gamma\sigma$ for all $\gamma\in S_{n}$ is $\sigma = \iota$, the identity permutation.
\end{exercise}

\begin{proof}
    Let $\alpha\in S_{n}$ be the transposition that swaps $1$ and $2$, let $\beta\in S_{n}$ be the transposition that swaps $1$ and $k$ ($k\geq 3$).

    $(\sigma\alpha)(1) = \sigma(\alpha(1)) = \sigma(2), (\alpha\sigma)(1) = \alpha(\sigma(1))$, so $\alpha$ swaps $\sigma(1)$ and $\sigma(2)$. So either $\sigma(1) = 1, \sigma(2) = 2$ or $\sigma(1) = 2$, $\sigma(2) = 1$.

    $(\sigma\beta)(1) = \sigma(\beta(1)) = \sigma(k), (\beta\sigma)(1) = \beta(\sigma(1))$, so $\beta$ swaps $\sigma(1)$ and $\sigma(k)$. So either $\sigma(1) = 1, \sigma(k) = k$ or $\sigma(1) = k$, $\sigma(k) = 1$.

    Therefore, $\sigma(1) = 1, \sigma(2) = 2, \sigma(k) = k$ (other cases lead to contradiction). Thus $\sigma = \iota$.
\end{proof}

% section 4/exercise 33
\begin{exercise}
    Orbits were defined before Exercise 15. Let $a, b\in A$ and $\sigma\in S_{A}$. Show that if $\mathcal{O}_{a,\sigma}$ and $\mathcal{O}_{b,\sigma}$ have an element in common, then $\mathcal{O}_{a,\sigma} = \mathcal{O}_{b,\sigma}$.
\end{exercise}

\begin{proof}
    Let $c$ be the common element of $\mathcal{O}_{a,\sigma}$ and $\mathcal{O}_{b,\sigma}$. According to the definition of orbits, there exists integers $p, q$ such that $c = \sigma^{p}(a)$ and $c = \sigma^{q}(b)$. Therefore, $\sigma^{p}(a) = \sigma^{q}(b)$, so $\sigma^{p-q}(a) = b$ and $\sigma^{q-p}(b) = a$.

    Let $x = \sigma^{n}(a)$ be an element of $\mathcal{O}_{a,\sigma}$, then $x = \sigma^{n}(\sigma^{q-p}(b)) = \sigma^{n+q-p}(b)$, so $x$ is also an element of $\mathcal{O}_{b,\sigma}$. Therefore, $\mathcal{O}_{a,\sigma} \subseteq \mathcal{O}_{b,\sigma}$.

    Let $y = \sigma^{m}(b)$ be an element of $\mathcal{O}_{b,\sigma}$, then $y = \sigma^{m}(\sigma^{p-q}(a)) = \sigma^{m+p-q}(a)$, so $y$ is also an element of $\mathcal{O}_{a,\sigma}$. Therefore, $\mathcal{P}_{b,\sigma} \subseteq \mathcal{P}_{a,\sigma}$.

    Thus $\mathcal{O}_{a,\sigma} = \mathcal{O}_{b,\sigma}$.
\end{proof}

% section 4/exercise 34
\begin{exercise}
    (See the warning following Theorem 4.8.) Let $G$ be a group with binary operation $*$. Let $G'$ be the same set as $G$, and define a binary operation $*'$ on $G'$ by $x *' y = y * x$ for all $x, y\in G'$.
    \begin{enumerate}[label={\textbf{\alph*.}}]
        \item (Intuitive argument that $G'$ under $*'$ is a group.) Suppose the front wall of your classroom were made of transparent glass, and that all possible products $a * b = c$ and all possible instances $a * (b * c) = (a * b) * c$ of the associative property for $G$ under $*$ were written on the wall with a magic marker. What would a person see when looking at the other side of the wall from the next room in front of yours?
        \item Show from the mathematical definition of $*'$ that $G'$ is a group under $*'$.
    \end{enumerate}
\end{exercise}

\begin{proof}
    $G'$ is closed under $*'$ since $G$ is closed under $*$.

    For all $x, y, z\in G'$, $(x *' y) *' z = (y * x) *' z = z * (y * x) = (z * y) * x = (y *' z) * x = x *' (y' * z')$. So $*'$ is associative.

    Let $e$ be the identity element in $G$. $x *' e = e * x = x = x * e = e *' x$. So $*'$ has an identity element, the same as $G$.

    Let $x'$ be the inverse of $x$ in $G$, $x *' x' = x' * x = e = x * x' = x' *' x$. So each element of $G'$ has an inverse with $*'$, the same as $*$.

    Hence $G'$ is a group under $*'$.
\end{proof}

% section 4/exercise 35
\begin{exercise}
    Give a careful proof using the definition of isomorphism that if $G$ and $G'$ are both groups with $G$ abelian and $G'$ not abelian, then $G$ and $G'$ are not isomorphic.
\end{exercise}

\begin{proof}
    Assume that $G$ and $G'$ are isomorphic. According to the definition of isomorphism, there exists a one-to-one function $\phi$ from $G$ onto $G'$ such that $\phi(x * y) = \phi(x) *' \phi(y)$ for all $x, y\in G$.

    Since $G'$ is not abelian, there exist two elements $a, b$ in $G'$ such that $a *' b \ne b *' a$. $\phi$ is onto $G'$ so there exist $x, y$ in $G$ such that $\phi(x) = a$ and $\phi(y) = b$. $a *' b = \phi(x) *' \phi(y) = \phi(x * y) = \phi(y * x) = \phi(y) *' \phi(x) = b *' a$, this contradicts $a *'b \ne b *' a$. So the initial assumption is false.

    Thus $G$ and $G'$ are not isomorphic.
\end{proof}

% section 4/exercise 36
\begin{exercise}
    Prove that for any integer $n\geq 2$, there are at least two nonisomorphic groups with exactly $2n$ elements.
\end{exercise}

\begin{proof}
    $D_{n}$ and $\mathbb{Z}_{2n}$, each has $2n$ elements. But $D_{n}$ is not abelian, $\mathbb{Z}_{2n}$ is abelian. Thus, there are at least two nonisomorphic groups with exactly $2n$ elements.
\end{proof}

% section 4/exercise 37
\begin{exercise}
    Let $n\geq 3$ and $0\leq k \leq n-1$. Prove that the map $\mu\rho^{k} \in D_{n}$ is the reflection about the line through the origin that makes an angle of $-\frac{\pi k}{n}$ with the $x$-axis.
\end{exercise}

\begin{proof}
    On the unit circle, $(\cos\frac{2m\pi}{n}, \sin\frac{2m\pi}{n})$ where $0\leq k \leq n$ are geometric representations of $n$-th roots of unity.

    The line through the origin that makes an angle of $-\frac{\pi k}{n}$ with the $x$-axis has equation
    \[
        \sin\frac{k\pi}{n} x + \cos\frac{k\pi}{n}y = 0
    \]

    Remind that the reflection of $(x_{0}, y_{0})$ about the line $ax + by + c = 0$ has coordinates
    \[
        \left(x_{0} - \frac{2a(ax_{0} + by_{0} + c)}{a^{2} + b^{2}}, y_{0} - \frac{2b(ax_{0} + by_{0} + c)}{a^{2} + b^{2}}\right)
    \]

    The reflection of $(\cos\frac{2m\pi}{n}, \sin\frac{2m\pi}{n})$ about this line has coordinates
    \begin{align*}
          & \left(\cos\frac{2m\pi}{n} - 2\sin\frac{k\pi}{n}\left(\sin\frac{k\pi}{n}\cos\frac{2m\pi}{n} + \cos\frac{k\pi}{n}\sin\frac{2m\pi}{n}\right), \sin\frac{2m\pi}{n} - 2\cos\frac{k\pi}{n}\left(\sin\frac{k\pi}{n}\cos\frac{2m\pi}{n} + \cos\frac{k\pi}{n}\sin\frac{2m\pi}{n}\right)\right) \\
        = & \left( \cos\frac{2m\pi}{n} - 2\sin\frac{k\pi}{n}\sin\frac{(2m+k)\pi}{n}, \sin\frac{2m\pi}{n} - 2\cos\frac{k\pi}{n}\sin\frac{(2m+k)\pi}{n} \right)                                                                                                                                     \\
        = & \left( \cos\frac{(2m+2k)\pi}{n}, \sin\frac{2m\pi}{n} - 2\cos\frac{-k\pi}{n}\sin\frac{(2m+k)\pi}{n} \right)                                                                                                                                                                            \\
        = & \left( \cos\frac{-(2m+2k)\pi}{n}, \sin\frac{-(2m+2k)\pi}{n} \right)
    \end{align*}

    Image of $(\cos\frac{2m\pi}{n}, \sin\frac{2m\pi}{n})$ under $\rho^{k}$ and $\mu$ is
    \[
        \mu\left(\rho^{k}\left(\left( \cos\frac{2m\pi}{n}, \frac{2m\pi}{n} \right)\right)\right) = \mu\left(\left( \cos\frac{(2m+2k)\pi}{n}, \sin\frac{(2m+2k)\pi}{n} \right)\right) = \left( \cos\frac{-(2m+2k)\pi}{n}, \sin\frac{-(2m+2k)\pi}{n} \right).
    \]

    Thus, $\mu\rho^{k} \in D_{n}$ is the reflection about the line through the origin and make an angle of $-\frac{\pi k}{n}$ with the $x$-axis.
\end{proof}

% section 4/exercise 38
\begin{exercise}
    Let $n\geq 3$ and $k, r\in\mathbb{Z}_{n}$. Based on Exercise 37, determine the element of $D_{n}$ that corresponds to first reflecting across the line through the origin at an angle of $-\frac{2\pi k}{n}$ and then reflection across the line through the origin making an angle of $-\frac{2\pi r}{n}$. Prove your answer.
\end{exercise}

\begin{proof}
    According to Exercise 37
    \begin{itemize}
        \item the element that corresponds to the reflection about the line through the origin at an angle of $-\frac{2\pi k}{n}$ is $\mu\rho^{2k}$,
        \item the element that corresponds to the reflection about the line through the origin at an angle of $-\frac{2\pi r}{n}$ is $\mu\rho^{2r}$.
    \end{itemize}

    Their composition is $\mu\rho^{2r}\mu\rho^{2k}$. According to Exercise 32
    \[
        \mu\rho^{2r}\mu\rho^{2k} = \mu(\mu\rho^{n-2r})\rho^{2k} = \rho^{n-2(r+k)}.
    \]

    Hence the desired element is $\rho^{n-2(r+k)}$.
\end{proof}

% \textbf{Computations}

% \begin{exercise}
% What three things must we check to determine whether a function $\phi: S\to S'$ is an isomorphism of a binary structure $\anglebracket{S, *}$ with $\anglebracket{S', *'}$.
% \end{exercise}

% \begin{proof}
% Those three things are: $\phi$ is a one-to-one function, $\phi$ is onto $S'$, and $\phi(x * y) = \phi(x) *' \phi(y)$ for every $x, y\in S$.
% \end{proof}

% In Exercises 2 through 10, determine whether the given map $\phi$ is an isomorphism of the first binary structure with the second. (See Exercise 1.) If it is not an isomorphism, why not?

% \begin{exercise}
% $\anglebracket{\mathbb{Z}, +}$ with $\anglebracket{\mathbb{Z}, +}$ where $\phi(n) = -n$ for $n\in\mathbb{Z}$
% \end{exercise}

% \begin{proof}
% $\phi(n) = \phi(m)$ if and only if $-n = -m$. $-n = -m$ if and only if $n = m$. So $\phi$ is a one-to-one function.

% For each $n\in\mathbb{Z}$, $\phi(-n) = n$. So $\phi$ is onto $\mathbb{Z}$.

% For every $m, n\in\mathbb{Z}$, $\phi(m + n) = -(m + n) = (-m) + (-n) = \phi(m) + \phi(n)$.

% Hence $\phi$ is an isomorphism.
% \end{proof}

% \begin{exercise}
% $\anglebracket{\mathbb{Z}, +}$ with $\anglebracket{\mathbb{Z}, +}$ where $\phi(n) = 2n$ for $n\in\mathbb{Z}$
% \end{exercise}

% \begin{proof}
% Since $\phi(n) = 2n$, there is no integer $x$ such that $\phi(x)$ is an odd integer. So $\phi$ is not onto $\mathbb{Z}$.

% Hence $\phi$ is not an isomorphism.
% \end{proof}

% \begin{exercise}
% $\anglebracket{\mathbb{Z}, +}$ with $\anglebracket{\mathbb{Z}, +}$ where $\phi(n) = n + 1$ for $n\in\mathbb{Z}$
% \end{exercise}

% \begin{proof}
% $\phi(n) = \phi(m)$ if and only if $n + 1 = m + 1$. $n + 1 = m + 1$ if and only if $n = m$. So $\phi$ is a one-to-one function.

% For each $n\in\mathbb{Z}$, $\phi(n - 1) = n$. So $\phi$ is onto $\mathbb{Z}$.

% For every $m, n\in\mathbb{Z}$, $\phi(m + n) = m + n + 1 = (m + 1) + (n + 1) - 1 = \phi(m) + \phi(n) - 1 \ne \phi(m) + \phi(n)$.

% Hence $\phi$ is not an isomorphism.
% \end{proof}

% \begin{exercise}
% $\anglebracket{\mathbb{Q}, +}$ with $\anglebracket{\mathbb{Q}, +}$ where $\phi(x) = x/2$ for $x\in\mathbb{Q}$
% \end{exercise}

% \begin{proof}
% $\phi(x) = \phi(y)$ if and only if $x/2 = y/2$. $x/2 = y/2$ if and only if $x = y$. So $\phi$ is a one-to-one function.

% For each $x\in\mathbb{Q}$, $\phi(2x) = x$. So $\phi$ is onto $\mathbb{Q}$.

% For every $x, y\in\mathbb{Q}$, $\phi(x\cdot y) = x\cdot y/2 = 2\cdot(x/2)\cdot(y/2) = 2\cdot\phi(x)\cdot\phi(y)$. If $x, y\ne 0$, $\phi(x\cdot y) \ne \phi(x)\cdot\phi(y)$.

% Hence $\phi$ is not an isomorphism.
% \end{proof}

% \begin{exercise}
% $\anglebracket{\mathbb{Q},\cdot}$ with $\anglebracket{\mathbb{Q},\cdot}$ where $\phi(x) = x^{2}$ for $x\in\mathbb{Q}$.
% \end{exercise}

% \begin{proof}
% There is no rational number $q$ such that $\phi(q) = q^{2} = 2$. So $\phi$ is not onto $\mathbb{Q}$.

% Hence $\phi$ is not an isomorphism.
% \end{proof}

% \begin{exercise}
% $\anglebracket{\mathbb{R},\cdot}$ with $\anglebracket{\mathbb{R},\cdot}$ where $\phi(x) = x^{3}$
% \end{exercise}

% \begin{proof}
% $\phi(x) = \phi(y)$ if and only if $x^{3} = y^{3}$. $x^{3} = y^{3}$ if and only if $x = y$ ($x, y\in\mathbb{R}$). So $\phi$ is a one-to-one function.

% For each $x\in\mathbb{R}$, there exists a real number $a$ such that $a^{3} = x$. Equivalently, $\phi(a) = a^{3} = x$. So $\phi$ is onto $\mathbb{R}$.

% For every $x, y\in\mathbb{R}$, $\phi(x\cdot y) = {(x\cdot y)}^{3} = x^{3}\cdot y^{3} = \phi(x)\cdot\phi(y)$.

% Hence $\phi$ is an isomorphism.
% \end{proof}

% \begin{exercise}
% $\anglebracket{M_{2}(\mathbb{R}), \cdot}$ with $\anglebracket{\mathbb{R},\cdot}$ where $\phi(A)$ is the determinant of matrix $A$
% \end{exercise}

% \begin{proof}
% There are different $2\times 2$ matrices with the same determinant. For examples:
% \[
% \begin{vmatrix}
% 1 & 0 \\
% 0 & 1
% \end{vmatrix}
% =
% \begin{vmatrix}
% 1 & 1 \\
% 0 & 1
% \end{vmatrix}.
% \]

% So $\phi$ is not a one-to-one function.

% Hence $\phi$ is not an isomorphism.
% \end{proof}

% \begin{exercise}
% $\anglebracket{M_{1}(\mathbb{R}),\cdot}$ with $\anglebracket{\mathbb{R},\cdot}$ where $\phi(A)$ is the determinant of matrix $A$
% \end{exercise}

% \begin{proof}
% $\phi(A) = \phi(B)$ if and only if $\det(A) = \det(B)$. Since $A, B$ are $1\times 1$ matrices, $\det(A) = \det(B)$ if and only if $A = B$. So $\phi$ is a one-to-one function.

% For every $x\in\mathbb{R}$, $\phi(\begin{bmatrix}x\end{bmatrix})\det\begin{bmatrix}x\end{bmatrix} = x$. So $\phi$ is onto $\mathbb{R}$.

% For every $A, B\in M_{1}(\mathbb{R})$, $\phi(A\cdot B) = \det(A\cdot B) = \det(A)\cdot\det(B) = \phi(A)\cdot\phi(B)$.

% Hence $\phi$ is an isomorphism.
% \end{proof}

% \begin{exercise}
% $\anglebracket{\mathbb{R}, +}$ with $\anglebracket{\mathbb{R}^{+},\cdot}$ where $\phi(r) = {0.5}^{r}$ for $r\in\mathbb{R}$
% \end{exercise}

% \begin{proof}
% $\phi(x) = \phi(y)$ if and only if ${0.5}^{x} = {0.5}^{y}$. ${0.5}^{x} = {0.5}^{y}$ if and only if $\log_{0.5}{0.5}^{x} = \log_{0.5}{0.5}^{y}$, in other words, $x = y$. So $\phi$ is a one-to-one function.

% For each $m\in\mathbb{R}^{+}$, $\phi(\log_{0.5}m) = {0.5}^{\log_{0.5}m} = m$. So $\phi$ is onto $\mathbb{R}^{+}$.

% For every $x, y\in\mathbb{R}$, $\phi(x + y) = {0.5}^{x+y} = {0.5}^{x}\cdot{0.5}^{y} = \phi(x)\cdot\phi(y)$.

% Hence $\phi$ is an isomorphism.
% \end{proof}

% In Exercises 11 through 15, let $F$ be the set of all functions $f$ mapping $\mathbb{R}$ into $\mathbb{R}$ that have deravatives of all orders. Follow the instructions for Exercies 2 through 10.

% \begin{exercise}
% $\anglebracket{F,+}$ with $\anglebracket{F,+}$ where $\phi(f) = f'$, the derivative of $f$
% \end{exercise}

% \begin{proof}
% Let $f(x) = x$ and $g(x) = x + 1$. $f'(x) = g'(x) = 1$. So $\phi$ is not a one-to-one function.

% Hence $\phi$ is not an isomorphism.
% \end{proof}

% \begin{exercise}
% $\anglebracket{F, +}$ with $\anglebracket{\mathbb{R}, +}$ where $\phi(f) = f'(0)$
% \end{exercise}

% \begin{proof}
% Let $f(x) = x$ and $g(x) = x + 1$. $f'(0) = g'(0) = 1$. So $\phi$ is not a one-to-one function.

% Hence $\phi$ is not an isomorphism.
% \end{proof}

% \begin{exercise}
% $\anglebracket{F,+}$ with $\anglebracket{F,+}$ where $\phi(f)(x) = \int^{x}_{0}f(t)dt$.
% \end{exercise}

% \begin{proof}
% If $f\in F$, $f$ has antiderivatives. If $g$ is an antiderivative of $f$, then every antiderivative $h$ of $f$ satisfies $h(x) = g(x) + C$, where $C$ is a constant. But there is a unique antiderivative that has zero value at $x = 0$, which is $\int^{x}_{0}f(t)dt$. So $\phi$ is a one-to-one function.

% Since $\phi(f)$ has zero value at $x = 0$, so for every $g\in F$ such that $g(0)\ne 0$, there is no $h\in F$ such that $\phi(h) = g$. So $\phi$ is not onto $\anglebracket{F,+}$.

% Hence $\phi$ is not an isomorphism.
% \end{proof}

% \begin{exercise}
% $\anglebracket{F,+}$ with $\anglebracket{F,+}$ where $\phi(f)(x) = \frac{d}{dx}\left(\int^{x}_{0} f(t)dt\right)$
% \end{exercise}

% \begin{proof}
% $f\in F$, then $f$ has antiderivate. According to \textit{the fundamental theorem of calculus}
% \[
% \frac{d}{dx}\left(\int^{x}_{0}f(t)dt\right) = f(x) - f(0)
% \]

% $\phi(f) = \phi(g)$ if and only if $\phi(f)(x) = \phi(g)(x)$ for every $x\in\mathbb{R}$. $\phi(f)(x) = \phi(g)(x)$ for every $x\in\mathbb{R}$ if and only if $f(x) - f(0) = g(x) - g(0)$ for every $x\in\mathbb{R}$, in other words, $f = g$. So $\phi$ is a one-to-one function.

% Let $f(x) = x + 1$, then $f(0) = 1\ne 0$. There is no function $g\in F$ such that $\phi(g)(x) = f(x)$, since $\phi(g)(0) = g(0) - g(0) = 0 \ne 1 = f(0)$.

% Hence $\phi$ is not an isomorphism.
% \end{proof}

% \begin{exercise}
% $\anglebracket{F,\cdot}$ with $\anglebracket{F,\cdot}$ where $\phi(f)(x) = x\cdot f(x)$
% \end{exercise}

% \begin{proof}
% $\phi(f) = \phi(g)$ if and only if $\phi(f)(x) = \phi(g)(x)$ for every $x\in\mathbb{R}$. $\phi(f)(x) = \phi(g)(x)$ for every $x\in\mathbb{R}$ if and only if $x\cdot f(x) = x\cdot g(x)$ for every $x\in\mathbb{R}$. If $x\ne 0$, then $f(x) = g(x)$. Otherwise, since $f$ and $g$ are continuous at $x = 0$, for every $\varepsilon > 0$, there exists $\delta_{f}(\varepsilon)$ and $\delta_{f}(\varepsilon)$ such that $\abs{f(x) - f(0)} < \frac{\varepsilon}{2}$ if $0 < \abs{x} < \delta_{f}(\varepsilon)$ and $\abs{g(x) - g(0)} < \frac{\varepsilon}{2}$ if $0 < \abs{x} < \delta_{g}(\varepsilon)$. For $0 < \abs{x} < \min\{ \delta_{f}(\varepsilon), \delta_{f}(\varepsilon) \}$, $\abs{f(0) - g(0)}\le \abs{f(0) - f(x)} + \abs{f(x) - g(x)} + \abs{g(x) - g(0)} = \varepsilon/2 + 0 + \varepsilon/2 = \varepsilon$, which implies $f(0) = g(0)$. So $f(x) = g(x)$ for every $x\in\mathbb{R}$. So $\phi$ is a one-to-one function.

% There is no function $f\in F$ such that $x\cdot f(x) = 1$ for every $x\in\mathbb{R}$. So $\phi$ is not onto $F$.

% For $x\ne 0$ and $x\ne 1$, $f(x) = 1, g(x) = 1$, we have
% \[
% \phi(f\cdot g)(x) = x\cdot (f\cdot g)(x) = x \ne x^{2} = (x\cdot f(x))\cdot (y\cdot g(x)) = \phi(f)(x) \cdot \phi(g)(x) = (\phi(f)\cdot\phi(g))(x)
% \]

% Hence $\phi$ is not an isomorphism.
% \end{proof}

% \begin{exercise}
% The map $\phi: \mathbb{Z} \to \mathbb{Z}$ define by $\phi(n) = n + 1$ for $n\in\mathbb{Z}$ is one-to-one and onto $\mathbb{Z}$. Give the definition of a binary operation $*$ on $\mathbb{Z}$ such that $\phi$ is an isomorphism mapping
% \begin{enumerate}[label={\textbf{\alph*}},itemsep=0pt,topsep=0pt]
% \item $\anglebracket{\mathbb{Z}, +}$ onto $\anglebracket{\mathbb{Z}, *}$
% \item $\anglebracket{\mathbb{Z}, *}$ onto $\anglebracket{\mathbb{Z}, +}$
% \end{enumerate}

% In each case, give the identity element for $*$ on $\mathbb{Z}$.
% \end{exercise}

% \begin{proof}
% \begin{enumerate}[label={\textbf{\alph*}},itemsep=0pt,topsep=0pt]
% \item Define $*: \mathbb{Z}\times\mathbb{Z}\to\mathbb{Z}$ as follows: $m * n = m + n - 1$ for every $m, n\in\mathbb{Z}$. Then
% \[
% \phi(m + n) = m + n + 1 = (m + 1) + (n + 1) - 1 = \phi(m) + \phi(n) - 1 = \phi(m) * \phi(n).
% \]

% The identity element of $*$ on $\mathbb{Z}$ is $1$.
% \item Define $*: \mathbb{Z}\times\mathbb{Z}\to\mathbb{Z}$ as follows: $m * n = m + n + 1$ for every $m, n\in\mathbb{Z}$. Then
% \[
% \phi(m * n) = \phi(m + n + 1) = m + n + 2 = (m + 1) + (n + 1) = \phi(m) + \phi(n).
% \]

% The identity element of $*$ on $\mathbb{Z}$ is $-1$.
% \end{enumerate}
% \end{proof}

% \begin{exercise}
% The map $\phi: \mathbb{Z} \to \mathbb{Z}$ define by $\phi(n) = n + 1$ for $n\in\mathbb{Z}$ is one-to-one and onto $\mathbb{Z}$. Give the definition of a binary operation $*$ on $\mathbb{Z}$ such that $\phi$ is an isomorphism mapping
% \begin{enumerate}[label={\textbf{\alph*}},itemsep=0pt,topsep=0pt]
% \item $\anglebracket{\mathbb{Z}, \cdot}$ onto $\anglebracket{\mathbb{Z}, *}$
% \item $\anglebracket{\mathbb{Z}, *}$ onto $\anglebracket{\mathbb{Z}, \cdot}$
% \end{enumerate}

% In each case, give the identity element for $*$ on $\mathbb{Z}$.
% \end{exercise}

% \begin{proof}
% \begin{enumerate}[label={\textbf{\alph*}},itemsep=0pt,topsep=0pt]
% \item Define $*: \mathbb{Z}\times\mathbb{Z}\to\mathbb{Z}$ as follows: $m * n = mn - m - n + 2$ for every $m, n\in\mathbb{Z}$. Then
% \begin{align*}
% \phi(m\cdot n) & = m\cdot n + 1                                 \\
% & = m\cdot n + m + n + 1 - (m + 1) - (n + 1) + 2 \\
% & = (m + 1)(n + 1) - (m + 1) - (n + 1) + 2       \\
% & = \phi(m)\cdot\phi(n) - \phi(m) - \phi(n) + 2  \\
% & = \phi(m) * \phi(n)
% \end{align*}

% The identity element of $*$ in $\mathbb{Z}$ is $2$.
% \item Define $*: \mathbb{Z}\times\mathbb{Z}\to\mathbb{Z}$ as follows: $m * n = mn + m + n$ for every $m, n\in\mathbb{Z}$. Then
% \begin{align*}
% \phi(m * n) & = m * n + 1            \\
% & = mn + m + n + 1       \\
% & = (m + 1)\cdot (n + 1) \\
% & = \phi(m)\cdot\phi(n)
% \end{align*}

% The identity element of $*$ in $\mathbb{Z}$ is $0$.
% \end{enumerate}
% \end{proof}

% \begin{exercise}
% The map $\phi: \mathbb{Q} \to \mathbb{Q}$ defined by $\phi(x) = 3x - 1$ for $x\in\mathbb{Q}$ is one-to-one and onto $\mathbb{Z}$. Give the definition of a binary operation $*$ on $\mathbb{Q}$ such that $\phi$ is an isomorphism mapping
% \begin{enumerate}[label={\textbf{\alph*}},itemsep=0pt,topsep=0pt]
% \item $\anglebracket{\mathbb{Q}, +}$ onto $\anglebracket{\mathbb{Q}, *}$
% \item $\anglebracket{\mathbb{Q}, *}$ onto $\anglebracket{\mathbb{Q}, +}$
% \end{enumerate}

% In each case, give the identity element for $*$ on $\mathbb{Q}$.
% \end{exercise}

% \begin{proof}
% \begin{enumerate}[label={\textbf{\alph*}},itemsep=0pt,topsep=0pt]
% \item Define $*: \mathbb{Q}\times\mathbb{Q} \to \mathbb{Q}$ as follows: $x * y = x + y + 1$ for every $x, y\in\mathbb{Q}$. Then
% \[
% \phi(x + y) = 3(x + y) - 1 = (3x - 1) + (3y - 1) + 1 = \phi(x) + \phi(y) + 1 = \phi(x) * \phi(y).
% \]

% The identity element of $*$ on $\mathbb{Q}$ is $-1$.
% \item Define $*: \mathbb{Q}\times\mathbb{Q} \to \mathbb{Q}$ as follows: $x * y = x + y - \frac{1}{3}$ for every $x, y\in\mathbb{Q}$. Then
% \[
% \phi(x * y) = 3\cdot(x * y) - 1 = 3\cdot\left(x + y - \frac{1}{3}\right) - 1 = 3x + 3y - 2 = (3x - 1) + (3y - 1) = \phi(x) + \phi(y).
% \]

% The identity element of $*$ on $\mathbb{Q}$ is $\frac{1}{3}$.
% \end{enumerate}
% \end{proof}

% \begin{exercise}
% The map $\phi: \mathbb{Q} \to \mathbb{Q}$ defined by $\phi(x) = 3x - 1$ for $x\in\mathbb{Q}$ is one-to-one and onto $\mathbb{Z}$. Give the definition of a binary operation $*$ on $\mathbb{Q}$ such that $\phi$ is an isomorphism mapping
% \begin{enumerate}[label={\textbf{\alph*}},itemsep=0pt,topsep=0pt]
% \item $\anglebracket{\mathbb{Q}, \cdot}$ onto $\anglebracket{\mathbb{Q}, *}$
% \item $\anglebracket{\mathbb{Q}, *}$ onto $\anglebracket{\mathbb{Q}, \cdot}$
% \end{enumerate}

% In each case, give the identity element for $*$ on $\mathbb{Q}$.
% \end{exercise}

% \begin{proof}
% \item Define $*: \mathbb{Q}\times\mathbb{Q} \to \mathbb{Q}$ as follows: $x * y = \frac{1}{3}xy + \frac{1}{3}x + \frac{1}{3}y - \frac{2}{3}$ for every $x, y\in\mathbb{Q}$. Then
% \begin{align*}
% \phi(x\cdot y) & = 3xy - 1 = \frac{1}{3}(9xy - 3x - 3y + 1) + x + y - \frac{4}{3}                        \\
% & = \frac{1}{3}(3x - 1)(3y - 1) + \frac{1}{3}(3x - 1) + \frac{1}{3}(3y - 1) - \frac{2}{3} \\
% & = \frac{1}{3}\phi(x)\phi(y) + \frac{1}{3}x + \frac{1}{3}y - \frac{2}{3}                 \\
% & = \phi(x) * \phi(y)
% \end{align*}

% The identity element of $*$ on $\mathbb{Q}$ is $2$.
% \item Define $*: \mathbb{Q}\times\mathbb{Q} \to \mathbb{Q}$ as follows: $x * y = 3xy - x - y + \frac{2}{3}$ for every $x, y\in\mathbb{Q}$. Then
% \begin{align*}
% \phi(x * y) & = 3\cdot (x * y) - 1    \\
% & = 9xy - 3x - 3y + 2 - 1 \\
% & = 9xy - 3x - 3y + 1     \\
% & = (3x - 1)(3y - 1)
% \end{align*}

% The identity element of $*$ on $\mathbb{Q}$ is $\frac{2}{3}$.
% \end{proof}

% \textbf{Concepts}

% \begin{exercise}
% The displayed homomorphism condition for an isomorphism $\phi$ in Definition 3.7 is sometimes summarized by saying ``$\phi$ must commute with the binary operation\@(s).\@'' Explain how that condition can be viewed in this manner.
% \end{exercise}

% \begin{proof}
% $\phi: S\to S'$, let's call $\phi(x)$ the image of $x$, where $x\in S$.

% The image of the combination of $x, y\in S$ (with respect to $*$) is the combination of the images of $x$ and $y$ (with respect to $*'$).
% \end{proof}

% In Exercises 21 and 22, correct the definition of the italicized term without reference to the text, if correction is needed, so that it is in a form acceptable for publication.

% \begin{exercise}
% A function $\phi: S\to S'$ is an \textit{isomorphism} if and only if $\phi(a * b) = \phi(a) *' \phi(b)$.
% \end{exercise}

% \begin{proof}
% Correction: A function $\phi: S\to S'$ is an \textit{isomorphism} if and only if $\phi$ is a one-to-one function from $S$ onto $S'$ and $\phi(a * b) = \phi(a) *' \phi(b)$.
% \end{proof}

% \begin{exercise}
% Let $*$ be a binary operation on a set $S$. An element $e$ of $S$ with the property $s * e = s = e * s$ is an \textit{identity element for $*$}  for all $s\in S$.
% \end{exercise}

% \begin{proof}
% No correction is needed.
% \end{proof}

% \textbf{Proof Synopnis}

% \begin{exercise}
% Give a proof synopsis of Theorem 3.13.
% \end{exercise}

% \begin{proof}
% We suppose that $e$ and $e'$ are identity elements of the binary operation $*$. We combine the two identity elements using the binary operation $*$ to deduce that $e = e'$.
% \end{proof}

% \textbf{Theory}

% \begin{exercise}
% An identity element for a binary operation $*$ as described by Definition 3.12 is sometimes referred to as ``a two-sided identity element.\@'' Using complete sentences, give analogous definition for
% \begin{enumerate}[label={\textbf{\alph*.}},itemsep=0pt]
% \item a \textit{left identity element $e_{L}$ for $*$},
% \item a \textit{right identity element $e_{R}$ for $*$}.
% \end{enumerate}

% Theorem 3.13 shows that if a two-sided identity element for $*$ exists, it is unique. Is the same true for a one-sided identity element you just defined? If so, prove it. If not, give a counterexample $\anglebracket{S,*}$ for a finite set $S$ and find the first place where the proof of Theorem 3.13 breaks down.
% \end{exercise}

% \begin{proof}
% \begin{enumerate}[label={\textbf{\alph*.}},itemsep=0pt]
% \item A \textit{left identity element $e_{L}$ for $*$} is an element that yields $x$ when combined from the left with $x$ for every element $x$ by binary operation $*$.
% \item A \textit{right identity element $e_{R}$ for $*$} is an element that yields $x$ when combined from the right with $x$ for every element $x$ by binary operation $*$.
% \end{enumerate}

% Left identity element, if exists, is not necessarily unique. Example: $S = \{ 0, 1 \}$, $*: S\times S\to S$ is defined as $0 * 0 = 0, 0 * 1 = 1, 1 * 0 = 0, 1 * 1 = 1$. In this example, $0$ and $1$ are left identity elements.

% Right identity element, if exists, is not necessarily unique. Example: $S = \{ 0, 1 \}$, $*: S\times S\to S$ is defined as $0 * 0 = 0, 0 * 1 = 0, 1 * 0 = 1, 1 * 1 = 1$. In this example, $0$ and $1$ are right identity elements.
% \end{proof}

% \begin{exercise}
% Continuing the ideas of Exercise 24, can a binary structure have a left identity $e_{L}$ and a right identity $e_{R}$ where $e_{L}\ne e_{R}$? If so, give an example, using an operation on a finite set $S$. If not, prove that it is impossible.
% \end{exercise}

% \begin{proof}
% Such binary structure does not exist.

% $e_{L} * x = x$ for every $x\in S$, so $e_{L} * e_{R} = e_{R}$. On the other hand, $x * e_{R} = x$ for every $x\in S$, so $e_{L} * e_{R} = e_{L}$. Hence $e_{L} = e_{R}$.

% Thus, if a binary structure have a left identity and a right identity, they must be identical.
% \end{proof}

% \begin{exercise}
% Recall that if $f: A \to B$ is a one-to-one function mapping $A$ onto $B$, then $\phi^{-1}(b)$ is the unique $a\in A$ such that $f(a) = b$. Prove that if $\phi: S \to S'$ is an isomorphism of $\anglebracket{S, *}$ with $\anglebracket{S', *'}$, then $\phi^{-1}$ is an isomorphism of $\anglebracket{S', *'}$ with $\anglebracket{S, *}$.
% \end{exercise}

% \begin{proof}
% $\phi$ is a one-to-one function mapping $A$ onto $B$, then $\phi^{-1}$ is a one-to-one function mapping $B$ onto $A$.

% For every $x', y'\in S$, there exist uniquely $x, y\in S$ such that $\phi(x) = x'$ and $\phi(y) = y'$.
% \begin{align*}
% \phi^{-1}(x' *' y') & = \phi^{-1}(\phi(x) *' \phi(y)) \\
% & = \phi^{-1}(\phi(x * y))        \\
% & = x * y                         \\
% & = \phi^{-1}(x') * \phi^{-1}(y')
% \end{align*}

% Hence $\phi^{-1}$ is an isomorphism of $\anglebracket{S', *'}$ with $\anglebracket{S, *}$.
% \end{proof}

% \begin{exercise}
% Prove that if $\phi: S \to S'$ is an isomorphism of $\anglebracket{S, *}$ with $\anglebracket{S', *}$ and $\psi: S' \to S''$ is an isomorphism of $\anglebracket{S', *'}$ with $\anglebracket{S'', *''}$, then the composite function $\psi\circ\phi$ is an isomorphism of $\anglebracket{S, *}$ with $\anglebracket{S'', *''}$.
% \end{exercise}

% \begin{proof}
% Because $\psi$ is one-to-one function, $(\psi\circ\phi)(x) = (\psi\circ\phi)(y)$ if and only if $\phi(x) = \phi(y)$. Because $\phi$ is one-to-one function, $\phi(x) = \phi(y)$ if and only if $x = y$. Therefore, $(\psi\circ\phi)(x) = (\psi\circ\phi)(y)$ if and only if $x = y$. So $\psi\circ\phi$ is a one-to-one function from $S$ into $S''$.

% Let $x''$ be an element of $S''$. Because $\psi$ is one-to-one function, there exists uniquely $x'\in S'$ such that $\psi(x') = x''$. Because $\phi$ is one to one function, there exists uniquely $x\in S$ such that $\phi(x) = x'$. So for every $x''\in S''$, there exists uniquely $x\in S$ such that $(\psi\circ\phi)(x) = x''$, equivalently, $\psi\circ\phi$ is onto $S''$.

% For every $x, y\in S$
% \begin{align*}
% (\psi\circ\phi)(x * y) & = \psi(\phi(x * y))                         \\
% & = \psi(\phi(x) *' \phi(y))                  \\
% & = \psi(\phi(x)) *'' \psi(\phi(y))           \\
% & = (\psi\circ\phi)(x) *'' (\psi\circ\phi)(y)
% \end{align*}

% Hence $\psi\circ\phi$ is an isomorphism of $\anglebracket{S, *}$ with $\anglebracket{S'', *''}$.
% \end{proof}

% \begin{exercise}
% Prove that the relation $\simeq$ of being isomorphic, described in Definition 3.7 is an equivalence relation on any set of binary structures. You may simply quote the results you were asked to prove in the preceding two exercises at appropriate places in your proof.
% \end{exercise}

% \begin{proof}
% A binary structure $\anglebracket{S, *}$ is isomorphic to itself (follows from the identity mapping from $S$ to $S$, which is an isomorphism). So $\simeq$ is reflexive.

% If $\anglebracket{S, *}$ is isomorphic to $\anglebracket{S', *'}$, then $\anglebracket{S', *'}$ is isomorphic to $\anglebracket{S, *}$ (Exercise 26). So $\simeq$ is symmetric.

% If $\anglebracket{S, *}$ is isomorphic to $\anglebracket{S', *'}$, and $\anglebracket{S', *'}$ is isomorphic to $\anglebracket{S'', *''}$, then $\anglebracket{S, *}$ is isomorphic to $\anglebracket{S'', *''}$ (Exercise 27). So $\simeq$ is transitive.

% Hence the relation $\simeq$ is an equivalence relation on any set of binary structures.
% \end{proof}

% In Exercises 29 through 32, give a careful proof for a skeptic that the indicated property of a binary structure $\anglebracket{S, *}$ is indeed a structural property. (In Theorem 3.14, we did this for the property, ``There is an identity element for *.\@'')

% \begin{exercise}
% The operation $*$ is commutative.
% \end{exercise}

% \begin{proof}
% Let $\phi: S\to S'$ be an isomorphism of $\anglebracket{S, *}$ with $\anglebracket{S', *'}$.

% For every $x', y'\in S$, there exist uniquely $x, y\in S$ such that $\phi(x) = x'$ and $\phi(y) = y'$.
% \begin{align*}
% x' *' y' & = \phi(x) *' \phi(y) \\
% & = \phi(x * y)        \\
% & = \phi(y * x)        \\
% & = \phi(y) *' \phi(x) \\
% & = y' * x'
% \end{align*}

% Hence $*'$ is also commutative.
% \end{proof}

% \begin{exercise}
% The operation $*$ is associative.
% \end{exercise}

% \begin{proof}
% Let $\phi: S\to S'$ be an isomorphism of $\anglebracket{S, *}$ with $\anglebracket{S', *'}$.

% For every $x', y', z'\in S$, there exist uniquely $x, y, z\in S$ such that $\phi(x) = x', \phi(y) = y'$ and $\phi(z) = z'$.
% \begin{align*}
% (x' *' y') *' z' & = (\phi(x) *' \phi(y)) *' \phi(z) \\
% & = \phi(x * y) *' \phi(z)          \\
% & = \phi((x * y) * z)               \\
% & = \phi(x * (y * z))               \\
% & = \phi(x) *' \phi(y * z)          \\
% & = \phi(x) *' (\phi(y) *' \phi(z)) \\
% & = x' *' (y' *' z')
% \end{align*}

% Hence $*'$ is also commutative.
% \end{proof}

% \begin{exercise}
% For each $c\in S$, the equation $x * x = c$ has a solution $x$ in $S$.
% \end{exercise}

% \begin{proof}
% Let $\phi: S\to S'$ be an isomorphism of $\anglebracket{S, *}$ with $\anglebracket{S', *'}$.

% For every $c'\in S'$, there exists uniquely $c\in S$ such that $\phi(c) = c'$.

% The equation $x * x = c$ has a solution $x$ in $S$, so $\phi(x) *' \phi(x) = \phi(x * x) = \phi(c) = c'$. Hence the equation $y * y = c'$ has a solution $y$ in $S'$.
% \end{proof}

% \begin{exercise}
% There exists an element $b$ in $S$ such that $b * b = b$.
% \end{exercise}

% \begin{proof}
% Let $\phi: S\to S'$ be an isomorphism of $\anglebracket{S, *}$ with $\anglebracket{S', *'}$.

% For every $b'\in S'$, there exists uniquely $b\in S$ such that $\phi(b) = b'$.

% There exists an element $b$ in $S$ such that $b * b = b$. So $b' *' b' = \phi(b) *' \phi(b) = \phi(b * b) = \phi(b) = b'$.

% Hence there exists an element $b'$ in $S'$ such that $b' * b' = b'$.
% \end{proof}

% \begin{exercise}
% Let $H$ be the subset $M_{2}(\mathbb{R})$ consisting of all matrices of the form $\begin{bmatrix}a & -b \\ b & a\end{bmatrix}$ for $a, b\in\mathbb{R}$. Exercise 23 of Section 2 shows that $H$ is closed under both matrix addition and matrix multiplication.
% \begin{enumerate}[label={\textbf{\alph*.}},topsep=0pt,itemsep=0pt]
% \item Show that $\anglebracket{\mathbb{C},+}$ is isomorphic to $\anglebracket{H,+}$.
% \item Show that $\anglebracket{\mathbb{C},\cdot}$ is isomorphic to $\anglebracket{H,\cdot}$.
% \end{enumerate}

% (We say that $H$ is a \textit{matrix representation} of the complex numbers $\mathbb{C}$.)
% \end{exercise}

% \begin{proof}
% Define the following one-to-one function $\phi$ from $\mathbb{C}$ onto $H$
% \[
% \phi(a + bi) = \begin{bmatrix}
% a & -b \\
% b & a
% \end{bmatrix}
% \]

% \begin{enumerate}[label={\textbf{\alph*.}},topsep=0pt,itemsep=0pt]
% \item According to Exercise 23 of Section 2
% \begin{align*}
% \phi((a + bi) + (c + di)) & = \phi((a + c) + (b + d)i)    \\
% & = \begin{bmatrix}
% a + c & -(b + d) \\
% b + d & a + c
% \end{bmatrix}            \\
% & = \begin{bmatrix}
% a & -b \\
% b & a
% \end{bmatrix} +
% \begin{bmatrix}
% c & -d \\
% d & c
% \end{bmatrix}                                            \\
% & = \phi(a + bi) + \phi(c + di)
% \end{align*}

% So $\phi$ is an isomorphism of $\anglebracket{\mathbb{C},+}$ and $\anglebracket{H,+}$.

% Hence $\anglebracket{\mathbb{C},+}$ is isomorphic to $\anglebracket{H,+}$.
% \item According to Exercise 23 of Section 2
% \begin{align*}
% \phi((a + bi) \cdot (c + di)) & = \phi((ac - bd) + (ad + bc)i)    \\
% & = \begin{bmatrix}
% ac - bd & -(ad + bc) \\
% ad + bc & ac - bd
% \end{bmatrix}            \\
% & = \begin{bmatrix}
% a & -b \\
% b & a
% \end{bmatrix} \cdot
% \begin{bmatrix}
% c & -d \\
% d & c
% \end{bmatrix}                                                    \\
% & = \phi(a + bi) \cdot \phi(c + di)
% \end{align*}

% So $\phi$ is an isomorphism of $\anglebracket{\mathbb{C},\cdot}$ and $\anglebracket{H,\cdot}$.

% Hence $\anglebracket{\mathbb{C},\cdot}$ is isomorphic to $\anglebracket{H,\cdot}$.
% \end{enumerate}

% \end{proof}

% \begin{exercise}
% There are 16 possible binary structures on the set $\{ a, b \}$ of two elements. How many nonisomorphic (that is, structurally different) structures are there among these 16? Phrased more precisely in terms of the isomorphism equivalence relation $\simeq$ on this set of 16 structures, how many equivalence classes are there? Write down one structure from each equivalence class.
% \end{exercise}

% \begin{proof}
% \end{proof}

\section{Subgroups}

\section{Cyclic Groups}

\section{Generating Sets and Cayley Digraphs}
