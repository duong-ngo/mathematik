\chapter{Some General Mathematical Concepts and Notation}

\section{Logical Symbolism}

% chapter 1:section 1:exercise 1
\begin{exercise}
    Check whether all of these tables agree with your concept of the corresponding logical operation. (In particular, pay attention to the fact that if $A$ is false, then the implication $A\implies B$ is always true.)
\end{exercise}

\begin{proof}
    I skip this exercise.
\end{proof}

% chapter 1:section 1:exercise 2
\begin{exercise}
    Show that the following simple, but very useful relations, which are widely used in mathematical reasoning, are true:
    \begin{enumerate}[label={(\alph*)}]
        \item $\neg(A\land B) \Leftrightarrow \neg A\lor \neg B$;
        \item $\neg(A\lor B) \Leftrightarrow \neg A\land \neg B$;
        \item $(A\implies B) \Leftrightarrow (\neg B\implies \neg A)$;
        \item $(A\implies B) \Leftrightarrow (\neg A\lor B)$;
        \item $\neg(A\implies B) \Leftrightarrow A\land \neg B$.
    \end{enumerate}
\end{exercise}

\begin{proof}
    I skip this exercise.
\end{proof}

\section{Sets and Elementary Operations on Them}

% chapter1:section2:exercise1
\begin{exercise}
    Verify the following relations.
    \begin{enumerate}[label={(\alph*)}]
        \item $(A\subset C)\land (B\subset C) \Leftrightarrow ((A\cup B)\subset C)$;
        \item $(C\subset A)\land (C\subset B) \Leftrightarrow (C\subset (A\cap B))$;
        \item $C_{M}(C_{M}A) = A$;
        \item $(A\subset C_{M}B) \Leftrightarrow (B\subset C_{M}A)$;
        \item $(A\subset B)\Leftrightarrow (C_{M}A \supset C_{M}B)$.
    \end{enumerate}
\end{exercise}

\begin{proof}
    I skip this exercise. These relations are true.
\end{proof}
\newpage

% chapter1:section2:exercise2
\begin{exercise}
    Prove the following statements.
    \begin{enumerate}[label={(\alph*)}]
        \item $A\cup (B\cup C) = (A\cup B)\cup C =: A\cup B\cup C$;
        \item $A\cap (B\cap C) = (A\cap B)\cap C =: A\cap B\cap C$;
        \item $A\cap (B\cup C) = (A\cap B)\cup (A\cap C)$;
        \item $A\cup (B\cap C) = (A\cup B)\cap (A\cup C)$.
    \end{enumerate}
\end{exercise}

\begin{proof}
\end{proof}
\newpage

% chapter1:section2:exercise3
\begin{exercise}
    Verify the connection (duality) between the operations of union and intersection:
    \begin{enumerate}[label={(\alph*)}]
        \item $C_{M}(A\cup B) = C_{M}A\cap C_{M}B$.
        \item $C_{M}(A\cap B) = C_{M}A\cup C_{M}B$.
    \end{enumerate}
\end{exercise}

\begin{proof}
    I skip this exercise.
\end{proof}
\newpage

% chapter1:section2:exercise4
\begin{exercise}
    Give geometric representations of the following Cartesian products.
    \begin{enumerate}[label={(\alph*)}]
        \item The product of two line segments (a rectangle).
        \item The product of two lines (a plane).
        \item The product of a line and a circle (an infinite cylindrical surface).
        \item The product of a line and a disk (an infinite solid cylinder).
        \item The product of two circles (a torus).
        \item The product of a circle and a disk (a solid torus).
    \end{enumerate}
\end{exercise}

\begin{proof}
    I skip this exercise.
\end{proof}
\newpage

% chapter1:section2:exercise5
\begin{exercise}
    The set $\Delta = \{ (x_{1}, x_{2})\in X^{2} \mid x_{1} = x_{2} \}$ is called the \textit{diagonal} of the Cartesian square $X^{2}$ of the set $X$. Give geometric representations of the diagonals of the sets
    obtained in parts (a), (b), and (e) of Exercise 4.
\end{exercise}

\begin{proof}
    (a) It is a diagonal of the rectangle.

    (b) It is the line having equation $y = x$.

    (e) It is another circle that passes through the intersection of the two circles of which Cartesian product is the given torus.
\end{proof}
\newpage

% chapter1:section2:exercise6
\begin{exercise}
    Show that
    \begin{enumerate}[label={(\alph*)}]
        \item $(X\times Y = \varnothing) \Leftrightarrow (X = \varnothing) \lor (Y = \varnothing)$, and if $X\times Y\ne \varnothing$, then
        \item $(A\times B\subset X\times Y) \Leftrightarrow (A\subset X)\land (B\subset Y)$,
        \item $(X\times Y)\cup (Z\times Y) \Leftrightarrow (X\cup Z)\times Y$,
        \item $(X \times Y) \cap (X'\times Y') = (X \cap X') \times (Y \cap Y')$.
    \end{enumerate}

    Here $\varnothing$ denotes the empty set, that is, the set having no elements.
\end{exercise}

\begin{proof}
    I skip this exercise.
\end{proof}
\newpage

% chapter1:section2:exercise7
\begin{exercise}
    By comparing the relations of Exercise 3 with relations (a) and (b) from Exercise 2 of Sect. 1.1, establish a correspondence between the logical operators $\neg, \land, \lor$ and the operations $C$, $\cap$, and $\cup$ on sets.
\end{exercise}

\begin{proof}
    \[
        CA = \{ x \mid \neg(x\in A) \}.
    \]
    \[
        A\cap B = \{ x \mid (x\in A)\land (x\in B) \}.
    \]
    \[
        A\cup B = \{ x \mid (x\in A)\lor (x\in B) \}.
    \]
\end{proof}
\newpage

\section{Functions}

% chapter1:section3:exercise1
\begin{exercise}
    The \textit{composition} $\mathcal{R}_{2}\circ \mathcal{R}_{1}$ of the relations $\mathcal{R}_{1}$ and $\mathcal{R}_{2}$ is defined as follows:
    \[
        \mathcal{R}_{2}\circ\mathcal{R}_{1} := \{ (x, z) \mid \exists y\, (x\mathcal{R}_{1}y \land y\mathcal{R}_{2}z) \}.
    \]

    In particular, if $\mathcal{R}_{1}\subset X\times Y$ and $\mathcal{R}_{2}\subset Y\times Z$, then $\mathcal{R} = \mathcal{R}_{2}\circ \mathcal{R}_{1}\subset X\times Z$, and
    \[
        x\mathcal{R}z := \exists y\, \left((y\in Y)\land (x\mathcal{R}_{1}y)\land (y\mathcal{R}_{2}z)\right).
    \]

    \begin{enumerate}[label={(\alph*)}]
        \item Let $\Delta_{X}$ be the diagonal of $X^{2}$ and $\Delta_{Y}$ the diagonal of $Y^{2}$. Show that if the relations $\mathcal{R}_{1}\subset X\times Y$ and $\mathcal{R}_{2}\subset Y\times X$ are such that $(\mathcal{R}_{2}\circ\mathcal{R}_{1} = \Delta_{X})\land (\mathcal{R}_{1}\circ\mathcal{R}_{2} = \Delta_{Y})$, then both relations are functional and defined mutually inverse mappings of $X$ and $Y$.
        \item Let $\mathcal{R}\subset X^{2}$. Show that the condition of transitivity of the relation $\mathcal{R}$ is equivalent to the condition $\mathcal{R}\circ \mathcal{R}\subset \mathcal{R}$.
        \item The relation $\mathcal{R}'\subset Y\times X$ is called the \textit{transpose} of the relation $\mathcal{R}\subset X\times Y$ if $(y\mathcal{R}'x) \Leftrightarrow (x\mathcal{R}y)$.

              Show that a relation $\mathcal{R}\subset X^{2}$ is antisymmetric if and only if $\mathcal{R}\cap \mathcal{R}'\subset \Delta_{X}$.
        \item Verify that any two elements of $X$ are connected (in some order) by the relation $\mathcal{R} \subseteq X^{2}$ if and only if $\mathcal{R}\cup \mathcal{R}' = X^{2}$.
    \end{enumerate}
\end{exercise}

\begin{proof}
    \begin{enumerate}[label={(\alph*)}]
        \item Assume that $x\mathcal{R}_{1}y$ and $x\mathcal{R}_{1}y'$. Because $\mathcal{R}_{2}\circ\mathcal{R}_{1} = \Delta_{X}$, it follows that $y\mathcal{R}_{2}x$ and $y'\mathcal{R}_{2}x$. Because $\mathcal{R}_{1}\circ\mathcal{R}_{2} = \Delta_{Y}$, it follows that $y\mathcal{R}y'$, so $y = y'$. Hence $\mathcal{R}_{1}$ is functional.

              Assume that $y\mathcal{R}_{2}x$ and $y\mathcal{R}_{2}x'$. Because $\mathcal{R}_{1}\circ\mathcal{R}_{2} = \Delta_{Y}$, it follows that $x\mathcal{R}_{1}y$ and $x'\mathcal{R}_{1}y$. Because $\mathcal{R}_{2}\circ\mathcal{R}_{1} = \Delta_{X}$, it follows that $x\mathcal{R}x'$, so $x = x'$. Hence $\mathcal{R}_{2}$ is functional.

              On the other hand $\mathcal{R}_{2}\circ\mathcal{R}_{1} = \Delta_{X}$ means $X$ is the domain of $\mathcal{R}_{1}$, $\mathcal{R}_{1}\circ\mathcal{R}_{2} = \Delta_{Y}$ means $Y$ is the domain of $\mathcal{R}_{2}$.

              $x\mathcal{R}_{1}y$ iff $y\mathcal{R}_{2}x$.
              \[
                  \begin{split}
                      (\mathcal{R}_{2}\circ\mathcal{R}_{1})(x) = \mathcal{R}_{2}(y) = x \\
                      (\mathcal{R}_{1}\circ\mathcal{R}_{2})(y) = \mathcal{R}_{1}(x) = y.
                  \end{split}
              \]

              So $\mathcal{R}_{1}$ and $\mathcal{R}_{2}$ are indeed mutually inverse mappings of $X$ and $Y$.
        \item $(\Rightarrow)$ $\mathcal{R}\circ\mathcal{R}\subset \mathcal{R}$ then $x\mathcal{R}y$ and $y\mathcal{R}z$ implies $x\mathcal{R}z$ (due to the definition of composition). So $\mathcal{R}$ is transitive.

              $(\Leftarrow)$ $\mathcal{R}$ is transitive then $x (\mathcal{R}\circ\mathcal{R}) z$ iff there exists $y$ such that $x\mathcal{R}y$ and $y\mathcal{R}z$. Therefore $x\mathcal{R}z$ because $\mathcal{R}$ is transitive. Hence $(x, z)\in \mathcal{R}$, which means $\mathcal{R}\circ\mathcal{R} \subset \mathcal{R}$.
        \item $(\Rightarrow)$ $\mathcal{R}$ is antisymmetric.

              If $(x, y)\in \mathcal{R}\cap\mathcal{R'}$, then $x\mathcal{R}y$ and $x\mathcal{R}'y$. It follows that $y\mathcal{R}'x$ and $y\mathcal{R}x$. So $x = y$ because $\mathcal{R}$ is antisymmetric. Therefore $(x, y)\in \Delta_{X}$. Hence $\mathcal{R}\cap \mathcal{R}'\subset \Delta_{X}$.

              $(\Leftarrow)$ $\mathcal{R}\cap\mathcal{R}'\subset \Delta_{X}$.

              If $x\mathcal{R}y$ and $y\mathcal{R}x$, then $y\mathcal{R}'x$ and $x\mathcal{R}'y$. So $(x, y)\in \mathcal{R}\cap\mathcal{R}'\subset \Delta_{X}$. Therefore $x = y$. Hence $\mathcal{R}$ is antisymmetric.
        \item $(\Rightarrow)$ For all $x, y\in X$, $(x\mathcal{R}y)\lor (y\mathcal{R}x)$.

              So for all $x, y\in X$, $(y\mathcal{R}'x)\lor (x\mathcal{R}'y)$. Therefore $(x, y)\in \mathcal{R}\cup \mathcal{R}'$. Hence $X^{2}\subset \mathcal{R}\cup \mathcal{R}'$. On the other hand, $\mathcal{R}\cup \mathcal{R}'\subset X^{2}$, so $\mathcal{R}\cup \mathcal{R}' = X^{2}$.

              $(\Leftarrow)$ $\mathcal{R}\cup \mathcal{R}' = X^{2}$.

              Let $x, y$ be any two elements of $X$. Since $(x, y) \in X^{2}$, it follows that $(x\mathcal{R}y) \lor (x\mathcal{R}'y)$. Equivalently, $(x\mathcal{R}y)\lor (y\mathcal{R}x)$. Hence for all $x, y\in X$, $(x\mathcal{R}y)\lor (y\mathcal{R}x)$.
    \end{enumerate}
\end{proof}
\newpage

% chapter1:section3:exercise2
\begin{exercise}
    Let $f: X\to Y$ be a mapping. The pre-image $f^{-1}(y)\subset X$ of the element $y\in Y$ is called the \textit{fiber} over $y$.
    \begin{enumerate}[label={(\alph*)}]
        \item Find the fibers for the following mappings:
              \[
                  \operatorname{pr}_{1}: X_{1}\times X_{2} \to X_{1},\qquad \operatorname{pr}_{2}: X_{1}\times X_{2} \to X_{2}.
              \]
        \item An element $x_{1}\in X$ will be considered to be connected with an element $x_{2}\in X$ by the relation $\mathcal{R}\subset X^{2}$, and we shall write $x_{1} \mathcal{R} x_{2}$ if $f(x_{1}) = f(x_{2})$, that is, $x_{1}$ and $x_{2}$ both lie in the same fiber.

              Verify that $\mathcal{R}$ is an equivalence relation.
        \item Show that the fibers of a mapping $f: X \to Y$ do not intersect one another and that the union of all the fibers is the whole set $X$.
        \item Verify that any equivalence relation between elements of a set makes it possible to represent the set as a union of mutually disjoint equivalence classes of elements.
    \end{enumerate}
\end{exercise}

\begin{proof}
    \begin{enumerate}[label={(\alph*)}]
        \item Fibers of $\operatorname{pr}_{1}$ are the sets $\{ x_{1} \}\times X_{2}$, where $x_{1}\in X_{1}$.

              Fibers of $\operatorname{pr}_{2}$ are the sets $X_{1}\times\{ x_{2} \}$, where $x_{2}\in X_{2}$.
        \item $f(x_{1}) = f(x_{1})$, so $x_{1}\mathcal{R}x_{1}$. Therefore $\mathcal{R}$ is reflexive.

              $f(x_{1}) = f(x_{2})$ if and only if $f(x_{2}) = f(x_{1})$, so $(x_{1}\mathcal{R}x_{2})\Leftrightarrow (x_{2}\mathcal{R}x_{1})$. Therefore $\mathcal{R}$ is symmetric.

              $f(x_{1}) = f(x_{2})$ and $f(x_{2}) = f(x_{3})$ implies $f(x_{1}) = f(x_{3})$, so $(x_{1}\mathcal{R}x_{2})\land (x_{2}\mathcal{R}x_{3})\implies (x_{1}\mathcal{R}x_{3})$. Therefore $\mathcal{R}$ is transitive.

              Hence $\mathcal{R}$ is an equivalence relation.
        \item The union of all fibers is a subset of $X$.

              Let $x\in X$, then $x$ is an element of the fiber over $f(x)$. Therefore $X$ is a subset of the union of all fibers.

              Thus the union of all fibers is the whole set $X$.
        \item I skip this exercise.
    \end{enumerate}
\end{proof}
\newpage

% chapter1:section3:exercise3
\begin{exercise}
    Let $f: X\to Y$ be a mapping from $X$ into $Y$. Show that if $A$ and $B$ are subsets of $X$, then
    \begin{enumerate}[label={(\alph*)}]
        \item $(A\subset B)\implies (f(A)\subset f(B))$,
        \item $(A\ne\varnothing) \implies (f(A) \ne \varnothing)$,
        \item $f(A\cap B) \subset f(A)\cap f(B)$
        \item $f(A\cup B) = f(A)\cup f(B)$;
    \end{enumerate}

    if $A'$ and $B'$ are subsets of $Y$, then
    \begin{enumerate}[label={(\alph*)}]
        \setcounter{enumi}{4}
        \item $(A'\subset B')\implies (f^{-1}(A')\subset f^{-1}(B'))$,
        \item $f^{-1}(A'\cap B') = f^{-1}(A')\cap f^{-1}(B')$,
        \item $f^{-1}(A'\cup B') = f^{-1}(A')\cup f^{-1}(B')$;
    \end{enumerate}

    if $Y\supset A'\supset B'$, then
    \begin{enumerate}[label={(\alph*)}]
        \setcounter{enumi}{7}
        \item $f^{-1}(A'\setminus B') = f^{-1}(A')\setminus f^{-1}(B')$,
        \item $f^{-1}(C_{Y}A') = C_{X}f^{-1}(A')$;
    \end{enumerate}

    and for any $A\subset X$ and $B'\subset Y$
    \begin{enumerate}[label={(\alph*)}]
        \setcounter{enumi}{9}
        \item $f^{-1}(f(A))\supset A$,
        \item $f(f^{-1}(B'))\subset B'$.
    \end{enumerate}
\end{exercise}

\begin{proof}
    \begin{enumerate}[label={(\alph*)}]
        \item I skip this exercise.
        \item I skip this exercise.
        \item If $f(A\cap B) = \varnothing$ then $f(A\cap B)\subset f(A)\cap f(B)$.

              If $f(A\cap B)\ne\varnothing$, let $y\in f(A\cap B)$. Then there exists $x\in A\cap B$ such that $y = f(x)$. Moreover, $x\in A$ and $x\in B$ so $y\in f(A)$ and $y\in f(B)$, which implies $y\in f(A)\cap f(B)$.

              Hence $f(A\cap B)\subset f(A)\cap f(B)$.
        \item If $y\in f(A\cup B)$ then there exists $x\in A\cup B$ such that $y = f(x)$. Because $x\in A\cup B$, it follows that $x\in A$ or $x\in B$, and $f(x)\in f(A)$ or $f(x)\in f(B)$. So $y = f(x)\in f(A)\cup f(B)$. Hence $f(A\cup B)\subset f(A)\cup f(B)$.

              If $y\in f(A)\cup f(B)$ then $y\in f(A)$ or $y\in f(B)$. So there exists $x_{1}\in A$ such that $y = f(x_{1})$ or there exists $x_{2}\in B$ such that $y = f(x_{2})$. Equivalently, this means there exists $x\in A\cup B$ such that $y = f(x)$. So $y\in f(A\cup B)$. Hence $f(A)\cup f(B)\subset f(A\cup B)$.

              Thus $f(A\cup B) = f(A)\cup f(B)$.
        \item Assume $A'\subset B'$. Let $x\in f^{-1}(A')$, then $f(x)\in A'\subset B'$. Therefore $x\in f^{-1}(B)$. It follows that $f^{-1}(A)\subset f^{-1}(B)$.

              Thus $(A'\subset B')\implies (f^{-1}(A')\subset f^{-1}(B'))$.
        \item \begin{align*}
                  x\in f^{-1}(A'\cap B') & \Leftrightarrow (f(x)\in A'\cap B') \Leftrightarrow ((f(x)\in A') \land (f(x)\in B')) \\
                                         & \Leftrightarrow ((x\in f^{-1}(A') \land x\in f^{-1}(B')))                             \\
                                         & \Leftrightarrow (x\in (f^{-1}(A')\cap f^{-1}(B)))
              \end{align*}

              Hence $f^{-1}(A'\cap B') = f^{-1}(A')\cap f^{-1}(B')$.
        \item \begin{align*}
                  x\in f^{-1}(A'\cup B') & \Leftrightarrow (f(x)\in A'\cup B') \Leftrightarrow ((f(x)\in A') \lor (f(x)\in B')) \\
                                         & \Leftrightarrow ((x\in f^{-1}(A') \lor x\in f^{-1}(B')))                             \\
                                         & \Leftrightarrow (x\in (f^{-1}(A')\cup f^{-1}(B)))
              \end{align*}

              Hence $f^{-1}(A'\cup B') = f^{-1}(A')\cup f^{-1}(B')$.
        \item \begin{align*}
                  x\in f^{-1}(A'\setminus B') & \Leftrightarrow (f(x)\in A'\setminus B') \Leftrightarrow ((f(x)\in A') \land (f(x)\notin B')) \\
                                              & \Leftrightarrow ((x\in f^{-1}(A') \land x\notin f^{-1}(B')))                                  \\
                                              & \Leftrightarrow (x\in (f^{-1}(A')\setminus f^{-1}(B)))
              \end{align*}

              Hence $f^{-1}(A'\setminus B') = f^{-1}(A')\setminus f^{-1}(B')$.
        \item \begin{align*}
                  f^{-1}(C_{Y}A') & = f^{-1}(Y\setminus A) = f^{-1}(Y)\setminus f^{-1}(A') \\
                                  & = X\setminus f^{-1}(A') = C_{X}f^{-1}(A').
              \end{align*}

              Hence $f^{-1}(A'\setminus B') = f^{-1}(A')\setminus f^{-1}(B')$.
        \item Let $x\in A$, then $f(x)\in f(A)$, and then the fiber over $f(x)$ is a subset of $f^{-1}(f(A))$. Therefore $x\in f^{-1}(f(A))$. Hence $f^{-1}(f(A))\supset A$.
        \item Let $y\in f(f^{-1}(B'))$, then there exists $x\in f^{-1}(B')$ such that $y = f(x)$. Since $x\in f^{-1}(B')$, it follows that $f(x)\in B'$, so $y\in B'$. Hence $f(f^{-1}(B'))\subset B'$.
    \end{enumerate}
\end{proof}
\newpage

% chapter1:section3:exercise4
\begin{exercise}
    Show that the mapping $f: X \to Y$ is
    \begin{enumerate}[label={(\alph*)}]
        \item surjective if and only if $f(f^{-1}(B')) = B'$ for every set $B'\subset Y$;
        \item bijective if and only if
              \[
                  \left(f^{-1}(f(A)) = A\right) \land \left( f(f^{-1}(B')) = B' \right)
              \]

              for every set $A\subset X$ and every $B'\subset Y$.
    \end{enumerate}
\end{exercise}

\begin{proof}
    \begin{enumerate}[label={(\alph*)}]
        \item $(\Rightarrow)$ $f$ is surjective.

              For every $B'\subset Y$, if $y\in B'$, then there exists $x\in X$ such that $y = f(x)$ because $f$ is surjective. Moreover, $x\in f^{-1}(B')$ according to the definition of pre-image. So $y = f(x)\in f(f^{-1}(B'))$. Therefore $B'\subset f(f^{-1}(B'))$. On the other hand, $f(f^{-1}(B'))\subset B'$, so $f(f^{-1}(B')) = B'$ for every set $B'\subset Y$.

              $(\Leftarrow)$ $f(f^{-1}(B')) = B'$ for every set $B'\subset Y$.

              Let $y\in Y$, then $f(f^{-1}(\{y\})) = \{y\}$. So $f^{-1}(\{y\})$ is not empty, hence there exists $x\in X$ such that $f(x) = y$. Therefore $f$ is surjective.
        \item $(\Rightarrow)$ $f$ is bijective.

              According to (a), $f(f^{-1}(B')) = B'$ for every $B'\subset Y$.

              For every $A\subset X$, $f^{-1}(f(A))\supset A$. Let $x\in f^{-1}(f(A))$, then $f(x)\in f(A)$. If $x\notin A$, it violates the injectivity of $f$, so $x\in A$. Therefore $f^{-1}(f(A))\subset A$. Hence $f^{-1}(f(A)) = A$.

              $(\Leftarrow)$ $\left(f^{-1}(f(A)) = A\right) \land \left( f(f^{-1}(B')) = B' \right)$ for every set $A\subset X$ and every $B'\subset Y$.

              According to (a), $f$ is surjective.

              If $f(x_{1}) = f(x_{2})$, then $f^{-1}(f(\{x_{1}\})) = f^{-1}(f(\{x_{2}\}))$. It follows that $\{ x_{1} \} = \{ x_{2} \}$, which implies $x_{1} = x_{2}$. Hence $f$ is injective. Therefore $f$ is bijective.

              \bigskip

              Thus $f$ is bijective iff $\left(f^{-1}(f(A)) = A\right) \land \left( f(f^{-1}(B')) = B' \right)$ for every set $A\subset X$ and every $B'\subset Y$.
    \end{enumerate}
\end{proof}
\newpage

% chapter1:section3:exercise5
\begin{exercise}
    Verify that the following statements about a mapping $f: X \to Y$ are equivalent:
    \begin{enumerate}[label={(\alph*)}]
        \item $f$ is injective;
        \item $f^{-1}(f(A)) = A$ for every $A\subset X$;
        \item $f(A\cap B) = f(A)\cap f(B)$ for any two subsets $A$ and $B$ of $X$;
        \item $f(A)\cap f(B) = \varnothing \Leftrightarrow A\cap B = \varnothing$;
        \item $f(A\setminus B) = f(A)\setminus f(B)$ whenever $X\supset A\supset B$.
    \end{enumerate}
\end{exercise}

\begin{proof}
    \begin{itemize}
        \item Assume (a) is true.

              For every $A\subset X$, $f^{-1}(f(A))\supset A$. Let $x\in f^{-1}(f(A))$, then $f(x)\in f(A)$. If $x\notin A$, it violates the injectivity of $f$, so $x\in A$. Therefore $f^{-1}(f(A))\subset A$. Hence $f^{-1}(f(A)) = A$. Hence $(a)\implies (b)$.
        \item Assume (b) is true.

              If $f(x_{1}) = f(x_{2})$, then $f^{-1}(f(\{x_{1}\})) = f^{-1}(f(\{x_{2}\}))$. It follows that $\{ x_{1} \} = \{ x_{2} \}$, which implies $x_{1} = x_{2}$. Hence $f$ is injective. Therefore $f$ is bijective. Hence $(b)\implies (a)$.
        \item Assume (a) is true.

              For every subsets $A, B$ of $X$, $f(A\cap B)\subset f(A)\cap f(B)$.

              Let $y\in f(A)\cap f(B)$, then $y\in f(A)$ and $y\in f(B)$. There exists $x_{1}\in A$ and $x_{2}\in B$ such that $f(x_{1}) = y$ and $f(x_{2}) = y$. Because $f$ is injective, it follows that $x_{1} = x_{2}$ and $x_{1}, x_{2}\in A\cap B$. So $y = f(x_{1}) = f(x_{2})\in f(A\cap B)$. Therefore $f(A\cap B) \supset f(A)\cap f(B)$, and we conclude that $f(A\cap B) = f(A)\cap f(B)$ for any two subsets $A$ and $B$ of $X$.

              Hence $(a)\implies (c)$.
        \item Assume (c) is true.

              $f(A)\cap f(B) = f(A\cap B)$. Moreover $f(A\cap B) = \varnothing \Leftrightarrow A\cap B = \varnothing$. Therefore $f(A)\cap f(B) = \varnothing \Leftrightarrow A\cap B = \varnothing$. Hence $(c)\implies (d)$.
        \item Assume (d) is true.

              If $x_{1}\ne x_{2}$ and $x_{1}, x_{2}\in X$, then $f(\{x_{1}\})\ne f(\{x_{2}\})$. Therefore $f(x_{1})\ne f(x_{2})$. Hence $f$ is injective. Hence $(d)\implies (a)$.
        \item Assume (a) is true.

              For every $B\subset A\subset X$, $f(A)\setminus f(B)\subset f(A\setminus B)$. Let $y\in f(A\setminus B)$, then there exists $x\in A\setminus B$ such that $f(x) = y$. Because $x\in A$, it follows that $y = f(x)\in f(A)$. Because $x\notin B$ and $f$ is injective, it follows that $y = f(x)\notin f(B)$ (proved by contradiction). Therefore $f(A\setminus B)\subset f(A)\setminus f(B)$. Hence $f(A\setminus B) = f(A)\setminus f(B)$. Hence $(a)\implies (e)$.
        \item Assume (e) is true.

              If $f(A) \cap f(B) = \varnothing$, then $A\cap B = \varnothing$, because $f(A\cap B)\subset f(A)\cap f(B)$, for every sets $A, B\subset X$.

              If $A\cap B = \varnothing$, then $A\setminus B = A$. Moreover
              \[
                  f(A)\setminus f(B) = f(A\setminus B) = f(A).
              \]

              So $f(A)\cap f(B) = \varnothing$. Hence $(e)\implies (d)$.
    \end{itemize}

    \bigskip
    Thus the statements are equivalent.
\end{proof}
\newpage

% chapter1:section3:exercise6
\begin{exercise}
    \begin{enumerate}[label={(\alph*)}]
        \item If the mappings $f: X\to Y$ and $g: Y\to X$ are such that $g\circ f = e_{X}$, where $e_{X}$ is the identity mapping on $X$, then $g$ is called a \textit{left inverse} of $f$ and $f$ a \textit{right inverse} of $g$. Show that, in contrast to the uniqueness of the inverse mapping, there
              may exist many one-sided inverse mappings.
    \end{enumerate}

    Consider, for example, the mappings $f: X \to Y$ and $g: Y \to X$, where $X$ is a one-element set and $Y$ a two-element set, or the mappings of sequences given by
    \begin{align*}
        (x_{1}, \ldots, x_{n}, \ldots) \stackrel{f_{a}}{\longmapsto} (a, x_{1}, \ldots, x_{n}, \ldots), \\
        (y_{2}, \ldots, y_{n}, \ldots) \stackrel{g}{\longmapsfrom} (y_{1}, y_{2}, \ldots, y_{n}, \ldots).
    \end{align*}
    \begin{enumerate}[label={(\alph*)}]
        \setcounter{enumi}{1}
        \item Let $f: X\to Y$ and $g: Y\to Z$ be bijective mappings. Show that the mapping $g\circ f: X\to Z$ is bijective and that ${(g\circ f)}^{-1} = f^{-1}\circ g^{-1}$.
        \item Show that the equality
              \[
                  {(g\circ f)}^{-1}(C) = f^{-1}(g^{-1}(C))
              \]

              holds for any mappings $f: X\to Y$ and $g: Y\to Z$ and set $C\subset Z$.
        \item Verify that the mapping $F: X\times Y\to Y\times X$ defined by the correspondence $(x, y)\mapsto (y, x)$ is bijective. Describe the connection between the graphs of mutually inverse mappings $f: X\to Y$ and $f^{-1}: Y\to X$.
    \end{enumerate}
\end{exercise}

\begin{proof}
    \begin{enumerate}[label={(\alph*)}]
        \item $g\circ f_{a} = e_{X}$. So $g$ has many right inverse.
        \item Let $z\in Z$. Because $g$ is bijective, there exists $y\in Y$ such that $g(y) = z$. Because $f$ is bijective, there exists $x\in X$ such that $f(x) = y$. Then $(g\circ f)(x) = z$, so $g\circ f$ is surjective.

              Let $x_{1}, x_{2}\in X$ and $x_{1}\ne x_{2}$. Because $f$ is bijective, $f(x_{1})\ne f(x_{2})$. Because $g$ is bijective, $g(f(x_{1}))\ne g(f(x_{2}))$. So $g\circ f$ is injective.

              Thus $g\circ f$ is bijective. Moreover, according to the definition of bijection
              \[
                  {(g\circ f)}^{-1} \circ (g\circ f) = e_{X}\qquad (g\circ f)\circ {(g\circ f)}^{-1} = e_{Z}.
              \]

              On the other hand
              \begin{align*}
                  (f^{-1}\circ g^{-1})\circ (g\circ f) & = f^{-1}\circ (g^{-1}\circ g)\circ f = f^{-1} \circ e_{Y} \circ f = f^{-1}\circ f = e_{X}, \\
                  (g\circ f)\circ (f^{-1}\circ g^{-1}) & = g\circ (f\circ f^{-1})\circ g^{-1} = g\circ e_{Y} \circ g^{-1} = g\circ g^{-1} = e_{Z}.
              \end{align*}

              Hence ${(g\circ f)}^{-1} = f^{-1}\circ g^{-1}$, due to the uniqueness of the inverse mapping.
        \item If $x\in {(g\circ f)}^{-1}(C)$, then $(g\circ f)(x)\in C$. Therefore  $f(x)\in g^{-1}(C)$, and then $x\in f^{-1}(g^{-1}(C))$. Hence ${(g\circ f)}^{-1}(C)\subset f^{-1}(g^{-1}(C))$.

              If $x\in f^{-1}(g^{-1}(C))$, then $f(x)\in f(f^{-1}(g^{-1}(C)))\subset g^{-1}(C)$. Therefore $g(f(x))\in g(g^{-1}(C))\subset C$. Hence $x\in {(g\circ f)}^{-1}(C)$.

              Therefore ${(g\circ f)}^{-1}(C) = f^{-1}(g^{-1}(C))$.
        \item Let $G: Y\times X\to X\times Y$ and $g: (y, x)\mapsto (x, y)$. Then $G\circ F = e_{X\times Y}$ and $F\circ G = e_{Y\times X}$. Therefore $F$ is bijective.

              The graph of $f^{-1}: Y\to X$ is the range of the graph of $f: X\to Y$ under $F$.
    \end{enumerate}
\end{proof}
\newpage

% chapter1:section3:exercise7
\begin{exercise}
    \begin{enumerate}[label={(\alph*)}]
        \item Show that for any mapping $f: X\to Y$ the mapping $F: X\to X\times Y$ defined by the correspondence $x\stackrel{F}{\longmapsto} (x, f(x))$ is injective.
        \item Suppose a particle is moving at uniform speed on a circle $Y$; let $X$ be the time axis and $x\stackrel{f}{\longmapsto} y$ the correspondence between the time $X$ and the position $y = f(x)\in Y$ of the particle. Describe the graph of the function $f: X\to Y$ in $X\times Y$.
    \end{enumerate}
\end{exercise}

\begin{proof}
    \begin{enumerate}[label={(\alph*)}]
        \item If $x_{1}\ne x_{2}$ and $x_{1}, x_{2}\in X$, then $(x_{1}, f(x_{1}))\ne (x_{2}, f(x_{2}))$. Therefore $F$ is injective.
        \item After a fixed period of time (let it be $T$ (seconds), the time the particle needs to go around the full circle), the particle gets back to the position where it was at $T$ seconds ago.
    \end{enumerate}
\end{proof}
\newpage

% chapter1:section3:exercise8
\begin{exercise}
    \begin{enumerate}[label={(\alph*)}]
        \item For each of the examples 1{-}12 considered in Sect. 1.3 determine whether the mapping defined in the example is surjective, injective, or bijective or whether it belongs to none of these classes.
        \item Ohm{'}s law $I = V/R$ connects the current $I$ in a conductor with the potential difference $V$ at the ends of the conductor and the resistance $R$ of the conductor. Give sets $X$ and $Y$ for which some mapping $O: X\to Y$ corresponds to Ohm{'}s law. What set is the relation corresponding to Ohm{'}s law a subset of?
        \item Find the mapping $G^{-1}$ and $L^{-1}$ inverse to the Galilean and Lorentz transformations.
    \end{enumerate}
\end{exercise}

\begin{proof}
    \begin{enumerate}[label={(\alph*)}]
        \item \begin{enumerate}[label={\arabic*.}]
                  \item These two functions are bijective.
                  \item These functions are not injective or surjective.
                  \item These functions are bijective.
                  \item These functions are surjective.
                  \item This function is bijective.
                  \item This function is not injective or surjective.
                  \item These functions are not injective or surjective (the conclusion varies).
                  \item These functions are not injective or surjective.
                  \item These functions are bijective.
                  \item These functions are not injective or surjective.
                  \item This function is not injective or surjective.
                  \item This function is not injective or surjective.
              \end{enumerate}
        \item The relation corresponding to Ohm's law is a subset of $A\times B$ where $A$ is the set of potential difference values and $B$ is the set of the current values.
        \item \[
                  G(t, x) = \left(t, x - vt\right)\qquad G^{-1}(t, x) = (t, x + vt)
              \]
              \[
                  L(t, x) = \left(\frac{t - \left(\frac{v}{c^{2}}\right)x}{\sqrt{1 - {\left(\frac{v}{c}\right)}^{2}}}, \frac{x - vt}{\sqrt{1 - {\left(\frac{v}{c}\right)}^{2}}}\right)\qquad L^{-1}(t, x) = \left(\frac{t + \left(\frac{v}{c^{2}}\right)x}{\sqrt{1 - {\left(\frac{v}{c}\right)}^{2}}}, \frac{x' + vt'}{\sqrt{1 - {\left(\frac{v}{c}\right)}^{2}}} \right).
              \]
    \end{enumerate}
\end{proof}
\newpage

% chapter1:section3:exercise9
\begin{exercise}
    \begin{enumerate}[label={(\alph*)}]
        \item A set $S \subset X$ is \textit{stable} with respect to a mapping $f: X \to X$ if $f(S) \subset S$. Describe the sets that are stable with respect to a shift of the plane by a given vector lying in the plane.
        \item A set $I \subset X$ is \textit{invariant} with respect to a mapping $f: X \to X$ if $f(I) = I$. Describe the sets that are invariant with respect to rotation of the plane about a fixed point.
        \item A point $p\in X$ is a \textit{fixed point} of a mapping $f: X\to X$ if $f(p) = p$. Verify that any composition of a shift, a rotation, and a similarity transformation of the plane has a fixed point, provided the coefficient of the similarity transformation is less than $1$.
        \item Regarding the Galilean and Lorentz transformations as mappings of the plane into itself for which the point with coordinates $(x, t)$ maps to the point with coordinates $(x' , t')$, find the invariant sets of these transformations.
    \end{enumerate}
\end{exercise}

\begin{proof}
    \begin{enumerate}[label={(\alph*)}]
        \item The sets that are stable with respect to a shift of the plane by vector $v$ are unions of lines that are parallel to $v$.
        \item The sets that are invariant with respect to rotation of the plane about a fixed point are unions of circles centered at the fixed point.
        \item Such a transformation is given by $z\mapsto \alpha z + \beta$ (direct) or $z\mapsto \alpha \overline{z} + \beta$ (inverse), where $0 < \abs{\alpha} < 1$. In both cases, the transformation has a fixed point.
        \item The invariant sets of the Galilean transformation are unions of $\{ (t_{0}, x_{0} - nvt_{0}) \mid n\in\mathbb{Z} \}$.

              Lorentz transformation is linear. The eigenvalues of it are
              \[
                  \frac{1 + \frac{v}{c}}{\sqrt{1 - {\left(\frac{v}{c}\right)}^{2}}}\qquad \frac{1 - \frac{v}{c}}{\sqrt{1 - {\left(\frac{v}{c}\right)}^{2}}}
              \]

              and the corresponding eigenvectors are
              \[
                  \operatorname{span}\left(\left(1, -c\right)\right)\qquad \operatorname{span}\left(\left(1, c\right)\right).
              \]
    \end{enumerate}
\end{proof}
\newpage

% chapter1:section3:exercise10
\begin{exercise}
    Consider the steady flow of a fluid (that is, the velocity at each point of the flow does not change over time). In time $t$ a particle at point $x$ of the flow will move to some new point $f_{t}(x)$ of space. The mapping $x \mapsto f_{t} (x)$ that arises thereby on the points of space occupied by the flow depends on time and is called the mapping after time $t$. Show that $f_{t_{2}} \circ f_{t_{1}} = f_{t_{1}} \circ f_{t_{2}} = f_{t_{1} +t_{2}}$ and $f_{t} \circ f_{-t} = e_{X}$.
\end{exercise}

\begin{proof}
    $f_{t_{1} + t_{2}}(x)$ is the coordinate of the particle after time $t_{1} + t_{2}$.

    $f_{t_{1}}(x)$ is the coordinates of the particle after time $t_{1}$, $f_{t_{2}}(x)$ is the coordinates of the particle after time $t_{2}$. So $f_{2}(f_{1}(x))$ and $f_{1}(f_{2}(x))$ are the coordinates of the particle after time $t_{1} + t_{2}$.

    Therefore $f_{t_{2}}\circ f_{t_{1}} = f_{t_{1}}\circ f_{t_{2}} = f_{t_{1} + t_{2}}$. So $f_{t}\circ f_{-t} = f_{t + (-t)} = f_{0} = e_{X}$.
\end{proof}
\newpage

\section{Supplementary Material}

% chapter1:section4:exercise1
\begin{exercise}
    \begin{enumerate}[label={(\alph*)}]
        \item Prove the equipotence of the closed interval $\{ x\in \mathbb{R} \mid 0\leq x\leq 1 \}$ and the open interval $\{ x\in \mathbb{R} \mid 0 < x < 1 \}$ of the real line $\mathbb{R}$ both using the Schröder-Bernstein theorem and by direct exhibition of a suitable bijection.
        \item Prove the Schröder-Bernstein theorem.
    \end{enumerate}
\end{exercise}

\begin{proof}
    \begin{enumerate}[label={(\alph*)}]
        \item Let $a$ and $b$ be two real numbers such that $0 < a < b < 1$. The mapping $f: \{ x\in \mathbb{R} \mid a\leq x\leq b \} \to \{ x\in \mathbb{R} \mid 0\leq x\leq 1 \}$ where
              \[
                  f(x) = \frac{x - a}{b - a}
              \]

              is bijective. So $\operatorname{card} \{ x\in \mathbb{R} \mid a\leq x\leq b \} = \operatorname{card}\{ x\in \mathbb{R} \mid 0\leq x\leq 1 \}$.

              Moreover, $\operatorname{card} \{ x\in \mathbb{R} \mid a\leq x\leq b \} \leq \operatorname{card}\{ x\in \mathbb{R} \mid 0 < x < 1 \}\leq \operatorname{card}\{ x\in \mathbb{R} \mid 0 \leq x \leq 1 \}$.

              Hence by Schröder-Bernstein theorem, the closed interval $\{ x\in \mathbb{R} \mid 0\leq x\leq 1 \}$ and the open interval $\{ x\in \mathbb{R} \mid 0 < x < 1 \}$ are equipotence.
        \item It suffices to prove that if the sets $X$, $Y$, and $Z$ are such that $X \supset Y \supset Z$ and $\operatorname{card} X = \operatorname{card} Z$, then $\operatorname{card} X = \operatorname{card} Y$.

              Let $f : X \to Z$ be a bijection. Let's define  $g: X\to Y$ as follows:
              \[
                  g(x) = \begin{cases}
                      f(x), & \text{if $x\in f^{n}(X)\setminus f^{n}(Y)$ for some $n\in\mathbb{N}$}, \\
                      x     & \text{otherwise.}
                  \end{cases}
              \]

              Let
              \[
                  A = \bigcup_{n\in\mathbb{N}_{0}} f^{n}(X)\setminus f^{n}(Y) \qquad B = C_{X}A.
              \]

              \begin{enumerate}[label={(\roman*)}]
                  \item $x\in A\Leftrightarrow f(x)\in A$.
                  \item $g(X)\subset Y$.

                        If $x\in A$, then $g(x) = f(x)\in Z\subset Y$.

                        If $x\in B$, then $\forall n\in\mathbb{N}_{0}, x\notin f^{n}(X) \lor x\in f^{n}(Y)$. Choose $n = 0$, we obtain $x\notin X$ or $x\in Y$. $x\notin X$ is false, so $x\in Y$, and $g(x) = x\in Y$.
                  \item $g$ is surjective.

                        Let $y\in Y$. If $y\in B$, then $g(y) = y$.

                        If $y\in A$, then $y\in f^{n}(X)\setminus f^{n}(Y)$ for some $n\in\mathbb{N}_{0}$. $n\geq 1$ (proved by contradiction). Hence $y\in f^{n}(X)\subset Z$. So there exists $x\in X$ such that $f(x) = y$. By (i), $x\in A$. So $g(x) = f(x) = y$.

                        Hence $g$ is surjective.
                  \item $g$ is injective.

                        Assume that $g(x_{1}) = g(x_{2}) = y$. If $x_{1}$ and $x_{2}$ are both in $A$ or $B$ then $x_{1} = x_{2}$.

                        Assume $x_{1}\in A$ and $x_{2}\in B$, then $f(x_{1}) = g(x_{1}) = g(x_{2}) = x_{2}\in B$. This contradicts (i).

                        Hence $g$ is injective.
              \end{enumerate}

              Thus $g$ is bijective.
    \end{enumerate}
\end{proof}
\newpage

% chapter1:section4:exercise2
\begin{exercise}
    \begin{enumerate}[label={(\alph*)}]
        \item Starting from the definition of a pair, verify that the definition of the direct product $X \times Y$ of sets $X$ and $Y$ given in Sect. 1.4.2 is unambiguous, that is, the set $\mathcal{P}(\mathcal{P}(X)\cup \mathcal{P}(Y))$ contains all ordered pairs $(x, y)$ in which $x \in X$ and $y \in Y$.
        \item Show that the mappings $f: X \to Y$ from one given set $X$ into another given set $Y$ themselves form a set $M(X, Y)$.
        \item Verify that if $\mathcal{R}$ is a set of ordered pairs (that is, a relation), then the first elements of the pairs belonging to $\mathcal{R}$ (like the second elements) form a set.
    \end{enumerate}
\end{exercise}

\begin{proof}
    \begin{enumerate}[label={(\alph*)}]
        \item For all $x\in X$ and $y\in Y$, $\set{x}\in \mathcal{P}(X)$ and $\set{y}\in \mathcal{P}(Y)$. So
              \[
                  \set{\set{\set{x}}, \set{\set{x}, \set{y}}}\in \powerset{\powerset{X}\cup\powerset{Y}}.
              \]

              Hence $(x, y)\in \powerset{\powerset{X}\cup\powerset{Y}}$ for all $x\in X$ and $y\in Y$.
        \item This follows from the axiom of replacement.
        \item So $\mathcal{R}\subset X\times Y$ for some sets $X, Y$.

              The first elements of the pairs belonging to $\mathcal{R}$ are determined by
              \[
                  \set{x\in X \mid \exists y\in Y \left( (x, y)\in \mathcal{R} \right) },
              \]

              and the second elements of the pairs belonging to $\mathcal{R}$ are determined by
              \[
                  \set{y\in Y \mid \exists x\in X\left( (x, y)\in \mathbb{R} \right)}.
              \]

              By the axiom of separation, the first elements of the pairs belonging to $\mathcal{R}$ form a set, and the second elements of the pairs belonging to $\mathcal{R}$ form a set.
    \end{enumerate}
\end{proof}
\newpage

% chapter1:section4:exercise3
\begin{exercise}
    \begin{enumerate}[label={(\alph*)}]
        \item Using the axioms of extensionality, pairing, separation, union, and infinity, verify that the following statements hold for the elements of the set $\mathbb{N}_{0}$ of natural numbers in the sense of von Neumann:
              \begin{enumerate}[label={${\arabic*}^{0}$}]
                  \item $x = y\implies x^{+} = y^{+}$;
                  \item $(\forall x\in\mathbb{N}_{0})(x^{+}\ne \varnothing)$;
                  \item $x^{+} = y^{+} \implies x = y$;
                  \item $(\forall x\in\mathbb{N}_{0})(x \ne \varnothing \implies (\exists y\in \mathbb{N}_{0})(x = y^{+}))$.
              \end{enumerate}
        \item Using the fact that $\mathbb{N}_{0}$ is an inductive set, show that the following statements hold for any of its elements $x$ and $y$ (which in turn are themselves sets):
              \begin{enumerate}[label={${\arabic*}^{0}$}]
                  \item $\operatorname{card} x\leq \operatorname{card} x^{+}$;
                  \item $\operatorname{card} \varnothing < \operatorname{card} x^{+}$;
                  \item $\operatorname{card} x < \operatorname{card} y \Leftrightarrow \operatorname{card} x^{+} < \operatorname{card} y^{+}$;
                  \item $\operatorname{card} x < \operatorname{card} x^{+}$;
                  \item $\operatorname{card} x < \operatorname{card} y \implies \operatorname{card} x^{+} \leq \operatorname{card} y$;
                  \item $x = y \Leftrightarrow \operatorname{card} x = \operatorname{card} y$;
                  \item $(x\subset y)\lor (x\supset y)$.
              \end{enumerate}
        \item Show that in any subset $X$ of $\mathbb{N}_{0}$ there exists a (minimal) element $x_{m}$ such that ($\forall x\in X$) ($\operatorname{card} x_{m}\leq \operatorname{card} x$).
    \end{enumerate}
\end{exercise}

\begin{proof}
    Solved in Chapter 2.
\end{proof}
\newpage

% chapter1:section4:exercise4
\begin{exercise}
    We shall deal only with sets. Since a set consisting of different elements may
    itself be an element of another set, logicians usually denote all sets by uppercase
    letters. In the present exercise, it is very convenient to do so.
    \begin{enumerate}[label={(\alph*)}]
        \item Verify that the statement
              \[
                  \forall x \exists y \forall z \left( z\in y \Leftrightarrow \exists w (z\in w \land w\in x) \right)
              \]

              expresses the axiom of union, according to which $y$ is the union of the sets belonging to $x$.
        \item State which axioms of set theory are represented by the following statements:
              \begin{align*}
                   & \forall x\forall y\forall z\left((z\in x\Leftrightarrow z\in y) \Leftrightarrow x = y\right),                                                                   \\
                   & \forall x\forall y\exists z\forall v\left( v\in z \Leftrightarrow (v = x\lor v = y) \right),                                                                    \\
                   & \forall x\exists y\forall z\left(z\in y\Leftrightarrow \forall u (u\in z\implies u\in x)\right),                                                                \\
                   & \exists x\left(\forall y \left(\neg \exists z(z\in y)\implies y\in x\right)\right) \land                                                                        \\
                   & \phantom{   } \land \forall w\left(w\in x\implies \forall u\left( \forall v\left(v\in u\Leftrightarrow (v = w\lor v\in w)\right)\implies u\in x \right)\right).
              \end{align*}
        \item Verify that the formula
              \begin{gather*}
                  \forall z\, \left( z\in f \implies \left( \exists x_{1}\,\exists y_{1} \left( x_{1}\in x\land y_{1}\in y\land z = (x_{1}, y_{1}) \right) \right) \right) \land \\
                  \land\, \forall x_{1}\left( x_{1}\in x \implies \exists y_{1}\,\exists z \left( y_{1}\in y\land z = (x_{1}, y_{1})\land z\in f \right) \right) \land \\
                  \land\, \forall x_{1}\,\forall y_{1}\,\forall y_{2}\, \left( \exists z_{1}\,\exists z_{2}\,\left( z_{1}\in f\land z_{2}\in f \land z_{1} = (x_{1}, y_{1}) \land z_{2} = (x_{1}, y_{2}) \right) \implies y_{1} = y_{2} \right)
              \end{gather*}

              imposes three successive restrictions on the set $f$: $f$ is a subset of $x \times y$; the projection of $f$ on $x$ is equal to $x$; to each element $x_{1}$ of $x$ there corresponds exactly one $y_{1}$ in $y$ such that $(x_{1} , y_{1}) \in f$.

              Thus what we have here is a definition of a mapping $f: x \to y$.
    \end{enumerate}
\end{exercise}

\begin{proof}
    Unsolved.
\end{proof}
\newpage

% chapter1:section4:exercise5
\begin{exercise}
    Let $f: X\to Y$ be a mapping. Write the logical negation of each of the following statements:
    \begin{enumerate}[label={(\alph*)}]
        \item $f$ is surjective;
        \item $f$ is injective;
        \item $f$ is bijective.
    \end{enumerate}
\end{exercise}

\begin{proof}
    \begin{enumerate}[label={(\alph*)}]
        \item $f$ is not surjective
              \[
                  \exists y\left((y\in Y)\land \neg \left( \exists x\left( (x\in X)\land f(x) = y \right) \right)\right).
              \]
        \item $f$ is not injective
              \[
                  \exists x_{1}\in X\, \exists x_{2}\in X\, \left( (x_{1}\ne x_{2})\land (f(x_{1}) = f(x_{2})) \right).
              \]
        \item $f$ is not bijective means $f$ is not surjective or not injective.
              \begin{gather*}
                  \neg\left(\exists y\left((y\in Y)\land \neg \left( \exists x\left( (x\in X)\land f(x) = y \right) \right)\right)\right)\, \lor \\
                  \lor\, \neg\left(\exists x_{1}\in X\, \exists x_{2}\in X\, \left( (x_{1}\ne x_{2})\land (f(x_{1}) = f(x_{2})) \right)\right).
              \end{gather*}
    \end{enumerate}
\end{proof}
\newpage

% chapter1:section4:exercise6
\begin{exercise}
    Let $X$ and $Y$ be sets and $f\subset X\times Y$. Write what it means to say that the set $f$ is not a function.
\end{exercise}

\begin{proof}
    \begin{multline*}
        \exists x\in X\left( \forall y\in Y \left((x, y)\notin f\right) \right) \lor \\
        \lor\,\exists x\in X\left(\exists y_{1}\in Y\,\exists y_{2}\in Y\,\left((y_{1}\ne y_{2})\land ((x, y_{1})\in f)\land ((x, y_{2})\in f) \right)\right).
    \end{multline*}
\end{proof}
\newpage
