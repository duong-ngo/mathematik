\chapter{The Real and Complex Number Systems}

\section{Axioms of real numbers}

\par I am familar with real numbers. However, I have taken the properties of real numbers for granted. I can't even answer these questions: ``What are real numbers? Are they real?\@'' Turns out, these questions are really difficult.

\par If all you have ever wanted is a definition of real numbers, then you could use the following axiomatic definition.

\par Real numbers are elements of a set $\mathbb{R}$ endowed with addition and multiplication, which satisfy the following properties
\begin{enumerate}[label = (\roman*)]
    \item $\mathbb{R}$ is totally ordered.
    \item $\mathbb{R}$ is a field under addition and multiplication.
    \item Order in $\mathbb{R}$ is preserved under addition and multiplication (with non-negative real number).
    \item Every upper-bounded non-empty set of $\mathbb{R}$ has a least upper bound.
\end{enumerate}

\par To those who ask ``Are real numbers real?\@'', we can establish a model (a mathematical structure) that satisfies every axiom above. So in this sense, i.e.\@ since the existence of such model, I would answer ``yes''. Since 19th century, mathematicians have given several constructions of the real numbers. IMHO, the two most notable constructions are \textit{Dedekind cuts} and \textit{Cauchy sequences}. In the following sections, we will reproduce the constructions by \textit{Dedekind cuts} and \textit{Cauchy sequences}.

\section{* Construction of the real numbers by Dedekind cuts}

\par We will give the definition of Dedekind cuts fromwhich we construct a model that satisfies the real numbers axioms.

\par We are gonna use the following notations
\begin{table}[htp]
    \centering
    \begin{tabular}{ll}
        Notations        & Meanings                             \\
        \bottomrule
        \toprule
        $\mathbb{Q}$     & Set of all rational numbers          \\
        $\mathbb{Q}_{-}$ & Set of all negative rational numbers \\
        $\mathbb{Q}_{+}$ & Set of all positive rational numbers
    \end{tabular}
\end{table}

Upcoming proofs in this section will use various properties of rational numbers, including:
\begin{itemize}
    \item Total order.
    \item The field structure.
    \item Compatibility of addition and multiplication with the order relation.
    \item There exists a rational number strictly between any two distinct rational numbers.
          \[
              \forall q_{1}\forall q{2}\left( q_{1}\in\mathbb{Q}\land q_{2}\in\mathbb{Q}\and q_{1} < q_{2} \rightarrow \exists q (q\in\mathbb{Q}\land q_{1} < q\land q < q_{2}) \right).
          \]
\end{itemize}

\subsection{Definition of Dedekind cuts}

To define Dedekind cuts, we will use rational numbers as the basis in the sense that \textit{the set of rational numbers satisfies all real numbers axioms, except for the least upper bound axiom}.

\begin{definition}[Dedekind cuts]
    A Dedekind cut $A$ is a subset of $\mathbb{Q}$ that:
    \begin{enumerate}[label = (DC\arabic*)]
        \item $A\ne\varnothing$; in other words, $A$ is not empty.
        \item $A\neq\mathbb{Q}$; in other words, $A$ is not the entire set of rational numbers.
        \item $\forall x\left(x\in A \rightarrow \exists y \left( y\in A \wedge x < y \right)\right)$; in other words, $A$ has no greatest element.
        \item $\forall x\in A\left(\forall y( y < x \rightarrow y\in A)\right)$; in other words, $A$ is downward closed.
    \end{enumerate}
\end{definition}

\par Due to (DC4), $\mathbb{Q}\setminus A$ contains every rational upper bound of $A$.

\par Our goal is from the definition of Dedekind cuts as well as operations (addition and multiplication) and relations (less than or equal) on them, we prove that Dedekind cuts satisfy the real numbers axioms.

\par Sequentially, we will define the following based on Dedekind cuts:
\begin{itemize}
    \item Rational and irrational cut,
    \item Addition,
    \item Subtraction and negation,
    \item Multiplication,
    \item Division,
\end{itemize}
\par then we will show that the set of all Dedekind cuts equipped with those operation will satisfy the real numbers axioms.

\par Denote the set of all Dedekind cuts by $\mathbb{R}$.

\begin{theorem}[Totally ordered]
    $\mathbb{R}$ is totally ordered with $\subseteq$ relation.
\end{theorem}

\begin{proof}
    \par Let $A$ and $B$ be two Dedekind cuts.
    \par The subset relation $\subseteq$ is reflexive, transitive, and anti-symmetric.
    \bigskip
    \par Suppose that $A\ne B$.
    \par Without loss of generality, let's suppose that there exists $b\in B$ such that $b\notin A$.
    \par $b\in B$, then $b$ is an upper bound of $A$.
    \par Let $a$ be an arbitrary element of $A$, then $a\le b$. According to (DC4), $a\in B$. Hence $\forall a(a\in A\rightarrow a\in B)$.
    \par Therefore, $A$ is a proper subset of $B$.
    \bigskip
    \par So for arbitrary two Dedekind cuts $A$, $B$, one of the following holds: $A\subseteq B, B\subseteq A$. Hence, the set of all Dedekind cuts is totally ordered with $\subseteq$ relation.
\end{proof}

We define the relation $\le$ on $\mathbb{R}$ as follow:
\[
    A\le B \Longleftrightarrow A\subseteq B.
\]

For convenience, in this section, we use the following notation:
\[
    {0}^{*} = \{ x : x\in\mathbb{Q} \wedge x < 0 \} = {\mathbb{Q}}_{-}.
\]

\begin{definition}
    A Dedekind cut $A$ is called:
    \begin{enumerate}[label = (\roman*)]
        \item positive if $A$ is a proper superset of ${0}^{*}$,
        \item negative if $A$ is a proper subset of ${0}^{*}$,
        \item non-positive if $A\subseteq {0}^{*}$,
        \item non-negative if $A\supseteq {0}^{*}$.
    \end{enumerate}
\end{definition}

\begin{definition}[Rational and irrational]
    A Dedekind cut $ A$ is called:
    \begin{enumerate}[label = (\roman*)]
        \item rational if $\mathbb{Q}\setminus A$ has least element,
        \item irrational if $\mathbb{Q}\setminus A$ has no least element.
    \end{enumerate}
\end{definition}

\par The following example gives us an example of rational cut, and an example of irrational cut.

\begin{example}
    \[
        A = \{ x\in\mathbb{Q}: x < 1 \}
    \]
    \par is a rational cut.
    \[
        B = \{ x\in\mathbb{Q}: {x}^{2} < 2 \} \cup \mathbb{Q}^{-}
    \]
    \par is an irrational cut.
\end{example}

\begin{proof}
    \par $\mathbb{Q}\setminus A = \{ x\in\mathbb{Q}: x\ge 1 \}$ has least element, which is $1$. So $ A$ is a rational cut.
    \bigskip
    \par $\mathbb{Q}\setminus B = \{ x\in\mathbb{Q}: {x}^{2}\ge 2 \wedge x > 0 \}$.
    \par Since there is no rational number $r$ of which square equals $2$, then $\mathbb{Q}\setminus B = \{ x\in\mathbb{Q}: {x}^{2} > 2 \wedge x > 0 \}$ (change from $\ge$ to $>$).
    \par Let $q\in\mathbb{Q}\setminus B$. Consider $r = \dfrac{q}{2} + \dfrac{1}{q}$.
    \begin{align*}
        r = \frac{q}{2} + \frac{1}{q} & = -\frac{q}{2} + \frac{1}{q} + q                       \\
                                      & = \frac{2 - {q}^{2}}{2q} + q                           \\
                                      & < q \quad\text{(Since $q > 0$ and $2 - {q}^{2} < 0$)}.
    \end{align*}
    \begin{align*}
        {r}^{2} & = {\left(\frac{q}{2} + \frac{1}{q}\right)}^{2} = \frac{q^{2}}{4} + \frac{1}{q^{2}} + 1                                                         \\
                & = \frac{q^{2}}{4} + \frac{1}{q^{2}} - 1 + 2 = {\left(\frac{q}{2} - \frac{1}{q}\right)}^{2} + 2 = {\left( \frac{q^{2} - 2}{2q} \right)}^{2} + 2 \\
                & > 2
    \end{align*}
    \par Therefore, $\forall q(q\in\mathbb{Q}\setminus B \rightarrow \exists r( r\in\mathbb{Q}\setminus B \wedge r < q ))$. Hence $\mathbb{Q}\setminus B$ has no minimal element. According to the definition, $ B$ is an irrational cut.
\end{proof}

\subsection{Addition}

\begin{theorem}[Addition]
    \par $A, B$ are Dedekind cuts.
    \[
        A + B = \{ x + y : x\in A \wedge y\in B \}
    \]
    \par is also a Dedekind cut.
\end{theorem}

\begin{proof}
    \begin{enumerate}[label = (\roman*)]
        \item Since $A\ne\varnothing$ and $B\ne\varnothing$, then there exists $a\in A$ and $b\in B$. By definition of $A + B$, we obtain that $a + b \in A + B$. This implies that $A + B$ is not empty.
        \item A Dedekind cut is downward closed and not the entire set of rational numbers, then it is bounded above.
              \par Therefore, $A$ and $B$ are bounded above. Let $a$ be an upper bound of $A$, $b$ be an upper bound of $B$.
              \par $\forall x\in A\forall y\in B$, then $x + y \le a + b$, which means $A + B$ is bounded above.
              \par Hence $A + B\ne\mathbb{Q}$.
        \item Let $c$ be an element of $A + B$. According to the definition of $A + B$, there exists $a\in A$ and $b\in B$ such that $a + b = c$.
              \par According to (DC3), there exists $a_{0}\in A$ such that $a < a_{0}$, and there exists $b_{0}\in B$ such that $b < b_{0}$.
              \par $c = a + b < a_{0} + b_{0}$. According to the definition of $A + B$, $a_{0} + b_{0} \in A + B$. Hence $A + B$ has no greatest element.
        \item Let $c$ be an element of $A + B$. According to the definition of $A + B$, there exists $a\in A$ and $b\in B$ such that $a + b = c$.
              \par Let $c_{1}$ be a rational number such that $c_{1} < c$.
              \par According to (DC4) $b + (c_{1} - c)\in B$.
              \par Hence
              \[
                  \underbrace{a}_{\in A} + \underbrace{b + (c_{1} - c)}_{\in B} \in A + B.
              \]
              \par Therefore
              \[
                  c_{1} \in A + B.
              \]
              \par Hence $A + B$ is downward closed.
    \end{enumerate}
    \par In conclusion, $A + B$ is a Dedekind cut.
\end{proof}

\par I have difficulty defining multiplication since there are positive numbers and negative numbers. So I define additive inverse/negation of a cut.

\begin{theorem}[Subtraction]
    \par Let $A$ and $B$ be Dedekind cuts
    \[
        A - B = \{ a - b: a\in A\land b\in\mathbb{Q}\setminus B \}
    \]
    \par is also a Dedekind cut.
\end{theorem}

\begin{proof}
    \begin{enumerate}[label = (\roman*)]
        \item Since $A\ne\varnothing$, there exists $a\in A$. Since $B\ne\mathbb{Q}$, there exists $b\in\mathbb{Q}\setminus B$. So $a - b\in A - B$, which implies that $A - B$ is not empty.
        \item Let $a_{0}\in\mathbb{Q}\setminus A, b_{0}\in B, a\in A, b\in\mathbb{Q}\setminus B$. Then $a\le a_{0}$ and $b_{0} < b$.
              \[
                  \underbrace{a - b}_{\in A - B} < a_{0} - b_{0},
              \]
              Therefore, $A - B$ is bounded above, so $A - B\ne\mathbb{Q}$.
        \item Let $c$ be an arbitrary element of $A - B$. According to the definition of $A - B$, there exists $a\in A$ and $b\in\mathbb{Q}\setminus B$ such that $c = a - b$.
              \par Due to (DC3), there exists $d\in A$ such that $a < d$.
              \par Then $c = a - b < d - b \in A - B$.
              \par So $A - B$ does not have greatest element.
        \item Let $c$ be an arbitrary element of $A - B$. According to the definition of $A - B$, there exists $a\in A$ and $b\in\mathbb{Q}\setminus B$ such that $c = a - b$.
              \par Let $d$ be a rational number such that $d < c$.
              \[
                  d = d - c + c = (d - c) + a - b = \underbrace{(d - c + a)}_{< a} - \underbrace{b}_{\in B}
              \]
              \par $(d - c + a) < a$, therefore $(d - c + a)\in A$, according to (DC4). Hence $d\in A - B$.
              \par So $A - B$ is downward closed.
    \end{enumerate}
\end{proof}

\par The following theorem is a special case of subtraction.

\begin{theorem}[Additive inverse/Negation]
    Let $A$ be a Dedekind cut.
    \[
        -A = \{ b - a : b\in{0}^{*} \wedge a\in\mathbb{Q}\setminus A \}
    \]
    is also a Dedekind cut.
\end{theorem}

\begin{theorem}\label{theorem:chapter1:negation-and-subtraction}
    Let $A$ and $B$ be Dedekind cuts, then
    \[
        A - B = A + (-B)
    \]
\end{theorem}

\begin{proof}
    \par According to the definitions of addition, subtraction, and negation
    \begin{align*}
        A - B    & = \{ a - b : a\in A\land b\in\mathbb{Q}\setminus B \},                      \\
        A + (-B) & = \{ a + w - b : a\in A\land w\in{0}^{*}\land b\in\mathbb{Q}\setminus B \}.
    \end{align*}
    \par\textbf{Step 1.} Prove that $A - B\subseteq A + (-B)$.
    \par Let $c = a - b$ be an arbitrary element in $A - B$, where $a\in A$ and $b\in\mathbb{Q}\setminus B$.
    \par According to (DC3), there exists $a_{0}\in A$ such that $a < a_{0}$.
    \[
        c = a - b = \underbrace{a_{0}}_{\in A} + \underbrace{(a - a_{0})}_{< 0} - \underbrace{b}_{\in\mathbb{Q}\setminus B}
    \]
    \par so $c\in A + (-B)$. Hence $A - B\subseteq A + (-B)$.
    \bigskip
    \par\textbf{Step 2.} Prove that $A + (-B)\subseteq A - B$.
    \par Let $c = a + (w - b)$ be an arbitrary element of $A + (-B)$, where $a\in A$, $w < 0$, and $b\in\mathbb{Q}\setminus B$.
    \par $a + w < a$. According to (DC4), $a + w\in A$.
    \[
        c = a + (w - b) = \underbrace{(a + w)}_{\in A} - \underbrace{b}_{\in\mathbb{Q}\setminus B}
    \]
    \par so $c\in A - B$. Hence $A + (-B)\subseteq A - B$.
    \bigskip
    \par In conclusion, $A - B = A + (-B)$.
\end{proof}

I need the following property to verify the group structure of $\mathbb{R}$.

\begin{lemma}[Archimedean property for rational numbers]
    \par Let $a, b$ be rational numbers where $a > 0$. Then there exists an integer $n$ such that
    \[
        (n - 1)\le \frac{b}{a} < n.
    \]
\end{lemma}

\begin{proof}
    \par Since $a, b$ are rational numbers, then $\dfrac{b}{a}$ is also rational number.
    \par Then there exists a positive integer $q$ and an integer $p$ such that $\dfrac{b}{a} = \dfrac{p}{q}$.
    \par Apply Euclid division algorithm, there exists $0\le r < q$ and $k\in\mathbb{Z}$ such that $p = k\cdot q + r$.
    \begin{align*}
                         & \frac{p}{q} = \frac{k\cdot q + r}{q} = k + \frac{r}{q} \\
        \Rightarrow\quad & k \le \frac{p}{q} < k + 1.\qedhere
    \end{align*}
\end{proof}

\begin{theorem}\label{theorem:chapter1:real-field-part-one}
    \par $\mathbb{R}$ with addition is a commutative group.
    \begin{enumerate}[label={(F\arabic*)}]
        \item Addition is associative.
        \item Addition has identity element.
        \item Each element has an additive inverse.
        \item Addition is commutative.
    \end{enumerate}
\end{theorem}

\begin{proof}
    Let $A, B, C$ be arbitrary Dedekind cuts.
    \begin{enumerate}[label = (F\arabic*)]
        \item Addition is associative.
              \begin{align*}
                  (A + B) + C & = \{ (a + b) + c : a\in A\land b\in B\land c\in C \}                                                       \\
                              & = \{ a + (b + c) : a\in A\land b\in B\land c\in C \} \quad\text{(Addition in $\mathbb{Q}$ is associative)} \\
                              & = A + (B + C).
              \end{align*}
        \item Addition has identity element.
              \begin{align*}
                  A + {0}^{*} & = \{ a + w : a\in A\land w < 0 \} \\
                              & = \{ w + a : a\in A\land w < 0 \} \\
                              & = {0}^{*} + A.
              \end{align*}
              \par \textbf{Step 1. Prove that $A \subseteq A + {0}^{*}$}.
              \par Let $x\in A$. According to (DC4), there exists $y\in A$ such that $y > x$.
              \[
                  x = \underbrace{y}_{\in A} + \underbrace{(x - y)}_{< 0, \in {0}^{*}}
              \]
              \par So $\forall x(x\in A \rightarrow x\in A + {0}^{*})$, which means $A \subseteq A + {0}^{*}$.
              \bigskip
              \par \textbf{Step 2. Prove that $A + {0}^{*} \subseteq A$}.
              \par Let $a_{0}\in A + {0^{*}}$. According to the definition of $A + {0}^{*}$, there exists $a\in A$ and $w\in\mathbb{Q}^{-}$ such that $a_{0} = a + w$.
              \par Since $w < 0$ then $a_{0} < a$. According to (DC4), $a_{0}\in A$.
              \par So $\forall a_{0}(a_{0}\in A + {0}^{*} \rightarrow A)$.
              \par Hence $A = A + {0}^{*} = {0}^{*} + A$.
        \item Every element has an additive inverse.
              \begin{align*}
                  A + (-A) & = \{ a + (w - a') : a\in A\land w\in\mathbb{Q}^{-}\land a'\in\mathbb{Q}\setminus A \} \\
                           & = \{ (w - a') + a : a\in A\land w\in\mathbb{Q}^{-}\land a'\in\mathbb{Q}\setminus A \} \\
                           & = (-A) + A.
              \end{align*}
              \par \textbf{Step 1. Prove that $A + (-A)\subseteq {0}^{*}$}.
              \par Let $a\in A, a'\in \mathbb{Q}\setminus A, w\in {0}^{*}$.
              \par Since $a'\notin A$ then $a' > a$, so $a - a' < 0$. Therefore
              \[
                  a + (w - a') = w + (a - a') < w < 0
              \]
              \par According to (DC4), $a + (w - a')\in {0}^{*}$. So $A + (-A) \subseteq {0}^{*}$.
              \bigskip
              \par \textbf{Step 2. Prove that ${0}^{*}\subseteq A + (-A)$}.
              \par Let $w\in {0}^{*}$.
              \par We will prove that the set $S = \{ n : n\in\mathbb{Z} \land n\cdot w\in A \}$ has least element. In equivalent, $S$ is the set of integers such that $n\cdot w\in A$.
              \par Let $x\in A, y\in\mathbb{Q}\setminus A$. $w < 0$, then according to the Archimedean property, there exists an integer $m$ such that
              \begin{align*}
                  m - 1    & \le \frac{x}{-w} < m \\
                  (1 - m)w & \le x < -m\cdot w
              \end{align*}
              \par According to (DC4), $(1 - m)w\in A$. Then $S$ is not empty.
              \par Since the set $\{ n\cdot w : n\in\mathbb{Z}\land n\cdot w\in A \}$ is bounded above, and $w < 0$, then $S$ is bounded below. On the other hand, $S$ is a set of integers, then $S$ has least element.
              \par
              \par Let $n$ be the least integer such that $n\cdot w\in A$. Then $(n - 1)\cdot w\notin A$. Hence $(n - 1)\cdot w\in\mathbb{Q}\setminus A$.
              \[
                  w = \underbrace{n\cdot w}_{\in A} - \underbrace{(n - 1)\cdot w}_{\in\mathbb{Q}\setminus A}
              \]
              \par so $w\in A - A$. According to Theorem~\ref{theorem:chapter1:negation-and-subtraction}, $A - A = A + (-A)$. Then $w\in A + (-A)$. Therefore, ${0}^{*}\subseteq A + (-A)$.
              \bigskip
              \par Hence $A + (-A) = (-A) + A = {0}^{*}$.
        \item Addition is commutative.
              \begin{align*}
                  A + B & = \{ a + b : a\in A\land b\in B \}                                                        \\
                        & = \{ b + a : b\in B\land a\in A \}\qquad\text{(In $\mathbb{Q}$, addition is commutative)} \\
                        & = B + A.
              \end{align*}
    \end{enumerate}
\end{proof}

\begin{theorem}\label{theorem:chapter1:negation-is-an-involution}
    \par Let $A$ be a Dedekind cut, then
    \[
        A = -(-A).
    \]
\end{theorem}

\begin{proof}
    Firstly, we prove that there exist a unique Dedekind cut $A'$ such that $A + A' = A' + A = {0}^{*}$.

    According to Theorem~\ref{theorem:chapter1:real-field-part-one}, $A + (-A) = (-A) + A = {0}^{*}$.

    Suppose that $A + A' = A' + A = {0}^{*}$.
    \begin{align*}
        A' & = A' + {0}^{*} = A' + (A + (-A))   \\
           & = (A' + A) + (-A) = {0}^{*} + (-A) \\
           & = -A.
    \end{align*}
    Hence there exists unique Dedekind cut $A'$ such that $A + A' = A' + A = {0}^{*}$.
    \begin{align*}
         & A + (-A) = (-A) + A = {0}^{*}             \\
         & (-(-A)) + (-A) = (-A) + (-(-A)) = {0}^{*}
    \end{align*}
    Thus, $A = -(-A)$.
\end{proof}

\begin{theorem}
    In $\mathbb{R}$, addition is compatible with $\subseteq$
    \[
        \forall A, B, C\in\mathbb{R}(A\subseteq B \rightarrow A + C\subseteq B + C).
    \]
\end{theorem}

\begin{proof}
    Let $A, B, C$ be Dedekind cuts such that $A\subseteq B$

    If $A = B$ then $A + C = B + C$.

    Otherwise, $A\subset B$, then there exists $b\in B$ such that $b\notin A$. Hence, for all $c\in C$, $b + c\notin A + C$, then $A + C\ne B + C$.

    Let $a\in A, c\in C$, then $a + c\in A + C$. Since $A\subset B$, then $a\in B$, so $a + c\in B + C$. Therefore $A + C\subseteq B + C$.

    So $A + C\subset B + C$.

    Thus $A + C\subseteq B + C$.
\end{proof}

\begin{theorem}\label{theorem:chapter1:negation-and-sign}
    \par Let $A$ be a Dedekind cut, then
    \[
        A\subset {0}^{*} \Longleftrightarrow -A\supset {0}^{*}.
    \]
\end{theorem}

\begin{proof}
    $(\Rightarrow)$ $A\subset {0}^{*}\Longrightarrow -A\supset {0}^{*}$.

    If $-A\subseteq {0}^{*}$, then
    \[
        A + (-A) \subset {0}^{*} + {0}^{*} = {0}^{*}
    \]
    which conflicts with $A + (-A) = {0}^{*}$. So $-A\supset {0}^{*}$.

    $(\Leftarrow)$ $-A\supset {0}^{*}\Longrightarrow A\subset {0}^{*}$.

    If $A\supseteq {0}^{*}$, then
    \[
        A + (-A) \supset {0}^{*} + {0}^{*} = {0}^{*}
    \]
    which conflicts with $A + (-A) = {0}^{*}$. So $A\subset {0}^{*}$.
\end{proof}

\subsection{Multiplication}

\begin{theorem}[Multiplication]\label{theorem:chapter1:multiplication}
    \par Let $A, B$ be Dedekind cuts.
    \par $A\cdot B$ is defined as the following.
    \par If $A\supseteq{0}^{*}$ and $B\supseteq{0}^{*}$
    \[
        A\cdot B = \{ a\cdot b : a\in A\wedge a\ge 0 \wedge b\in B\wedge b\ge 0 \} \cup \mathbb{Q}^{-}.
    \]
    \par If $A\subseteq{0}^{*}$ and $B\subseteq{0}^{*}$
    \[
        A\cdot B = (-A)\cdot (-B).
    \]
    \par If $A\subseteq{0}^{*}$ and $B\supseteq{0}^{*}$
    \[
        A\cdot B = -\left((-A)\cdot B\right).
    \]
    \par If $A\supseteq{0}^{*}$ and $B\subseteq{0}^{*}$
    \[
        A\cdot B = -\left(A\cdot (-B)\right).
    \]
    \par $A\cdot B$ is also a Dedekind cut.
\end{theorem}

Thanks to the definition of negation, we only have to cover that first case: $A\supseteq{0}^{*}$ and $B\supseteq{0}^{*}$.

\begin{proof}
    There are four cases.

    \begin{enumerate}[label={\textbf{Case \arabic*.}},itemindent={0.5cm}]
        \item $A\supseteq {0}^{*}\land B\supseteq {0}^{*}$.

              Let's consider the following three sub-cases.
              \begin{enumerate}
                  \item $A = {0}^{*}$.

                        Since $A = {0}^{*}$, then $\nexists a\in A$ such that $a\ge 0$. Hence
                        \[
                            A\cdot B = \{ a\cdot b: a\in A\land a\ge 0\land b\in B\land b\ge 0 \} \cup\mathbb{Q}_{-} = \varnothing\cup\mathbb{Q}_{-} = \mathbb{Q}_{-} = {0}^{*}.
                        \]
                  \item $B = {0}^{*}$.

                        Since $B = {0}^{*}$, then $\nexists b\in B$ such that $b\ge 0$. Hence
                        \[
                            A\cdot B = \{ a\cdot b: a\in A\land a\ge 0\land b\in B\land b\ge 0 \} \cup\mathbb{Q}_{-} = \varnothing\cup\mathbb{Q}_{-} = \mathbb{Q}_{-} = {0}^{*}.
                        \]
                  \item $A\supset{0}^{*}$ and $B\supset{0}^{*}$.
                        \begin{enumerate}[label = (\roman*)]
                            \item Since $A\cdot B$ is a superset of $\mathbb{Q}^{-}$, then $A\cdot B$ is not empty.
                            \item Let $a_{0}$ be an upper bound of $A$, $b_{0}$ be an upper bound of $B$.
                                  \par Since $A\supset{0}^{*}$ and $B\supset{0}^{*}$, then $a_{0}\ge 0$ and $b_{0}\ge 0$.
                                  \par Then for any non-negative elements $a$ and $b$ of $A$ and $B$, $a\cdot b \le a_{0}\cdot b_{0}$.
                                  \par Hence $a_{0}\cdot b_{0}$ is an upper bound of $A\cdot B$, which implies that $A\cdot B\ne\mathbb{Q}$.
                            \item Let $c$ be an arbitrary element of $A\cdot B$.

                                  If $c$ is negative or zero, then there exists an element which is greater than $c$, since $A\supset {0}^{*}$ and $B\supset {0}^{*}$ (zero is not their greatest element).

                                  Otherwise, $c$ is positive, then there exists $a\in A$ and $a > 0$, $b\in B$ and $b > 0$ such that $a\cdot b = c$. Due to (DC3), there exists $a_{0} > a > 0$ and $a_{0}\in A$, $b_{0} > b > 0$ and $b_{0}\in B$.

                                  Furthermore, $a_{0}\cdot b_{0} > a\cdot b$ and $a_{0}\cdot b_{0}\in A\cdot B$ according to the definition of $A\cdot B$.

                                  So $A\cdot B$ has no greatest element.
                            \item Let $c$ be an arbitrary element of $A\cdot B$.

                                  Let $d$ be a rational number such that $d < c$.

                                  If $d$ is non-positive, then $d\in A\cdot B$, since $A\cdot B$ contains $0$ and is a superset of $\mathbb{Q}^{-}$.

                                  Otherwise, $d$ is positive, then $c$ is also positive. Since $c$ is positive, there exists $a\in A$ and $a > 0$, $b\in B$ and $b > 0$ such that $c = a\cdot b$.
                                  \[
                                      d = c - (c - d) = a\cdot b - (c - d) = a\cdot\left(b - \frac{c - d}{a}\right)
                                  \]
                                  Since $a\in A$ and $a > 0$, $b - \dfrac{c - d}{a}\in B$ (due to (DC4)) and $0 < b - \dfrac{c - d}{a} < b$, then $d \in A\cdot B$.

                                  Hence $A\cdot B$ is downward closed.
                        \end{enumerate}
              \end{enumerate}
        \item $A\subseteq {0}^{*}, B\subseteq {0}^{*}$.

              Then $-A\supseteq {0}^{*}$, $-B\supseteq {0}^{*}$. According to 1st case, $(-A)\cdot (-B)$ is a Dedekind cut.
        \item $A\subseteq {0}^{*}, B\supseteq {0}^{*}$.

              Then $-A\supseteq {0}^{*}$. According to 1st case, $(-A)\cdot B$ is a Dedekind cut. So $-((-A)\cdot B)$ is a Dedekind cut.
        \item $A\supseteq {0}^{*}, B\subseteq {0}^{*}$.

              Then $-B\supseteq {0}^{*}$. According to 1st case, $A\cdot (-B)$ is a Dedekind cut. So $-(A\cdot (-B))$ is a Dedekind cut.
    \end{enumerate}

    In conclusion, $A\cdot B$ is a Dedekind cut.
\end{proof}

\begin{theorem}\label{theorem:chapter1:multiplication-and-negation}
    Let $A, B$ be Dedekind cuts. Then
    \[
        \begin{cases}
            (-A)\cdot B = A\cdot (-B) = -(A\cdot B), \\
            (-A)\cdot (-B) = A\cdot B.
        \end{cases}
    \]
\end{theorem}

\begin{proof}
    If $A = {0}^{*}$ or $B = {0}^{*}$, then $-A = {0}^{*}, -B = {0}^{*}$. So that
    \[
        A\cdot B = A\cdot (-B) = -A\cdot B = (-A)\cdot B = (-A)\cdot (-B) = {0}^{*}.
    \]
    Otherwise, let's consider the following cases. I am going to use the definition of multiplication (there are four cases) and Theorem~\ref{theorem:chapter1:negation-is-an-involution} \textit{implicitly}.

    \noindent\textbf{Step 1. Prove that $A\cdot (-B) = (-A)\cdot B = -A\cdot B$.}

    \noindent\textbf{Case 1. $A\supset {0}^{*}, B\supset {0}^{*}$.}
    \begin{align*}
        A\cdot (-B) & = -A\cdot (-(-B)) = -A\cdot B, \\
        (-A)\cdot B & = -(-(-A))\cdot B = -A\cdot B, \\
    \end{align*}
    \textbf{Case 2. $A\supset {0}^{*}, B\subset {0}^{*}$.}
    \begin{align*}
        A\cdot (-B) & = (-A)\cdot (-(-B)) = (-A)\cdot B, \\
        A\cdot (-B) & = -A\cdot (-(-B)) = -A\cdot B,     \\
    \end{align*}
    \textbf{Case 3. $A\subset {0}^{*}, B\supset {0}^{*}$.}
    \begin{align*}
        A\cdot (-B) & = (-A)\cdot (-(-B)) = (-A)\cdot B, \\
        (-A)\cdot B & = -(-(-A))\cdot B = -A\cdot B.     \\
    \end{align*}
    \textbf{Case 4. $A\subset {0}^{*}, B\subset {0}^{*}$.}
    \begin{align*}
        A\cdot (-B) & = -(-A)\cdot (-B) = -A\cdot B, \\
        (-A)\cdot B & = -(-A)\cdot (-B) = -A\cdot B.
    \end{align*}

    \noindent\textbf{Step 2. Prove that $(-A)\cdot (-B) = A\cdot B$}

    \noindent Apply the result in Step 1, $(-A)\cdot (-B) = -A\cdot (-B) = -(-(A\cdot B)) = A\cdot B$.
\end{proof}

\begin{theorem}
    Multiplication in $\mathbb{R}$ is compatible with $\subseteq$.
    \[
        \forall A, B\in\mathbb{R}(A\supseteq {0}^{*}\land B\supseteq {0}^{*}\rightarrow A\cdot B\supseteq {0}^{*}).
    \]
\end{theorem}

\begin{proof}
    \textbf{Case 1. $A = {0}^{*}$ or $B = {0}^{*}$.}

    According to the proof of Theorem~\ref{theorem:chapter1:multiplication}, $A\cdot B = {0}^{*}$.

    \textbf{Case 2. $A\supset {0}^{*}$ and $B\supset {0}^{*}$.}

    Since $A\supset {0}^{*}$, then there exists $a\in A$ such that $a > 0$. $B\supset {0}^{*}$, then there exists $b\in B$ such that $b > 0$.

    According to the definition of multiplication, $a\cdot b\in A\cdot B$. On the other hand, $a\cdot b > 0$. Therefore, $A\cdot B\supset {0}^{*}$.

    Thus $A\cdot B\supseteq {0}^{*}$. Equality holds if and only if $A = {0}^{*}$ or $B\cdot {0}^{*}$.
\end{proof}

The following result follows Theorem~\ref{theorem:chapter1:multiplication} and Theorem~\ref{theorem:chapter1:negation-and-sign}.

\begin{corollary}
    Let $A, B$ be Dedekind cuts.
    \[
        \begin{split}
            A\supset {0}^{*}\land B\supset {0}^{*}\rightarrow A\cdot B\supset {0}^{*}, \\
            A\supset {0}^{*}\land B\subset {0}^{*}\rightarrow A\cdot B\subset {0}^{*}, \\
            A\subset {0}^{*}\land B\supset {0}^{*}\rightarrow A\cdot B\subset {0}^{*}, \\
            A\subset {0}^{*}\land B\subset {0}^{*}\rightarrow A\cdot B\supset {0}^{*}.
        \end{split}
    \]
\end{corollary}

Let ${1}^{*}$ be the following (rational) Dedekind cut
\[
    \{ x: x\in\mathbb{Q}\land x < 1 \}.
\]

\begin{theorem}
    $\mathbb{R}$ with addition and multiplication is a commutative ring.
\end{theorem}

\begin{proof}
    \par Let $A, B, C$ be arbitrary Dedekind cuts.
    \begin{enumerate}[label={(F\arabic*)}, start=5]
        \item Multiplication is associative.

              \textbf{Case 1.} $A\supseteq {0}^{*}, B\supseteq {0}^{*}, C\supseteq {0}^{*}$, then
              \begin{align*}
                  (A\cdot B)\cdot C & = \{ (a\cdot b)\cdot c : a\in A\land b\in B\land c\in C\land a\ge 0\land b\ge 0\land c\ge 0 \} \\
                                    & = \{ a\cdot (b\cdot c) : a\in A\land b\in B\land c\in C\land a\ge 0\land b\ge 0\land c\ge 0 \} \\
                                    & = A\cdot (B\cdot C).
              \end{align*}
              \textbf{Case 2.} $A\supseteq {0}^{*}, B\supseteq {0}^{*}, C\subseteq {0}^{*}$, then $-C\supseteq {0}^{*}$ and
              \begin{align*}
                  (A\cdot B)\cdot C & = -\left( (A\cdot B)\cdot (-C) \right)  \\
                                    & = -\left( A\cdot (B \cdot (-C)) \right) \\
                                    & = A\cdot (-(B\cdot (-C)))               \\
                                    & = A\cdot (B\cdot C).
              \end{align*}
              \textbf{Case 3.} $A\supseteq {0}^{*}, B\subseteq {0}^{*}, C\supseteq {0}^{*}$, then $-B\supseteq {0}^{*}$ and
              \begin{align*}
                  (A\cdot B)\cdot C & = (-(A\cdot (-B)))\cdot C \\
                                    & = -((A\cdot (-B))\cdot C) \\
                                    & = -(A\cdot ((-B)\cdot C)) \\
                                    & = A\cdot (-((-B)\cdot C)) \\
                                    & = A\cdot (B\cdot C).
              \end{align*}
              \textbf{Case 4.} $A\supseteq {0}^{*}, B\subseteq {0}^{*}, C\subseteq {0}^{*}$, then $-B\supseteq {0}^{*}$, $-C\supseteq {0}^{*}$, and
              \begin{align*}
                  (A\cdot B)\cdot C & = (-(A\cdot (-B)))\cdot C       \\
                                    & = (-(-(A\cdot (-B))))\cdot (-C) \\
                                    & = (A\cdot (-B))\cdot (-C)       \\
                                    & = A\cdot ((-B)\cdot (-C))       \\
                                    & = A\cdot (B\cdot C).
              \end{align*}
              \textbf{Case 5.} $A\subseteq {0}^{*}, B\supseteq {0}^{*}, C\supseteq {0}^{*}$, then $-A\supseteq {0}^{*}$, and
              \begin{align*}
                  (A\cdot B)\cdot C & = (-((-A)\cdot B))\cdot C \\
                                    & = -(((-A)\cdot B)\cdot C) \\
                                    & = -((-A)\cdot (B\cdot C)) \\
                                    & = A\cdot (B\cdot C).
              \end{align*}
              \textbf{Case 6.} $A\subseteq {0}^{*}, B\supseteq {0}^{*}, C\subseteq {0}^{*}$, then $-A\supset {0}^{*}$, $-C\supseteq {0}^{*}$, and
              \begin{align*}
                  (A\cdot B)\cdot C & = (-(A\cdot B))\cdot (-C) \\
                                    & = ((-A)\cdot B)\cdot (-C) \\
                                    & = (-A)\cdot (B\cdot (-C)) \\
                                    & = (-A)\cdot (-(B\cdot C)) \\
                                    & = A\cdot (B\cdot C).
              \end{align*}
              \textbf{Case 7.} $A\subseteq {0}^{*}, B\subseteq {0}^{*}, C\supseteq {0}^{*}$, then $-A\supseteq {0}^{*}, -B\supset {0}^{*}$, and
              \begin{align*}
                  (A\cdot B)\cdot C & = ((-A)\cdot (-B))\cdot C \\
                                    & = (-A)\cdot ((-B)\cdot C) \\
                                    & = (-A)\cdot (-(B\cdot C)) \\
                                    & = A\cdot (B\cdot C).
              \end{align*}
              \textbf{Case 8.} $A\subseteq {0}^{*}, B\subseteq {0}^{*}, C\subseteq {0}^{*}$, then $-A\supseteq {0}^{*}$, $-B\supseteq {0}^{*}$, $-C\supseteq {0}^{*}$, and
              \begin{align*}
                  (A\cdot B)\cdot C & = -((A\cdot B)\cdot (-C))       \\
                                    & = -(((-A)\cdot (-B))\cdot (-C)) \\
                                    & = -((-A)\cdot ((-B)\cdot (-C))) \\
                                    & = -((-A)\cdot (B\cdot C))       \\
                                    & = A\cdot (B\cdot C).
              \end{align*}
        \item Multiplication is distributive over addition.

              We will prove that
              \[
                  \begin{cases}
                      (A + B)\cdot C = A\cdot C + B\cdot C, \\
                      C\cdot (A + B) = C\cdot A + C\cdot B.
                  \end{cases}
              \]

              \textbf{Case 1.} $A = {0}^{*}$.
              \[
                  \begin{split}
                      (A + B)\cdot C = B\cdot C = {0}^{*} + B\cdot C = A\cdot C + B\cdot C, \\
                      C\cdot (A + B) = C\cdot B = {0}^{*} + C\cdot B = C\cdot A + C\cdot B.
                  \end{split}
              \]
              \textbf{Case 2.} $B = {0}^{*}$.
              \[
                  \begin{split}
                      (A + B)\cdot C = A\cdot C = A\cdot C + {0}^{*} = A\cdot C + B\cdot C, \\
                      C\cdot (A + B) = C\cdot A = C\cdot A + {0}^{*} = C\cdot A + C\cdot B.
                  \end{split}
              \]
              \textbf{Case 3.} $C = {0}^{*}$.
              \[
                  \begin{split}
                      (A + B)\cdot C = {0}^{*} = {0}^{*} + {0}^{*} = A\cdot C + B\cdot C, \\
                      C\cdot (A + B) = {0}^{*} = {0}^{*} + {0}^{*} = C\cdot A + C\cdot B.
                  \end{split}
              \]
              \textbf{Case 4.} $C\supset {0}^{*}, A\supset {0}^{*}, B\supset {0}^{*}$.
              \begin{align*}
                  (A + B)\cdot C      & = \{ (a + b)\cdot c : a\in A\land b\in B\land c\in C\land a+b\ge 0\land c\ge 0 \} \cup\mathbb{Q}_{-}                                                              \\
                                      & = \{ a\cdot c + b\cdot c : a\in A\land b\in B\land c\in C\land a+b\ge 0\land c\ge 0 \} \cup\mathbb{Q}_{-}                                                         \\
                  \\
                  A\cdot C + B\cdot C & = \{ a\cdot c : a\in A\land c\in C\land a\ge 0\land c\ge 0 \} \cup\mathbb{Q}_{-} + \{ b\cdot c : b\in B\land c\in C\land b\ge 0\land c\ge 0 \} \cup\mathbb{Q}_{-}
              \end{align*}

              \textbf{Part 1. Prove that $(A + B)\cdot C = A\cdot C + B\cdot C$.}

              \textbf{Step 1. Prove that $(A + B)\cdot C\subseteq A\cdot C + B\cdot C$.}

              Let $x\in (A + B)\cdot C$.

              If $x\le 0$, then $x\in A\cdot C + B\cdot C$, since both $A\cdot C + B\cdot C$ are supersets of $\{0\}\cup\mathbb{Q}$.

              \bigskip

              Otherwise $x > 0$, then there exists $a\in A, b\in B, c\in C$ such that $a + b > 0, c > 0$ and $(a + b)\cdot c = x$.

              $a\in A, c\in C$ and $c > 0$, then $a\cdot c\in A\cdot C$, no matter whether $a$ is greater than or not exceeding zero.

              $b\in B, c\in C$ and $c > 0$, then $b\cdot c\in B\cdot C$, no matter whether $b$ is greater than or not exceeding zero.

              So $x\in A\cdot C + B\cdot C$. Hence $(A + B)\cdot C\subseteq A\cdot C + B\cdot C$.

              \textbf{Step 2. Prove that $A\cdot C + B\cdot C\subseteq (A + B)\cdot C$.}

              Let $x\in A\cdot C + B\cdot C$.

              If $x\le 0$, then $x$ is also in $(A + B)\cdot C$ since $(A + B)\cdot C \supset \{0\}\cup\mathbb{Q}_{-}$.

              Otherwise, $x > 0$, then there exists $y\in A\cdot C$ and $z\in B\cdot C$ such that $y > 0$, $z > 0$ and $y + z = x$. We prove this result in the following.

              If $A\cdot C = B\cdot C$, then we choose $y = \dfrac{x}{2}, z = \dfrac{x}{2}$.

              Without loss of generality, suppose that $A$ is a proper subset of $B$.

              \begin{itemize}
                  \item $x\in A\cdot C$. Choose $y = \dfrac{x}{2}$ and $z = \dfrac{x}{2}$.
                  \item $x\notin A\cdot C\land x\in B\cdot C$.
                        Choose an element $y\in A\cdot C$ such that $y > 0$.

                        According to (DC4), $x - y\in B\cdot C$ ($x - y < x$). Since $x\notin A\cdot C$ and $y\in A\cdot C$, then $x - y > 0$.

                        We choose $z = x - y$.
                  \item $x\notin B\cdot C$. Then $x$ is greater than any elements of $A\cdot C$ and $B\cdot C$. According to the definition of addition, there exists $y\in A\cdot C$ and $z\in B\cdot C$ such that $y + z = x$.

                        Since $x > y$ and $x > z$, then $y > 0$ and $z > 0$.
              \end{itemize}

              $y\in A\cdot C$ and $y > 0$, there exists $a\in A$, $c_{1}\in C$ such that $a > 0, c_{1} > 0$ and $a\cdot c_{1} = y$.

              $z\in B\cdot C$ and $z > 0$, there exists $b\in B$, $c_{2}\in C$ such that $b > 0, c_{2} > 0$ and $b\cdot c_{2} = z$.

              (This choice of $c$ is from Hayden\footnote{\url{https://math.stackexchange.com/questions/1205640/proof-that-real-multiplication-distributes-over-addition-using-dedekind-cuts}} on MathOverflow) Let $c = \dfrac{a\cdot c_{1} + b\cdot c_{2}}{a + b}$. $c$ is between $c_{1}$ and $c_{2}$ so $c\in C$ (according to (DC4)).
              \begin{align*}
                  (a + b)\cdot c & = (a + b)\cdot\dfrac{a\cdot c_{1} + b\cdot c_{2}}{a + b} \\
                                 & = a\cdot c_{1} + b\cdot c_{2}                            \\
                                 & = y + z                                                  \\
                                 & = x.
              \end{align*}

              So $x\in (A + B)\cdot C$. Hence $A\cdot C + B\cdot C\subseteq (A + B)\cdot C$.

              Thus, $(A + B)\cdot C = A\cdot C + B\cdot C$.

              \bigskip

              \textbf{Part 2. $C\cdot (A + B) = C\cdot A + C\cdot B$}
              \begin{align*}
                  C\cdot (A + B) & = \{ c\cdot (a + b) : c\ge 0\land c\in C\land a\in A\land b\in B\land (a+b)\ge 0 \}\cup\mathbb{Q}_{-} \\
                                 & = \{ (a + b)\cdot c : c\ge 0\land c\in C\land a\in A\land b\in B\land (a+b)\ge 0 \}\cup\mathbb{Q}_{-} \\
                                 & = (A + B)\cdot C.
              \end{align*}
              On the other hand
              \begin{align*}
                  C\cdot A + C\cdot B & = \{ c\cdot a : c\in C\land c\ge 0\land a\in A\land a\ge 0 \}\cup\mathbb{Q}_{-} + \{ c\cdot b :  c\in C\land c\ge 0\land b\in B\land b\ge 0 \}\cup\mathbb{Q}_{-} \\
                                      & = \{ a\cdot c : c\in C\land c\ge 0\land a\in A\land a\ge 0 \}\cup\mathbb{Q}_{-} + \{ b\cdot c :  c\in C\land c\ge 0\land b\in B\land b\ge 0 \}\cup\mathbb{Q}_{-} \\
                                      & = A\cdot C + B\cdot C.
              \end{align*}
              According to step 1, $(A + B)\cdot C = A\cdot C + B\cdot C$.

              Thus, $C\cdot (A + B) = C\cdot A + C\cdot B$.

              \textbf{Case 5.} $C\supset {0}^{*}, A\subset {0}^{*}, B\subset {0}^{*}$.
              \begin{align*}
                  ((-B) + (-A)) + (A + B) & = ((-B) + ((-A) + A)) + B \\
                                          & = ((-B) + {0}^{*}) + B    \\
                                          & = (-B) + B                \\
                                          & = {0}^{*},                \\
                  (A + B) + ((-B) + (-A)) & = (A + (B + (-B))) + (-A) \\
                                          & = (A + {0}^{*}) + (-A)    \\
                                          & = A + (-A)                \\
                                          & = {0}^{*}.
              \end{align*}
              On the other hand, the addition over Dedekind cuts is commutative, it follows that $(-B) + (-A) = (-A) + (-B)$. So $-(A + B) = (-A) + (-B)$.

              $A\subset {0}^{*}$ and $B\subset {0}^{*}$ implies that $-A\supset {0}^{*}$ and $-B\supset {0}^{*}$. Apply Case 4 and Theorem~\ref{theorem:chapter1:multiplication-and-negation}
              \begin{align*}
                  (A + B)\cdot C & = -((-A) + (-B))\cdot C           \\
                                 & = -((-A)\cdot C + (-B)\cdot C)    \\
                                 & = (-(-A)\cdot C) + (-(-B)\cdot C) \\
                                 & = A\cdot C + B\cdot C,            \\
                  C\cdot (A + B) & = -C\cdot ((-A) + (-B))           \\
                                 & = -(C\cdot (-A) + C\cdot (-B))    \\
                                 & = (-C\cdot (-A)) + (-C\cdot (-B)) \\
                                 & = C\cdot A + C\cdot B
              \end{align*}

              \textbf{Case 6.} $C\supset {0}^{*}, A\supset {0}^{*}, B\subset {0}^{*}$.

              \textbf{Case 6.1.} $A + B = {0}^{*}$.
              \[
                  \begin{split}
                      (A + B)\cdot C = {0}^{*} = A\cdot C + (-A)\cdot C = A\cdot C + B\cdot C, \\
                      C\cdot (A + B) = {0}^{*} = C\cdot A + C\cdot (-A) = C\cdot A + C\cdot B.
                  \end{split}
              \]
              \textbf{Case 6.2.} $A + B > {0}^{*}$.

              I apply Case 4 and Theorem~\ref{theorem:chapter1:multiplication-and-negation}
              {\allowdisplaybreaks{}
                  \begin{align*}
                      (A + B)\cdot C + (-B)\cdot C   & = ((A + B) + (-B))\cdot C   \\
                                                     & = (A + (B + (-B)))\cdot C   \\
                                                     & = A\cdot C,                 \\
                      \Longrightarrow (A + B)\cdot C & = A\cdot C - (-B)\cdot C    \\
                                                     & = A\cdot C + (-(-B)\cdot C) \\
                                                     & = A\cdot C + B\cdot C.      \\
                      \bigskip
                      C\cdot (A + B) + C\cdot (-B)   & = C\cdot ((A + B) + (-B))   \\
                                                     & = C\cdot (A + (B + (-B)))   \\
                                                     & = C\cdot A,                 \\
                      \Longrightarrow C\cdot (A + B) & = C\cdot A - C\cdot (-B)    \\
                                                     & = C\cdot A + C\cdot B.
                  \end{align*}}

              \textbf{Case 6.3.} If $A + B < {0}^{*}$.
                  {\allowdisplaybreaks{}
                      \begin{align*}
                          (A + B)\cdot C & = -((-A) + (-B))\cdot C                                                                                   \\
                                         & = -((-A)\cdot C + (-B)\cdot C) & \quad\text{(Case 6.2)}                                                   \\
                                         & = A\cdot C + B\cdot C,         & \quad\text{(Theorem~\ref{theorem:chapter1:multiplication-and-negation})} \\
                          C\cdot (A + B) & = -C\cdot ((-A) + (-B))                                                                                   \\
                                         & = -(C\cdot (-A) + C\cdot (-B)) & \quad\text{(Case 6.2)}                                                   \\
                                         & = C\cdot A + C\cdot B.         & \quad\text{(Theorem~\ref{theorem:chapter1:multiplication-and-negation})}
                      \end{align*}}

              \textbf{Case 7.} $C\supset {0}^{*}, A\subset {0}^{*}, B\supset {0}^{*}$.

              This case is similar to Case 6.

              \textbf{Case 7.1.} $A + B = {0}^{*}$.
              \[
                  \begin{split}
                      (A + B)\cdot C = {0}^{*} = A\cdot C + (-A)\cdot C = A\cdot C + B\cdot C, \\
                      C\cdot (A + B) = {0}^{*} = C\cdot A + C\cdot (-A) = C\cdot A + C\cdot B.
                  \end{split}
              \]
              \textbf{Case 7.2.} $A + B > {0}^{*}$.
              \begin{align*}
                  (-A)\cdot C + (A + B)\cdot C   & = ((-A) + (A + B))\cdot C \\
                                                 & = (((-A) + A) + B)\cdot C \\
                                                 & = B\cdot C                \\
                  \Longrightarrow (A + B)\cdot C & = -(-A)\cdot C + B\cdot C \\
                                                 & = A\cdot C + B\cdot C.    \\
                  \bigskip
                  C\cdot (-A) + C\cdot (A + B)   & = C\cdot ((-A) + (A + B)) \\
                                                 & = C\cdot (((-A) + A) + B) \\
                                                 & = C\cdot B                \\
                  \Longrightarrow C\cdot (A + B) & = -C\cdot (-A) + C\cdot B \\
                                                 & = C\cdot A + C\cdot B.
              \end{align*}

              \textbf{Case 7.3.} $A + B < {0}^{*}$.
              \begin{align*}
                  (A + B)\cdot C & = -((-A) + (-B))\cdot C                                                                                    \\
                                 & = -((-A)\cdot C + (-B)\cdot C) & \quad\text{(Case 7.2)}                                                    \\
                                 & = A\cdot C + B\cdot C,         & \quad\text{(Theorem~\ref{theorem:chapter1:multiplication-and-negation})}  \\
                  C\cdot (A + B) & = -C\cdot ((-A) + (-B))                                                                                    \\
                                 & = -(C\cdot (-A) + C\cdot (-B)) & \quad\text{(Case 7.2)}                                                    \\
                                 & = C\cdot A + C\cdot B.         & \quad\text{(Theorem~\ref{theorem:chapter1:multiplication-and-negation})}.
              \end{align*}

              \textbf{Case 8.} $C\subset {0}^{*}$.

              Apply Case 4, 5, 6, 7 and Theorem~\ref{theorem:chapter1:multiplication-and-negation}
              \[
                  \begin{split}
                      (A + B)\cdot C = -(A + B)\cdot (-C) = -(A\cdot (-C) + B\cdot (-C)) = A\cdot C + B\cdot C, \\
                      C\cdot (A + B) = -(-C)\cdot (A + B) = -((-C)\cdot A + (-C)\cdot B) = C\cdot A + C\cdot B.
                  \end{split}
              \]
        \item Multiplication has identity element.

              I will prove that
              \[
                  A\cdot {1}^{*} = {1}^{*}\cdot A = A.
              \]
              \textbf{Case 1.} $A = {0}^{*}$.

              According to the proof of Theorem~\ref{theorem:chapter1:multiplication}
              \[
                  A\cdot {1}^{*} = {0}^{*}\cdot {1}^{*} = {0}^{*} = {1}^{*}\cdot {0}^{*} = {1}^{*}\cdot A.
              \]
              \textbf{Case 2.} $A > {0}^{*}$.
              \begin{align*}
                  A\cdot {1}^{*} & = \{ a\cdot b : a\in A\land a\ge 0\land b\ge 0\land b < 1 \}\cup\mathbb{Q}_{-} \\
                                 & = \{ b\cdot a : a\in A\land a\ge 0\land b\ge 0\land b < 1 \}\cup\mathbb{Q}_{-} \\
                                 & = {1}^{*}\cdot A.
              \end{align*}
              Let $x\in A$.

              If $x\le 0$, then $x\in A\cdot {1}^{*}$.

              Otherwise, $x > 0$. According to (DC3), there exists $y\in A$ such that $x < y$. Then
              \[
                  x = \underbrace{y}_{>0, \in A}\cdot\underbrace{\dfrac{x}{y}}_{>0, <1}
              \]
              Hence $x\in A\cdot {1}^{*}$, so $A\subseteq A\cdot {1}^{*}$.

              \bigskip

              Let $x\in A\cdot {1}^{*}$.

              If $x\le 0$, then $x\in A$.

              Otherwise, $x > 0$. Then there exists $a\in A$ and $b\in {1}^{*}$ such that $a > 0, b > 0$ and $x = a\cdot b$.

              Since $0 < b < 1$, then $a\cdot b < a$.

              According to (DC4), $x = a\cdot b\in A$, so $x\in A$. Therefore $A\cdot {1}^{*}\subseteq A$.

              \bigskip

              Thus, $A\cdot {1}^{*} = {1}^{*}\cdot A = A$.

              \textbf{Case 3.} $A < 0^{*}$

              Apply Case 2 and Theorem~\ref{theorem:chapter1:multiplication-and-negation}
              \[
                  \begin{split}
                      A\cdot {1}^{*} = -(-A)\cdot {1}^{*} = -(-A) = A, \\
                      {1}^{*}\cdot A = -{1}^{*}\cdot (-A) = -(-A) = A.
                  \end{split}
              \]
              \bigskip
        \item Multiplication is commutative.

              I will prove that $A\cdot B = B\cdot A$.

              If $A = {0}^{*}$ or $B^{*}$, then according to the proof of Theorem~\ref{theorem:chapter1:multiplication}, $A\cdot B = {0}^{*} = B\cdot A$.

              Otherwise, let's consider the following four cases.

              \textbf{Case 1.} $A > {0}^{*}, B > {0}^{*}$.
              \begin{align*}
                  A\cdot B & = \{ a\cdot b : a\in A\land b\in B\land a\ge 0\land b\ge 0 \}\cup\mathbb{Q}_{-} \\
                           & = \{ b\cdot a : a\in A\land b\in B\land a\ge 0\land b\ge 0 \}\cup\mathbb{Q}_{-} \\
                           & = B\cdot A.
              \end{align*}

              In the following cases, I use Case 1 and Theorem~\ref{theorem:chapter1:multiplication-and-negation}.

              \textbf{Case 2.} $A > {0}^{*}, B < {0}^{*}$.
              \[
                  A\cdot B = -A\cdot (-B) = -(-B)\cdot A = B\cdot A.
              \]
              \textbf{Case 3.} $A < {0}^{*}, B > {0}^{*}$.
              \[
                  A\cdot B = -(-A)\cdot B = -B\cdot (-A) = B\cdot A.
              \]
              \textbf{Case 4.} $A < {0}^{*}, B < {0}^{*}$.
              \[
                  A\cdot B = (-A)\cdot (-B) = (-B)\cdot (-A) = B\cdot A.
              \]
    \end{enumerate}
\end{proof}

(F1) (F2) (F3) (F4) (F5) (F6) (F7) (F8) make $\mathbb{R}$ a commutative ring. Together with the following, $\mathbb{R}$ is a field.

\begin{theorem}\label{theorem:chapter1:division}
    Let $A, B$ be Dedekind cuts such that $B\ne {0}^{*}$. Define the set $A/B$ as the following

    \noindent If $A\ge {0}^{*}, B > {0}^{*}$
    \[
        A/B = \left\{ \frac{a}{b} : a\in A\land a\ge 0\land b\in\mathbb{Q}\setminus B \right\}\cup\mathbb{Q}_{-}.
    \]
    If $A\le {0}^{*}, B > {0}^{*}$
    \[
        A/B = -(-A)/B.
    \]
    If $A\ge {0}^{*}, B < {0}^{*}$
    \[
        A/B = -A/(-B).
    \]
    If $A\le {0}^{*}, B < {0}^{*}$
    \[
        A/B = (-A)/(-B).
    \]
    Then $A/B$ is a Dedekind cut.
\end{theorem}

\begin{proof}
    It suffices to prove 1st case: $A\supseteq {0}^{*}$ and $B\supset {0}^{*}$.

    \begin{enumerate}[label={\textbf{Case \arabic*.}},itemindent=0.4cm]
        \item $A\ge {0}^{*}, B > {0}^{*}$.
              $A/B$ is a superset of $\mathbb{Q}_{-}$, so $A/B$ is not empty. (DC1) is satisfies.

              Let $c\in\mathbb{Q}\setminus A$ and $d\in B$ such that $d > 0$.


        \item $A\le {0}^{*}, B > {0}^{*}$.
        \item $A\ge {0}^{*}, B < {0}^{*}$.
        \item $A\le {0}^{*}, B < {0}^{*}$.
    \end{enumerate}
\end{proof}

\begin{theorem}
    \begin{enumerate}[label={(F\arabic*)},start=9]
        \item Each non-zero element of $\mathbb{R}$ has a multiplicative inverse.
    \end{enumerate}
\end{theorem}

\begin{proof}
\end{proof}

\begin{theorem}
    Let $A$ be a Dedekind cut and $A\ne {0}^{*}$, then
    \[
        {1}^{*}/({1}^{*}/A) = A.
    \]
\end{theorem}

\begin{proof}
\end{proof}

\begin{theorem}
    $\mathbb{R}$ is a field with characteristic zero.
\end{theorem}

\begin{proof}
\end{proof}

\subsection{Embed $\mathbb{Q}$ into $\mathbb{R}$}

\subsection{Completeness}

\section{Construction of the real numbers by Cauchy sequences}

\section{Complex numbers}

