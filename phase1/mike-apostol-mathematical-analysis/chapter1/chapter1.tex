\documentclass[class=mike-apostol-mathematical-analysis,crop=false]{standalone}

\begin{document}

\chapter{The Real and Complex Number Systems}

\section{Introduction}

\section{The field axiom}

\section{The order axiom}

\section{Geometric representation of real numbers}

\section{Intervals}

\section{Integers}

\section{The unique factorization theorem for integers}

\section{Rational numbers}

\section{Irrational numbers}

\section{Upper bounds, maximum element, least upper bound (supremum)}

\section{The completeness axiom}

\section{Some properties of the supremum}

\section{Properties if the integers deduced from the completness axiom}

\section{The Archimedean property of the real-number system}

\section{Rational numbers with finite decimal representation}

\section{Finite decimal approximations to real numbers}

\section{Infinite decimal representation of real numbers}

\section{Absolute values and the triangle inequality}

\section{The Cauchy-Schwarz inequality}

\section{Plus and minus infinity and the extended real number system $\mathbb{R}^{*}$}

\section*{Construction of the real numbers by Dedekind cuts}\addcontentsline{toc}{section}{[Note] Construction of the real numbers by Dedekind cuts}

\par In this section, we are going to build the real numbers from the rational numbers.

\par We re-introduce \textit{the Dedekind cuts} --- a model for real numbers. Then we will prove that the set of Dedekind cuts satisfy all properties of the real numbers (things we have taken for granted: field structure, total order, axiom of completeness).

\subsection*{Dedekind cuts}

\subsection*{Properties}

\par In this subsection, $\mathbb{R}$ is the set of all Dedekind cuts.

\begin{definition}
    \begin{enumerate}
        \item Addition.
        \item Multiplication.
    \end{enumerate}
\end{definition}

\begin{theorem}
    $\mathbb{R}$ is a field with characteristic zero.
\end{theorem}

\begin{theorem}
    $\mathbb{R}$ is totally ordered with relation $\leq$.
\end{theorem}

\begin{theorem}
    The embedding $\iota: \mathbb{Q} \to \mathbb{R}, r \mapsto {r}^{*}$ is an order-preserving field monomorphism.
\end{theorem}

\section{Complex numbers}

\section{Geometric representation of complex numbers}

\section{The imaginary unit}

\section{Absolute value of a complex number}

\section{Impossibility of ordering the complex numbers}

\section{Complex exponentials}

\section{Further properties of complex exponentials}

\section{The argument of a complex number}

\section{Integral powers and roots of complex numbers}

\section{Complex logarithms}

\section{Complex sines and cosines}

\section{Infinity and the extended complex plane $\mathbb{C}^{*}$}

\end{document}
