\chapter{The Real and Complex Number Systems}

\section{Axioms of real numbers}

I am familar with the real numbers. However, I have taken the properties of real numbers for granted. I can't even answer these questions: ``What are real numbers? Are they real?\@'' Turns out, these questions are really difficult.

Real numbers are elements of a set $\mathbb{R}$ endowed with addition and multiplication, which satisfy the following properties
\begin{enumerate}[label = (\roman*)]
    \item $\mathbb{R}$ is totally ordered (there exists an order relation on $\mathbb{R}$ where any two elements can be compared).
    \item $\mathbb{R}$ is a field under addition and multiplication.
    \item Order in $\mathbb{R}$ is preserved under addition and multiplication (with non-negative real number).
    \item Every non-empty set of $\mathbb{R}$ which is bounded above has a least upper bound.
\end{enumerate}

To those who ask ``Are real numbers real?\@'', we can establish a model (a mathematical structure) that satisfies every axiom above. Since 19th century, mathematicians have given several constructions of the real numbers. IMHO, the two most notable and commonlys used constructions are \textit{Dedekind cuts} and \textit{Cauchy sequences}. In the following sections, we will reproduce the two constructions.

\section{* Construction of the real numbers by Dedekind cuts of $\mathbb{Q}$}

We will give the definition of Dedekind cuts from which we construct a model that satisfies the real numbers axioms.

\subsection{Definition of Dedekind cuts of $\mathbb{Q}$}

\begin{definition}[Dedekind cuts]
    A Dedekind cut $A$ of $\mathbb{Q}$ is a subset of $\mathbb{Q}$ such that
    \begin{enumerate}[label = (DC\arabic*),itemindent=0.3cm]
        \item $A\ne\varnothing$; in other words, $A$ is not empty.
        \item $A\neq\mathbb{Q}$; in other words, $A$ is not the entire set of rational numbers.
        \item $\forall x\left(x\in A \rightarrow \exists y \left( y\in A \wedge x < y \right)\right)$; in other words, $A$ has no greatest element.
        \item $\forall x\in A\left(\forall y( y < x \rightarrow y\in A)\right)$; in other words, $A$ is downward closed.
    \end{enumerate}
\end{definition}

It follows from the definition that $\mathbb{Q}\setminus A$ contains every rational upper bound of $A$.

We will use Dedekind cuts of $\mathbb{Q}$ to ``cut'' the rational number line.

Our goal is from the definition of Dedekind cuts as well as operations (addition and multiplication) and order relation (less than or equal) on them, we prove that Dedekind cuts satisfy the axioms of the real numbers.

Sequentially, we will define the following based on Dedekind cuts of $\mathbb{Q}$:
\begin{itemize}[itemsep=0pt]
    \item Order,
    \item Rational and irrational cut,
    \item Addition,
    \item Subtraction and negation,
    \item Multiplication and division.
\end{itemize}

then we will show that the set of all Dedekind cuts equipped with those operations will satisfy the axioms of the real numbers.

Upcoming proofs in this section will use various properties of rational numbers, including:
\begin{itemize}[itemsep=0pt]
    \item Total order.
    \item The field structure.
    \item Compatibility of addition and multiplication with the order relation.
    \item There exists a rational number strictly between any two distinct rational numbers.
          \[
              \forall q_{1}\forall q{2}\left( q_{1}\in\mathbb{Q}\land q_{2}\in\mathbb{Q}\land q_{1} < q_{2} \rightarrow \exists q (q\in\mathbb{Q}\land q_{1} < q\land q < q_{2}) \right).
          \]
\end{itemize}

Denote the set of all Dedekind cuts by $\mathbb{R}$.

\begin{theorem}[Totally ordered]
    $\mathbb{R}$ is totally ordered with $\subseteq$ relation.
\end{theorem}

\begin{proof}
    Let $A$ and $B$ be two Dedekind cuts of $\mathbb{Q}$.

    The subset relation $\subseteq$ is reflexive, transitive, and anti-symmetric.
    \bigskip

    Suppose that $A\ne B$.

    Without loss of generality, suppose that there exists $b\in B$ such that $b\notin A$.

    $b\in B$, then $b$ is an upper bound of $A$.

    Let $a$ be an arbitrary element of $A$, then $a\le b$. According to (DC4), $a\in B$. Hence $\forall a(a\in A\rightarrow a\in B)$.

    Therefore, $A$ is a proper subset of $B$.
    \bigskip

    So for every two Dedekind cuts $A$, $B$, one of the following holds: $A\subset B$, $A = B$, $A\supset B$. Hence, the set of all Dedekind cuts of $\mathbb{Q}$ is totally ordered with $\subseteq$ relation.
\end{proof}

We define the relation $\le$ on $\mathbb{R}$ as follow:
\[
    A\le B \Longleftrightarrow A\subseteq B.
\]

For convenience, in this section, we use the following notation:
\[
    \begin{split}
        {0}^{*} & = \{ x : x\in\mathbb{Q} \land x < 0 \}, \\
        {q}^{*} & = \{ x : x\in\mathbb{Q} \land x < q \}\qquad\text{($q\in\mathbb{Q}$)}.
    \end{split}
\]

\begin{definition}
    A Dedekind cut $A$ of $\mathbb{Q}$ is called:
    \begin{enumerate}[label={(\roman*)},itemsep=0pt]
        \item positive if $A$ is a proper superset of ${0}^{*}$,
        \item negative if $A$ is a proper subset of ${0}^{*}$,
        \item non-positive if $A\le {0}^{*}$,
        \item non-negative if $A\ge {0}^{*}$.
    \end{enumerate}
\end{definition}

\begin{definition}[Rational and irrational]
    A Dedekind cut $A$ of $\mathbb{Q}$ is called:
    \begin{enumerate}[label={(\roman*)}]
        \item rational if $\mathbb{Q}\setminus A$ has a least element,
        \item irrational if $\mathbb{Q}\setminus A$ has no least element.
    \end{enumerate}
\end{definition}

The following example gives us an example of rational cut, and an example of irrational cut.

\begin{example}
    \[
        A = \{ x : x\in\mathbb{Q}\land x < q \}
    \]

    where $q$ is a rational number, is a rational cut.
    \[
        B = \{ x\in\mathbb{Q}: {x}^{2} < 2 \} \cup \mathbb{Q}^{-}
    \]
    is an irrational cut.
\end{example}

\begin{proof}
    \begin{itemize}[topsep=0pt]
        \item Rational cut.
              \begin{enumerate}[label={(\roman*)},topsep=0pt]
                  \item $A$ is not empty since there are rational numbers which are less than $q$.
                  \item $A\ne\mathbb{Q}$ since $A$ is bounded above by $q$.
                  \item Let $x\in A$.

                        If $x = \dfrac{a}{b} > 0$ and $q = \dfrac{c}{d} > 0$, where $a, b, c, d > 0$, then
                        \[
                            x = \frac{a}{b} < \frac{a + c}{b + d} < \frac{c}{d} = q
                        \]

                        so there exists an element of $A$ which is greater than $x$.

                        If $x < 0$ and $q > 0$, then $0$ is an element of $A$ which is greater than $x$.

                        If $x = -\dfrac{a}{b} < 0$ and $q = -\dfrac{c}{d} < 0$ where $a, b, c, d > 0$, then
                        \[
                            x = -\frac{a}{b} < -\frac{a + c}{b + d} < -\frac{c}{d} = q
                        \]

                        so there exists an element of $A$ which is greater than $x$.

                        Therefore, $A$ does not have a greatest element.
                  \item Let $x\in A$.

                        Let $y$ be a rational number such that $y < x$. Since $x < q$, then $y < q$. Therefore, $y\in A$.

                        So $A$ is closed downward.
              \end{enumerate}
              $\mathbb{Q}\setminus A = \{ x\in\mathbb{Q}: x\ge 1 \}$ has least element, which is $1$. So $ A$ is a rational cut.
        \item Irrational cut.
              \begin{enumerate}[label={(\roman*)},topsep=0pt]
                  \item $B$ contains $0$, so $B$ is not empty.
                  \item $B$ does not contain $2$, so $B\ne\mathbb{Q}$.
                  \item Let $x\in B$.

                        If $x\le 0$, then there exists elements of $B$ which are greater than $x$. For example, $1$.

                        Otherwise, $x > 0$. Let $y = \dfrac{4\cdot x}{{x}^{2} + 2}$.
                        \begin{align*}
                            y & = \frac{4\cdot x}{{x}^{2} + 2} - x + x                  \\
                              & = \frac{4\cdot x - {x}^{3} - 2\cdot x}{{x}^{2} + 2} + x \\
                              & = \frac{2\cdot x - {x}^{3}}{{x}^{2} + 2} + x            \\
                              & = \frac{x\cdot (2 - {x}^{2})}{{x}^{2} + 2} + x > x
                        \end{align*}

                        On the other hand
                        \begin{align*}
                            {y}^{2} & = \frac{16\cdot {x}^{2}}{{({x}^{2} + 2)}^{2}} - 2 + 2                                  \\
                                    & = \frac{16\cdot {x}^{2} - 2\cdot{x}^{4} - 8 - 8\cdot {x}^{2}}{{({x}^{2} + 2)}^{2}} + 2 \\
                                    & = \frac{-2\cdot {({x}^{2} - 2)}^{2}}{{({x}^{2} + 2)}^{2}} + 2 < 2
                        \end{align*}

                        Therefore, $y\in A$ and $x < y$, which implies that $A$ does not have a greatest element.
                  \item Let $x\in B$.

                        Let $y$ be a rational number such that $y < x$.

                        If $y\le 0$, then $y\in B$, since $B$ is a superset of $\mathbb{Q}$.

                        Otherwise, $y > 0$, then
                        \[
                            {y}^{2} = {y}^{2} - {x}^{2} + {x}^{2} = (y - x)\cdot(y + x) + {x}^{2} < 0 + 2 = 2
                        \]

                        so $y\in B$. Therefore, $B$ is closed downward.
              \end{enumerate}

              Hence, $B$ is a Dedekind cut.

              $\mathbb{Q}\setminus B = \{ x\in\mathbb{Q}: {x}^{2}\ge 2 \wedge x > 0 \}$.

              Since there is no rational number $r$ of which square equals $2$, then $\mathbb{Q}\setminus B = \{ x\in\mathbb{Q}: {x}^{2} > 2 \wedge x > 0 \}$ (change from $\ge$ to $>$).

              Let $q\in\mathbb{Q}\setminus B$. Consider $r = \dfrac{q}{2} + \dfrac{1}{q}$.
              \begin{align*}
                  r = \frac{q}{2} + \frac{1}{q} & = -\frac{q}{2} + \frac{1}{q} + q                       \\
                                                & = \frac{2 - {q}^{2}}{2q} + q                           \\
                                                & < q \quad\text{(Since $q > 0$ and $2 - {q}^{2} < 0$)}.
              \end{align*}
              \begin{align*}
                  {r}^{2} & = {\left(\frac{q}{2} + \frac{1}{q}\right)}^{2} = \frac{q^{2}}{4} + \frac{1}{q^{2}} + 1                                                         \\
                          & = \frac{q^{2}}{4} + \frac{1}{q^{2}} - 1 + 2 = {\left(\frac{q}{2} - \frac{1}{q}\right)}^{2} + 2 = {\left( \frac{q^{2} - 2}{2q} \right)}^{2} + 2 \\
                          & > 2
              \end{align*}
              Therefore, $\forall q(q\in\mathbb{Q}\setminus B \rightarrow \exists r( r\in\mathbb{Q}\setminus B \wedge r < q ))$. Hence $\mathbb{Q}\setminus B$ does not have a least element. According to the definition, $B$ is an irrational cut.
    \end{itemize}
\end{proof}

\subsection{Addition}

\begin{theorem}[Addition]
    $A, B$ are Dedekind cuts. Then
    \[
        A + B = \{ x + y : x\in A \wedge y\in B \}
    \]
    is also a Dedekind cut.
\end{theorem}

\begin{proof}
    \begin{enumerate}[label={(\roman*)},itemsep=0pt]
        \item Since $A\ne\varnothing$ and $B\ne\varnothing$, then there exists $a\in A$ and $b\in B$. By definition of $A + B$, we obtain that $a + b \in A + B$. This implies that $A + B$ is not empty.
        \item A Dedekind cut of $\mathbb{Q}$ is downward closed and not the entire set of rational numbers, then it is bounded above.

              Therefore, $A$ and $B$ are bounded above. Let $a$ be an upper bound of $A$, $b$ be an upper bound of $B$.

              $\forall x\in A\forall y\in B$, then $x + y \le a + b$, which means $A + B$ is bounded above.

              Hence $A + B\ne\mathbb{Q}$.
        \item Let $c$ be an element of $A + B$. According to the definition of $A + B$, there exists $a\in A$ and $b\in B$ such that $a + b = c$.

              According to (DC3), there exists $a_{0}\in A$ such that $a < a_{0}$, and there exists $b_{0}\in B$ such that $b < b_{0}$.

              $c = a + b < a_{0} + b_{0}$. According to the definition of $A + B$, $a_{0} + b_{0} \in A + B$. Hence $A + B$ has no greatest element.
        \item Let $c$ be an element of $A + B$. According to the definition of $A + B$, there exists $a\in A$ and $b\in B$ such that $a + b = c$.

              Let $c_{1}$ be a rational number such that $c_{1} < c$.

              According to (DC4) $b + (c_{1} - c)\in B$.

              Hence
              \[
                  \underbrace{a}_{\in A} + \underbrace{b + (c_{1} - c)}_{\in B} \in A + B.
              \]
              Therefore
              \[
                  c_{1} \in A + B.
              \]
              Hence $A + B$ is downward closed.
    \end{enumerate}
    In conclusion, $A + B$ is a Dedekind cut of $\mathbb{Q}$.
\end{proof}

I have difficulty defining multiplication since there are positive numbers and negative numbers. So I define additive inverse/negation of a cut.

\begin{theorem}[Subtraction]
    Let $A$ and $B$ be Dedekind cuts of $\mathbb{Q}$
    \[
        A - B = \{ a - b: a\in A\land b\in\mathbb{Q}\setminus B \}
    \]
    is also a Dedekind cut.
\end{theorem}

\begin{proof}
    \begin{enumerate}[label = (\roman*)]
        \item Since $A\ne\varnothing$, there exists $a\in A$. Since $B\ne\mathbb{Q}$, there exists $b\in\mathbb{Q}\setminus B$. So $a - b\in A - B$, which implies that $A - B$ is not empty.
        \item Let $a_{0}\in\mathbb{Q}\setminus A, b_{0}\in B, a\in A, b\in\mathbb{Q}\setminus B$. Then $a\le a_{0}$ and $b_{0} < b$.
              \[
                  \underbrace{a - b}_{\in A - B} < a_{0} - b_{0},
              \]
              Therefore, $A - B$ is bounded above, so $A - B\ne\mathbb{Q}$.
        \item Let $c$ be an arbitrary element of $A - B$. According to the definition of $A - B$, there exists $a\in A$ and $b\in\mathbb{Q}\setminus B$ such that $c = a - b$.

              Due to (DC3), there exists $d\in A$ such that $a < d$.

              Then $c = a - b < d - b \in A - B$.

              So $A - B$ does not have greatest element.
        \item Let $c$ be an arbitrary element of $A - B$. According to the definition of $A - B$, there exists $a\in A$ and $b\in\mathbb{Q}\setminus B$ such that $c = a - b$.

              Let $d$ be a rational number such that $d < c$.
              \[
                  d = d - c + c = (d - c) + a - b = \underbrace{(d - c + a)}_{< a} - \underbrace{b}_{\in\mathbb{Q}\setminus B}
              \]
              $(d - c + a) < a$, therefore $(d - c + a)\in A$, according to (DC4). Hence $d\in A - B$.

              So $A - B$ is downward closed.
    \end{enumerate}
    Thus, $A - B$ is a Dedekind cut.
\end{proof}

The following theorem is a special case of subtraction.

\begin{theorem}[Additive inverse/Negation]
    Let $A$ be a Dedekind cut.
    \[
        -A = \{ b - a : b\in{0}^{*} \wedge a\in\mathbb{Q}\setminus A \}
    \]
    is also a Dedekind cut.
\end{theorem}

\begin{theorem}\label{theorem:chapter1:negation-and-subtraction}
    Let $A$ and $B$ be Dedekind cuts of $\mathbb{Q}$, then
    \[
        A - B = A + (-B)
    \]
\end{theorem}

\begin{proof}
    According to the definitions of addition, subtraction, and negation
    \begin{align*}
        A - B    & = \{ a - b : a\in A\land b\in\mathbb{Q}\setminus B \},                      \\
        A + (-B) & = \{ a + w - b : a\in A\land w\in{0}^{*}\land b\in\mathbb{Q}\setminus B \}.
    \end{align*}

    \textbf{Step 1.} Prove that $A - B\subseteq A + (-B)$.

    Let $c = a - b$ be an arbitrary element in $A - B$, where $a\in A$ and $b\in\mathbb{Q}\setminus B$.

    According to (DC3), there exists $a_{0}\in A$ such that $a < a_{0}$.
    \[
        c = a - b = \underbrace{a_{0}}_{\in A} + \underbrace{(a - a_{0})}_{< 0} - \underbrace{b}_{\in\mathbb{Q}\setminus B}
    \]

    so $c\in A + (-B)$. Hence $A - B\subseteq A + (-B)$.
    \bigskip

    \textbf{Step 2.} Prove that $A + (-B)\subseteq A - B$.

    Let $c = a + (w - b)$ be an arbitrary element of $A + (-B)$, where $a\in A$, $w < 0$, and $b\in\mathbb{Q}\setminus B$.

    $a + w < a$. According to (DC4), $a + w\in A$.
    \[
        c = a + (w - b) = \underbrace{(a + w)}_{\in A} - \underbrace{b}_{\in\mathbb{Q}\setminus B}
    \]

    so $c\in A - B$. Hence $A + (-B)\subseteq A - B$.
    \bigskip

    Thus, $A - B = A + (-B)$.
\end{proof}

I need the following property to verify the group structure of $\mathbb{R}$.

\begin{lemma}[Archimedean property for rational numbers]
    Let $a, b$ be rational numbers where $a > 0$. Then there exists an integer $n$ such that
    \[
        (n - 1)\le \frac{b}{a} < n.
    \]
\end{lemma}

\begin{proof}
    Since $a, b$ are rational numbers, then $\dfrac{b}{a}$ is also rational number.

    Then there exists a positive integer $q$ and an integer $p$ such that $\dfrac{b}{a} = \dfrac{p}{q}$.

    Apply Euclid division algorithm, there exists two integers $k, r$ such that $0\le r < q$ and $p = k\cdot q + r$.
    \begin{align*}
                         & \frac{p}{q} = \frac{k\cdot q + r}{q} = k + \frac{r}{q} \\
        \Rightarrow\quad & k \le \frac{p}{q} < k + 1.\qedhere
    \end{align*}
\end{proof}

\begin{theorem}\label{theorem:chapter1:real-field-part-one}
    $\mathbb{R}$ with addition is a commutative group.
    \begin{enumerate}[label={(F\arabic*)},itemsep=0pt]
        \item Addition is associative.
        \item Addition has identity element.
        \item Each element has an additive inverse.
        \item Addition is commutative.
    \end{enumerate}
\end{theorem}

\begin{proof}
    Let $A, B, C$ be arbitrary Dedekind cuts.
    \begin{enumerate}[label = (F\arabic*)]
        \item Addition is associative.
              \begin{align*}
                  (A + B) + C & = \{ (a + b) + c : a\in A\land b\in B\land c\in C \}                                                       \\
                              & = \{ a + (b + c) : a\in A\land b\in B\land c\in C \} \quad\text{(Addition in $\mathbb{Q}$ is associative)} \\
                              & = A + (B + C).
              \end{align*}
        \item Addition has identity element.
              \begin{align*}
                  A + {0}^{*} & = \{ a + w : a\in A\land w < 0 \} \\
                              & = \{ w + a : a\in A\land w < 0 \} \\
                              & = {0}^{*} + A.
              \end{align*}
              \textbf{Step 1. Prove that $A \subseteq A + {0}^{*}$}.

              Let $x\in A$. According to (DC3), there exists $y\in A$ such that $x < y$.
              \[
                  x = \underbrace{y}_{\in A} + \underbrace{(x - y)}_{< 0, \in {0}^{*}}
              \]
              So $\forall x(x\in A \rightarrow x\in A + {0}^{*})$, which means $A \subseteq A + {0}^{*}$.
              \bigskip

              \textbf{Step 2. Prove that $A + {0}^{*} \subseteq A$}.

              Let $a_{0}\in A + {0^{*}}$. According to the definition of $A + {0}^{*}$, there exists $a\in A$ and $w\in\mathbb{Q}^{-}$ such that $a_{0} = a + w$.

              Since $w < 0$ then $a_{0} < a$. According to (DC4), $a_{0}\in A$.

              So $\forall a_{0}(a_{0}\in A + {0}^{*} \rightarrow A)$.

              Hence $A = A + {0}^{*} = {0}^{*} + A$.
        \item Every element has an additive inverse.
              \begin{align*}
                  A + (-A) & = \{ a + (w - a') : a\in A\land w\in\mathbb{Q}^{-}\land a'\in\mathbb{Q}\setminus A \} \\
                           & = \{ (w - a') + a : a\in A\land w\in\mathbb{Q}^{-}\land a'\in\mathbb{Q}\setminus A \} \\
                           & = (-A) + A.
              \end{align*}
              \textbf{Step 1. Prove that $A + (-A)\subseteq {0}^{*}$}.

              Let $a\in A, a'\in \mathbb{Q}\setminus A, w\in {0}^{*}$.

              Since $a'\notin A$ then $a' > a$, so $a - a' < 0$. Therefore
              \[
                  a + (w - a') = w + (a - a') < w
              \]
              According to (DC4), $a + (w - a')\in {0}^{*}$. So $A + (-A) \subseteq {0}^{*}$.
              \bigskip

              \textbf{Step 2. Prove that ${0}^{*}\subseteq A + (-A)$}.

              Let $w\in {0}^{*}$.

              We will prove that the set $S = \{ n : n\in\mathbb{Z} \land n\cdot w\in A \}$ has a least element. In equivalent, $S$ is the set of integers such that $n\cdot w\in A$.
              Let $x\in A, y\in\mathbb{Q}\setminus A$. $w < 0$, then according to the Archimedean property, there exists an integer $m$ such that
              \begin{align*}
                  m - 1    & \le \frac{x}{-w} < m \\
                  (1 - m)w & \le x < -m\cdot w
              \end{align*}
              According to (DC4), $(1 - m)w\in A$. Then $S$ is not empty.

              Since the set $\{ n\cdot w : n\in\mathbb{Z}\land n\cdot w\in A \}$ is bounded above, and $w < 0$, then $S$ is bounded below. On the other hand, $S$ is a set of consecutive integers and $S$ is bounded below, then $S$ has a least element (this is another form of the well-ordering principle).

              Let $n$ be the least integer such that $n\cdot w\in A$. Then $(n - 1)\cdot w\notin A$. Hence $(n - 1)\cdot w\in\mathbb{Q}\setminus A$.
              \[
                  w = \underbrace{n\cdot w}_{\in A} - \underbrace{(n - 1)\cdot w}_{\in\mathbb{Q}\setminus A}
              \]

              so $w\in A - A$. According to Theorem~\ref{theorem:chapter1:negation-and-subtraction}, $A - A = A + (-A)$. Then $w\in A + (-A)$. Therefore, ${0}^{*}\subseteq A + (-A)$.
              \bigskip

              Hence $A + (-A) = (-A) + A = {0}^{*}$.
        \item Addition is commutative.
              \begin{align*}
                  A + B & = \{ a + b : a\in A\land b\in B \}                                                      \\
                        & = \{ b + a : b\in B\land a\in A \}\qquad\text{(Addition n $\mathbb{Q}$ is commutative)} \\
                        & = B + A.\qedhere
              \end{align*}
    \end{enumerate}
\end{proof}

\begin{theorem}\label{theorem:chapter1:negation-is-an-involution}
    Let $A$ be a Dedekind cut, then
    \[
        A = -(-A).
    \]
\end{theorem}

\begin{proof}
    Firstly, we prove that there exist a unique Dedekind cut $A'$ such that $A + A' = A' + A = {0}^{*}$.

    According to Theorem~\ref{theorem:chapter1:real-field-part-one}, $A + (-A) = (-A) + A = {0}^{*}$.

    Suppose that $A + A' = A' + A = {0}^{*}$.
    \begin{align*}
        A' & = A' + {0}^{*} = A' + (A + (-A))   \\
           & = (A' + A) + (-A) = {0}^{*} + (-A) \\
           & = -A.
    \end{align*}

    Hence there exists unique Dedekind cut $A'$ of $\mathbb{Q}$ such that $A + A' = A' + A = {0}^{*}$.
    \begin{align*}
         & A + (-A) = (-A) + A = {0}^{*}             \\
         & (-(-A)) + (-A) = (-A) + (-(-A)) = {0}^{*}
    \end{align*}

    Thus, $A = -(-A)$.
\end{proof}

\begin{theorem}
    In $\mathbb{R}$, addition is compatible with $\subseteq$
    \[
        \forall A, B, C\in\mathbb{R}(A\subseteq B \rightarrow A + C\subseteq B + C).
    \]
\end{theorem}

\begin{proof}
    Let $A, B, C$ be Dedekind cuts of $\mathbb{Q}$ such that $A\subseteq B$

    If $A = B$ then $A + C = B + C$.

    Otherwise, $A\subset B$, then there exists $b\in B$ such that $b\notin A$. Hence, for all $c\in C$, $b + c\notin A + C$, then $A + C\ne B + C$.

    Let $a\in A, c\in C$, then $a + c\in A + C$. Since $A\subset B$, then $a\in B$, so $a + c\in B + C$. Therefore $A + C\subseteq B + C$.

    So $A + C\subset B + C$.

    Thus $A + C\subseteq B + C$, equality holds if and only if $A = B$.
\end{proof}

\begin{theorem}\label{theorem:chapter1:negation-and-sign}
    Let $A$ be a Dedekind cut of $\mathbb{Q}$, then
    \[
        A\subset {0}^{*} \Longleftrightarrow -A\supset {0}^{*}.
    \]
\end{theorem}

\begin{proof}
    $(\Rightarrow)$ $A\subset {0}^{*}\Longrightarrow -A\supset {0}^{*}$.

    If $-A\subseteq {0}^{*}$, then
    \[
        A + (-A) \subset {0}^{*} + {0}^{*} = {0}^{*}
    \]

    which conflicts with $A + (-A) = {0}^{*}$. So $-A\supset {0}^{*}$.
    \bigskip

    $(\Leftarrow)$ $-A\supset {0}^{*}\Longrightarrow A\subset {0}^{*}$.

    If $A\supseteq {0}^{*}$, then
    \[
        A + (-A) \supset {0}^{*} + {0}^{*} = {0}^{*}
    \]

    which conflicts with $A + (-A) = {0}^{*}$. So $A\subset {0}^{*}$.
\end{proof}

\subsection{Multiplication}

\begin{theorem}[Multiplication]\label{theorem:chapter1:multiplication}
    Let $A, B$ be Dedekind cuts of $\mathbb{Q}$. The set $A\cdot B$ is defined as the following.

    If $A\supseteq{0}^{*}$ and $B\supseteq{0}^{*}$
    \[
        A\cdot B = \{ a\cdot b : a\in A\wedge a\ge 0 \wedge b\in B\wedge b\ge 0 \} \cup \mathbb{Q}^{-}.
    \]

    If $A\subseteq{0}^{*}$ and $B\subseteq{0}^{*}$
    \[
        A\cdot B = (-A)\cdot (-B).
    \]

    If $A\subseteq{0}^{*}$ and $B\supseteq{0}^{*}$
    \[
        A\cdot B = -\left((-A)\cdot B\right).
    \]

    If $A\supseteq{0}^{*}$ and $B\subseteq{0}^{*}$
    \[
        A\cdot B = -\left(A\cdot (-B)\right).
    \]

    $A\cdot B$ is also a Dedekind cut.
\end{theorem}

\begin{proof}
    There are four cases.

    \begin{enumerate}[label={\textbf{Case \arabic*.}},itemindent={0.5cm}]
        \item $A\supseteq {0}^{*}\land B\supseteq {0}^{*}$.

              Let's consider the following three sub-cases.
              \begin{enumerate}
                  \item $A = {0}^{*}$.

                        Since $A = {0}^{*}$, then $\nexists a\in A$ such that $a\ge 0$. Hence
                        \[
                            A\cdot B = \{ a\cdot b: a\in A\land a\ge 0\land b\in B\land b\ge 0 \} \cup\mathbb{Q}_{-} = \varnothing\cup\mathbb{Q}_{-} = \mathbb{Q}_{-} = {0}^{*}.
                        \]
                  \item $B = {0}^{*}$.

                        Since $B = {0}^{*}$, then $\nexists b\in B$ such that $b\ge 0$. Hence
                        \[
                            A\cdot B = \{ a\cdot b: a\in A\land a\ge 0\land b\in B\land b\ge 0 \} \cup\mathbb{Q}_{-} = \varnothing\cup\mathbb{Q}_{-} = \mathbb{Q}_{-} = {0}^{*}.
                        \]
                  \item $A\supset{0}^{*}$ and $B\supset{0}^{*}$.
                        \begin{enumerate}[label = (\roman*)]
                            \item Since $A\cdot B$ is a superset of $\mathbb{Q}^{-}$, then $A\cdot B$ is not empty.
                            \item Let $a_{0}$ be an upper bound of $A$, $b_{0}$ be an upper bound of $B$.

                                  Since $A\supset{0}^{*}$ and $B\supset{0}^{*}$, then $a_{0}\ge 0$ and $b_{0}\ge 0$.

                                  Then for any non-negative elements $a$ and $b$ of $A$ and $B$, $a\cdot b \le a_{0}\cdot b_{0}$.

                                  Hence $a_{0}\cdot b_{0}$ is an upper bound of $A\cdot B$, which implies that $A\cdot B\ne\mathbb{Q}$.
                            \item Let $c$ be an arbitrary element of $A\cdot B$.

                                  If $c$ is negative or zero, then there exists an element which is greater than $c$, since $A\supset {0}^{*}$ and $B\supset {0}^{*}$ (zero is not their greatest element).

                                  Otherwise, $c$ is positive, then there exists $a\in A$ and $a > 0$, $b\in B$ and $b > 0$ such that $a\cdot b = c$. Due to (DC3), there exists $a_{0} > a > 0$ and $a_{0}\in A$, $b_{0} > b > 0$ and $b_{0}\in B$.

                                  Furthermore, $a_{0}\cdot b_{0} > a\cdot b$ and $a_{0}\cdot b_{0}\in A\cdot B$ according to the definition of $A\cdot B$.

                                  So $A\cdot B$ has no greatest element.
                            \item Let $c$ be an arbitrary element of $A\cdot B$.

                                  Let $d$ be a rational number such that $d < c$.

                                  If $d$ is non-positive, then $d\in A\cdot B$, since $A\cdot B$ contains $0$ and is a superset of $\mathbb{Q}^{-}$.

                                  Otherwise, $d$ is positive, then $c$ is also positive. Since $c$ is positive, there exists $a\in A$ and $a > 0$, $b\in B$ and $b > 0$ such that $c = a\cdot b$.
                                  \[
                                      d = c - (c - d) = a\cdot b - (c - d) = a\cdot\left(b - \frac{c - d}{a}\right)
                                  \]
                                  Since $a\in A$ and $a > 0$, $b - \dfrac{c - d}{a}\in B$ (due to (DC4)) and $0 < b - \dfrac{c - d}{a} < b$, then $d \in A\cdot B$.

                                  Hence $A\cdot B$ is downward closed.
                        \end{enumerate}
              \end{enumerate}
        \item $A\subseteq {0}^{*}, B\subseteq {0}^{*}$.

              Then $-A\supseteq {0}^{*}$, $-B\supseteq {0}^{*}$. According to 1st case, $(-A)\cdot (-B)$ is a Dedekind cut.
        \item $A\subseteq {0}^{*}, B\supseteq {0}^{*}$.

              Then $-A\supseteq {0}^{*}$. According to 1st case, $(-A)\cdot B$ is a Dedekind cut. So $-((-A)\cdot B)$ is a Dedekind cut.
        \item $A\supseteq {0}^{*}, B\subseteq {0}^{*}$.

              Then $-B\supseteq {0}^{*}$. According to 1st case, $A\cdot (-B)$ is a Dedekind cut. So $-(A\cdot (-B))$ is a Dedekind cut.
    \end{enumerate}

    Thus, $A\cdot B$ is a Dedekind cut.
\end{proof}

\begin{theorem}\label{theorem:chapter1:multiplication-and-negation}
    Let $A, B$ be Dedekind cuts. Then
    \[
        \begin{cases}
            (-A)\cdot B = A\cdot (-B) = -(A\cdot B), \\
            (-A)\cdot (-B) = A\cdot B.
        \end{cases}
    \]
\end{theorem}

\begin{proof}
    \noindent\textbf{Step 1. Prove that $A\cdot (-B) = (-A)\cdot B = -A\cdot B$.}

    \begin{enumerate}[label={\textbf{Case \arabic*.}},itemsep=0pt,itemindent=0.5cm]
        \item $A\supseteq {0}^{*}, B\supseteq {0}^{*}$.
              \begin{align*}
                  A\cdot (-B) & = -A\cdot (-(-B)) = -A\cdot B, \\
                  (-A)\cdot B & = -(-(-A))\cdot B = -A\cdot B.
              \end{align*}
        \item $A\supseteq {0}^{*}, B\subseteq {0}^{*}$.
              \begin{align*}
                  A\cdot (-B) & = (-A)\cdot (-(-B)) = (-A)\cdot B, \\
                  A\cdot (-B) & = -A\cdot (-(-B)) = -A\cdot B.
              \end{align*}
        \item $A\subseteq {0}^{*}, B\supseteq {0}^{*}$.
              \begin{align*}
                  A\cdot (-B) & = (-A)\cdot (-(-B)) = (-A)\cdot B, \\
                  (-A)\cdot B & = -(-(-A))\cdot B = -A\cdot B.
              \end{align*}
        \item $A\subseteq {0}^{*}, B\subseteq {0}^{*}$.
              \begin{align*}
                  A\cdot (-B) & = -(-A)\cdot (-B) = -A\cdot B, \\
                  (-A)\cdot B & = -(-A)\cdot (-B) = -A\cdot B.
              \end{align*}
    \end{enumerate}

    \noindent\textbf{Step 2. Prove that $(-A)\cdot (-B) = A\cdot B$}

    \noindent Apply the result in Step 1, $(-A)\cdot (-B) = -A\cdot (-B) = -(-(A\cdot B)) = A\cdot B$.
\end{proof}

\begin{theorem}\label{theorem:chapter1:multiplication-and-order}
    Multiplication in $\mathbb{R}$ is compatible with $\subseteq$.
    \[
        \forall A, B\in\mathbb{R}(A\supseteq {0}^{*}\land B\supseteq {0}^{*}\rightarrow A\cdot B\supseteq {0}^{*}).
    \]
\end{theorem}

\begin{proof}
    \textbf{Case 1. $A = {0}^{*}$ or $B = {0}^{*}$.}

    According to the proof of Theorem~\ref{theorem:chapter1:multiplication} and Theorem~\ref{theorem:chapter1:multiplication-and-negation}, $A\cdot B = {0}^{*}$.
    \bigskip

    \textbf{Case 2. $A\supset {0}^{*}$ and $B\supset {0}^{*}$.}

    Since $A\supset {0}^{*}$, then there exists $a\in A$ such that $a > 0$. $B\supset {0}^{*}$, then there exists $b\in B$ such that $b > 0$.

    According to the definition of multiplication, $a\cdot b\in A\cdot B$. On the other hand, $a\cdot b > 0$. Therefore, $A\cdot B\supset {0}^{*}$.

    Thus $A\cdot B\supseteq {0}^{*}$. Equality holds if and only if $A = {0}^{*}$ or $B\cdot {0}^{*}$.
\end{proof}

The following result follows Theorem~\ref{theorem:chapter1:negation-and-sign} Theorem~\ref{theorem:chapter1:multiplication}, and Theorem~\ref{theorem:chapter1:multiplication-and-order}.

\begin{corollary}
    Let $A, B$ be Dedekind cuts of $\mathbb{Q}$.
    \[
        \begin{split}
            A\supset {0}^{*}\land B\supset {0}^{*}\rightarrow A\cdot B\supset {0}^{*}, \\
            A\supset {0}^{*}\land B\subset {0}^{*}\rightarrow A\cdot B\subset {0}^{*}, \\
            A\subset {0}^{*}\land B\supset {0}^{*}\rightarrow A\cdot B\subset {0}^{*}, \\
            A\subset {0}^{*}\land B\subset {0}^{*}\rightarrow A\cdot B\supset {0}^{*}.
        \end{split}
    \]
\end{corollary}

Let ${1}^{*}$ be the following (rational) Dedekind cut
\[
    \{ x: x\in\mathbb{Q}\land x < 1 \}.
\]

\begin{theorem}
    $\mathbb{R}$ with addition and multiplication is a commutative ring.
\end{theorem}

\begin{proof}
    Let $A, B, C$ be arbitrary Dedekind cuts of $\mathbb{Q}$.
    \begin{enumerate}[label={(F\arabic*)}, start=5]
        \item Multiplication is associative.
              \begin{enumerate}[label={\textbf{Case \arabic*.}},topsep=0pt,itemsep=0pt]
                  \item $A\supseteq {0}^{*}, B\supseteq {0}^{*}, C\supseteq {0}^{*}$, then
                        \begin{align*}
                            (A\cdot B)\cdot C & = \{ (a\cdot b)\cdot c : a\in A\land b\in B\land c\in C\land a\ge 0\land b\ge 0\land c\ge 0 \} \\
                                              & = \{ a\cdot (b\cdot c) : a\in A\land b\in B\land c\in C\land a\ge 0\land b\ge 0\land c\ge 0 \} \\
                                              & = A\cdot (B\cdot C).
                        \end{align*}
                  \item $A\supseteq {0}^{*}, B\supseteq {0}^{*}, C\subseteq {0}^{*}$, then $-C\supseteq {0}^{*}$ and
                        \begin{align*}
                            (A\cdot B)\cdot C & = -\left( (A\cdot B)\cdot (-C) \right)  \\
                                              & = -\left( A\cdot (B \cdot (-C)) \right) \\
                                              & = A\cdot (-(B\cdot (-C)))               \\
                                              & = A\cdot (B\cdot C).
                        \end{align*}
                  \item $A\supseteq {0}^{*}, B\subseteq {0}^{*}, C\supseteq {0}^{*}$, then $-B\supseteq {0}^{*}$ and
                        \begin{align*}
                            (A\cdot B)\cdot C & = (-(A\cdot (-B)))\cdot C \\
                                              & = -((A\cdot (-B))\cdot C) \\
                                              & = -(A\cdot ((-B)\cdot C)) \\
                                              & = A\cdot (-((-B)\cdot C)) \\
                                              & = A\cdot (B\cdot C).
                        \end{align*}
                  \item $A\supseteq {0}^{*}, B\subseteq {0}^{*}, C\subseteq {0}^{*}$, then $-B\supseteq {0}^{*}$, $-C\supseteq {0}^{*}$, and
                        \begin{align*}
                            (A\cdot B)\cdot C & = (-(A\cdot (-B)))\cdot C       \\
                                              & = (-(-(A\cdot (-B))))\cdot (-C) \\
                                              & = (A\cdot (-B))\cdot (-C)       \\
                                              & = A\cdot ((-B)\cdot (-C))       \\
                                              & = A\cdot (B\cdot C).
                        \end{align*}
                  \item $A\subseteq {0}^{*}, B\supseteq {0}^{*}, C\supseteq {0}^{*}$, then $-A\supseteq {0}^{*}$, and
                        \begin{align*}
                            (A\cdot B)\cdot C & = (-((-A)\cdot B))\cdot C \\
                                              & = -(((-A)\cdot B)\cdot C) \\
                                              & = -((-A)\cdot (B\cdot C)) \\
                                              & = A\cdot (B\cdot C).
                        \end{align*}
                  \item $A\subseteq {0}^{*}, B\supseteq {0}^{*}, C\subseteq {0}^{*}$, then $-A\supset {0}^{*}$, $-C\supseteq {0}^{*}$, and
                        \begin{align*}
                            (A\cdot B)\cdot C & = (-(A\cdot B))\cdot (-C) \\
                                              & = ((-A)\cdot B)\cdot (-C) \\
                                              & = (-A)\cdot (B\cdot (-C)) \\
                                              & = (-A)\cdot (-(B\cdot C)) \\
                                              & = A\cdot (B\cdot C).
                        \end{align*}
                  \item $A\subseteq {0}^{*}, B\subseteq {0}^{*}, C\supseteq {0}^{*}$, then $-A\supseteq {0}^{*}, -B\supset {0}^{*}$, and
                        \begin{align*}
                            (A\cdot B)\cdot C & = ((-A)\cdot (-B))\cdot C \\
                                              & = (-A)\cdot ((-B)\cdot C) \\
                                              & = (-A)\cdot (-(B\cdot C)) \\
                                              & = A\cdot (B\cdot C).
                        \end{align*}
                  \item $A\subseteq {0}^{*}, B\subseteq {0}^{*}, C\subseteq {0}^{*}$, then $-A\supseteq {0}^{*}$, $-B\supseteq {0}^{*}$, $-C\supseteq {0}^{*}$, and
                        \begin{align*}
                            (A\cdot B)\cdot C & = -((A\cdot B)\cdot (-C))       \\
                                              & = -(((-A)\cdot (-B))\cdot (-C)) \\
                                              & = -((-A)\cdot ((-B)\cdot (-C))) \\
                                              & = -((-A)\cdot (B\cdot C))       \\
                                              & = A\cdot (B\cdot C).
                        \end{align*}
              \end{enumerate}
        \item Multiplication is distributive over addition.

              We will prove that
              \[
                  \begin{cases}
                      (A + B)\cdot C = A\cdot C + B\cdot C, \\
                      C\cdot (A + B) = C\cdot A + C\cdot B.
                  \end{cases}
              \]

              \begin{enumerate}[label={\textbf{Case \arabic*.}}]
                  \item $A = {0}^{*}$.
                        \[
                            \begin{split}
                                (A + B)\cdot C = B\cdot C = {0}^{*} + B\cdot C = A\cdot C + B\cdot C, \\
                                C\cdot (A + B) = C\cdot B = {0}^{*} + C\cdot B = C\cdot A + C\cdot B.
                            \end{split}
                        \]
                  \item $B = {0}^{*}$.
                        \[
                            \begin{split}
                                (A + B)\cdot C = A\cdot C = A\cdot C + {0}^{*} = A\cdot C + B\cdot C, \\
                                C\cdot (A + B) = C\cdot A = C\cdot A + {0}^{*} = C\cdot A + C\cdot B.
                            \end{split}
                        \]
                  \item $C = {0}^{*}$.
                        \[
                            \begin{split}
                                (A + B)\cdot C = {0}^{*} = {0}^{*} + {0}^{*} = A\cdot C + B\cdot C, \\
                                C\cdot (A + B) = {0}^{*} = {0}^{*} + {0}^{*} = C\cdot A + C\cdot B.
                            \end{split}
                        \]
                  \item $C\supset {0}^{*}, A\supset {0}^{*}, B\supset {0}^{*}$.
                        \begin{align*}
                            (A + B)\cdot C      & = \{ (a + b)\cdot c : a\in A\land b\in B\land c\in C\land a+b\ge 0\land c\ge 0 \} \cup\mathbb{Q}_{-}                                                              \\
                                                & = \{ a\cdot c + b\cdot c : a\in A\land b\in B\land c\in C\land a+b\ge 0\land c\ge 0 \} \cup\mathbb{Q}_{-}                                                         \\
                            \\
                            A\cdot C + B\cdot C & = \{ a\cdot c : a\in A\land c\in C\land a\ge 0\land c\ge 0 \} \cup\mathbb{Q}_{-} + \{ b\cdot c : b\in B\land c\in C\land b\ge 0\land c\ge 0 \} \cup\mathbb{Q}_{-}
                        \end{align*}
              \end{enumerate}

              \textbf{Part 1. Prove that $(A + B)\cdot C = A\cdot C + B\cdot C$.}

              \textbf{Step 1. Prove that $(A + B)\cdot C\subseteq A\cdot C + B\cdot C$.}

              Let $x\in (A + B)\cdot C$.

              If $x\le 0$, then $x\in A\cdot C + B\cdot C$, since both $A\cdot C + B\cdot C$ are supersets of $\{0\}\cup\mathbb{Q}$.

              \bigskip

              Otherwise $x > 0$, then there exists $a\in A, b\in B, c\in C$ such that $a + b > 0, c > 0$ and $(a + b)\cdot c = x$.

              $a\in A, c\in C$ and $c > 0$, then $a\cdot c\in A\cdot C$, no matter whether $a$ is greater than zero or not.

              $b\in B, c\in C$ and $c > 0$, then $b\cdot c\in B\cdot C$, no matter whether $b$ is greater than zero or not.

              So $x\in A\cdot C + B\cdot C$. Hence $(A + B)\cdot C\subseteq A\cdot C + B\cdot C$.

              \textbf{Step 2. Prove that $A\cdot C + B\cdot C\subseteq (A + B)\cdot C$.}

              Let $x\in A\cdot C + B\cdot C$.

              If $x\le 0$, then $x$ is also in $(A + B)\cdot C$ since $(A + B)\cdot C \supset \{0\}\cup\mathbb{Q}_{-}$.

              Otherwise, $x > 0$. I will show that here exists $y\in A\cdot C$ and $z\in B\cdot C$ such that $y > 0$, $z > 0$ and $y + z = x$.

              If $A\cdot C = B\cdot C$, then we choose $y = \dfrac{x}{2}, z = \dfrac{x}{2}$.

              Without loss of generality, suppose that $A$ is a proper subset of $B$.

              \begin{itemize}
                  \item $x\in A\cdot C$. Choose $y = \dfrac{x}{2}$ and $z = \dfrac{x}{2}$.
                  \item $x\notin A\cdot C\land x\in B\cdot C$.
                        Choose an element $y\in A\cdot C$ such that $y > 0$.

                        $x - y < x$. According to (DC4), $x - y\in B\cdot C$. Since $x\notin A\cdot C$ and $y\in A\cdot C$, then $x - y > 0$.

                        We choose $z = x - y$.
                  \item $x\notin B\cdot C$. Then $x$ is greater than any elements of $A\cdot C$ and $B\cdot C$. According to the definition of addition, there exists $y\in A\cdot C$ and $z\in B\cdot C$ such that $y + z = x$.

                        Since $x > y$ and $x > z$, then $y > 0$ and $z > 0$.
              \end{itemize}

              Now, for every positive rational number $x\in A\cdot C + B\cdot C$, there exists positive rational numbers $y\in A\cdot C$ and $z\in B\cdot C$ such that $y + z = x$.

              According to the definition of multiplication
              \begin{itemize}
                  \item $y\in A\cdot C$ and $y > 0$, then there exists $a\in A$, $c_{1}\in C$ such that $a > 0, c_{1} > 0$ and $a\cdot c_{1} = y$.
                  \item $z\in B\cdot C$ and $z > 0$, then there exists $b\in B$, $c_{2}\in C$ such that $b > 0, c_{2} > 0$ and $b\cdot c_{2} = z$.
              \end{itemize}

              (This choice of $c$ is from Hayden\footnote{\url{https://math.stackexchange.com/questions/1205640/proof-that-real-multiplication-distributes-over-addition-using-dedekind-cuts}} on MathOverflow) Let $c = \dfrac{a\cdot c_{1} + b\cdot c_{2}}{a + b}$. $c$ is between $c_{1}$ and $c_{2}$ so $c\in C$ (according to (DC4)).
              \begin{align*}
                  (a + b)\cdot c & = (a + b)\cdot\dfrac{a\cdot c_{1} + b\cdot c_{2}}{a + b} \\
                                 & = a\cdot c_{1} + b\cdot c_{2}                            \\
                                 & = y + z                                                  \\
                                 & = x.
              \end{align*}

              So $x\in (A + B)\cdot C$. Hence $A\cdot C + B\cdot C\subseteq (A + B)\cdot C$.

              Thus, $(A + B)\cdot C = A\cdot C + B\cdot C$.

              \bigskip

              \textbf{Part 2. $C\cdot (A + B) = C\cdot A + C\cdot B$}
              \begin{align*}
                  C\cdot (A + B) & = \{ c\cdot (a + b) : c\ge 0\land c\in C\land a\in A\land b\in B\land (a+b)\ge 0 \}\cup\mathbb{Q}_{-} \\
                                 & = \{ (a + b)\cdot c : c\ge 0\land c\in C\land a\in A\land b\in B\land (a+b)\ge 0 \}\cup\mathbb{Q}_{-} \\
                                 & = (A + B)\cdot C.
              \end{align*}
              On the other hand
              \begin{align*}
                  C\cdot A + C\cdot B & = \{ c\cdot a : c\in C\land c\ge 0\land a\in A\land a\ge 0 \}\cup\mathbb{Q}_{-} + \{ c\cdot b :  c\in C\land c\ge 0\land b\in B\land b\ge 0 \}\cup\mathbb{Q}_{-} \\
                                      & = \{ a\cdot c : c\in C\land c\ge 0\land a\in A\land a\ge 0 \}\cup\mathbb{Q}_{-} + \{ b\cdot c :  c\in C\land c\ge 0\land b\in B\land b\ge 0 \}\cup\mathbb{Q}_{-} \\
                                      & = A\cdot C + B\cdot C.
              \end{align*}
              According to step 1, $(A + B)\cdot C = A\cdot C + B\cdot C$.

              Thus, $C\cdot (A + B) = C\cdot A + C\cdot B$.

              \textbf{Case 5.} $C\supset {0}^{*}, A\subset {0}^{*}, B\subset {0}^{*}$.
              \begin{align*}
                  ((-B) + (-A)) + (A + B) & = ((-B) + ((-A) + A)) + B \\
                                          & = ((-B) + {0}^{*}) + B    \\
                                          & = (-B) + B                \\
                                          & = {0}^{*},                \\
                  (A + B) + ((-B) + (-A)) & = (A + (B + (-B))) + (-A) \\
                                          & = (A + {0}^{*}) + (-A)    \\
                                          & = A + (-A)                \\
                                          & = {0}^{*}.
              \end{align*}

              On the other hand, the addition over Dedekind cuts of $\mathbb{Q}$ is commutative, it follows that $(-B) + (-A) = (-A) + (-B)$. So $-(A + B) = (-A) + (-B)$.

              $A\subset {0}^{*}$ and $B\subset {0}^{*}$ implies that $-A\supset {0}^{*}$ and $-B\supset {0}^{*}$. Apply Case 4 and Theorem~\ref{theorem:chapter1:multiplication-and-negation}
              \begin{align*}
                  (A + B)\cdot C & = -((-A) + (-B))\cdot C           \\
                                 & = -((-A)\cdot C + (-B)\cdot C)    \\
                                 & = (-(-A)\cdot C) + (-(-B)\cdot C) \\
                                 & = A\cdot C + B\cdot C,            \\
                  C\cdot (A + B) & = -C\cdot ((-A) + (-B))           \\
                                 & = -(C\cdot (-A) + C\cdot (-B))    \\
                                 & = (-C\cdot (-A)) + (-C\cdot (-B)) \\
                                 & = C\cdot A + C\cdot B
              \end{align*}

              \textbf{Case 6.} $C\supset {0}^{*}, A\supset {0}^{*}, B\subset {0}^{*}$.

              \textbf{Case 6.1.} $A + B = {0}^{*}$.
              \[
                  \begin{split}
                      (A + B)\cdot C = {0}^{*} = A\cdot C + (-A)\cdot C = A\cdot C + B\cdot C, \\
                      C\cdot (A + B) = {0}^{*} = C\cdot A + C\cdot (-A) = C\cdot A + C\cdot B.
                  \end{split}
              \]
              \textbf{Case 6.2.} $A + B > {0}^{*}$.

              I apply Case 4 and Theorem~\ref{theorem:chapter1:multiplication-and-negation}
              {\allowdisplaybreaks{}
                  \begin{align*}
                      (A + B)\cdot C + (-B)\cdot C   & = ((A + B) + (-B))\cdot C   \\
                                                     & = (A + (B + (-B)))\cdot C   \\
                                                     & = A\cdot C,                 \\
                      \Longrightarrow (A + B)\cdot C & = A\cdot C - (-B)\cdot C    \\
                                                     & = A\cdot C + (-(-B)\cdot C) \\
                                                     & = A\cdot C + B\cdot C.      \\
                      \bigskip
                      C\cdot (A + B) + C\cdot (-B)   & = C\cdot ((A + B) + (-B))   \\
                                                     & = C\cdot (A + (B + (-B)))   \\
                                                     & = C\cdot A,                 \\
                      \Longrightarrow C\cdot (A + B) & = C\cdot A - C\cdot (-B)    \\
                                                     & = C\cdot A + C\cdot B.
                  \end{align*}}

              \textbf{Case 6.3.} If $A + B < {0}^{*}$.
                  {\allowdisplaybreaks{}
                      \begin{align*}
                          (A + B)\cdot C & = -((-A) + (-B))\cdot C                                                                                   \\
                                         & = -((-A)\cdot C + (-B)\cdot C) & \quad\text{(Case 6.2)}                                                   \\
                                         & = A\cdot C + B\cdot C,         & \quad\text{(Theorem~\ref{theorem:chapter1:multiplication-and-negation})} \\
                          C\cdot (A + B) & = -C\cdot ((-A) + (-B))                                                                                   \\
                                         & = -(C\cdot (-A) + C\cdot (-B)) & \quad\text{(Case 6.2)}                                                   \\
                                         & = C\cdot A + C\cdot B.         & \quad\text{(Theorem~\ref{theorem:chapter1:multiplication-and-negation})}
                      \end{align*}}

              \textbf{Case 7.} $C\supset {0}^{*}, A\subset {0}^{*}, B\supset {0}^{*}$.

              This case is similar to Case 6.

              \textbf{Case 7.1.} $A + B = {0}^{*}$.
              \[
                  \begin{split}
                      (A + B)\cdot C = {0}^{*} = A\cdot C + (-A)\cdot C = A\cdot C + B\cdot C, \\
                      C\cdot (A + B) = {0}^{*} = C\cdot A + C\cdot (-A) = C\cdot A + C\cdot B.
                  \end{split}
              \]
              \textbf{Case 7.2.} $A + B > {0}^{*}$.
              \begin{align*}
                  (-A)\cdot C + (A + B)\cdot C   & = ((-A) + (A + B))\cdot C \\
                                                 & = (((-A) + A) + B)\cdot C \\
                                                 & = B\cdot C                \\
                  \Longrightarrow (A + B)\cdot C & = -(-A)\cdot C + B\cdot C \\
                                                 & = A\cdot C + B\cdot C.    \\
                  \bigskip
                  C\cdot (-A) + C\cdot (A + B)   & = C\cdot ((-A) + (A + B)) \\
                                                 & = C\cdot (((-A) + A) + B) \\
                                                 & = C\cdot B                \\
                  \Longrightarrow C\cdot (A + B) & = -C\cdot (-A) + C\cdot B \\
                                                 & = C\cdot A + C\cdot B.
              \end{align*}

              \textbf{Case 7.3.} $A + B < {0}^{*}$.
              \begin{align*}
                  (A + B)\cdot C & = -((-A) + (-B))\cdot C                                                                                    \\
                                 & = -((-A)\cdot C + (-B)\cdot C) & \quad\text{(Case 7.2)}                                                    \\
                                 & = A\cdot C + B\cdot C,         & \quad\text{(Theorem~\ref{theorem:chapter1:multiplication-and-negation})}  \\
                  C\cdot (A + B) & = -C\cdot ((-A) + (-B))                                                                                    \\
                                 & = -(C\cdot (-A) + C\cdot (-B)) & \quad\text{(Case 7.2)}                                                    \\
                                 & = C\cdot A + C\cdot B.         & \quad\text{(Theorem~\ref{theorem:chapter1:multiplication-and-negation})}.
              \end{align*}

              \textbf{Case 8.} $C\subset {0}^{*}$.

              Apply Case 4, 5, 6, 7 and Theorem~\ref{theorem:chapter1:multiplication-and-negation}
              \[
                  \begin{split}
                      (A + B)\cdot C = -(A + B)\cdot (-C) = -(A\cdot (-C) + B\cdot (-C)) = A\cdot C + B\cdot C, \\
                      C\cdot (A + B) = -(-C)\cdot (A + B) = -((-C)\cdot A + (-C)\cdot B) = C\cdot A + C\cdot B.
                  \end{split}
              \]
        \item Multiplication has identity element.

              I will prove that
              \[
                  A\cdot {1}^{*} = {1}^{*}\cdot A = A.
              \]

              \begin{enumerate}[label={\textbf{Case \arabic*.}}]
                  \item $A = {0}^{*}$.

                        According to the proof of Theorem~\ref{theorem:chapter1:multiplication}
                        \[
                            A\cdot {1}^{*} = {0}^{*}\cdot {1}^{*} = {0}^{*} = {1}^{*}\cdot {0}^{*} = {1}^{*}\cdot A.
                        \]

                  \item $A > {0}^{*}$.
                        \begin{align*}
                            A\cdot {1}^{*} & = \{ a\cdot b : a\in A\land a\ge 0\land b\ge 0\land b < 1 \}\cup\mathbb{Q}_{-} \\
                                           & = \{ b\cdot a : a\in A\land a\ge 0\land b\ge 0\land b < 1 \}\cup\mathbb{Q}_{-} \\
                                           & = {1}^{*}\cdot A.
                        \end{align*}
                        Let $x\in A$.

                        If $x\le 0$, then $x\in A\cdot {1}^{*}$.

                        Otherwise, $x > 0$. According to (DC3), there exists $y\in A$ such that $x < y$. Then
                        \[
                            x = \underbrace{y}_{>0, \in A}\cdot\underbrace{\dfrac{x}{y}}_{>0, <1}
                        \]
                        Hence $x\in A\cdot {1}^{*}$, so $A\subseteq A\cdot {1}^{*}$.

                        \bigskip

                        Let $x\in A\cdot {1}^{*}$.

                        If $x\le 0$, then $x\in A$.

                        Otherwise, $x > 0$. Then there exists $a\in A$ and $b\in {1}^{*}$ such that $a > 0, b > 0$ and $x = a\cdot b$.

                        Since $0 < b < 1$, then $a\cdot b < a$.

                        According to (DC4), $x = a\cdot b\in A$, so $x\in A$. Therefore $A\cdot {1}^{*}\subseteq A$.

                        \bigskip

                        Thus, $A\cdot {1}^{*} = {1}^{*}\cdot A = A$.

                  \item $A < 0^{*}$

                        Apply Case 2 and Theorem~\ref{theorem:chapter1:multiplication-and-negation}
                        \[
                            \begin{split}
                                A\cdot {1}^{*} = -(-A)\cdot {1}^{*} = -(-A) = A, \\
                                {1}^{*}\cdot A = -{1}^{*}\cdot (-A) = -(-A) = A.
                            \end{split}
                        \]
              \end{enumerate}
        \item Multiplication is commutative.

              I will prove that $A\cdot B = B\cdot A$.

              If $A = {0}^{*}$ or $B^{*} = {0}^{*}$, then according to the proof of Theorem~\ref{theorem:chapter1:multiplication}, $A\cdot B = {0}^{*} = B\cdot A$.

              Otherwise, let's consider the following four cases.

              \textbf{Case 1.} $A > {0}^{*}, B > {0}^{*}$.
              \begin{align*}
                  A\cdot B & = \{ a\cdot b : a\in A\land b\in B\land a\ge 0\land b\ge 0 \}\cup\mathbb{Q}_{-} \\
                           & = \{ b\cdot a : a\in A\land b\in B\land a\ge 0\land b\ge 0 \}\cup\mathbb{Q}_{-} \\
                           & = B\cdot A.
              \end{align*}

              In the following cases, I use Case 1 and Theorem~\ref{theorem:chapter1:multiplication-and-negation}.

              \textbf{Case 2.} $A > {0}^{*}, B < {0}^{*}$.
              \[
                  A\cdot B = -A\cdot (-B) = -(-B)\cdot A = B\cdot A.
              \]

              \textbf{Case 3.} $A < {0}^{*}, B > {0}^{*}$.
              \[
                  A\cdot B = -(-A)\cdot B = -B\cdot (-A) = B\cdot A.
              \]

              \textbf{Case 4.} $A < {0}^{*}, B < {0}^{*}$.
              \[
                  A\cdot B = (-A)\cdot (-B) = (-B)\cdot (-A) = B\cdot A.
              \]

              Thus, multiplication in $\mathbb{R}$ is commutative.
    \end{enumerate}
\end{proof}

\begin{theorem}\label{theorem:chapter1:division}
    Let $A, B$ be Dedekind cuts such that $B\ne {0}^{*}$. Define the set $A/B$ as the following

    if $A\ge {0}^{*}, B > {0}^{*}$
    \[
        A/B = \left\{ \frac{a}{b} : a\in A\land a\ge 0\land b\in\mathbb{Q}\setminus B \right\}\cup\mathbb{Q}_{-},
    \]

    if $A\le {0}^{*}, B > {0}^{*}$
    \[
        A/B = -(-A)/B,
    \]

    if $A\ge {0}^{*}, B < {0}^{*}$
    \[
        A/B = -A/(-B),
    \]

    if $A\le {0}^{*}, B < {0}^{*}$
    \[
        A/B = (-A)/(-B).
    \]

    Then $A/B$ is a Dedekind cut of $\mathbb{Q}$.
\end{theorem}

\begin{proof}
    It suffices to prove 1st case: $A\supseteq {0}^{*}$ and $B\supset {0}^{*}$.

    \begin{enumerate}[label={\textbf{Case \arabic*.}},itemindent=0.4cm]
        \item $A\ge {0}^{*}, B > {0}^{*}$.
              $A/B$ is a superset of $\mathbb{Q}_{-}$, so $A/B$ is not empty. (DC1) is satisfied.
              \bigskip

              Let $c\in\mathbb{Q}\setminus A$, $d\in B$ such that $d > 0$. Then for all $a\in A, b\in\mathbb{Q}\setminus B$ such that $a \ge 0, b > 0$, we obtain that
              \[
                  \frac{a}{b} < \frac{c}{d}
              \]
              which means $A/B$ is bounded above. (DC2) is satisfied.
              \bigskip

              Let $x\in A/B$. If $x\le 0$, then there exists an element $y\in A/B$ such that $x < y$. For example, zero.

              Otherwise, $x > 0$, then there exists $a\in A, b\in\mathbb{Q}\setminus B$ such that $a > 0, b > 0$ and $x = \dfrac{a}{b}$.

              Since $A$ is a Dedekind cut, then there exists $a_{0}\in A$ such that $a < a_{0}$. So $\dfrac{a}{b} < \dfrac{a_{0}}{b}$.

              According to the definition of $A/B$, $\dfrac{a_{0}}{b}\in A/B$. So $A/B$ does not have a greatest element. (DC3) is satisfied.
              \bigskip

              Let $x\in A/B$, $y$ be a rational number such that $y < x$.

              If $x\le 0$, then $y < 0$. Since $A/B$ is a superset of $\mathbb{Q}_{-}$, then $y\in A/B$.

              Otherwise, $x > 0$, then there exists $a\in A, b\in\mathbb{Q}\setminus B$ such that $a > 0, b > 0$ and $x = \dfrac{a}{b}$.

              Since $y < \dfrac{a}{b}$, then $b\cdot y < a$. According to (DC4), $b\cdot y\in A$.

              If $y\ge 0$, then $y = \dfrac{b\cdot y}{b} \in A/B$. Otherwise, $y < 0$, then $y\in A/B$ since $A/B$ is a superset of $\mathbb{Q}_{-}$.
        \item $A\le {0}^{*}, B > {0}^{*}$.

              According to Case 1, $(-A)/B$ is a Dedekind cut, so $-(-A)/B$ is a Dedekind cut.
        \item $A\ge {0}^{*}, B < {0}^{*}$.

              According to Case 1, $A/(-B)$ is a Dedekind cut, so $-A/(-B)$ is a Dedekind cut.
        \item $A\le {0}^{*}, B < {0}^{*}$.

              According to Case 1, $(-A)/(-B)$ is a Dedekind cut.
    \end{enumerate}

    Thus, $A/B$ is a Dedekind cut.
\end{proof}

(F1) (F2) (F3) (F4) (F5) (F6) (F7) (F8) make $\mathbb{R}$ a commutative ring. Together with the following, $\mathbb{R}$ is a field.

\begin{theorem}\label{theorem:chapter1:multiplicative-inverse}
    \begin{enumerate}[label={(F\arabic*)},start=9]
        \item Each non-zero element of $\mathbb{R}$ has a multiplicative inverse.
    \end{enumerate}
\end{theorem}

\begin{proof}
    Let $A$ be a non-zero element of $\mathbb{R}$. I will show that
    \[
        A\cdot ({1}^{*}/A) = ({1}^{*}/A)\cdot A = {1}^{*}.
    \]

    Since multiplication over $\mathbb{R}$ is commutative, then $A\cdot ({1}^{*}/A) = ({1}^{*}/A)\cdot A$.

    \textbf{Case 1.} $A > {0}^{*}$.
    \[
        \begin{split}
            A\cdot ({1}^{*}/A) & = \left\{ a\cdot\frac{b}{a_{0}} : a\in A\land 0\le b\land b < 1\land a_{0}\in\mathbb{Q}\setminus A\land a\ge 0\land a_{0} > 0 \right\}\cup\mathbb{Q}_{-} \\
            {1}^{*} & = \{ b: b\in\mathbb{Q}\land b < 1 \}
        \end{split}
    \]

    Let $x\in A\cdot ({1}^{*}/A)$.

    If $x\le 0$, then $x\in {1}^{*}$ as well.

    Otherwise, $x > 0$, then there exists $a\in A, b\in {1}^{*}, a_{0}\in \mathbb{Q}\setminus A$ such that $a > 0, b > 0, a_{0} > 0$ and $x = a\cdot\dfrac{b}{a_{0}}$.

    Since $a < a_{0}$ (because $a\in A$ and $a_{0}\in\mathbb{Q}\setminus A$), then
    \[
        x = a\cdot\dfrac{b}{a_{0}} = b\cdot\dfrac{a}{a_{0}} < b < 1.
    \]

    So $x\in {1}^{*}$. Hence $A\cdot ({1}^{*}/A)\subseteq {1}^{*}$.

    \bigskip

    Let $x\in {1}^{*}$.

    If $x\le 0$, then $x\in A\cdot ({1}^{*}/A)$ as well.

    Otherwise, $x > 0$. Then $0 < x < 1$.

    I will show that there exists an integer $n$ such that $x^{n}\in A$ and $x^{n-1}\notin A$.

    Let's consider the following sets:
    \[
        \begin{split}
            S_{1} & = \{ {x}^{n} : n\in\mathbb{Z} \} \cap A, \\
            S_{2} & = \{ {x}^{n} : n\in\mathbb{Z} \} \cap \mathbb{Q}\setminus A, \\
            S & = S_{1}\cup S_{2}
        \end{split}
    \]
    \begin{enumerate}[label={\textbf{Step \arabic*.}},itemindent=1cm]
        \item Prove that $S_{1}$ and $S_{2}$ are not empty.

              Let $y$ be a positive rational number. I will find a natural number $m$ ($m\ge 1$) such that $x^{-m} > y$.

              Due to binomial theorem
              \[
                  x^{-m} = {\left(1 + \left(\frac{1}{x} - 1\right)\right)}^{m} = \sum^{m}_{k=1}\binom{m}{k}{\left(\frac{1}{x} - 1\right)}^{k} \ge 1 + m\cdot\left(\frac{1}{x} - 1\right)
              \]
              So for any natural number $m$ that satisfies $m > \dfrac{y - 1}{\dfrac{1}{x} - 1}$, we will obtain that $x^{-m} > y$.

              Let $a_{1}\in A$ and $a_{2}\in\mathbb{Q}\setminus A$ such that $a_{1} > 0$ and $a_{2} > 0$.

              According to the result that we have just proved, there exists natural numbers $n_{1}$ and $n_{2}$ such that
              \[
                  x^{-{n}_{2}} > a_{2}\qquad\text{and}\qquad x^{-{n}_{1}} > \frac{1}{a_{1}}
              \]

              Hence $S_{1}$ and $S_{2}$ are not empty.

        \item Prove that there exists an integer $n$ such that $x^{n}\in A$ and $x^{n}\notin A$.

              Since $0 < x < 1$, then ${x}^{n} < {x}^{n-1}$.

              $\{ {x}^{n} : n\in\mathbb{Z}\land {x}^{n}\in A \}$ is bounded above, then $\{ n : n\in\mathbb{Z}\land {x}^{n}\in A \}$ is bounded below. According to the well-ordering principle, $\{ n : n\in\mathbb{Z}\land {x}^{n}\in A \}$ contains a least element.

              Hence, there exists an integer $n$ such that ${x}^{n}\in A$ and ${x}^{n-1}\notin A$.
    \end{enumerate}

    \[
        x = {x^{n}} / {x^{n-1}}.
    \]

    According to (DC3), there exists $x_{0}\in A$ such that $x^{n} < x_{0}$.
    \[
        x = x_{0}\cdot\frac{x^{n}/x_{0}}{x^{n-1}}.
    \]

    Since $x_{0}\in A, x^{n-1}\in\mathbb{Q}\setminus A, 0 < x^{n}/x_{0} < 1$, then $x\in A\cdot ({1}^{*}/A)$.

    Hence ${1}^{*}\subseteq A\cdot ({1}^{*}/A)$.

    \bigskip
    \textbf{Case 2.} $A < {0}^{*}$.

    According to the definition of division
    \[
        {1}^{*}/A = -{1}^{*}/(-A)
    \]

    Apply Theorem~\ref{theorem:chapter1:multiplication-and-negation} and Case 1
    \[
        A\cdot ({1}^{*}/A) = (-(-A))\cdot (-{1}^{*}/(-A)) = (-A)\cdot ({1}^{*}/(-A)) = {1}^{*}
    \]

    Thus, $A\cdot ({1}^{*}/A) = ({1}^{*}/A)\cdot A = {1}^{*}$.
\end{proof}

\begin{theorem}
    Let $A$ be a Dedekind cut of $\mathbb{Q}$ and $A\ne {0}^{*}$, then
    \[
        {1}^{*}/({1}^{*}/A) = A.
    \]
\end{theorem}

\begin{proof}
    Theorem~\ref{theorem:chapter1:multiplicative-inverse} asserts that there exists a Dedekind cut $B$ such that $A\cdot B = B\cdot A = {1}^{*}$.

    Suppose that $B$ and $C$ are two Dedekind cuts such that
    \[
        \begin{split}
            A\cdot B = B\cdot A = {1}^{*}, \\
            A\cdot C = C\cdot A = {1}^{*}.
        \end{split}
    \]

    Apply (F5) and (F7)
    \[
        B = B\cdot {1}^{*} = B\cdot (A\cdot C) = (B\cdot A)\cdot C = {1}^{*}\cdot C = C.
    \]

    Hence there exists a unique Dedekind cut $B$ such that $A\cdot B = B\cdot A = {1}^{*}$.

    Since this uniqueness, we call $B$ \textit{the multiplicative inverse} of $A$.

    According to Theorem~\ref{theorem:chapter1:multiplicative-inverse}
    \[
        A\cdot ({1}^{*}/A) = ({1}^{*}/A)\cdot A = {1}^{*}.
    \]

    Therefore, $A$ is the multiplicative inverse of ${1}^{*}/A$.

    Thus ${1}^{*}/({1}^{*}/A) = A$.
\end{proof}

\begin{theorem}
    $\mathbb{R}$ is a field and it is characteristic zero.
\end{theorem}

\begin{proof}
    (F1) (F2) (F3) (F4) (F5) (F6) (F7) (F8) (F9) together assert that $\mathbb{R}$ is a field.

    Since ${0}^{*} < {1}^{*}$, then for all positive integer $n$
    \[
        \underbrace{{1}^{*} + \cdots + {1}^{*}}_{n} > {0}^{*}.
    \]
    Therefore, $\mathbb{R}$ is characteristic zero.
\end{proof}

\subsection{Least-upper-bound property}

So far, we prove that the set of Dedekind cuts with addition and multiplication is a totally ordered field. But those properties make no distinction between $\mathbb{R}$ and $\mathbb{Q}$. The property that makes distinction is the least-upper-bound property.

\begin{theorem}
    Let $\mathcal{D}$ be a set of Dedekind cuts of $\mathbb{Q}$ and $\mathcal{D}$ has an upper bound.

    Then $\mathcal{D}$ has a least upper bound.
\end{theorem}

\begin{proof}
    Let $S = \bigcup\limits_{A\in\mathcal{D}} A$. $S$ consists of rational numbers.

    \textbf{Step 1. Prove that $S$ is a Dedekind cut of $\mathbb{Q}$.}

    $S$ is non-empty by definition. (DC1) is satisfied.

    $S$ is bounded above. (DC2) is satisfied.

    Let $a\in S$. Then there exists $A\in\mathcal{D}$ such that $a\in A$. Since $A$ has no greatest element, then there exists $x_{0}\in A$ such that $x < x_{0}$. $x_{0}\in A$ then $x_{0}\in S$. (DC3) is satisfied.

    Let $a\in S$, and $b$ be a rational number such that $b < a$. Since $a\in S$, then there exists $A\in\mathcal{D}$ such that $a\in A$. Since $b < a$, then $b\in A$, so $b\in S$ also. (DC4) is satisfied.

    Hence $S$ is a Dedekind cut of $\mathbb{Q}$.
    \bigskip

    \textbf{Step 2. Prove that $S$ is the least upper bound of $\mathcal{D}$.}

    Let $A\in\mathcal{D}$. Since $A\subseteq S$, then $S$ is an upper bound of $\mathcal{D}$.

    Let $S'$ be an upper bound of $\mathcal{D}$. According to the definition of upper bound, then $\forall A (A\in\mathcal{D}\rightarrow A\subseteq S')$. So $S\subseteq S'$, or equivalently, $S\le S'$. Hence $S$ is the least upper bound of $\mathcal{D}$.

    Thus $\mathcal{D}$ has a least upper bound.
\end{proof}

Now we have proved that $\mathbb{R}$ satisfies all real numbers axioms.

\subsection{Embed $\mathbb{Q}$ into $\mathbb{R}$}

I don't feel so good after proving that $\mathbb{R}$ is a totally ordered field. It is because I haven't showed that the addition and multiplication in $\mathbb{R}$ are ``the same'' as addition and multiplication $\mathbb{Q}$.

\begin{theorem}
    There exists a ring monomorphism $\iota$
    \[
        \begin{split}
            \iota:\quad & \mathbb{Q}\to\mathbb{R}, \\
            & q\mapsto {q}^{*}
        \end{split}
    \]
    which preserves order.
\end{theorem}

\begin{proof}
    We define a mapping $\iota$ as the following:
    \[
        \begin{split}
            \iota:\quad & \mathbb{Q}\to\mathbb{R}, \\
            & q\mapsto \{ x : x\in\mathbb{Q}\land x < q \}.
        \end{split}
    \]

    Let $q_{1}$ and $q_{2}$ be two rational numbers.
    \begin{itemize}
        \item If $q_{1} = q_{2}$, then $\iota(q_{1}) = \iota(q_{2})$.

              Otherwise, $q_{1}\ne q_{2}$. Without loss of generality, suppose that $q_{1} < q_{2}$.

              Since every element of $\iota(q_{1})$ is less than $q_{1}$, then according to (DC4), every element of $\iota(q_{1})$ is an element of $\iota(q_{2})$. Therefore $\iota(q_{1})\subseteq\iota(q_{2})$.

              On the other hand, $q_{1}\notin\iota(q_{1})$ and $q_{1}\in\iota(q_{2})$. Therefore $\iota(q_{1})$ is a proper subset of $\iota(q_{2})$.

              Hence $\iota$ preserves order and $\iota$ is injective.
        \item To prove that $\iota$ is a monomorphism, I will show that
              \[
                  \begin{split}
                      \iota(q_{1} + q_{2}) = \iota(q_{1}) + \iota(q_{2}) \\
                      \iota(q_{1}\cdot q_{2}) = \iota(q_{1})\cdot\iota(q_{2}).
                  \end{split}
              \]
              \begin{itemize}
                  \item Prove that $\iota(q_{1} + q_{2}) = \iota(q_{1}) + \iota(q_{2})$.

                        Let $x\in\iota(q_{1}) + \iota(q_{2})$.

                        According to the definition of addition, there exists $y\in\iota(q_{1})$ and $z\in\iota(q_{2})$ such that $y + z = x$.

                        On the other hand
                        \[
                            x = y + z < q_{1} + q_{2}
                        \]
                        So $x\in\iota(q_{1} + q_{2})$. Therefore, $\iota(q_{1}) + \iota(q_{2})\le\iota(q_{1} + q_{2})$.
                        \bigskip

                        Let $x\in\iota(q_{1} + q_{2})$, then $x < q_{1} + q_{2}$.

                        Choose $y = q_{1} + \dfrac{x - (q_{1} + q_{2})}{2}$ and $z = q_{2} + \dfrac{x - (q_{1} + q_{2})}{2}$, then $y + z = x$.

                        Since $x < q_{1} + q_{2}$, then $y < q_{1}$ and $z < q_{2}$. $y$ and $z$ are also rational numbers, so $y\in\iota(q_{1})$ and $z\in\iota(q_{2})$.

                        Therefore, $\iota(q_{1} + q_{2})\le\iota(q_{1}) + \iota(q_{2})$.

                        Thus $\iota(q_{1}) + \iota(q_{2}) = \iota(q_{1} + q_{2})$.
                  \item Prove that $\iota(-q) = - \iota(q)$.
                        \[
                            \iota(-q) = \iota(0) - \iota(q) = {0}^{*} - \iota(q) = -\iota(q).
                        \]
                  \item Prove that $\iota(q_{1}\cdot q_{2}) = \iota(q_{1})\cdot\iota(q_{2})$.

                        If $q_{1} = 0$ or $q_{2} = 0$, then $\iota(q_{1})\cdot\iota(q_{2}) = {0}^{*} = \iota(0) = \iota(q_{1}\cdot q_{2})$.

                        Otherwise, $q_{1}\ne 0$ and $q_{2}\ne 0$.

                        \textbf{Case 1.} $q_{1} > 0$ and $q_{2} > 0$.
                        \[
                            \begin{split}
                                \iota(q_{1})\cdot\iota(q_{2}) & = \{ a\cdot b : a\ge 0\land a < q_{1}\land b\ge 0\land b < q_{2} \}\cup\mathbb{Q}_{-}, \\
                                \iota(q_{1}\cdot\iota(q_{2})) & = \{ x: x\in\mathbb{Q}\land x < q_{1}\cdot q_{2} \}.
                            \end{split}
                        \]
                        Then $\iota(q_{1})\cdot\iota(q_{2})\subseteq\iota(q_{1}\cdot q_{2})$.

                        Let $x\in\iota(q_{1}\cdot q_{2})$.

                        If $x\le 0$, then $x\in\iota(q_{1})\cdot\iota(q_{2})$.

                        Otherwise, $0 < x < q_{1}\cdot q_{2}$. Choose $y$ such that $\dfrac{x}{q_{1}\cdot q_{2}} < y < 1$, then
                        \[
                            \dfrac{x}{q_{1}q_{2}} = \underbrace{y}_{< 1}\cdot\underbrace{\dfrac{x}{q_{1}\cdot q_{2}\cdot y}}_{< 1}
                        \]
                        Let $a = q_{1}\cdot y$ and $b = q_{2}\cdot\dfrac{x}{q_{1}\cdot q_{2}\cdot y}$, then $a\in\iota(q_{1}), b\in\iota(q_{2})$, and $a\cdot b = x$. This means $x\in\iota(q_{1})\cdot\iota(q_{2})$.

                        So $\iota(q_{1}\cdot q_{2})\subseteq\iota(q_{1})\cdot\iota(q_{2})$.

                        Hence $\iota(q_{1})\cdot\iota(q_{2}) = \iota(q_{1}\cdot q_{2})$.

                        \textbf{Case 2.} $q_{1} < 0$ and $q_{2} < 0$.
                        \begin{align*}
                            \iota(q_{1}\cdot q_{2}) & = \iota((-q_{1})\cdot (-q_{2}))        \\
                                                    & = \iota(-q_{1})\cdot\iota(-q_{2})      \\
                                                    & = (-\iota(q_{1}))\cdot (-\iota(q_{2})) \\
                                                    & = \iota(q_{1})\cdot\iota(q_{2}).
                        \end{align*}
                        \textbf{Case 3.} $q_{1} > 0$ and $q_{2} < 0$.
                        \begin{align*}
                            \iota(q_{1}\cdot q_{2}) & = -\iota(-q_{1}\cdot q_{2})         \\
                                                    & = -\iota(q_{1}\cdot (-q_{2}))       \\
                                                    & = -(\iota(q_{1})\cdot\iota(-q_{2})) \\
                                                    & = \iota(q_{1})\cdot\iota(q_{2}).
                        \end{align*}
                        \textbf{Case 4.} $q_{1} < 0$ and $q_{2} > 0$.
                        \begin{align*}
                            \iota(q_{1}\cdot q_{2}) & = -\iota(-q_{1}\cdot q_{2})         \\
                                                    & = -\iota((-q_{1})\cdot q_{2})       \\
                                                    & = -(\iota(-q_{1})\cdot\iota(q_{2})) \\
                                                    & = \iota(q_{1})\cdot\iota(q_{2}).
                        \end{align*}
                        Thus, $\iota(q_{1}\cdot q_{2}) = \iota(q_{1})\cdot\iota(q_{2})$.
              \end{itemize}
    \end{itemize}

    We have constructed a ring monomorphism $\iota:\mathbb{Q}\to\mathbb{R}$ that preserves order.
\end{proof}

\subsection{Completeness (Dedekind completeness)}

So far, we have successfully constructed a model (by using Dedekind cuts), which satisfies the axioms of the real numbers.

In this subsection, we will show that $\mathbb{R}$ is complete, in the sense that every cut is represented by a real number. To do so, we will use a generalization of Dedekind cut.

\begin{definition}[Dedekind cut (generalization)]
    Let $S$ be a totally ordered set. A Dedekind cut of $S$ is a partition $(L, R)$ ($L\cap R = \varnothing$ and $L\cup R = S$) such that
    \begin{enumerate}[label={(\roman*)}]
        \item $L$ is not empty.
        \item $R$ is not empty (i.e. $L\ne S$).
        \item $L$ does not contain a greatest element.
        \item $L$ is closed downward.
    \end{enumerate}
\end{definition}

\begin{definition}[Dedekind-complete]
    A totally order set $S$ is called Dedekind-complete if every Dedekind cut $A$ of it satisfies: \textit{$S\setminus A$ has a least element.}
\end{definition}

\begin{theorem}
    $\mathbb{R}$ is Dedekind-complete.
\end{theorem}

\begin{proof}
    Let $(L, R)$ be a Dedekind cut of $\mathbb{R}$.

    $L$ is bounded above. According to the least-upper-bound property, $L$ has a least upper bound $\ell$.

    On the other hand, $L$ does not contain a greatest element, then $\ell\in R$.

    Since every element of $R$ is an upper bound of $L$, then $\ell$ is the least element of $R$.

    Therefore, for all Dedekind cut $(L, R)$ of $\mathbb{R}$, $R$ has a least element. Thus, $\mathbb{R}$ is Dedekind-complete.
\end{proof}

The least-upper-bound property is also called the axiom of completeness.

\section{Construction of the real numbers by Cauchy sequences}

Another way to construct the real numbers from the rational numbers is using Cauchy sequences.

In the upcoming \textit{four subsections}, we consider sequences with rational terms only.

\subsection{Definition of Cauchy sequence}

In this subsection and the upcoming three subsections, we work with Cauchy sequences which contain rational numbers only.

\begin{definition}
    A sequence ${\{ a_{n} \}}^{\infty}_{n=1}$ (or just $(a_{n})$) is called a Cauchy sequence if and only if
    \begin{quotation}
        \noindent for every positive rational number $\varepsilon$, there exists a natural number $N$ which is a function of $\varepsilon$ such that for all natural numbers $m, n$ which are greater than $N$, $\left\vert a_{m} - a_{n}\right\vert < \varepsilon$.
    \end{quotation}

    Equivalently
    \[
        (\forall\varepsilon > 0)(\exists N=N(\varepsilon))(\forall m, n > N)\left(\left\vert a_{m} - a_{n} \right\vert < \varepsilon\right).
    \]
\end{definition}

Informally, the differences of any two terms can be arbitrarily small as the indices increase.

Such sequences do exist. For example: $(a_{n})$, where $a_{n} = 0$ for every natural number $n$.

\begin{definition}[Equivalent Cauchy sequences]
    Two Cauchy sequences $(a_{n})$ and $(b_{n})$ are called equivalent if and only if
    \[
        (\forall\varepsilon > 0)(\exists N=N(\varepsilon))(\forall n > N)(\left\vert a_{n} - b_{n} \right\vert < \varepsilon).
    \]
\end{definition}

\begin{theorem}
    The relation in the previous definition is an equivalence relation. In other words, this relation is reflexive, symmetric, transitive.
\end{theorem}

\begin{proof}
    \begin{itemize}
        \item Reflexive.

              A Cauchy sequence $(a_{n})$ is equivalent to itself, since
              \[
                  (\forall\varepsilon > 0)(\forall n > 1)(\left\vert a_{n} - a_{n} \right\vert = 0 < \varepsilon)
              \]
        \item Symmetric.

              Let $(a_{n})$ and $(b_{n})$ be two Cauchy sequences.

              $(a_{n})$ is equivalent to $(b_{n})$ if and only if
              \[
                  \begin{split}
                      & (\forall\varepsilon > 0)(\exists N=N(\varepsilon))(\forall n > N)(\left\vert a_{n} - b_{n} \right\vert < \varepsilon) \\
                      \Leftrightarrow\quad & (\forall\varepsilon > 0)(\exists N=N(\varepsilon))(\forall n > N)(\left\vert b_{n} - a_{n} \right\vert < \varepsilon)
                  \end{split}
              \]

              which implies that $(b_{n})$ is equivalent to $(a_{n})$.
        \item Transitive.

              Assume that among three Cauchy sequences $(a_{n})$, $(b_{n})$, $(c_{n})$, $(a_{n})$ is equivalent to $(b_{n})$ and $(b_{n})$ is equivalent to $(c_{n})$.

              Let $\varepsilon$ be an arbitrary positive rational number.
              \[
                  \begin{split}
                      (\forall\varepsilon > 0)(\exists N_{1}=N_{1}(\varepsilon))(\forall n > N_{1})\left(\left\vert a_{n} - b_{n} \right\vert < \frac{\varepsilon}{2}\right), \\
                      (\forall\varepsilon > 0)(\exists N_{2}=N_{2}(\varepsilon))(\forall n > N_{2})\left(\left\vert b_{n} - c_{n} \right\vert < \frac{\varepsilon}{2}\right).
                  \end{split}
              \]

              Let's choose $N = \max\{ N_{1}, N_{2} \}$, then for all $n > N$
              \[
                  \abs{a_{n} - c_{n}} \le \abs{a_{n} - b_{n}} + \abs{b_{n} - c_{n}} < \frac{\varepsilon}{2} + \frac{\varepsilon}{2} = \varepsilon.
              \]

              Hence
              \[
                  (\forall\varepsilon > 0)(\forall n > N)(\abs{a_{n} - c_{n}} < \varepsilon).
              \]

              Thus, $(a_{n})$ and $(c_{n})$ are equivalent.
    \end{itemize}
\end{proof}

A Cauchy sequence can converge to a rational number or not.

\begin{definition}
    A Cauchy sequence $(a_{n})$ converges to a rational number $a$ if and only if
    \[
        (\forall\varepsilon > 0)(\exists N=N(\varepsilon))(\forall n > N)(\abs{a_{n} - a} < \varepsilon).
    \]

    Then we say $a$ is a limit point of $(a_{n})$.
\end{definition}

\begin{example}
    $(a_{n})$ where $a_{n} = q$, $q$ is a rational number converges to a rational number.
\end{example}

\begin{proof}
    For every rational number $\varepsilon > 0$, let's choose $N = 1$, then $\forall n > N$, $\abs{a_{n} - q} = 0 < \varepsilon$.

    Therefore, $(a_{n})$ converges to $q$.
\end{proof}

\begin{theorem}
    If a Cauchy sequence converges, then it has a unique limit point.
\end{theorem}

\begin{proof}
    Let $(a_{n})$ be a Cauchy sequence.

    Assume that two rational numbers $a$ and $b$ are limit points of $(a_{n})$ and $a\ne b$.

    Pick a positive rational number $\varepsilon$ such that $\abs{a - b}\ge\varepsilon$.
    \begin{itemize}
        \item there exists $N_{1} = N_{1}(\varepsilon)$ such that for all $n > N_{1}$, $\abs{a_{n} - a} < \dfrac{\varepsilon}{2}$,
        \item there exists $N_{2} = N_{2}(\varepsilon)$ such that for all $n > N_{2}$, $\abs{a_{n} - b} < \dfrac{\varepsilon}{2}$.
    \end{itemize}

    Choose $N = \max\{ N_{1}, N_{2} \}$, then for all $n > N$
    \[
        \abs{a - b}\le \abs{a - a_{n}} + \abs{a_{n} - b} < \frac{\varepsilon}{2} + \frac{\varepsilon}{2} = \varepsilon.
    \]

    $\abs{a - b} < \varepsilon$ contradicts $\abs{a - b}\ge\varepsilon$.

    Hence $a = b$.

    Thus if a Cauchy sequence converges, it has a unique limit point.
\end{proof}

\begin{theorem}
    If the Cauchy sequence $(a_{n})$ converges to $a$, then every Cauchy sequence  which is equivalent to $(a_{n})$ also converges to $a$.
\end{theorem}

\begin{proof}
    Let $(a_{n})$ and $(b_{n})$ be equivalent Cauchy sequences.

    Let $\varepsilon$ be a positive rational number.

    $(a_{n})$ converges to $a$ means
    \[
        (\exists N_{1}=N_{1}(\varepsilon))(\forall n > N_{1})\left(\abs{a_{n} - a} < \frac{\varepsilon}{2}\right)
    \]

    Since $(a_{n})$ and $(b_{n})$ are equivalent
    \[
        (\exists N_{2}=N_{2}(\varepsilon))(\forall n > N_{2})\left(\abs{b_{n} - a_{n}} < \frac{\varepsilon}{2}\right)
    \]

    Let $N = \max\{ N_{1}, N_{2} \}$, then
    \[
        (\forall n > N)\left(\abs{b_{n} - a} \le \abs{b_{n} - a_{n}} + \abs{a_{n} - a} < \frac{\varepsilon}{2} + \frac{\varepsilon}{2} = \varepsilon\right)
    \]

    So $(b_{n})$ converges to $a$.

    Thus, every Cauchy sequence which is equivalent to $(a_{n})$ converges to $a$.
\end{proof}

\begin{corollary}
    If the Cauchy sequence $(a_{n})$ does not converge to any rational number, then every Cauchy sequence which is equivalent to $(a_{n})$ does not converge to any rational number, either.
\end{corollary}

\begin{proof}
    Let $(a_{n})$ and $(b_{n})$ be equivalent Cauchy sequences.

    Assume that $(b_{n})$ converges to a rational number. Then according to the previous theorem, $(a_{n})$ converges to the same rational number, which is a contradiction. So $(b_{n})$ does not converge to a ratioanl number.

    Thus, every Cauchy sequence which is equivalent to $(a_{n})$ does not converge to any rational number.
\end{proof}

\begin{example}
    $(b_{n})$ which is defined by
    \[
        b_{n} =
        \begin{cases}
            2                                       & \text{if $n = 1$} \\
            \dfrac{b_{n-1}}{2} + \dfrac{1}{b_{n-1}} & \text{if $n > 1$}
        \end{cases}
    \]

    does not converge to a rational number.
\end{example}

\begin{proof}
    Assume that $(b_{n})$ converges to a rational number $b$.
    \begin{enumerate}[label={\textbf{Step \arabic*.}},itemindent=1cm]
        \item ${b^{2}_{n}} > 2$.

              For $n = 1$, $b_{1} = 2$, so $b^{2}_{1} = 4 > 2$.

              For $n > 1$
              \begin{align*}
                  {b^{2}_{n}} - 2 & = {\left(\frac{b_{n-1}}{2} + \frac{1}{b_{n-1}}\right)}^{2} - 2         \\
                                  & = {\left(\frac{b^{2}_{n-1}}{4} + \frac{1}{b^{2}_{n-1}} + 1\right)} - 2 \\
                                  & = \frac{b^{2}_{n-1}}{4} + \frac{1}{b^{2}_{n-1}} - 1                    \\
                                  & = {\left(\frac{b_{n-1}}{2} - \frac{1}{b_{n-1}}\right)}^{2} > 0
              \end{align*}

              Hence $b^{2}_{n} > 2$.
        \item Prove that $(b_{n})$ is a decrease sequence.

              For every $n$, ${b^{2}_{n}}$ is greater than $2$.

              For $n > 1$
              \[
                  b_{n} - b_{n-1} = \left(\frac{b_{n-1}}{2} + \frac{1}{b_{n-1}}\right) - b_{n-1} = \frac{1}{b_{n-1}} - \frac{b_{n-1}}{2} = \frac{2 - {b}^{2}_{n-1}}{2\cdot b_{n}} < 0
              \]

              Then $(b_{n})$ is a decrease sequence.
        \item Prove that $b$ is smaller than all terms of $(b_{n})$.

              Assume that there exists a natural number $n_{1}$ such that $b_{n_{1}}\le b$. Since $(b_{n})$ is decrease, then $b_{n} < b$, for every $n > n_{1}$.

              Choose $\varepsilon = \varepsilon - b_{n_{1} + 1}$, then for all $n > n_{1}$, $\abs{b_{n} - b} = b - b_{n} \ge b - b_{n_{1} + 1} = \varepsilon$. Hence, $(b_{n})$ does not converge to $b$.

              Therefore, $b < b_{n}$, for every natural number $n$.
        \item Prove that $0 < b < 2$.

              Since $b$ is smaller than all terms of $(b_{n})$, then $b < b_{1} = 2$.

              Suppose that $b\le 0$.

              Choose $\varepsilon = 1$, then for all $n$, $\abs{b_{n} - b} = b_{n} - b \ge b_{n} > \varepsilon$. This means $(b_{n})$ does not converge to $b$.

              Therefore, $b > 0$.
        \item Prove that ${b}^{2} = 2$.

              Let's consider another sequence $(x_{n})$, where $x_{n} = {b^{2}_{n}}$.

              Since $b_{n}$ converges to $b$, then
              \[
                  (\forall\varepsilon > 0)(\exists N=N(\varepsilon))(\forall n > N)\left(\abs{b_{n} - b} < \frac{\varepsilon}{4}\right)
              \]

              It follows that
              \[
                  (\forall\varepsilon > 0)(\exists N=N(\varepsilon))(\forall n > N)\left(\abs{b^{2}_{n} - b^{2}} < \abs{b_{n} - b}\cdot\abs{b_{n} + b} < \frac{\varepsilon}{4}\cdot 4 = \varepsilon \right)
              \]

              So $(x_{n})$ converges to $b^{2}$.

              For $n > 2$
              \begin{align*}
                  b_{n}^{2} - 2 & = \frac{{\left( b^{2}_{n-1} - 2 \right)}^{2}}{4\cdot b^{2}_{n-1}}                               \\
                                & \le \frac{(b^{2}_{n-1} - 2)\cdot (b^{2}_{1} - 2)}{4\cdot b^{2}_{n-1}}                           \\
                                & \le \frac{b^{2}_{n-1} - 2}{2\cdot b^{2}_{n-1}}                                                  \\
                                & < \frac{b^{2}_{n-1} - 2}{4}                                                                     \\
                                & \le \frac{b^{2}_{1} - 2}{4^{n-1}} = \frac{2}{4^{n-1}} = \frac{1}{2^{2n-3}} \le \frac{1}{2^{n}}.
              \end{align*}

              \[
                  (\forall\varepsilon > 0)\left(\forall n > N(\varepsilon) = \abs{\floor{\frac{1}{\varepsilon} - 1}}\right)\left( \frac{1}{2^{n}} < \varepsilon \right).
              \]

              So for every $\varepsilon > 0$, choose $N = \abs{\floor{\dfrac{1}{\varepsilon} - 1}} + 3$ (I plus 3 to make sure $N > 2$), then for all $n > N$, $\abs{b^{2}_{n} - 2} < \dfrac{1}{2^{n}} < \varepsilon$.

              Therefore, $(x_{n})$ converges to $2$.

              Hence $b^{2} = 2$.
    \end{enumerate}

    $b^{2} = 2$ contradicts that there is no rational number whose square equals $2$.

    Hence the initial assumption ($b$ is a rational number) is false. Therefore, $(b_{n})$ does not converge to a rational number.
\end{proof}

The equivalence relation amongst Cauchy sequences partitions the set of all Cauchy sequences into \textit{equivalance classes}. In the upcoming subsections, I will prove that these equivalence classes satisfy the axioms of the real numbers. Firstly, I have to define the order relation, operations (addition, multiplication) on them.

\subsection{Field structure}

\subsection{Total ordering}

\subsection{Least-upper-bound property}

\subsection{Completeness (Cauchy-complete)}

\section{Compare the two constructions}

\subsection{Pros and Cons of the two constructions}

\subsection{Equivalence}

\section{Complex numbers}
