\documentclass{mike-apostol-mathematical-analysis}

\title{Extended version of Tom Mike Apostol's ``Mathematical Analysis''}
\author{Ngo Quang Duong}
\date{\today}

\begin{document}

\maketitle

\tableofcontents

\chapter*{Conventions}

In this note, I aim to provide a comprehensive and constructive approach to mathematical analysis, which means I will not leave any proofs as exercises to readers and I will restrict the use of proof by contradiction to prove the existence of an object. However, this approach would be really verbose, and it can even be tedious because of my dry writing style. I don't want my writing to be tedious. Instead of listing theorems to theorems and proofs to proofs, I will add an abstract to (almost) every subsection to recap the content of the subsection, explain my solution along the way. In every subsection, the main results will be emphasized. On the other hand, there are main theorems which require some intermediate results (let's call them lemmas). I will write them sequentially (lemmas and then proofs), but the lemmas (I might mark them as theorems) will have an italic heading. For example

\noindent\textit{\textbf{Theorem 1.}} This is an intermediate theorem.

\noindent\textbf{Theorem 2.} This is a main theorem.

Readers can safely jump right in the main results and then revisit the intermediate results when ever needed.

\chapter{First Examples}

\section{The Simplest Examples}

\section{Linear Systems with Constant Coefficients}

\chapter{Smooth Maps}

\section{Smooth Functions and Smooth Maps}

\section{Partitions of Unity}

\documentclass[class=linearalgebra,crop=false]{standalone}

\newcommand{\sgn}[1]{\text{sgn}\left({#1}\right)}
\setcounter{lemma}{0}

\begin{document}

\chapter{Định thức và hệ phương trình tuyến tính}

\par Thực hiện các phép nhân sau đây, viết các phép thế thu được thành tích của những xích rời rạc và tính dấu của chúng.

\begin{exercise}
    $
        \begin{pmatrix}
            1 & 2 & 3 & 4 & 5 \\
            2 & 4 & 5 & 1 & 3
        \end{pmatrix}
        \begin{pmatrix}
            1 & 2 & 3 & 4 & 5 \\
            4 & 3 & 5 & 1 & 2
        \end{pmatrix}
    $.
\end{exercise}

\begin{proof}[Lời giải]
    \[
        \begin{pmatrix}
            1 & 2 & 3 & 4 & 5 \\
            2 & 4 & 5 & 1 & 3
        \end{pmatrix}
        \begin{pmatrix}
            1 & 2 & 3 & 4 & 5 \\
            4 & 3 & 5 & 1 & 2
        \end{pmatrix}
        =
        \begin{pmatrix}
            4 & 3 & 5 & 1 & 2 \\
            1 & 5 & 3 & 2 & 4
        \end{pmatrix}
        \begin{pmatrix}
            1 & 2 & 3 & 4 & 5 \\
            4 & 3 & 5 & 1 & 2
        \end{pmatrix}
        =
        \begin{pmatrix}
            1 & 2 & 3 & 4 & 5 \\
            1 & 5 & 3 & 2 & 4
        \end{pmatrix}.
    \]
    \[
        \begin{pmatrix}
            1 & 2 & 3 & 4 & 5 \\
            1 & 5 & 3 & 2 & 4
        \end{pmatrix}
        =
        (1)(2,5,4)(3).
    \]
    \[
        \sgn{
            \begin{matrix}
                1 & 2 & 3 & 4 & 5 \\
                1 & 5 & 3 & 2 & 4
            \end{matrix}
        }
        = \sgn{1}\sgn{2,5,4}\sgn{3}
        = 1.
    \]
\end{proof}

\begin{exercise}
    $
        \begin{pmatrix}
            1 & 2 & 3 & 4 & 5 \\
            3 & 5 & 4 & 1 & 2
        \end{pmatrix}
        \begin{pmatrix}
            1 & 2 & 3 & 4 & 5 \\
            4 & 3 & 1 & 5 & 2
        \end{pmatrix}
    $.
\end{exercise}

\begin{proof}[Lời giải]
    \[
        \begin{pmatrix}
            1 & 2 & 3 & 4 & 5 \\
            3 & 5 & 4 & 1 & 2
        \end{pmatrix}
        \begin{pmatrix}
            1 & 2 & 3 & 4 & 5 \\
            4 & 3 & 1 & 5 & 2
        \end{pmatrix}
        =
        \begin{pmatrix}
            4 & 3 & 1 & 5 & 2 \\
            1 & 4 & 3 & 2 & 5
        \end{pmatrix}
        \begin{pmatrix}
            1 & 2 & 3 & 4 & 5 \\
            4 & 3 & 1 & 5 & 2
        \end{pmatrix}
        =
        \begin{pmatrix}
            1 & 2 & 3 & 4 & 5 \\
            1 & 4 & 3 & 2 & 5
        \end{pmatrix}.
    \]
    \[
        \begin{pmatrix}
            1 & 2 & 3 & 4 & 5 \\
            1 & 4 & 3 & 2 & 5
        \end{pmatrix}
        =
        (1)(2,4)(3)(5).
    \]
    \[
        \sgn{
            \begin{matrix}
                1 & 2 & 3 & 4 & 5 \\
                1 & 4 & 3 & 2 & 5
            \end{matrix}
        }
        = \sgn{1}\sgn{2,4}\sgn{3}\sgn{5}
        = -1.
    \]
\end{proof}

\begin{exercise}
    $(1,2)(2,3)\ldots (n-1,n)$.
\end{exercise}

\begin{lemma}\label{chapter3:cycles-product}
    $(a_{1}, a_{2}, \ldots, a_{k})(a_{k},a_{k+1}) = (a_{1},a_{2},\ldots, a_{k+1})$.
\end{lemma}

\begin{proof}[Chứng minh bổ đề]
    \par Xét dãy
        \[
            a_{1}, a_{2}, \ldots, a_{k-1}, a_{k}, a_{k+1}.
        \]
    \par Sau khi tác động bằng $(a_{k},a_{k+1})$, dãy trên trở thành:
        \[
            a_{1}, a_{2}, \ldots, a_{k-1}, a_{k+1}, a_{k}.
        \]
    \par Sau khi tác động bằng $(a_{1}, a_{2}, \ldots, a_{k})$, dãy trên (liên trên) trở thành:
        \[
            a_{2}, a_{3}, \ldots, a_{k}, a_{k+1}, a_{1}.
        \]
    \par Theo định nghĩa về xích, ta có điều phải chứng minh.
\end{proof}

\begin{proof}[Lời giải]
    \par Theo bổ đề~\ref{chapter3:cycles-product}:
        \[
            (1,2)(2,3)\ldots (n-1,n) = (1,2,\ldots,n)
            =
            \begin{pmatrix}
                1 & 2 & \cdots & n-1 & n \\
                2 & 3 & \cdots & n   & 1
            \end{pmatrix}
        \]
    \par $(1,2,\ldots, n)$ chính là một xích.
        \[
            \sgn{1,2,\ldots,n} = \sgn{1,2}\sgn{2,3}\ldots\sgn{n-1,n} = (-1){}^{n-1}.
        \]
\end{proof}

\begin{exercise}
    $(1,2,3)(2,3,4)(3,4,5)\ldots (n-2,n-1,n)$.
\end{exercise}

\begin{proof}[Lời giải]
    \par Theo bổ đề~\ref{chapter3:cycles-product}, nếu $n > 3$:
    \begin{align*}
        (1,2,3)(2,3,4)(3,4,5)\ldots (n-2,n-1,n)
        & = (1,2)(2,3)(2,3)(3,4)(3,4)(4,5) \ldots (n-2,n-1)(n-1,n) \\
        & = (1,2)(2,3){}^{2}(3,4){}^{2}\ldots (n-2,n-1){}^{2}(n-1,n) \\
        & = (1,2)(n-1,n)\qquad\text{(đây là 2 xích rời nhau)} \\
        & =
        \begin{pmatrix}
            1 & 2 & 3 & \cdots & n-2 & n-1 & n   \\
            2 & 1 & 3 & \cdots & n-2 & n   & n-1
        \end{pmatrix}.
    \end{align*}
    \par Nếu $n = 3$:
    \[
        (1,2,3) =
        \begin{pmatrix}
            1 & 2 & 3 \\
            2 & 3 & 1
        \end{pmatrix}.
    \]
    \par Trong cả hai trường hợp, dấu của phép thế (kết quả) là 1.
\end{proof}

\begin{exercise}
    Cho hai cách sắp thành dãy $a_{1}$, $a_{2}$, \ldots, $a_{n}$ và $b_{1}$, $b_{2}$, \ldots, $b_{n}$ của $n$ số tự nhiên đầu tiên. Chứng minh rằng có thể đưa cách sắp này về cách sắp kia bằng cách sử dụng không quá $n-1$ phép thế sơ cấp.
\end{exercise}

\begin{lemma}\label{chapter3:product-of-disjoint-cycles}
    Mọi phép thế đều có thể được viết dưới dạng tích của các xích rời nhau.
\end{lemma}

\begin{proof}[Chứng minh bổ đề~\ref{chapter3:product-of-disjoint-cycles}]
\end{proof}

\begin{lemma}\label{chapter3:product-of-transpositions}
    Một xích độ dài $k$ ($k > 1$) có thể viết được dưới dạng tích của $k-1$ phép thế sơ cấp.
\end{lemma}

\begin{proof}[Chứng minh bổ đề~\ref{chapter3:product-of-transpositions}]
\end{proof}

\begin{proof}
\end{proof}

\end{document}

\chapter{Cartesian products}

\section{Cartesian product topology}

\begin{problem}{IV.1.1}
Let \( \left\{ Y_{\alpha} \mid \alpha \in \mathscr{A} \right\} \) be a family of spaces. Assume that each \( Y_{\alpha} \) has a basis of cardinal number \( \le \aleph \). What is the cardinal of a basis for \( \prod_{\alpha} Y_{\alpha} \)?
\end{problem}

\begin{proof}
	% TODO
	It is \( \aleph \cdot \aleph(\mathscr{A}) \).
\end{proof}

\begin{problem}{IV.1.2}
Let \( \aleph(\mathscr{A}) \) be arbitrary and \( \prod_{\alpha} A_{\alpha} \subset \prod_{\alpha} Y_{\alpha} \). If all but at most finitely many factors \( A_{\alpha} = Y_{\alpha} \), prove \( \operatorname{Int}\left( \prod_{\alpha} A_{\alpha} \right) = \prod_{\alpha} \operatorname{Int}(A_{\alpha}) \).
\end{problem}

\begin{proof}
	Let \( \beta \in \mathscr{A} \). We will show that \( \operatorname{Int}\left( \prod_{\alpha} A_{\alpha} \right) = \prod_{\alpha} \operatorname{Int}(A_{\alpha}) \) when \( A_{\alpha} = Y_{\alpha} \) for all \(\alpha \ne \beta\).
	\begingroup
	\allowdisplaybreaks%
	\begin{align*}
		\operatorname{Int}\left( \prod_{\alpha} A_{\alpha} \right) & = \mathscr{C}\overline{\prod_{\alpha} Y_{\alpha} - \prod_{\alpha}A_{\alpha}}                         \\
		                                                           & = \mathscr{C}\overline{\mathscr{C}A_{\beta} \times \prod_{\alpha \ne \beta} Y_{\alpha}}              \\
		                                                           & = \mathscr{C}\left( \overline{\mathscr{C}A_{\beta}} \times \prod_{\alpha\ne\beta} Y_{\alpha} \right) \\
		                                                           & = \mathscr{C}\overline{\mathscr{C}A_{\beta}} \times \prod_{\alpha\ne\beta} Y_{\alpha}                \\
		                                                           & = \operatorname{Int}(A_{\beta}) \times \prod_{\alpha\ne\beta} Y_{\alpha}                             \\
		                                                           & = \prod_{\alpha} \operatorname{Int}(A_{\alpha}).
	\end{align*}
	\endgroup

	Now let \( \mathscr{B} \) be a finite subset of \( \mathscr{A} \) and \( A_{\alpha} = Y_{\alpha} \) whenever \( \alpha \notin \mathscr{B} \). From the previous case, we deduce that
	\begingroup
	\allowdisplaybreaks%
	\begin{align*}
		\operatorname{Int}\left( \prod_{\alpha} A_{\alpha} \right) & = \operatorname{Int}\left( \bigcap_{\beta \in \mathscr{B}} A_{\beta} \times \prod_{\alpha\ne\beta} Y_{\alpha} \right) \\
		                                                           & = \bigcap_{\beta \in \mathscr{B}} \operatorname{Int}\left( A_{\beta} \times \prod_{\alpha\ne\beta} Y_{\alpha} \right) \\
		                                                           & = \bigcap_{\beta \in \mathscr{B}} \operatorname{Int}(A_{\beta}) \times \prod_{\alpha\ne\beta} Y_{\alpha}              \\
		                                                           & = \prod_{\alpha} \operatorname{Int}(A_{\alpha}).
	\end{align*}
	\endgroup
\end{proof}

\begin{problem}{IV.1.3}
Let \( R \) be the real numbers with upper-limit topology (Chapter III, Section 3, Example 4). Show that \( R \times R \) is not a discrete space, but that \( A = \left\{ (x, y) \mid x + y = 1 \right\} \), as a subspace of \( R \times R \), has the discrete topology.
\end{problem}

\begin{proof}
	The singleton \( \left\{ (0, 0) \right\} \) is not open in \( R \times R \) as it doesn't contain any basic open set of the for \( \halfopenleft{a, b} \times \halfopenleft{c, d} \).

	On the other hand, if \( (x, y) \in A \) then \( \left\{ (x, y) \right\} = A \cap (\halfopenleft{x - 1, x} \times \halfopenleft{y - 1, y}) \) so \( \left\{ (x, y) \right\} \) is open in \( A \). Thus \( A \) has the discrete topology.
\end{proof}

\begin{problem}{IV.1.4}
Prove: \( \prod_{\alpha} A_{\alpha} \) is dense in \( \prod_{\alpha} Y_{\alpha} \) if and only if each \( A_{\alpha} \subset Y_{\alpha} \) is dense.
\end{problem}

\begin{proof}
	Because \( \overline{\prod_{\alpha} A_{\alpha}} = \prod_{\alpha} \overline{A_{\alpha}} \), the result follows.
\end{proof}

\section{Continuity of maps}

\begin{problem}{IV.2.1}
Prove: The cartesian product topology in \( \prod_{\alpha} Y_{\alpha} \) is the smallest topology for which all projections \( p_{\beta}: \prod_{\alpha} Y_{\alpha} \to Y_{\beta} \) are continuous.
\end{problem}

\begin{proof}
	Let \( \mathscr{T} \) be a topology on \( \prod_{\alpha} Y_{\alpha} \) such that all projections \( p_{\beta} \) are continuous. We need to show that \( \mathscr{T} \) contains the cartesian product topology.

	For each open set \( U_{\beta} \subset Y_{\beta} \), \( \left\langle U_{\beta} \right\rangle = p_{\beta}^{-1}(U_{\beta}) \in \mathscr{T} \) because \( p_{\beta} \) is continuous. Therefore the cartesian product topology is contained in \( \mathscr{T} \).

	Thus the cartesian product topology is the smallest topology for which all projections are continuous.
\end{proof}

\begin{problem}{IV.2.2}
Let \( \left\{ Y_{\alpha} \mid \alpha \in \mathscr{A} \right\} \) be a family of spaces. For each \( \mathscr{B} \subset \mathscr{A} \), let
\[
	p_{\mathscr{B}}: \prod_{\alpha \in \mathscr{A}} Y_{\alpha} \to \prod_{\beta \in \mathscr{B}} Y_{\beta}
\]

be the projection. Let \( A \subset \prod_{\alpha} Y_{\alpha} \) be closed. Prove:
\[
	A = \bigcap_{\mathscr{B} \text{ finite}} p_{\mathscr{B}}^{-1}\left( p_{\mathscr{B}}(A) \right).
\]
\end{problem}

\begin{proof}
	For each \( \mathscr{B} \subset \mathscr{A} \), \( p_{\mathscr{B}}^{-1}\left( p_{\mathscr{B}}(A) \right) \supset A \) so
	\[
		\bigcap_{\mathscr{B} \text{ finite}} p_{\mathscr{B}}^{-1}\left( p_{\mathscr{B}}(A) \right) \supset A
	\]

	Let \( c \in \bigcap_{\mathscr{B} \text{ finite}} p_{\mathscr{B}}^{-1}\left( p_{\mathscr{B}}(A) \right) \). Suppose on the contrary that \( c \notin A \). Since \( A \) is closed, there exists a basic open set \( U = \left\langle U_{\alpha_{1}}, \ldots, U_{\alpha_{n}} \right\rangle \) that contains \( c \) and is contained in \( \mathscr{C}A \). Let \( \mathscr{B} = \left\{ \alpha_{1}, \ldots, \alpha_{n} \right\} \). Because \( c \in p_{\mathscr{B}}^{-1}(p_{\mathscr{B}}(A)) \), \( p_{\mathscr{B}}(c) \in p_{\mathscr{B}}(A) \), so there exists \( f \in A \) such that \( p_{\mathscr{B}}(c) = p_{\mathscr{B}}(f) \). This means \( p_{\alpha_{i}}(f) = f(\alpha_{i}) = c(\alpha_{i}) \in U_{\alpha_{i}} \) for each \( i = 1, \ldots, n \), so \( f \in \left\langle U_{\alpha_{i}} \right\rangle \), which means \( f \in U = \bigcap^{n}_{i=1} \left\langle U_{\alpha_{i}} \right\rangle \). Therefore \( f \in A \cap U \), which contradicts \( U \subset \mathscr{C}A \). Thus \( c \in A \) and we conclude that \( \bigcap_{\mathscr{B} \text{ finite}} p_{\mathscr{B}}^{-1}\left( p_{\mathscr{B}}(A) \right) \subset A \).

	Hence \( A = \bigcap_{\mathscr{B} \text{ finite}} p_{\mathscr{B}}^{-1}\left( p_{\mathscr{B}}(A) \right) \).
\end{proof}


\section{Slices in Cartesian Products}

\begin{problem}{IV.3.1}
Let \( \mathscr{S} \) be the Sierpiński space, and \( \mathscr{S} \times \left\{ 0 \right\} \) the slice in \( \mathscr{S} \times \mathscr{S} \) parallel to the first factor. Is \( \mathscr{S} \times \left\{ 0 \right\} \) closed in \( \mathscr{S} \times \mathscr{S} \)?
\end{problem}

\begin{proof}
	The complement of \( \mathscr{S} \times \left\{0\right\} \) in \( \mathscr{S} \times \mathscr{S} \) is \( \mathscr{S} \times \left\{1\right\} \), which is not open. Hence \( \mathscr{S} \times \left\{ 0 \right\} \) is not closed in \( \mathscr{S} \times \mathscr{S} \).
\end{proof}

\section{Peano curves}



\end{document}
