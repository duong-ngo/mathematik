\chapter{Linear Maps and Matrices}

\section{The Linear Map associated with a Matrix}
\setcounter{exercise}{0}

\begin{exercise}
    In each case, find the vector $L_{A}(X)$
    \begin{enumerate}[label={(\alph*)}]
        \item $A = \begin{bmatrix}2 & 1 \\ 1 & 0\end{bmatrix}, X = \begin{bmatrix}3 \\ -1\end{bmatrix}$
        \item $A = \begin{bmatrix}1 & 0 \\ 0 & 0\end{bmatrix}, X = \begin{bmatrix}5 \\ 1\end{bmatrix}$
        \item $A = \begin{bmatrix}1 & 1 \\ 0 & 1\end{bmatrix}, X = \begin{bmatrix}4 \\ 1\end{bmatrix}$
        \item $A = \begin{bmatrix}0 & 0 \\ 0 & 1\end{bmatrix}, X = \begin{bmatrix}7 \\ -3\end{bmatrix}$
    \end{enumerate}
\end{exercise}

\begin{proof}
    \begin{enumerate}[label={(\alph*)}]
        \item $L_{A}(X) = \begin{bmatrix}5 \\ 3\end{bmatrix}$.
        \item $L_{A}(X) = \begin{bmatrix}5 \\ 0\end{bmatrix}$.
        \item $L_{A}(X) = \begin{bmatrix}5 \\ 1\end{bmatrix}$.
        \item $L_{A}(X) = \begin{bmatrix}0 \\ 4\end{bmatrix}$.
    \end{enumerate}
\end{proof}

\section{The Matrix associated with a Linear Map}
\setcounter{exercise}{0}

\begin{exercise}
    Find the matrix associated with the following linear maps. The vectors are written horizontally with a transpose sign for typographical reasons.
    \begin{enumerate}[label={(\alph*)}]
        \item $F: \mathbf{R}^{4} \to \mathbf{R}^{2}$ given by $F(\prescript{t}{}(x_{1}, x_{2}, x_{3}, x_{4})) = \prescript{t}{}(x_{1}, x_{2})$ (the projection)
        \item The projection from $\mathbf{R}^{4}$ to $\mathbf{R}^{3}$
        \item $F: \mathbf{R}^{2} \to \mathbf{R}^{2}$ given by $F(\prescript{t}{}(x, y)) = \prescript{t}{}(3x, 3y)$
        \item $F: \mathbf{R}^{n} \to \mathbf{R}^{n}$ given by $F(X) = 7X$
        \item $F: \mathbf{R}^{n} \to \mathbf{R}^{n}$ given by $F(X) = -X$
        \item $F: \mathbf{R}^{4} \to \mathbf{R}^{4}$ given by $F(\prescript{t}{}(x_{1}, x_{2}, x_{3}, x_{4})) = \prescript{t}{}(x_{1}, x_{2}, 0, 0)$
    \end{enumerate}
\end{exercise}

\begin{proof}
    \begin{enumerate}[label={(\alph*)}]
        \item \[
                  \begin{bmatrix}
                      1 & 0 & 0 & 0 \\
                      0 & 1 & 0 & 0
                  \end{bmatrix}
              \]
        \item \[
                  \begin{bmatrix}
                      1 & 0 & 0 & 0 \\
                      0 & 1 & 0 & 0 \\
                      0 & 0 & 1 & 0
                  \end{bmatrix}
              \]
        \item \[
                  \begin{bmatrix}
                      3 & 0 \\
                      0 & 3
                  \end{bmatrix}
              \]
        \item \[
                  \begin{bmatrix}
                      7      & 0      & \cdots & 0      \\
                      0      & 7      & \cdots & 0      \\
                      \vdots & \vdots &        & \vdots \\
                      0      & 0      & \cdots & 7
                  \end{bmatrix}
              \]
        \item \[
                  \begin{bmatrix}
                      -1     & 0      & \cdots & 0      \\
                      0      & -1     & \cdots & 0      \\
                      \vdots & \vdots &        & \vdots \\
                      0      & 0      & \cdots & -1
                  \end{bmatrix}
              \]
        \item \[
                  \begin{bmatrix}
                      1 & 0 & 0 & 0 \\
                      0 & 1 & 0 & 0 \\
                      0 & 0 & 0 & 0 \\
                      0 & 0 & 0 & 0
                  \end{bmatrix}
              \]
    \end{enumerate}
\end{proof}

\begin{exercise}
    Find the matrix $R(\theta)$ associated with the rotation for each of the following values of $\theta$.
    \begin{enumerate}[label={(\alph*)}]
        \item $\pi/2$
        \item $\pi/4$
        \item $\pi$
        \item $-\pi$
        \item $-\pi/3$
        \item $\pi/6$
        \item $5\pi/4$
    \end{enumerate}
\end{exercise}

\begin{proof}
    \begin{enumerate}[label={(\alph*)}]
        \item \[
                  \begin{bmatrix}
                      0 & -1 \\
                      1 & 0
                  \end{bmatrix}
              \]
        \item \[
                  \begin{bmatrix}
                      \frac{\sqrt{2}}{2} & \frac{-\sqrt{2}}{2} \\
                      \frac{\sqrt{2}}{2} & \frac{\sqrt{2}}{2}
                  \end{bmatrix}
              \]
        \item \[
                  \begin{bmatrix}
                      -1 & 0  \\
                      0  & -1
                  \end{bmatrix}
              \]
        \item \[
                  \begin{bmatrix}
                      -1 & 0  \\
                      0  & -1
                  \end{bmatrix}
              \]
        \item \[
                  \begin{bmatrix}
                      \frac{1}{2}         & \frac{\sqrt{3}}{2} \\
                      \frac{-\sqrt{3}}{2} & \frac{1}{2}
                  \end{bmatrix}
              \]
        \item \[
                  \begin{bmatrix}
                      \frac{\sqrt{3}}{2} & \frac{-1}{2}       \\
                      \frac{1}{2}        & \frac{\sqrt{3}}{2}
                  \end{bmatrix}
              \]
        \item \[
                  \begin{bmatrix}
                      \frac{-\sqrt{2}}{2} & \frac{\sqrt{2}}{2}  \\
                      \frac{-\sqrt{2}}{2} & \frac{-\sqrt{2}}{2}
                  \end{bmatrix}
              \]
    \end{enumerate}
\end{proof}

\begin{exercise}
    In general, let $\theta > 0$. What is the matrix associated with the rotation by an angle $-\theta$?
\end{exercise}

\begin{proof}
    \[
        \begin{bmatrix}
            \cos\theta  & \sin\theta \\
            -\sin\theta & \cos\theta
        \end{bmatrix}
    \]
\end{proof}

\begin{exercise}
    Let $X = \prescript{t}{}(1, 2)$ be a point of the plane. Let $F$ be the rotation through an angle of $\pi/4$. What are the coordinates of $F(X)$ relative to the usual basis $\{ E^{1}, E^{2} \}$?
\end{exercise}

\begin{proof}
    $F(X) = \prescript{t}{}(\frac{-\sqrt{2}}{2}, \frac{3\sqrt{2}}{2}) = \frac{-\sqrt{2}}{2}E^{1} + \frac{3\sqrt{2}}{2}E^{2}$.
\end{proof}

\begin{exercise}
    Same question when $X = \prescript{t}{}(-1, 3)$, and $F$ is the rotation through $\pi/2$.
\end{exercise}

\begin{proof}
    $F(X) = \prescript{t}{}(-3, -1) = (-3)E^{1} + (-1)E^{2}$.
\end{proof}

\begin{exercise}
    Let $F: \mathbf{R}^{n} \to \mathbf{R}^{n}$ be a linear map which is invertible. Show that if $A$ is the matrix associated with $F$, then $A^{-1}$ is the matrix associated the inverse of $F$.
\end{exercise}

\begin{proof}
    $X\in\mathbf{R}^{n}$. $F(X) = AX$. Since $F^{-1}$ is also a linear mapping, then there exists matrix $B$ such that $F^{-1}(X) = BX$. $(F^{-1}\circ F)(X) = BAX$, $(F\circ F^{-1})(X) = ABX$. Because $F^{-1}\circ F = F\circ F^{-1} = \text{id}_{\mathbf{R}^{n}}$, and the matrix associated with the identity mapping is $I$, so $AB = BA = I$, where $I$ is the identity $n\times n$ matrix. Thus $B = A^{-1}$, $A^{-1}$ is the matrix associated with the inverse of $F$.
\end{proof}

\begin{exercise}
    Let $F$ be a rotation through an angle $\theta$. Show that for any vector $X$ in $\mathbf{R}^{2}$ we have $\norm{X} = \norm{F(X)}$ (i.e. $F$ preserves norms), where $\norm{(a, b)} = \sqrt{a^{2} + b^{2}}$.
\end{exercise}

\begin{proof}
    Let $X = \prescript{t}{}(a, b)$. $F(X) = \prescript{t}{}(a\cos\theta - b\sin\theta, a\sin\theta + b\cos\theta)$.
    \begin{align*}
        \norm{F(X)} & = \sqrt{{(a\cos\theta - b\sin\theta)}^{2} + {(a\sin\theta + b\cos\theta)}^{2}}                                                                     \\
                    & = \sqrt{a^{2}\cos^{2}\theta + b^{2}\sin^{2}\theta - 2ab\cos\theta\sin\theta + a^{2}\sin^{2}\theta + b^{2}\cos^{2}\theta + 2ab\sin\theta\cos\theta} \\
                    & = \sqrt{a^{2}(\cos^{2}\theta + \sin^{2}\theta) + b^{2}(\cos^{2}\theta + \sin^{2}\theta)}                                                           \\
                    & = \sqrt{a^{2} + b^{2}}                                                                                                                             \\
                    & = \norm{X}
    \end{align*}

    Thus $\norm{X} = \norm{F(X)}$.
\end{proof}

\begin{exercise}
    Let $c$ be a number, and let $L: \mathbf{R}^{n} \to \mathbf{R}^{n}$ be the linear map such that $L(X) = cX$. What is the matrix associated with this linear map?
\end{exercise}

\begin{proof}
    \[
        \begin{bmatrix}
            c      & 0      & \cdots & 0      \\
            0      & c      & \cdots & 0      \\
            \vdots & \vdots &        & \vdots \\
            0      & 0      & \cdots & c
        \end{bmatrix}
    \]
\end{proof}

\begin{exercise}
    Let $F_{\theta}$ be rotation by an angle $\theta$. If $\theta, \varphi$ are numbers, compute the matrix of the linear map $F_{\theta}\circ F_{\varphi}$ and show that is the matrix of $F_{\theta + \varphi}$.
\end{exercise}

\begin{proof}
    The matrices of the linear maps $F_{\theta}, F_{\varphi}$ are
    \[
        \begin{bmatrix}
            \cos\theta & -\sin\theta \\
            \sin\theta & \cos\theta
        \end{bmatrix},\quad
        \begin{bmatrix}
            \cos\varphi & -\sin\varphi \\
            \sin\varphi & \cos\varphi
        \end{bmatrix}
    \]

    The matrix of the linear map $F_{\theta}\circ F_{\varphi}$ is
    \[
        \begin{bmatrix}
            \cos\varphi & -\sin\varphi \\
            \sin\varphi & \cos\varphi
        \end{bmatrix}
        \begin{bmatrix}
            \cos\theta & -\sin\theta \\
            \sin\theta & \cos\theta
        \end{bmatrix}
        =
        \begin{bmatrix}
            \cos(\varphi + \theta) & -\sin(\varphi + \theta) \\
            \sin(\varphi + \theta) & \cos(\varphi + \theta)
        \end{bmatrix}
    \]

    which is also the matrix of $F_{\theta\circ\varphi}$.
\end{proof}

\begin{exercise}
    Let $F_{\theta}$ be rotation by angle $\theta$. Show that $F_{\theta}$ is invertible, and determine the matrix associated with $F_{\theta}^{-1}$.
\end{exercise}

\begin{proof}
    $F_{\theta}\circ F_{-\theta} = F_{-\theta}\circ F_{\theta} = F_{0} = I$. The matrix associated with $F_{\theta}^{-1}$ is
    \[
        \begin{bmatrix}
            \cos\theta  & \sin\theta \\
            -\sin\theta & \cos\theta
        \end{bmatrix}
    \]
\end{proof}

\section{Bases, Matrices, and Linear Maps}
\setcounter{exercise}{0}

\begin{exercise}
    In each one of the following cases, find $M^{\mathscr{B}}_{\mathscr{B'}}(\text{id})$. The vector space in each case is $\mathbf{R}^{3}$.
    \begin{enumerate}[label={(\alph*)}]
        \item $\mathscr{B} = \{ (1, 1, 0), (-1, 1, 1), (0, 1, 2) \}$, $\mathscr{B'} = \{ (2, 1, 1), (0, 0, 1), (-1, 1, 1) \}$
        \item $\mathscr{B} = \{ (3, 2, 1), (0, -2, 5), (1, 1, 2) \}$, $\mathscr{B'} = \{ (1, 1, 0), (-1, 2, 4), (2, -1, 1) \}$
    \end{enumerate}
\end{exercise}

\begin{proof}
    \begin{enumerate}[label={(\alph*)}]
        \item \[
                  M^{\mathscr{B}}_{\mathscr{B'}} =
                  \begin{bmatrix}
                      \frac{2}{3} & 0 & \frac{1}{3} \\
                      -1          & 0 & 1           \\
                      \frac{1}{3} & 1 & \frac{2}{3}
                  \end{bmatrix}
              \]
        \item \[
                  M^{\mathscr{B}}_{\mathscr{B'}} =
                  \begin{bmatrix}
                      \frac{11}{5} & \frac{-11}{5} & \frac{3}{5} \\
                      \frac{2}{15} & \frac{13}{15} & \frac{2}{5} \\
                      \frac{7}{15} & \frac{23}{15} & \frac{2}{5}
                  \end{bmatrix}
              \]
    \end{enumerate}
\end{proof}

\begin{exercise}
    Let $L: V\to V$ be a linear map. Let $\mathscr{B} = \{ v_{1}, \ldots, v_{n} \}$ be a basis of $V$. Suppose that there are numbers $c_{1}, \ldots, c_{n}$ such that $L(v_{i}) = c_{i}v_{i}$ for $i = 1,\ldots, n$. What is $M^{\mathscr{B}}_{\mathscr{B'}}(L)$?
\end{exercise}

\begin{proof}
    \[
        M^{\mathscr{B}}_{\mathscr{B'}} =
        \text{diag}(c_{1}, c_{2}, \ldots, c_{n}) =
        \begin{bmatrix}
            c_{1}  & 0      & \cdots & 0      \\
            0      & c_{2}  & \cdots & 0      \\
            \vdots & \vdots & \ddots & \vdots \\
            0      & 0      & \cdots & c_{n}
        \end{bmatrix}
    \]
\end{proof}

\begin{exercise}
    For each real number $\theta$, let $F_{\theta}: \mathbf{R}^{2} \to \mathbf{R}^{2}$ be the linear map represented by the matrix
    \[
        R(\theta) = \begin{bmatrix}
            \cos\theta & -\sin\theta \\
            \sin\theta & \cos\theta
        \end{bmatrix}.
    \]

    Show that if $\theta, \theta'$ are real numbers, then $F_{\theta}F_{\theta'} = F_{\theta + \theta'}$. Also show that $F_{\theta}^{-1} = F_{-\theta}$.
\end{exercise}

\begin{proof}
    \begin{align*}
        R(\theta)R(\theta') & =
        \begin{bmatrix}
            \cos\theta & -\sin\theta \\
            \sin\theta & \cos\theta
        \end{bmatrix}.
        \begin{bmatrix}
            \cos\theta' & -\sin\theta' \\
            \sin\theta' & \cos\theta'
        \end{bmatrix}                                                                                             \\
                            & = \begin{bmatrix}
                                    \cos\theta\cos\theta' - \sin\theta\sin\theta' & -\sin\theta'\cos\theta - \cos\theta'\sin\theta \\
                                    \sin\theta\cos\theta' + \cos\theta\sin\theta' & \cos\theta\cos\theta' - \sin\theta\sin\theta'
                                \end{bmatrix} \\
                            & = \begin{bmatrix}
                                    \cos(\theta + \theta') & -\sin(\theta + \theta') \\
                                    \sin(\theta + \theta') & \cos(\theta + \theta')
                                \end{bmatrix}                                               \\
                            & = R(\theta + \theta').
    \end{align*}

    Therefore, $F_{\theta}F_{\theta'} = F_{\theta + \theta'}$.

    Particularly, $F_{\theta}F_{-\theta} = F_{-\theta}F_{\theta} = F_{0} = \text{id}$. Hence $F_{\theta}^{-1} = F_{-\theta}$.
\end{proof}

\begin{exercise}
    In general, let $\theta > 0$. What is the matrix associated with the identity map, add rotation of bases by an angle $-\theta$?
\end{exercise}

\begin{proof}
    \[
        \mathscr{B} = \{ (1, 0), (0, 1) \} \qquad \mathscr{B'} = \{ (\cos\theta, -\sin\theta), (\sin\theta, \cos\theta) \}
    \]
    \[
        M^{\mathscr{B}}_{\mathscr{B'}}(\text{id}) =
        \begin{bmatrix}
            \cos\theta & -\sin\theta \\
            \sin\theta & \cos\theta
        \end{bmatrix}
    \]
\end{proof}

\begin{exercise}
    Let $X = \prescript{t}{}(1, 2)$ be a point of the plane. Let $F$ be the rotation through an angle of $\pi/4$. What are the coordinates of $F(X)$ relative to the usual basis $\{ E^{1}, E^{2} \}$?
\end{exercise}

\begin{proof}
    $F(X) = \prescript{t}{}(\frac{-\sqrt{2}}{2}, \frac{3\sqrt{2}}{2}) = \frac{-\sqrt{2}}{2}E^{1} + \frac{3\sqrt{2}}{2}E^{2}$.
\end{proof}

\begin{exercise}
    Same question when $X = \prescript{t}{}(-1, 3)$, and $F$ is the rotation through $\pi/2$.
\end{exercise}

\begin{proof}
    $F(X) = \prescript{t}{}(-3, -1) = (-3)E^{1} + (-1)E^{2}$.
\end{proof}

\begin{exercise}
    In general, let $F$ be the rotation through an angle $\theta$. Let $(x, y)$ be a point of the plane in the standard coordinate system. Let $(x', y')$ be the coordinates of this point in the rotated system. Express $x', y'$ in terms of $x, y$, and $\theta$.
\end{exercise}

\begin{proof}
    \[
        \begin{cases}
            x' = x\cos\theta - y\sin\theta \\
            y' = x\sin\theta + y\cos\theta
        \end{cases}
    \]
\end{proof}

\begin{exercise}
    In each of the following cases, let $D = d/dt$ be the derivative. We give a set of linearly independent functions $\mathscr{B}$. These generate a vector space $V$, and $D$ is a linear map from $V$ into itself. Find the matrix associated with $D$ relative to the bases $\mathscr{B}, \mathscr{B}$.
    \begin{enumerate}[label={(\alph*)}]
        \item $\{ e^{t}, e^{2t} \}$
        \item $\{ 1, t \}$
        \item $\{ e^{t}, te^{t} \}$
        \item $\{ 1, t, t^{2} \}$
        \item $\{ 1, t, e^{t}, e^{2t}, te^{2t} \}$
        \item $\{ \sin t, \cos t \}$
    \end{enumerate}
\end{exercise}

\begin{proof}
    \begin{enumerate}[label={(\alph*)}]
        \item \[
                  M^{\mathscr{B}}_{\mathscr{B}}(D) =
                  \begin{bmatrix}
                      1 & 0 \\
                      0 & 2
                  \end{bmatrix}
              \]
        \item \[
                  M^{\mathscr{B}}_{\mathscr{B}}(D) =
                  \begin{bmatrix}
                      0 & 1 \\
                      0 & 0
                  \end{bmatrix}
              \]
        \item \[
                  M^{\mathscr{B}}_{\mathscr{B}}(D) =
                  \begin{bmatrix}
                      1 & 1 \\
                      0 & 1
                  \end{bmatrix}
              \]
        \item \[
                  M^{\mathscr{B}}_{\mathscr{B}}(D) =
                  \begin{bmatrix}
                      0 & 1 & 0 \\
                      0 & 0 & 2 \\
                      0 & 0 & 0
                  \end{bmatrix}
              \]
        \item \[
                  M^{\mathscr{B}}_{\mathscr{B}}(D) =
                  \begin{bmatrix}
                      0 & 1 & 0 & 0 & 0 \\
                      0 & 0 & 0 & 0 & 0 \\
                      0 & 0 & 1 & 0 & 0 \\
                      0 & 0 & 0 & 2 & 1 \\
                      0 & 0 & 0 & 0 & 2
                  \end{bmatrix}
              \]
        \item \[
                  M^{\mathscr{B}}_{\mathscr{B}}(D) =
                  \begin{bmatrix}
                      0 & -1 \\
                      1 & 0
                  \end{bmatrix}
              \]
    \end{enumerate}
\end{proof}

\begin{exercise}
    \begin{enumerate}[label={(\alph*)}]
        \item Let $N$ be a square matrix. We say that $N$ is \textbf{nilpotent} if there exists a positive integer $r$ such that $N^{r} = 0$. Prove that if $N$ is nilpotent, then $I - N$ is invertible.
        \item State and prove that analogous statement for linear maps of a vector space into itself.
    \end{enumerate}
\end{exercise}

\begin{proof}
    \begin{enumerate}[label={(\alph*)}]
        \item \[
                  \begin{split}
                      (I - N)(I + N + \cdots + N^{r-1}) = (I + N + \cdots + N^{r-1}) - (N + N^{2} + \cdots + N^{r}) = I - N^{r} = I \\
                      (I + N + \cdots + N^{r-1})(I - N) = (I + N + \cdots + N^{r-1}) - (N + N^{2} + \cdots + N^{r}) = I - N^{r} = I
                  \end{split}
              \]

              Hence $I - N$ is invertible.
        \item A linear map $F$ of a vector space into itself is nilpotent if there exists a positive integer $r$ such that $F^{r} = 0$. If $F$ is nilpotent, then $\text{id} - F$ is invertible.

              Let $\mathscr{B}$ be a basis of the given vector space. $F^{r} = 0$ iff ${(M^{\mathscr{B}}_{\mathscr{B}}(F))}^{r} = 0$. According to part (a), $I - M^\mathscr{B}_{\mathscr{B}}(F)$ is invertible. Therefore, $\text{id} - F$ is invertible.
    \end{enumerate}
\end{proof}

\begin{exercise}
    Let $P_{n}$ be the vector space of polynomials of degree $\leqq n$. Then the derivative $D: P_{n} \to P_{n}$ is a linear map of $P_{n}$ into itself. Let $I$ be the identity mapping. Prove that the following linear maps are invertible:
    \begin{enumerate}[label={(\alph*)}]
        \item $I - D^{2}$.
        \item $D^{m} - I$ for any positive integer $m$.
        \item $D^{m} - cI$ for any number $c\ne 0$.
    \end{enumerate}
\end{exercise}

\begin{proof}
    \begin{enumerate}[label={(\alph*)}]
        \item $D^{n+1} = 0, D^{2n} = 0$. Therefore $D^{2}$ is nilpotent. According to Exercise 10, $I - D^{2}$ is invertible.
        \item $D^{mn} = 0$. Therefore $D^{m}$ is nilpontent. According to Exercise 10, $I - D^{m}$ is invertible. So $D^{m} - I$ is invertible.
        \item $D^{mn} = 0$
              \begin{align*}
                    & (D^{m} - cI)(D^{(n-1)m} + cD^{(n-2)m} + \cdots + c^{n-1}I)                                         \\
                  = & (D^{nm} + cD^{(n-1)m} + \cdots + c^{n-1}D^{m}) - (cD^{(n-1)m} + c^{2}D^{(n-2)m} + \cdots + c^{n}I) \\
                  = & D^{nm} - c^{n}I                                                                                    \\
                  = & -c^{n}I,                                                                                           \\
                    & (D^{(n-1)m} + cD^{(n-2)m} + \cdots + c^{n-1}I)(D^{m} - cI)                                         \\
                  = & (D^{nm} + cD^{(n-1)m} + \cdots + c^{n-1}D^{m}) - (cD^{(n-1)m} + c^{2}D^{(n-2)m} + \cdots + c^{n})  \\
                  = & D^{nm} - c^{n}I                                                                                    \\
                  = & -c^{n}I
              \end{align*}

              Hence $D^{m} - cI$ is invertible, because its inverse map is $\frac{-1}{c^{n}}D^{(n-1)m} + \frac{-1}{c^{n-1}}D^{(n-2)m} + \cdots + \frac{-1}{c}I$.
    \end{enumerate}
\end{proof}

\begin{exercise}
    Let $A$ be the $n\times n$ matrix
    \[
        A = \begin{bmatrix}
            0      & 1      & 0      & \cdots & 0      \\
            0      & 0      & 1      & \cdots & 0      \\
            0      & 0      & 0      & \cdots & 0      \\
            \vdots & \vdots & \vdots & \ddots & \vdots \\
            0      & 0      & 0      & \cdots & 1      \\
            0      & 0      & 0      & \cdots & 0
        \end{bmatrix}
    \]

    which is upper triangular, with zeros on the diagonal, $1$ just above the diagonal, and zeros elsewhere as shown.
    \begin{enumerate}[label={(\alph*)}]
        \item How would you describe the effect of $L_{A}$ on the standard basis vectors $\{ E^{1}, \ldots, E^{n} \}$ of $K^{n}$?
        \item Show that $A^{n} = O$ and $A^{n-1}\ne O$ by using the effect of power of $A$ on the basis vector.
    \end{enumerate}
\end{exercise}

\begin{proof}
    \begin{enumerate}[label={(\alph*)}]
        \item $L_{A}(E^{1}) = O, L_{A}(E^{2}) = E_{1}, L_{A}(E^{3}) = E^{2}, \ldots, L_{A}(E^{n}) = E^{n-1}$
        \item $L_{A}^{n}(E^{1}) = O, L_{A}^{n}(E^{2}) = O, \ldots, L_{A}^{n}(E^{n}) = O$. Therefore, $L_{A}^{n}(X) = O$ for any column vector $X\in\mathbf{R}^{n}$. Hence $A^{n} = O$.

              $L_{A}^{n-1}(E^{n}) = L_{A}^{n-2}(E^{n-1}) = \cdots = L_{A}(E^{2}) = E^{1}\ne O$. Hence $A^{n-1} \ne O$.
    \end{enumerate}
\end{proof}
