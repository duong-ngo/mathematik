\chapter{Vector Spaces}

\section{Definitions}

\setcounter{exercise}{0}

\begin{exercise}
    Let $V$ be a vector space. Using the properties \textbf{VS 1} through \textbf{VS 8}, show that if $c$ is a number, then $cO = O$.
\end{exercise}

\begin{proof}
    Let $v$ be a vector in $V$.
    \begin{align*}
        0v         & = (0 + 0)v                                 \\
                   & = 0v + 0v           & \text{\textbf{VS 6}} \\
        0v + (-0v) & = 0v + (0v + (-0v)) & \text{\textbf{VS 1}} \\
        O          & = 0v + O            & \text{\textbf{VS 3}} \\
        O          & = 0v                & \text{\textbf{VS 2}}
    \end{align*}

    \begin{align*}
        cO         & = c(O + O)                                 \\
                   & = cO + cO           & \text{\textbf{VS 5}} \\
        cO + (-cO) & = cO + (cO + (-cO)) & \text{\textbf{VS 1}} \\
        O          & = cO + 0            & \text{\textbf{VS 3}} \\
        O          & = cO                & \text{\textbf{VS 2}}
    \end{align*}
\end{proof}

\begin{exercise}
    Let $c$ be a number $\ne 0$, and $v$ an element of $V$. Prove that if $cv = O$, then $v = O$.
\end{exercise}

\begin{proof}
    Since $c\ne 0$, there exists $c^{-1}$ such that $cc^{-1} = c^{-1}c = 1$.

    According to \textbf{VS7} and \textbf{VS 8}
    \[
        v = 1v = (c^{-1}c)v = c^{-1}(cv) = c^{-1}O = O.
    \]
\end{proof}

\begin{exercise}
    In the vector space of functions, what is the function satisfying the condition \textbf{VS 2}?
\end{exercise}

\begin{proof}
    The zero function $O(x) = 0$ for every number $x$, since
    \[
        f(x) + O(x) = f(x) + 0 = f(x) = 0 + f(x) = O(x) + f(x).
    \]
\end{proof}

\begin{exercise}
    Let $V$ be a vector space and $v, w$ two elements of $V$. If $v + w = O$, show that $w = -v$.
\end{exercise}

\begin{proof}
    According to \textbf{VS 1, 2, 3, 4}
    \[
        w = O + w = ((-v) + v) + w = (-v) + (v + w) = (-v) + O = -v.
    \]

    Hence $w = -v$.
\end{proof}

\begin{exercise}
    Let $V$ be a vector space, and $v, w$ two elements of $V$ such that $v + w = v$. Show that $w = O$.
\end{exercise}

\begin{proof}
    \begin{align*}
        w & = O + w          & \text{\textbf{VS 2}} \\
          & = ((-v) + v) + w & \text{\textbf{VS 3}} \\
          & = (-v) + (v + w) & \text{\textbf{VS 1}} \\
          & = (-v) + v                              \\
          & = O              & \text{\textbf{VS 3}}
    \end{align*}

    Thus, $w = O$.
\end{proof}

\begin{exercise}
    Let $A_{1}, A_{2}$ be vectors in $\mathbb{R}^{n}$. Show that the set of all vectors $B$ in $\mathbb{R}^{n}$ such that $B$ is perpendicular to both $A_{1}$ and $A_{2}$ is a subspace.
\end{exercise}

\begin{proof}
    Denote by $S_{i}$ the set of all vector $B$ in $\mathbb{R}^{n}$ such that $B$ is perpendicular to $A_{i}$.

    $O$ is in $S_{1}$, since $O\cdot A_{1} = 0$.

    Let $B = (b_{1}, b_{2}, \ldots, b_{n}), C = (c_{1}, c_{2}, \ldots, c_{n})$ be elements of $S_{1}$, and $A_{1} = (a_{1.1}, a_{1.2}, \ldots, a_{1.n})$.
    \begin{align*}
        (B + C)\cdot A_{1} & = (b_{1} + c_{1})a_{1.1} + (b_{2} + c_{2})a_{1.2} + \cdots + (b_{n} + c_{n})a_{1.n}                             \\
                           & = (b_{1}a_{1.1} + b_{2}a_{1.2} + \cdots + b_{n}a_{1.n}) + (c_{1}a_{1.1} + c_{2}a_{1.2} + \cdots + c_{n}a_{1.n}) \\
                           & = B\cdot A_{1} + C\cdot A_{1}                                                                                   \\
                           & = 0 + 0 = 0                                                                                                     \\
        (cB)\cdot A_{1}    & = cb_{1}a_{1.1} + cb_{2}a_{1.2} + \cdots + cb_{n}a_{1.n}                                                        \\
                           & = c(b_{1}a_{1.1} + b_{2}a_{1.2} + \cdots + b_{n}a_{1.n})                                                        \\
                           & = c(B\cdot A_{1})                                                                                               \\
                           & = c0 = 0.
    \end{align*}

    Hence $S_{1}$ is a subspace of $\mathbb{R}^{n}$. Analogously, $S_{2}$ is a subspace of $\mathbb{R}^{n}$. Thus $S_{1}\cap S_{2}$ (the set of all vectors $B$ in $\mathbb{R}^{n}$ such that $B$ is perpendicular to both $A_{1}$ and $A_{2}$) is a subspace of $\mathbb{R}^{n}$.
\end{proof}

\begin{exercise}
    Generalize Exercise 6, and prove: Let $A_{1}, \ldots, A_{r}$ be vectors in $\mathbb{R}^{n}$. Let $W$ be the set of vectors $B$ in $\mathbb{R}^{n}$ such that $B\cdot A_{i} = 0$ for every $i = 1,\ldots, r$. Show that $S$ is a subspace of $\mathbb{R}^{n}$.
\end{exercise}

\begin{proof}
    Denote by $S_{i}$ the set of vectors $B$ in $\mathbb{R}^{n}$ such that $B\cdot A_{i} = 0$.

    Similar to Exercise 6, $S_{i}$ is a subspace of $\mathbb{R}^{n}$ for every $i = 1,\ldots, r$.

    Hence $S = S_{1}\cap S_{2}\cap \cdots\cap S_{r}$ is also a subspace of $\mathbb{R}^{n}$.
\end{proof}

\begin{exercise}
    Show that the following sets of elements in $\mathbb{R}^{2}$ form subspaces.
    \begin{enumerate}[label={(\alph*)}]
        \item The set of all $(x, y)$ such that $x = y$.
        \item The set of all $(x, y)$ such that $x - y = 0$.
        \item The set of all $(x, y)$ such that $x + 4y = 0$.
    \end{enumerate}
\end{exercise}

\begin{proof}
    $O = (0, 0)$ is in all of these sets.

    \begin{enumerate}[label={(\alph*)}]
        \item If $(x_{1}, y_{1})$ and $(x_{2}, y_{2})$ are in the set, $x_{1} + x_{2} = y_{1} + y_{2}$. Therefore $(x_{1} + x_{2}, y_{1} + y_{2})$ is in the set.

              If $(x, y)$ is in the set, then $cx = cy$. Therefore, $c(x, y) = (cx, cy)$ is in the set.

              Hence the set forms a subspace.
        \item If $(x_{1}, y_{1})$ and $(x_{2}, y_{2})$ are in the set, $x_{1} + x_{2} - (y_{1} + y_{2}) = (x_{1} - y_{1}) + (x_{2} - y_{2}) = O - O = O$. Therefore $(x_{1} + x_{2}, y_{1} + y_{2})$ is in the set.

              If $(x, y)$ is in the set, then $cx - cy = c(x - y) = 0$. Therefore, $c(x, y) = (cx, cy)$ is in the set.

              Hence the set forms a subspace.
        \item If $(x_{1}, y_{1})$ and $(x_{2}, y_{2})$ are in the set, $x_{1} + x_{2} + 4(y_{1} + y_{2}) = (x_{1} + 4y_{1}) + (x_{2} + 4y_{2}) = O + O = O$. Therefore $(x_{1} + x_{2}, y_{1} + y_{2})$ is in the set.

              If $(x, y)$ is in the set, the $cx + 4(cy) = cx + c(4y) = c(x + 4y) = 0$. Therefore, $c(x, y) = (cx, cy)$ is in the set.

              Hence the set forms a subspace.
    \end{enumerate}
\end{proof}

\begin{exercise}
    Show the the following sets of elements in $\mathbb{R}^{3}$ form subspaces.
    \begin{enumerate}[label={(\alph*)}]
        \item The set of all $(x, y, z)$ such that $x + y + z = 0$.
        \item The set of all $(x, y, z)$ such that $x = y$ and $2y = z$.
        \item The set of all $(x, y, z)$ such that $x + y = 3z$.
    \end{enumerate}
\end{exercise}

\begin{proof}
    $O = (0, 0, 0)$ is in all of these sets.

    \begin{enumerate}[label={(\alph*)}]
        \item If $(x_{1}, y_{1}, z_{1})$ and $(x_{2}, y_{2}, z_{2})$ are in the set, $(x_{1} + x_{2}) + (y_{1} + y_{2}) + (z_{1} + z_{2}) = (x_{1} + y_{1} + z_{1}) + (x_{2} + y_{2} + z_{2}) = O + O = O$. Therefore, $(x_{1} + x_{2}, y_{1} + y_{2}, z_{1} + z_{2})$ is in the set.

              If $(x, y, z)$ is in the set, $cx + cy + cz = c(x + y + z) = c0 = 0$. Therefore, $c(x, y, z) = (cx, cy, cz)$ is in the set.

              Hence the set forms a subspace.
        \item If $(x_{1}, y_{1}, z_{1})$ and $(x_{2}, y_{2}, z_{2})$ are in the set, $x_{1} + x_{2} = y_{1} + y_{2}$ and $2(y_{1} + y_{2}) = 2y_{1} + 2y_{2} = z_{1} + z_{2}$. Therefore, $(x_{1} + x_{2}, y_{1} + y_{2}, z_{1} + z_{2})$ is in the set.

              If $(x, y, z)$ is in the set, $cx = cy$ and $2(cy) = c(2y) = cz$. Therefore, $c(x, y, z) = (cx, cy, cz)$ is in the set.

              Hence the set forms a subspace.
        \item If $(x_{1}, y_{1}, z_{1})$ and $(x_{2}, y_{2}, z_{2})$ are in the set, $(x_{1} + x_{2}) + (y_{1} + y_{2}) = (x_{1} + y_{1}) + (x_{2} + y_{2}) = 3z_{1} + 3z_{2} = 3(z_{1} + z_{2})$. Therefore, $(x_{1} + x_{2}, y_{1} + y_{2}, z_{1} + z_{2})$ is in the set.

              If $(x, y, z)$ is in the set $cx + cy = c(x + y) = c(3z) = 3(cz)$. Therefore, $c(x, y, z) = (cx, cy, cz)$ is in the set.

              Hence the set forms a subspace.
    \end{enumerate}
\end{proof}

\begin{exercise}
    If $U, W$ are subspaces of a vector space $V$, show that $U\cap W$ and $U + W$ are subspaces.
\end{exercise}

\begin{proof}
    Since $O\in U, W$, $O\in U\cap W, U + W$.

    Let $v_{1}, v_{2}$ be elements of $U\cap W$ and $c$ a number. Since $U$ is a subspace, $v_{1} + v_{2}\in U, cv_{1}\in U$. Since $W$ is a subspace, $v_{1} + v_{2}\in W, cv_{1}\in W$. Therefore, $v_{1} + v_{2}\in V\cap W$ and $cv_{1}\in V\cap W$. Therefore $U\cap W$ is a subspace.

    Let $t_{1}, t_{2}$ be elements of $U + W$. According to the definition of sum of subspaces, there exists vectors $u_{1}, u_{2}\in U$ and $w_{1}, w_{2}\in W$ such that $t_{1} = u_{1} + w_{1}$ and $t_{2} = u_{2} + w_{2}$.
    \begin{align*}
        t_{1} + t_{2} & = (u_{1} + w_{1}) + (u_{2} + w_{2})                                                     \\
                      & = \underbrace{(u_{1} + u_{2})}_{\in U} + \underbrace{(w_{1} + w_{2})}_{\in W} \in U + W \\
        ct_{1}        & = c(u_{1} + w_{1})                                                                      \\
                      & = \underbrace{cu_{1}}_{\in U} + \underbrace{cw_{1}}_{\in W} \in U + W.
    \end{align*}

    Therefore, $U + W$ is a subspace.
\end{proof}

\begin{exercise}
    Let $K$ be a subfield of a field $L$. Show that $L$ is a vector space over $K$. In particular, $\mathbf{C}$ and $\mathbf{R}$ are vector spaces over $\mathbf{Q}$.
\end{exercise}

\begin{proof}
    The field $L$ contains $0$ and $1$.

    Since $L$ is a field, addition in $L$ is associative, has an identity element ($0$), every element has an additive inverse, and commutative. So $L$ satisfies \textbf{VS 1, 2, 3, 4}.

    Let $k_{1}, k_{2}$ be elements of $K$, and $\ell_{1}, \ell_{2}$ elements of $L$.
    \begin{align*}
        k_{1}(\ell_{1} + \ell_{2}) & = \underbrace{k_{1}\ell_{1}}_{\in L} + \underbrace{k_{2}\ell_{2}}_{\in L} \in L, \\
        (k_{1} + k_{2})\ell_{1}    & = \underbrace{k_{1}\ell_{1}}_{\in L} + \underbrace{k_{2}\ell_{1}}_{\in L} \in L, \\
        (k_{1}k_{2})\ell_{1}       & = k_{1}(k_{2}\ell_{1}),                                                          \\
        1\ell_{1}                  & = \ell_{1}.
    \end{align*}

    Thus $L$ is a vector space over $K$.
\end{proof}

\begin{exercise}
    Let $K$ be the set of all numbers which can be written in the form $a + b\sqrt{2}$, where $a, b$ are rational numbers. Show that $K$ is a field.
\end{exercise}

\begin{proof}
    $K$ is a subset of $\mathbf{R}$, which is a field.

    $0 = 0 + 0\sqrt{2}, 1 = 1 + 0\sqrt{2}$. So $0$ and $1$ are in $K$.

    $a + b\sqrt{2} = 0$ if and only if $a = b = 0$ (proof by contradiction, follows the irrationality of $\sqrt{2}$).

    If $a_{1} + b_{1}\sqrt{2}$ and $a_{2} + b_{2}\sqrt{2}$ are in $K$,
    \begin{itemize}
        \item $(a_{1} + b_{1}\sqrt{2}) + (a_{2} + b_{2}\sqrt{2}) = (a_{1} + a_{2}) + (b_{1} + b_{2})\sqrt{2} \in K$.
        \item $(a_{1} + b_{1}\sqrt{2})\cdot (a_{2} + b_{2}\sqrt{2}) = (a_{1}a_{2} + 2b_{1}b_{2}) + (a_{1}b_{2} + a_{2}b_{1})\sqrt{2} \in K$.
        \item $(a_{1} + b_{1}\sqrt{2}) + ((-a_{1}) + (-b_{1})\sqrt{2}) = (a_{1} + (-a_{1})) + (b_{1} + (-b_{1}))\sqrt{2} = 0 + 0 = 0$.
        \item If $a_{1}$ and $b_{1}$ are not both zero, then $a_{1} + b_{1}\sqrt{2}\ne 0$ and
              \[
                  (a_{1} + b_{1}\sqrt{2})\frac{a_{1} - b_{1}\sqrt{2}}{{a_{1}}^{2} - 2{b_{1}}^{2}} = 1.
              \]
    \end{itemize}

    Thus, $K$ is a field.
\end{proof}

\begin{exercise}
    Let $K$ be the set of all numbers which can be written in the form $a + bi$, where $a, b$ are rational numbers. Show that $K$ is a field.
\end{exercise}

\begin{proof}
    $K$ is a subset of $\mathbf{C}$, which is a field.

    $0 = 0 + 0i, 1 = 1 + 0i$. So $0$ and $1$ are in $K$.

    $a + bi = 0$ if and only if $a = b = 0$.

    If $a_{1} + b_{1}i$ and $a_{2} + b_{2}i$ are in $K$,
    \begin{itemize}
        \item $(a_{1} + b_{1}i) + (a_{2} + b_{2}i) = (a_{1} + a_{2}) + (b_{1} + b_{2})i \in K$.
        \item $(a_{1} + b_{1}i)\cdot (a_{2} + b_{2}i) = (a_{1}a_{2} - b_{1}b_{2}) + (a_{1}b_{2} + a_{2}b_{1})i \in K$.
        \item $(a_{1} + b_{1}i) + ((-a_{1}) + (-b_{1})i) = (a_{1} + (-a_{1})) + (b_{1} + (-b_{1}))i = 0 + 0i = 0$.
        \item If $a_{1}$ and $b_{1}$ are not both zero, then $a_{1} + b_{1}i \ne 0$ and
              \[
                  (a_{1} + b_{1}i)\frac{a_{1} - b_{1}i}{{a_{1}}^{2} + {b_{1}}^{2}} = 1.
              \]
    \end{itemize}

    Thus, $K$ is a field.
\end{proof}

\begin{exercise}
    Let $c$ be a rational number $> 0$, and let $\gamma$ be a real number such that ${\gamma}^{2} = c$. Show that the set of all numbers which can be written in the form $a + b\gamma$, where $a, b$ are rational numbers, is a field.
\end{exercise}

\begin{proof}
    Denote the set by $K$.

    $0 = 0 + 0\gamma, 1 = 1 + 0\gamma$. So $0$ and $1$ are in $K$.

    If $a_{1} + b_{1}\gamma$ and $a_{2} + b_{2}\gamma$ are in $K$,
    \begin{itemize}
        \item $(a_{1} + b_{1}\gamma) + (a_{2} + b_{2}\gamma) = (a_{1} + a_{2}) + (b_{1} + b_{2})\gamma \in K$.
        \item $(a_{1} + b_{1}\gamma)\cdot (a_{2} + b_{2}\gamma) = (a_{1}a_{2} + cb_{1}b_{2}) + (a_{1}b_{2} + a_{2}b_{1})\gamma \in K$.
        \item $(a_{1} + b_{1}\gamma) + ((-a_{1}) + (-b_{1})\gamma) = (a_{1} + (-a_{1})) + (b_{1} + (-b_{1}))\gamma = 0 + 0\gamma = 0$.
        \item If $a + b\gamma \ne 0$ and $a - b\gamma\ne 0$, $(a + b\gamma)\frac{a - b\gamma}{{a}^{2} - c{b}^{2}} = 1$.

              If $a + b\gamma \ne 0$ and $a - b\gamma = 0$, $(a + b\gamma)\frac{1}{2a} = 2a\cdot\frac{1}{2a} = 1$.
    \end{itemize}

    Thus, $K$ is a field.
\end{proof}

\section{Bases}

\setcounter{exercise}{0}

\begin{exercise}
    Show that the following vectors are linearly independent (over $\mathbf{C}$ or $\mathbf{R}$).
    \begin{enumerate}[label={(\alph*)},itemsep=0pt]
        \item $(1, 1, 1)$ and $(0, 1, -2)$
        \item $(1, 0)$ and $(1, 1)$
        \item $(-1, 1, 0)$ and $(0, 1, 2)$
        \item $(2, -1)$ and $(1, 0)$
        \item $(\pi, 0)$ and $(0, 1)$
        \item $(1, 2)$ and $(1, 3)$
        \item $(1, 1, 0)$, $(1, 1, 1)$, and $(0, 1, -1)$,
        \item $(0, 1, 1)$, $(0, 2, 1)$, and $(1, 5, 3)$.
    \end{enumerate}
\end{exercise}

\begin{proof}
    \begin{enumerate}[label={(\alph*)},itemsep=0pt]
        \item Let $x, y$ be two numbers such that
              \[
                  x(1, 1, 1) + y(0, 1, -2) = (0, 0, 0).
              \]

              It follows that
              \[
                  \begin{split}
                      x = 0 \\
                      x + y = 0 \\
                      x - 2y = 0
                  \end{split}.
              \]

              Solving for $x$ and $y$, we obtain $x = y = 0$. Therefore, the two vectors are linearly independent.
        \item Let $x, y$ be two numbers such that
              \[
                  x(1, 0) + y(1, 1) = (0, 0).
              \]

              It follows that
              \[
                  \begin{split}
                      x + y = 0 \\
                      y = 0
                  \end{split}.
              \]

              Solving for $x$ and $y$, we obtain $x = y = 0$. Therefore, the two vectors are linearly independent.
        \item Let $x, y$ be two numbers such that
              \[
                  x(-1, 1, 0) + y(0, 1, 2) = (0, 0, 0).
              \]

              It follows that
              \[
                  \begin{split}
                      -x = 0 \\
                      x + y = 0 \\
                      2y = 0
                  \end{split}.
              \]

              Solving for $x$ and $y$, we obtain $x = y = 0$. Therefore, the two vectors are linear independent.
        \item Let $x, y$ be two numbers such that
              \[
                  x(2, -1) + y(1, 0) = (0, 0).
              \]

              It follows that
              \[
                  \begin{split}
                      2x + y = 0 \\
                      -x = 0
                  \end{split}.
              \]

              Solving for $x$ and $y$, we obtain $x = y = 0$. Therefore, the two vectors are linear independent.
        \item Let $x, y$ be two numbers such that
              \[
                  x(\pi, 0) + y(0, 1) = (0, 0).
              \]

              It follows that
              \[
                  \begin{split}
                      \pi x = 0 \\
                      y = 0
                  \end{split}.
              \]

              Solving for $x$ and $y$, we obtain $x = y = 0$. Therefore, the two vectors are linear independent.
        \item Let $x, y$ be two numbers such that
              \[
                  x(1, 2) + y(1, 3) = (0, 0)
              \]

              It follows that
              \[
                  \begin{split}
                      x + y = 0 \\
                      2x + 3y = 0
                  \end{split}.
              \]

              Subtract twice the 1st equation from the 2nd equation, we obtain $0 = (2x + 3y) - 2(x + y) = y$, so $y = 0$ and $x = 0 - y = 0$.

              Hence $x = y = 0$, which means the two vectors are linearly independent.
        \item Let $x, y, z$ be three numbers such that
              \[
                  x(1, 1, 0) + y(1, 1, 1) + z(0, 1, -1) = (0, 0, 0).
              \]

              It follows that
              \[
                  \begin{split}
                      x + y = 0 \\
                      x + y + z = 0 \\
                      y - z = 0
                  \end{split}.
              \]

              Subtracting the 1st equation from the 2nd equation yields $z = 0$. So that from the 3rd equation, $y = z = 0$. From the 2nd equation, $x = 0$.

              Hence $x = y = z = 0$, and the three vectors are linearly independent.
        \item Let $x, y, z$ be three numbers such that
              \[
                  x(0, 1, 1) + y(0, 2, 1) + z(1, 5, 3) = (0, 0, 0).
              \]

              It follows that
              \[
                  \begin{split}
                      z = 0 \\
                      x + 2y + 5z = 0 \\
                      x + y + 3z = 0
                  \end{split}.
              \]

              The 1st equation means $z = 0$. Subtracting the 3rd equation from the 2nd equation yields $0 = (x + 2y + 5z) - (x + y + 3z) = y + 2z$. Together with $z = 0$, we obtain $y = 0$. Also, $x = 0 - y - 3z = 0$.

              Hence $x = y = z = 0$, and the three vectors are linearly independent.
    \end{enumerate}
\end{proof}

\begin{exercise}
    Express the given vector $X$ as a linear combination of the given vectors $A, B$, and find the coordinates of $X$ with respect to $A, B$.
    \begin{enumerate}[label={(\alph*)},itemsep=0pt]
        \item $X = (1, 0), A = (1, 1), B = (0, 1)$
        \item $X = (2, 1), A = (1, -1), B = (1, 1)$
        \item $X = (1, 1), A = (2, 1), B = (-1, 0)$
        \item $X = (4, 3), A = (2, 1), B = (-1, 0)$
    \end{enumerate}
\end{exercise}

\begin{proof}
    \begin{enumerate}[label={(\alph*)},itemsep=0pt]
        \item Let $a, b$ be numbers such that
              \[
                  a(1, 1) + b(0, 1) = (1, 0).
              \]

              In term of coordinates,
              \[
                  \begin{split}
                      a = 1 \\
                      a + b = 0
                  \end{split}.
              \]

              Solving for $a$ and $b$, we obtain $a = 1$ and $b = -1$. Hence $X = 1\cdot A + (-1)\cdot B$.
        \item Let $a, b$ be numbers such that
              \[
                  a(1, -1) + b(1, 1) = (2, 1).
              \]

              In term of coordinates,
              \[
                  \begin{split}
                      a + b = 2 \\
                      -a + b = 1
                  \end{split}.
              \]

              Solving for $a$ and $b$, we obtain $a = \frac{1}{2}$ and $b = \frac{3}{2}$. Hence $X = \frac{1}{2}\cdot A + \frac{3}{2}\cdot B$.
        \item Let $a, b$ be numbers such that
              \[
                  a(2, 1) + b(-1, 0) = (1, 1).
              \]

              In term of coordinates,
              \[
                  \begin{split}
                      2a - b = 1 \\
                      a = 1
                  \end{split}.
              \]

              Solving for $a$ and $b$, we obtain $a = 1$ and $b = 1$. Hence $X = 1\cdot A + 1\cdot B$.
        \item Let $a, b$ be numbers such that
              \[
                  a(2, 1) + b(-1, 0) = (4, 3).
              \]

              In term of coordinates,
              \[
                  \begin{split}
                      2a - b = 4 \\
                      a = 3
                  \end{split}.
              \]

              Solving for $a$ and $b$, we obtain $a = 3$ and $b = 2$. Hence $X = 3\cdot A + 2\cdot B$.
    \end{enumerate}
\end{proof}

\begin{exercise}
    Find the coordinates of the vector $X$ with respect to the vectors $A, B, C$.
    \begin{enumerate}[label={(\alph*)},itemsep=0pt]
        \item $X = (1, 0, 0), A = (1, 1, 1), B = (-1, 1, 0), C = (1, 0, -1)$
        \item $X = (1, 1, 1), A = (0, 1, -1), B = (1, 1, 0), C = (1, 0, 2)$
        \item $X = (0, 0, 1), A = (1, 1, 1), B = (-1, 1, 0), C = (1, 0, -1)$
    \end{enumerate}
\end{exercise}

\begin{proof}
    \begin{enumerate}[label={(\alph*)},itemsep=0pt]
        \item Let $a, b, c$ be numbers such that
              \[
                  a(1, 1, 1) + b(-1, 1, 0) + c(1, 0, -1) = (1, 0, 0).
              \]

              In term of coordinates,
              \[
                  \begin{split}
                      a - b + c = 1 \\
                      a + b = 0 \\
                      a - c = 0
                  \end{split}
              \]

              Add the 2nd and 3rd equation to the 1st equation, we obtain $1 = (a - b + c) + (a + b) + (a - c) = 3a$, so $a = \frac{1}{3}$. From the 2nd and 3rd equation,  $b = \frac{-1}{3}$ and $c = \frac{1}{3}$.

              Hence $X = \frac{1}{3}\cdot A + \frac{-1}{3}\cdot B + \frac{1}{3}\cdot C$.
        \item Let $a, b, c$ be numbers such that
              \[
                  a(0, 1, -1) + b(1, 1, 0) + c(1, 0, 2) = (1, 1, 1).
              \]

              In term of coordinates,
              \[
                  \begin{split}
                      b + c = 1 \\
                      a + b = 1 \\
                      -a + 2c = 1
                  \end{split}.
              \]

              Adding the 2nd equation to and subtracting the 1st equation from the 3rd equation yield $1 = (-a + 2c) + (a + b) - (b + c) = c$. From the 2nd equation, $b = 0$. From the 3rd equation, $a = 1$.

              Hence $X = 1\cdot A + 0\cdot B + 1\cdot C$.
        \item Let $a, b, c$ be numbers such that
              \[
                  a(1, 1, 1) + b(-1, 1, 0) + c(1, 0, -1) = (0, 0, 1).
              \]

              In term of coordinates,
              \[
                  \begin{split}
                      a - b + c = 0 \\
                      a + b = 0 \\
                      a - c = 1
                  \end{split}
              \]

              Adding the 2nd and 3rd equation to the 1st equation yields $1 = (a - b + c) + (a + b) + (a - c) = 3a$, so $a = \frac{1}{3}$. From the 2nd equation, $b = \frac{-1}{3}$. From the 3rd equation, $c = \frac{-2}{3}$.

              Hence $X = \frac{1}{3}\cdot A + \frac{-1}{3}\cdot B + \frac{-2}{3}\cdot C$.
    \end{enumerate}
\end{proof}

\begin{exercise}
    Let $(a, b)$ and $(c, d)$ be two vectors in the plane. If $ad - bc = 0$, show that they are linearly dependent. If $ad - bc\ne 0$, show that they are linearly independent.
\end{exercise}

\begin{proof}
    \begin{itemize}
        \item $ad - bc = 0$.

              If $a = b = 0$, the two vectors are linearly dependent.

              Otherwise, $a$ and $b$ are not all zero. No matter which number is non-zero, the following linear combinations hold
              \[
                  \begin{split}
                      d(a, b) + (-b)(c, d) = (ad - bc, db - bd) = (0, 0), \\
                      c(a, b) + (-a)(c, d) = (ca - ac, bc - ad) = (0, 0).
                  \end{split}
              \]

              Hence, the two vectors are linear dependent.
        \item $ad - bc\ne 0$.

              Let $x, y$ be numbers such that $x(a, b) + y(c, d) = (0, 0)$. In term of coordinates,
              \[
                  \begin{split}
                      ax + cy = 0 \\
                      bx + dy = 0
                  \end{split}.
              \]

              We deduce that
              \[
                  \begin{split}
                      0 = d(ax + cy) - c(bx + dy) = (ad - bc)x \\
                      0 = b(ax + cy) - a(bx + dy) = (bc - ad)y
                  \end{split}.
              \]

              Since $ad - bc\ne 0$, $x = y = 0$. Therefore, the two vectors are linearly independent.
    \end{itemize}
\end{proof}

\begin{exercise}
    Consider the vector space of all functions of a variable $t$. Show that the following pairs of functions are linearly independent.
    \begin{enumerate}[label={(\alph*)},itemsep=0pt]
        \item $1, t$
        \item $t, t^{2}$
        \item $t, t^{4}$
        \item $e^{t}, t$
        \item $t{e}^{t}, e^{2t}$
        \item $\sin t, \cos t$
        \item $t, \sin t$
        \item $\sin t, \sin 2t$
        \item $\cos t, \cos 3t$
    \end{enumerate}
\end{exercise}

\begin{proof}
    \begin{enumerate}[label={(\alph*)}]
        \item Let $a, b$ be numbers such that $a\cdot 1 + b\cdot t = 0$ for every $t$.

              When $t = 0$, $a = 0$. When $t = 1$, $a + b = 0$. So $a = b = 0$. Hence $1, t$ are linearly independent.
        \item Let $a, b$ be numbers such that $a\cdot t + b\cdot t^{2} = 0$ for every $t$.

              When $t\ne 0$, $a + b\cdot t = 0$. According to (a), $a = b = 0$. Hence $t, t^{2}$ are linearly independent.
        \item Let $a, b$ be numbers such that $a\cdot t + b\cdot t^{4} = 0$ for every $t$.

              When $t\ne 0$, $a + b{t}^{3} = 0$. When $t = 1$, $a + b = 0$. When $t = -1$, $a - b = 0$. So $a = b = 0$. Hence $t, t^{4}$ are linearly independent.
        \item Let $a, b$ be numbers such that $a{e}^{t} + b\cdot t = 0$ for every $t$.

              Differentiating yields $a{e}^{t} + b = 0$. Therefore $b = bt$ for every $t$, which means $b = 0$. Substitute $b = 0$ to the linear combination, we obtain $a{e}^{t} = 0$, which yields $a = 0$.

              Hence $e^{t}, t$ are linearly independent.
        \item Let $a, b$ be numbers such that $a\cdot t{e}^{t} + b\cdot e^{2t} = 0$ for every $t$.

              Substituting $t = 0$ yields $b = 0$. Substituting $t = 1$ yields $a\cdot e + b\cdot e^{2} = 0$. So $a = b = 0$.

              Hence $t{e}^{t}, e^{2t}$ are linearly independent.
        \item Let $a, b$ be numbers such that $a\sin t + b\cos t = 0$ for every $t$.

              Differentiating yields $a\cos t - b\sin t = 0$. So $0 = \cos t (a\sin t + b\cos t) - \sin t(a\cos t - b\sin t) = b({(\cos t)}^{2} + {(\sin t)}^{2}) + a(\cos t\sin t - \sin t\cos t) = b$. Subtituting $b = 0$ to the linear combination yields $a = 0$.

              Hence $\sin t, \cos t$ are linearly independent.
        \item Let $a, b$ be numbers such that $at + b\sin t = 0$ for every $t$.

              Differentiating yields $a + b\cos t = 0$. When $t = 0$, $a + b = 0$. When $t = \frac{\pi}{2}$, $a = 0$. So $a = b = 0$.

              Hence $t, \sin t$ are linearly independent.
        \item Let $a, b$ be numbers such that $a\sin t + b\sin 2t = 0$ for every $t$.

              When $t = \frac{\pi}{2}$, $a = 0$. When $t = \frac{\pi}{4}$, $\frac{\sqrt{2}}{2}\cdot a + b = 0$. From these relations, we deduce that $a = b = 0$.

              Hence $\sin t, \sin 2t$ are linearly independent.
        \item Let $a, b$ be numbers such that $a\cos t + b\cos 3t = 0$ for every $t$.

              When $t = 0$, $a + b = 0$. When $t = \frac{\pi}{4}$, $\frac{\sqrt{2}}{2}\cdot a + \frac{-\sqrt{2}}{2}b = 0$. So $a = b = 0$.

              Hence $\cos t, \cos 3t$ are linearly independent.
    \end{enumerate}
\end{proof}

\begin{exercise}
    Consider the vector space of functions defined for $t > 0$. Show that the following pairs of functions are linearly independent.
    \begin{enumerate}[label={(\alph*)},itemsep=0pt]
        \item $t, 1/t$
        \item $e^{t}, \log t$
    \end{enumerate}
\end{exercise}

\begin{proof}
    \begin{enumerate}[label={(\alph*)},itemsep=0pt]
        \item Let $a, b$ be numbers such that $at + b/t = 0$ for every $t > 0$.

              Differentiate both sides, we obtain $a - b/t^{2} = 0$ for every $t > 0$. Equivalently, $a{t}^{2} - b = 0$ for every $t > 0$. Multiply both sides by $t$, we obtain  $at^{2} + b = 0$ for every $t > 0$. So $0 = (at^{2} + b) - (at^{2} - b) = 2b$, and $b = 0$. Subtitute $b = 0$ to the linear combination, we obtain $a = 0$.

              Hence $t, 1/t$ are linearly independent.
        \item Let $a, b$ be numbers such that $a{e}^{t} + b\log t = 0$ for every $t > 0$.

              When $t = 1$, $ae = 0$, which implies $a = 0$. When $t = e$, $a{e}^{e} + b = 0$. Together with $a = 0$, we obtain $b = 0$.

              Hence $e^{t}, \log t$ are linearly independent.
    \end{enumerate}
\end{proof}

\begin{exercise}\label{chapter1:section2:exercise7}
    What are the coordinates of the function $3\sin t + 5\cos t = f(t)$ with respect to the basis $\{ \sin t, \cos t \}$?
\end{exercise}

\begin{proof}
    The coordinates are $(3, 5)$.
\end{proof}

\begin{exercise}
    Let $D$ be the derivative $d/dt$. Let $f(t)$ be as in Exercise~\ref{chapter1:section2:exercise7}. What are the coordinates of the function $Df(t)$ with respect to the basis of Exercise~\ref{chapter1:section2:exercise7}?
\end{exercise}

\begin{proof}
    $Df(t) = 3\cos t - 5\sin t$. The coordinates of $Df(t)$ are $(-5, 3)$.
\end{proof}

\begin{exercise}
    Let $A_{1},\ldots, A_{r}$ be vectors in $\mathbb{R}^{n}$ and assume that they are mutually perpendicular (i.e.\@ any two of them are perpendicular), and that none of them is equal to $O$. Prove that they are linearly independent.
\end{exercise}

\begin{proof}
    Let $a_{1},\ldots, a_{r}$ be real numbers such that
    \[
        a_{1}A_{1} + \cdots + a_{r}A_{r} = O.
    \]

    Since $A_{i}$ is perpendicular to any $A_{j}$ ($j\ne i$)
    \[
        0 = A_{i}\cdot O = A_{i}(a_{1}A_{1} + \cdots + a_{r}A_{r}) = a_{i}(A_{i}\cdot A_{i}).
    \]

    Suppose that $A_{i} = (x_{i.1}, x_{i.2}, \ldots, x_{i.n})$, then $A_{i}\cdot A_{i} = {x_{i.1}}^{2} + \cdots + {x_{i.n}}^{2}$. Since $A_{i}\ne O$, the coordinates of $A_{i}$ are not all zero, therefore $A_{i}\cdot A_{i} > 0$ and $a_{i} = 0$.

    Hence $a_{i} = 0$ for every $1\le i \le r$. Thus, $A_{1}, \ldots, A_{r}$ are linearly independent.
\end{proof}

\begin{exercise}
    Let $v, w$ be elements of a vector space and assume that $v\ne O$. If $v, w$ are linearly dependent, show that there is a number $a$ such that $w = av$.
\end{exercise}

\begin{proof}
    Since $v, w$ are linearly dependent, there exist numbers $x, y$ such that $xv + yw = O$ and $x, y$ are not both zero.

    Assume that $y = 0$, then $xv = O$. Because $v\ne O$, $x = 0$. So $x = y = 0$, which contradicts the definition of $x$ and $y$. Therefore $y\ne 0$, and there exists number $y^{-1}$ such that $yy^{-1} = y^{-1}y = 1$. So $(y^{-1}x)v + w = O$, equivalently, $w = (-y^{-1}x)v$.

    Thus, there is a number $a$ such that $w = av$.
\end{proof}

\section{Dimension of a vector space}

There is no exercises in this section.

\section{Sums and Direct sums}

\begin{exercise}
\end{exercise}

\begin{proof}
\end{proof}

\begin{exercise}
\end{exercise}

\begin{proof}
\end{proof}

\begin{exercise}
\end{exercise}

\begin{proof}
\end{proof}

\begin{exercise}
\end{exercise}

\begin{proof}
\end{proof}
