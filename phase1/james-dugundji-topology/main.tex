\documentclass[12pt,oneside]{book}

\usepackage[left=2cm,right=2cm,top=2.5cm,bottom=2.5cm]{geometry}
\usepackage[unicode=true,hidelinks]{hyperref}

\usepackage{amsmath}
\usepackage{amsfonts}
\usepackage{amssymb}
\usepackage{amsthm}
\usepackage{mathtools}
\usepackage{mathrsfs}
\usepackage{cases}
\usepackage{pgf,tikz}
\usepackage{pgfplots}
\usetikzlibrary{shapes}
\usetikzlibrary{shapes.arrows}
\usetikzlibrary{arrows.meta}
\usetikzlibrary{calc}
\usetikzlibrary{math}
\usetikzlibrary{decorations,
    decorations.footprints,
    decorations.fractals,
    decorations.markings,
    decorations.pathmorphing,
    decorations.pathreplacing,
    decorations.shapes,
    decorations.shapes,
    decorations.text}
\usetikzlibrary{positioning}
\usetikzlibrary{angles}
\usetikzlibrary{matrix}
\usepackage{tikz-cd}

\usepackage{fancyhdr}
\usepackage{xcolor}
\usepackage{titlesec}
\usepackage{indentfirst}
\usepackage{chngcntr}
\usepackage{caption}
\usepackage{subcaption}
\usepackage{booktabs}
\usepackage[inline]{enumitem}
\usepackage{setspace}

\setstretch{1.4142}
\setcounter{chapter}{0}
\setcounter{section}{0}
\counterwithout{section}{chapter}

\captionsetup{labelfont={bf},labelsep=period}
\counterwithin{figure}{chapter}
\counterwithin{table}{chapter}

\theoremstyle{definition}

\newtheorem{exercise}{Exercise}
\counterwithin{exercise}{section}
\newenvironment{sqcases}{%
    \matrix@check\sqcases\env@sqcases
}{%
    \endarray\right.%
}
\def\env@sqcases{%
\let\@ifnextchar\new@ifnextchar
\left\lbrack{}
\def\arraystretch{1.2}%
\array{@{}l@{\quad}l@{}}%
}

\newcommand{\tr}[1]{\left({#1}\right)}
\newcommand{\card}[1]{\left\vert{#1}\right\vert}
\newcommand{\rank}{\text{rank}}
\newcommand{\abs}[1]{\left\vert{#1}\right\vert}
\newcommand{\norm}[1]{\left\Vert{#1}\right\Vert}
\newcommand{\anglebracket}[1]{\left\langle{#1}\right\rangle}
\newcommand{\floor}[1]{\left\lfloor{#1}\right\rfloor}
\newcommand{\ceil}[1]{\left\lceil{#1}\right\rceil}
\newcommand{\openinterval}[1]{\left]{#1}\right[}
\newcommand{\closedinterval}[1]{\left[{#1}\right]}
\newcommand{\halfopenleft}[1]{\left]{#1}\right]}
\newcommand{\halfopenright}[1]{\left[{#1}\right[}
\newcommand{\set}[1]{\left\{{#1}\right\}}
\newcommand{\tuple}[1]{\left({#1}\right)}
\newenvironment{section*}[1]{% \begin{section*}{section title}....\end{section*}
  \section*{#1}
  \renewcommand\thesection{\thechapter.S}
  \setcounter{exercise}{0}}{}

\title{James Dugundji's Topology: Notes and Problems}
\author{Ngo Quang Duong}
\date{\today}

\begin{document}

\maketitle

\tableofcontents

\documentclass[class=mike-apostol-mathematical-analysis,crop=false]{standalone}

\begin{document}

\chapter{The Real and Complex Number Systems}

\section{Axioms of real numbers}

\par I can assure that most people familiar with real numbers. We have taken many properties of real numbers for granted. But what is real number, are they real? Turns out, these questions are really difficult.

\par If all you have ever wanted is a definition of real numbers, then you could use the following axiomatic definition.

\par Real numbers are elements of a set $\mathbb{R}$, which satisfy the following properties
\begin{enumerate}[label = (\roman*)]
    \item $\mathbb{R}$ is a field under addition and multiplication.
    \item $\mathbb{R}$ is totally ordered.
    \item Order in $\mathbb{R}$ is preserved under addition and multiplication (with non-negative real number).
    \item Every upper-bounded non-empty set of $\mathbb{R}$ has a least upper bound.
\end{enumerate}

\par To those who ask ``Are real numbers real?\@'', we can establish a model (a mathematical structure) that satisfies every axiom above. So in this sense, or since the existence of such model, I would answer ``yes''. Since 19th century, mathematicians have given several constructions of the real numbers. IMHO, the two most notable constructions are \textit{Dedekind cuts} and \textit{Cauchy sequences}. In the following section, we will try to reproduce the construction by \textit{Dedekind cuts}.

\section{Construction of the real numbers by Dedekind cuts}\addcontentsline{toc}{section}{[Note] Construction of the real numbers by Dedekind cuts}

\par We will give the definition of Dedekind cuts and construct a model that satisfies the real numbers axioms.

\subsection*{Dedekind cuts}

\par To define Dedekind cuts, we will use rational numbers as the basis in the sense that the set of rational numbers satisfies all real numbers axioms, except for the least upper bound axiom.

\begin{definition}[Dedekind cuts]
    A Dedekind cut $ A$ is a subset of $\mathbb{Q}$ that:
    \begin{enumerate}[label = (DC\arabic*)]
        \item $ A\ne\varnothing$; in other words, $ A$ is not empty.
        \item $ A\neq\mathbb{Q}$; in other words, $ A$ is not the entire set of rational numbers.
        \item $\forall x\left(x\in A \rightarrow \exists y \left( y\in A \wedge x < y \right)\right)$; in other words, $ A$ has no maximum element.
        \item $\forall x\in A\left(\forall y( y < x \rightarrow y\in A)\right)$; in other words, $ A$ is downward closed.
    \end{enumerate}
\end{definition}

\par Our goal is from the definition of Dedekind cuts as well as operations (addition and multiplication) and relations (less than or equal) on them, we can prove that Dedekind cuts satisfy the real numbers axioms.

\begin{theorem}
    The set of all Dedekind cuts is totally ordered with $\subseteq$ relation.
\end{theorem}

\begin{proof}
    \par Let $ A$ and $ B$ be two Dedekind cuts.
    \par Suppose that $ A\ne B$.
    \par Without loss of generality, let's suppose that there exists $b\in B$ such that $b\notin A$.
    \par $b\in B$, then $b$ is a rational number and an upper bound of $ A$.
    \par Let $a$ be an arbitrary element of $ A$, then $a\le b$. According to (DC4), $a\in B$. Hence $\forall a(a\in A\rightarrow a\in B)$.
    \par Therefore, $ A$ is a proper subset of $ B$.
    \bigskip
    \par So for arbitrary two Dedekind cuts $ A$, $ B$, one of the following holds: $ A\subseteq B,  B\subseteq A$. Hence, the set of all Dedekind cuts is totally ordered with $\subseteq$ relation.
\end{proof}

\par For convenience, in this section, we use the following notation:
\[
    {0}^{*} = \{ x : x\in\mathbb{Q} \wedge x < 0 \}
\]

\begin{definition}
    A Dedekind cut $ A$ is called:
    \begin{enumerate}[label = (\roman*)]
        \item positive if $ A$ is a proper superset of ${0}^{*}$,
        \item negative if $ A$ is a proper subset of ${0}^{*}$,
        \item non-positive if $ A\subset {0}^{*}$,
        \item non-negative if $ A\supseteq {0}^{*}$.
    \end{enumerate}
\end{definition}

\begin{definition}[Rational and irrational]
    A Dedekind cut $ A$ is called:
    \begin{enumerate}[label = (\roman*)]
        \item rational if $\mathbb{Q}\setminus A$ has minimum element,
        \item irrational if $\mathbb{Q}\setminus A$ has no minimum element.
    \end{enumerate}
\end{definition}

\par The following example gives us an example of rational cut, and an example of irrational cut.

\begin{example}
    \[
        A = \{ x\in\mathbb{Q}: x < 1 \}
    \]
    \par is a rational cut.
    \[
        B = \{ x\in\mathbb{Q}: {x}^{2} < 2 \} \cup \mathbb{Q}^{-}
    \]
    \par is an irrational cut.
\end{example}

\begin{proof}
    \par $\mathbb{Q}\setminus A = \{ x\in\mathbb{Q}: x\ge 1 \}$ has minimum element, which is $1$. So $ A$ is a rational cut.
    \bigskip
    \par $\mathbb{Q}\setminus B = \{ x\in\mathbb{Q}: {x}^{2}\ge 2 \wedge x > 0 \}$.
    \par Since there is no rational number $r$ of which square equals $2$, then $\mathbb{Q}\setminus B = \{ x\in\mathbb{Q}: {x}^{2} > 2 \wedge x > 0 \}$ (change from $\ge$ to $>$).
    \par Let $q\in\mathbb{Q}\setminus B$, choose $r = \frac{q}{2} + \frac{1}{q}$.
    \begin{align*}
        \frac{q}{2} + \frac{1}{q} & = -\frac{q}{2} + \frac{1}{q} + q                       \\
                                  & = \frac{2 - {q}^{2}}{2q} + q                           \\
                                  & < q \quad\text{(Since $q > 0$ and $2 - {q}^{2} < 0$)}.
    \end{align*}
    \begin{align*}
        {r}^{2} & = {\left(\frac{q}{2} + \frac{1}{q}\right)}^{2} = \frac{q^{2}}{4} + \frac{1}{q^{2}} + 1                                                         \\
                & = \frac{q^{2}}{4} + \frac{1}{q^{2}} - 1 + 2 = {\left(\frac{q}{2} - \frac{1}{q}\right)}^{2} + 2 = {\left( \frac{q^{2} - 2}{2q} \right)}^{2} + 2 \\
                & > 2
    \end{align*}
    \par Therefore, $\forall q(q\in\mathbb{Q}\setminus B \rightarrow \exists r( r\in\mathbb{Q}\setminus B \wedge r < q ))$. Hence $\mathbb{Q}\setminus B$ has no minimal element. According to the definition, $ B$ is an irrational cut.
\end{proof}

\par Next, we will define addition and multiplication with Dedekind cuts.

\begin{definition}[Addition]
    \par $ A,  B$ are Dedekind cuts.
    \[
        A +  B = \{ x + y : x\in A \wedge y\in B \}.
    \]
\end{definition}

\par However, we have to prove that $ A +  B$ is also a Dedekind cut.

\begin{proof}
    \begin{enumerate}[label = (\roman*)]
        \item Since $ A\ne\varnothing$ and $ B\ne\varnothing$, then there exists $a\in A$ and $b\in B$. By definition of $ A +  B$, we obtain that $a + b \in  A +  B$. This implies that $ A +  B$ is not empty.
        \item A Dedekind cut is downward closed and not the entire set of rational numbers, then it is upper bounded.
              \par Therefore, $ A$ and $ B$ are upper bounded. Let $a$ be an upper bound of $ A$, $b$ be an upper bound of $ B$.
              \par $\forall x\in A\forall y\in B$, then $x + y \le a + b$, which means $ A +  B$ is upper bounded.
              \par Hence $ A +  B\ne\mathbb{Q}$.
        \item Let $c$ be an element of $ A +  B$. According to the definition of $ A +  B$, there exists $a\in A$ and $b\in B$ such that $a + b = c$.
              \par According to (DC3), there exists $a_{0}\in A$ such that $a < a_{0}$, and there exists $b_{0}\in B$ such that $b < b_{0}$.
              \par $c = a + b < a_{0} + b_{0}$. According to the definition of $ A +  B$, $a_{0} + b_{0} \in  A +  B$. Hence $ A +  B$ has no maximum element.
        \item Let $c$ be an element of $ A +  B$. According to the definition of $ A +  B$, there exists $a\in A$ and $b\in B$ such that $a + b = c$.
              \par Let $c_{1}$ be a rational number such that $c_{1} < c$.
              \par According to (DC4)
              \[
                  a + \frac{c_{1} - c}{2}\in A\qquad\text{and}\qquad b + \frac{c_{1} - c}{2}\in B
              \]
              \par Hence
              \[
                  \left( a + \dfrac{c_{1} - c}{2} \right) + \left( b + \dfrac{c_{1} - 2}{2} \right) \in  A +  B
              \]
              \par Therefore
              \[
                  \left( a + \dfrac{c_{1} - c}{2} \right) + \left( b + \dfrac{c_{1} - c}{2} \right) = (a + b) + (c_{1} - c) = c + (c_{1} - c) = c_{1}
              \]
              \par Hence $ A +  B$ is downward closed.
    \end{enumerate}
    \par In conclusion, $ A +  B$ is a Dedekind cut.
\end{proof}

\par I have difficulty defining multiplication since there are positive numbers and negative numbers. So I define additive inverse/negation of a cut.

\begin{definition}[Additive inverse/Negation]
    \par Let $ A$ be a Dedekind cut.
    % % an alternative definition of additive inverse
    %\[
    %    - A = {\bigcup}_{x\in A}\{ y: y\in\mathbb{Q} \wedge y < -x \}
    %\]
    \[
        - A = \{ b - a' : b < 0 \wedge b\in\mathbb{Q} \wedge a'\in\mathbb{Q}\setminus A \}
    \]
\end{definition}

\begin{proof}
    \begin{enumerate}[label = (\roman*)]
        \item Since $ A\ne\mathbb{Q}$ then $\mathbb{Q}\setminus A$ is not empty. Therefore, $- A$ is not empty.
        \item Since $ A$ is downward closed and has no maximum element, then $\mathbb{Q}\setminus A$ contains all upper bounds of $ A$.
              \par Let $a\in A$, then $a$ is a lower bound of $\mathbb{Q}\setminus A$.
              \par $\forall b < 0 \wedge b\in\mathbb{Q}, \forall a'\in\mathbb{Q}\setminus A$,
              \[
                  b - a' < -a' < -a.
              \]
              \par Therefore $- A$ is upper bounded. So $- A\ne\mathbb{Q}$.
        \item Let $c$ be an arbitrary element of $- A$. According to the definition of $- A$, there exists $b < 0\wedge b\in\mathbb{Q}$ and $a'\in\mathbb{Q}\setminus A$ such that $b - a' = c$.
              \par Choose $c' = \dfrac{b}{2} - a'$. Due to the definition of $- A$, $c'\in - A$. On the other hand
              \[
                  c = b - a' < \dfrac{b}{2} - a' = c'.
              \]
              \par Therefore, $- A$ does not have maximum element.
        \item Let $c$ be an arbitrary element of $- A$. According to the definition of $- A$, there exists $b < 0\wedge b\in\mathbb{Q}$ and $a'\in\mathbb{Q}\setminus A$ such that $b - a' = c$.
              \par Let $c_{0}$ be a rational number such that $c_{0} < c$.
              \[
                  c_{0} = c + (c_{0} - c) = (b - a') + (c_{0} - c) = \underbrace{(b + c_{0} - c)}_{< 0, \in\mathbb{Q}} + a'
              \]
              \par So $c_{0}\in - A$. Hence $c_{0}\in - A$.
    \end{enumerate}
    \par In conclusion, $- A$ is a Dedekind cut.
\end{proof}

\begin{definition}[Multiplication]
    \par Let $A, B$ be Dedekind cuts.
    \par $A\cdot B$ is defined as the following.
    \par If $A = {0}^{*}$ or $B = {0}^{*}$, then
    \[
        A\cdot B = {0}^{*}.
    \]
    \par If $A\supset\supset{0}^{*}$ and $B\supset\supset{0}^{*}$
    \[
        A\cdot B = \{ a\cdot b : a\in A\wedge a\ge 0 \wedge b\in B\wedge b\ge 0 \} \cup \mathbb{Q}^{-}.
    \]
    \par If $A\subset\subset{0}^{*}$ and $B\subset\subset{0}^{*}$
    \[
        A\cdot B = (-A)\cdot (-B).
    \]
    \par If $A\subset\subset{0}^{*}$ and $B\supset\supset{0}^{*}$
    \[
        A\cdot B = -\left((-A)\cdot B\right).
    \]
    \par If $A\supset\supset{0}^{*}$ and $B\subset\subset{0}^{*}$
    \[
        A\cdot B = -\left(A\cdot (-B)\right).
    \]
\end{definition}

\par We will show that $A\cdot B$ is also a Dedekind cut. But, thanks to the definition of negation, we only have to cover that first case: $A\supset\supset{0}^{*}$ and $B\supset\supset{0}^{*}$.

\begin{proof}
    \begin{enumerate}[label = (\roman*)]
        \item Since $A\cdot B$ is a superset of $\mathbb{Q}^{-}$, then $A\cdot B$ is not empty.
        \item Let $a_{0}$ be an upper bound of $A$, $b_{0}$ be an upper bound of $B$.
              \par Since $A\supset\supset{0}^{*}$ and $B\supset\supset{0}^{*}$, then $a_{0}\ge 0$ and $b_{0}\ge 0$.
              \par Then for any non-negative elements $a$ and $b$ of $A$ and $B$, $a\cdot b \le a_{0}\cdot b_{0}$.
              \par Hence $a_{0}\cdot b_{0}$ is an upper bound of $A\cdot B$, which implies that $A\cdot B\ne\mathbb{Q}$.
        \item Let $c$ be an arbitrary element of $A\cdot B$.
              \par If $c$ is negative or zero, then there exists an element which is greater than $c$, since $A\supset\supset {0}^{*}$ and $B\supset\supset {0}^{*}$ (zero is not their maximum element).
              \par Otherwise, $c$ is positive, then there exists $a\in A$ and $a > 0$, $b\in B$ and $b > 0$ such that $a\cdot b = c$. Due to (DC3), there exists $a_{0} > a > 0$ and $a_{0}\in A$, $b_{0} > b > 0$ and $b_{0}\in B$.
              \par Furthermore, $a_{0}\cdot b_{0} > a\cdot b$ and $a_{0}\cdot b_{0}$ according to the definition of $A\cdot B$.
              \par So $A\cdot B$ has no maximum element.
        \item Let $c$ be an arbitrary element of $A\cdot B$.
              \par Let $d$ be a rational number such that $d < c$.
              \par If $d$ is non-positive, then $d\in A\cdot B$, since $A\cdot B$ contains $0$ and is a superset of $\mathbb{Q}^{-}$.
              \par Otherwise, $d$ is positive, then $c$ is also positive. Since $c$ is positive, there exists $a\in A$ and $a > 0$, $b\in B$ and $b > 0$ such that $c = a\cdot b$.
              \[
                  d = c - (c - d) = a\cdot b - (c - d) = a\cdot\left(b - \frac{c - d}{a}\right)
              \]
              \par Since $a\in A$ and $a > 0$, $b - \dfrac{c - d}{a}\in B$ (due to (DC4)) and $b - \dfrac{c - d}{a} > 0$, then $d \in A\cdot B$.
              \par Hence $A\cdot B$ is downward closed.
    \end{enumerate}
    \par In conclusion, $A\cdot B$ is a Dedekind cut.
\end{proof}

\subsection*{Properties}

\par In this subsection, $\mathbb{R}$ is the set of all Dedekind cuts.

\begin{theorem}
    $\mathbb{R}$ is a field with the defined addition and multiplication.
\end{theorem}

\begin{theorem}
    $\mathbb{R}$ is a field with characteristic zero.
\end{theorem}

\begin{theorem}
    The embedding $\iota: \mathbb{Q} \to \mathbb{R}, r \mapsto {r}^{*}$ is an order-preserving field monomorphism.
    $\mathbb{R}$ is totally ordered with relation $\leq$.
\end{theorem}

\section{Complex numbers}

\end{document}

\chapter{Manifolds}

\section{Manifolds}

\begin{problem}{5.1}[The real line with two origins]
Let \( A \) and \( B \) be two points not on the real line \( \mathbb{R} \). Consider the set \( S = (\mathbb{R} \smallsetminus \left\{0\right\}) \cup \left\{A, B\right\} \).

For any two positive real numbers \( c, d \), define
\[
	I_{A}(-c, d) = \openinterval{-c, 0} \cup \left\{A\right\} \cup \openinterval{0, d}
\]

and similarly for \( I_{B}(-c, d) \), with \( B \) instead of \( A \). Define a topology on \( S \) as follows: On \( \mathbb{R} \smallsetminus \left\{0\right\} \), use the subspace topology inherited from \( \mathbb{R} \), with open intervals as a basis. A basis of neighborhoods at \( A \) is the set \( \left\{ I_{A}(-c, d) \mid c, d > 0 \right\} \); similarly, a basis of neighborhoods at \( B \) is \( \left\{ I_{B}(-c, d) \mid c, d > 0 \right\} \).

\begin{enumerate}[label={(\alph*)}]
	\item Prove that the map \( h: I_{A}(-c,d) \to \openinterval{-c, d} \) defined by
	      \begin{align*}
		      h(x) = x & \qquad \text{for \( x \in \openinterval{-c, 0} \cup \openinterval{0, d} \)}, \\
		      h(A) = 0
	      \end{align*}

	      is a homeomorphism.
	\item Show that \( S \) is locally Euclidean and second countable, but not Hausdorff.
\end{enumerate}
\end{problem}

\begin{proof}
	\begin{enumerate}[label={(\alph*)}]
		\item By definition, \( h \) is a bijection.

		      Let \( V \) be an open subset of \( \openinterval{-c, d} \) then \( V \) is open in \( \mathbb{R} \) as \( \openinterval{-c, d} \) is open in \( \mathbb{R} \). Hence \( V = \bigcup_{\alpha} V_{\alpha} \) in which each \( V_{\alpha} \) is an open interval. If \( V_{\alpha} \) doesn't contain \( 0 \) then either \( V_{\alpha} \subseteq \openinterval{-c, 0} \) or \( V_{\alpha} \subseteq \openinterval{0, d} \) so \( h^{-1}(V_{\alpha}) = V_{\alpha} \) which is open in \( I_{A}(-c, d) \). If \( V_{\alpha} \) contains \( 0 \) then \( V_{\alpha} \) is of the form \( \openinterval{-c_{\alpha}, d_{\alpha}} \) in which \( 0 < c_{\alpha} \leq c \) and \( 0 < d_{\alpha} \leq d \), then \( h^{-1}(V_{\alpha}) = I_{A}(-c_{\alpha}, d_{\alpha}) \), which is open in \( I_{A}(-c, d) \). Hence \( h^{-1}(V) = h^{-1}\left(\bigcup_{\alpha} V_{\alpha}\right) = \bigcup_{\alpha} h^{-1}(V_{\alpha}) \) is open in \( I_{A}(-c, d) \), so \( h \) is continuous.

		      Let \( U \) be an open subset of \( I_{A}(-c, d) \). If \( U \) doesn't contain \( A \) then \( U \) is an open subset of \( \openinterval{-c, 0} \cup \openinterval{0, d} \) so \( h(U) = U \), which is open in \( \openinterval{-c, d} \). Otherwise \( U \) contains \( A \) then \( U \) contains a maximal set of the form \( I_{A}(-c_{U}, y_{U}) \). If \( U \smallsetminus I_{A}(-c_{U}, y_{U}) \ne \varnothing \) then each point \( p \) in \( U \smallsetminus I_{A}(-c_{U}, y_{U}) \) is contained in some open set \( U_{p} \) that is disjoint from \( I_{A}(-c_{U}, y_{U}) \) (otherwise, it will contradict maximality of \( I_{A}(-c_{U}, y_{U}) \)). Since \( U_{p} \) is an open subset of \( \openinterval{-c, 0} \cup \openinterval{d, 0} \), it follows that \( U_{p} \) is a union of open intervals. Therefore \( h(U_{p}) \) is open in \( \openinterval{-c, d} \). Moreover, \( h(I_{A}(-c_{U}, d_{U})) = \openinterval{-c_{U}, d_{U}} \) is open in \( \openinterval{-c, d} \), hence \( h \) is open.

		      Thus \( h \) is a homeomorphism as it is a bicontinuous bijection.
		\item At the point \( A \) (or \(B\)), any neighborhood of the form \( I_{A}(-c, d) \) (or \( I_{B}(-c, d) \)) is homeomorphic to \( \openinterval{-c, d} \) according to part (a).

		      At a point \( p \in \mathbb{R} \smallsetminus \left\{0\right\} \), the neighborhood \( \openinterval{\min\{ p/2; 2p \}, \max\{ p/2; 2p \}} \) is homeomorphic to \( \openinterval{\min\{ p/2; 2p \}, \max\{ p/2; 2p \}} \).

		      Therefore \( S \) is locally Euclidean of dimension \(1\).

		      Let \( \mathscr{B} \) be the collection consisting of \( I_{A}(-c, d) \) for positive rational numbers \( c, d \) and the intersections of \( \mathbb{R} \smallsetminus \left\{0\right\} \) with open intervals \( \openinterval{x, y} \) such that \( x, y \) are rational numbers. Each set in \( \mathscr{B} \) is an open subset of \( S \). Let \( U \) be an open set of \( S \) and \( p \in U \).

		      If \( p = A \) then there exists some \( I_{A}(-c, d) \subseteq U \). There exists positive rational numbers \( q_{c} < c, q_{d} < d \) so \( p \in I_{A}(-q_{c}, q_{d}) \subseteq I_{A}(-c, d) \subseteq U \).

		      If \( p \ne A \) then either \( p < 0 \) or \( p > 0 \). When \( p < 0 \), \( U \cap \openinterval{-\infty, 0} \) is a neighborhood of \( p \) and it is open in \( \mathbb{R}\smallsetminus\left\{0\right\} \) so there exist positive rational numbers \( q_{1}, q_{2} \) such that \( p \in \openinterval{-q_{1}, -q_{2}} \subseteq U \cap \openinterval{-\infty, 0} \subseteq U \). Similarly, when \( p > 0 \), there exist positive rational numbers \( q_{1}, q_{2} \) such that \( p \in \openinterval{q_{1}, q_{2}} \subseteq U \).

		      Thus \( \mathscr{B} \) is a basis for the topology on \( S \). Because \( \mathscr{B} \) is countable, it follows that \( S \) is second countable.

		      Let \( U_{A} \) be a neighborhood of \( A \) and \( U_{B} \) a neighborhood of \( B \). Because \( \left\{ I_{A}(-c, d) \mid c, d \right\} \) is a neighborhood basis at \( A \) and \( \left\{ I_{B}(-c, d) \mid c, d \right\} \) is a neighborhood basis at \( B \) so there exist \( c_{A}, d_{A}, c_{B}, d_{B} > 0 \) such that \( I_{A}(-c_{A}, d_{A}) \subseteq U_{A} \) and \( I_{B}(-c_{B}, d_{B}) \subseteq U_{B} \). Since \( I_{A}(-c_{A}, d_{A}) \) and \( I_{B}(-c_{B}, d_{B}) \) are not disjoint, it follows that \( U_{A}, U_{B} \) are not disjoint. Therefore \( S \) is not Hausdorff.
	\end{enumerate}
\end{proof}

\begin{problem}{5.2}[A sphere with a hair]
A fundamental theorem of topology, the theorem on invariance of dimension, states that if two nonempty open sets \( U \subseteq \mathbb{R}^{n} \) and \( V \subseteq \mathbb{R}^{m} \) are homeomorphic, then \( n = m \). Use the idea of Example 5.4 as well as the theorem on invariance of dimension to prove that the sphere with a hair in \( \mathbb{R}^{3} \) (see Figure 5.10 in the book) is not locally Euclidean at \( q \). Hence it cannot be a topological manifold.
\end{problem}

\begin{proof}
	Assume for the sake of contrary that the sphere with a hair in \( \mathbb{R}^{3} \) is locally Euclidean at \( q \) (the common point of the sphere and the hair) of dimension \( n \).

	Each point other than \( q \) on the sphere has a neighborhood that is homeomorphic to \( \mathbb{R}^{2} \). On the other hand, each point other than \( q \) on the hair other than \( q \) has a neighborhood that is homeomorphic to \( \mathbb{R}^{1} \). From the theorem on invariance of dimension, it follows that \( n = 1 \) and \( n = 2 \), which is a contradiction.
\end{proof}

\begin{problem}{5.3}[Charts on a sphere]
Let \( S^{2} \) be the unit sphere
\[
	x^{2} + y^{2} + z^{2} = 1
\]

in \( \mathbb{R}^{3} \). Define in \( S^{2} \) the six charts corresponding to the six hemispheres --- the front, rear, right, left, upper, and lower hemispheres:
\begin{align*}
	U_{1} = \left\{ (x, y, z) \in S^{2} \mid x > 0 \right\}, \qquad \phi_{1}(x, y, z) = (y, z), \\
	U_{2} = \left\{ (x, y, z) \in S^{2} \mid x < 0 \right\}, \qquad \phi_{2}(x, y, z) = (y, z), \\
	U_{3} = \left\{ (x, y, z) \in S^{2} \mid y > 0 \right\}, \qquad \phi_{3}(x, y, z) = (x, z), \\
	U_{4} = \left\{ (x, y, z) \in S^{2} \mid y < 0 \right\}, \qquad \phi_{4}(x, y, z) = (x, z), \\
	U_{5} = \left\{ (x, y, z) \in S^{2} \mid z > 0 \right\}, \qquad \phi_{5}(x, y, z) = (x, y), \\
	U_{6} = \left\{ (x, y, z) \in S^{2} \mid z < 0 \right\}, \qquad \phi_{6}(x, y, z) = (x, y).
\end{align*}

Describe the domain \( \phi_{4}(U_{14}) \) of \( \phi_{1} \circ \phi_{4}^{-1} \) and show that \( \phi_{1} \circ \phi_{4}^{-1} \) is \( C^{\infty} \) on \( \phi_{4}(U_{14}) \). Do the same for \( \phi_{6} \circ \phi_{1}^{-1} \).
\end{problem}

\begin{proof}
	\( U_{14} = \left\{ (x, y, z) \in S^{2} \mid x > 0, y < 0 \right\} \) so \( \phi_{4}(U_{14}) = \left\{ (x, z) \mid x > 0, x^{2} + z^{2} < 1 \right\} \). For every \( (x, z) \in \phi_{4}(U_{14}) \)
	\[
		(\phi_{1}\circ \phi_{4}^{-1})(x, z) = \phi_{1}(x, -\sqrt{1 - x^{2} - z^{2}}, z) = (-\sqrt{1 - x^{2} - z^{2}}, z)
	\]

	which shows that \( \phi_{1} \circ \phi_{4}^{-1} \) is \( C^{\infty} \) on \( \phi_{4}(U_{14}) \).

	\( U_{16} = \left\{ (x, y, z) \in S^{2} \mid x > 0, z < 0 \right\} \) so \( \phi_{6}(U_{16}) = \left\{ (x, y) \mid x > 0, x^{2} + y^{2} < 1 \right\} \). For each \( (x, y) \in \phi_{6}(U_{16}) \)
	\[
		(\phi_{1}\circ \phi_{6}^{-1})(x, y) = \phi_{1}(x, y, -\sqrt{1 - x^{2} - y^{2}}) = (y, -\sqrt{1 - x^{2} - y^{2}})
	\]

	which shows that \( \phi_{1}\circ \phi_{6}^{-1} \) is \( C^{\infty} \) on \( \phi_{6}(U_{16}) \).
\end{proof}

\begin{problem}{5.4}[Existence of a coordinate neighborhood]
Let \( \left\{ (U_{\alpha}, \phi_{\alpha}) \right\} \) be the maximal atlas on a manifold \( M \). For any open set \( U \) in \( M \) and a point \( p \in U \), prove the existence of a coordinate open set \( U_{\alpha} \) such that \( p \in U_{\alpha} \subset U \).
\end{problem}

\begin{proof}
	Since \( \bigcup_{\alpha} U_{\alpha} = M \), there exists \( \alpha \) such that \( p \in U_{\alpha} \). Therefore \( (U \cap U_{\alpha}, \phi_{\alpha}\vert_{U \cap U_{\alpha}}) \) is a chart about \( p \). Consider an arbitrary chart \( (U_{\beta}, \phi_{\beta}) \) in the given maximal atlas on \( M \). By the definition of a smooth manifold, \( \phi_{\beta} \circ \phi_{\alpha}^{-1} \) is \( C^{\infty} \) on \( \phi_{\alpha}(U_{\alpha} \cap U_{\beta}) \) and \( \phi_{\alpha} \circ \phi_{\beta}^{-1} \) is \( C^{\infty} \) on \( \phi_{\beta}(U_{\alpha}\cap U_{\beta}) \). Therefore, as restrictions of \( C^{\infty} \) functions, \( \phi_{\beta} \circ {(\phi_{\alpha}\vert_{U\cap U_{\alpha}})}^{-1} \) is \( C^{\infty} \) on \( \phi_{a}(U \cap U_{\alpha} \cap U_{\beta}) \) and \( {(\phi_{\alpha}\vert_{U\cap U_{\alpha} \cap U_{\beta}})} \circ \phi_{\beta}^{-1} \) is \( C^{\infty} \) on \( \phi_{\beta}(U \cap U_{\alpha} \cap U_{\beta}) \). Due to the arbitrariness of \( \beta \), we conclude that the chart \( (U \cap U_{\alpha}, \phi_{\alpha}\vert_{U\cap U_{\alpha}}) \) is compatible with the given atlas. Because the given atlas is maximal, then the chart \( (U \cap U_{\alpha}, \phi_{\alpha}\vert_{U \cap U_{\alpha}}) \) is contained in the atlas as it is compatible with the atlas.

	Thus there exists a coordinate neighborhood \( U_{i} \) such that \( p \in U_{i} \subseteq U \).
\end{proof}

\begin{problem}{5.5}[An atlas for a product manifold]\label{problem:5.5}
Prove Proposition 5.18.

If \( \left\{ (U_{\alpha}, \phi_{\alpha}) \right\} \) and \( \left\{ (V_{i}, \psi_{i}) \right\} \) are \( C^{\infty} \) atlases for the manifolds \( M \) and \( N \) of dimensions \( m \) and \( n \), respectively, then the collection
\[
	\left\{ (U_{\alpha} \times V_{i}, \phi_{\alpha} \times \psi_{i}: U_{\alpha} \times V_{i} \to \mathbb{R}^{m} \times \mathbb{R}^{n}) \right\}
\]

of charts is a \( C^{\infty} \) atlas on \( M\times N \). Therefore, \( M\times N \) is a \( C^{\infty} \) manifold of dimension \( m + n \).
\end{problem}

\begin{proof}
	\( \phi_{\alpha} \) is a homeomorphism from \( U_{\alpha} \) onto an open subset of \( \mathbb{R}^{m} \) and \( \psi_{i} \) is a homeomorphism from \( V_{i} \) onto an open subset of \( \mathbb{R}^{n} \) so \( \phi_{\alpha} \times \psi_{i} \) is a homeomorphism onto \( \phi_{\alpha}(U_{\alpha}) \times \psi_{i}(V_{i}) \). Moreover, \( \phi_{\alpha}(U_{\alpha}) \times \psi_{i}(V_{i}) \) is a product open set of \( \mathbb{R}^{m} \times \mathbb{R}^{n} \) hence open in \( \mathbb{R}^{m} \times \mathbb{R}^{n} \simeq \mathbb{R}^{m + n} \). Therefore \( (U_{\alpha} \times V_{i}, \phi_{\alpha} \times \psi_{i}: U_{\alpha} \times V_{i} \to \mathbb{R}^{m} \times \mathbb{R}^{n}) \) is a chart.

	Consider two charts \( (U_{\alpha} \times V_{i}, \phi_{\alpha} \times \psi_{i}: U_{\alpha} \times V_{i} \to \mathbb{R}^{m} \times \mathbb{R}^{n})  \) and \( (U_{\beta} \times V_{j}, \phi_{\beta} \times \psi_{j}: U_{\beta} \times V_{j} \to \mathbb{R}^{m} \times \mathbb{R}^{n}) \). Let \( (p_{1}, q_{1}) \in (\phi_{\alpha} \times \psi_{i})(U_{\alpha\beta} \times V_{ij}) \subseteq \mathbb{R}^{m} \times \mathbb{R}^{n} \) and \( (p_{2}, q_{2}) \in (\phi_{\beta} \times \psi_{j})(U_{\alpha\beta} \times V_{ij}) \) then
	\[
		\begin{split}
			{(\phi_{\beta} \times \psi_{j})} \circ {(\phi_{\alpha} \times \psi_{i})}^{-1}(p_{1}, q_{1}) = ((\phi_{\beta} \circ \phi_{\alpha}^{-1})(p_{1}), (\psi_{j} \circ \psi_{i}^{-1})(q_{1})), \\
			{(\phi_{\alpha} \times \psi_{i})} \circ {(\phi_{\beta} \times \psi_{j})}^{-1}(p_{2}, q_{2}) = ((\phi_{\alpha} \circ \phi_{\beta}^{-1})(p_{2}), (\psi_{i} \circ \psi_{j}^{-1})(q_{2})).
		\end{split}
	\]

	Since \( \phi_{\beta} \circ \phi_{\alpha}^{-1} \) and \( \phi_{\alpha} \circ \phi_{\beta}^{-1} \) are \( C^{\infty} \) on \( U_{\alpha\beta} \), \( \psi_{j} \circ \psi_{i}^{-1} \) and \( \psi_{i} \circ \psi_{j}^{-1} \) are \( C^{\infty} \) on \( V_{ij} \), then \( {(\phi_{\beta} \times \psi_{j})} \circ {(\phi_{\alpha} \times \psi_{i})}^{-1} \) and \( {(\phi_{\alpha} \times \psi_{i})} \circ {(\phi_{\beta} \times \psi_{j})}^{-1} \) are \( C^{\infty} \) on \( U_{\alpha\beta} \times V_{ij} \). Therefore the charts in the given collection are pairwise \( C^{\infty} \)-compatible hence an atlas on \( M\times N \). Thus, \( M\times N \) is a \( C^{\infty} \) manifold of dimension \( m + n \).
\end{proof}

\section{Smooth Maps on a Manifold}

\begin{exercise}{6.14}[Smoothness of a map to a circle]
	Prove that the map \( F: \mathbb{R} \to S^{1}, F(t) = (\cos t, \sin t) \) is \( C^{\infty} \).
\end{exercise}

\begin{proof}
	The component functions of \( F \) are \( \cos \) and \( \sin \), which are \( C^{\infty} \). According to Proposition 6.13, \( F \) is \( C^{\infty} \).
\end{proof}

\begin{exercise}{6.18}[Smoothness of a map to a Cartesian product]\label{exercise:6.18}
	Let \( M_{1}, M_{2} \), and \( N \) be manifolds of dimensions \( m_{1}, m_{2}, \) and \( n \) respectively. Prove that a map \( (f_{1}, f_{2}): N \to M_{1} \times M_{2} \) is \( C^{\infty} \) if and only if \( f_{i}: N \to M_{i}, i = 1, 2, \) are both \( C^{\infty} \).
\end{exercise}

\begin{proof}
	Firstly, we show that \( \pi_{1}: M_{1} \times M_{2} \to M_{1} \) and \( \pi_{2}: M_{1} \times M_{2} \to M_{2} \) given by \( \pi_{1}(p_{1}, p_{2}) = p_{1} \) and \( \pi_{2}(p_{1}, p_{2}) = p_{2} \) are \( C^{\infty} \).

	Let \( \left\{ (U_{\alpha}, \phi_{\alpha}: U_{\alpha} \to \mathbb{R}^{m_{1}}) \right\} \) and \( \left\{ (V_{\beta}, \psi_{\beta}: V_{\beta} \to \mathbb{R}^{m_{2}}) \right\} \) be atlases for \( M_{1} \) and \( M_{2} \), respectively. The collection \( \left\{ U_{\alpha} \times V_{\beta}, \phi_{\alpha} \times \psi_{\beta}: U_{\alpha} \times V_{\beta} \to \mathbb{R}^{m_{1} + m_{2}} \right\} \) is an atlas for \( M_{1} \times M_{2} \).

	Each composition
	\[
		(\phi_{\alpha} \circ \pi_{1}\circ {(\phi_{\alpha} \times \psi_{\beta})}^{-1})(a^{1}, \ldots, a^{m_{1}}, b^{1}, \ldots, b^{m_{2}}) = (a^{1}, \ldots, a^{m_{1}})
	\]

	is a \( C^{\infty} \) function from \( (\phi_{\alpha} \times \psi_{\beta})(U_{\alpha} \times V_{\beta}) \) to \( \phi_{\alpha}(U_{\alpha}) \). Thus \( \pi_{1} \) is \( C^{\infty} \). Analogously, \( \pi_{2} \) is \( C^{\infty} \).

	If \( (f_{1}, f_{2}) \) is \( C^{\infty} \) then \( f_{1} = \pi_{1} \circ (f_{1}, f_{2}) \) and \( f_{2} = \pi_{2} \circ (f_{1}, f_{2}) \) are \( C^{\infty} \).

	Otherwise, suppose that \( f_{1} \) and \( f_{2} \) are \( C^{\infty} \). Let \( (W, \tau) \) be an arbitrary chart on \( N \). Because \( f_{1}, f_{2} \) are  \( C^{\infty} \) maps, it follows that \( \phi_{\alpha} \circ f_{1}\circ \tau^{-1} \) is \( C^{\infty} \) on \( \tau(W \cap f_{1}^{-1}(U_{\alpha})) \) and \( \psi_{\beta} \circ f_{2} \circ \tau^{-1} \) is \( C^{\infty} \) on \( \tau(W \cap f_{2}^{-1}(V_{\beta})) \).

	Each composition
	\[
		((\phi_{\alpha} \times \psi_{\beta}) \circ (f_{1}, f_{2}) \circ \tau^{-1})(a^{1}, \ldots, a^{n}) = ((\phi_{\alpha} \circ f_{1} \circ \tau^{-1})(a^{1}, \ldots, a^{n}), (\psi_{\beta} \circ f_{2} \circ \tau^{-1})(a^{1}, \ldots, a^{n}))
	\]

	is then a \( C^{\infty} \) function from \( \tau(W \cap {(f_{1}, f_{2})}^{-1}(U_{\alpha} \times V_{\beta})) \subseteq \mathbb{R}^{n} \) to \( \mathbb{R}^{m_{1} + m_{2}} \). Hence the map \( (f_{1}, f_{2}) \) is \( C^{\infty} \).
\end{proof}

\begin{problem}{6.1}[Differentiable Structures on \(\mathbb{R}\)]
Let \(\mathbb{R}\) be the real line with the differentiable structure given by the maximal atlas of the chart \((\mathbb{R}, \phi = \operatorname{id} \colon \mathbb{R} \to \mathbb{R})\), and let \(\mathbb{R}'\) be the real line with the differentiable structure given by the maximal atlas of the chart \((\mathbb{R}, \psi \colon \mathbb{R} \to \mathbb{R})\), where \(\psi(x) = x^{1/3}\).

\begin{enumerate}[label={(\alph*)},leftmargin=*]
	\item Show that these two differentiable structures are distinct.
	\item Show that there is a diffeomorphism between \(\mathbb{R}\) and \(\mathbb{R}'\).\@ (\textit{Hint}: The identity map \(\mathbb{R} \to \mathbb{R}\) is not the desired diffeomorphism; in fact, this map is not smooth.)
\end{enumerate}
\end{problem}

\begin{proof}
	\begin{enumerate}[label={(\alph*)},leftmargin=*]
		\item The composition \( \psi \circ \phi^{-1}: \mathbb{R} \to \mathbb{R}, (\psi \circ \phi^{-1})(x) = \psi(x) = x^{1/3} \) is not \( C^{\infty} \) at \( x = 0 \). Therefore the charts \( (\mathbb{R}, \phi) \) and \( (\mathbb{R}, \psi) \) are not \( C^{\infty} \)-compatible, hence the two differentiable structures are distinct.
		\item Consider the map \( f: \mathbb{R} \to \mathbb{R}' \) given by \( f(x) = x^{3} \). The map \( f \) is \( C^{\infty} \) because the composition \( \phi \circ f \circ \psi^{-1} = \phi \) is \( C^{\infty} \). It is also bijective as it admits \( x \mapsto x^{1/3} \) as an inverse. This map is also \( C^{\infty} \) because \( \psi \circ f^{-1} \circ \phi = \phi \) is \( C^{\infty} \).
	\end{enumerate}
\end{proof}

\begin{problem}{6.2}[The Smoothness of an Inclusion Map]
Let \(M\) and \(N\) be manifolds and let \(q_{0}\) be a point in \(N\). Prove that the inclusion map \(i_{q_{0}}: M \to M \times N\), \(i_{q_{0}}(p) = (p, q_{0})\), is \(C^\infty\).
\end{problem}

\begin{proof}
	Denote the dimensions of \( M, N \) be \( m, n \), respectively.

	The identity map \( \operatorname{id}: M \to M \) is \( C^{\infty} \). Consider the constant map \( f: M \to N \) given by \( f(p) = q_{0} \). For every chart \( (U, \phi) \) on \( M \) and \( (V, \psi) \) on \( N \) such that \( U \cap f^{-1}(V) \ne \varnothing \), the function
	\[
		\psi \circ f \circ \phi^{-1}: \phi(U \cap f^{-1}(V)) \to \mathbb{R}^{n}
	\]

	is \( C^{\infty} \) because it is a constant function, as
	\[
		(\psi \circ f \circ \phi^{-1})(\phi(x)) = (\psi \circ f)(x) = \psi(q_{0})
	\]

	for every \( x \in U \cap f^{-1}(V) \). According to Exercise~\ref{exercise:6.18}, \( i_{q_{0}} \) is \( C^{\infty} \).
\end{proof}

\begin{problem}{6.3}[Group of Automorphisms of a Vector Space]
Let \(V\) be a finite-dimensional vector space over \(\mathbb{R}\), and \(\mathrm{GL}(V)\) the group of all linear automorphisms of \(V\). Relative to an ordered basis \(e = (e_{1}, \ldots, e_{n})\) for \(V\), a linear automorphism \(L \in \mathrm{GL}(V)\) is represented by a matrix \([a_{j}^{i}]\) defined by
\[
	L(e_{j}) = \sum_{i} a_{j}^{i} e_{i}.
\]

The map
\[
	\phi_{e} : \mathrm{GL}(V) \to \mathrm{GL}(n, \mathbb{R}), \quad L \mapsto [a_{j}^{i}],
\]

is a bijection with an open subset of \(\mathbb{R}^{n \times n}\) that makes \(\mathrm{GL}(V)\) into a \(C^{\infty}\) manifold, which we denote temporarily by \({\mathrm{GL}(V)}_{e}\). If \({\mathrm{GL}(V)}_{u}\) is the manifold structure induced from another ordered basis \(u = (u_{1}, \ldots, u_{n})\) for \(V\), show that \({\mathrm{GL}(V)}_{e}\) is the same as \({\mathrm{GL}(V)}_{u}\).
\end{problem}

\begin{proof}
	Let \( C = \mathcal{M}(I_{n}, e, u) \) be the change-of-basis matrix in which \( I_{n} \) is the \( n\times n \) identity matrix.

	Consider the map \( \phi_{e} \circ \phi_{u}^{-1}: \mathrm{GL}(n, \mathbb{R}) \to \mathrm{GL}(n, \mathbb{R}) \). For each \( A \in \mathrm{GL}(n, \mathbb{R}) \)
	\[
		(\phi_{e} \circ \phi_{u}^{-1})(A) = C^{-1}AC
	\]

	so \( \phi_{e} \circ \phi_{u}^{-1} \) is a \( C^{\infty} \) function. Similarly, the function \( \phi_{u} \circ \phi_{e}^{-1} \) is also \( C^{\infty} \). Therefore the two charts \( ({\mathrm{GL}(V)}_{e}, \phi_{e}) \) and \( ({\mathrm{GL}(V)}_{u}, \phi_{u}) \) are \( C^{\infty} \)-compatible, which implies that the two charts belong to the same differentiable structure (maximal atlas). Hence \( {\mathrm{GL}(V)}_{e} \) and \( {\mathrm{GL}(V)}_{u} \) are the same smooth manifold.
\end{proof}

\begin{problem}{6.4}[Local Coordinate Systems]
Find all points in \(\mathbb{R}^{3}\) in a neighborhood of which the functions \(x\), \(x^{2} + y^{2} + z^{2} - 1\), \(z\) can serve as a local coordinate system.
\end{problem}

\begin{proof}
	Let \( p = (p^{1}, p^{2}, p^{3}) \) be a point in \( \mathbb{R}^{3} \), the Jacobian determinant
	\[
		\det\begin{bmatrix}
			\dfrac{\partial x}{\partial x}(p)                           & \dfrac{\partial x}{\partial y}(p)                           & \dfrac{\partial x}{\partial z}(p)                           \\
			\dfrac{\partial (x^{2} + y^{2} + z^{2} - 1)}{\partial x}(p) & \dfrac{\partial (x^{2} + y^{2} + z^{2} - 1)}{\partial y}(p) & \dfrac{\partial (x^{2} + y^{2} + z^{2} - 1)}{\partial z}(p) \\
			\dfrac{\partial z}{\partial x}(p)                           & \dfrac{\partial z}{\partial y}(p)                           & \dfrac{\partial z}{\partial z}(p)
		\end{bmatrix} = 2p^{2}
	\]

	is nonzero if and only if \( p^{2} \ne 0 \). Hence \( \mathbb{R}^{3} \smallsetminus \left\{ (x, y, z) \in \mathbb{R}^{3} \mid y \ne 0 \right\} \) consists of points in a neighborhood such that \( x, x^{2} + y^{2} + z^{2} - 1, z \) can serve as a local coordinate system.
\end{proof}

\section{Quotients}

\begin{exercise}{7.11}[Real projective space as a quotient of a sphere]\label{exercise:7.11}
	For \( x = (x^{1}, \ldots, x^{n}) \in \mathbb{R}^{n} \), let \( \left\Vert x \right\Vert = \sqrt{\sum_{i}{(x^{i})}^{2}} \) be the modulus of \( x \). Prove that the map \( f: \mathbb{R}^{n+1}\smallsetminus \left\{0\right\} \to S^{n} \) given by
	\[
		f(x) = \dfrac{x}{\left\Vert x \right\Vert}
	\]

	induces a homeomorphism  \( \bar{f}: \mathbb{R}P^{n} \to S^{n}/\!\sim \).
\end{exercise}

\begin{proof}
	Let \( \pi_{1}: \mathbb{R}^{n+1}\smallsetminus \left\{0\right\} \to \mathbb{R}P^{n} \) and \( \pi_{2}: S^{n} \to S^{n}/\!\sim \) be the quotient maps in this exercise.

	We define \( \bar{f}: \mathbb{R}P^{n} \to S^{n}/\!\sim \) by
	\[
		\bar{f}([x]) = [f(x)].
	\]

	This map is well-defined, since if \( x \sim y \) then \( f(x) = \dfrac{x}{\left\Vert x\right\Vert} \sim \dfrac{y}{\left\Vert y\right\Vert} = f(y) \). Let me remind the second \( \sim \): \( x \sim y \Longleftrightarrow x = \pm y \). From this definition we obtain the following commutative diagram
	\[
		\begin{tikzcd}
			\mathbb{R}^{n+1}\smallsetminus \left\{0\right\} \arrow{r}{f} \arrow{d}{\pi_{1}} & S^{n} \arrow{d}{\pi_{2}} \\
			\mathbb{R}P^{n} \arrow{r}{\bar{f}}                                & S^{n}/\!\sim
		\end{tikzcd}
	\]

	Since \( \pi_{2} \circ f = \bar{f} \circ \pi_{1} \) is continuous and \( \pi_{1} \) is continous, it follows that \( \bar{f} \) is continuous (see Proposition 7.1).

	Consider the maps \( g: S^{n} \to \mathbb{R}^{n+1}\smallsetminus\left\{0\right\} \) and \( \bar{g}: S^{n}/\!\sim \to \mathbb{R}P^{n} \) defined by \( g(x) = x \) and \( \bar{g}([x]) = [x] \) then \( \bar{g} \) is an inverse of \( \bar{f} \). Moreover, the following diagram commutes
	\[
		\begin{tikzcd}
			S^{n} \arrow{r}{g} \arrow{d}{\pi_{2}} & \mathbb{R}^{n+1}\smallsetminus\left\{0\right\} \arrow{d}{\pi_{1}} \\
			S^{n}/\!\sim \arrow{r}{\bar{g}}         & \mathbb{R}P^{n}
		\end{tikzcd}
	\]

	Because \( \pi_{1} \circ g \) is continuous and \( \pi_{1} \circ g = \bar{g} \circ \pi_{2} \) then \( \bar{g} \circ \pi_{2} \) is continuous. From Proposition 7.1, it follows that \( \bar{g} \) is continuous.

	Thus \( \bar{f} \) is a homeomorphism from \( \mathbb{R}P^{n} \) onto \( S^{n}/\!\sim \) as it is a continuous bijection that has a continuous inverse.
\end{proof}

\begin{problem}{7.1}[Image of the inverse image of a map]
Let \( f: X \to Y \) be a map of sets, and let \( B \subseteq Y \). Prove that \( f(f^{-1}(B)) = B \cap f(X) \). Therefore, if \( f \) is surjective, then \( f(f^{-1}(B)) = B \).
\end{problem}

\begin{proof}
	Assume that \( y \in f(f^{-1}(B)) \) then there exists \( x \in f^{-1}(B) \) such that \( f(x) = y \). Hence \( f(x) \in f(X) \) and \( f(x) \in B \), which means \( y = f(x) \in B \cap f(X) \), so \( f(f^{-1}(B)) \subseteq B \cap f(X) \).

	If \( y \in B \cap f(X) \) then \( f^{-1}(\left\{ y \right\}) \subseteq f^{-1}(B)\). Let \( x \) be an element of \( f^{-1}(\left\{ y \right\}) \) then \( f(x) \in f(f^{-1}(B)) \). Hence \( B \cap f(X) \subseteq f(f^{-1}(B)) \).

	Thus \( f(f^{-1}(B)) = B \cap f(X) \).
\end{proof}

\begin{problem}{7.2}[Real projective plane]
Let \( H^{2} \) be the closed upper hemisphere in the unit sphere \( S^{2} \), and let \( i: H^{2} \to S^{2} \) be the inclusion map. In the notation of Example 7.13, prove that the induced map \( f: H^{2}/\!\sim \to S^{2}/\!\sim \) is a homeomorphism.
\end{problem}

\begin{proof}
	Let \( \pi_{1}: H^{2} \to H^{2}/\!\sim \) and \( \pi_{2}: S^{2} \to S^{2}/\!\sim \) be the induced quotient maps.

	For every \( U \subseteq S^{2}/\!\sim \)
	\[
		{(\pi_{2} \circ i)}^{-1}(U) = i^{-1}(\pi_{2}^{-1}(U)) = \pi_{2}^{-1}(U) \cap H^{2}.
	\]

	So \( U \) is open in \( S^{2}/\!\sim \) if and only if \(  {(\pi_{2} \circ i)}^{-1}(U) \) is open in \( H^{2} \). Therefore \( \pi_{2} \circ i \) is a quotient map. On the other hand, \( \pi_{1} \) is constant on each equivalence class of \( \pi_{2}\circ i \) so it induces a map \( f \) that commutes the following diagram.
	\[
		\begin{tikzcd}
			H^{2} \arrow{r}{i} \arrow{d}{\pi_{1}} & S^{2} \arrow{d}{\pi_{2}} \\
			H^{2}/\!\sim \arrow{r}{f} & S^{2}/\!\sim
		\end{tikzcd}
	\]

	From Proposition 7.1, it follows that \( f \) is continuous.

	We define two maps \( j: S^{2} \to H^{2} \) and \( g: S^{2}/\!\sim \to H^{2}/\!\sim \) as follows
	\begin{itemize}
		\item \( j \) identifies each point on the open upper hemisphere to its antipodal on the open lower hemisphere and \( j \) leaves points on the equator unchanged.
		\item \( g \) maps \( [x] \in S^{2}/\!\sim \) to \( [x] \in H^{2}/\!\sim \).
	\end{itemize}

	The following diagram commutes
	\[
		\begin{tikzcd}
			S^{2} \arrow{r}{j} \arrow{d}{\pi_{2}} & H^{2} \arrow{d}{\pi_{1}} \\
			S^{2}/\!\sim \arrow{r}{g}         & H^{2}/\!\sim
		\end{tikzcd}
	\]

	\( j \) and \( \pi_{1} \) are quotient maps. The composition of two quotient maps is a quotient map so \( g \circ \pi_{2} = \pi_{1} \circ j \) is a quotient map. From Proposition 7.1, it follows that \( g \) is continuous.

	On the other hand, \( g \) is an inverse of \( f \), so \( f \) is a homeomorphism.
\end{proof}

\begin{problem}{7.3}[Closedness of the diagonal of a Hausdorff space]
Deduce Theorem 7.7 from Corollary 7.8.

Theorem 7.7. Suppose \( \sim \) is an open equivalence relation on a topological space \( S \). Then the quotient space \( S/\!\sim \) is Hausdorff if and only if the graph \( R \) of \( \sim \) is closed in \( S\times S \).

Corollary 7.8. A topological space \( S \) is Hausdorff if and only if the diagonal \( \Delta \) in \( S\times S \) is closed.
\end{problem}

\begin{proof}
	Let \( \pi: S \to S/\!\sim \) be the induced quotient map.

	First, assume that \( S/\!\sim \) is Hausdorff. Let \( (x, y) \in (S\times S)\smallsetminus R \) then \( \pi(x) \ne \pi(y) \). Due to Hausdorffness of \( S/\!\sim \), there exist neighborhoods \( U \ni \pi(x) \) and \( V \ni \pi(x) \) that are disjoint. According to the continuity of \( \pi \), the preimages \( \pi^{-1}(U) \) and \( \pi^{-1}(V) \) are open and disjoint. Moreover, \( \pi^{-1}(U) \) and \( \pi^{-1}(V) \) are neighborhoods of \( x \) and \( y \), respectively. None of the elements of \( \pi^{-1}(U) \) is equivalent to any element of \( \pi^{-1}(V) \) and vice versa. Hence \( \pi^{-1}(U) \times \pi^{-1}(V) \) is a neighborhood of \( (x, y) \) and is contained in \( (S \times S)\smallsetminus R \), which means \( (S\times S)\smallsetminus R \) is open. Hence \( R \) is closed in \( S\times S \).

	Now assume that \( R \) is closed in \( S\times S \) then \( (S\times S)\smallsetminus R \) is open in \( S\times S \). Let \( [x] \) and \( [y] \) be two distinct elements of \( S/\!\sim \) then \( (x, y) \in (S\times S)\smallsetminus R \). So \( (x, y) \) is contained in a product open set \( U\times V \) contained in \( (S\times S)\smallsetminus R \). From the definition of \( R \), it follows that \( \pi(U) \) and \( \pi(V) \) are disjoint. Because \( \pi \) is an open map, \( \pi(U) \) and \( \pi(V) \) are open. Hence \( \pi(U) \) and \( \pi(V) \) are disjoint neighborhoods of \( [x] \) and \( [y] \). Thus \( S/\!\sim \) is Hausdorff.
\end{proof}

\begin{problem}{7.4}[Quotient of a sphere with antipodal points identified]
Let \( S^{n} \) be the unit sphere centered at the origin in \( \mathbb{R}^{n+1} \). Define an equivalence relation \( \sim \) on \( S^{n} \) by identifying antipodal points:
\[
	x \sim y \Longleftrightarrow x = \pm y,\quad x, y\in S^{n}.
\]

\begin{enumerate}[label={(\alph*)},leftmargin=*]
	\item Show that \( \sim \) is an open equivalence relation.
	\item Apply Theorem 7.7 and Corollary 7.8 to prove that the quotient space \( S^{n}/\!\sim \) is Hausdorff, without making use of the homeomorphism \( \mathbb{R}P^{n} \simeq S^{n}/\!\sim \).
\end{enumerate}
\end{problem}

\begin{proof}
	\begin{enumerate}[label={(\alph*)},leftmargin=*]
		\item Let \( \pi: S^{n} \to S^{n}/\!\sim \) be the induced quotient map and \( U \) is an open subset of \( S^{n} \). The map \( a: S^{n} \to S^{n} \) given by \( a(x) = -x \) is a homeomorphism so \( a(U) \) is an open subset of \( S^{n} \). Therefore
		      \[
			      \pi^{-1}(\pi(U)) = \bigcup_{x\in U} \left\{ x, -x \right\} = \left(\bigcup_{x\in U} \left\{ x \right\}\right) \cup \left( \bigcup_{x\in U} \left\{ -x \right\} \right) = U \cup a(U)
		      \]

		      is an open subset of \( S^{n} \), so \( \pi(U) \) is open. Hence \( \pi \) is open, which means \( \sim \) is an open equivalence relation.
		\item The graph \( R \) of \( \sim \) in \( S^{n} \times S^{n} \) is
		      \[
			      \left\{ (x, y) \in S^{n}\times S^{n} \mid x = \pm y \right\} = \left\{ (x, y) \in S^{n}\times S^{n} \mid x = y \right\} \cup \left\{ (x, y) \in S^{n}\times S^{n} \mid x = -y \right\}.
		      \]

		      Since \( S^{n} \) is Hausdorff (as a subspace of the Hausdorff space \( \mathbb{R}^{n+1} \)) then \( \left\{ (x, y) \in S^{n}\times S^{n} \mid x = y \right\} \) is closed in \( S^{n} \times S^{n} \). The map \( f: S^{n} \times S^{n} \to S^{n} \times S^{n} \) given by \( f(x, y) = (x, -y) \) is a homeomorphism so \( \left\{ (x, y) \in S^{n}\times S^{n} \mid x = -y \right\} \) is closed in \( S^{n} \times S^{n} \). Therefore \( R \) is closed in \( S^{n} \times S^{n} \) as it is the union of two closed sets.

		      From Theorem 7.7, it follows that \( S^{n}/\!\sim \) is Hausdorff.
	\end{enumerate}
\end{proof}

\begin{problem}{7.5}[Orbit space of a continuous group action]\label{problem:7.5}
Suppose a right action of a topological group \(G\) on a topological space \(S\) is continuous; this simply means that the map \( S\times G \to S \) describing the action is continuous. Define two points \( x,y \) of \(S\) to be equivalent if they are in the same orbit; i.e., there is an element \( g\in G \) such that \( y = xg \). Let \( S/G \) be the quotient space; it is called the \textit{orbit space} of the action. Prove that the projection map \( \pi: S \to S/G \) is an open map.
\end{problem}

\begin{proof}
	The projection map \( \pi: S \to S/G \) is a quotient map.

	Denote the right action of \( G \) on \( S \) by \( r \). Let \( U \) be an open subset of \( S \). Denote by \( \pi_{S} \) the canonical projection \( S\times G \to S \).
	\begingroup
	\allowdisplaybreaks%
	\begin{align*}
		\pi^{-1}(\pi(U)) & = \bigcup_{x\in U} \pi^{-1}(\pi(x))                                     \\
		                 & = \bigcup_{x\in U} \left\{ y \in S \mid \exists g\in G, x = yg \right\} \\
		                 & = \bigcup_{x\in U} \pi_{S}(r^{-1}(x))                                   \\
		                 & = \pi_{S}\left(\bigcup_{x\in U} r^{-1}(x)\right)                        \\
		                 & = \pi_{S}(r^{-1}(U)).
	\end{align*}
	\endgroup

	The canonical projection \( \pi_{S} \) is an open map and \( r^{-1}(U) \) is an open subset of \( S\times G \) so \( \pi_{S}(r^{-1}(U)) \) is an open subset of \( S \). Therefore \( \pi^{-1}(\pi(U)) \) for every open subset \( U \) of \( S \). From the definition of quotient map, it follows that \( \pi(U) \) is an open subset of \( S/G \). Thus \( \pi \) is an open quotient map.
\end{proof}

\begin{problem}{7.6}[Quotient of \( \mathbb{R} \) by \( 2\pi\mathbb{Z} \)]\label{problem:7.6}
Let the additive group \( 2\pi\mathbb{Z} \) act on \( \mathbb{R} \) on the right by \( x\cdot 2\pi n = x + 2\pi n \), where \( n \) is an integer. Show that the orbit space \( \mathbb{R}/2\pi\mathbb{Z} \) is a smooth manifold.
\end{problem}

\begin{proof}
	Denote the equivalence relation and the quotient map induced by the right group action by \( \sim \) and \( f \), as in Problem~\ref{problem:7.5}.

	According to Problem~\ref{problem:7.5} and theorems in Section 7.5, \( \sim \) and \( f \) are open and the orbit space \( \mathbb{R}/2\pi\mathbb{Z} \) is Hausdorff and second countable. It remains to construct an atlas.

	For each integer \(n\) we define \( V_{n} = f(\openinterval{n\pi - \pi, n\pi + \pi}) \) and \( \psi_{n}: V_{n} \to \openinterval{n\pi - \pi, n\pi + \pi} \) given by the restriction of \( f \) to \( \openinterval{n\pi - \pi, n\pi + \pi} \). The restriction of \( f \) to \( \openinterval{n\pi - \pi, n\pi + \pi} \) is one-to-one. Moreover, \( f \) is open and continuous. Hence \( \psi_{n} \) is a homeomorphism. Therefore \( \mathbb{R}/2\pi\mathbb{Z} \) is locally Euclidean of dimension \(1\).

	\( f(\halfopenright{0, 2\pi}) = \mathbb{R}/2\pi\mathbb{Z} \) so \( V_{0}, V_{1} \) constitutes an open cover for \( \mathbb{R}/2\pi\mathbb{Z} \).

	For every \( t \in \psi_{0}(V_{0} \cap V_{1}) \), \( t \ne 0 \) and
	\[
		(\psi_{1} \circ \psi_{0}^{-1})(t) = \begin{cases}
			t        & \text{if } t\in \openinterval{0, \pi},  \\
			t + 2\pi & \text{if } t\in \openinterval{-\pi, 0}.
		\end{cases}
	\]

	For every \( s \in \psi_{1}(V_{0} \cap V_{1}) \)
	\[
		(\psi_{0} \circ \psi_{1}^{-1})(s) = \begin{cases}
			s        & \text{if } s\in \openinterval{0, \pi},    \\
			s - 2\pi & \text{if } s\in \openinterval{\pi, 2\pi}.
		\end{cases}
	\]

	The function \( \psi_{1}\circ \psi_{0}^{-1} \) is defined on disjoint open intervals and it is \( C^{\infty} \) on each open interval, so is \( \psi_{0} \circ \psi_{1}^{-1} \). Hence the transition functions are \( C^{\infty} \), which means the charts \( (V_{0}, \psi_{0}) \) and \( (V_{1}, \psi_{1}) \) are \( C^{\infty} \)-compatible, which means the two charts constitute an atlas for \( \mathbb{R}/2\pi\mathbb{Z} \).

	Thus \( \mathbb{R}/2\pi\mathbb{Z} \) is a smooth manifold of dimension \(1\).
\end{proof}

\begin{problem}{7.7}[The circle as a quotient space]
\begin{enumerate}[label={(\alph*)},leftmargin=*]
	\item Let \( {\left\{ (U_{\alpha}, \phi_{\alpha}) \right\}}^{2}_{\alpha=1} \) be the atlas of the circle \( S^{1} \) in Example 5.7, and let \( \bar{\phi}_{\alpha} \) be the map \( \phi_{\alpha} \) followed by the projection \( \mathbb{R} \to \mathbb{R}/2\pi\mathbb{Z} \). On \( U_{1} \cap U_{2} = A \amalg B \), since \( \phi_{1} \) and \( \phi_{2} \) differ by an integer multiple of \( 2\pi \), \( \bar{\phi}_{1} = \bar{\phi}_{2} \). Therefore \( \bar{\phi}_{1} \) and \( \bar{\phi}_{2} \) piece together to give a well-defined map \( \bar{\phi}: S^{1} \to \mathbb{R}/2\pi\mathbb{Z} \). Prove that \( \bar{\phi} \) is \( C^{\infty} \).
	\item The complex exponential \( \mathbb{R} \to S^{1} \), \( t \mapsto e^{it} \), is constant on each orbit of the action of \( 2\pi\mathbb{Z} \) on \( \mathbb{R} \). Therefore, there is an induced map \( F: \mathbb{R}/2\pi\mathbb{Z} \to S^{1} \), \( F([t]) = e^{it} \). Prove that \( F \) is \( C^{\infty} \).
	\item Prove that \( F: \mathbb{R}/2\pi\mathbb{Z} \to S^{1} \) is a diffeomorphism.
\end{enumerate}
\end{problem}

\begin{proof}
	Firstly, I rewrite here the definitions of \( \phi_{1}, \phi_{2}, A, B \).
	\[
		\begin{split}
			U_{1} = \left\{ e^{it} \in \mathbb{C} \mid -\pi < t < \pi \right\}, \\
			U_{2} = \left\{ e^{it} \in \mathbb{C} \mid 0 < t < 2\pi \right\},
		\end{split}
	\]

	and
	\[
		\begin{split}
			\phi_{1}: U_{1} \to \mathbb{R}, \quad \phi_{1}(e^{it}) = t, \quad -\pi < t < \pi, \\
			\phi_{2}: U_{2} \to \mathbb{R}, \quad \phi_{2}(e^{it}) = t, \quad 0 < t < 2\pi.
		\end{split}
	\]

	\( A, B \) are the connected components of \( U_{1} \cap U_{2} \)
	\[
		\begin{split}
			A = \left\{ e^{it} \mid -\pi < t < 0 \right\}, \\
			B = \left\{ e^{it} \mid 0 < t < \pi \right\}.
		\end{split}
	\]

	Denote the projection \( \mathbb{R} \to \mathbb{R}/2\pi\mathbb{Z} \) by \( f \).

	\begin{enumerate}[label={(\alph*)},leftmargin=*]
		\item \( \bar{\phi} \) is continuous due to the local criterion for continuity. We reuse the atlas in Problem~\ref{problem:7.6}.

		      For every \( t \in \phi_{1}(U_{1} \cap \bar{\phi}^{-1}(V_{0})) \), \( (\psi_{0} \circ \bar{\phi} \circ \phi_{1}^{-1})(t) = t \) so \( \psi_{0} \circ \bar{\phi} \circ \phi_{1}^{-1} \) is a \( C^{\infty} \) function.

		      For every \( t \in \phi_{1}(U_{1} \cap \bar{\phi}^{-1}(V_{1})) \), \( (\psi_{1} \circ \bar{\phi} \circ \phi_{1}^{-1})(t) = t + 2\pi \) so \( \psi_{1} \circ \bar{\phi} \circ \phi_{1}^{-1} \) is a \( C^{\infty} \) function.

		      For every \( t \in \phi_{2}(U_{2} \cap \bar{\phi}^{-1}(V_{0})) \), \( (\psi_{0} \circ \bar{\phi} \circ \phi_{2}^{-1})(t) = t - 2\pi \) so \( \psi_{0} \circ \bar{\phi} \circ \phi_{2}^{-1} \) is a \( C^{\infty} \) function.

		      For every \( t \in \phi_{2}(U_{2} \cap \bar{\phi}^{-1}(V_{1})) \), \( (\psi_{1} \circ \bar{\phi} \circ \phi_{2}^{-1})(t) = t \) so \( \psi_{1} \circ \bar{\phi} \circ \phi_{2}^{-1} \) is a \( C^{\infty} \) function.

		      Hence \( \bar{\phi}: S^{1} \to \mathbb{R}/2\pi\mathbb{Z} \) is a \( C^{\infty} \) map between smooth manifolds.
		\item \( F \) is continuous due to Proposition 7.1.

		      For every \( t \in \psi_{0}(V_{0} \cap F^{-1}(U_{1})) \), \( (\phi_{1} \circ F \circ \psi_{0}^{-1})(t) = t \) so \( \phi_{1} \circ F \circ \psi_{0}^{-1} \) is a \( C^{\infty} \) function.

		      For every \( t \in \psi_{0}(V_{0} \cap F^{-1}(U_{2})) \), \( (\phi_{2} \circ F \circ \psi_{0}^{-1})(t) = t + 2\pi \) so \( \phi_{2} \circ F \circ \psi_{0}^{-1} \) is a \( C^{\infty} \) function.

		      For every \( t \in \psi_{1}(V_{1} \cap F^{-1}(U_{1})) \), \( (\phi_{1} \circ F \circ \psi_{1}^{-1})(t) = t - 2\pi \) so \( \phi_{1} \circ F \circ \psi_{1}^{-1} \) is a \( C^{\infty} \) function.

		      For every \( t \in \psi_{1}(V_{1} \cap F^{-1}(U_{2})) \), \( (\phi_{2} \circ F \circ \psi_{1}^{-1})(t) = t \) so \( \phi_{2} \circ F \circ \psi_{1}^{-1} \) is a \( C^{\infty} \) function.

		      Hence \( F: \mathbb{R}/2\pi\mathbb{Z} \to S^{1} \) is a \( C^{\infty} \) map between smooth manifolds.
		\item Because \( (\bar{\phi} \circ F)([t]) = \bar{\phi}(e^{it}) = [t] \) and \( (F\circ \bar{\phi})(e^{it}) = F([t]) = e^{it} \), \( F \) is the inverse of \( \bar{\phi} \). Together with parts (a) and (b), we deduce that \( F \) is a diffeomorphism.
	\end{enumerate}
\end{proof}

\begin{problem}{7.8}[The Grassmannian \(G(k, n)\)]
The Grassmannian \( G(k, n) \) is the set of all \(k\)-planes through the origin in \( \mathbb{R}^{n} \). Such a \(k\)-plane is a linear subspace of dimension \(k\) of \( \mathbb{R}^{n} \) and has a basis consisting of \(k\) linearly independent vectors \( a_{1}, \ldots, a_{k} \) in \( \mathbb{R}^{n} \). It is therefore completely specified by an \( n\times k \) matrix \(A = [a_{1} \cdots a_{k}]\) of rank \(k\), where the \textit{rank} of a matrix \(A\), denoted by \(\operatorname{rk}A\), is defined to be the number of linearly independent columns of \(A\). This matrix is called a \textit{matrix representative} of the \(k\)-plane.

Two bases \( a_{1}, \ldots, a_{k} \) and \( b_{1}, \ldots, b_{k} \) determine the same \(k\)-plane if there is a change-of-basis matrix \(g = [g_{ij}] \in \mathrm{GL}(k, \mathbb{R})\) such that
\[
	b_{j} = \sum_{i} a_{i}g_{ij},\quad 1 \leq i, j\leq k.
\]

In matrix notation, \( B = Ag \).

Let \(F(k, n)\) be the set of all \(n\times k\) matrices of rank \(k\), topologized as a subspace of \( \mathbb{R}^{n\times k} \), and \(\sim\) the equivalence relation
\[
	A \sim B \text{  iff  } \text{there is a matrix } g \in \mathrm{GL}(k, \mathbb{R}) \text{ such that } B = Ag.
\]

In the notation of Problem B.3, \( F(k, n) \) is the set \( D_{\max} \) in \( \mathbb{R}^{n+k} \) and is therefore an open subset. There is a bijection between \( G(k, n) \) and the quotient space \( F(k,n)/\!\sim \). We give the Grassmannian \(G(k, n)\) the quotient topology on \( F(k,n)/\!\sim \).
\begin{enumerate}[label={(\alph*)},leftmargin=*]
	\item Show that \( \sim \) is an open equivalence relation.
	\item Prove that the Grassmannian \( G(k, n) \) is second countable.
	\item Let \( S = F(k,n) \). Prove that the graph \( R \) in \( S\times S \) of the equivalence relation \( \sim \) is closed.
	\item Prove that the Grassmannian \( G(k, n) \) is Hausdorff.
\end{enumerate}

Next we want to find a \( C^{\infty} \) atlas on the Grassmanian \( G(k, n) \). For simplicity, we specialize to \( G(2, 4) \). For any \( 4\times 2 \) matrix \(A\), let \( A_{i,j} \) be the \(2\times 2\) submatrix consisting of its \(i\)-th row and \(j\)-th row. Define
\[
	V_{ij} = \left\{ A \in F(2, 4) \mid A_{ij} \text{ is nonsingular} \right\}.
\]

Because the complement of \(V_{ij}\) in \(F(2,4)\) is defined by the vanishing of \( \det A_{ij} \), we conclude that \( V_{ij} \) is an open subset of \( F(2, 4) \).

\begin{enumerate}[label={(\alph*)}, resume]
	\item Prove that if \( A \in V_{ij} \), then \( Ag \in V_{ij} \) for any nonsingular matrix \( g \in \mathrm{GL}(2, \mathbb{R}) \).
\end{enumerate}

Define \( U_{ij} = V_{ij}/\!\sim \). Since \( \sim \) is an open equivalence relation, \( U_{ij} = V_{ij}/\!\sim \) is an open subset of \( G(2, 4) \).

For \( A \in V_{12} \),
\[
	A \sim AA_{12}^{-1} = \begin{bmatrix}
		1 & 0 \\
		0 & 1 \\
		* & * \\
		* & *
	\end{bmatrix} = \begin{bmatrix}
		I \\
		A_{34}A_{12}^{-1}
	\end{bmatrix}
\]

This shows that the matrix representatives of a \(2\)-plane in \( U_{12} \) have a canonical form \(B\) in which \( B_{12} \) is the identity matrix.
\begin{enumerate}[label={(\alph*)}, resume]
	\item Show that the map \( \tilde{\phi}_{12}: V_{12} \to \mathbb{R}^{2\times 2} \),
	      \[
		      \tilde{\phi}_{12}(A) = A_{34}A_{12}^{-1},
	      \]

	      induces a homeomorphism \( \phi_{12}: U_{12} \to \mathbb{R}^{2\times 2} \).
	\item Define similarly homeomorphisms \( \phi_{ij}: U_{ij} \to \mathbb{R}^{2\times 2} \). Compute \( \phi_{12} \circ \phi_{23}^{-1} \), and show that it is \( C^{\infty} \).
	\item Show that \( \left\{ U_{ij} \mid 1 \leq i, j\leq 4 \right\} \) is an open cover of \( G(2, 4) \) and that \( G(2, 4) \) is a smooth manifold.
\end{enumerate}
\end{problem}

\begin{proof}
	\begin{enumerate}[label={(\alph*)},leftmargin=*]
		\item \( F(k,n)/\!\sim \) is the orbit space of the right action of \( \mathrm{GL}(k, \mathbb{R}) \) on \( F(k, n) \). From Problem~\ref{problem:7.5}, the projection \( F(k,n) \to F(k,n)/\!\sim \) is an open map. Therefore \( \sim \) is an open equivalence relation.
		\item \( F(k, n) \) is second countable and \( \sim \) is an open equivalence relation so it follows from Corollary 7.10 that \( F(k, n)/\!\sim \) is second countable and so is \( G(k, n) \).
		\item Let \( (A, B) \in S \times S \). Denote \( A = [a_{1} \cdots a_{k}] \) and \( B = [b_{1} \cdots b_{k}] \). Two matrices \( A \) and \( B \) are equivalent if and only if every column of \( B \) is a linear combination of the columns of \( A \) (and vice versa). So \( A \sim B \) if and only if \( \operatorname{rk}[a_{1} \cdots a_{k}\, b_{1} \cdots b_{k}] \leq k \). Moreover, \( \operatorname{rk}[a_{1} \cdots a_{k}\, b_{1} \cdots b_{k}] \leq k \) if and only if all \( (k+1)\times (k+1) \) minors of \( [A\, B] \) are zero. Consider the maps that each takes some \( n\times (2k) \) matrix to a \( (k + 1)\times (k+1) \) minor. These maps are continuous so \( R \) is an intersection of finitely many closed subsets of \( S\times S \). Thus \( R \) is closed in \( S\times S \).
		\item From Theorem 7.7, it follows that \( F(k,n)/\!\sim \) is Hausdorff. Hence \( G(k, n) \) is Hausdorff as the topology on it is defined to be that of \( F(k, n)/\!\sim \).
		\item Let \( B = Ag \) in which \( A \in V_{ij} \) and \( g \in \mathrm{GL}(2, \mathbb{R}) \). Therefore \( A = Bg^{-1} \) so if \( \operatorname{rk}B < 2 \) then so does \(A\) as every column of \(A\) is a linear combination of the columns of \(B\). Hence \( B \) is nonsingular, which means \( B \in V_{ij} \).
		\item If \( A, B \in V_{12} \) and \( A \sim B \) then there exists \( g \in \mathrm{GL}(2, \mathbb{R}) \) such that \( B = Ag \).
		      \[
			      \tilde{\phi}_{12}(B) = B_{34}B_{12}^{-1} = (A_{34}g){(A_{12}g)}^{-1} = (A_{34}g)(g^{-1}A_{12}^{-1}) = A_{34}A_{12}^{-1} = \tilde{\phi}_{12}(A).
		      \]

		      Hence \( \tilde{\phi}_{12} \) is constant on each equivalence class of \( \sim \) so it induces a continuous map \( \phi_{12}([A]) = A_{34}A_{12}^{-1} \). Moreover, if \( \tilde{\phi}_{12}(A) = \tilde{\phi}_{12}(B) \) then \( A_{34}A_{12}^{-1} = B_{34}B_{12}^{-1} \) and
		      \[ A \sim \begin{bmatrix}I \\ A_{34}A_{12}^{-1} \end{bmatrix} \sim \begin{bmatrix}I \\ B_{34}B_{12}^{-1} \end{bmatrix} \sim B \]

		      which means \( A \sim B \). Hence \( \tilde{\phi}_{12}(A) = \tilde{\phi}_{12}(B) \) if and only if \( A \sim B \). Therefore the induced map is bijective.

		      Its inverse is
		      \[
			      \phi_{12}^{-1}(X) = \left[\begin{bmatrix}
					      1       & 0       \\
					      0       & 1       \\
					      X_{1,1} & X_{1,2} \\
					      X_{2,1} & X_{2,2}
				      \end{bmatrix}\right]
		      \]

		      which is continuous. Thus the induced map is a homeomorphism.
		\item For each \( X \in \phi_{23}(U_{23} \cap U_{12}) \)
		      \begingroup
		      \allowdisplaybreaks%
		      \begin{align*}
			      (\phi_{12} \circ \phi_{23}^{-1})\left( \begin{bmatrix} X_{1,1} & X_{1,2} \\ X_{2,1} & X_{2,2} \end{bmatrix} \right) & = \phi_{12}\left(\left[\begin{bmatrix}
					                                                                                                                                                   X_{1,1} & X_{1,2} \\
					                                                                                                                                                   1       & 0       \\
					                                                                                                                                                   0       & 1       \\
					                                                                                                                                                   X_{2,1} & X_{2,2}
				                                                                                                                                                   \end{bmatrix}\right]\right) = \begin{bmatrix}
				                                                                                                                                                                                 0       & 1       \\
				                                                                                                                                                                                 X_{2,1} & X_{2,2}
			                                                                                                                                                                                 \end{bmatrix} {\begin{bmatrix}
				                                                                                                                                                                                                X_{1,1} & X_{1,2} \\
				                                                                                                                                                                                                1       & 0
			                                                                                                                                                                                                \end{bmatrix}}^{-1}
		      \end{align*}
		      \endgroup

		      so \( \phi_{12} \circ \phi_{23}^{-1}: \phi_{23}(U_{23} \cap U_{12}) \to \phi_{12}(U_{23} \cap U_{12}) \) is a \( C^{\infty} \) function, according to Cramer's rule.
		\item Let
		      \[
			      A = \begin{bmatrix}
                      a_{1,1} & a_{1,2} \\
                      a_{2,1} & a_{2,2} \\
                      a_{3,1} & a_{3,2} \\
                      a_{4,1} & a_{4,2}
                  \end{bmatrix}
		      \]

		      be an element of \( G(2, 4) \). Because \( A \) has maximal rank, it has a \( 2\times 2 \) minor that doesn't vanish. Hence \( A \) belongs to at least one of \( U_{ij} \), which means \( \left\{ U_{ij} \mid 1\leq i, j\leq 4 \right\} \) covers \( G(2, 4) \). Moreover, we showed that each \( U_{ij} \) is open, so  \( \left\{ U_{ij} \mid 1\leq i, j\leq 4 \right\} \) is an open cover for \( G(2, 4) \).

              Similar to the previous part, one can show that each transition function \( \phi_{ij} \circ \phi_{k\ell}^{-1}: \phi_{k\ell}(U_{k\ell} \cap U_{ij}) \to \phi_{ij}(U_{k\ell}\cap U_{ij}) \) is \( C^{\infty} \), hence the charts \( \left\{ (U_{ij}, \phi_{ij}) \mid 1\leq i, j\leq 4 \right\} \) are pairwsie \( C^{\infty} \)-compatible.

              Hence the Grassmannian \( G(2, 4) \) is a smooth manifold of dimension \( 2\times 2 \).
	\end{enumerate}
\end{proof}

\begin{problem}{7.9}[Compactness of real projective space]
Show that the real projective space \( \mathbb{R}P^{n} \) is compact.
\end{problem}

\begin{proof}
	According to Exercise~\ref{exercise:7.11}, \( \mathbb{R}P^{n} \) and \( S^{n}/\!\sim \) are homeomorphic. On the other hand, \( S^{n} \) is compact (according to Heine-Borel theorem) and the quotient map \( \pi: S^{n} \to S^{n}/\!\sim \) is continuous so \( S^{n}/\!\sim \) is also compact. Therefore \( \mathbb{R}P^{n} \) is compact.
\end{proof}

% chktex-file 44
\documentclass[class=linear-algebra,crop=false]{standalone}

\newcommand{\sgn}[1]{\text{sgn}\left({#1}\right)}
\setcounter{lemma}{0}

\begin{document}

\chapter{Định thức và hệ phương trình tuyến tính}

\par Thực hiện các phép nhân sau đây, viết các phép thế thu được thành tích của những xích rời rạc và tính dấu của chúng.

% exercise 3.1
\begin{exercise}
    $
        \begin{pmatrix}
            1 & 2 & 3 & 4 & 5 \\
            2 & 4 & 5 & 1 & 3
        \end{pmatrix}
        \begin{pmatrix}
            1 & 2 & 3 & 4 & 5 \\
            4 & 3 & 5 & 1 & 2
        \end{pmatrix}
    $.
\end{exercise}

\begin{proof}[Lời giải]
    \[
        \begin{pmatrix}
            1 & 2 & 3 & 4 & 5 \\
            2 & 4 & 5 & 1 & 3
        \end{pmatrix}
        \begin{pmatrix}
            1 & 2 & 3 & 4 & 5 \\
            4 & 3 & 5 & 1 & 2
        \end{pmatrix}
        =
        \begin{pmatrix}
            4 & 3 & 5 & 1 & 2 \\
            1 & 5 & 3 & 2 & 4
        \end{pmatrix}
        \begin{pmatrix}
            1 & 2 & 3 & 4 & 5 \\
            4 & 3 & 5 & 1 & 2
        \end{pmatrix}
        =
        \begin{pmatrix}
            1 & 2 & 3 & 4 & 5 \\
            1 & 5 & 3 & 2 & 4
        \end{pmatrix}.
    \]
    \[
        \begin{pmatrix}
            1 & 2 & 3 & 4 & 5 \\
            1 & 5 & 3 & 2 & 4
        \end{pmatrix}
        =
        (1)(2,5,4)(3).
    \]
    \[
        \sgn{
            \begin{matrix}
                1 & 2 & 3 & 4 & 5 \\
                1 & 5 & 3 & 2 & 4
            \end{matrix}
        }
        = \sgn{1}\sgn{2,5,4}\sgn{3}
        = 1.
    \]
\end{proof}

% exercise 3.2
\begin{exercise}
    $
        \begin{pmatrix}
            1 & 2 & 3 & 4 & 5 \\
            3 & 5 & 4 & 1 & 2
        \end{pmatrix}
        \begin{pmatrix}
            1 & 2 & 3 & 4 & 5 \\
            4 & 3 & 1 & 5 & 2
        \end{pmatrix}
    $.
\end{exercise}

\begin{proof}[Lời giải]
    \[
        \begin{pmatrix}
            1 & 2 & 3 & 4 & 5 \\
            3 & 5 & 4 & 1 & 2
        \end{pmatrix}
        \begin{pmatrix}
            1 & 2 & 3 & 4 & 5 \\
            4 & 3 & 1 & 5 & 2
        \end{pmatrix}
        =
        \begin{pmatrix}
            4 & 3 & 1 & 5 & 2 \\
            1 & 4 & 3 & 2 & 5
        \end{pmatrix}
        \begin{pmatrix}
            1 & 2 & 3 & 4 & 5 \\
            4 & 3 & 1 & 5 & 2
        \end{pmatrix}
        =
        \begin{pmatrix}
            1 & 2 & 3 & 4 & 5 \\
            1 & 4 & 3 & 2 & 5
        \end{pmatrix}.
    \]
    \[
        \begin{pmatrix}
            1 & 2 & 3 & 4 & 5 \\
            1 & 4 & 3 & 2 & 5
        \end{pmatrix}
        =
        (1)(2,4)(3)(5).
    \]
    \[
        \sgn{
            \begin{matrix}
                1 & 2 & 3 & 4 & 5 \\
                1 & 4 & 3 & 2 & 5
            \end{matrix}
        }
        = \sgn{1}\sgn{2,4}\sgn{3}\sgn{5}
        = -1.
    \]
\end{proof}

% exercise 3.3
\begin{exercise}
    $(1,2)(2,3)\ldots (n-1,n)$.
\end{exercise}

\begin{lemma}\label{chapter3:cycles-product}
    $(a_{1}, a_{2}, \ldots, a_{k})(a_{k},a_{k+1}) = (a_{1},a_{2},\ldots, a_{k+1})$.
\end{lemma}

\begin{proof}[Chứng minh bổ đề]
    \par Xét dãy
    \[
        a_{1}, a_{2}, \ldots, a_{k-1}, a_{k}, a_{k+1}.
    \]
    \par Sau khi tác động bằng $(a_{k},a_{k+1})$, dãy trên trở thành:
    \[
        a_{1}, a_{2}, \ldots, a_{k-1}, a_{k+1}, a_{k}.
    \]
    \par Sau khi tác động bằng $(a_{1}, a_{2}, \ldots, a_{k})$, dãy trên (liền trên) trở thành:
    \[
        a_{2}, a_{3}, \ldots, a_{k}, a_{k+1}, a_{1}.
    \]
    \par Theo định nghĩa về xích, ta có điều phải chứng minh.
\end{proof}

\begin{proof}[Lời giải]
    \par Theo bổ đề~\ref{chapter3:cycles-product}:
    \[
        (1,2)(2,3)\ldots (n-1,n) = (1,2,\ldots,n)
        =
        \begin{pmatrix}
            1 & 2 & \cdots & n-1 & n \\
            2 & 3 & \cdots & n   & 1
        \end{pmatrix}
    \]
    \par $(1,2,\ldots, n)$ chính là một xích.
    \[
        \sgn{1,2,\ldots,n} = \sgn{1,2}\sgn{2,3}\ldots\sgn{n-1,n} = (-1){}^{n-1}.
    \]
\end{proof}

% exercise 3.4
\begin{exercise}
    $(1,2,3)(2,3,4)(3,4,5)\ldots (n-2,n-1,n)$.
\end{exercise}

\begin{proof}[Lời giải]
    \par Theo bổ đề~\ref{chapter3:cycles-product}, nếu $n > 3$:
    \begin{align*}
        (1,2,3)(2,3,4)(3,4,5)\ldots (n-2,n-1,n)
         & = (1,2)(2,3)(2,3)(3,4)(3,4)(4,5) \ldots (n-2,n-1)(n-1,n)   \\
         & = (1,2)(2,3){}^{2}(3,4){}^{2}\ldots (n-2,n-1){}^{2}(n-1,n) \\
         & = (1,2)(n-1,n)\qquad\text{(đây là 2 xích rời nhau)}        \\
         & =
        \begin{pmatrix}
            1 & 2 & 3 & \cdots & n-2 & n-1 & n   \\
            2 & 1 & 3 & \cdots & n-2 & n   & n-1
        \end{pmatrix}.
    \end{align*}
    \par Nếu $n = 3$:
    \[
        (1,2,3) =
        \begin{pmatrix}
            1 & 2 & 3 \\
            2 & 3 & 1
        \end{pmatrix}.
    \]
    \par Trong cả hai trường hợp, dấu của phép thế (kết quả) là 1.
\end{proof}

% exercise 3.5
\begin{exercise}
    Cho hai cách sắp thành dãy $a_{1}$, $a_{2}$, \ldots, $a_{n}$ và $b_{1}$, $b_{2}$, \ldots, $b_{n}$ của $n$ số tự nhiên đầu tiên. Chứng minh rằng có thể đưa cách sắp này về cách sắp kia bằng cách sử dụng không quá $n-1$ phép thế sơ cấp.
\end{exercise}

\begin{lemma}\label{chapter3:fixed-point}
    $\forall\sigma,\ \forall x\in\{ 1,2,\ldots,n \}, \exists k\in\mathbb{N}, k\le n$ sao cho $\sigma^{k}(x) = x$.
\end{lemma}

\begin{proof}[Chứng minh bổ đề~\ref{chapter3:existence-of-cycle}]
    \par Giả sử phản chứng, $\sigma^{k}(x)\ne x,\ \forall k=\overline{1,n}$.
    \par Có hai trường hợp cần xem xét.
    \begin{enumerate}[label = Trường hợp \arabic*:,itemindent=2cm]
        \item $n$ số $\sigma(x)$, $\sigma^{2}(x)$, \ldots, $\sigma^{n}(x)$ đôi một khác nhau.
              \par Vì $n$ số này đôi một khác nhau và đều là các phần tử của tập hợp $n$ số tự nhiên đầu tiên nên tồn tại một số tự nhiên $k\le n$ sao cho $\sigma^{k}(x) = x$.
        \item Trong $n$ số $\sigma(x)$, $\sigma^{2}(x)$, \ldots, $\sigma^{n}(x)$, có hai số bằng nhau.
              \par Giả sử $\sigma^{a}(x) = \sigma^{a+b}(x)$, trong đó $1\le a < a + b\le n$.
              \[
                  \begin{split}
                      &(\underbrace{\sigma\circ\cdots\circ\sigma}_{a})(x) = (\underbrace{\sigma\circ\cdots\circ\sigma}_{a+b})(x) \\
                      \Leftrightarrow& ((\underbrace{\sigma^{-1}\circ\ldots\circ\sigma^{-1}}_{a})\circ\underbrace{(\sigma\circ\cdots\circ\sigma)}_{a})(x) = ((\underbrace{\sigma^{-1}\circ\ldots\circ\sigma^{-1}}_{a})\circ\underbrace{(\sigma\circ\cdots\circ\sigma)}_{a+b})(x) \\
                      \Leftrightarrow& x = \sigma^{b}(x).
                  \end{split}
              \]
    \end{enumerate}
    \par Bổ đề được chứng minh.
\end{proof}

\begin{lemma}\label{chapter3:existence-of-cycle}
    Mỗi phần tử của $n$ số tự nhiên đầu tiên luôn thuộc một xích nào đó.
\end{lemma}

\begin{proof}
    \par Theo bổ đề~\ref{chapter3:fixed-point}, cùng với nguyên lý sắp thứ tự tốt (well-ordering principle), ta suy ra luôn chọn được số tự nhiên $k$ nhỏ nhất sao cho $\sigma^{k}(x) = x$.
    \par $(x, \sigma(x), \ldots, \sigma^{k-1}(x))$ chính là một xích độ dài $k$.
\end{proof}

\begin{lemma}\label{chapter3:product-of-disjoint-cycles}
    Mọi phép thế đều có thể được viết dưới dạng tích của các xích rời nhau.
\end{lemma}

\begin{proof}[Chứng minh bổ đề~\ref{chapter3:product-of-disjoint-cycles}]
    \par Theo bổ đề~\ref{chapter3:existence-of-cycle}, mỗi phần tử đều thuộc một xích nào đó.
    \par Nếu hai xích cùng chứa một phần tử $x$ thì hai xích đó trùng nhau.
    \par Do đó hai xích bất kì hoặc trùng nhau, hoặc rời nhau.
    \par Ta tiến hành phân tích một phép thế thành các xích rời nhau.
    \begin{enumerate}[label = (\arabic*)]
        \item Chọn lấy một phần tử bất kì của tập hợp $n$ số tự nhiên đầu tiên.
        \item Xác định xích của phần tử đó.
        \item Chọn lấy một phần tử bất kì không thuộc xích, tiếp xúc xác định xích của phần tử mới này.
    \end{enumerate}
    \par Quy trình trên được làm liên tục đến khi không còn phần tử nào để chọn. Quy trình này sẽ dừng lại sau hữu hạn bước vì số phần tử của tập hợp $n$ số tự nhiên là hữu hạn, và độ dài của một xích cũng không vượt quá số phần tử của tập hợp $n$ số tự nhiên đầu tiên.
\end{proof}

\begin{lemma}\label{chapter3:product-of-transpositions}
    Một xích độ dài $k$ ($k > 1$) có thể viết được dưới dạng tích của $k-1$ phép thế sơ cấp.
\end{lemma}

\begin{proof}[Chứng minh bổ đề~\ref{chapter3:product-of-transpositions}]
    \par Một xích độ dài $k$ sẽ tác động lên dãy $x_{1}$, $x_{2}$, \ldots, $x_{k}$ như sau:
    \[
        x_{1}\mapsto x_{2} \mapsto x_{3} \mapsto \cdots \mapsto x_{k} \mapsto x_{1}.
    \]
    \par Xích trên có thể phân tích được thành tích của $k - 1$ phép thế sơ cấp, theo bổ đề~\ref{chapter3:cycles-product}:
    \begin{align*}
        (x_{1}, x_{2}, \ldots, x_{k}) & = (x_{1}, x_{2}, \ldots, x_{k-1})(x_{k-1}, x_{k})                   \\
                                      & = (x_{1}, x_{2}, \ldots, x_{k-2})(x_{k-2}, x_{k-1})(x_{k-1}, x_{k}) \\
                                      & = \ldots                                                            \\
                                      & = (x_{1}, x_{2})(x_{2}, x_{3})\ldots (x_{k-1}, x_{k}).
    \end{align*}
\end{proof}

\begin{proof}
    \par Xét phép thế $\sigma$ trên tập hợp $n$ số tự nhiên đầu tiên.
    \par Theo bổ đề~\ref{chapter3:product-of-disjoint-cycles}, ta phân tích phép thế $\sigma$ thành các xích rời nhau. Tổng độ dài của các xích này bằng $n$. Gọi số lượng xích rời rạc là $\ell$.
    \par Theo bổ đề~\ref{chapter3:product-of-transpositions}, mỗi xích độ dài $k$ đều phân tích được thành tích của $k-1$ phép thế sơ cấp.
    \par Kết hợp hai điều trên, một phép thế có thể phân tích được thành tích của $n - \ell$ phép thế sơ cấp. Mà $n - \ell \le n - 1$ nên ta có điều phải chứng minh.
    \par Điều này tương đương với việc có thể đưa cách sắp dãy $a_{1}$, $a_{2}$, \ldots, $a_{n}$ thành $b_{1}$, $b_{2}$, \ldots, $b_{n}$ và ngược lại bằng cách thực hiện không quá $n - 1$ phép thế sơ cấp.
\end{proof}

% exercise 3.6
\begin{exercise}
    Với giả thiết như bài trên, chứng minh rằng có thể đưa cách sắp này về cách sắp kia bằng cách sử dụng không quá $n(n-1)/2$ phép chuyển vị của hai phần tử đứng kề nhau.
\end{exercise}

\begin{proof}
    \par Ta chỉ cần chứng minh có thể đưa cách sắp thứ nhất thành cách sắp thứ hai bằng cách sử dụng không quá $n(n-1)/2$ phép chuyển vị của hai phần tử đứng kề nhau.
    \par Như giả thiết, cách sắp dãy thứ nhất là $a_{1}$, $a_{2}$, \ldots, $a_{n}$; cách sắp dãy thứ hai là $b_{1}$, $b_{2}$, \ldots, $b_{n}$.
    \par Lưu ý rằng $\{ a_{1}, a_{2}, \ldots, a_{n} \} = \{ b_{1}, b_{2}, \ldots, b_{n} \}$.

    \begin{enumerate}[label = (\arabic*)]
        \item Tồn tại duy nhất $a_{k_{1}}$ sao cho $a_{k_{1}} = b_{n}$.
        \item Ta lần lượt thực hiện cách phép chuyển vị hai phần tử kề nhau: $(a_{k_{1}}, a_{k_{1}+1})$, $(a_{k_{1}}, a_{k_{1}+2})$, \ldots $(a_{k_{1}}, a_{n})$.
        \item Sau khi thực hiện các phép chuyển vị như trên, $a_{k}$ ở vị trí cuối cùng trong dãy.
    \end{enumerate}
    \par Trong quy trình trên, ta thực hiện $n - k_{1}$ phép chuyển vị hai phần tử kề nhau.
    \par Trong $n - 1$ phần tử đầu tiên của dãy $a_{1}$, $a_{2}$, \ldots, $a_{n}$, tồn tại $a_{k_{2}} = b_{n-1}$,  ta thực hiện các phép chuyển vị $(a_{k_{2}}, a_{k_{2}+1})$, $(a_{k_{2}}, a_{k_{2}+2})$, \ldots $(a_{k_{2}}, a_{n-1})$. Ta đã thực hiện $(n - 1) - k_{2}$ phép chuyển vị hai phần tử kề nhau.
    \par Liên tiếp thực hiện việc dồn dãy như trên, ta đưa được cách sắp thứ nhất về cách sắp thứ hai $-$ chỉ bằng các phép chuyển vị hai phần tử kề nhau.
    \par Số phép chuyển vị hai phần tử kề nhau đã được sử dụng không vượt quá
    \[
        (n - 1) + (n - 2) + \cdots + 2 + 1 = \dfrac{n(n-1)}{2}.
    \]
    \par Đó cũng là điều phải chứng minh.
\end{proof}

% exercise 3.7
\begin{exercise}
    Cho ví dụ về một cách sắp dãy $n$ số tự nhiên đầu tiên thành dãy sao cho dãy này không thể đưa về dãy sắp tự nhiên bằng cách dùng ít hơn $n - 1$ phép thế sơ cấp.
\end{exercise}

\begin{proof}[Lời giải]
    \par Cách sắp như vậy được tạo ra từ một xích độ dài $n$.
    \[
        2, 3, \ldots, n, 1.
    \]
\end{proof}

% exercise 3.8
\begin{exercise}
    Biết số nghịch thế của dãy $a_{1}$, $a_{2}$, \ldots, $a_{n}$ là $k$. Hãy tìm số nghịch thế của dãy $a_{n}$, $a_{n-1}$, \ldots, $a_{1}$.
\end{exercise}

\begin{proof}[Lời giải]
    \par Không mất tính tổng quát, ta giả sử $a_{1}$, $a_{2}$, \ldots, $a_{n}$ là một hoán vị của $n$ số tự nhiên đầu tiên.
    \par Nếu cặp $(a_{i}, a_{j})$ trong dãy $a_{1}$, $a_{2}$, \ldots, $a_{n}$ là nghịch thế thì trong dãy $a_{n}$, $a_{n-1}$, \ldots, $a_{1}$ lại không phải nghịch thế.
    \par Nếu cặp $(a_{i}, a_{j})$ trong dãy $a_{1}$, $a_{2}$, \ldots, $a_{n}$ không phỉa nghịch thế thì trong dãy $a_{n}$, $a_{n-1}$, \ldots, $a_{1}$ lại là phải nghịch thế.
    \par Do đó, số nghịch thế trong $a_{n}$, $a_{n-1}$, \ldots, $a_{1}$ là $\dfrac{n(n-1)}{2} - k$.
\end{proof}

% exercise 3.9
\begin{exercise}
    Tính các định thức sau đây
    \begin{enumerate}[label = (\alph*)]
        \item $
                  \begin{vmatrix}
                      2  & -5 & 4  & 3 \\
                      3  & -4 & 7  & 5 \\
                      4  & -9 & 8  & 5 \\
                      -3 & 2  & -5 & 3
                  \end{vmatrix}
              $
        \item $
                  \begin{vmatrix}
                      3 & -3 & -2 & -5 \\
                      2 & 5  & 4  & 6  \\
                      5 & 5  & 8  & 7  \\
                      4 & 4  & 5  & 6
                  \end{vmatrix}
              $
    \end{enumerate}
\end{exercise}

\begin{proof}[Lời giải]
    \begin{enumerate}[label = (\alph*)]
        \item
              \begin{align*}
                  \begin{vmatrix}
                      2  & -5 & 4  & 3 \\
                      3  & -4 & 7  & 5 \\
                      4  & -9 & 8  & 5 \\
                      -3 & 2  & -5 & 3
                  \end{vmatrix}
                   & =
                  \begin{vmatrix}
                      2 & -5 & 4 & 3  \\
                      3 & -4 & 7 & 5  \\
                      0 & 1  & 0 & -1 \\
                      0 & -2 & 2 & 8
                  \end{vmatrix}
                  =
                  \begin{vmatrix}
                      2 & -5  & 4 & 3   \\
                      0 & 3.5 & 1 & 0.5 \\
                      0 & 1   & 0 & -1  \\
                      0 & -2  & 2 & 8
                  \end{vmatrix}
                  =
                  \begin{vmatrix}
                      2 & -5  & 4 & 3   \\
                      0 & 1   & 0 & -1  \\
                      0 & -2  & 2 & 8   \\
                      0 & 3.5 & 1 & 0.5
                  \end{vmatrix} \\
                   & =
                  \begin{vmatrix}
                      2 & -5 & 4 & 3  \\
                      0 & 1  & 0 & -1 \\
                      0 & 0  & 2 & 6  \\
                      0 & 0  & 1 & 4
                  \end{vmatrix}
                  =
                  \begin{vmatrix}
                      2 & -5 & 4 & 3  \\
                      0 & 1  & 0 & -1 \\
                      0 & 0  & 2 & 6  \\
                      0 & 0  & 0 & 1
                  \end{vmatrix}
                  = 2\cdot 1\cdot 2\cdot 1 = 4.
              \end{align*}
        \item
              \begin{align*}
                  \begin{vmatrix}
                      3 & -3 & -2 & -5 \\
                      2 & 5  & 4  & 6  \\
                      5 & 5  & 8  & 7  \\
                      4 & 4  & 5  & 6
                  \end{vmatrix}
                   & =
                  \begin{vmatrix}
                      1 & -8 & -6 & -11 \\
                      2 & 5  & 4  & 6   \\
                      5 & 5  & 8  & 7   \\
                      4 & 4  & 5  & 6
                  \end{vmatrix}
                  =
                  \begin{vmatrix}
                      1 & -8 & -6 & -11 \\
                      0 & 21 & 16 & 28  \\
                      0 & 45 & 38 & 62  \\
                      0 & 36 & 29 & 50
                  \end{vmatrix}
                  =
                  \begin{vmatrix}
                      21 & 16 & 28 \\
                      45 & 38 & 62 \\
                      36 & 29 & 50
                  \end{vmatrix}                                                                                     \\
                   & = 21(38\cdot 50 - 62\cdot 29) + 16(62\cdot 36 - 45\cdot 50) + 28(45\cdot 29 - 36\cdot 38) = 90.
              \end{align*}
    \end{enumerate}
\end{proof}

% exercise 3.10
\begin{exercise}
    \par Tính các định thức sau bằng cách \textit{đưa về dạng tam giác}:
    \begin{enumerate}[label = (\alph*)]
        \item $\begin{vmatrix} 1 & 2 & 3 & \cdots & n \\ -1 & 0 & 3 & \cdots & n \\ -1 & -2 & 0 & \cdots & n \\ \vdots & \vdots & \vdots & \ddots & \vdots \\ -1 & -2 & -3 & \cdots & 0 \end{vmatrix}$,
        \item $\begin{vmatrix} a_{0} & a_{1} & a_{2} & \cdots & a_{n} \\ -x & x & 0 & \cdots & 0 \\ 0 & -x & x & \cdots & 0 \\ \vdots & \vdots & \vdots & \ddots & \vdots \\ 0 & 0 & 0 & \cdots & x \end{vmatrix}$,
        \item $\begin{vmatrix} a_{1} & a_{2} & a_{3} & \cdots & a_{n} \\ -x_{1} & x_{2} & 0 & \cdots & 0 \\ 0 & -x_{2} & x_{3} & \cdots & 0 \\ \vdots & \vdots & \vdots & \ddots & \vdots \\ 0 & 0 & 0 & \cdots & x_{n} \end{vmatrix}$.
    \end{enumerate}
\end{exercise}

\begin{proof}[Lời giải]
    \begin{enumerate}[label = (\alph*)]
        \item $r_{k}:= r_{k} + r_{1}$, $\forall 1 < k\le n$.
              \begin{align*}
                  \begin{vmatrix}
                      1      & 2      & 3      & \cdots & n      \\
                      -1     & 0      & 3      & \cdots & n      \\
                      -1     & -2     & 0      & \cdots & n      \\
                      \vdots & \vdots & \vdots & \ddots & \vdots \\
                      -1     & -2     & -3     & \cdots & 0
                  \end{vmatrix}
                  =
                  \begin{vmatrix}
                      1      & 2      & 3      & \cdots & n      \\
                      0      & 2      & 6      & \cdots & 2n     \\
                      0      & 0      & 3      & \cdots & 2n     \\
                      \vdots & \vdots & \vdots & \ddots & \vdots \\
                      0      & 0      & 0      & \cdots & n
                  \end{vmatrix}
                  = n!
              \end{align*}
        \item Thực hiện lần lượt các biến đổi sơ cấp sau:
              \begin{itemize}
                  \item $c_{0} = \displaystyle\sum^{n}_{k=0}c_{k}$.
                  \item $c_{1} = \displaystyle\sum^{n}_{k=1}c_{k}$.
                  \item $\cdots$
                  \item $c_{n} = \displaystyle\sum^{n}_{k=n}c_{k}$.
              \end{itemize}
              \begingroup{}
              \allowdisplaybreaks{}
              \begin{align*}
                  \begin{vmatrix}
                      a_{0}  & a_{1}  & a_{2}  & \cdots & a_{n}  \\
                      -x     & x      & 0      & \cdots & 0      \\
                      0      & -x     & x      & \cdots & 0      \\
                      \vdots & \vdots & \vdots & \ddots & \vdots \\
                      0      & 0      & 0      & \cdots & x
                  \end{vmatrix}
                   & =
                  \begin{vmatrix}
                      \sum^{n}_{k=0}a_{k} & a_{1}  & a_{2}  & \cdots & a_{n} \\
                      0                   & x      & 0      & \cdots & 0     \\
                      0                   & -x     & x      & \cdots & 0     \\
                      \vdots              & \vdots & \vdots & \ddots & 0     \\
                      0                   & 0      & 0      & \cdots & x
                  \end{vmatrix}
                  =
                  \begin{vmatrix}
                      \sum^{n}_{k=0}a_{k} & \sum^{n}_{k=1}a_{k} & a_{2}  & \cdots & a_{n} \\
                      0                   & x                   & 0      & \cdots & 0     \\
                      0                   & 0                   & x      & \cdots & 0     \\
                      \vdots              & \vdots              & \vdots & \ddots & 0     \\
                      0                   & 0                   & 0      & \cdots & x
                  \end{vmatrix} \\
                   & =
                  \begin{vmatrix}
                      \sum^{n}_{k=0}a_{k} & \sum^{n}_{k=1}a_{k} & \sum^{n}_{k=n}a_{k} & \cdots & a_{n} \\
                      0                   & x                   & 0                   & \cdots & 0     \\
                      0                   & 0                   & x                   & \cdots & 0     \\
                      \vdots              & \vdots              & \vdots              & \ddots & 0     \\
                      0                   & 0                   & 0                   & \cdots & x
                  \end{vmatrix}
                  = \left(\sum^{n}_{k=0}a_{k}\right)x^{n}.
              \end{align*}
              \endgroup{}
        \item Nếu $x_{1} = 0$
              \begin{align*}
                  \begin{vmatrix}
                      a_{1}  & a_{2}  & a_{3}  & \cdots & a_{n}  \\
                      -x_{1} & x_{2}  & 0      & \cdots & 0      \\
                      0      & -x_{2} & x_{3}  & \cdots & 0      \\
                      \vdots & \vdots & \vdots & \ddots & \vdots \\
                      0      & 0      & 0      & \cdots & x_{n}
                  \end{vmatrix}
                   & =
                  \begin{vmatrix}
                      a_{1}  & a_{2}  & a_{3}  & \cdots & a_{n}  \\
                      0      & x_{2}  & 0      & \cdots & 0      \\
                      0      & -x_{2} & x_{3}  & \cdots & 0      \\
                      \vdots & \vdots & \vdots & \ddots & \vdots \\
                      0      & 0      & 0      & \cdots & x_{n}
                  \end{vmatrix}             \\
                   & = a_{1} \begin{vmatrix}
                                 x_{2}  & 0      & \cdots & 0      \\
                                 -x_{2} & x_{3}  & \cdots & 0      \\
                                 \vdots & \vdots & \ddots & \vdots \\
                                 0      & 0      & \cdots & x_{n}
                             \end{vmatrix} = a_{1}x_{2}x_{3}\cdots x_{n}
              \end{align*}

              \par Nếu $x_{k} = 0$
              \begin{align*}
                  \begin{vmatrix}
                      a_{1}  & a_{2}  & a_{3}  & \cdots & a_{n}  \\
                      -x_{1} & x_{2}  & 0      & \cdots & 0      \\
                      0      & -x_{2} & x_{3}  & \cdots & 0      \\
                      \vdots & \vdots & \vdots & \ddots & \vdots \\
                      0      & 0      & 0      & \cdots & x_{n}
                  \end{vmatrix}
                   & = a_{k}\prod^{n}_{i=1,i\ne k} x_{i}
              \end{align*}
              \par Nếu $x_{k} \ne 0, \forall k=\overline{1, n}$.
              \begingroup{}
              \allowdisplaybreaks{}
              \begin{align*}
                  \begin{vmatrix}
                      a_{1}  & a_{2}  & a_{3}  & \cdots & a_{n}  \\
                      -x_{1} & x_{2}  & 0      & \cdots & 0      \\
                      0      & -x_{2} & x_{3}  & \cdots & 0      \\
                      \vdots & \vdots & \vdots & \ddots & \vdots \\
                      0      & 0      & 0      & \cdots & x_{n}
                  \end{vmatrix}
                   & =
                  \frac{1}{x_{1}x_{2}}
                  \begin{vmatrix}
                      a_{1}x_{2}  & a_{2}x_{1}  & a_{3}  & \cdots & a_{n}  \\
                      -x_{1}x_{2} & x_{1}x_{2}  & 0      & \cdots & 0      \\
                      0           & -x_{1}x_{2} & x_{3}  & \cdots & 0      \\
                      \vdots      & \vdots      & \vdots & \ddots & \vdots \\
                      0           & 0           & 0      & \cdots & x_{n}
                  \end{vmatrix}                                                                                                \\
                   & = \frac{1}{x_{1}x_{2}}
                  \begin{vmatrix}
                      a_{1}x_{2} + a_{2}x_{1} & a_{2}x_{1}  & a_{3}  & \cdots & a_{n}  \\
                      0                       & x_{1}x_{2}  & 0      & \cdots & 0      \\
                      -x_{1}x_{2}             & -x_{1}x_{2} & x_{3}  & \cdots & 0      \\
                      \vdots                  & \vdots      & \vdots & \ddots & \vdots \\
                      0                       & 0           & 0      & \cdots & x_{n}
                  \end{vmatrix}                                                                                    \\
                   & = \frac{1}{x^{2}_{1}x^{2}_{2}x^{2}_{3}}
                  \begin{vmatrix}
                      a_{1}x_{2}x_{3} + a_{2}x_{1}x_{3} & a_{2}x_{1}x_{3}  & a_{3}x_{1}x_{2} & \cdots & a_{n}  \\
                      0                                 & x_{1}x_{2}x_{3}  & 0               & \cdots & 0      \\
                      -x_{1}x_{2}x_{3}                  & -x_{1}x_{2}x_{3} & x_{1}x_{2}x_{3} & \cdots & 0      \\
                      \vdots                            & \vdots           & \vdots          & \ddots & \vdots \\
                      0                                 & 0                & 0               & \cdots & x_{n}
                  \end{vmatrix}                                                            \\
                   & = \frac{1}{x^{2}_{1}x^{2}_{2}x^{2}_{3}}
                  \begin{vmatrix}
                      a_{1}x_{2}x_{3} + a_{2}x_{1}x_{3} + a_{3}x_{1}x_{2} & a_{2}x_{1}x_{3}  & a_{3}x_{1}x_{2} & \cdots & a_{n}  \\
                      0                                                   & x_{1}x_{2}x_{3}  & 0               & \cdots & 0      \\
                      0                                                   & -x_{1}x_{2}x_{3} & x_{1}x_{2}x_{3} & \cdots & 0      \\
                      \vdots                                              & \vdots           & \vdots          & \ddots & \vdots \\
                      0                                                   & 0                & 0               & \cdots & x_{n}
                  \end{vmatrix}                                          \\
                   & = \frac{1}{(x_{1}x_{2}\ldots x_{n}){}^{n-1}}(a_{1}x_{2}x_{3}\ldots x_{n} + a_{2}x_{1}x_{3}\ldots x_{n} + \cdots + a_{n}x_{1}x_{2}\ldots x_{n-1}) \\
                   & \times (x_{2}x_{3}\ldots x_{n}) (x_{1}x_{3}\ldots x_{n}) \cdots (x_{1}x_{2}\ldots x_{n-1})                                                       \\
                   & = a_{1}x_{2}x_{3}\ldots x_{n} + a_{2}x_{1}x_{3}\ldots x_{n} + \cdots + a_{n}x_{1}x_{2}\ldots x_{n-1}                                             \\
                   & = \sum^{n}_{k = 1}\left(a_{k}\prod^{n}_{i\ne k}x_{i}\right)
              \end{align*}
              \endgroup{}
    \end{enumerate}
\end{proof}

% exercise 3.11
\begin{exercise}
    Tính định thức của ma trận vuông cỡ $n$ với yếu tố nằm ở hàng $i$ cột $j$ bằng $\abs{i - j}$.
\end{exercise}

\begin{proof}[Lời giải]
    \begingroup{}
    \allowdisplaybreaks{}
    \begin{align*}
        \begin{vmatrix}
            0      & 1      & 2      & \cdots & n - 1  \\
            1      & 0      & 1      & \cdots & n - 2  \\
            2      & 1      & 0      & \cdots & n - 3  \\
            \vdots & \vdots & \vdots & \ddots & \vdots \\
            n - 1  & n - 2  & n - 3  & \cdots & 0
        \end{vmatrix}
         & =
        \begin{vmatrix}
            0      & 1      & 2      & \cdots & n - 1  \\
            1      & -1     & -1     & \cdots & -1     \\
            1      & 1      & -1     & \cdots & -1     \\
            \vdots & \vdots & \vdots & \ddots & \vdots \\
            1      & 1      & 1      & \cdots & -1
        \end{vmatrix} (r_{k}:= r_{k} - r_{k - 1}) \\
         & =
        \begin{vmatrix}
            n - 1  & 1      & 2      & \cdots & n - 1  \\
            0      & -1     & -1     & \cdots & -1     \\
            0      & 1      & -1     & \cdots & -1     \\
            \vdots & \vdots & \vdots & \ddots & \vdots \\
            0      & 1      & 1      & \cdots & -1
        \end{vmatrix} (c_{1}:= c_{1} + c_{n}) \\
         & =
        \begin{vmatrix}
            n - 1  & 1      & 2      & \cdots & n - 1  \\
            0      & -1     & -1     & \cdots & -1     \\
            0      & 0      & -2     & \cdots & -2     \\
            \vdots & \vdots & \vdots & \ddots & \vdots \\
            0      & 0      & 0      & \cdots & -2
        \end{vmatrix} (r_{k}:= r_{k} + r_{k-1}) \\
         & = (-2){}^{n-1}(n - 1).
    \end{align*}
    \endgroup{}
\end{proof}

% exercise 3.12
\begin{exercise}
    \par Tính các định thức sau đây bằng \textit{phương pháp rút ra các nhân tử tuyến tính}:
    \begin{enumerate}[label = (\alph*)]
        \item $\begin{vmatrix} 1 & 2 & 3 & \cdots & n \\ 1 & x + 1 & 3 & \cdots & n \\ 1 & 2 & x + 1 & \cdots & n \\ \vdots & \vdots & \vdots & \ddots & \vdots \\ 1 & 2 & 3 & \cdots & x + 1 \end{vmatrix}$,
        \item $\begin{vmatrix} 1 + x & 1 & 1 & 1 \\ 1 & 1 - x & 1 & 1 \\ 1 & 1 & 1 + y & 1 \\ 1 & 1 & 1 & 1 - y \end{vmatrix}$.
    \end{enumerate}
\end{exercise}

\begin{proof}[Lời giải]
    \begin{enumerate}[label = (\alph*)]
        \item
              \begin{align*}
                  \begin{vmatrix}
                      1      & 2      & 3      & \cdots & n      \\
                      1      & x + 1  & 3      & \cdots & n      \\
                      1      & 2      & x + 1  & \cdots & n      \\
                      \vdots & \vdots & \vdots & \ddots & \vdots \\
                      1      & 2      & 3      & \cdots & x + 1
                  \end{vmatrix}
                   & =
                  \begin{vmatrix}
                      1      & 0      & 0      & \cdots & 0           \\
                      1      & x - 1  & 0      & \cdots & 0           \\
                      1      & 0      & x - 2  & \cdots & 0           \\
                      \vdots & \vdots & \vdots & \ddots & \vdots      \\
                      1      & 0      & 0      & \cdots & x - (n - 1) \\
                  \end{vmatrix} (c_{k}:= c_{k} - k\times c_{1}) \\
                   & =
                  \begin{vmatrix}
                      1      & 0      & 0      & \cdots & 0           \\
                      0      & x - 1  & 0      & \cdots & 0           \\
                      0      & 0      & x - 2  & \cdots & 0           \\
                      \vdots & \vdots & \vdots & \ddots & \vdots      \\
                      0      & 0      & 0      & \cdots & x - (n - 1) \\
                  \end{vmatrix} (r_{k}:= r_{k} - r_{1}) \\
                   & = (x - 1)(x - 2)\cdots (x - n + 1).
              \end{align*}
        \item Đặt $\varepsilon_{1} = \begin{pmatrix}1 \\ 0 \\ 0 \\ 0\end{pmatrix}$, $\varepsilon_{2} = \begin{pmatrix}0 \\ 1 \\ 0 \\ 0\end{pmatrix}$, $\varepsilon_{3} = \begin{pmatrix}0 \\ 0 \\ 1 \\ 0\end{pmatrix}$, $\varepsilon_{4} = \begin{pmatrix}0 \\ 0 \\ 0 \\ 1\end{pmatrix}$, $\varepsilon = \begin{pmatrix}1 \\ 1 \\ 1 \\ 1\end{pmatrix}$.
              \begin{align*}
                  \begin{vmatrix}
                      1 + x & 1     & 1     & 1     \\
                      1     & 1 - x & 1     & 1     \\
                      1     & 1     & 1 + y & 1     \\
                      1     & 1     & 1     & 1 - y \\
                  \end{vmatrix}
                   & = \det(\varepsilon + x\varepsilon_{1}, \varepsilon - x\varepsilon_{2}, \varepsilon + y\varepsilon_{3}, \varepsilon - y\varepsilon_{4})                       \\
                   & = \det(\varepsilon, -x\varepsilon_{2}, y\varepsilon_{3}, -y\varepsilon_{4}) + \det(x\varepsilon_{1}, \varepsilon, y\varepsilon_{3}, -y\varepsilon_{4})       \\
                   & \quad + \det(x\varepsilon_{1}, -x\varepsilon_{2}, \varepsilon, -y\varepsilon_{4}) + \det(x\varepsilon_{1}, -x\varepsilon_{2}, y\varepsilon_{3}, \varepsilon) \\
                   & \quad + \det(x\varepsilon_{1}, -x\varepsilon_{2}, y\varepsilon_{3}, -y\varepsilon_{4})                                                                       \\
                   & = xy^{2} - xy^{2} + x^{2}y - x^{2}y + x^{2}y^{2}                                                                                                             \\
                   & = x^{2}y^{2}.
              \end{align*}
    \end{enumerate}
\end{proof}

% exercise 3.13
\begin{exercise}
    \par Tính các định thức sau đây bằng cách \textit{sử dụng các quan hệ hồi quy}:
    \begin{enumerate}[label = (\alph*)]
        \item $\begin{vmatrix}
                      a_{1}b_{1} & a_{1}b_{2} & a_{1}b_{3} & \cdots & a_{1}b_{n} \\
                      a_{1}b_{2} & a_{2}b_{2} & a_{2}b_{3} & \cdots & a_{2}b_{n} \\
                      a_{1}b_{3} & a_{2}b_{3} & a_{3}b_{3} & \cdots & a_{3}b_{n} \\
                      \vdots     & \vdots     & \vdots     & \ddots & \vdots     \\
                      a_{1}b_{n} & a_{2}b_{n} & a_{3}b_{n} & \cdots & a_{n}b_{n}
                  \end{vmatrix}$
        \item $\begin{vmatrix}
                      a_{0}  & a_{1}  & a_{2}  & \cdots & a_{n}  \\
                      -y_{1} & x_{1}  & 0      & \cdots & 0      \\
                      0      & -y_{2} & x_{2}  & \cdots & 0      \\
                      \vdots & \vdots & \vdots & \ddots & \vdots \\
                      0      & 0      & 0      & \cdots & x_{n}
                  \end{vmatrix}$
    \end{enumerate}
\end{exercise}

\begin{proof}[Lời giải]
    \begin{enumerate}[label = (\alph*)]
        \item
              \begin{align*}
                  \begin{vmatrix}
                      a_{1}b_{1} & a_{1}b_{2} & a_{1}b_{3} & \cdots & a_{1}b_{n} \\
                      a_{1}b_{2} & a_{2}b_{2} & a_{2}b_{3} & \cdots & a_{2}b_{n} \\
                      a_{1}b_{3} & a_{2}b_{3} & a_{3}b_{3} & \cdots & a_{3}b_{n} \\
                      \vdots     & \vdots     & \vdots     & \ddots & \vdots     \\
                      a_{1}b_{n} & a_{2}b_{n} & a_{3}b_{n} & \cdots & a_{n}b_{n}
                  \end{vmatrix}
                   & =
                  a_{1}b_{n}
                  \begin{vmatrix}
                      b_{1}  & a_{1}b_{2} & a_{1}b_{3} & \cdots & a_{1}b_{n} \\
                      b_{2}  & a_{2}b_{2} & a_{2}b_{3} & \cdots & a_{2}b_{n} \\
                      b_{3}  & a_{2}b_{3} & a_{3}b_{3} & \cdots & a_{3}b_{n} \\
                      \vdots & \vdots     & \vdots     & \ddots & \vdots     \\
                      1      & a_{2}      & a_{3}      & \cdots & a_{n}
                  \end{vmatrix}                                            \\
                   & =
                  a_{1}b_{n}
                  \begin{vmatrix}
                      0      & a_{1}b_{2} - a_{2}b_{1} & a_{1}b_{3} - a_{3}b_{1} & \cdots & a_{1}b_{n} - a_{n}b_{1} \\
                      0      & 0                       & a_{2}b_{3} - a_{3}b_{2} & \cdots & a_{2}b_{n} - a_{n}b_{2} \\
                      0      & 0                       & 0                       & \cdots & a_{3}b_{n} - a_{n}b_{3} \\
                      \vdots & \vdots                  & \vdots                  & \ddots & \vdots                  \\
                      1      & a_{2}                   & a_{3}                   & \cdots & a_{n}
                  \end{vmatrix} (r_{k}:= r_{k} - b_{k}r_{n})     \\
                   & =
                  (-1){}^{n-1}a_{1}b_{n}
                  \begin{vmatrix}
                      1      & a_{2}                   & a_{3}                   & \cdots & a_{n}                       \\
                      0      & a_{1}b_{2} - a_{2}b_{1} & a_{1}b_{3} - a_{3}b_{1} & \cdots & a_{1}b_{n} - a_{n}b_{1}     \\
                      0      & 0                       & a_{2}b_{3} - a_{3}b_{2} & \cdots & a_{2}b_{n} - a_{n}b_{2}     \\
                      0      & 0                       & 0                       & \cdots & a_{3}b_{n} - a_{n}b_{3}     \\
                      \vdots & \vdots                  & \vdots                  & \ddots & \vdots                      \\
                      0      & 0                       & 0                       & \cdots & a_{n-1}b_{n} - a_{n}b_{n-1}
                  \end{vmatrix} \\
                   & =
                  (-1){}^{n-1}a_{1}b_{n}(a_{1}b_{2} - a_{2}b_{1})(a_{2}b_{3} - a_{3}b_{2})\cdots (a_{n-1}b_{n} - a_{n}b_{n-1}).
              \end{align*}
        \item
              \begin{align*}
                  \begin{vmatrix}
                      a_{0}  & a_{1}  & a_{2}  & \cdots & a_{n}  \\
                      -y_{1} & x_{1}  & 0      & \cdots & 0      \\
                      0      & -y_{2} & x_{2}  & \cdots & 0      \\
                      \vdots & \vdots & \vdots & \ddots & \vdots \\
                      0      & 0      & 0      & \cdots & x_{n}
                  \end{vmatrix}
                   & =
                  \begin{vmatrix}
                      a_{0}  & a_{1}  & a_{2}  & \cdots & a_{n}  \\
                      0      & x_{1}  & 0      & \cdots & 0      \\
                      0      & -y_{2} & x_{2}  & \cdots & 0      \\
                      \vdots & \vdots & \vdots & \ddots & \vdots \\
                      0      & 0      & 0      & \cdots & x_{n}
                  \end{vmatrix}
                  +
                  \begin{vmatrix}
                      0      & a_{1}  & a_{2}  & \cdots & a_{n}  \\
                      -y_{1} & x_{1}  & 0      & \cdots & 0      \\
                      0      & -y_{2} & x_{2}  & \cdots & 0      \\
                      \vdots & \vdots & \vdots & \ddots & \vdots \\
                      0      & 0      & 0      & \cdots & x_{n}
                  \end{vmatrix}                                                         \\
                   & =
                  a_{0}x_{1}x_{2}\cdots x_{n}
                  +
                  \begin{vmatrix}
                      y_{1}  & -x_{1} & 0      & \cdots & 0      \\
                      0      & a_{1}  & a_{2}  & \cdots & a_{n}  \\
                      0      & -y_{2} & x_{2}  & \cdots & 0      \\
                      \vdots & \vdots & \vdots & \ddots & \vdots \\
                      0      & 0      & 0      & \cdots & x_{n}
                  \end{vmatrix}                                                         \\
                   & =
                  a_{0}x_{1}x_{2}\cdots x_{n}
                  +
                  y_{1}
                  \begin{vmatrix}
                      a_{1}  & a_{2}  & \cdots & a_{n}  \\
                      -y_{2} & x_{2}  & \cdots & 0      \\
                      \vdots & \vdots & \ddots & \vdots \\
                      0      & 0      & \cdots & x_{n}
                  \end{vmatrix} \text{(hệ thức truy hồi)}                                                            \\
                   & = a_{0}x_{1}x_{2}\cdots x_{n}
                  + y_{1} (a_{1}x_{2}\cdots x_{n})
                  + \cdots
                  + y_{1}y_{2}\cdots y_{n} (a_{n})                                                                   \\
                   & = \sum^{n}_{k=0} a_{k}\left(\prod^{k}_{i=1}y_{i}\right)\left(\prod^{n-1}_{i=k}x_{i + 1}\right).
              \end{align*}
    \end{enumerate}
\end{proof}

% exercise 3.14
\begin{exercise}
    Tính các định thức sau đây bằng cách biểu diễn chúng thành tổng của các định thức nào đó:
    \begin{enumerate}[label = (\alph*)]
        \item $\begin{vmatrix}
                      x + a_{1} & a_{2}     & \cdots & a_{n}     \\
                      a_{1}     & x + a_{2} & \cdots & a_{n}     \\
                      \vdots    & \vdots    & \ddots & \vdots    \\
                      a_{1}     & a_{2}     & \cdots & x + a_{n}
                  \end{vmatrix}$,
        \item $\begin{vmatrix}
                      x_{1}  & a_{2}  & \cdots & a_{n}  \\
                      a_{1}  & x_{2}  & \cdots & a_{n}  \\
                      \vdots & \vdots & \ddots & \vdots \\
                      a_{1}  & a_{2}  & \cdots & x_{n}
                  \end{vmatrix}$.
    \end{enumerate}
\end{exercise}

\begin{proof}[Lời giải]
    \par Đặt $\varepsilon_{1} = (1, 0, \ldots, 0)$, $\varepsilon_{2} = (0, 1, \ldots, 0)$, \ldots $\varepsilon_{n} = (0, 0, \ldots, 1)$, $\varepsilon = \displaystyle\sum^{n}_{i}\varepsilon_{i}$.
    \begin{enumerate}[label = (\alph*)]
        \item
              \begin{align*}
                  \begin{vmatrix}
                      x + a_{1} & a_{2}     & \cdots & a_{n}     \\
                      a_{1}     & x + a_{2} & \cdots & a_{n}     \\
                      \vdots    & \vdots    & \ddots & \vdots    \\
                      a_{1}     & a_{2}     & \cdots & x + a_{n}
                  \end{vmatrix}
                   & = \det(x\varepsilon_{1} + a_{1}\varepsilon, x\varepsilon_{2} + a_{2}\varepsilon, \ldots, x\varepsilon_{n} + a_{n}\varepsilon)             \\
                   & = \det(x\varepsilon_{1}, x\varepsilon_{2}, \ldots, x\varepsilon_{n})                                                                      \\
                   & + \det(a_{1}\varepsilon, x\varepsilon_{2}, \ldots, x\varepsilon_{n}) + \det(x\varepsilon_{1}, a_{2}\varepsilon, \ldots, x\varepsilon_{n}) \\
                   & + \cdots + \det(x\varepsilon_{1}, x\varepsilon_{2}, \ldots, a_{n}\varepsilon)                                                             \\
                   & = x^{n} + x^{n-1}\sum^{n}_{i=1}a_{i}.
              \end{align*}
        \item
              \begin{align*}
                  \begin{vmatrix}
                      x_{1}  & a_{2}  & \cdots & a_{n}  \\
                      a_{1}  & x_{2}  & \cdots & a_{n}  \\
                      \vdots & \vdots & \ddots & \vdots \\
                      a_{1}  & a_{2}  & \cdots & x_{n}
                  \end{vmatrix}
                   & =
                  \begin{vmatrix}
                      (x_{1} - a_{1}) + a_{1} & a_{2}                   & \cdots & a_{n}                   \\
                      a_{1}                   & (x_{2} - a_{2}) + a_{2} & \cdots & a_{n}                   \\
                      \vdots                  & \vdots                  & \ddots & \vdots                  \\
                      a_{1}                   & a_{2}                   & \cdots & (x_{n} - a_{n}) + a_{n}
                  \end{vmatrix}                                                                                       \\
                   & = \det((x_{1} - a_{1})\varepsilon_{1} + a_{1}\varepsilon, (x_{2} - a_{2})\varepsilon_{2} + a_{2}\varepsilon, \ldots, (x_{n} - a_{n})\varepsilon_{n} + a_{n}\varepsilon) \\
                   & = (x_{1} - a_{1})\cdots (x_{n} - a_{n})\det(\varepsilon_{1}, \ldots, \varepsilon_{n})                                                                                   \\
                   & + \sum^{n}_{i = 1}a_{i}\left(\prod^{n}_{\stackrel{j=1}{j\ne i}}(x_{j} - a_{j})\right)\det(\ldots, \varepsilon_{i-1}, \varepsilon, \varepsilon_{i+1}, \ldots)            \\
                   & = \prod^{n}_{i=1}(x_{i} - a_{i}) + \sum^{n}_{i=1}a_{i}\left(\prod^{n}_{\stackrel{j=1}{j\ne i}}(x_{j} - a_{j})\right).
              \end{align*}
    \end{enumerate}
\end{proof}

\par Kí hiệu định thức Vandermonde cỡ $n$ với $n$ biến:

\[
    D_{n} = D_{n}(x_{1}, \ldots, x_{n}) =
    \begin{vmatrix}
        1      & x_{1}  & x_{1}^{2} & \cdots & x_{1}^{n-1} \\
        1      & x_{2}  & x_{2}^{2} & \cdots & x_{2}^{n-1} \\
        \vdots & \vdots & \vdots    & \ddots & \vdots      \\
        1      & x_{n}  & x_{n}^{2} & \cdots & x_{1}^{n-1}
    \end{vmatrix}.
\]

\par Kí hiệu đa thức đối xứng sơ cấp:

\[
    e_{k}(x_{1}, \ldots, x_{n}) = \sum_{1\le i_{1} < \cdots < i_{k} \le n}\left(\prod^{k}_{j=1}x_{i_{j}}\right).
\]

\par Ví dụ
\begin{gather*}
    e_{1}(x_{1}, x_{2}, x_{3}) = x_{1} + x_{2} + x_{3} \\
    e_{2}(x_{1}, x_{2}, x_{3}, x_{4}) = x_{1}x_{2} + x_{1}x_{3} + x_{1}x_{4} + x_{2}x_{3} + x_{3}x_{4} + x_{2}x_{4} \\
    e_{3}(x_{1}, x_{2}, x_{3}) = x_{1}x_{2}x_{3}.
\end{gather*}

\par Kí hiệu đa thức đối xứng thuần nhất đầy đủ:
\[
    h_{k}(x_{1}, \ldots, x_{n}) = \sum_{i_{1}+\cdots+i_{n}=k}\prod^{n}_{j=1}x^{i_{j}}_{j}
\]

\par Ví dụ
\begin{gather*}
    h_{1}(x_{1}, x_{2}, x_{3}) = x_{1} + x_{2} + x_{3} \\
    h_{2}(x_{1}, x_{2}, x_{3}) = x_{1}^{2} + x_{2}^{2} + x_{3}^{2} + x_{1}x_{2} + x_{2}x_{3} + x_{1}x_{3} \\
    h_{3}(x_{1}, x_{2}, x_{3}) = x_{1}^{3} + x_{2}^{3} + x_{3}^{3} + x_{1}^{2}x_{2} + x_{1}^{2}x_{3} + x_{2}^{2}x_{1} + x_{2}^{2}x_{3} + x_{3}^{2}x_{1} + x_{3}^{2}x_{2} + x_{1}x_{2}x_{3}.
\end{gather*}

\par Tính các định thức sau đây:

% exercise 3.15
\begin{exercise}
    $\begin{vmatrix}
            a_{1}  & x_{1}  & x_{1}^{2} & \cdots & x_{1}^{n-1} \\
            a_{2}  & x_{2}  & x_{2}^{2} & \cdots & x_{2}^{n-1} \\
            \vdots & \vdots & \vdots    & \ddots & \vdots      \\
            a_{n}  & x_{n}  & x_{n}^{2} & \cdots & x_{n}^{n-1}
        \end{vmatrix}$.
\end{exercise}

\begin{proof}[Lời giải]
    \par Khai triển Laplace theo cột thứ nhất:
    \begin{align*}
        \begin{vmatrix}
            a_{1}  & x_{1}  & x_{1}^{2} & \cdots & x_{1}^{n-1} \\
            a_{2}  & x_{2}  & x_{2}^{2} & \cdots & x_{2}^{n-1} \\
            \vdots & \vdots & \vdots    & \ddots & \vdots      \\
            a_{n}  & x_{n}  & x_{n}^{2} & \cdots & x_{n}^{n-1}
        \end{vmatrix}
         & =
        \sum^{n}_{i=1}(-1){}^{1+i}a_{i}
        \begin{vmatrix}
            \vdots  & \vdots      & \vdots      & \ddots & \vdots        \\
            x_{i-1} & x_{i-1}^{2} & x_{i-1}^{3} & \cdots & x_{i-1}^{n-1} \\
            x_{i+1} & x_{i+1}^{2} & x_{i+1}^{3} & \cdots & x_{i+1}^{n-1} \\
            \vdots  & \vdots      & \vdots      & \ddots & \vdots
        \end{vmatrix} \\
         & =
        \sum^{n}_{i=1}(-1){}^{1+i}a_{i}\prod^{n}_{j\ne i}
        \begin{vmatrix}
            \vdots & \vdots  & \vdots      & \ddots & \vdots        \\
            1      & x_{i-1} & x_{i-1}^{2} & \cdots & x_{i-1}^{n-2} \\
            1      & x_{i+1} & x_{i+1}^{2} & \cdots & x_{i+1}^{n-2} \\
            \vdots & \vdots  & \vdots      & \ddots & \vdots
        \end{vmatrix}      \\
         & =
        \sum^{n}_{i=1}(-1){}^{1+i}a_{i}\left(\prod^{n}_{j\ne i}x_{j}\right)
        D_{n-1}(\ldots, x_{i-1}, x_{i+1}, \ldots).
    \end{align*}
\end{proof}

% exercise 3.16
\begin{exercise}\label{chapter3:vandermonde-and-symmetric-polynomials}
    \begin{enumerate}[label = (\alph*)]
        \item $D^{(1)}_{n} = \begin{vmatrix}
                      1      & x_{1}^{2} & x_{1}^{3} & \cdots & x_{1}^{n} \\
                      1      & x_{2}^{2} & x_{2}^{3} & \cdots & x_{2}^{n} \\
                      \vdots & \vdots    & \vdots    & \ddots & \vdots    \\
                      1      & x_{n}^{2} & x_{n}^{3} & \cdots & x_{n}^{n}
                  \end{vmatrix}$,
        \item $D^{(s)}_{n} = \begin{vmatrix}
                      1      & x_{1}  & x_{1}^{2} & \cdots & x_{1}^{s-1} & x_{1}^{s+1} & \cdots & x_{1}^{n} \\
                      1      & x_{2}  & x_{2}^{2} & \cdots & x_{2}^{s-1} & x_{2}^{s+1} & \cdots & x_{2}^{n} \\
                      \vdots & \vdots & \vdots    & \ddots & \vdots      & \vdots      & \ddots & \vdots    \\
                      1      & x_{n}  & x_{n}^{2} & \cdots & x_{n}^{s-1} & x_{n}^{s+1} & \cdots & x_{n}^{n}
                  \end{vmatrix}$.
    \end{enumerate}
\end{exercise}

\begin{proof}[Lời giải]
    \begin{enumerate}[label = (\alph*)]
        \item
              \begin{align*}
                  \begin{vmatrix}
                      1      & x_{1}^{2} & x_{1}^{3} & \cdots & x_{1}^{n} \\
                      1      & x_{2}^{2} & x_{2}^{3} & \cdots & x_{2}^{n} \\
                      \vdots & \vdots    & \vdots    & \ddots & \vdots    \\
                      1      & x_{n}^{2} & x_{n}^{3} & \cdots & x_{n}^{n}
                  \end{vmatrix}
                   & =
                  \begin{vmatrix}
                      1      & x_{1}^{2} & x_{1}^{3} & \cdots & x_{1}^{n-1}(x_{1} - x_{n}) \\
                      1      & x_{2}^{2} & x_{2}^{3} & \cdots & x_{2}^{n-1}(x_{2} - x_{n}) \\
                      \vdots & \vdots    & \vdots    & \ddots & \vdots                     \\
                      1      & x_{n}^{2} & x_{n}^{3} & \cdots & 0
                  \end{vmatrix}\quad(c_{n}:= c_{n} - x_{n}c_{n-1})                                     \\
                   & = \vdots                                                                                              \\
                   & =
                  \begin{vmatrix}
                      1      & x_{1}^{2} & x_{1}^{2}(x_{1} - x_{n}) & \cdots & x_{1}^{n-1}(x_{1} - x_{n}) \\
                      1      & x_{2}^{2} & x_{2}^{2}(x_{2} - x_{n}) & \cdots & x_{2}^{n-1}(x_{2} - x_{n}) \\
                      \vdots & \vdots    & \vdots                   & \ddots & \vdots                     \\
                      1      & x_{n}^{2} & 0                        & \cdots & 0
                  \end{vmatrix}\quad(c_{3}:=c_{3} - x_{n}c_{2})                      \\
                   & =
                  \begin{vmatrix}
                      1      & (x_{1} + x_{n})(x_{1} - x_{n}) & x_{1}^{2}(x_{1} - x_{n}) & \cdots & x_{1}^{n-1}(x_{1} - x_{n}) \\
                      1      & (x_{2} + x_{n})(x_{2} - x_{n}) & x_{2}^{2}(x_{2} - x_{n}) & \cdots & x_{2}^{n-1}(x_{2} - x_{n}) \\
                      \vdots & \vdots                         & \vdots                   & \ddots & \vdots                     \\
                      1      & 0                              & 0                        & \cdots & 0
                  \end{vmatrix}\quad(c_{2}:=c_{2} - x_{n}^{2}c_{1}) \\
              \end{align*}
              \par Khai triển Laplace theo cột thứ nhất
              \begin{align*}
                   & = (-1){}^{n+1}(x_{1} - x_{n})(x_{2} - x_{n})\cdots (x_{n-1} - x_{n})
                  \begin{vmatrix}
                      x_{1} + x_{n}   & x_{1}^{2}   & \cdots & x_{1}^{n-1}   \\
                      x_{2} + x_{n}   & x_{2}^{2}   & \cdots & x_{2}^{n-1}   \\
                      \vdots          & \vdots      & \ddots & \vdots        \\
                      x_{n-1} + x_{n} & x_{n-1}^{2} & \cdots & x_{n-1}^{n-1}
                  \end{vmatrix}                  \\
                   & = \prod^{n-1}_{i=1}(x_{n} - x_{i})\left(
                  \begin{vmatrix}
                          x_{1}   & x_{1}^{2}   & \cdots & x_{1}^{n-1}   \\
                          x_{2}   & x_{2}^{2}   & \cdots & x_{2}^{n-1}   \\
                          \vdots  & \vdots      & \ddots & \vdots        \\
                          x_{n-1} & x_{n-1}^{2} & \cdots & x_{n-1}^{n-1}
                      \end{vmatrix}
                  +
                  \begin{vmatrix}
                          x_{n}  & x_{1}^{2}   & \cdots & x_{1}^{n-1}   \\
                          x_{n}  & x_{2}^{2}   & \cdots & x_{2}^{n-1}   \\
                          \vdots & \vdots      & \ddots & \vdots        \\
                          x_{n}  & x_{n-1}^{2} & \cdots & x_{n-1}^{n-1}
                      \end{vmatrix}
                  \right)
              \end{align*}
              \begingroup{}
              \allowdisplaybreaks{}
              \begin{align*}
                  D^{(1)}_{n} & = (x_{n} - x_{1})\cdots(x_{n} - x_{n-1})\left( x_{1}x_{2}\cdots x_{n-1} D_{n-1} + x_{n}D^{(1)}_{n-1}\right)                                                                                        \\
                              & = x_{n}(x_{n} - x_{1})\cdots(x_{n} - x_{n-1})D^{(1)}_{n-1} + (x_{n} - x_{1})\cdots (x_{n} - x_{n-1}) x_{1}x_{2}\cdots x_{n-1}D_{n-1}                                                               \\
                              & = x_{n}\prod^{n-1}_{i=1}(x_{n} - x_{i}) \cdot D^{(1)}_{n-1} + \prod^{n-1}_{i=1}x_{i} \cdot D_{n}                                                                                                   \\
                              & = x_{n}\prod^{n-1}_{i=1}(x_{n} - x_{i}) \cdot \left(x_{n-1}\prod^{n-2}_{i=1}(x_{n-1} - x_{i})\cdot D^{(1)}_{n-2} + \prod^{n-2}_{i=1}x_{i}\cdot D_{n-1}\right) + \prod^{n-1}_{i=1}x_{i} \cdot D_{n} \\
                              & = x_{n}x_{n-1}\prod^{n-1}_{i=1}(x_{n} - x_{i})\prod^{n-2}_{i=1}(x_{n-1} - x_{i})\cdot D^{(1)}_{n-2} + \prod^{n}_{1\le i\ne n-1}x_{i}\cdot D_{n} + \prod^{n}_{1\le i\ne n}x_{i}\cdot D_{n}          \\
                              & = \cdots                                                                                                                                                                                           \\
                              & = D_{n}x_{1}x_{2}\cdots x_{n}\sum^{n}_{i=1}\frac{1}{x_{i}}                                                                                                                                         \\
                              & = D_{n}\sum^{n}_{i=1}\left(\prod^{n}_{1\le j\ne i}x_{j}\right)                                                                                                                                     \\
                              & = D_{n}e_{n-1}(x_{1},\ldots, x_{n}).
              \end{align*}
              \endgroup{}
        \item
              \par Ta xét trường hợp đặc biệt
              \begin{align*}
                   & D^{(n-1)}_{n} =
                  \begin{vmatrix}
                      1      & x_{1}  & \cdots & x_{1}^{n-2} & x_{1}^{n} \\
                      1      & x_{2}  & \cdots & x_{2}^{n-2} & x_{2}^{n} \\
                      \vdots & \vdots & \ddots & \cdots      & \vdots    \\
                      1      & x_{n}  & \cdots & x_{n}^{n-2} & x_{n}^{n}
                  \end{vmatrix}                                                                                                                                                                                 \\
                   & =
                  \begin{vmatrix}
                      1      & x_{1} - x_{n} & \cdots & x_{1}^{n-3}(x_{1} - x_{n}) & x_{1}^{n-2}(x_{1} + x_{n})(x_{1} - x_{n}) \\
                      1      & x_{2} - x_{n} & \cdots & x_{2}^{n-3}(x_{2} - x_{n}) & x_{2}^{n-2}(x_{2} + x_{n})(x_{2} - x_{n}) \\
                      \vdots & \vdots        & \ddots & \vdots                     & \vdots                                    \\
                      1      & 0             & \cdots & 0                          & 0
                  \end{vmatrix}                                                                                                                           \\
                   & = \prod^{n}_{1\le i\ne n}(x_{n} - x_{i})
                  \begin{vmatrix}
                      1      & x_{1}   & \cdots & x_{1}^{n-3}   & x_{1}^{n-1} + x_{1}^{n-2}x_{n}     \\
                      1      & x_{2}   & \cdots & x_{2}^{n-3}   & x_{2}^{n-1} + x_{2}^{n-2}x_{n}     \\
                      \vdots & \vdots  & \ddots & \cdots        & \vdots                             \\
                      1      & x_{n-1} & \cdots & x_{n-1}^{n-3} & x_{n-1}^{n-1} + x_{n-1}^{n-2}x_{n}
                  \end{vmatrix}                                                                                                                                                     \\
                   & = \prod^{n}_{1\le i\ne n}(x_{n} - x_{i})\cdot D^{(n-2)}_{n-1} + x_{n}\prod^{n}_{1\le i\ne n}(x_{n} - x_{i})\cdot D_{n-1}                                                                                                        \\
                   & = \prod^{n}_{1\le i\ne n}(x_{n} - x_{i})\cdot D^{(n-2)}_{n-1} + x_{n}D_{n}                                                                                                                                                      \\
                   & = \prod^{n}_{1\le i\ne n}(x_{n} - x_{i})\prod^{n-1}_{1\le i\ne n-1}(x_{n-1} - x_{i})\cdot D^{(n-3)}_{n-2} + \prod^{n}_{1\le i\ne n}(x_{n} - x_{i})\prod^{n-1}_{1\le i\ne n-1}(x_{n-1} - x_{i})\cdot x_{n-1}D_{n-2} + x_{n}D_{n} \\
                   & = \prod^{n}_{1\le i\ne n}(x_{n} - x_{i})\prod^{n-1}_{1\le i\ne n-1}(x_{n-1} - x_{i})\cdot D^{(n-3)}_{n-2} + (x_{n-1} + x_{n})D_{n}                                                                                              \\
                   & = \prod^{n}_{1\le i\ne n}(x_{n} - x_{i})\cdots \prod^{3}_{1\le i\ne 3}(x_{3} - x_{i})\cdot D^{(1)}_{2} + (x_{3} + \cdots + x_{n})D_{n}                                                                                          \\
                   & = \prod^{n}_{1\le i\ne n}(x_{n} - x_{i})\cdots \prod^{3}_{1\le i\ne 3}(x_{3} - x_{i})(x_{2} - x_{1})(x_{1} + x_{2}) + (x_{3} + \cdots + x_{n})D_{n}                                                                             \\
                   & = D_{n}\sum^{n}_{i=1}x_{i}                                                                                                                                                                                                      \\
                   & = D_{n}e_{1}(x_{1},\ldots, x_{n}).
              \end{align*}

              \par Ta sẽ chứng minh $D_{n}^{(s)} = D_{n}e_{n-s}(x_{1},\ldots, x_{n})\ \forall n, 0\le s\le n$.
              \par Khẳng định trên đúng với $n = 1, 2, 3$.
              \par Nếu khẳng định này đúng với $n - 1, \forall\ 0\le s\le n-1$, ta cần chứng minh khẳng định vẫn đúng với $n, \forall\ 0\le s\le n$.
              \begin{align*}
                   & \phantom{=}\begin{vmatrix}
                                    1      & x_{1}  & x_{1}^{2} & \cdots & x_{1}^{s-1} & x_{1}^{s+1} & \cdots & x_{1}^{n} \\
                                    1      & x_{2}  & x_{2}^{2} & \cdots & x_{2}^{s-1} & x_{2}^{s+1} & \cdots & x_{2}^{n} \\
                                    \vdots & \vdots & \vdots    & \ddots & \vdots      & \vdots      & \ddots & \vdots    \\
                                    1      & x_{n}  & x_{n}^{2} & \cdots & x_{n}^{s-1} & x_{n}^{s+1} & \cdots & x_{n}^{n}
                                \end{vmatrix}                                          \\
                   & =
                  \begin{vmatrix}
                      1      & x_{1}  & x_{1}^{2} & \cdots & x_{1}^{s-1} & x_{1}^{s+1} & \cdots & x_{1}^{n-1}(x_{1} - x_{n}) \\
                      1      & x_{2}  & x_{2}^{2} & \cdots & x_{2}^{s-1} & x_{2}^{s+1} & \cdots & x_{2}^{n-1}(x_{2} - x_{n}) \\
                      \vdots & \vdots & \vdots    & \ddots & \vdots      & \vdots      & \ddots & \vdots                     \\
                      1      & x_{n}  & x_{n}^{2} & \cdots & x_{n}^{s-1} & x_{n}^{s+1} & \cdots & 0
                  \end{vmatrix}\quad(c_{s}:=c_{s} - x_{n}c_{s-1})                                       \\
                   & = \cdots                                                                                                                                  \\
                   & =
                  \begin{vmatrix}
                      1      & x_{1}  & x_{1}^{2} & \cdots & x_{1}^{s-1} & x_{1}^{s-1}(x_{1}^{2} - x_{n}^{2}) & \cdots & x_{1}^{n-1}(x_{1} - x_{n}) \\
                      1      & x_{2}  & x_{2}^{2} & \cdots & x_{2}^{s-1} & x_{2}^{s-1}(x_{2}^{2} - x_{n}^{2}) & \cdots & x_{2}^{n-1}(x_{2} - x_{n}) \\
                      \vdots & \vdots & \vdots    & \ddots & \vdots      & \vdots                             & \ddots & \vdots                     \\
                      1      & x_{n}  & x_{n}^{2} & \cdots & x_{n}^{s-1} & 0                                  & \cdots & 0
                  \end{vmatrix}\quad(c_{s}:=c_{s} - x_{n}c_{s-1})                \\
                   & =
                  \begin{vmatrix}
                      1      & x_{1}  & x_{1}^{2} & \cdots & x_{1}^{s-2}(x_{1} - x_{n}) & x_{1}^{s-1}(x_{1}^{2} - x_{n}^{2}) & \cdots & x_{1}^{n-1}(x_{1} - x_{n}) \\
                      1      & x_{2}  & x_{2}^{2} & \cdots & x_{2}^{s-2}(x_{2} - x_{n}) & x_{2}^{s-1}(x_{2}^{2} - x_{n}^{2}) & \cdots & x_{2}^{n-1}(x_{2} - x_{n}) \\
                      \vdots & \vdots & \vdots    & \ddots & \vdots                     & \vdots                             & \ddots & \vdots                     \\
                      1      & x_{n}  & x_{n}^{2} & \cdots & 0                          & 0                                  & \cdots & 0
                  \end{vmatrix} \\
                   & =
                  \begin{vmatrix}
                      1      & x_{1} - x_{n} & x_{1}(x_{1} - x_{n}) & \cdots & x_{1}^{s-2}(x_{1} - x_{n}) & x_{1}^{s-1}(x_{1}^{2} - x_{n}^{2}) & \cdots & x_{1}^{n-1}(x_{1} - x_{n}) \\
                      1      & x_{2} - x_{n} & x_{2}(x_{2} - x_{n}) & \cdots & x_{2}^{s-2}(x_{2} - x_{n}) & x_{2}^{s-1}(x_{2}^{2} - x_{n}^{2}) & \cdots & x_{2}^{n-1}(x_{2} - x_{n}) \\
                      \vdots & \vdots        & \vdots               & \ddots & \vdots                     & \vdots                             & \ddots & \vdots                     \\
                      1      & 0             & 0                    & \cdots & 0                          & 0                                  & \cdots & 0
                  \end{vmatrix}
              \end{align*}
              \par Khai triển Laplace theo dòng thứ $n$:
              \begin{align*}
                   & = {(-1)}^{n+1}(x_{1} - x_{n})(x_{2} - x_{n})\cdots (x_{n-1} - x_{n})
                  \begin{vmatrix}
                      1      & x_{1}   & \cdots & x_{1}^{s-2}   & x_{1}^{s-1}(x_{1} + x_{n})     & \cdots & x_{1}^{n-1}   \\
                      1      & x_{2}   & \cdots & x_{2}^{s-2}   & x_{2}^{s-1}(x_{2} + x_{n})     & \cdots & x_{2}^{n-1}   \\
                      \vdots & \vdots  & \ddots & \vdots        & \vdots                         & \ddots & \vdots        \\
                      1      & x_{n-1} & \cdots & x_{n-1}^{s-2} & x_{n-1}^{s-1}(x_{n-1} + x_{n}) & \cdots & x_{n-1}^{n-1}
                  \end{vmatrix}                                                    \\
                   & = (x_{n} - x_{1})\cdots (x_{n} - x_{n-1})
                  \begin{vmatrix}
                      1      & x_{1}   & \cdots & x_{1}^{s-2}   & x_{1}^{s}   & \cdots & x_{1}^{n-1} \\
                      1      & x_{2}   & \cdots & x_{2}^{s-2}   & x_{2}^{s}   & \cdots & x_{2}^{n-1} \\
                      \vdots & \vdots  & \ddots & \vdots        & \vdots      & \ddots & \vdots      \\
                      1      & x_{n-1} & \cdots & x_{n-1}^{s-2} & x_{n-1}^{s} & \cdots & x_{s}^{n-1}
                  \end{vmatrix}                                                                         \\
                   & + (x_{n} - x_{1})\cdots (x_{n} - x_{n-1})
                  x_{n}\begin{vmatrix}
                           1      & x_{1}   & \cdots & x_{1}^{s-1}   & x_{1}^{s+1}   & \cdots & x_{1}^{n-1} \\
                           1      & x_{2}   & \cdots & x_{2}^{s-1}   & x_{2}^{s+1}   & \cdots & x_{2}^{n-1} \\
                           \vdots & \vdots  & \ddots & \vdots        & \vdots        & \ddots & \vdots      \\
                           1      & x_{n-1} & \cdots & x_{n-1}^{s-1} & x_{n-1}^{s+1} & \cdots & x_{s}^{n-1}
                       \end{vmatrix}                                                                  \\
                   & = \prod^{n}_{1\le i\ne n}(x_{n} - x_{i})\cdot \left(D^{(s-1)}_{n-1} + x_{n}D^{(s)}_{n-1}\right)                                                     \\
                   & = \prod^{n}_{1\le i\ne n}(x_{n} - x_{i})\cdot \left( D_{n-1}e_{n-s}(x_{1}, \ldots, x_{n-1}) + x_{n}D_{n-1}e_{n-1-s}(x_{1}, \ldots, x_{n-1}) \right) \\
                   & = \prod^{n}_{1\le i\ne n}(x_{n} - x_{i})\cdot D_{n-1} \left( e_{n-s}(x_{1}, \ldots, x_{n-1}) + x_{n}e_{n-1-s}(x_{1}, \ldots, x_{n-1}) \right)       \\
                   & = D_{n} e_{n-s}(x_{1}, \ldots, x_{n-1}, x_{n}).
              \end{align*}
              \par Phép chứng minh quy nạp hoàn tất.
              \par Vậy $D^{(s)}_{n} = D_{n}e_{n-s}(x_{1}, \ldots, x_{n})$.
    \end{enumerate}
\end{proof}

\par Ta kí hiệu các vector cột gồm $n$ thành phần:
\[
    \alpha_{0} = \begin{pmatrix}
        1 \\ 1 \\ \vdots \\ 1
    \end{pmatrix}\qquad
    \alpha_{k} = \begin{pmatrix}
        x_{1}^{k} \\ x_{2}^{k} \\ \vdots \\ x_{n}^{k}
    \end{pmatrix}
\]

% exercise 3.17
\begin{exercise}
    $\begin{vmatrix}
            1      & x_{1}(x_{1} - 1) & x_{1}^{2}(x_{1} - 1) & \cdots & x_{1}^{n-1}(x_{1} - 1) \\
            1      & x_{2}(x_{2} - 1) & x_{2}^{2}(x_{2} - 1) & \cdots & x_{2}^{n-1}(x_{2} - 1) \\
            \vdots & \vdots           & \vdots               & \ddots & \vdots                 \\
            1      & x_{n}(x_{n} - 1) & x_{n}^{2}(x_{n} - 1) & \cdots & x_{n}^{n-1}(x_{n} - 1)
        \end{vmatrix}$.
\end{exercise}

\begin{proof}[Lời giải]
    \par Ta lần lượt thực hiện các biến đổi sau:
    \[
        \begin{cases}
            c_{n}   & := c_{n} + c_{n-1} + \cdots + c_{2}   \\
            c_{n-1} & := c_{n-1} + c_{n-2} + \cdots + c_{2} \\
                    & \vdots                                \\
            c_{3}   & := c_{3} + c_{2}
        \end{cases}
    \]
    \begingroup{}
    \allowdisplaybreaks{}
    \begin{align*}
          & \begin{vmatrix}
                1      & x_{1}(x_{1} - 1) & x_{1}^{2}(x_{1} - 1) & \cdots & x_{1}^{n-1}(x_{1} - 1) \\
                1      & x_{2}(x_{2} - 1) & x_{2}^{2}(x_{2} - 1) & \cdots & x_{2}^{n-1}(x_{2} - 1) \\
                \vdots & \vdots           & \vdots               & \ddots & \vdots                 \\
                1      & x_{n}(x_{n} - 1) & x_{n}^{2}(x_{n} - 1) & \cdots & x_{n}^{n-1}(x_{n} - 1)
            \end{vmatrix}                                                                                                                                               \\
        = &
        \begin{vmatrix}
            1      & x_{1}^{2} - x_{1} & x_{1}^{3} - x_{1}^{2} & \cdots & x_{1}^{n} - x_{1}^{n-1} \\
            1      & x_{2}^{2} - x_{2} & x_{2}^{3} - x_{2}^{2} & \cdots & x_{2}^{n} - x_{2}^{n-1} \\
            \vdots & \vdots            & \vdots                & \ddots & \vdots                  \\
            1      & x_{n}^{2} - x_{n} & x_{n}^{3} - x_{n}^{2} & \cdots & x_{n}^{n} - x_{n}^{n-1}
        \end{vmatrix}
        =
        \begin{vmatrix}
            1      & x_{1}^{2} - x_{1} & x_{1}^{3} - x_{1} & \cdots & x_{1}^{n} - x_{1} \\
            1      & x_{2}^{2} - x_{2} & x_{2}^{3} - x_{2} & \cdots & x_{2}^{n} - x_{2} \\
            \vdots & \vdots            & \vdots            & \ddots & \vdots            \\
            1      & x_{n}^{2} - x_{n} & x_{n}^{3} - x_{n} & \cdots & x_{n}^{n} - x_{n}
        \end{vmatrix}                                                                                                                                                          \\
        = & \det(\alpha_{0}, \alpha_{2} - \alpha_{1}, \alpha_{3} - \alpha_{1}, \ldots, \alpha_{n} - \alpha_{1})                                                                                                                              \\
        = & \det(\alpha_{0}, \alpha_{2}, \alpha_{3}, \ldots, \alpha_{n}) + \left(\det(\alpha_{0}, -\alpha_{1}, \alpha_{3}, \ldots, \alpha_{n}) + \cdots + \det(\alpha_{0}, \alpha_{2}, \alpha_{3}, \ldots, \alpha_{n-1}, -\alpha_{1})\right) \\
        = & D_{n}e_{n-1}(x_{1}, \ldots, x_{n}) + \sum^{n}_{s=2}{(-1)}^{s+1}D^{(s)}_{n}                                                                                                                                                       \\
        = & D_{n}e_{n-1}(x_{1}, \ldots, x_{n}) + \sum^{n}_{s=2}{(-1)}^{s+1}D_{n}e_{n-s}(x_{1}, \ldots, x_{n})                                                                                                                                \\
        = & D_{n}\sum^{n}_{s=1}{(-1)}^{s+1}e_{n-s}(x_{1}, \ldots, x_{n})                                                                                                                                                                     \\
        = & D_{n}\sum^{n}_{s=0}{(-1)}^{s+1}e_{n-s}(x_{1}, \ldots, x_{n}) + D_{n}\prod^{n}_{i=1}x_{i}                                                                                                                                         \\
        = & D_{n}\prod^{n}_{i=1}x_{i} - D_{n}\prod^{n}_{i=1}(x_{i} - 1).
    \end{align*}
    \endgroup{}
\end{proof}

% exercise 3.18
\begin{exercise}
    $\begin{vmatrix}
            1 + x_{1} & 1 + x_{1}^{2} & \cdots & 1 + x_{1}^{n} \\
            1 + x_{2} & 1 + x_{2}^{2} & \cdots & 1 + x_{2}^{n} \\
            \vdots    & \vdots        & \ddots & \vdots        \\
            1 + x_{n} & 1 + x_{n}^{2} & \cdots & 1 + x_{n}^{n}
        \end{vmatrix}$.
\end{exercise}

\begin{proof}[Lời giải]
    \begingroup{}
    \allowdisplaybreaks{}
    \begin{align*}
        \begin{vmatrix}
            1 + x_{1} & 1 + x_{1}^{2} & \cdots & 1 + x_{1}^{n} \\
            1 + x_{2} & 1 + x_{2}^{2} & \cdots & 1 + x_{2}^{n} \\
            \vdots    & \vdots        & \ddots & \vdots        \\
            1 + x_{n} & 1 + x_{n}^{2} & \cdots & 1 + x_{n}^{n}
        \end{vmatrix}
         & =
        \begin{vmatrix}
            1 + x_{1} & x_{1}^{2} - x_{1} & \cdots & x_{1}^{n} - x_{1} \\
            1 + x_{2} & x_{2}^{2} - x_{2} & \cdots & x_{2}^{n} - x_{2} \\
            \vdots    & \vdots            & \ddots & \vdots            \\
            1 + x_{n} & x_{n}^{2} - x_{n} & \cdots & x_{n}^{n} - x_{n}
        \end{vmatrix}                                   \\
         & =
        \begin{vmatrix}
            1      & x_{1}^{2} - x_{1} & \cdots & x_{1}^{n} - x_{1} \\
            1      & x_{2}^{2} - x_{2} & \cdots & x_{2}^{n} - x_{2} \\
            \vdots & \vdots            & \ddots & \vdots            \\
            1      & x_{n}^{2} - x_{n} & \cdots & x_{n}^{n} - x_{n}
        \end{vmatrix}
        +
        \begin{vmatrix}
            x_{1}  & x_{1}^{2} - x_{1} & \cdots & x_{1}^{n} - x_{1} \\
            x_{2}  & x_{2}^{2} - x_{2} & \cdots & x_{2}^{n} - x_{2} \\
            \vdots & \vdots            & \ddots & \vdots            \\
            x_{n}  & x_{n}^{2} - x_{n} & \cdots & x_{n}^{n} - x_{n}
        \end{vmatrix}                                      \\
         & =
        \begin{vmatrix}
            1      & x_{1}^{2} - x_{1} & \cdots & x_{1}^{n} - x_{1} \\
            1      & x_{2}^{2} - x_{2} & \cdots & x_{2}^{n} - x_{2} \\
            \vdots & \vdots            & \ddots & \vdots            \\
            1      & x_{n}^{2} - x_{n} & \cdots & x_{n}^{n} - x_{n}
        \end{vmatrix}
        +
        \begin{vmatrix}
            x_{1}  & x_{1}^{2} & \cdots & x_{1}^{n} \\
            x_{2}  & x_{2}^{2} & \cdots & x_{2}^{n} \\
            \vdots & \vdots    & \ddots & \vdots    \\
            x_{n}  & x_{n}^{2} & \cdots & x_{n}^{n}
        \end{vmatrix}                                                      \\
         & = D_{n}\prod^{n}_{i=1}x_{i} - D_{n}\prod^{n}_{i=1}(x_{i} - 1) + D_{n}\prod^{n}_{i=1}x_{i} \\
         & = 2D_{n}\prod^{n}_{i=1}x_{i} - D_{n}\prod^{n}_{i=1}(x_{i} - 1).
    \end{align*}
    \endgroup{}
\end{proof}

% exercise 3.19
\begin{exercise}
    $\begin{vmatrix}
            1      & \cos(\varphi_{1}) & \cos(2\varphi_{1}) & \cdots & \cos((n-1)\varphi_{1}) \\
            1      & \cos(\varphi_{2}) & \cos(2\varphi_{2}) & \cdots & \cos((n-1)\varphi_{2}) \\
            \vdots & \vdots            & \vdots             & \ddots & \vdots                 \\
            1      & \cos(\varphi_{n}) & \cos(2\varphi_{n}) & \cdots & \cos((n-1)\varphi_{n})
        \end{vmatrix}$.
\end{exercise}

\begin{proof}[Lời giải]
    \par Áp dụng các công thức cộng, trừ cung:
    \begin{align*}
        \cos(n\varphi) & = \cos(\varphi + (n-1)\varphi) = \cos(\varphi)\cos((n-1)\varphi) - \sin(\varphi)\sin((n-1)\varphi) \\
                       & = 2\cos(\varphi)\cos((n-1)\varphi) - \cos((n-2)\varphi).
    \end{align*}
    \par Dựa vào liên hệ này và nhận xét rằng $\cos(0\varphi) = 1$, $\cos(\varphi) = \cos(\varphi)$, ta có thể chứng minh quy nạp được cho khẳng định: $\cos(n\varphi)$ là một đa thực bậc $n$ với biến $\cos(\varphi)$, hệ số cao nhất là $2^{n-1}$.
    \begingroup{}
    \allowdisplaybreaks{}
    \begin{align*}
        \begin{vmatrix}
            1      & \cos(\varphi_{1}) & \cos(2\varphi_{1}) & \cdots & \cos((n-1)\varphi_{1}) \\
            1      & \cos(\varphi_{2}) & \cos(2\varphi_{2}) & \cdots & \cos((n-1)\varphi_{2}) \\
            \vdots & \vdots            & \vdots             & \ddots & \vdots                 \\
            1      & \cos(\varphi_{n}) & \cos(2\varphi_{n}) & \cdots & \cos((n-1)\varphi_{n})
        \end{vmatrix}
         & =
        \begin{vmatrix}
            1      & \cos(\varphi_{1}) & 2{\cos(\varphi_{1})}^{2} & \cdots & \cos((n-1)\varphi_{1}) \\
            1      & \cos(\varphi_{2}) & 2{\cos(\varphi_{2})}^{2} & \cdots & \cos((n-1)\varphi_{2}) \\
            \vdots & \vdots            & \vdots                   & \ddots & \vdots                 \\
            1      & \cos(\varphi_{n}) & 2{\cos(\varphi_{n})}^{2} & \cdots & \cos((n-1)\varphi_{n})
        \end{vmatrix}                     \\
         & =
        \begin{vmatrix}
            1      & \cos(\varphi_{1}) & 2{\cos(\varphi_{1})}^{2} & \cdots & 2^{n-2}\cos((n-1)\varphi_{1}) \\
            1      & \cos(\varphi_{2}) & 2{\cos(\varphi_{2})}^{2} & \cdots & 2^{n-2}\cos((n-1)\varphi_{2}) \\
            \vdots & \vdots            & \vdots                   & \ddots & \vdots                        \\
            1      & \cos(\varphi_{n}) & 2{\cos(\varphi_{n})}^{2} & \cdots & 2^{n-2}\cos((n-1)\varphi_{n})
        \end{vmatrix}              \\
         & = \frac{1}{2^{n-1}}
        \begin{vmatrix}
            1      & 2\cos(\varphi_{1}) & 2^{2}{\cos(\varphi_{1})}^{2} & \cdots & 2^{n-1}{\cos(\varphi_{1})}^{n-1} \\
            1      & 2\cos(\varphi_{2}) & 2^{2}{\cos(\varphi_{2})}^{2} & \cdots & 2^{n-1}{\cos(\varphi_{2})}^{n-2} \\
            \vdots & \vdots             & \vdots                       & \ddots & \vdots                           \\
            1      & 2\cos(\varphi_{n}) & 2^{2}{\cos(\varphi_{n})}^{2} & \cdots & 2^{n-1}{\cos(\varphi_{n})}^{n-1}
        \end{vmatrix} \\
         & = \frac{1}{2^{n-1}}\prod_{i>j}2(\cos(\varphi_{i}) - \cos(\varphi_{j}))                                        \\
         & = 2^{(n-1)(n-2)/2}\prod_{i>j}(\cos(\varphi_{i}) - \cos(\varphi_{j})).                                         \\
    \end{align*}
    \endgroup{}
\end{proof}

% exercise 3.20
\begin{exercise}
    $\begin{vmatrix}
            x_{1}y_{1}     & 1 + x_{1}y_{2} & \cdots & 1 + x_{1}y_{n} \\
            1 + x_{2}y_{1} & x_{2}y_{2}     & \cdots & 1 + x_{2}y_{n} \\
            \vdots         & \vdots         & \ddots & \vdots         \\
            1 + x_{n}y_{1} & 1 + x_{n}y_{2} & \cdots & x_{n}y_{n}
        \end{vmatrix}$.
\end{exercise}

\begin{lemma}
    \begin{enumerate}[label = (\roman*)]
        \item
              \[
                  \underbrace{\begin{vmatrix}
                          0      & 1      & 1      & \cdots & 1      \\
                          1      & 0      & 1      & \cdots & 1      \\
                          1      & 1      & 0      & \cdots & 1      \\
                          \vdots & \vdots & \vdots & \ddots & \vdots \\
                          1      & 1      & 1      & \cdots & 0
                      \end{vmatrix}}_{n\times n}
                  = (-1){}^{n-1}(n-1).
              \]
        \item Thay cột thứ $k$ của định thức trên bởi cột $\begin{pmatrix}x_{1} & x_{2} & \cdots & x_{n} \end{pmatrix}^{T}$ thì định thức mới bằng:
              \[
                  {(-1)}^{n-1}\left((1-n)x_{k} + \sum^{n}_{i=1}x_{i}\right)
              \]
    \end{enumerate}
\end{lemma}

\begin{proof}[Chứng minh bổ đề]
    \begin{enumerate}[label = (\roman*)]
        \item
              \begingroup{}
              \allowdisplaybreaks{}
              \begin{align*}
                  \begin{vmatrix}
                      0      & 1      & 1      & \cdots & 1      \\
                      1      & 0      & 1      & \cdots & 1      \\
                      1      & 1      & 0      & \cdots & 1      \\
                      \vdots & \vdots & \vdots & \ddots & \vdots \\
                      1      & 1      & 1      & \cdots & 0
                  \end{vmatrix}
                   & =
                  \begin{vmatrix}
                      0      & 1      & 1      & \cdots & 1      \\
                      1      & -1     & 0      & \cdots & 0      \\
                      1      & 0      & -1     & \cdots & 0      \\
                      \vdots & \vdots & \vdots & \ddots & \vdots \\
                      1      & 0      & 0      & \cdots & -1
                  \end{vmatrix}\quad(r_{i}:= r_{i} - r_{1}) \\
                   & =
                  \begin{vmatrix}
                      n-1    & 1      & 1      & \cdots & 1      \\
                      0      & -1     & 0      & \cdots & 0      \\
                      0      & 0      & -1     & \cdots & 0      \\
                      \vdots & \vdots & \vdots & \ddots & \vdots \\
                      0      & 0      & 0      & \cdots & -1
                  \end{vmatrix}
                  = {(-1)}^{n-1}(n-1).
              \end{align*}
              \endgroup{}
        \item
              \par Trường hợp $k = 1$:
              \begingroup{}
              \allowdisplaybreaks{}
              \begin{align*}
                  \begin{vmatrix}
                      x_{1}  & 1      & 1      & \cdots & 1      \\
                      x_{2}  & 0      & 1      & \cdots & 1      \\
                      x_{3}  & 1      & 0      & \cdots & 1      \\
                      \vdots & \vdots & \vdots & \ddots & \vdots \\
                      x_{n}  & 1      & 1      & \cdots & 0
                  \end{vmatrix}
                   & =
                  \begin{vmatrix}
                      x_{1}         & 1      & 1      & \cdots & 1      \\
                      x_{2} - x_{1} & -1     & 0      & \cdots & 0      \\
                      x_{3} - x_{1} & 0      & -1     & \cdots & 0      \\
                      \vdots        & \vdots & \vdots & \ddots & \vdots \\
                      x_{n} - x_{1} & 0      & 0      & \cdots & -1
                  \end{vmatrix}\quad(r_{i}:= r_{i} - r_{1})                                 \\
                   & =
                  \begin{vmatrix}
                      (1-n)x_{1} + \displaystyle\sum^{n}_{i=1}x_{i} & 0      & 0      & \cdots & 0      \\
                      x_{2} - x_{1}                                 & -1     & 0      & \cdots & 0      \\
                      x_{3} - x_{1}                                 & 0      & -1     & \cdots & 0      \\
                      \vdots                                        & \vdots & \vdots & \ddots & \vdots \\
                      x_{n} - x_{1}                                 & 0      & 0      & \cdots & -1
                  \end{vmatrix}\quad(r_{1}:= r_{1} + \sum^{n}_{i=2}r_{i}) \\
                   & = {(-1)}^{n-1}\left((1-n)x_{1} + \sum^{n}_{i=1}x_{i}\right).
              \end{align*}
              \endgroup{}

              \par Trường hợp $k\ne 1$:
              \begingroup{}
              \allowdisplaybreaks{}
              \begin{align*}
                  \begin{vmatrix}
                      0      & 1      & \cdots & x_{1}  & \cdots & 1      \\
                      1      & 0      & \cdots & x_{2}  & \cdots & 1      \\
                      \vdots & \vdots & \ddots & \vdots & \ddots & 1      \\
                      1      & 1      & \cdots & x_{k}  & \cdots & 1      \\
                      \vdots & \vdots & \ddots & \vdots & \ddots & \vdots \\
                      1      & 1      & \cdots & x_{n}  & \cdots & 0
                  \end{vmatrix}
                   & =
                  \begin{vmatrix}
                      x_{1}  & 1      & \cdots & 0      & \cdots & 1      \\
                      x_{2}  & 0      & \cdots & 1      & \cdots & 1      \\
                      \vdots & \vdots & \ddots & \vdots & \ddots & 1      \\
                      x_{k}  & 1      & \cdots & 1      & \cdots & 1      \\
                      \vdots & \vdots & \ddots & \vdots & \ddots & \vdots \\
                      x_{n}  & 1      & \cdots & 1      & \cdots & 0
                  \end{vmatrix}\quad(c_{1} \leftrightarrow c_{k}) \\
                   & =
                  \begin{vmatrix}
                      x_{k}  & 1      & \cdots & 1      & \cdots & 1      \\
                      x_{2}  & 0      & \cdots & 1      & \cdots & 1      \\
                      \vdots & \vdots & \ddots & \vdots & \ddots & 1      \\
                      x_{1}  & 1      & \cdots & 0      & \cdots & 1      \\
                      \vdots & \vdots & \ddots & \vdots & \ddots & \vdots \\
                      x_{n}  & 1      & \cdots & 1      & \cdots & 0
                  \end{vmatrix}\quad(r_{1} \leftrightarrow r_{k}) \\
                   & =
                  {(-1)}^{n-1}\left((1-n)x_{k} + \sum^{n}_{i=1}x_{i}\right) \quad\text{(áp dụng trường hợp $k=1$)}
              \end{align*}
              \endgroup{}
    \end{enumerate}
\end{proof}

\begin{proof}[Lời giải]
    \begingroup{}
    \allowdisplaybreaks{}
    \begin{align*}
          & \begin{vmatrix}
                x_{1}y_{1}     & 1 + x_{1}y_{2} & \cdots & 1 + x_{1}y_{n} \\
                1 + x_{2}y_{1} & x_{2}y_{2}     & \cdots & 1 + x_{2}y_{n} \\
                \vdots         & \vdots         & \ddots & \vdots         \\
                1 + x_{n}y_{1} & 1 + x_{n}y_{2} & \cdots & x_{n}y_{n}
            \end{vmatrix}                                                                                     \\
        = &
        \det\begin{pmatrix}
                \begin{pmatrix}
                0      \\
                1      \\
                \vdots \\
                1
            \end{pmatrix}
                +
                y_{1}\begin{pmatrix}
                     x_{1}  \\
                     x_{2}  \\
                     \vdots \\
                     x_{n}
                 \end{pmatrix},
                \ldots,
                \begin{pmatrix}
                1      \\
                1      \\
                \vdots \\
                0
            \end{pmatrix}
                +
                y_{n}\begin{pmatrix}
                     x_{1}  \\
                     x_{2}  \\
                     \vdots \\
                     x_{n}
                 \end{pmatrix}
            \end{pmatrix}                                                                                                              \\
        = &
        \begin{vmatrix}
            0      & 1      & \cdots & 1      \\
            1      & 0      & \cdots & 1      \\
            \vdots & \vdots & \ddots & \vdots \\
            1      & 1      & \cdots & 0
        \end{vmatrix}
        +
        \sum^{n}_{k=1}y_{k}
        \begin{vmatrix}
            0      & 1      & \cdots & x_{1}  & \cdots & 1      \\
            1      & 0      & \cdots & x_{2}  & \cdots & 1      \\
            \vdots & \vdots & \ddots & \vdots & \ddots & \vdots \\
            1      & 1      & \cdots & x_{k}  & \cdots & 1      \\
            \vdots & \vdots & \ddots & \vdots & \ddots & \vdots \\
            1      & 1      & \cdots & x_{n}  & \cdots & 0
        \end{vmatrix}                                                                                               \\
        = &
        {(-1)}^{n-1}(n-1) + {(-1)}^{n-1}\sum^{n}_{k=1}y_{k}\left((1-n)x_{k} + \sum^{n}_{i=1}x_{i}\right)                                                  \\
        = & {(-1)}^{n-1}(n-1) + {(-1)}^{n-1}\left(\sum^{n}_{i=1}x_{i}\right)\left(\sum^{n}_{i=1}y_{i}\right) - {(-1)}^{n-1}(n-1)\sum^{n}_{i=1}x_{i}y_{i}.
    \end{align*}
    \endgroup{}
\end{proof}

% exercise 3.21
\begin{exercise}
    $C_{n} = \begin{vmatrix}
            (a_{1} + b_{1}){}^{-1} & (a_{1} + b_{2}){}^{-1} & \cdots & (a_{1} + b_{n}){}^{-1} \\
            (a_{2} + b_{1}){}^{-1} & (a_{2} + b_{2}){}^{-1} & \cdots & (a_{2} + b_{n}){}^{-1} \\
            \vdots                 & \vdots                 & \ddots & \vdots                 \\
            (a_{n} + b_{1}){}^{-1} & (a_{n} + b_{2}){}^{-1} & \cdots & (a_{n} + b_{n}){}^{-1}
        \end{vmatrix}$.
\end{exercise}

\begin{proof}[Lời giải]
    \begingroup{}
    \allowdisplaybreaks{}
    \begin{align*}
          & \begin{vmatrix}
                (a_{1} + b_{1}){}^{-1} & (a_{1} + b_{2}){}^{-1} & \cdots & (a_{1} + b_{n}){}^{-1} \\
                (a_{2} + b_{1}){}^{-1} & (a_{2} + b_{2}){}^{-1} & \cdots & (a_{2} + b_{n}){}^{-1} \\
                \vdots                 & \vdots                 & \ddots & \vdots                 \\
                (a_{n} + b_{1}){}^{-1} & (a_{n} + b_{2}){}^{-1} & \cdots & (a_{n} + b_{n}){}^{-1}
            \end{vmatrix}                                                                 \\
        = &
        \begin{vmatrix}
            (b_{n}-b_{1})(a_{1}+b_{1}){}^{-1}(a_{1}+b_{n}){}^{-1} & (b_{n} - b_{2})(a_{1} + b_{2}){}^{-1}(a_{1} + b_{n}){}^{-1} & \cdots & (a_{1} + b_{n}){}^{-1} \\
            (b_{n}-b_{1})(a_{2}+b_{1}){}^{-1}(a_{2}+b_{n}){}^{-1} & (b_{n} - b_{2})(a_{2} + b_{2}){}^{-1}(a_{2} + b_{n}){}^{-1} & \cdots & (a_{2} + b_{n}){}^{-1} \\
            \vdots                                                & \vdots                                                      & \ddots & \vdots                 \\
            (b_{n}-b_{1})(a_{n}+b_{1}){}^{-1}(a_{n}+b_{n}){}^{-1} & (b_{n} - b_{2})(a_{n} + b_{2}){}^{-1}(a_{n} + b_{n}){}^{-1} & \cdots & (a_{n} + b_{n}){}^{-1} \\
        \end{vmatrix}\quad (c_{i}:= c_{i} - c_{n}) \\
        = &
        \prod^{n}_{1\le i<n}(b_{n} - b_{i})\prod^{n}_{i=1}{(a_{i}+b_{n})}^{-1}
        \begin{vmatrix}
            (a_{1}+b_{1}){}^{-1} & (a_{1} + b_{2}){}^{-1} & \cdots & 1      \\
            (a_{2}+b_{1}){}^{-1} & (a_{2} + b_{2}){}^{-1} & \cdots & 1      \\
            \vdots               & \vdots                 & \ddots & \vdots \\
            (a_{n}+b_{1}){}^{-1} & (a_{n} + b_{2}){}^{-1} & \cdots & 1      \\
        \end{vmatrix}                                                                                       \\
        = &
        \prod^{n}_{1\le i<n}(b_{n} - b_{i})\prod^{n}_{i=1}{(a_{i}+b_{n})}^{-1}
        \begin{vmatrix}
            (a_{n}-a_{1}){(a_{1}+b_{1})}^{-1}{(a_{n}+b_{1})}^{-1} & (a_{n}-a_{1}){(a_{1}+b_{2})}^{-1}{(a_{n}+b_{2})}^{-1} & \cdots & 0      \\
            (a_{n}-a_{2}){(a_{2}+b_{1})}^{-1}{(a_{n}+b_{1})}^{-1} & (a_{n}-a_{2}){(a_{2}+b_{2})}^{-1}{(a_{n}+b_{2})}^{-1} & \cdots & 0      \\
            \vdots                                                & \vdots                                                & \ddots & \vdots \\
            {(a_{n}+b_{1})}^{-1}                                  & {(a_{n}+b_{2})}^{-1}                                  & \cdots & 1
        \end{vmatrix}
    \end{align*}
    \par Áp dụng khai triển Laplace cho cột thứ $n$ và tách nhân tử chung:
    \begin{align*}
        = &
        \prod^{n}_{1\le i<n}(b_{n} - b_{i})\prod^{n}_{i=1}{(a_{i}+b_{n})}^{-1}\prod^{n}_{1\le i<n}(a_{n} - a_{i})\prod^{n}_{i=1}{(a_{n} + b_{i})}^{-1}
        \begin{vmatrix}
            {(a_{1}+b_{1})}^{-1}   & {(a_{1}+b_{2})}^{-1}   & \cdots & {(a_{1}+b_{n-1})}^{-1}   \\
            {(a_{2}+b_{1})}^{-1}   & {(a_{2}+b_{2})}^{-1}   & \cdots & {(a_{2}+b_{n-1})}^{-1}   \\
            \vdots                 & \vdots                 & \ddots & \vdots                   \\
            {(a_{n-1}+b_{1})}^{-1} & {(a_{n-1}+b_{2})}^{-1} & \cdots & {(a_{n-1}+b_{n-1})}^{-1}
        \end{vmatrix}
    \end{align*}
    \par Từ hệ thức truy hồi trên, và nhận xét $C_{1} = {(a_{1} + b_{1})}^{-1}$, ta được:
    \[
        C_{n} = \prod_{i>j} (a_{i}-a_{j})(b_{i}-b_{j}) \times \prod_{i\ne j}{(a_{i}+b_{j})}^{-1}.
    \]
    \endgroup{}
\end{proof}

% exercise 3.22
\begin{exercise}
    \par Dãy Fibonacci là dãy số bắt đầu với các số hạng 1, 2 và mỗi số hạng, kể từ số hạng thứ ba, đều bằng tổng của hai số hạng đứng ngay trước nó. Chứng minh rằng số hạng thứ $n$ của dãy Fibonacci bằng định thức cỡ $n$ sau đây:
    \[
        \begin{vmatrix}
            1      & 1      & 0      & \cdots & 0      & 0      \\
            -1     & 1      & 1      & \cdots & 0      & 0      \\
            0      & -1     & 1      & \cdots & 0      & 0      \\
            \vdots & \vdots & \vdots & \ddots & \vdots & \vdots \\
            0      & 0      & 0      & \cdots & -1     & 1
        \end{vmatrix}.
    \]
\end{exercise}

\begin{proof}[Lời giải]
    \[
        F_{1} = 1 = \begin{vmatrix}1\end{vmatrix}
    \]
    \[
        F_{2} = 2 = \begin{vmatrix}1 & 1 \\ -1 & 1 \end{vmatrix}
    \]
    \par Ta chứng minh bằng quy nạp, giả sử đẳng thức đúng đến $n-1$.
    \par Với $n > 2$, áp dụng khai triển Laplace cho hàng thứ nhất:
    \begingroup{}
    \allowdisplaybreaks{}
    \begin{align*}
        \underbrace{\begin{vmatrix}
                            1      & 1      & 0      & \cdots & 0      & 0      \\
                            -1     & 1      & 1      & \cdots & 0      & 0      \\
                            0      & -1     & 1      & \cdots & 0      & 0      \\
                            \vdots & \vdots & \vdots & \ddots & \vdots & \vdots \\
                            0      & 0      & 0      & \cdots & -1     & 1
                        \end{vmatrix}}_{n\times n}
         & =
        \underbrace{\begin{vmatrix}
                            1      & 1      & 0      & \cdots & 0      & 0      \\
                            -1     & 1      & 1      & \cdots & 0      & 0      \\
                            0      & -1     & 1      & \cdots & 0      & 0      \\
                            \vdots & \vdots & \vdots & \ddots & \vdots & \vdots \\
                            0      & 0      & 0      & \cdots & -1     & 1
                        \end{vmatrix}}_{(n-1)\times (n-1)}
        +
        {(-1)}^{1+2}
        \underbrace{\begin{vmatrix}
                            -1     & 1      & 0      & \cdots & 0      & 0      \\
                            -1     & 1      & 1      & \cdots & 0      & 0      \\
                            0      & -1     & 1      & \cdots & 0      & 0      \\
                            \vdots & \vdots & \vdots & \ddots & \vdots & \vdots \\
                            0      & 0      & 0      & \cdots & -1     & 1
                        \end{vmatrix}}_{(n-2)\times (n-2)} \\
         & =
        \underbrace{\begin{vmatrix}
                            1      & 1      & 0      & \cdots & 0      & 0      \\
                            -1     & 1      & 1      & \cdots & 0      & 0      \\
                            0      & -1     & 1      & \cdots & 0      & 0      \\
                            \vdots & \vdots & \vdots & \ddots & \vdots & \vdots \\
                            0      & 0      & 0      & \cdots & -1     & 1
                        \end{vmatrix}}_{(n-1)\times (n-1)}
        +
        \underbrace{\begin{vmatrix}
                            1      & 1      & 0      & \cdots & 0      & 0      \\
                            -1     & 1      & 1      & \cdots & 0      & 0      \\
                            0      & -1     & 1      & \cdots & 0      & 0      \\
                            \vdots & \vdots & \vdots & \ddots & \vdots & \vdots \\
                            0      & 0      & 0      & \cdots & -1     & 1
                        \end{vmatrix}}_{(n-2)\times (n-2)} \\
         & = F_{n-1} + F_{n-2}                                          \\
         & = F_{n}.
    \end{align*}
    \endgroup{}
    \par Vậy giả thiết quy nạp đúng, ta có được điều phải chứng minh.
\end{proof}

% exercise 3.23
\begin{exercise}
    \par Tính định thức sau đây bằng cách \textit{viết nó thành tích của hai định thức:}
    \[
        \begin{vmatrix}
            s_{0}  & s_{1}   & s_{2}   & \cdots & s_{n-1}  & 1      \\
            s_{1}  & s_{2}   & s_{3}   & \cdots & s_{n}    & x      \\
            s_{2}  & s_{3}   & s_{4}   & \cdots & s_{n+1}  & x^{2}  \\
            \vdots & \vdots  & \vdots  & \ddots & \vdots   & \vdots \\
            s_{n}  & s_{n+1} & s_{n+2} & \cdots & s_{2n-1} & x^{n}
        \end{vmatrix}.
    \]
    \par trong đó $s_{k} = \displaystyle\sum^{n}_{i=1}x_{i}^{k}$.
\end{exercise}

\begin{proof}[Lời giải]
    \[
        \begin{pmatrix}
            s_{0}  & s_{1}   & s_{2}   & \cdots & s_{n-1}  & 1      \\
            s_{1}  & s_{2}   & s_{3}   & \cdots & s_{n}    & x      \\
            s_{2}  & s_{3}   & s_{4}   & \cdots & s_{n+1}  & x^{2}  \\
            \vdots & \vdots  & \vdots  & \ddots & \vdots   & \vdots \\
            s_{n}  & s_{n+1} & s_{n+2} & \cdots & s_{2n-1} & x^{n}
        \end{pmatrix}
        =
        \begin{pmatrix}
            1           & 1           & 1           & \cdots & 1           & 1      \\
            x_{1}       & x_{2}       & x_{3}       & \cdots & x_{n}       & x      \\
            x_{1}^{2}   & x_{2}^{2}   & x_{3}^{2}   & \cdots & x_{n}^{2}   & x^{2}  \\
            \vdots      & \vdots      & \vdots      & \ddots & \vdots      & \vdots \\
            x_{1}^{n-1} & x_{2}^{n-1} & x_{3}^{n-1} & \cdots & x_{n}^{n-1} & x^{n}
        \end{pmatrix}
        \begin{pmatrix}
            1      & x_{1}  & x_{1}^{2} & \cdots & x_{1}^{n-1} & 0      \\
            1      & x_{2}  & s_{2}^{2} & \cdots & x_{2}^{n-1} & 0      \\
            \vdots & \vdots & \vdots    & \ddots & \vdots      & \vdots \\
            1      & x_{n}  & x_{n}^{2} & \cdots & x_{n}^{n-1} & 0      \\
            0      & 0      & 0         & \cdots & 0           & 1
        \end{pmatrix}
    \]
    \[
        \Rightarrow
        \begin{vmatrix}
            s_{0}  & s_{1}   & s_{2}   & \cdots & s_{n-1}  & 1      \\
            s_{1}  & s_{2}   & s_{3}   & \cdots & s_{n}    & x      \\
            s_{2}  & s_{3}   & s_{4}   & \cdots & s_{n+1}  & x^{2}  \\
            \vdots & \vdots  & \vdots  & \ddots & \vdots   & \vdots \\
            s_{n}  & s_{n+1} & s_{n+2} & \cdots & s_{2n-1} & x^{n}
        \end{vmatrix}
        =
        \prod_{i>j}(x_{i} - x_{j})\prod^{n}_{i=1}(x - x_{i})\prod_{i>j}(x_{i} - x_{j}) = \prod_{i>j}{(x_{i}-x_{j})}^{2}\prod^{n}_{i=1}(x-x_{i}).
    \]
\end{proof}

% exercise 3.24
\begin{exercise}
    \par Chứng minh rằng
    \[
        \begin{vmatrix}
            a_{1}   & a_{2}  & a_{3}  & \cdots & a_{n}   \\
            a_{n}   & a_{1}  & a_{2}  & \cdots & a_{n-1} \\
            a_{n-1} & a_{n}  & a_{1}  & \cdots & a_{n-2} \\
            \vdots  & \vdots & \vdots & \ddots & \vdots  \\
            a_{2}   & a_{3}  & a_{4}  & \cdots & a_{1}
        \end{vmatrix}
        = f(\varepsilon_{1})f(\varepsilon_{2})\cdots f(\varepsilon_{n}),
    \]
    \par trong đó, $f(X) = a_{1} + a_{2}X + \cdots + a_{n}X^{n-1}$ và $\varepsilon_{1}, \varepsilon_{2}, \ldots,\varepsilon_{n}$ là các căn bậc $n$ khác nhau của 1.
\end{exercise}

\begin{proof}
    \begingroup{}
    \allowdisplaybreaks{}
    \begin{align*}
        \begin{pmatrix}
            a_{1}   & a_{2}  & a_{3}  & \cdots & a_{n}   \\
            a_{n}   & a_{1}  & a_{2}  & \cdots & a_{n-1} \\
            a_{n-1} & a_{n}  & a_{1}  & \cdots & a_{n-2} \\
            \vdots  & \vdots & \vdots & \ddots & \vdots  \\
            a_{2}   & a_{3}  & a_{4}  & \cdots & a_{1}
        \end{pmatrix}
        \begin{pmatrix}
            1                   \\
            \varepsilon_{k}     \\
            \varepsilon_{k}^{2} \\
            \vdots              \\
            \varepsilon_{k}^{n-1}
        \end{pmatrix}
         & =
        \begin{pmatrix}
            f(\varepsilon_{k})                    \\
            \varepsilon_{k}f(\varepsilon_{k})     \\
            \varepsilon_{k}^{2}f(\varepsilon_{k}) \\
            \vdots                                \\
            \varepsilon_{k}^{n-1}f(\varepsilon_{k})
        \end{pmatrix}
        = f(\varepsilon_{k})
        \begin{pmatrix}
            1                   \\
            \varepsilon_{k}     \\
            \varepsilon_{k}^{2} \\
            \vdots              \\
            \varepsilon_{k}^{n-1}
        \end{pmatrix}                   \\
        \Rightarrow
        \begin{pmatrix}
            a_{1}   & a_{2}  & a_{3}  & \cdots & a_{n}   \\
            a_{n}   & a_{1}  & a_{2}  & \cdots & a_{n-1} \\
            a_{n-1} & a_{n}  & a_{1}  & \cdots & a_{n-2} \\
            \vdots  & \vdots & \vdots & \ddots & \vdots  \\
            a_{2}   & a_{3}  & a_{4}  & \cdots & a_{1}
        \end{pmatrix}
        \begin{pmatrix}
            1                     & 1                     & \cdots & 1                     \\
            \varepsilon_{1}       & \varepsilon_{2}       & \cdots & \varepsilon_{n}       \\
            \varepsilon_{1}^{2}   & \varepsilon_{2}^{2}   & \cdots & \varepsilon_{n}^{2}   \\
            \vdots                & \vdots                & \ddots & \vdots                \\
            \varepsilon_{1}^{n-1} & \varepsilon_{2}^{n-1} & \cdots & \varepsilon_{n}^{n-1}
        \end{pmatrix}
         & = \prod^{n}_{i=1}f(\varepsilon_{i})
        \begin{pmatrix}
            1                     & 1                     & \cdots & 1                     \\
            \varepsilon_{1}       & \varepsilon_{2}       & \cdots & \varepsilon_{n}       \\
            \varepsilon_{1}^{2}   & \varepsilon_{2}^{2}   & \cdots & \varepsilon_{n}^{2}   \\
            \vdots                & \vdots                & \ddots & \vdots                \\
            \varepsilon_{1}^{n-1} & \varepsilon_{2}^{n-1} & \cdots & \varepsilon_{n}^{n-1}
        \end{pmatrix}
    \end{align*}
    \endgroup{}

    \par Bên cạnh đó
    \[
        \begin{vmatrix}
            1                     & 1                     & \cdots & 1                     \\
            \varepsilon_{1}       & \varepsilon_{2}       & \cdots & \varepsilon_{n}       \\
            \varepsilon_{1}^{2}   & \varepsilon_{2}^{2}   & \cdots & \varepsilon_{n}^{2}   \\
            \vdots                & \vdots                & \ddots & \vdots                \\
            \varepsilon_{1}^{n-1} & \varepsilon_{2}^{n-1} & \cdots & \varepsilon_{n}^{n-1}
        \end{vmatrix}
        = \prod_{i>j}(\varepsilon_{i} - \varepsilon_{j}) \ne 0
    \]
    \par nên ma trận
    \[
        \begin{pmatrix}
            1                     & 1                     & \cdots & 1                     \\
            \varepsilon_{1}       & \varepsilon_{2}       & \cdots & \varepsilon_{n}       \\
            \varepsilon_{1}^{2}   & \varepsilon_{2}^{2}   & \cdots & \varepsilon_{n}^{2}   \\
            \vdots                & \vdots                & \ddots & \vdots                \\
            \varepsilon_{1}^{n-1} & \varepsilon_{2}^{n-1} & \cdots & \varepsilon_{n}^{n-1}
        \end{pmatrix}
    \]
    \par khả nghịch, suy ra
    \begin{align*}
        \begin{vmatrix}
            a_{1}   & a_{2}  & a_{3}  & \cdots & a_{n}   \\
            a_{n}   & a_{1}  & a_{2}  & \cdots & a_{n-1} \\
            a_{n-1} & a_{n}  & a_{1}  & \cdots & a_{n-2} \\
            \vdots  & \vdots & \vdots & \ddots & \vdots  \\
            a_{2}   & a_{3}  & a_{4}  & \cdots & a_{1}
        \end{vmatrix}
        \begin{vmatrix}
            1                     & 1                     & \cdots & 1                     \\
            \varepsilon_{1}       & \varepsilon_{2}       & \cdots & \varepsilon_{n}       \\
            \varepsilon_{1}^{2}   & \varepsilon_{2}^{2}   & \cdots & \varepsilon_{n}^{2}   \\
            \vdots                & \vdots                & \ddots & \vdots                \\
            \varepsilon_{1}^{n-1} & \varepsilon_{2}^{n-1} & \cdots & \varepsilon_{n}^{n-1}
        \end{vmatrix}
         & =
        \prod^{n}_{i=1}f(\varepsilon_{i})
        \begin{vmatrix}
            1                     & 1                     & \cdots & 1                     \\
            \varepsilon_{1}       & \varepsilon_{2}       & \cdots & \varepsilon_{n}       \\
            \varepsilon_{1}^{2}   & \varepsilon_{2}^{2}   & \cdots & \varepsilon_{n}^{2}   \\
            \vdots                & \vdots                & \ddots & \vdots                \\
            \varepsilon_{1}^{n-1} & \varepsilon_{2}^{n-1} & \cdots & \varepsilon_{n}^{n-1}
        \end{vmatrix} \\
        \Leftrightarrow
        \begin{vmatrix}
            a_{1}   & a_{2}  & a_{3}  & \cdots & a_{n}   \\
            a_{n}   & a_{1}  & a_{2}  & \cdots & a_{n-1} \\
            a_{n-1} & a_{n}  & a_{1}  & \cdots & a_{n-2} \\
            \vdots  & \vdots & \vdots & \ddots & \vdots  \\
            a_{2}   & a_{3}  & a_{4}  & \cdots & a_{1}
        \end{vmatrix}
         & =
        \prod^{n}_{i=1}f(\varepsilon_{i}).
    \end{align*}
\end{proof}

% exercise 3.25
\begin{exercise}
    \par Dùng khai triển Laplace chứng minh rằng nếu một định thức cỡ $n$ có các yếu tố nằm trên giao của $k$ hàng và $\ell$ cột xác định nào đó đều bằng 0, trong đó $k + \ell > n$, thì định thức đó bằng 0.
\end{exercise}

\begin{proof}
    \par Chú ý rằng: khi đổi chỗ các hàng, hay đổi chỗ các cột thì giá trị định thức chỉ đổi dấu..
    \par Do đó, không mất tính tổng quát, ta có thể giả sử các yếu tố nằm trên giao của $k$ hàng đầu tiên và $\ell$ cột đầu tiên đều bằng 0.
    \par Ma trận khi đó có dạng sau:
    \[
        A=
        \begin{pmatrix}
            0          & 0          & \cdots & 0             & a_{1(\ell+1)}     & \cdots & a_{1n}     \\
            0          & 0          & \cdots & 0             & a_{2(\ell+1)}     & \cdots & a_{2n}     \\
            \vdots     & \vdots     & \ddots & \vdots        & \vdots            & \ddots & \vdots     \\
            0          & 0          & \cdots & 0             & a_{k(\ell+1)}     & \cdots & a_{kn}     \\
            a_{(k+1)1} & a_{(k+1)2} & \cdots & a_{(k+1)\ell} & a_{(k+1)(\ell+1)} & \cdots & a_{(k+1)n} \\
            \vdots     & \vdots     & \ddots & \vdots        & \vdots            & \ddots & \vdots     \\
            a_{n1}     & a_{n2}     & \cdots & a_{n\ell}     & a_{n(\ell+1)}     & \cdots & a_{nn}
        \end{pmatrix}.
    \]
    \par Áp dụng khai triển Laplace cho $k$ hàng đầu tiên
    \[
        \det A = \sum_{1\le j_{1} < \cdots < j_{k}\le n}(-1){}^{1+\cdots+k+j_{1}+\cdots+j_{k}}D_{1,\ldots,k}^{j_{1},\ldots,j_{k}}\overline{D}_{1,\ldots,k}^{j_{1},\ldots,j_{k}}.
    \]
    \par Giả sử trong các cột $1\le j_{1} < \cdots < j_{k}\le n$, không có cột nào thuộc $\ell$ cột đầu tiên. Như vậy, số cột của ma trận $A$ sẽ lớn hơn hoặc bằng $k + \ell$, tức là $n \ge k + \ell$. Điều này mâu thuẫn với giả thiết $k + \ell > n$.
    \par Do đó, trong các cột $1\le j_{1} < \cdots < j_{k}\le n$, có ít nhất một cột thuộc $\ell$ cột đầu tiên. Điều này dẫn tới việc, bất kể chọn $k$ cột nào thì giá trị của định thức con  $D^{j_{1},\ldots,j_{k}}_{1,\ldots,k}$ bằng không.
    \par Vậy $\det A = 0$.
\end{proof}

% exercise 3.26
\begin{exercise}
    \par Giải hệ phương trình sau đây bằng phương pháp Cramer và phương pháp khử:
    \[
        \begin{array}{ccccccccccc}
            3x_{1} & + & 4x_{2} & + & x_{3}  & + & 2x_{4} & + & 3 & = & 0, \\
            3x_{1} & + & 5x_{2} & + & 3x_{3} & + & 5x_{4} & + & 6 & = & 0, \\
            6x_{1} & + & 8x_{2} & + & x_{3}  & + & 5x_{4} & + & 8 & = & 0, \\
            3x_{1} & + & 5x_{2} & + & 3x_{3} & + & 7x_{4} & + & 8 & = & 0.
        \end{array}
    \]
\end{exercise}

\begin{proof}[Lời giải]
    \par Sử dụng công thức Cramer.
    \par Định thức của ma trận hệ số:
    \begingroup{}
    \allowdisplaybreaks{}
    \begin{align*}
        \det A & =
        \begin{vmatrix}
            3 & 4 & 1 & 2 \\
            3 & 5 & 3 & 5 \\
            6 & 8 & 1 & 5 \\
            3 & 5 & 3 & 7
        \end{vmatrix}
        =
        \begin{vmatrix}
            3 & 4 & 1  & 2 \\
            0 & 1 & 2  & 3 \\
            0 & 0 & -1 & 1 \\
            0 & 1 & 2  & 5
        \end{vmatrix} \\
               & =
        \begin{vmatrix}
            3 & 4 & 1  & 2 \\
            0 & 1 & 2  & 3 \\
            0 & 0 & -1 & 1 \\
            0 & 0 & 0  & 2
        \end{vmatrix}
        = -6.
    \end{align*}
    \endgroup{}
    \par Áp dụng công thức Cramer
    \[
        x_{1} = \dfrac{
            \begin{vmatrix}
                -3 & 4 & 1 & 2 \\
                -6 & 5 & 3 & 5 \\
                -8 & 8 & 1 & 5 \\
                -8 & 5 & 3 & 7
            \end{vmatrix}
        }{\det A} = \dfrac{-12}{-6} = 2,
    \]
    \[
        x_{2} = \dfrac{
            \begin{vmatrix}
                3 & -3 & 1 & 2 \\
                3 & -6 & 3 & 5 \\
                6 & -8 & 1 & 5 \\
                3 & -8 & 3 & 7
            \end{vmatrix}
        }{\det A} = \dfrac{12}{-6} = -2,
    \]
    \[
        x_{3} = \dfrac{
            \begin{vmatrix}
                3 & 4 & -3 & 2 \\
                3 & 5 & -6 & 5 \\
                6 & 8 & -8 & 5 \\
                3 & 5 & -8 & 7
            \end{vmatrix}
        }{\det A} = \dfrac{-6}{-6} = 1,
    \]
    \[
        x_{4} = \dfrac{
            \begin{vmatrix}
                3 & 4 & 1 & -3 \\
                3 & 5 & 3 & -6 \\
                6 & 8 & 1 & -8 \\
                3 & 5 & 3 & -8
            \end{vmatrix}
        }{\det A} = \dfrac{6}{-6} = -1.
    \]
    \bigskip
    \par Sử dụng phương pháp khử.
    \begingroup{}
    \allowdisplaybreaks{}
    \begin{gather*}
        \left(\begin{array}{cccc|c}
                3 & 4 & 1 & 2 & -3 \\
                3 & 5 & 3 & 5 & -6 \\
                6 & 8 & 1 & 5 & -8 \\
                3 & 5 & 3 & 7 & -8
            \end{array}
        \right)
        \Longleftrightarrow{}
        \left(\begin{array}{cccc|c}
                3 & 4 & 1  & 2 & -3 \\
                0 & 1 & 2  & 3 & -3 \\
                0 & 0 & -1 & 1 & -2 \\
                0 & 1 & 2  & 5 & -5
            \end{array}
        \right)
        \Longleftrightarrow{}
        \left(\begin{array}{cccc|c}
                3 & 4 & 1  & 2 & -3 \\
                0 & 1 & 2  & 3 & -3 \\
                0 & 0 & -1 & 1 & -2 \\
                0 & 0 & 0  & 2 & -2
            \end{array}
        \right).
    \end{gather*}
    \endgroup{}
    \par Suy ra $x_{4} = -1$, $x_{3} = 1$, $x_{2} = -2$, $x_{1} = 2$.
\end{proof}

% exercise 3.27
\begin{exercise}
    \par Chứng minh rằng một đa thức bậc $n$ trong $\mathbb{F}[X]$ được hoàn toàn xác định bởi giá trị của nó lại $(n+1)$ điểm khác nhau của trường $\mathbb{F}$. Tìm ví dụ về hai đa thức khác nhau cùng bậc $n$ nhận các giá trị bằng nhau tại mọi điểm của $\mathbb{F}$, nếu số phần tử của $\mathbb{F}$ không vượt quá $n$.
\end{exercise}

\begin{proof}[Lời giải]
    \par Giả sử ta có $n + 1$ cặp giá trị $(x_{i}, y_{i})$, trong đó $i \in \{ 0; 1; \ldots; n \}$ sao cho $x_{i} \ne x_{j}, \forall i\ne j$.
    \par Một đa thức $f(X) = a_{0} + a_{1}X + \cdots + a_{n}X^{n}$ bậc $n$ thỏa mãn $f(x_{i}) = y_{i}, \forall i$ nếu và chỉ nếu hệ phương trình tuyến tính sau có nghiệm
    \begin{align*}
         & a_{0} + a_{1}x_{0} + \cdots + a_{n}x_{0}^{n} = y_{0} \\
         & a_{0} + a_{1}x_{1} + \cdots + a_{n}x_{1}^{n} = y_{1} \\
         & a_{0} + a_{1}x_{2} + \cdots + a_{n}x_{2}^{n} = y_{2} \\
         & \vdots                                               \\
         & a_{0} + a_{1}x_{n} + \cdots + a_{n}x_{n}^{n} = y_{n} \\
    \end{align*}
    \par Hệ phương trình tuyến tính này có $(n+1)$ ẩn $a_{0}, a_{1}, \ldots, a_{n}$ và $(n+1)$ phương trình. Bên cạnh đó, hệ này có ma trận hệ số là ma trận Vandermonde của $(n+1)$ biến đôi một khác nhau, tức là định thức của ma trận hệ số khác không.
    \par Do đó hệ phương trình tuyến tính trên có nghiệm duy nhất. Điều này cũng chứng tỏ đa thức $f(X)$ bậc $n$ được xác định duy nhất bởi giá trị của nó tại $(n+1)$ điểm khác nhau.
    \bigskip
    \bigskip
    \par Nếu $n < 3$, không có ví dụ nào như vậy, vì số phần tử của một trường luôn lớn hơn hoặc bằng 2.
    \par Nếu $n\ge 3$, ta chọn trường $\mathbb{F}_{2}$ và lấy hai đa thức phân biệt:
    \[
        \begin{cases}
            f(X) = X{(X-1)}^{n-1}, \\
            g(X) = X^{n-1}(X-1).
        \end{cases}
    \]
    \par Hai đa thức này luôn cùng nhận giá trị 0 tại mọi điểm của $\mathbb{F}_{2}$.
\end{proof}

\par Giải các hệ phương trình sau đây bằng phương pháp thích hợp:

% exercise 3.28
\begin{exercise}
    \begin{align*}
        \phantom{-}ax + by + cz + dt & = p, \\
        -bx + ay + dz - ct           & = q, \\
        -cx - dy + az + bt           & = r, \\
        -dx + cy - bz + at           & = s.
    \end{align*}
\end{exercise}

\begin{proof}[Lời giải]
    \par Ta tính định thức của ma trận hệ số
    \begingroup{}
    \allowdisplaybreaks{}
    \begin{align*}
        \begin{vmatrix}
            a  & b  & c  & d  \\
            -b & a  & d  & -c \\
            -c & -d & a  & b  \\
            -d & c  & -b & a
        \end{vmatrix}
         & = {(-1)}^{1+2+1+2}
        \begin{vmatrix}
            a  & b \\
            -b & a
        \end{vmatrix}
        \begin{vmatrix}
            a  & b \\
            -b & a
        \end{vmatrix}
        + {(-1)}^{1+2+1+3}
        \begin{vmatrix}
            a  & b  \\
            -c & -d
        \end{vmatrix}
        \begin{vmatrix}
            d  & -c \\
            -b & a
        \end{vmatrix}
        + {(-1)}^{1+2+1+4}
        \begin{vmatrix}
            a  & b \\
            -d & c
        \end{vmatrix}
        \begin{vmatrix}
            d & -c \\
            a & b
        \end{vmatrix}                                                                                                 \\
         & + {(-1)}^{1+2+2+3}
        \begin{vmatrix}
            -b & a  \\
            -c & -d
        \end{vmatrix}
        \begin{vmatrix}
            c  & d \\
            -b & a
        \end{vmatrix}
        + {(-1)}^{1+2+2+4}
        \begin{vmatrix}
            -b & a \\
            -d & c
        \end{vmatrix}
        \begin{vmatrix}
            c & d \\
            a & b
        \end{vmatrix}
        + {(-1)}^{1+2+3+4}
        \begin{vmatrix}
            -c & -d \\
            -d & c
        \end{vmatrix}
        \begin{vmatrix}
            c & d  \\
            d & -c
        \end{vmatrix}                                                                                                 \\
         & = {(a^{2}+b^{2})}^{2} + {(ad-bc)}^{2} + {(ac+bd)}^{2} + {(ac+bd)}^{2} + {(bc-ad)}^{2} + {(c^{2}+d^{2})}^{2} \\
         & = {(a^{2}+b^{2})}^{2} + {(c^{2}+d^{2})}^{2} + 2{(ac+bd)}^{2} + 2{(ad-bc)}^{2}                               \\
         & = {(a^{2}+b^{2})}^{2} + {(c^{2}+d^{2})}^{2} + 2(a^{2}+b^{2})(c^{2}+d^{2})                                   \\
         & = {(a^{2}+b^{2}+c^{2}+d^{2})}^{2}.
    \end{align*}
    \endgroup{}
    \par Nếu $a^{2} + b^{2} + c^{2} + d^{2} \ne 0$ thì hệ phương trình tuyến tính trên có nghiệm duy nhất:
    \[
        x_{1} = \dfrac{1}{(a^{2} + b^{2} + c^{2} + d^{2}){}^{2}}
        \begin{vmatrix}
            p & b  & c  & d  \\
            q & a  & d  & -c \\
            r & -d & a  & b  \\
            s & c  & -b & a
        \end{vmatrix},
    \]
    \[
        x_{2} = \dfrac{1}{(a^{2} + b^{2} + c^{2} + d^{2}){}^{2}}
        \begin{vmatrix}
            a  & p & c  & d  \\
            -b & q & d  & -c \\
            -c & r & a  & b  \\
            -d & s & -b & a
        \end{vmatrix},
    \]
    \[
        x_{3} = \dfrac{1}{(a^{2} + b^{2} + c^{2} + d^{2}){}^{2}}
        \begin{vmatrix}
            a  & b  & p & d  \\
            -b & a  & q & -c \\
            -c & -d & r & b  \\
            -d & c  & s & a
        \end{vmatrix},
    \]
    \[
        x_{4} = \dfrac{1}{(a^{2} + b^{2} + c^{2} + d^{2}){}^{2}}
        \begin{vmatrix}
            a  & b  & c  & p \\
            -b & a  & d  & q \\
            -c & -d & a  & r \\
            -d & c  & -b & s
        \end{vmatrix}.
    \]
    \par Ngược lại, nếu $a^{2} + b^{2} + c^{2} + d^{2} = 0$ thì $a = b = c = d = 0$. Nếu $(p, q, r, s) \ne (0, 0, 0, 0)$ thì hệ phương trình vô nghiệm, ngược lại, tập nghiệm của hệ phương trình là $\mathbb{R}{}^{4}$.
\end{proof}

% exercise 3.29
\begin{exercise}
    \[
        \begin{array}{ccccccccccccc}
            x_{n}  & +      & a_{1}x_{n-1} & +      & a_{1}^{2}x_{n-2} & +      & \cdots & +      & a_{1}^{n-1}x_{1} & +      & a_{1}^{n} & =      & 0      \\
            x_{n}  & +      & a_{2}x_{n-1} & +      & a_{2}^{2}x_{n-2} & +      & \cdots & +      & a_{2}^{n-1}x_{1} & +      & a_{2}^{n} & =      & 0      \\
            \vdots & \vdots & \vdots       & \vdots & \vdots           & \vdots & \ddots & \vdots & \vdots           & \vdots & \vdots    & \vdots & \vdots \\
            x_{n}  & +      & a_{n}x_{n-1} & +      & a_{n}^{2}x_{n-2} & +      & \cdots & +      & a_{n}^{n-1}x_{1} & +      & a_{n}^{n} & =      & 0      \\
        \end{array}
    \]
\end{exercise}

\begin{proof}[Lời giải]
    \par $A$ là ma trận hệ số của hệ phương trình tuyến tính trên.
    \par Như vậy $A$ là một ma trận Vandermonde.

    \begin{enumerate}[label = \textbf{Trường hợp \arabic*.},itemindent=2cm]
        \item $a_{1}$, $a_{2}$, \ldots $a_{n}$ đôi một khác nhau.
              \par Lúc này, định thức Vandermonde của nó khác không.
              \par Áp dụng công thức Cramer:
              \[
                  x_{n-k} = \dfrac{\det A_{k}}{\det A}
              \]
              \par trong đó $A_{k}$ là ma trận $A$ sau khi thay cột $\begin{pmatrix} a_{1}^{k} \\ a_{2}^{k} \\ \vdots \\ a_{n}^{k} \end{pmatrix}$ bởi $\begin{pmatrix}-a_{1}^{n} \\ -a_{2}^{n} \\ \vdots \\ -a_{n}^{n} \end{pmatrix}$.
              \par Sử dụng kết quả từ bài toán~\ref{chapter3:vandermonde-and-symmetric-polynomials}:
              \[
                  \det A_{k} = (-1)(-1){}^{n-k-1}
                  \begin{vmatrix}
                      1      & a_{1}  & \cdots & a_{1}^{k-1} & a_{1}^{k+1} & \cdots & a_{1}^{n} \\
                      1      & a_{1}  & \cdots & a_{1}^{k-1} & a_{1}^{k+1} & \cdots & a_{1}^{n} \\
                      \vdots & \vdots & \ddots & \vdots      & \vdots      & \ddots & \vdots    \\
                      1      & a_{1}  & \cdots & a_{1}^{k-1} & a_{1}^{k+1} & \cdots & a_{1}^{n} \\
                  \end{vmatrix}
                  = (-1){}^{n-k}D_{n}e_{n-k}.
              \]
              \par Suy ra $x_{n-k} = (-1){}^{n-k}e_{n-k}(a_{1},\ldots, a_{n})$.
              \[
                  \begin{cases}
                      x_{1} = (-1)e_{1}(a_{1},\ldots,a_{n})       \\
                      x_{2} = (-1){}^{2}e_{2}(a_{1},\ldots,a_{n}) \\
                      \vdots                                      \\
                      x_{n} = (-1){}^{n}e_{n}(a_{1},\ldots,a_{n})
                  \end{cases}
              \]
              \par trong đó, nhắc lại rằng $e_{k}$ là đa thức đối xứng sơ cấp bậc $k$.
        \item Trong $n$ hệ số $a_{1}$, $a_{2}$, \ldots $a_{n}$, có ít nhất hai hệ số bằng nhau.
              \par Giả sử rằng, sau khi loại bỏ các hệ số dư thừa, ta còn lại $m$ hệ số ($m < n$). Không giảm tổng quát, có thể đánh số lại các hệ số. $m$ hệ số đôi một khác nhau được đánh số lại là $a_{1}$, $a_{2}$, \ldots $a_{m}$.
              \par Thực hiện các phép biến đổi sơ cấp trên các hàng của ma trận $m\times(n+1)$ sau:
              \begingroup{}
              \allowdisplaybreaks{}
              \begin{align*}
                                      &
                  \begin{pmatrix}
                      1      & a_{1}  & a_{1}^{2} & \cdots & a_{1}^{n-1} & a_{1}^{n} \\
                      1      & a_{2}  & a_{2}^{2} & \cdots & a_{2}^{n-1} & a_{2}^{n} \\
                      \vdots & \vdots & \vdots    & \ddots & \vdots      & \vdots    \\
                      1      & a_{m}  & a_{m}^{2} & \cdots & a_{m}^{n-1} & a_{m}^{n}
                  \end{pmatrix}                                                                                                                         \\
                  \Longleftrightarrow &
                  \begin{pmatrix}
                      1      & a_{1}         & a_{1}^{2}             & \cdots & a_{1}^{n-1}               & a_{1}^{n}             \\
                      0      & a_{2} - a_{1} & a_{2}^{2} - a_{1}^{2} & \cdots & a_{2}^{n-1} - a_{1}^{n-1} & a_{2}^{n} - a_{1}^{n} \\
                      \vdots & \vdots        & \vdots                & \ddots & \vdots                    & \vdots                \\
                      0      & a_{m} - a_{1} & a_{m}^{2} - a_{1}^{2} & \cdots & a_{m}^{n-1} - a_{1}^{n-1} & a_{m}^{n} - a_{m}^{n}
                  \end{pmatrix}                                                                            \\
                  \Longleftrightarrow &
                  \begin{pmatrix}
                      1      & a_{1}  & a_{1}^{2}           & a_{1}^{3}           & \cdots & a_{1}^{n-1}           & a_{1}^{n}             \\
                      0      & 1      & h_{1}(a_{1}, a_{2}) & h_{2}(a_{1}, a_{2}) & \cdots & h_{n-2}(a_{1}, a_{2}) & h_{n-1}(a_{1}, a_{2}) \\
                      0      & 1      & h_{1}(a_{1}, a_{3}) & h_{2}(a_{1}, a_{3}) & \cdots & h_{n-2}(a_{1}, a_{2}) & h_{n-1}(a_{1}, a_{2}) \\
                      \vdots & \vdots & \vdots              & \vdots              & \ddots & \vdots                & \vdots                \\
                      0      & 1      & h_{1}(a_{1}, a_{m}) & h_{2}(a_{1}, a_{m}) & \cdots & h_{n-2}(a_{1}, a_{m}) & h_{n-1}(a_{1}, a_{m})
                  \end{pmatrix}                                                                   \\
                  \Longleftrightarrow &
                  \begin{pmatrix}
                      1      & a_{1}  & a_{1}^{2}           & a_{1}^{3}                                 & \cdots & a_{1}^{n-1}                                 & a_{1}^{n}                                   \\
                      0      & 1      & h_{1}(a_{1}, a_{2}) & h_{2}(a_{1}, a_{2})                       & \cdots & h_{n-2}(a_{1}, a_{2})                       & h_{n-1}(a_{1}, a_{2})                       \\
                      0      & 0      & a_{3} - a_{2}       & (a_{3} - a_{2})h_{1}(a_{1}, a_{2}, a_{3}) & \cdots & (a_{3} - a_{2})h_{n-3}(a_{1}, a_{2}, a_{3}) & (a_{3} - a_{2})h_{n-2}(a_{1}, a_{2}, a_{3}) \\
                      \vdots & \vdots & \vdots              & \vdots                                    & \ddots & \vdots                                      & \vdots                                      \\
                      0      & 0      & a_{m} - a_{2}       & (a_{m} - a_{2})h_{1}(a_{1}, a_{2}, a_{m}) & \cdots & (a_{m} - a_{2})h_{n-3}(a_{1}, a_{2}, a_{m}) & (a_{m} - a_{2})h_{n-2}(a_{1}, a_{m})
                  \end{pmatrix} \\
                  \Longleftrightarrow &
                  \begin{pmatrix}
                      1      & a_{1}  & a_{1}^{2}           & a_{1}^{3}                  & \cdots & a_{1}^{n-1}                  & a_{1}^{n}                    \\
                      0      & 1      & h_{1}(a_{1}, a_{2}) & h_{2}(a_{1}, a_{2})        & \cdots & h_{n-2}(a_{1}, a_{2})        & h_{n-1}(a_{1}, a_{2})        \\
                      0      & 0      & 1                   & h_{1}(a_{1}, a_{2}, a_{3}) & \cdots & h_{n-3}(a_{1}, a_{2}, a_{3}) & h_{n-2}(a_{1}, a_{2}, a_{3}) \\
                      \vdots & \vdots & \vdots              & \vdots                     & \ddots & \vdots                       & \vdots                       \\
                      0      & 0      & 1                   & h_{1}(a_{1}, a_{2}, a_{m}) & \cdots & h_{n-2}(a_{1}, a_{2}, a_{m}) & h_{n-2}(a_{1}, a_{2}, a_{m})
                  \end{pmatrix}                                              \\
                                      & \ddots                                                                                                                                                           \\
                                      & \Longleftrightarrow
                  \begin{pmatrix}
                      1      & a_{1}  & a_{1}^{2}           & \cdots & a_{1}^{m-1}                  & \cdots & a_{1}^{n-1}                  & a_{1}^{n}                      \\
                      0      & 1      & h_{1}(a_{1}, a_{2}) & \cdots & h_{m-2}(a_{1}, a_{2})        & \cdots & h_{n-2}(a_{1}, a_{2})        & h_{n-1}(a_{1}, a_{2})          \\
                      0      & 0      & 1                   & \cdots & h_{m-3}(a_{1}, a_{2}, a_{3}) & \cdots & h_{n-3}(a_{1}, a_{2}, a_{3}) & h_{n-2}(a_{1}, a_{2}, a_{3})   \\
                      \vdots & \vdots & \vdots              & \vdots & \ddots                       & \vdots & \vdots                       & \vdots                         \\
                      0      & 0      & 0                   & \cdots & 1                            & \cdots & h_{n-m}(a_{1},\ldots, a_{m}) & h_{n-m+1}(a_{1},\ldots, a_{m})
                  \end{pmatrix}                                 \\
              \end{align*}
              \endgroup{}
              \par (trong đó, nhắc lại $h_{k}$ là đa thức đối xứng thuần nhất đầy đủ).
              \par Như vậy, hệ phương trình tuyến tính có nghiệm.
              \par $x_{n-m}$, $x_{n-m-1}$, \ldots, $x_{0}$ có thể nhận giá trị bất kì.
              \par $x_{n-m+1}$ được xác định từ phương trình thứ $m$.
              \par $x_{n-m+2}$ được xác định từ phương trình thứ $m-1$.
              \par $\ddots$
              \par $x_{n}$ được xác định từ phương trình thứ 1.
    \end{enumerate}
\end{proof}

% exercise 3.30
\begin{exercise}
    \par Đặt $s_{n}(k) = 1^{n} + 2^{n} + \cdots + (k-1){}^{n}$. Hãy thiết lập phương trình
    \[
        k^{n} = 1 + \binom{n}{n-1}s_{n-1}(k) + \cdots + \binom{n}{1}s_{1}(k) + s_{0}(k)
    \]
    \par và chứng minh rằng
    \[
        s_{n-1}(k) = \frac{1}{n!}
        \begin{vmatrix}
            k^{n}   & \binom{n}{n-2}   & \binom{n}{n-3}   & \cdots & \binom{n}{1}   & 1      \\
            k^{n-1} & \binom{n-1}{n-2} & \binom{n-1}{n-3} & \cdots & \binom{n-1}{1} & 1      \\
            k^{n-2} & 0                & \binom{n-2}{n-3} & \cdots & \binom{n-2}{1} & 1      \\
            \vdots  & \vdots           & \vdots           & \ddots & \vdots         & \vdots \\
            k^{2}   & 0                & 0                & \cdots & \binom{2}{1}   & 1      \\
            k       & 0                & 0                & \cdots & 0              & 1
        \end{vmatrix}.
    \]
\end{exercise}

\begin{proof}
    \par Áp dụng định lý nhị thức Newton:
    \begin{align*}
        k^{n}       & = 1 + \binom{n}{1}(k-1) + \cdots + \binom{n}{n-1}(k-1){}^{n-1} + (k-1){}^{n} \\
        (k-1){}^{n} & = 1 + \binom{n}{1}(k-2) + \cdots + \binom{n}{n-1}(k-2){}^{n-1} + (k-2){}^{n} \\
                    & \vdots                                                                       \\
        2^{n}       & = 1 + \binom{n}{1}1     + \cdots + \binom{n}{n-1}1^{n-1} + 1^{n}
    \end{align*}
    \par Cộng vế theo vế của tất cả đẳng thức trên:
    \begin{align*}
        k^{n} + (k-1){}^{n} + \cdots + 2^{n} & = s_{0}(k) + \binom{n}{1}s_{1}(k) + \cdots + \binom{n}{n-1}s_{n-1}(k) + (k-1){}^{n} + \cdots + 1 \\
        \Longleftrightarrow k^{n}            & = s_{0}(k) + \binom{n}{1}s_{1}(k) + \cdots + \binom{n}{n-1}s_{n-1}(k) + 1                        \\
        \Longleftrightarrow k^{n}            & = 1 + \binom{n}{n-1}s_{n-1}(k) + \cdots + \binom{n}{1}s_{1}(k) + s_{0}(k).
    \end{align*}

    \par Áp dụng công thức trên với các giá trị $n$ nhỏ hơn:
    \begin{align*}
        k^{n}   & = 1 + \binom{n}{n-1}s_{n-1}(k) + \cdots + \binom{n}{1}s_{1}(k) + s_{0}(k)     \\
        k^{n-1} & = 1 + \binom{n-1}{n-2}s_{n-2}(k) + \cdots + \binom{n-1}{1}s_{1}(k) + s_{0}(k) \\
                & \ddots                                                                        \\
        k       & = 1 + s_{0}(k)
    \end{align*}
    \par $s_{n-1}(k)$, $s_{n-2}(k)$, \ldots, $s_{0}(k)$ là nghiệm của hệ phương trình tuyến tính:
    \[
        \begin{array}{ccccccccccc}
            \binom{n}{n-1}x_{n-1} & + & \binom{n}{n-2}x_{n-2}   & + & \cdots & + & \binom{n}{1}x_{1}   & + & x_{0} & =      & k^{n} - 1   \\
                                  &   & \binom{n-1}{n-2}x_{n-2} & + & \cdots & + & \binom{n-1}{1}x_{1} & + & x_{0} & =      & k^{n-1} - 1 \\
                                  &   &                         &   & \ddots &   &                     &   &       & \vdots &             \\
                                  &   &                         &   &        &   &                     &   & x_{0} & =      & k - 1
        \end{array}
    \]
    \par Định thức của ma trận hệ số bằng:
    \[
        \begin{vmatrix}
            \binom{n}{n-1} & \binom{n}{n-2}   & \cdots & \binom{n}{1}   & 1      \\
            0              & \binom{n-1}{n-2} & \cdots & \binom{n-1}{1} & 1      \\
            \vdots         & \vdots           & \ddots & \vdots         & \vdots \\
            0              & 0                & \cdots & 0              & 1
        \end{vmatrix}
        = \binom{n}{n-1}\binom{n-1}{n-2}\cdots\binom{2}{1}
        = n!
    \]
    \par Áp dụng công thức Cramer:
    \[
        s_{n-1}(k) = x_{n-1} = \dfrac{1}{n!}
        \begin{vmatrix}
            k^{n} - 1   & \binom{n}{n-2} & \binom{n}{n-3}   & \cdots & \binom{n}{1}   & 1      \\
            k^{n-1} - 1 & 0              & \binom{n-1}{n-3} & \cdots & \binom{n-1}{1} & 1      \\
            k^{n-2} - 1 & 0              & 0                & \cdots & \binom{n-2}{1} & 1      \\
            \vdots      & \vdots         & \vdots           & \ddots & \vdots         & \vdots \\
            k - 1       & 0              & 0                & \cdots & 0              & 1
        \end{vmatrix}
        = \dfrac{1}{n!}
        \begin{vmatrix}
            k^{n}   & \binom{n}{n-2} & \binom{n}{n-3}   & \cdots & \binom{n}{1}   & 1      \\
            k^{n-1} & 0              & \binom{n-1}{n-3} & \cdots & \binom{n-1}{1} & 1      \\
            k^{n-2} & 0              & 0                & \cdots & \binom{n-2}{1} & 1      \\
            \vdots  & \vdots         & \vdots           & \ddots & \vdots         & \vdots \\
            k       & 0              & 0                & \cdots & 0              & 1
        \end{vmatrix}.
    \]
\end{proof}

% exercise 3.31
\begin{exercise}
    \par Xét khai triển $\frac{x}{e^{x} - 1} = 1 + b_{1}x + b_{2}x^{2} + b_{3}x^{3} + \cdots$. Ta đặt $b_{2n} = \frac{(-1){}^{n-1}B_{n}}{(2n)!}$, trong đó $B_{n}$ được gọi là số Bernoulli thứ $n$. Chứng minh rằng
    \[
        B_{n} = (-1){}^{n-1}(2n)!
        \begin{vmatrix}
            \frac{1}{2!}      & 1               & 0                 & 0                 & \cdots & 0            \\
            \frac{1}{3!}      & \frac{1}{2!}    & 1                 & 0                 & \cdots & 0            \\
            \frac{1}{4!}      & \frac{1}{3!}    & \frac{1}{2!}      & 1                 & \cdots & 0            \\
            \vdots            & \vdots          & \vdots            & \vdots            & \ddots & \vdots       \\
            \frac{1}{(2n+1)!} & \frac{1}{(2n)!} & \frac{1}{(2n-1)!} & \frac{1}{(2n-2)!} & \cdots & \frac{1}{2!}
        \end{vmatrix},
    \]
    \par và chỉ ra rằng
    \[
        b_{2n-1} =
        \begin{vmatrix}
            \frac{1}{2!}    & 1                 & 0                 & 0                 & \cdots & 0            \\
            \frac{1}{3!}    & \frac{1}{2!}      & 1                 & 0                 & \cdots & 0            \\
            \frac{1}{4!}    & \frac{1}{3!}      & \frac{1}{2!}      & 1                 & \cdots & 0            \\
            \vdots          & \vdots            & \vdots            & \vdots            & \ddots & \vdots       \\
            \frac{1}{(2n)!} & \frac{1}{(2n-1)!} & \frac{1}{(2n-2)!} & \frac{1}{(2n-3)!} & \cdots & \frac{1}{2!}
        \end{vmatrix}
        = 0
    \]
    \par với mọi $n > 1$.
\end{exercise}

\begin{proof}
    \begingroup{}
    \allowdisplaybreaks{}
    \begin{align*}
        \dfrac{x}{e^{x}-1} & = 1 + b_{1}x + b_{2}x^{2} + b_{3}x^{3} + \cdots                                                                                  \\
        \Leftrightarrow 1  & = (1 + b_{1}x + b_{2}x^{2} + b_{3}x^{3} + \cdots)\left(1 + \dfrac{x}{2!} + \dfrac{x^{2}}{3!} + \dfrac{x^{3}}{4!} + \cdots\right)
    \end{align*}
    \endgroup{}
    \par Đồng nhất hệ số của $x$, $x^{2}$, $x^{3}$, \ldots, $x^{2n}$ trong khai triển trên, ta được hệ phương trình tuyến tính gồm $(2n)$ phương trình:
    \begin{align*}
         & b_{1}                                                               & = \dfrac{-1}{2!}      \\
         & \dfrac{b_{1}}{2!} + b_{2}                                           & = \dfrac{-1}{3!}      \\
         & \dfrac{b_{1}}{3!} + \dfrac{b_{2}}{2!} + b_{3}                       & = \dfrac{-1}{4!}      \\
         &                                                                     & \vdots                \\
         & \dfrac{b_{1}}{(2n-1)!} + \dfrac{b_{2}}{(2n-2)!} + \cdots + b_{2n-1} & = \dfrac{-1}{(2n)!}   \\
         & \dfrac{b_{1}}{(2n)!} + \dfrac{b_{2}}{(2n-1)!} + \cdots + b_{2n}     & = \dfrac{-1}{(2n+1)!}
    \end{align*}
    \par Định thức của ma trận hệ số của hệ $(2n)$ phương trình tuyến tính này bằng:
    \[
        \begin{vmatrix}
            1               & 0                 & 0                 & \cdots & 0            & 0      \\
            \frac{1}{2!}    & 1                 & 0                 & \cdots & 0            & 0      \\
            \frac{1}{3!}    & \frac{1}{2!}      & 1                 & \cdots & 0            & 0      \\
            \vdots          & \vdots            & \vdots            & \ddots & \vdots       & \vdots \\
            \frac{1}{(2n)!} & \frac{1}{(2n-1)!} & \frac{1}{(2n-2)!} & \cdots & \frac{1}{2!} & 1
        \end{vmatrix} = 1 \ne 0.
    \]
    \par Áp dụng công thức Cramer
    \begingroup{}
    \allowdisplaybreaks{}
    \begin{align*}
        b_{2n} & =
        \begin{vmatrix}
            1               & 0                 & 0                 & \cdots & 0            & \frac{-1}{2!}      \\
            \frac{1}{2!}    & 1                 & 0                 & \cdots & 0            & \frac{-1}{3!}      \\
            \frac{1}{3!}    & \frac{1}{2!}      & 1                 & \cdots & 0            & \frac{-1}{4!}      \\
            \vdots          & \vdots            & \vdots            & \ddots & \vdots       & \vdots             \\
            \frac{1}{(2n)!} & \frac{1}{(2n-1)!} & \frac{1}{(2n-2)!} & \cdots & \frac{1}{2!} & \frac{-1}{(2n+1)!}
        \end{vmatrix} \\
               & = (-1){}^{2n-1}
        \begin{vmatrix}
            \frac{-1}{2!}      & 1               & 0                 & 0                 & \cdots & 0            \\
            \frac{-1}{3!}      & \frac{1}{2!}    & 1                 & 0                 & \cdots & 0            \\
            \frac{-1}{4!}      & \frac{1}{3!}    & \frac{1}{2!}      & 1                 & \cdots & 0            \\
            \vdots             & \vdots          & \vdots            & \vdots            & \ddots & \vdots       \\
            \frac{-1}{(2n+1)!} & \frac{1}{(2n)!} & \frac{1}{(2n-1)!} & \frac{1}{(2n-2)!} & \cdots & \frac{1}{2!}
        \end{vmatrix} \\
               & =
        \begin{vmatrix}
            \frac{1}{2!}      & 1               & 0                 & 0                 & \cdots & 0            \\
            \frac{1}{3!}      & \frac{1}{2!}    & 1                 & 0                 & \cdots & 0            \\
            \frac{1}{4!}      & \frac{1}{3!}    & \frac{1}{2!}      & 1                 & \cdots & 0            \\
            \vdots            & \vdots          & \vdots            & \vdots            & \ddots & \vdots       \\
            \frac{1}{(2n+1)!} & \frac{1}{(2n)!} & \frac{1}{(2n-1)!} & \frac{1}{(2n-2)!} & \cdots & \frac{1}{2!}
        \end{vmatrix}  \\
    \end{align*}
    \endgroup{}
    \par Do đó
    \[
        B_{n} = (-1){}^{n-1}(2n)!
        \begin{vmatrix}
            \frac{1}{2!}      & 1               & 0                 & 0                 & \cdots & 0            \\
            \frac{1}{3!}      & \frac{1}{2!}    & 1                 & 0                 & \cdots & 0            \\
            \frac{1}{4!}      & \frac{1}{3!}    & \frac{1}{2!}      & 1                 & \cdots & 0            \\
            \vdots            & \vdots          & \vdots            & \vdots            & \ddots & \vdots       \\
            \frac{1}{(2n+1)!} & \frac{1}{(2n)!} & \frac{1}{(2n-1)!} & \frac{1}{(2n-2)!} & \cdots & \frac{1}{2!}
        \end{vmatrix}.
    \]

    \par Áp dụng công thức Cramer cho $2n-1$ phương trình tuyến tính đầu tiên:
    \begin{align*}
        b_{2n-1} & =
        \begin{vmatrix}
            1                 & 0                 & 0                 & \cdots & 0            & \frac{-1}{2!}    \\
            \frac{1}{2!}      & 1                 & 0                 & \cdots & 0            & \frac{-1}{3!}    \\
            \frac{1}{3!}      & \frac{1}{2!}      & 1                 & \cdots & 0            & \frac{-1}{4!}    \\
            \vdots            & \vdots            & \vdots            & \ddots & \vdots       & \vdots           \\
            \frac{1}{(2n-1)!} & \frac{1}{(2n-2)!} & \frac{1}{(2n-3)!} & \cdots & \frac{1}{2!} & \frac{-1}{(2n)!}
        \end{vmatrix} \\
                 & = -
        \begin{vmatrix}
            \frac{1}{2!}    & 1                 & 0                 & \cdots & 0            & 0            \\
            \frac{1}{3!}    & \frac{1}{2!}      & 1                 & \cdots & 0            & 0            \\
            \frac{1}{4!}    & \frac{1}{3!}      & \frac{1}{2!}      & \cdots & 0            & 0            \\
            \vdots          & \vdots            & \vdots            & \ddots & \vdots       & \vdots       \\
            \frac{1}{(2n)!} & \frac{1}{(2n-1)!} & \frac{1}{(2n-2)!} & \cdots & \frac{1}{3!} & \frac{1}{2!}
        \end{vmatrix}
    \end{align*}
    \begin{align*}
        \dfrac{x}{e^{x}-1}                                             & = 1 + b_{1}x + b_{2}x^{2} + b_{3}x^{3} + \cdots \\
        \dfrac{-x}{e^{-x}-1}                                           & = 1 - b_{1}x + b_{2}x^{2} - b_{3}x^{3} + \cdots \\
        \Longleftrightarrow \dfrac{x}{e^{x} - 1} + \dfrac{x}{e^{-x}-1} & = 2b_{1}x + 2b_{3}x^{3} + \cdots                \\
        \Longleftrightarrow -x                                         & = 2b_{1}x + 2b_{3}x^{3} + \cdots
    \end{align*}
    \par Đồng nhất hệ số của hai đa thức, ta được $b_{1} = \dfrac{-1}{2}$, $b_{3} = b_{5} = \cdots = b_{2n-1} = \cdots = 0$.
    \par Vậy với $n > 1$
    \[
        b_{2n-1} =
        \begin{vmatrix}
            \frac{1}{2!}    & 1                 & 0                 & \cdots & 0            & 0            \\
            \frac{1}{3!}    & \frac{1}{2!}      & 1                 & \cdots & 0            & 0            \\
            \frac{1}{4!}    & \frac{1}{3!}      & \frac{1}{2!}      & \cdots & 0            & 0            \\
            \vdots          & \vdots            & \vdots            & \ddots & \vdots       & \vdots       \\
            \frac{1}{(2n)!} & \frac{1}{(2n-1)!} & \frac{1}{(2n-2)!} & \cdots & \frac{1}{3!} & \frac{1}{2!}
        \end{vmatrix}
        = 0.
    \]
\end{proof}

% exercise 3.32
\begin{exercise}
    \par Diễn đạt hệ số $a_{n}$ trong khai triển
    \[
        e^{-x} = 1 - a_{1}x + a_{2}x^{2} - a_{3}x^{3} + \cdots ,
    \]
    \par như một định thức cỡ $n$, từ đó tính định thức thu được.
\end{exercise}

\begin{proof}[Lời giải]
    \begingroup{}
    \allowdisplaybreaks{}
    \begin{align*}
        e^{-x} & = 1 - a_{1}x + a_{2}x^{2} - a_{3}x^{3} + \cdots                                                                      \\
        1      & = (1 - a_{1}x + a_{2}x^{2} - a_{3}x^{3} + \cdots)\left(1 + x + \dfrac{x^{2}}{2!} + \dfrac{x^{3}}{3!} + \cdots\right) \\
    \end{align*}
    \endgroup{}
    \par Đồng nhất hệ số của các hạng tử $x$, $x^{2}$, $x^{3}$, \ldots, $x^{n}$, ta thu được hệ phương trình tuyến tính:
    \begin{align*}
        (-1){}^{1}a_{1} + \frac{1}{1!}                                                                              & = 0    \\
        (-1){}^{2}a_{2} + (-1){}^{1}a_{1}\frac{1}{1!} + \frac{1}{2!}                                                & = 0    \\
        (-1){}^{3}a_{3} + (-1){}^{2}a_{2}\frac{1}{1!} + (-1){}^{1}a_{1}\frac{1}{2!} + \frac{1}{3!}                  & = 0    \\
                                                                                                                    & \ddots \\
        (-1){}^{n}a_{n} + (-1){}^{n-1}a_{n-1}\frac{1}{1!} + \cdots + (-1){}^{1}a_{1}\frac{1}{(n-1)!} + \frac{1}{n!} & = 0
    \end{align*}
    \par $(-1){}^{1}a_{1}$, $(-1){}^{2}a_{2}$, \ldots, $(-1){}^{n}a_{n}$ là nghiệm của hệ phương tình tuyến tính:
    \begin{align*}
         & x_{1}                                                                                & = \frac{-1}{1!} \\
         & \frac{1}{1!}x_{1} + x_{2}                                                            & = \frac{-1}{2!} \\
         & \frac{1}{2!}x_{1} + \frac{1}{1!}x_{2} + x_{3}                                        & = \frac{-1}{3!} \\
         & \ddots                                                                               &                 \\
         & \frac{1}{(n-1)!}x_{1} + \frac{1}{(n-2)!}x_{2} + \cdots + \frac{1}{1!}x_{n-1} + x_{n} & = \frac{-1}{n!}
    \end{align*}
    \par Định thức của ma trận hệ số bằng 1. Áp dụng công thức Cramer:
    \begin{align*}
        (-1){}^{n}a_{n} = x_{n}   & =
        \begin{vmatrix}
            1                & 0                & 0                & \cdots & 0            & \frac{-1}{1!} \\
            \frac{1}{1!}     & 1                & 0                & \cdots & 0            & \frac{-1}{2!} \\
            \frac{1}{2!}     & \frac{1}{1!}     & 1                & \cdots & 0            & \frac{-1}{3!} \\
            \vdots           & \vdots           & \vdots           & \ddots & \vdots       & \vdots        \\
            \frac{1}{(n-1)!} & \frac{1}{(n-2)!} & \frac{1}{(n-3)!} & \cdots & \frac{1}{1!} & \frac{-1}{n!}
        \end{vmatrix} \\
                                  & = {(-1)}^{n-1}
        \begin{vmatrix}
            \frac{-1}{1!} & 1                & 0                & 0                & \cdots & 0            \\
            \frac{-1}{2!} & \frac{1}{1!}     & 1                & 0                & \cdots & 0            \\
            \frac{-1}{3!} & \frac{1}{2!}     & \frac{1}{1!}     & 1                & \cdots & 0            \\
            \vdots        & \vdots           & \vdots           & \vdots           & \ddots & \vdots       \\
            \frac{-1}{n!} & \frac{1}{(n-1)!} & \frac{1}{(n-2)!} & \frac{1}{(n-3)!} & \cdots & \frac{1}{1!}
        \end{vmatrix} \\
                                  & = {(-1)}^{n}
        \begin{vmatrix}
            \frac{1}{1!} & 1                & 0                & 0                & \cdots & 0            \\
            \frac{1}{2!} & \frac{1}{1!}     & 1                & 0                & \cdots & 0            \\
            \frac{1}{3!} & \frac{1}{2!}     & \frac{1}{1!}     & 1                & \cdots & 0            \\
            \vdots       & \vdots           & \vdots           & \vdots           & \ddots & \vdots       \\
            \frac{1}{n!} & \frac{1}{(n-1)!} & \frac{1}{(n-2)!} & \frac{1}{(n-3)!} & \cdots & \frac{1}{1!}
        \end{vmatrix}  \\
        \Longleftrightarrow a_{n} & =
        \begin{vmatrix}
            \frac{1}{1!} & 1                & 0                & 0                & \cdots & 0            \\
            \frac{1}{2!} & \frac{1}{1!}     & 1                & 0                & \cdots & 0            \\
            \frac{1}{3!} & \frac{1}{2!}     & \frac{1}{1!}     & 1                & \cdots & 0            \\
            \vdots       & \vdots           & \vdots           & \vdots           & \ddots & \vdots       \\
            \frac{1}{n!} & \frac{1}{(n-1)!} & \frac{1}{(n-2)!} & \frac{1}{(n-3)!} & \cdots & \frac{1}{1!}
        \end{vmatrix}.
    \end{align*}

    \par Áp dụng khai triển Taylor:
    \[
        e^{-x} = \sum^{+\infty}_{n=0}\frac{(-1){}^{n}x^{n}}{n!}
    \]
    \par Đồng nhất hệ số với $e^{-x} = 1 - a_{1}x + a_{2}x^{2} - a_{3}x^{3} + \cdots$, ta được:
    \[
        a_{n} =
        \begin{vmatrix}
            \frac{1}{1!} & 1                & 0                & 0                & \cdots & 0            \\
            \frac{1}{2!} & \frac{1}{1!}     & 1                & 0                & \cdots & 0            \\
            \frac{1}{3!} & \frac{1}{2!}     & \frac{1}{1!}     & 1                & \cdots & 0            \\
            \vdots       & \vdots           & \vdots           & \vdots           & \ddots & \vdots       \\
            \frac{1}{n!} & \frac{1}{(n-1)!} & \frac{1}{(n-2)!} & \frac{1}{(n-3)!} & \cdots & \frac{1}{1!}
        \end{vmatrix}
        = \frac{1}{n!}.
    \]
\end{proof}

% exercise 3.33
\begin{exercise}
    \par Không dùng ma trận của tự đồng cấu, hãy chứng minh trực tiếp rằng nếu $f$ là một tự đồng cấu của không gian vector hữu hạn chiều $V$ và $f^{*}$ là đồng cấu đối ngẫu của $f$, thì $\det(f^{*}) = \det(f)$. (Gợi ý: Xét định thức của ma trận ${\left(\dotprod{\alpha_{i},\xi_{j}}\right)}_{n\times n}$, trong đó $\alpha_{1},\ldots,\alpha_{n}\in V$, $\xi_{1},\ldots ,\xi_{n}\in V^{*}$.)
\end{exercise}

\begin{proof}
    \par $(\alpha_{1}, \alpha_{2}, \ldots, \alpha_{n})$ là một cơ sở của $V$.
    \par $(\xi_{1}, \xi_{2}, \ldots, \xi_{n})$ là một cơ sở đối ngẫu của $(\alpha_{1}, \alpha_{2}, \ldots, \alpha_{n})$.
    \par Ta sử dụng hai dạng đa tuyến tính thay phiên\footnote{Oversimplified: Hai ánh xạ này đa tuyến tính thay phiên bởi biểu thức mà chúng được định nghĩa theo là một định thức, và định thức có tính chất đa tuyến tính thay phiên (ở đây định thức được nhìn nhận là một hàm của các vector cột).} $\phi \in {\Lambda^{n}(V)}^{*}$, $\psi \in {\Lambda^{n}({V}^{*})}^{*}$ sau:
    \[
        \begin{split}
            \phi(v_{1}, v_{2}, \ldots, v_{n}) = \begin{vmatrix}
                \dotprod{v_{1}, \xi_{1}} & \dotprod{v_{1}, \xi_{2}} & \cdots & \dotprod{v_{1}, \xi_{n}} \\
                \dotprod{v_{2}, \xi_{1}} & \dotprod{v_{2}, \xi_{2}} & \cdots & \dotprod{v_{2}, \xi_{n}} \\
                \vdots                   & \vdots                   & \ddots & \vdots                   \\
                \dotprod{v_{n}, \xi_{1}} & \dotprod{v_{n}, \xi_{2}} & \cdots & \dotprod{v_{n}, \xi_{n}}
            \end{vmatrix}, \\
            \psi(\tau_{1}, \tau_{2}, \ldots, \tau_{n}) = \begin{vmatrix}
                \dotprod{\alpha_{1}, \tau_{1}} & \dotprod{\alpha_{1}, \tau_{2}} & \cdots & \dotprod{\alpha_{1}, \tau_{n}} \\
                \dotprod{\alpha_{2}, \tau_{1}} & \dotprod{\alpha_{2}, \tau_{2}} & \cdots & \dotprod{\alpha_{2}, \tau_{n}} \\
                \vdots                         & \vdots                         & \ddots & \vdots                         \\
                \dotprod{\alpha_{n}, \tau_{1}} & \dotprod{\alpha_{n}, \tau_{2}} & \cdots & \dotprod{\alpha_{n}, \tau_{n}}
            \end{vmatrix}.
        \end{split}
    \]
    \par Theo định nghĩa định thức của tự đồng cấu:
    \[
        \begin{split}
            \phi(f(\alpha_{1}), f(\alpha_{2}), \ldots, f(\alpha_{n})) = \det(f)\cdot\phi(\alpha_{1}, \alpha_{2}, \ldots, \alpha_{n}), \\
            \psi(f^{*}(\xi_{1}), f^{*}(\xi_{2}), \ldots, f^{*}(\xi_{n})) = \det(f^{*})\cdot\psi(\xi_{1}, \xi_{2}, \ldots, \xi_{n}).
        \end{split}
    \]
    \par Khai triển các đẳng thức trên, ta thu được:
    \begingroup{}
    \allowdisplaybreaks{}
    \begin{align*}
        \begin{vmatrix}
            \dotprod{f(\alpha_{1}),\xi_{1}} & \dotprod{f(\alpha_{1}),\xi_{2}} & \cdots & \dotprod{f(\alpha_{1}),\xi_{n}} \\
            \dotprod{f(\alpha_{2}),\xi_{1}} & \dotprod{f(\alpha_{2}),\xi_{2}} & \cdots & \dotprod{f(\alpha_{2}),\xi_{n}} \\
            \vdots                          & \vdots                          & \ddots & \vdots                          \\
            \dotprod{f(\alpha_{n}),\xi_{1}} & \dotprod{f(\alpha_{n}),\xi_{2}} & \cdots & \dotprod{f(\alpha_{n}),\xi_{n}} \\
        \end{vmatrix}
         & = \det(f)
        \begin{vmatrix}
            \dotprod{\alpha_{1},\xi_{1}} & \dotprod{\alpha_{1},\xi_{2}} & \cdots & \dotprod{\alpha_{1},\xi_{n}} \\
            \dotprod{\alpha_{2},\xi_{1}} & \dotprod{\alpha_{2},\xi_{2}} & \cdots & \dotprod{\alpha_{2},\xi_{n}} \\
            \vdots                       & \vdots                       & \ddots & \vdots                       \\
            \dotprod{\alpha_{n},\xi_{1}} & \dotprod{\alpha_{n},\xi_{2}} & \cdots & \dotprod{\alpha_{n},\xi_{n}} \\
        \end{vmatrix}, \\
        \begin{vmatrix}
            \dotprod{\alpha_{1},f^{*}(\xi_{1})} & \dotprod{\alpha_{1},f^{*}(\xi_{2})} & \cdots & \dotprod{\alpha_{1},f^{*}(\xi_{n})} \\
            \dotprod{\alpha_{2},f^{*}(\xi_{1})} & \dotprod{\alpha_{2},f^{*}(\xi_{2})} & \cdots & \dotprod{\alpha_{2},f^{*}(\xi_{n})} \\
            \vdots                              & \vdots                              & \ddots & \vdots                              \\
            \dotprod{\alpha_{n},f^{*}(\xi_{1})} & \dotprod{\alpha_{n},f^{*}(\xi_{2})} & \cdots & \dotprod{\alpha_{n},f^{*}(\xi_{n})} \\
        \end{vmatrix}
         & = \det(f^{*})
        \begin{vmatrix}
            \dotprod{\alpha_{1},\xi_{1}} & \dotprod{\alpha_{1},\xi_{2}} & \cdots & \dotprod{\alpha_{1},\xi_{n}} \\
            \dotprod{\alpha_{2},\xi_{1}} & \dotprod{\alpha_{2},\xi_{2}} & \cdots & \dotprod{\alpha_{2},\xi_{n}} \\
            \vdots                       & \vdots                       & \ddots & \vdots                       \\
            \dotprod{\alpha_{n},\xi_{1}} & \dotprod{\alpha_{n},\xi_{2}} & \cdots & \dotprod{\alpha_{n},\xi_{n}} \\
        \end{vmatrix}, \\
    \end{align*}
    \par Theo định nghĩa của đồng cấu tuyến tính đối ngẫu, $f^{*}(\varphi) = \varphi\circ f, \forall\varphi\in V^{*}$, suy ra:
    \[
        \det(f)
        \begin{vmatrix}
            \dotprod{\alpha_{1},\xi_{1}} & \dotprod{\alpha_{1},\xi_{2}} & \cdots & \dotprod{\alpha_{1},\xi_{n}} \\
            \dotprod{\alpha_{2},\xi_{1}} & \dotprod{\alpha_{2},\xi_{2}} & \cdots & \dotprod{\alpha_{2},\xi_{n}} \\
            \vdots                       & \vdots                       & \ddots & \vdots                       \\
            \dotprod{\alpha_{n},\xi_{1}} & \dotprod{\alpha_{n},\xi_{2}} & \cdots & \dotprod{\alpha_{n},\xi_{n}} \\
        \end{vmatrix}
        =
        \det(f^{*})
        \begin{vmatrix}
            \dotprod{\alpha_{1},\xi_{1}} & \dotprod{\alpha_{1},\xi_{2}} & \cdots & \dotprod{\alpha_{1},\xi_{n}} \\
            \dotprod{\alpha_{2},\xi_{1}} & \dotprod{\alpha_{2},\xi_{2}} & \cdots & \dotprod{\alpha_{2},\xi_{n}} \\
            \vdots                       & \vdots                       & \ddots & \vdots                       \\
            \dotprod{\alpha_{n},\xi_{1}} & \dotprod{\alpha_{n},\xi_{2}} & \cdots & \dotprod{\alpha_{n},\xi_{n}} \\
        \end{vmatrix}
    \]
    \par Bên cạnh đó, theo định nghĩa của cơ sở đối ngẫu, $\dotprod{\alpha_{i}, \xi_{j}} = \delta_{ij}, \forall i, j\in \{ 1, 2, \ldots, n \}$. Do đó ${(\dotprod{\alpha_{i}, \xi_{j}})}_{i\times j} = I_{n}$, và $\det{(\dotprod{\alpha_{i}, \xi_{j}})}_{i\times j} = 1$, kéo theo:
    \[
        \det(f) = \det(f^{*}).\qedhere
    \]
    \endgroup{}
\end{proof}

% exercise 3.34
\begin{exercise}
    \par Tính hạng của các ma trận sau đây bằng phương pháp biến đổi sơ cấp và phương pháp dùng định thức con:
    \begin{enumerate}[label = (\alph*)]
        \item $
                  \begin{pmatrix}
                      2 & -1 & 3 & -2 & 4 \\
                      4 & -2 & 5 & 1  & 7 \\
                      2 & -1 & 1 & 8  & 2
                  \end{pmatrix}
              $,
        \item $
                  \begin{pmatrix}
                      3 & -1 & 3  & 2 & 5  \\
                      5 & -3 & 2  & 3 & 4  \\
                      1 & -3 & -5 & 0 & -7 \\
                      7 & -5 & 1  & 4 & 1
                  \end{pmatrix}
              $.
    \end{enumerate}
\end{exercise}

\begin{proof}[Lời giải]
    \begin{enumerate}[label = (\alph*)]
        \item
              \begingroup{}
              \allowdisplaybreaks{}
              \begin{gather*}
                  \rank\begin{pmatrix}
                      2 & -1 & 3 & -2 & 4 \\
                      4 & -2 & 5 & 1  & 7 \\
                      2 & -1 & 1 & 8  & 2
                  \end{pmatrix}
                  =\rank\begin{pmatrix}
                      2 & -1 & 3  & -2 & 4  \\
                      0 & 0  & -1 & 5  & -1 \\
                      0 & 0  & -2 & 10 & -2
                  \end{pmatrix}
                  =\rank\begin{pmatrix}
                      2 & -1 & 3  & -2 & 4  \\
                      0 & 0  & -1 & 5  & -1 \\
                      0 & 0  & 0  & 0  & 0
                  \end{pmatrix}
                  = 2.
              \end{gather*}
              \endgroup{}
        \item
              \begingroup{}
              \allowdisplaybreaks{}
              \begin{gather*}
                  \rank\begin{pmatrix}
                      3 & -1 & 3  & 2 & 5  \\
                      5 & -3 & 2  & 3 & 4  \\
                      1 & -3 & -5 & 0 & -7 \\
                      7 & -5 & 1  & 4 & 1
                  \end{pmatrix} =
                  \rank\begin{pmatrix}
                      1 & -3 & -5 & 0 & -7 \\
                      3 & -1 & 3  & 2 & 5  \\
                      5 & -3 & 2  & 3 & 4  \\
                      7 & -5 & 1  & 4 & 1
                  \end{pmatrix} =
                  \rank\begin{pmatrix}
                      1 & -3 & -5 & 0 & -7 \\
                      0 & 8  & 18 & 2 & 26 \\
                      0 & 12 & 27 & 3 & 39 \\
                      0 & 16 & 36 & 4 & 50
                  \end{pmatrix} \\
                  =\rank\begin{pmatrix}
                      1 & -3 & -5 & 0 & -7 \\
                      0 & 4  & 9  & 1 & 13 \\
                      0 & 4  & 9  & 1 & 13 \\
                      0 & 8  & 18 & 2 & 25
                  \end{pmatrix}
                  =\rank\begin{pmatrix}
                      1 & -3 & -5 & 0 & -7 \\
                      0 & 4  & 9  & 1 & 13 \\
                      0 & 0  & 0  & 0 & 0  \\
                      0 & 0  & 0  & 0 & -1
                  \end{pmatrix}
                  = 3.
              \end{gather*}
              \endgroup{}
    \end{enumerate}
\end{proof}

% exercise 3.35
\begin{exercise}
    \par Tìm giá trị của $\lambda$ sao cho ma trận sau đây có hạng thấp nhất
    \[
        \begin{pmatrix}
            3       & 1 & 1  & 4 \\
            \lambda & 4 & 10 & 1 \\
            1       & 7 & 17 & 3 \\
            2       & 2 & 4  & 3
        \end{pmatrix}.
    \]
\end{exercise}

\begin{proof}[Lời giải]
    \begingroup{}
    \allowdisplaybreaks{}
    \begin{gather*}
        \rank\begin{pmatrix}
            3       & 1 & 1  & 4 \\
            \lambda & 4 & 10 & 1 \\
            1       & 7 & 17 & 3 \\
            2       & 2 & 4  & 3
        \end{pmatrix}
        =\rank\begin{pmatrix}
            \lambda & 4 & 10 & 1 \\
            1       & 7 & 17 & 3 \\
            2       & 2 & 4  & 3 \\
            3       & 1 & 1  & 4 \\
        \end{pmatrix}
        =\rank\begin{pmatrix}
            \lambda & 4   & 10  & 1  \\
            1       & 7   & 17  & 3  \\
            0       & -12 & -30 & -3 \\
            0       & -20 & -50 & -5 \\
        \end{pmatrix} \\
        =\rank\begin{pmatrix}
            \lambda & 4 & 10 & 1 \\
            1       & 7 & 17 & 3 \\
            0       & 4 & 10 & 1 \\
            0       & 4 & 10 & 1 \\
        \end{pmatrix}
        =\rank\begin{pmatrix}
            \lambda & 0 & 0  & 0 \\
            1       & 7 & 17 & 3 \\
            0       & 4 & 10 & 1 \\
            0       & 0 & 0  & 0 \\
        \end{pmatrix}.
    \end{gather*}
    \endgroup{}
    \par Để ma trận trên có hạng thấp nhất, $\lambda = 0$.
\end{proof}

% exercise 3.36
\begin{exercise}
    \par Tìm hạng của ma trận sau đây như một hàm phụ thuộc $\lambda$:
    \[
        \begin{pmatrix}
            1 & \lambda & -1      & 2 \\
            2 & -1      & \lambda & 5 \\
            1 & 10      & -6      & 1
        \end{pmatrix}.
    \]
\end{exercise}

\begin{proof}[Lời giải]
    \par Ma trận trên có định thức con $\begin{vmatrix}2 & -1 \\ 1 & 10\end{vmatrix} = 21 \ne 0$.
    \par Do đó hạng của ma trận trên lớn hơn hoặc bằng 2.
    \par Tính tất cả các định thức con cỡ 3:
    \begin{align*}
         & \begin{vmatrix}
               1 & \lambda & -1      \\
               2 & -1      & \lambda \\
               1 & 10      & -6
           \end{vmatrix} = (\lambda - 3)(\lambda + 5)  \\
         & \begin{vmatrix}
               1 & \lambda & 2 \\
               2 & -1      & 5 \\
               1 & 10      & 1
           \end{vmatrix} = 3(\lambda - 3)              \\
         & \begin{vmatrix}
               1 & -1      & 2 \\
               2 & \lambda & 5 \\
               1 & -6      & 1
           \end{vmatrix} = -(\lambda - 3)              \\
         & \begin{vmatrix}
               \lambda & -1      & 2 \\
               -1      & \lambda & 5 \\
               10      & -6      & 1
           \end{vmatrix} = (\lambda - 3)(\lambda + 13)
    \end{align*}
    \par Vậy hạng của ma trận trên bằng 3 nếu $\lambda \ne 3$, bằng 2 nếu $\lambda = 3$.
\end{proof}

% exercise 3.37
\begin{exercise}
    \par Chứng minh rằng nếu hạng của một ma trận bằng $r$ thì mỗi định thức con nằm trên giao của bất kì $r$ hàng độc lập tuyến tính và $r$ cột độc lập tuyến tính của ma trận đó đều khác 0.
\end{exercise}

\begin{proof}
    \par Giả sử ma trận $A$ đang xét có $m$ hàng và $n$ cột.
    \par Việc thay đổi thứ tự các hàng hay thay đổi thứ tự các cột không làm ảnh hưởng đến sự độc lập tuyến tính/phụ thuộc tuyến tính của các vector hàng, vector cột mà chỉ thay đổi thứ tự các vector đó, cũng như thứ tự của các yếu tố trong các vector đó.
    \par Do đó, không mất tính tổng quát, giả sử $r$ hàng đầu tiên của $A$ độc lập tuyến tính và $r$ cột đầu tiên của $A$ độc lập tuyến tính.
    \[
        A =
        \begin{pmatrix}
            a_{11}     & a_{12}     & \cdots & a_{1r}     & a_{1(r+1)}     & \cdots & a_{1n}     \\
            a_{21}     & a_{22}     & \cdots & a_{2r}     & a_{2(r+1)}     & \cdots & a_{2n}     \\
            \vdots     & \vdots     & \ddots & \vdots     & \vdots         & \ddots & \vdots     \\
            a_{r1}     & a_{r2}     & \cdots & a_{rr}     & a_{r(r+1)}     & \cdots & a_{rn}     \\
            a_{(r+1)1} & a_{(r+1)2} & \cdots & a_{(r+1)r} & a_{(r+1)(r+1)} & \cdots & a_{(r+1)n} \\
            \vdots     & \vdots     & \ddots & \vdots     & \vdots         & \ddots & \vdots     \\
            a_{m1}     & a_{m2}     & \cdots & a_{mr}     & a_{m(r+1)}     & \cdots & a_{mn}
        \end{pmatrix}
    \]
    \par Do $\rank(A) = r$ và $r$ hàng đầu tiên của $A$ độc lập tuyến tính nên hàng cột $r+1$, \ldots, $n$ đều biểu thị tuyến tính được theo $r$ hàng đầu tiên, ta đặt:
    \begin{align*}
        (a_{(r+1)1}, a_{(r+1)2}, \ldots, a_{(r+1)r}, \ldots, a_{(r+1)n}) & = \sum^{r}_{k=1}b_{k}^{(r+1)}(a_{k1}, a_{k2}, \ldots, a_{kr}, \ldots, a_{kn}) \\
        (a_{(r+2)1}, a_{(r+2)2}, \ldots, a_{(r+2)r}, \ldots, a_{(r+2)n}) & = \sum^{r}_{k=1}b_{k}^{(r+2)}(a_{k1}, a_{k2}, \ldots, a_{kr}, \ldots, a_{kn}) \\
                                                                         & \ddots                                                                        \\
        (a_{m1}, a_{m2}, \ldots, a_{mr}, \ldots, a_{mn})                 & = \sum^{r}_{k=1}b_{k}^{(m)}(a_{k1}, a_{k2}, \ldots, a_{kr}, \ldots, a_{kn})   \\
    \end{align*}
    \par Từ các biểu thị tuyến tính trên, ta suy ra:
    \[
        \begin{cases}
            (a_{(r+1)1}, a_{(r+1)2}, \ldots, a_{(r+1)r}, 0, \ldots, 0) & = \sum^{r}_{k=1}b_{k}^{(r+1)}(a_{k1}, a_{k2}, \ldots, a_{kr}, 0, \ldots, 0) \\
            (a_{(r+2)1}, a_{(r+2)2}, \ldots, a_{(r+2)r}, 0, \ldots, 0) & = \sum^{r}_{k=1}b_{k}^{(r+2)}(a_{k1}, a_{k2}, \ldots, a_{kr}, 0, \ldots, 0) \\
                                                                       & \ddots                                                                      \\
            (a_{m1}, a_{m2}, \ldots, a_{mr}, 0, \ldots, 0)             & = \sum^{r}_{k=1}b_{k}^{(m)}(a_{k1}, a_{k2}, \ldots, a_{kr}, 0, \ldots, 0)   \\
        \end{cases}
        \tag{$\star$}
    \]
    \par Do $\rank(A) = r$ và $r$ cột đầu tiên của $A$ độc lập tuyến tính nên từng cột $r+1$, \ldots, $n$ đều biểu thị tuyến tính được theo $r$ cột đầu tiên, do đó nếu xóa các cột $r+1$, \ldots, $n$ thì hạng của $A$ không đổi. Nói cách khác:
    \[
        \rank
        \begin{pmatrix}
            a_{11}     & a_{12}     & \cdots & a_{1r}     & a_{1(r+1)}     & \cdots & a_{1n}     \\
            a_{21}     & a_{22}     & \cdots & a_{2r}     & a_{2(r+1)}     & \cdots & a_{2n}     \\
            \vdots     & \vdots     & \ddots & \vdots     & \vdots         & \ddots & \vdots     \\
            a_{r1}     & a_{r2}     & \cdots & a_{rr}     & a_{r(r+1)}     & \cdots & a_{rn}     \\
            a_{(r+1)1} & a_{(r+1)2} & \cdots & a_{(r+1)r} & a_{(r+1)(r+1)} & \cdots & a_{(r+1)n} \\
            \vdots     & \vdots     & \ddots & \vdots     & \vdots         & \ddots & \vdots     \\
            a_{m1}     & a_{m2}     & \cdots & a_{mr}     & a_{m(r+1)}     & \cdots & a_{mn}
        \end{pmatrix}
        =
        \rank
        \begin{pmatrix}
            a_{11}     & a_{12}     & \cdots & a_{1r}     & 0      & \cdots & 0      \\
            a_{21}     & a_{22}     & \cdots & a_{2r}     & 0      & \cdots & 0      \\
            \vdots     & \vdots     & \ddots & \vdots     & \vdots & \ddots & \vdots \\
            a_{r1}     & a_{r2}     & \cdots & a_{rr}     & 0      & \cdots & 0      \\
            a_{(r+1)1} & a_{(r+1)2} & \cdots & a_{(r+1)r} & 0      & \cdots & 0      \\
            \vdots     & \vdots     & \ddots & \vdots     & \vdots & \ddots & \vdots \\
            a_{m1}     & a_{m2}     & \cdots & a_{mr}     & 0      & \cdots & 0
        \end{pmatrix}
    \]
    \par Ma trận mới có các vector cột thứ $r+1$, \ldots, $n$ bằng không.
    \par Mặt khác, theo các đẳng thức $(\star)$, ta có thể xóa các hàng $r+1$, \ldots, $n$ của ma trận mới mà vẫn bảo toàn hạng, điều này kéo theo:
    \[
        \rank
        \begin{pmatrix}
            a_{11} & a_{12} & \cdots & a_{1r} & 0      & \cdots & 0      \\
            a_{21} & a_{22} & \cdots & a_{2r} & 0      & \cdots & 0      \\
            \vdots & \vdots & \ddots & \vdots & \vdots & \ddots & \vdots \\
            a_{r1} & a_{r2} & \cdots & a_{rr} & 0      & \cdots & 0      \\
            0      & 0      & \cdots & 0      & 0      & \cdots & 0      \\
            \vdots & \vdots & \ddots & \vdots & \vdots & \ddots & \vdots \\
            0      & 0      & \cdots & 0      & 0      & \cdots & 0
        \end{pmatrix}
        = \rank(A) = r.
    \]
    \par Giả sử phản chứng rằng:
    \[
        \begin{vmatrix}
            a_{11} & a_{12} & \cdots & a_{1r} \\
            a_{21} & a_{22} & \cdots & a_{2r} \\
            \vdots & \vdots & \ddots & \vdots \\
            a_{r1} & a_{r2} & \cdots & a_{rr}
        \end{vmatrix} = 0.
    \]
    \par Khi đó, mọi định thức con cỡ $r$ của ma trận trên đều bằng không, kéo theo $\rank(A) < r$.
    \par Điều mâu thuẫn này chứng tỏ giả sử phản chứng là sai. Như vậy, định thức con nằm trên giao của $r$ hàng độc lập tuyến tính và $r$ cột độc lập tuyến tính khác không.
\end{proof}

% exercise 3.38
\begin{exercise}
    \par Cho $A$ là một ma trận vuông cỡ $n > 1$ và $\tilde{A}$ là ma trận phụ hợp (gồm những phần bù đại số của các yếu tố) của $A$. Hãy xác định $\rank\tilde{A}$ như một hàm của $\rank A$.
\end{exercise}

\begin{proof}
    \par Đặt $\tilde{a}_{ij}$ là phần bù đại số của $a_{ij}$ trong ma trận $A$. Khi đó, theo định nghĩa của ma trận phụ hợp:
    \[
        \tilde{A} =
        \begin{pmatrix}
            \tilde{a}_{11} & \tilde{a}_{21} & \cdots & \tilde{a}_{n1} \\
            \tilde{a}_{12} & \tilde{a}_{22} & \cdots & \tilde{a}_{n2} \\
            \vdots         & \vdots         & \ddots & \vdots         \\
            \tilde{a}_{1n} & \tilde{a}_{2n} & \cdots & \tilde{a}_{nn}
        \end{pmatrix}
    \]
    \begin{enumerate}[label = \textbf{Trường hợp \arabic*.},itemindent=2cm]
        \item $\rank(A) = n$.
              \par Lúc này, $\det(A)\ne 0$. Mà $\det(A)\cdot\det(\tilde{A}) = 1$ nên $\det(\tilde{A})$, dẫn tới $\rank(\tilde{A}) = n$.
        \item $\rank(A) < n - 1$.
              \par Lúc này, mọi định thức con cỡ $(n-1)$ của $A$ đều bằng không, do đó $\tilde{A}$ là ma trận không. Như vậy $\rank(\tilde{A}) = 0$.
        \item $\rank(A) = n - 1$.
              \par Vì $\rank(A) = n - 1$ nên $A$ sẽ có ít nhất một định thức con cỡ $(n-1)$ với giá trị khác không. Điều này đảm bảo $\rank(\tilde{A})\ge 1$.
              \par Nhận xét rằng:
              \begin{itemize}
                  \item Đổi chỗ hai cột $j$ và $j'$ của $A$ thì ma trận phụ hợp mới là ma trận phụ hợp cũ sau khi đổi chỗ hai hàng $j$ và $j'$.
                  \item Đổi chỗ hai hàng $i$ và $i'$ của $A$ thì ma trận phụ hợp mới là ma trận phụ hợp cũ sau khi đổi chỗ hai cột $i$ và $i'$.
              \end{itemize}
              \par Như vậy, không mất tính tổng quát, giả sử $(n-1)$ cột đầu tiên của $A$ độc lập tuyến tính và $(n-1)$ hàng đầu tiên của $A$ độc lập tuyến tính.
              \par Khi đó, $\tilde{a}_{nn}\ne 0$ và trong ma trận $A$, hàng cuối cùng biểu thị tuyến tính được duy nhất theo $(n-1)$ hàng đầu tiên.
              \par Đặt $\alpha'_{n} = \lambda_{1}\alpha'_{1} + \lambda_{2}\alpha'_{2} + \cdots + \lambda_{n-1}\alpha'_{n-1}$, trong đó $\alpha'_{k}$ và vector hàng thứ $k$ của $A$, $\lambda_{i}$ là các vô hướng thuộc trường $\mathbb{F}$.
              \[
                  \tilde{A} =
                  \begin{pmatrix}
                      \tilde{a}_{11} & \tilde{a}_{21} & \cdots & \tilde{a}_{n1} \\
                      \tilde{a}_{12} & \tilde{a}_{22} & \cdots & \tilde{a}_{n2} \\
                      \vdots         & \vdots         & \ddots & \vdots         \\
                      \tilde{a}_{1n} & \tilde{a}_{2n} & \cdots & \tilde{a}_{nn}
                  \end{pmatrix}.
              \]
              \begingroup{}
              \allowdisplaybreaks{}
              \begin{align*}
                  \tilde{a}_{n\ell} & = (-1){}^{n+\ell}
                  \begin{vmatrix}
                      a_{11}     & \cdots & a_{1(\ell-1)}     & a_{1(\ell+1)}     & \cdots & a_{1n}     \\
                      \vdots     & \ddots & \vdots            & \vdots            & \ddots & \vdots     \\
                      a_{(n-1)1} & \cdots & a_{(n-1)(\ell-1)} & a_{(n-1)(\ell+1)} & \cdots & a_{(n-1)n}
                  \end{vmatrix}                             \\
                  \tilde{a}_{k\ell} & = (-1){}^{k+\ell}
                  \begin{vmatrix}
                      a_{11}     & \cdots & a_{1(\ell-1)}     & a_{1(\ell+1)}     & \cdots & a_{1n}     \\
                      \vdots     & \ddots & \vdots            & \vdots            & \ddots & \vdots     \\
                      a_{(k-1)1} & \cdots & a_{(k-1)(\ell-1)} & a_{(k-1)(\ell+1)} & \cdots & a_{(k-1)n} \\
                      a_{(k+1)1} & \cdots & a_{(k+1)(\ell-1)} & a_{(k+1)(\ell+1)} & \cdots & a_{(k+1)n} \\
                      \vdots     & \ddots & \vdots            & \vdots            & \ddots & \vdots     \\
                      a_{n1}     & \cdots & a_{n(\ell-1)}     & a_{n(\ell+1)}     & \cdots & a_{nn}
                  \end{vmatrix}                             \\
                                    & = (-1){}^{k+\ell}
                  \begin{vmatrix}
                      a_{11}            & \cdots & a_{1(\ell-1)}            & a_{1(\ell+1)}            & \cdots & a_{1n}            \\
                      \vdots            & \ddots & \vdots                   & \vdots                   & \ddots & \vdots            \\
                      a_{(k-1)1}        & \cdots & a_{(k-1)(\ell-1)}        & a_{(k-1)(\ell+1)}        & \cdots & a_{(k-1)n}        \\
                      a_{(k+1)1}        & \cdots & a_{(k+1)(\ell-1)}        & a_{(k+1)(\ell+1)}        & \cdots & a_{(k+1)n}        \\
                      \vdots            & \ddots & \vdots                   & \vdots                   & \ddots & \vdots            \\
                      \lambda_{k}a_{k1} & \cdots & \lambda_{k}a_{k(\ell-1)} & \lambda_{k}a_{k(\ell+1)} & \cdots & \lambda_{k}a_{kn}
                  \end{vmatrix} \\
                                    & = (-1){}^{n+\ell-1}\lambda_{k}
                  \begin{vmatrix}
                      a_{11}     & \cdots & a_{1(\ell-1)}     & a_{1(\ell+1)}     & \cdots & a_{1n}     \\
                      \vdots     & \ddots & \vdots            & \vdots            & \ddots & \vdots     \\
                      a_{(k-1)1} & \cdots & a_{(k-1)(\ell-1)} & a_{(k-1)(\ell+1)} & \cdots & a_{(k-1)n} \\
                      a_{k1}     & \cdots & a_{k(\ell-1)}     & a_{k(\ell+1)}     & \cdots & a_{kn}     \\
                      a_{(k+1)1} & \cdots & a_{(k+1)(\ell-1)} & a_{(k+1)(\ell+1)} & \cdots & a_{(k+1)n} \\
                      \vdots     & \ddots & \vdots            & \vdots            & \ddots & \vdots     \\
                      a_{(n-1)1} & \cdots & a_{(n-1)(\ell-1)} & a_{(n-1)(\ell+1)} & \cdots & a_{(n-1)n}
                  \end{vmatrix}, \quad\forall 1\le\ell\le n
              \end{align*}
              \endgroup{}
              \par Như vậy, trong ma trận phụ hợp $\tilde{A}$, cột thứ $k$ bằng cột thứ $n$ nhân với $\lambda_{k}$, tức là từng cột trong $(n-1)$ cột đầu tiên của $\tilde{A}$ đều biểu thị tuyến tính được theo cột thứ $n$.
              \par Mà $\tilde{a}_{nn}\ne 0$ nên cột thứ $n$ của ma trận phụ hợp khác không.
              \par Hai điều nêu trên kéo theo $\rank(\tilde{A}) = 1$.
    \end{enumerate}

    \par Vậy, với $A\in M(n\times n,\mathbb{F})$, $n > 1$:
    \begin{align*}
        \rank(A) = n   & \Longrightarrow \rank(\tilde{A}) = n, \\
        \rank(A) = n-1 & \Longrightarrow \rank(\tilde{A}) = 1, \\
        \rank(A) < n-1 & \Longrightarrow \rank(\tilde{A}) = 0.
    \end{align*}
\end{proof}

% exercise 3.39
\begin{exercise}
    \par Chứng minh rằng nếu các vector
    \[
        \alpha_{i} = (a_{i1}, a_{i2}, \ldots, a_{in})\in\mathbb{R}_{n}\quad (i = 1, 2, \ldots, s; s\le n),
    \]
    \par thỏa mãn điều kiện $\abs{a_{jj}} > \sum_{i\ne j}\abs{a_{ij}}$, thì chúng độc lập tuyến tính.
\end{exercise}

\par\MakeUppercase{Chưa làm được}.

\begin{proof}
\end{proof}

% exercise 3.40
\begin{exercise}\label{chapter3:rank-of-sum}
    \par Chứng minh rằng nếu $A$ và $B$ là các ma trận cùng số hàng và số cột thì
    \[
        \rank(A + B)\le \rank(A) + \rank(B).
    \]
\end{exercise}

\begin{proof}
    \par $\alpha_{1}$, \ldots, $\alpha_{n}$ là các vector cột của $A$.
    \par $\beta_{1}$, \ldots, $\beta_{n}$ là các vector cột của $B$.
    \par Như vậy, các vector cột của $(A + B)$ là $\alpha_{1} + \beta_{1}$, \ldots, $\alpha_{n} + \beta_{n}$.
    \par Đặt $V_{A} = \text{span}(\alpha_{1}, \ldots, \alpha_{n})$, $V_{B} = \text{span}(\beta_{1}, \ldots, \beta_{n})$ và $V_{A+B} = \text{span}(\alpha_{1}+\beta_{1}, \ldots, \alpha_{n} + \beta_{n})$.
    \[
        \gamma = \underbrace{\sum^{n}_{i=1}c_{i}(\alpha_{i} + \beta_{i})}_{\in V_{A+B}} = \underbrace{\sum^{n}_{i=1}c_{i}\alpha_{i}}_{\in V_{A}} + \underbrace{\sum^{n}_{i=1}c_{i}\beta_{i}}_{\in V_{B}} \in V_{A} + V_{B}.
    \]
    \par Do đó $V_{A+B}$ là không gian con của $V_{A} + V_{B}$.
    \par $\dim V_{A+B}\le \dim(V_{A} + V_{B}) = \dim(V_{A}) + \dim(V_{B}) - \dim(V_{A}\cap V_{B})\le \dim(V_{A}) + \dim(V_{B})$.
    \par Bất đẳng thức $\dim V_{A+B} \le \dim{V_{A}} + \dim{V_{B}}$ tương đương với $\rank(A + B)\le \rank(A) + \rank(B)$.
\end{proof}

% exercise 3.41
\begin{exercise}
    \par Chứng minh rằng mỗi ma trận có hạng $r$ có thể viết thành tổng của $r$ ma trận có hạng 1, nhưng không thể viết thành tổng của một số ít hơn $r$ ma trận có hạng 1.
\end{exercise}

\begin{proof}
    \par Giả sử ma trận $A$ có $n$ cột và các vector cột $\alpha_{1}$, \ldots, $\alpha_{r}$ của $A$ độc lập tuyến tính cực đại.
    \par Khi đó, từng vector cột $\alpha_{r+1}$, \ldots, $\alpha_{n}$ của $A$ có thể biểu thị tuyến tính theo $\alpha_{1}$, \ldots, $\alpha_{r}$. Ta đặt:
    \[
        \begin{cases}
            \alpha_{r+1} = a_{1}^{(r+1)}\alpha_{1} + \cdots + a_{r}^{(r+1)}\alpha_{r}, \\
            \alpha_{r+2} = a_{1}^{(r+2)}\alpha_{1} + \cdots + a_{r}^{(r+2)}\alpha_{r}, \\
            \ddots                                                                     \\
            \alpha_{n} = a_{1}^{(n)}\alpha_{1} + \cdots + a_{r}^{(n)}\alpha_{r}.
        \end{cases}
    \]
    \par Dựa vào những biểu thị tuyến tính này, ta có thể tách $A$ thành tổng của $r$ ma trận như sau:
    \begin{align*}
        A = \begin{pmatrix}
                \alpha_{1} & \cdots & \alpha_{r} & \alpha_{r+1} & \cdots & \alpha_{n}
            \end{pmatrix}
         & =
        \begin{pmatrix}
            \alpha_{1} & \cdots & \alpha_{r} & \sum^{r}_{i=1}a_{i}^{(r+1)}\alpha_{i} & \cdots & \sum^{r}_{i=1}a_{i}^{(n)}\alpha_{i}
        \end{pmatrix} \\
         & =
        \begin{pmatrix}
            \alpha_{1} & \cdots & 0 & a_{1}^{(r+1)}\alpha_{1} & \cdots & a_{1}^{(n)}\alpha_{1}
        \end{pmatrix}                                      \\
         & \phantom{=} + \cdots                                                                                                 \\
         & \phantom{=} + \begin{pmatrix}
                             0 & \cdots & \alpha_{r} & a_{r}^{(r+1)}\alpha_{r} & \cdots & a_{r}^{(n)}\alpha_{r}
                         \end{pmatrix}
    \end{align*}
    \par Như vậy $A$ có thể viết thành tổng của $r$ ma trận có hạng 1.
    \bigskip
    \par Giả sử phản chứng rằng $A$ có thể viết thành tổng của $s$ ma trận $A_{i}$, $i = \overline{1,s}$ có hạng 1, trong đó $s < r$.
    \par Theo bài toán~\ref{chapter3:rank-of-sum}, $\rank(A)\le \sum^{s}_{i=1}\rank(A_{i}) = s < r$.
    \par Bất đẳng thức trên là sai vì $\rank(A) = r$.
    \par Vậy $A$ không thể viết thành tổng của ít hơn $r$ ma trận có hạng 1.
\end{proof}

% exercise 3.42
\begin{exercise}
    \par Chứng minh bất đẳng thức Sylvester cho các ma trận vuông cỡ $n$ bất kì $A$ và $B$:
    \[
        \rank(A) + \rank(B) - n \le \rank(AB) \le \min\{ \rank(A), \rank(B) \}.
    \]
\end{exercise}

\begin{proof}
    \par $\rank(A) = a$, $\rank(B) = b$.
    \par Nhận xét rằng:
    \begin{enumerate}[label = (\roman*)]
        \item Nếu đổi chỗ hai hàng của $A$ thì hai hàng tương ứng của $AB$ đổi chỗ,
        \item Nếu đổi chỗ hai cột của $B$ thì hai cột tương ứng của $AB$ đổi chỗ,
        \item Nếu cộng một hàng của $A$ với tổ hợp tuyến tính của các hàng còn lại thì hàng tương ứng của $AB$ cũng được cộng với tổ hợp tuyến tính của các hàng tương ứng còn lại (cùng hệ số tổ hợp tuyến tính),
        \item Nếu cộng một cột của $B$ với tổ hợp tuyến tính của các hàng còn lại thì cột tương ứng của $AB$ cũng được cộng với tổ hợp tuyến tính của các cột tương ứng còn lại (cùng hệ số tổ hợp tuyến tính).
        \item Nhân một hàng của $A$ với một vô hướng $\lambda$ khác không thì hàng tương ứng của $AB$ cũng được nhân với $\lambda$.
        \item Nhân một cột của $B$ với một vô hướng $\lambda$ khác không thì cột tương ứng của $AB$ cũng được nhân với $\lambda$.
    \end{enumerate}
    \par Do đó, nếu thực hiện các phép biến đổi như trên, $\rank(A)$, $\rank(B)$, $\rank(AB)$ không đổi.
    \bigskip
    \par Vì những lý do trên, không mất tính tổng quát, ta có thể giả sử:
    \begin{itemize}
        \item $a$ hàng đầu tiên của $A$ độc lập tuyến tính (i)
        \item $b$ cột đầu tiên của $B$ độc lập tuyến tính (ii)
        \item $n - a$ hàng cuối của $A$ là các vector không (iii)
        \item $n - b$ cột cuối của $B$ là các vector không (iv)
    \end{itemize}
    \par $\alpha_{1}^{T}$, $\alpha_{2}^{T}$, \ldots, $\alpha_{a}^{T}$ là $a$ vector hàng đầu tiên của $A$.
    \par $\beta_{1}$, $\beta_{2}$, \ldots, $\beta_{b}$ là $b$ vector cột đầu tiên của $B$.
    \[
        \underbrace{\begin{pmatrix}
                \alpha_{1}^{T} \\
                \alpha_{2}^{T} \\
                \vdots         \\
                \alpha_{a}^{T} \\
                0              \\
                \vdots         \\
                0
            \end{pmatrix}}_{A}
        \underbrace{\begin{pmatrix}
                \beta_{1} & \beta_{2} & \cdots & \beta_{b} & 0 & \cdots & 0
            \end{pmatrix}}_{B}
        =
        \underbrace{\begin{pmatrix}
                \alpha_{1}^{T}\beta_{1} & \cdots & \alpha_{1}^{T}\beta_{b} & 0      & \cdots & 0      \\
                \vdots                  & \ddots & \vdots                  & \vdots & \ddots & \vdots \\
                \alpha_{a}^{T}\beta_{1} & \cdots & \alpha_{a}^{T}\beta_{b} & 0      & \cdots & 0      \\
                0                       & \cdots & 0                       & 0      & \cdots & 0      \\
                \vdots                  & \ddots & \vdots                  & \vdots & \ddots & \vdots \\
                0                       & \cdots & 0                       & 0      & \cdots & 0
            \end{pmatrix}}_{AB}
    \]
    \par Do đó, $\rank(AB)\le \min\{ a, b \} = \min\{ \rank(A), \rank(B) \}$.
    \bigskip
    \par Nếu $A = 0$ thì $\rank(A) + \rank(B) - n = \rank(B) - n\le 0 = \rank(AB)$.
    \par Ngược lại:
    \par Ta sử dụng các phép biến đổi sơ cấp khác để đưa ma trận $A$ về dạng đơn giản hơn:
    \begin{enumerate}[label = (\roman*)]
        \setcounter{enumi}{6}
        \item Nhân một \textit{cột} $i$, $j$ của $A$ với vô hướng $c\ne 0$ và \textit{hàng} $i$, $j$ của $B$ với $c^{-1}$. Biến đổi này không làm thay đổi $\rank(A)$, $\rank(B)$ và không làm thay đổi $AB$.
        \item Đổi chỗ hai \textit{cột} $i$, $j$ của $A$ và đổi chỗ hai \textit{hàng} $i$, $j$ của $B$. Biến đổi này không làm thay đổi $\rank(A)$, $\rank(B)$ và không làm thay đổi $AB$.
        \item Cộng thêm cột $j$ vào cột $i$ của $A$ và trừ hàng $i$ khỏi hàng $j$ của $B$. Biến đổi này không làm thay đổi $\rank(A)$, $\rank(B)$ và không làm thay đổi $AB$.
    \end{enumerate}
    \par Như vậy, cả chín phép biến đổi trên sẽ bảo toàn $\rank(A)$, $\rank(B)$, $\rank(AB)$. Ta thực hiện các biến đổi sau, với ma trận $A$ có $a$ hàng đầu tiên độc lập tuyến tính và $n - a$ hàng còn lại bằng không:
    \begin{itemize}
        \item Tồn tại $a_{ij}\ne 0$ (vì $A\ne 0$). Ta đổi chỗ hàng 1 và hàng $i$ của $A$.
        \item Đổi chỗ cột 1 và cột $j$ của $A$, đồng thời đổi chỗ hàng 1 và hàng $j$ của $B$.
        \item Nhân hàng thứ 1 của $A$ với một vô hướng khác không, sao cho yếu tố hàng 1 cột 1 bằng 1.
        \item Với $2\le i\le a$, cộng hàng $i$ với hàng 1 (sau khi nhân hàng 1 với đối của yếu tố hàng $i$, cột 1). Đến lúc này tất cả yêu tố trên cột 1 đều bằng 0, trừ yếu tố hàng 1 cột 1.
        \item Tiếp tục quá trình trên, đến khi $A$ đạt được dạng sau (hàng $i$ có yếu tố hàng $i$, cột $i$ bằng 1, các yếu tố đứng trước bằng 0):
              \[
                  \begin{pmatrix}
                      1      & *      & \cdots & *      & *      & \cdots & *      \\
                      0      & 1      & \cdots & *      & *      & \cdots & *      \\
                      \vdots & \vdots & \ddots & \vdots & \vdots & \ddots & \vdots \\
                      0      & 0      & \cdots & 1      & *      & \cdots & *      \\
                      0      & 0      & \cdots & 0      & 0      & \cdots & 0      \\
                      \vdots & \vdots & \ddots & \vdots & \vdots & \ddots & \vdots \\
                      0      & 0      & \cdots & 0      & 0      & \cdots & 0
                  \end{pmatrix}
              \]
        \item Đối với các cột $i > a$. Bắt đầu từ hàng $a$, rồi đến $a-1$, \ldots, 1, ta áp dụng biến đổi (vii), (ix) để đưa $A$ về dạng:
              \[
                  \begin{pmatrix}
                      1      & *      & \cdots & *      & 0      & \cdots & 0      \\
                      0      & 1      & \cdots & *      & 0      & \cdots & 0      \\
                      \vdots & \vdots & \ddots & \vdots & \vdots & \ddots & \vdots \\
                      0      & 0      & \cdots & 1      & 0      & \cdots & 0      \\
                      0      & 0      & \cdots & 0      & 0      & \cdots & 0      \\
                      \vdots & \vdots & \ddots & \vdots & \vdots & \ddots & \vdots \\
                      0      & 0      & \cdots & 0      & 0      & \cdots & 0
                  \end{pmatrix}
              \]
        \item Đối với các hàng $i < a$. Ta thực hiện biến đổi (iii) để đưa $A$ về dạng:
              \[
                  \begin{pmatrix}
                      1      & 0      & \cdots & 0      & 0      & \cdots & 0      \\
                      0      & 1      & \cdots & 0      & 0      & \cdots & 0      \\
                      \vdots & \vdots & \ddots & \vdots & \vdots & \ddots & \vdots \\
                      0      & 0      & \cdots & 1      & 0      & \cdots & 0      \\
                      0      & 0      & \cdots & 0      & 0      & \cdots & 0      \\
                      \vdots & \vdots & \ddots & \vdots & \vdots & \ddots & \vdots \\
                      0      & 0      & \cdots & 0      & 0      & \cdots & 0
                  \end{pmatrix}
              \]
    \end{itemize}
    \par Như vậy, ta chỉ cần chứng minh bất đẳng thức cho trường hợp $A$ có dạng:
    \[
        \begin{pmatrix}
            1      & 0      & \cdots & 0      & 0      & \cdots & 0      \\
            0      & 1      & \cdots & 0      & 0      & \cdots & 0      \\
            \vdots & \vdots & \ddots & \vdots & \vdots & \ddots & \vdots \\
            0      & 0      & \cdots & 1      & 0      & \cdots & 0      \\
            0      & 0      & \cdots & 0      & 0      & \cdots & 0      \\
            \vdots & \vdots & \ddots & \vdots & \vdots & \ddots & \vdots \\
            0      & 0      & \cdots & 0      & 0      & \cdots & 0
        \end{pmatrix}
    \]
    \par Phần còn lại của chứng minh được triển khai chi tiết hơn từ chứng minh sau: \url{https://math.stackexchange.com/questions/298836/sylvester-rank-inequality-operatornamerank-a-operatornamerankb-leq-o}
    \begin{align*}
        \rank(A) + \rank(B) & = \rank(A) + \rank(AB + B(I_{n} - A))          \\
                            & \le \rank(A) + \rank(AB) + \rank(B(I_{n} - A))
    \end{align*}
    \[
        B(I_{n} - A) =
        \begin{pmatrix}
            b_{11} & b_{12} & \cdots & b_{1n} \\
            b_{21} & b_{22} & \cdots & b_{2n} \\
            \vdots & \vdots & \ddots & \vdots \\
            b_{n1} & b_{n2} & \cdots & b_{nn}
        \end{pmatrix}
        \begin{pmatrix}
            0      & \cdots & 0      & 0      & \cdots & 0      \\
            \vdots & \ddots & \vdots & \vdots & \ddots & \vdots \\
            0      & \cdots & 0      & 0      & \cdots & 0      \\
            0      & \cdots & 0      & 1      & \cdots & 0      \\
            \vdots & \ddots & \vdots & \vdots & \ddots & \vdots \\
            0      & \cdots & 0      & 0      & \cdots & 1
        \end{pmatrix}
        =
        \begin{pmatrix}
            0      & \cdots & 0      & b_{1(a+1)} & \cdots & b_{1n} \\
            0      & \cdots & 0      & b_{2(a+1)} & \cdots & b_{2n} \\
            \vdots & \ddots & \vdots & \vdots     & \ddots & \vdots \\
            0      & \cdots & 0      & b_{a(a+1)} & \cdots & b_{an} \\
            \vdots & \ddots & \vdots & \vdots     & \ddots & \vdots \\
            0      & \cdots & 0      & b_{n(a+1)} & \cdots & b_{nn}
        \end{pmatrix}.
    \]
    \par Đẳng thức trên chứng tỏ $\rank(B(I_{n} - A))\le n - \rank(A)$.
    \begin{align*}
        \rank(A) + \rank(B) & = \rank(A) + \rank(AB + B(I_{n} - A))          \\
                            & \le \rank(A) + \rank(AB) + \rank(B(I_{n} - A)) \\
                            & \text{(tiếp tục)}                              \\
                            & \le \rank(A) + \rank(AB) + n - \rank(A)        \\
                            & = \rank(AB) + n.
    \end{align*}
    \par Vậy $\rank(A) + \rank(B) - n \le \rank(AB)$.
\end{proof}

% exercise 3.43
\begin{exercise}
    \par Chứng minh rằng nếu trường $\mathbb{F}$ có đặc số khác 2 và $A$ là một ma trận vuông cỡ $n$ với các yếu tố trong $\mathbb{F}$ sao cho $A^{2} = E$, thì $\rank(A + E) + \rank(A - E) = n$. Tìm phản ví dụ cho kết luận nói trên khi đặc số của $\mathbb{F}$ bằng 2.
\end{exercise}

\begin{proof}
    \par Nếu $\text{Char}(\mathbb{F})\ne 2$, áp dụng bất đẳng thức Sylvester và bài toán~\ref{chapter3:rank-of-sum}:
    \[
        \rank(A) = \rank(2A) = \rank(A+E + A-E) \le \rank(A+E) + \rank(A-E) \le n + \rank(A^{2} - E) = n.
    \]
    \par Vì $A^{2} = E$ nên $A$ khả nghịch, do đó $\rank(A) = n$.
    \par Nếu $\rank(A + E) + \rank(A - E) < n$ thì $\rank(A) < n$, mâu thuẫn với đẳng thức $\rank(A) = n$.
    \par Do đó, $\rank(A + E) + \rank(A - E) = n$.
    \bigskip
    \par Nếu $\text{Char}(\mathbb{F}) = 2$, chọn $A = E$.
    \par $A = E$ và $\text{Char}(\mathbb{F}) = 2$ kéo theo $A + E = A - E = 0$, suy ra $\rank(A + E) + \rank(A - E) = 0 < n$.
\end{proof}

% exercise 3.44
\begin{exercise}
    \par Tìm ma trận nghịch đảo của các ma trận sau đây bằng phương pháp định thức và phương pháp biến đổi sơ cấp:
    \begin{center}
        \begin{enumerate*}[label = (\alph*)]
            \item $\begin{pmatrix}
                          0 & 1 & 3 \\
                          2 & 3 & 5 \\
                          3 & 6 & 7
                      \end{pmatrix}$,
            \item $\begin{pmatrix}
                          1 & 2 & -1 & -2 \\
                          3 & 8 & 0  & -4 \\
                          2 & 2 & -4 & -3 \\
                          3 & 8 & -1 & -6
                      \end{pmatrix}$.
        \end{enumerate*}
    \end{center}
\end{exercise}

\begin{proof}[Lời giải]
    \begin{enumerate}[label = (\alph*)]
        \item
              \begingroup{}
              \allowdisplaybreaks{}
              \begin{gather*}
                  \left(\begin{array}{ccc|ccc}
                          0 & 1 & 3 & 1 & 0 & 0 \\
                          2 & 3 & 5 & 0 & 1 & 0 \\
                          3 & 6 & 7 & 0 & 0 & 1
                      \end{array}
                  \right)
                  \stackrel{r_{3}:= r_{3}-r_{2}}{\Longleftrightarrow}
                  \left(\begin{array}{ccc|ccc}
                          0 & 1 & 3 & 1 & 0  & 0 \\
                          2 & 3 & 5 & 0 & 1  & 0 \\
                          1 & 3 & 2 & 0 & -1 & 1
                      \end{array}
                  \right) \\
                  \stackrel{
                      \substack{
                          r_{1}:= r_{1} + r_{2} \\
                          r_{2}:= r_{2} {-} 2r_{1}
                      }
                  }{\Longleftrightarrow}
                  \left(\begin{array}{ccc|ccc}
                          1 & 4  & 5 & 1 & -1 & 1  \\
                          0 & -3 & 1 & 0 & 3  & -2 \\
                          1 & 3  & 2 & 0 & -1 & 1
                      \end{array}
                  \right)
                  \stackrel{
                      r_{3}:= r_{3} {-} r_{1}
                  }{\Longleftrightarrow}
                  \left(\begin{array}{ccc|ccc}
                          1 & 4  & 5  & 1  & -1 & 1  \\
                          0 & -3 & 1  & 0  & 3  & -2 \\
                          0 & -1 & -3 & -1 & 0  & 0
                      \end{array}
                  \right) \\
                  \stackrel{
                      r_{3}:= 3r_{3} {-} r_{2}
                  }{\Longleftrightarrow}
                  \left(\begin{array}{ccc|ccc}
                          1 & 4  & 5   & 1  & -1 & 1  \\
                          0 & -3 & 1   & 0  & 3  & -2 \\
                          0 & 0  & -10 & -3 & -3 & 2
                      \end{array}
                  \right)
                  \stackrel{
                      r_{2}:= 10r_{2} + r_{3}
                  }{\Longleftrightarrow}
                  \left(\begin{array}{ccc|ccc}
                          1 & 4   & 5   & 1  & -1 & 1   \\
                          0 & -30 & 0   & -3 & 27 & -18 \\
                          0 & 0   & -10 & -3 & -3 & 2
                      \end{array}
                  \right) \\
                  \stackrel{
                  r_{2}:= \frac{1}{3}r_{2}
                  }{\Longleftrightarrow}
                  \left(\begin{array}{ccc|ccc}
                          1 & 4   & 5   & 1  & -1 & 1  \\
                          0 & -10 & 0   & -1 & 9  & -6 \\
                          0 & 0   & -10 & -3 & -3 & 2
                      \end{array}
                  \right)
                  \stackrel{
                      r_{1}:= 2r_{1} + r_{3}
                  }{\Longleftrightarrow}
                  \left(\begin{array}{ccc|ccc}
                          2 & 8   & 0   & -1 & -5 & 4  \\
                          0 & -10 & 0   & -1 & 9  & -6 \\
                          0 & 0   & -10 & -3 & -3 & 2
                      \end{array}
                  \right) \\
                  \stackrel{
                  r_{1}:= r_{1} + \frac{4}{5}r_{2}
                  }{\Longleftrightarrow}
                  \left(\begin{array}{ccc|ccc}
                          2 & 0   & 0   & \frac{-9}{5} & \frac{11}{5} & \frac{-4}{5} \\
                          0 & -10 & 0   & -1           & 9            & -6           \\
                          0 & 0   & -10 & -3           & -3           & 2
                      \end{array}
                  \right)
                  \stackrel{
                  \substack{
                  r_{1}:= \frac{1}{2}r_{1} \\
                  r_{2}:= \frac{-1}{10}r_{2} \\
                  r_{3}:= \frac{-1}{10}r_{3}
                  }
                  }{\Longleftrightarrow}
                  \left(\begin{array}{ccc|ccc}
                          1 & 0 & 0 & \frac{-9}{10} & \frac{11}{10} & \frac{-2}{5} \\
                          0 & 1 & 0 & \frac{1}{10}  & \frac{-9}{10} & \frac{3}{5}  \\
                          0 & 0 & 1 & \frac{3}{10}  & \frac{3}{10}  & \frac{-1}{5}
                      \end{array}
                  \right)
              \end{gather*}
              \par Vậy
              \[
                  \begin{pmatrix}
                      0 & 1 & 3 \\
                      2 & 3 & 5 \\
                      3 & 6 & 7
                  \end{pmatrix}^{-1}
                  =
                  \begin{pmatrix}
                      \frac{-9}{10} & \frac{11}{10} & \frac{-2}{5} \\
                      \frac{1}{10}  & \frac{-9}{10} & \frac{3}{5}  \\
                      \frac{3}{10}  & \frac{3}{10}  & \frac{-1}{5}
                  \end{pmatrix}.
              \]
              \endgroup{}
        \item
              \begingroup{}
              \allowdisplaybreaks{}
              \begin{gather*}
                  \left(\begin{array}{cccc|cccc}
                          1 & 2 & -1 & -2 & 1 & 0 & 0 & 0 \\
                          3 & 8 & 0  & -4 & 0 & 1 & 0 & 0 \\
                          2 & 2 & -4 & -3 & 0 & 0 & 1 & 0 \\
                          3 & 8 & -1 & -6 & 0 & 0 & 0 & 1
                      \end{array}
                  \right)
                  \stackrel{
                      \substack{
                          r_{2}:= r_{2} {-} 3r_{1} \\
                          r_{3}:= r_{3} {-} 2r_{1} \\
                          r_{4}:= r_{4} {-} 3r_{1}
                      }
                  }{\Longleftrightarrow}
                  \left(\begin{array}{cccc|cccc}
                          1 & 2  & -1 & -2 & 1  & 0 & 0 & 0 \\
                          0 & 2  & 3  & 2  & -3 & 1 & 0 & 0 \\
                          0 & -2 & -2 & 1  & -2 & 0 & 1 & 0 \\
                          0 & 2  & 2  & 0  & -3 & 0 & 0 & 1
                      \end{array}
                  \right) \\
                  \stackrel{
                      \substack{
                          r_{3}:= r_{3} + r_{2} \\
                          r_{4}:= r_{4} {-} r_{2}
                      }
                  }{\Longleftrightarrow}
                  \left(\begin{array}{cccc|cccc}
                          1 & 2 & -1 & -2 & 1  & 0  & 0 & 0 \\
                          0 & 2 & 3  & 2  & -3 & 1  & 0 & 0 \\
                          0 & 0 & 1  & 3  & -5 & 1  & 1 & 0 \\
                          0 & 0 & -1 & -2 & 0  & -1 & 0 & 1
                      \end{array}
                  \right)
                  \stackrel{
                      r_{4}:= r_{4} + r_{3}
                  }{\Longleftrightarrow}
                  \left(\begin{array}{cccc|cccc}
                          1 & 2 & -1 & -2 & 1  & 0 & 0 & 0 \\
                          0 & 2 & 3  & 2  & -3 & 1 & 0 & 0 \\
                          0 & 0 & 1  & 3  & -5 & 1 & 1 & 0 \\
                          0 & 0 & 0  & 1  & -5 & 0 & 1 & 1
                      \end{array}
                  \right) \\
                  \stackrel{
                      \substack{
                          r_{1}:= r_{1} + 2r_{4} \\
                          r_{2}:= r_{2} {-} 2r_{4} \\
                          r_{3}:= r_{3} {-} 3r_{4}
                      }
                  }{\Longleftrightarrow}
                  \left(\begin{array}{cccc|cccc}
                          1 & 2 & -1 & 0 & -9 & 0 & 2  & 2  \\
                          0 & 2 & 3  & 0 & 7  & 1 & -2 & -2 \\
                          0 & 0 & 1  & 0 & 10 & 1 & -2 & -3 \\
                          0 & 0 & 0  & 1 & -5 & 0 & 1  & 1
                      \end{array}
                  \right)
                  \stackrel{
                      \substack{
                          r_{1}:= r_{1} + r_{3} \\
                          r_{2}:= r_{2} {-} 3r_{3}
                      }
                  }{\Longleftrightarrow}
                  \left(\begin{array}{cccc|cccc}
                          1 & 2 & 0 & 0 & 1   & 1  & 0  & -1 \\
                          0 & 2 & 0 & 0 & -23 & -2 & 4  & 7  \\
                          0 & 0 & 1 & 0 & 10  & 1  & -2 & -3 \\
                          0 & 0 & 0 & 1 & -5  & 0  & 1  & 1
                      \end{array}
                  \right) \\
                  \stackrel{
                      r_{1}:= r_{1} {-} r_{2}
                  }{\Longleftrightarrow}
                  \left(\begin{array}{cccc|cccc}
                          1 & 0 & 0 & 0 & 24  & 3  & -4 & -8 \\
                          0 & 2 & 0 & 0 & -23 & -2 & 4  & 7  \\
                          0 & 0 & 1 & 0 & 10  & 1  & -2 & -3 \\
                          0 & 0 & 0 & 1 & -5  & 0  & 1  & 1
                      \end{array}
                  \right)
                  \stackrel{
                  r_{2}:= \frac{1}{2}r_{2}
                  }{\Longleftrightarrow}
                  \left(\begin{array}{cccc|cccc}
                          1 & 0 & 0 & 0 & 24            & 3  & -4 & -8          \\
                          0 & 1 & 0 & 0 & \frac{-23}{2} & -1 & 2  & \frac{7}{2} \\
                          0 & 0 & 1 & 0 & 10            & 1  & -2 & -3          \\
                          0 & 0 & 0 & 1 & -5            & 0  & 1  & 1
                      \end{array}
                  \right)
              \end{gather*}
              \par Vậy
              \[
                  \begin{pmatrix}
                      1 & 2 & -1 & -2 \\
                      3 & 8 & 0  & -4 \\
                      2 & 2 & -4 & -3 \\
                      3 & 8 & -1 & -6
                  \end{pmatrix}^{-1}
                  =
                  \begin{pmatrix}
                      24            & 3  & -4 & -8          \\
                      \frac{-23}{2} & -1 & 2  & \frac{7}{2} \\
                      10            & 1  & -2 & -3          \\
                      -5            & 0  & 1  & 1
                  \end{pmatrix}.
              \]
              \endgroup{}
    \end{enumerate}
\end{proof}

\par Nghiên cứu tính tương thích của các hệ phương trình sau, tìm một nghiệm riêng và nghiệm tổng quát của chúng:

% exercise 3.45
\begin{exercise}
    \[
        \begin{array}{ccccccccc}
            3x & - & 2y & + & 5z & + & 4t & = & 2, \\
            6x & - & 4y & + & 4z & + & 3t & = & 3, \\
            9x & - & 6y & + & 3z & + & 2t & = & 4.
        \end{array}
    \]
\end{exercise}

\begin{proof}[Lời giải]
    \par Thực hiện phép biến đổi sơ cấp trên ma trận hệ số mở rộng:
    \begingroup{}
    \allowdisplaybreaks{}
    \begin{gather*}
        \left(
        \begin{array}{cccc|c}
                3 & -2 & 5 & 4 & 2 \\
                6 & -4 & 4 & 3 & 3 \\
                9 & -6 & 3 & 2 & 4
            \end{array}
        \right)
        \stackrel{
            \substack{
                r_{2}:= r_{2} {-} 2r_{1} \\
                r_{3}:= r_{3} {-} 3r_{1}
            }
        }{\Longleftrightarrow}
        \left(\begin{array}{cccc|c}
                3 & -2 & 5   & 4   & 2  \\
                0 & 0  & -6  & -5  & -1 \\
                0 & 0  & -12 & -10 & -2
            \end{array}
        \right)
        \stackrel{
            r_{3}:= r_{3} {-} 2r_{2}
        }{\Longleftrightarrow}
        \left(\begin{array}{cccc|c}
                3 & -2 & 5  & 4  & 2  \\
                0 & 0  & -6 & -5 & -1 \\
                0 & 0  & 0  & 0  & 0
            \end{array}
        \right)
    \end{gather*}
    \endgroup{}
    \par Theo định lý Kronecker-Capelli, hệ phương trình tuyến tính trên có nghiệm.
    \par Một nghiệm riêng của hệ trên là:
    \[
        (x, y, z, t) = (1, 1, 1, -1).
    \]
    \par Nghiệm của hệ phương trình tuyến tính thuần nhất:
    \[
        (x, y, z, t) = (a, b, -15a + 10b, 18a - 12b).
    \]
    \par Nghiệm tổng quát của hệ phương trình tuyến tính trên là:
    \[
        (x, y, z, t) = (1 + a, 1 + b, 1 - 15a + 10b, -1 + 18a - 12b).
    \]
\end{proof}

% exercise 3.46
\begin{exercise}
    \[
        \begin{array}{ccccccccc}
            8x & + & 6y & + & 5z & + & 2t & = & 21, \\
            3x & + & 3y & + & 2z & + & t  & = & 10, \\
            4x & + & 2y & + & 3z & + & t  & = & 8,  \\
            3x & + & 5y & + & z  & + & t  & = & 15, \\
            7x & + & 4y & + & 5z & + & 2t & = & 18.
        \end{array}
    \]
\end{exercise}

\begin{proof}[Lời giải]
    \par Thực hiện phép biến đổi sơ cấp trên ma trận hệ số mở rộng:
    \begingroup{}
    \allowdisplaybreaks{}
    \begin{gather*}
        \left(
        \begin{array}{cccc|c}
                8 & 6 & 5 & 2 & 21 \\
                3 & 3 & 2 & 1 & 10 \\
                4 & 2 & 3 & 1 & 8  \\
                3 & 5 & 1 & 1 & 15 \\
                7 & 4 & 5 & 2 & 18
            \end{array}
        \right)
        \stackrel{
            \substack{
                r_{2}:= 2r_{2} {-} r_{1} \\
                r_{3}:= 2r_{3} {-} r_{1} \\
                r_{4}:= 2r_{4} {-} r_{1} \\
                r_{5}:= r_{5} {-} r_{1}
            }
        }{\Longleftrightarrow}
        \left(\begin{array}{cccc|c}
                8  & 6  & 5  & 2 & 21 \\
                -2 & 0  & -1 & 0 & -1 \\
                0  & -2 & 1  & 0 & -5 \\
                -2 & 4  & -3 & 0 & 9  \\
                -1 & -2 & 0  & 0 & -3
            \end{array}
        \right)
        \stackrel{
        \substack{
        r_{3}:= r_{3} + r_{2} \\
        r_{4}:= r_{4} {-} 3r_{2} \\
        r_{5}:= -r_{5}
        }
        }{\Longleftrightarrow}
        \left(\begin{array}{cccc|c}
                8  & 6  & 5  & 2 & 21 \\
                -2 & 0  & -1 & 0 & -1 \\
                -2 & -2 & 0  & 0 & -6 \\
                4  & 4  & 0  & 0 & 12 \\
                1  & 2  & 0  & 0 & 3
            \end{array}
        \right) \\
        \stackrel{
            r_{4}:= r_{4} + 2r_{3}
        }{\Longleftrightarrow}
        \left(\begin{array}{cccc|c}
                8  & 6  & 5  & 2 & 21 \\
                -2 & 0  & -1 & 0 & -1 \\
                -2 & -2 & 0  & 0 & -6 \\
                0  & 0  & 0  & 0 & 0  \\
                1  & 2  & 0  & 0 & 3
            \end{array}
        \right)
        \stackrel{
        \substack{
        r_{2}:= -r_{2} \\
        r_{3}:= \frac{-1}{2}r_{3}
        }
        }{\Longleftrightarrow}
        \left(\begin{array}{cccc|c}
                8 & 6 & 5 & 2 & 21 \\
                2 & 0 & 1 & 0 & 1  \\
                1 & 1 & 0 & 0 & 3  \\
                0 & 0 & 0 & 0 & 0  \\
                1 & 2 & 0 & 0 & 3
            \end{array}
        \right)
        \stackrel{
            r_{5}:= r_{5} {-} r_{3}
        }{\Longleftrightarrow}
        \left(\begin{array}{cccc|c}
                8 & 6 & 5 & 2 & 21 \\
                2 & 0 & 1 & 0 & 1  \\
                1 & 1 & 0 & 0 & 3  \\
                0 & 0 & 0 & 0 & 0  \\
                0 & 1 & 0 & 0 & 0
            \end{array}
        \right) \\
        \stackrel{
            r_{3}:= r_{3} {-} r_{5}
        }{\Longleftrightarrow}
        \left(\begin{array}{cccc|c}
                8 & 6 & 5 & 2 & 21 \\
                2 & 0 & 1 & 0 & 1  \\
                1 & 0 & 0 & 0 & 3  \\
                0 & 0 & 0 & 0 & 0  \\
                0 & 1 & 0 & 0 & 0
            \end{array}
        \right)
        \stackrel{
            r_{2}:= r_{2} {-} 2r_{3}
        }{\Longleftrightarrow}
        \left(\begin{array}{cccc|c}
                8 & 6 & 5 & 2 & 21 \\
                0 & 0 & 1 & 0 & -5 \\
                1 & 0 & 0 & 0 & 3  \\
                0 & 0 & 0 & 0 & 0  \\
                0 & 1 & 0 & 0 & 0
            \end{array}
        \right)
        \stackrel{
            r_{1}:= r_{1} {-} 8r_{3} {-} 5r_{2} {-} 6r_{5}
        }{\Longleftrightarrow}
        \left(\begin{array}{cccc|c}
                0 & 0 & 0 & 2 & 22 \\
                0 & 0 & 1 & 0 & -5 \\
                1 & 0 & 0 & 0 & 3  \\
                0 & 0 & 0 & 0 & 0  \\
                0 & 1 & 0 & 0 & 0
            \end{array}
        \right)
    \end{gather*}
    \par Vậy hệ phương trình có nghiệm duy nhất:
    \[
        (x, y, z, t) = (3, 0, -5, 11).
    \]
    \endgroup{}
\end{proof}

\end{document}

% chktex-file 8
\chapter{Connectedness and Compactness}

\section*{Connectedness}\addcontentsline{toc}{section}{Connectedness}

\subsection*{Definitions and Basic Properties}\addcontentsline{toc}{subsection}{Definitions and Basic Properties}

\begin{exercise}{4.3}
	Suppose $X$ is a connected topological space, and $\sim$ is an equivalence relation on $X$ such that every equivalence class is open. Show that there is exactly one equivalence class, namely $X$ itself.
\end{exercise}

\begin{proof}
	Suppose $X$ has exactly $n$ equivalence classes, all of which are open. Because all equivalence classes constitute a partition of $X$, then $X$ is the union of these open, disjoint equivalence classes. On the other hand, $X$ is connected, so $n = 1$. Hence $X$ has exactly one equivalence class.
\end{proof}

\begin{exercise}{4.4}
	Prove that a topological space $X$ is disconnected if and only if there exists a nonconstant continuous function from $X$ to the discrete space $\set{0,1}$.
\end{exercise}

\begin{proof}
	Suppose there is a nonconstant continuous function $f: X\to \set{0,1}$, then $f^{-1}(0)$ and $f^{-1}(1)$ are nonempty, open, disjoint, and their union is $X$. Therefore $X$ is disconnected.

	Conversely, suppose $X$ is disconnected, then there are nonempty, open, disjoint subsets $U, V\subseteq X$ such that $X = U\cup V$. Define $f: X\to \set{0,1}$ as follows: $f(x) = 0$ if $x\in U$ and $f(x) = 1$ if $x\notin U$, then $f$ is nonconstant and $f$ is continuous (because the preimages under $f$ of $\varnothing, \set{0}, \set{1}, \set{0, 1}$ are open in $X$).
\end{proof}

\begin{exercise}{4.5}
	Prove that a topological space is disconnected if and only if it is homeomorphic to a disjoint union of two or more nonempty spaces.
\end{exercise}

\begin{proof}
	Let $X$ be a topological space.

	Suppose $X$ is disconnected, then there are subsets $U, V\subseteq X$ that disconnect $X$. The map $\varphi: X \to U\sqcup V$ given by
	\begin{equation*}
		\varphi(x) = \begin{cases}
			(x, 0) & \text{if $x \in U$} \\
			(x, 1) & \text{if $x \in V$}
		\end{cases}
	\end{equation*}

	is a homeomorphism, so $X$ is homeomorphic to the disjoint union $U\sqcup V$. Hence $X$ is homeomorphic to a disjoint union of at least two nonempty spaces.

	Conversely, suppose $X$ is homeomorphic to a disjoint union of $n\geq 2$ nonempty spaces $U_{1}, \ldots, U_{n}$. Let $f: X\to \coprod^{n}_{i=1}U_{i}$ be a homeomorphism, then $f^{-1}(U_{1}), \ldots, f^{-1}(U_{n})$ constitutes a partition on $X$, and $f^{-1}(U_{1}), \ldots, f^{-1}(U_{n})$ are nonempty open subsets of $X$. Because $n\geq 2$, it follows that $X$ is disconnected.
\end{proof}

\begin{exercise}{4.10}
	Suppose $M$ is a connected manifold with nonempty boundary. Show that its double $D(M)$ is connected.
\end{exercise}

\begin{proof}
	Let $h: \partial M\to \partial M$ be the identity map from the boundary of $M$ onto itself. $\sim$ is the equivalence relation on $D(M)$ defined by $x\sim h(x)$ for $x\in\partial M$, and other points are equivalent to itself. By the definition of the double of a manifold with boundary
	\begin{equation*}
		D(M) = M\cup_{h}M = (M\sqcup M)/_{\sim}.
	\end{equation*}

	Assume that $D(M)$ is disconnected by open subsets $U, V$. Let $q: M \sqcup M \to (M \sqcup M)/_{\sim}$ be the quotient map, then $q^{-1}(U)$ and $q^{-1}(V)$ disconnect $M\sqcup M \approx (M\times\set{0}) \cup (M\times\set{1})$. Because $M\times\set{0}, M\times\set{1} \subseteq q^{-1}(U) \cup q^{-1}(V)$ and $M\times\set{0}, M\times\set{1}$ are connected, it follows from Proposition 4.9 that either $M\times\set{0} \subseteq q^{-1}(U)$ or $M\times\set{0} \subseteq q^{-1}(V)$, either $M\times\set{1} \subseteq q^{-1}(U)$ or $M\times\set{1} \subseteq q^{-1}(V)$. Because $q^{-1}(U)$ and $q^{-1}(V)$ are nonempty, it follows that either $M\times\set{0} \subseteq q^{-1}(U)$ and $M\times\set{1} \subseteq q^{-1}(V)$ or $M\times\set{0} \subseteq q^{-1}(V)$ and $M\times\set{1} \subseteq q^{-1}(U)$. Without loss of generality, suppose that the former is the case. Since the boundary $\partial M$ is nonempty, there is $a \in \partial M$. $a \sim h(a)$, $(a, 0) \in M\times\set{0} = q^{-1}(U)$ and $(a, 1) \in M\times\set{1} = q^{-1}(V)$ so $q(a) = q(h(a)) \in U, V$. Hence $U$ and $V$ are not disjoint, which contradicts our assumption.

	Thus $D(M)$ is connected.
\end{proof}

\subsection*{Path Connectedness}\addcontentsline{toc}{subsection}{Path Connectedness}

\begin{exercise}{4.14}\label{exercise:4.14}
	Prove Proposition 4.13 (Properties of Path-Connected Spaces).
	\begin{enumerate}[label={(\alph*)}]
		\item Every continuous image of a path-connected space is path-connected.
		\item Let $X$ be a space, and let ${\{ B_{\alpha} \}}_{\alpha\in A}$ be a collection of path-connected subspaces of $X$ with a point in common. Then $\bigcup_{\alpha\in A}B_{\alpha}$ is path-connected.
		\item Every product of finitely many path-connected spaces is path-connected.
		\item Every quotient space of a path-connected space is path-connected.
	\end{enumerate}
\end{exercise}

\begin{proof}
	\begin{enumerate}[label={(\alph*)}]
		\item Let $f: X\to Y$ be a continuous map and $X$ is a path-connected space. Let $p, q$ be two points of $f(X)$, let $a \in f^{-1}(p)$ and $b \in f^{-1}(q)$. Because $X$ is path-connected, there is a continuous map $g: \closedinterval{0, 1} \to X$ such that $g(0) = a$ and $g(1) = b$, so the composition $f\circ g: \closedinterval{0, 1} \to f(X)$ is continuous and $(f\circ g)(0) = p$ and $(f\circ g)(1) = q$. Hence for every two points $p, q$ of $f(X)$, there is a continuous map from $\closedinterval{0, 1}$ to $f(X)$ such that the images of $0, 1$ are $p, q$, respectively, which implies $f(X)$ is path-connected.
		\item Let $p, q$ be two points of $\bigcup_{\alpha\in A}B_{\alpha}$.

		      $p \in B_{\alpha_{p}}$ and $q \in B_{\alpha_{q}}$ for some $\alpha_{p}, \alpha_{q} \in A$. Let $x \in \bigcap_{\alpha\in A}B_{\alpha}$ (these sets have a point in common). Because $B_{\alpha_{p}}, B_{\alpha_{q}}$ are path-connected, there are continuous maps $f_{p}: \closedinterval{0, 1} \to B_{\alpha_{p}}$ such that $f_{p}(0) = p, f_{p}(1) = x$, and $f_{q}: \closedinterval{0, 1} \to B_{\alpha_{q}}$ such that $f_{q}(0) = x, f_{q}(1) = q$.

		      The maps $g: \closedinterval{0, \frac{1}{2}} \to \closedinterval{0, 1}$ given by $g(t) = 2t$ and $h: \closedinterval{\frac{1}{2}, 1} \to \closedinterval{0, 1}$ given by $h(t) = 2t - 1$ are continuous. The compositions $f_{p}\circ g$ and $f_{q}\circ h$ are therefore continuous, and they agree on $\closedinterval{0, \frac{1}{2}} \cap \closedinterval{\frac{1}{2}, 1}$, since $(f_{p}\circ g)(1/2) = x = (f_{q}\circ h)(1/2)$. $\closedinterval{0, \frac{1}{2}}, \closedinterval{\frac{1}{2}, 1}$ constitute a finite closed cover of $\closedinterval{0, 1}$, so by the gluing lemma, there is a unique continuous map $f: \closedinterval{0, 1} \to B_{\alpha_{p}} \cup B_{\alpha_{q}}$ such that $f\vert_{\closedinterval{0, \frac{1}{2}}} = f_{p}\circ g$ and $f\vert_{\closedinterval{\frac{1}{2}, 1}} = f_{q}\circ h$. Moreover, $f(0) = p, f(1) = q$. Hence there is a path in $\bigcup_{\alpha\in A}B_{\alpha}$ from $p$ to $q$.

		      Thus $\bigcup_{\alpha\in A}B_{\alpha}$ is path-connected.
		\item It suffices to prove that the product of two path-connected spaces is path-connected.

		      Let $X, Y$ be path-connected spaces and $(x_{1}, y_{1}), (x_{2}, y_{2})$ are two points of $X\times Y$. The maps $i_{y_{0}}: X\to X\times Y$ given by $i_{y_{0}}(x) = (x, y_{0})$ and $i_{x_{0}}: Y\to X\times Y$ given by $i_{x_{0}}(y) = (x_{0}, y)$ are continuous. From part (a), it follows that $X\times\set{y_{0}}$ and $\set{x_{0}}\times Y$ are path-connected for every $y_{0} \in Y, x_{0}\in X$. Hence there is a path $f_{1}$ in $X\times Y$ from $(x_{1}, y_{1})$ to $(x_{2}, y_{1})$ and a path $f_{2}$ in $X\times Y$ from $(x_{2}, y_{1})$ to $(x_{2}, y_{2})$.

		      The maps $g: \closedinterval{0, \frac{1}{2}} \to \closedinterval{0, 1}$ given by $g(t) = 2t$ and $h: \closedinterval{\frac{1}{2}, 1} \to \closedinterval{0, 1}$ given by $h(t) = 2t - 1$ are continuous. The compositions $f_{1}\circ g$ and $f_{2}\circ h$ are therefore continuous, and they agree on $\closedinterval{0, \frac{1}{2}} \cap \closedinterval{\frac{1}{2}, 1}$, since $(f_{1}\circ g)(1/2) = (x_{2}, y_{1}) = (f_{2}\circ h)(1/2)$. From the gluing lemma, it follows that there is a unique continuous map $f: \closedinterval{0, 1} \to X\times Y$ such that $f\vert_{\closedinterval{0, \frac{1}{2}}} = f_{1}\circ g$ and $f\vert_{\closedinterval{\frac{1}{2}, 1}} = f_{2}\circ h$. Moreover, $f(0) = (x_{1}, y_{1})$ and $f(1) = (x_{2}, y_{2})$.

		      Therefore $X\times Y$ is path-connected. From mathematical induction, it follows that the finite product of path-connected spaces is path-connected.
		\item Since every quotient map is continuous and surjective, from part (a), it follows that every quotient space of a path-connected space is path-connected.
	\end{enumerate}
\end{proof}

\subsection*{Components and Path Components}\addcontentsline{toc}{subsection}{Components and Path Components}

\begin{exercise}{4.22}
	Prove Proposition 4.21 (Properties of Path Components).

	Let $X$ be any space.
	\begin{enumerate}[label={(\alph*)}]
		\item The path components of $X$ form a partition of $X$.
		\item Each path component is contained in a single component, and each component is a disjoint union of path components.
		\item Any nonempty path-connected subset of $X$ is contained in a single path component.
	\end{enumerate}
\end{exercise}

\begin{proof}
	\begin{enumerate}[label={(\alph*)}]
		\item Let $U, V$ be non-disjoint path components of $X$. From Exercise~\ref{exercise:4.14} (b), $U\cup V$ is path-connected. Due to the maximality of path components, $U = V = U\cup V$, from which we deduce that non-disjoint path components are identical. Hence distinct path components are disjoint.

		      Let $x$ be a point of $X$. The singleton $\set{x}$ is a path component containing $x$. Let ${(B_{\alpha})}_{\alpha\in A}$ be the family of all path-connected sets containing $x$, then $\bigcup_{\alpha\in A}B_{\alpha}$ is path-connected (according to Exercise~\ref{exercise:4.14} (b)). Moreover $\bigcup_{\alpha\in A}B_{\alpha}$ is a maximal path-connected set so it is a path component containing $x$. Therefore every element of $X$ is in a path component.

		      Thus the path components of $X$ form a partition of $X$.

			      [Another approach to part (a) is to prove path-connectedness of two points is an equivalence relation on the given topological space.]
		\item Let $P$ be a path component of $X$. Because the components of $X$ form a partition of $X$, $P$ has a common point with some component $C$ of $X$. Since a path-connected set is also connected, $P$ is a connected set. Because the union of connected sets with a point in common is connected, $P\cup C$ is connected. From the maximality of $C$, we deduce that $P\cup C = C$, which means $P\subseteq C$. Moreover, distinct components are disjoint, so $P$ is contained in the component $C$ only. Hence every path component is contained in a single component.

		      Let $C$ be a component of $X$. Every point $p$ of $C$ is in some path component of $X$, so the path component containing $p$ is contained in $C$. On the other hand, distinct path components are disjoint. Hence $C$ is a disjoint union of path components.
		\item Let $A$ be a nonempty path-connected subset of $X$.

		      Let $x$ be an element of $A$. According to part (a), $x$ is a point of a path component $P$. According to Exercise~\ref{exercise:4.14} (b), $A\cup P$ is path-connected. Because $P$ is a maximal path-connected set, it follows that $A\cup P = P$, which means $A\subseteq P$.

		      If $A$ is contained in two path components, then the two path components are not disjoint (because $A$ is nonempty) and it follows that the two path components are identical (according to part (a)).

		      Therefore any nonempty path-connected subset of $X$ is contained in a single path component of $X$.
	\end{enumerate}
\end{proof}

\begin{exercise}{4.24}
	Prove Proposition 4.23.

	Every manifold (with or without boundary) is locally connected and locally path-connected.
\end{exercise}

\begin{proof}
	Let $M$ be an $n$-manifold (with or without boundary).

	\textbf{Case 1. $M$ is an $n$-manifold without boundary.}

	According to Problem~\ref{problem:2-23}, every manifold has a basis of coordinate balls. On the other hand, every coordinate ball is homeomorphic to an open ball of $\mathbb{R}^{n}$ and every open ball of $\mathbb{R}^{n}$ is path-connected because every open ball of $\mathbb{R}^{n}$ is homeomorphic to $\mathbb{R}^{n}$ (which is path-connected). Therefore every coordinate ball is path-connected (and hence connected). Hence $M$ is locally path-connected and locally connected.

	\textbf{Case 2. $M$ is an $n$-manifold with boundary.}

	Firstly, we construct a basis for $M$. Let $U$ be a nonempty open subset of $M$ and $x\in U$, then $x$ is in the domain of an interior chart or that of a boundary chart.

	If $x$ is in the domain of an interior chart $(V, \varphi_{x})$, then $\varphi_{x}(V)$ is an open subset of $\mathbb{R}^{n}$. $U\cap V$ is homeomorphic to $\varphi_{x}(U\cap V)$ and $\varphi_{x}(U\cap V)$ is an open subset of $\mathbb{R}^{n}$. Because $\varphi_{x}(U\cap V)$ is open and $\varphi_{x}(x)$ is a point of this set, there is an open ball $B_{r}(\varphi_{x}(x)) \subseteq \varphi_{x}(U\cap V)$. Therefore $\varphi_{x}^{-1}(B_{r}(\varphi_{x}(x)))$ and $B_{r}(\varphi_{x}(x))$ are homeomorphic. So $x$ is in the domain of the following interior chart $(\varphi_{x}^{-1}(B_{r}(\varphi_{x}(x))), \varphi_{x}\vert_{\varphi_{x}^{-1}(B_{r}(\varphi_{x}(x)))})$ where the domain is contained in $U$ and homeomorphic to an open ball in $\mathbb{R}^{n}$.

	If $x$ is in the domain of a boundary chart $(V, \varphi_{x})$, then $\varphi_{x}(x) \in \partial\mathbb{H}^{n}$ and $\varphi_{x}(V)$ is an open subset of $\mathbb{H}^{n}$. $U\cap V$ is homeomorphic to $\varphi_{x}(U\cap V)$ and $\varphi_{x}(U\cap V)$ is an open subset of $\mathbb{H}^{n}$. Because $\varphi_{x}(U\cap V)$ is open and $\varphi_{x}(x)$ is a point of this set, there is an open ball $B_{r}(\varphi_{x}(x))$ in $\mathbb{R}^{n}$ such that $B_{r}(\varphi_{x}(x)) \cap \mathbb{H}^{n} \subseteq \varphi_{x}(U\cap V)$ (here we make use of the subspace topology on $\mathbb{H}^{n}$ and the basis for $\mathbb{R}^{n}$ containing open balls). Therefore $W = B_{r}(\varphi_{x}(x)) \cap \mathbb{H}^{n}$ and $\varphi_{x}^{-1}(W)$ are homeomorphic, and $x$ is in the domain of the following boundary chart $(\varphi_{x}^{-1}(W), \varphi_{x}\vert_{W})$ where the domain is contained in $U$ and is homeomorphic to an open half-ball in $\mathbb{H}^{n}$ (it is halved by taking intersection of $\mathbb{H}^{n}$ and an open ball in $\mathbb{R}^{n}$ with center on $\partial \mathbb{H}^{n}$).

	Hence the collection of open sets of $M$ which are domains of some interior chart (and homeomorphic to some open ball in $\mathbb{R}^{n}$) or some boundary chart  (and homeomorphic to some open half-ball in $\mathbb{H}^{n}$) constitutes a basis for the $n$-manifold with boundary $M$.

	On the other hand, an open ball or an open half-ball is path-connected (because it is a convex set), so $M$ has a basis of path-connected (hence connected) open sets. Therefore $M$ is locally path-connected and locally connected.

	From the two cases, we conclude that every manifold (without or with boundary) is locally path-connected and locally connected.
\end{proof}

\section*{Compactness}\addcontentsline{toc}{section}{Compactness}

\subsection*{Definitions and Basic Properties}\addcontentsline{toc}{subsection}{Definitions and Basic Properties}

\begin{exercise}{4.28}
	Prove Lemma 4.27 (Compactness Criterion for Subspaces).

	If $X$ is any topological space, a subset $A\subseteq X$ is compact (in the subspace topology) if and only if every cover of $A$ by open subsets of $X$ has a finite subcover.
\end{exercise}

\begin{proof}
	Suppose $A\subseteq X$ is compact with the subspace topology. Let ${(U_{i})}_{i\in I}$ be a cover of $A$ by open subsets of $X$, then ${(U_{i}\cap A)}_{i\in I}$ is a cover of $A$ by open subsets of $A$. Because $A$ is compact with the subspace topology, ${(U_{i}\cap A)}_{i\in I}$ contains a finite subcover of $A$ by open subsets of $A$, which implies ${(U_{i})}_{i\in I}$ contains a finite subcover of $A$ by open subsets of $X$.

	Conversely, suppose that every cover of $A$ by open subsets of $X$ has a finite subcover. Let ${(V_{i})}_{i\in I}$ be a cover of $A$ by open subsets of $A$. From the definition of subspace topology, for each $i\in I$, there is an open subset $U_{i}\subseteq X$ such that $V_{i} = U_{i}\cap A$. Therefore ${(U_{i})}_{i\in I}$ is a cover of $A$ by open subsets of $X$, so ${(U_{i})}_{i\in I}$ contains a finite subcover of $A$ by open subsets of $X$, which implies that ${(V_{i})}_{i\in I}$ contains a finite subcover of $A$ by open subsets of $A$. Hence $A\subseteq X$ is compact with the subspace topology.
\end{proof}

\begin{exercise}{4.29}
	In any topological space $X$, show that every union of finitely many compact subsets of $X$ is compact.
\end{exercise}

\begin{proof}
	It suffices to prove that the union of two compact subsets of $X$ is compact. Let $A, B$ be compact subsets of $X$ and $\mathcal{O}$ be an open cover of $A\cup B$ by open subsets of $X$. Since $A$ is compact, there exist finitely many open subsets $A_{1}, \ldots, A_{m}$ from $\mathcal{O}$ that covers $A$ and finitely many open subsets $B_{1}, \ldots, B_{n}$ from $\mathcal{O}$ that covers $B$. Hence $\mathcal{O}$ has a finite subcover, which consists of $A_{1}, \ldots, A_{m}, B_{1}, \ldots, B_{n}$. Therefore $A\cup B$ is compact.

	The union of zero compact subsets of $X$ is compact (because it is the empty set). Assume that the union of $n - 1$ compact subsets of $X$ is compact. Let $A_{1}, \ldots, A_{n}$ be $n$ compact subsets of $X$. By the inductive hypothesis, $\bigcup^{n-1}_{i=1}A_{i}$ is a compact subset of $X$. From the previous paragraph, we deduce that $\bigcup^{n}_{i=1}A_{i}$ is a compact subset of $X$. By the principle of mathematical induction, we conclude that every union of finitely many compact subsets of $X$ is compact.
\end{proof}

\begin{exercise}{4.37}
	Suppose $M$ is a compact manifold with boundary. Show that the double of $M$ is compact.
\end{exercise}

\begin{proof}
	Let $h$ be the identity map $\partial M\to \partial M$. From the definition of the double of a manifold with boundary, $D(M) = M\cup_{h} M = {(M\sqcup M)}/_{\sim}$ (where $(a, 0) \sim (h(a), 1)$ for $a\in \partial H$). Because $M\sqcup M \approx (M\times\set{0}) \cup (M\times\set{1})$ is the union of two compact manifolds (which are homeomorphic to the compact manifold $M$), it follows that $M\sqcup M$ is compact.

	Let $q: M\sqcup M \to (M\sqcup M)/_{\sim} = D(M)$ be the quotient map. Because every quotient of a compact space is compact, it follows that $D(M)$ is compact.
\end{proof}

\begin{exercise}{4.38}
	Let $X$ be a compact space, and suppose $\set{F_{n}}$ is a countable collection of nonempty closed subsets of $X$ that are \textbf{nested}, which means that $F_{n}\supseteq F_{n+1}$ for each $n$. Show that $\bigcap_{n}F_{n}$ is nonempty.
\end{exercise}

\begin{proof}
	Each $F_{n}$ is a closed subset $X$, so $X\smallsetminus F_{n}$ is open. Assume for the sake of contradiction that $\bigcap_{n}F_{n}$ is empty then
	\begin{equation*}
		X = X\smallsetminus \left(\bigcap_{n}F_{n}\right) = \bigcup_{n}(X\smallsetminus F_{n})
	\end{equation*}

	so $\set{X\smallsetminus F_{n}}_{n}$ is an open cover of $X$. Because of the compactness of $X$, $\set{X\smallsetminus F_{n}}_{n}$ contains a finite subcover, say $X\smallsetminus F_{k_{1}}, \ldots, X\smallsetminus F_{k_{m}}$, and we can relabel these sets such that $k_{1} < \cdots < k_{m}$. Since $F_{k_{1}} \supseteq \cdots \supseteq F_{k_{m}}$, we have $X\smallsetminus F_{k_{1}} \subseteq \cdots \subseteq X\smallsetminus F_{k_{m}}$ hence $X = \bigcup^{m}_{i=1}(X\smallsetminus F_{k_{i}}) = X \smallsetminus F_{k_{m}}$, which is a contradiction since $X$ is a proper superset of $X\smallsetminus F_{k_{m}}$, because $F_{k_{m}}$ is a nonempty subset of $X$. Hence $\bigcap_{n}F_{n}$ is nonempty.
\end{proof}

\subsection*{Sequential and Limit Point Compactness}\addcontentsline{toc}{subsection}{Sequential and Limit Point Compactness}

\begin{exercise}{4.49}
	Prove the preceding three theorems.

	Theorem 4.46 (Bolzano-Weierstraß). Every bounded sequence in $\mathbb{R}^{n}$ has a convergent subsequence.

	Theorem 4.47. Endowed with the Euclidean metric, a subset of $\mathbb{R}^{n}$ is a complete metric space if and only if it is closed in $\mathbb{R}^{n}$. In particular, $\mathbb{R}^{n}$ is complete.

	Theorem 4.48. Every compact metric space is complete.
\end{exercise}

\begin{proof}
	\textbf{Proof for Theorem 4.46.} Let ${(x_{i})}_{i\in\mathbb{N}}$ be a bounded sequence in $\mathbb{R}^{n}$, then there exists $a > 0$ such that $x_{i} \in {\closedinterval{-a, a}}^{n}$ for every $i\in\mathbb{N}$. If the sequence ${(x_{i})}_{i\in\mathbb{N}}$ takes on finitely many values then it has a subsequence that is eventually constant. Otherwise, suppose ${(x_{i})}_{i\in\mathbb{N}}$ takes on infinitely many values. ${\closedinterval{-a, a}}^{n}$ is closed and bounded in $\mathbb{R}^{n}$ so it is compact, due to Heine-Borel's theorem. The space ${\closedinterval{-a, a}}^{n}$ with subspace topology inherited from $\mathbb{R}^{n}$ is second countable (implies first countable) and Hausdorff so from Lemma 4.42 (compactness implies limit point compactness) and Lemma 4.43 (In first countable Hausdorff spaces, limit point compactness implies sequential compactness), it follows that ${(x_{i})}_{i\in\mathbb{N}}$ has a convergent subsequence.

	\textbf{Proof for Theorem 4.47.} Let $M$ be subset of $\mathbb{R}^{n}$, then $M$ is a metric space with the Euclidean metric restricted to $M$. If $M$ is complete, let $x\in \mathbb{R}^{n}$ be a limit point of $M$. For every $i\in\mathbb{N}$, $B_{1/2^{i}}(x)$ contains a point $a_{i}$ in $M$, then the sequence ${(a_{i})}_{i\in\mathbb{N}}$ is a Cauchy sequence. Since $M$ is complete, the sequence converges to a point of $M$. Because $a_{i}$ converges to $x$, it follows that $x\in M$, hence $M$ contains all of its limit points, which means $M$ is closed. Conversely, if $M$ is closed, let ${(x_{i})}_{i\in\mathbb{N}}$ be a Cauchy sequence of points in $M$. In a metric space, a Cauchy sequence is a bounded sequence. By Theorem 4.46, ${(x_{i})}_{i\in\mathbb{N}}$ has a convergent subsequence. A Cauchy sequence having a convergent subsequence is convergent, hence ${(x_{i})}_{i\in\mathbb{N}}$ is convergent, so $M$ is complete.

	\textbf{Proof for Theorem 4.48.} Let $M$ be a compact metric space and ${(x_{i})}_{i\in\mathbb{N}}$ be a Cauchy sequence of points in $M$. $M$ is a metric space so it is first countable and Hausdorff. By Lemma 4.42, $M$ is limit point compact. By Lemma 4.43 (In first countable Hausdorff spaces, limit point compactness implies sequential compactness), $M$ is sequentially compact. Hence ${(x_{i})}_{i\in\mathbb{N}}$ has a convergent subsequence. A Cauchy sequence having a convergent subsequence is convergent, hence ${(x_{i})}_{i\in\mathbb{N}}$ is convergent, so $M$ is complete.
\end{proof}

\subsection*{The Closed Map Lemma}\addcontentsline{toc}{subsection}{The Closed Map Lemma}

\begin{exercise}{4.58}
	Using the map of Example 4.55, show that there is a coordinate ball in $\mathbb{S}^{n}$ whose closure is equal to all of $\mathbb{S}^{n}$.
\end{exercise}

\begin{proof}
	The quotient map of Example 4.55 is $q: {\overline{B}}^{n} \to \mathbb{S}^{n}$ given by
	\begin{equation*}
		q(x) = \tuple{2\sqrt{1 - {\abs{x}}^{2}}x, 2{\abs{x}}^{2} - 1}.
	\end{equation*}

	Let $N$ be the point of $\mathbb{R}^{n+1}$ whose coordinates are $\tuple{0, \ldots, 0, 1}$ then $N\in \mathbb{S}^{n}$. The singleton set $\set{N}$ is closed in $\mathbb{S}^{n}$ so its complement $\mathbb{S}^{n}\smallsetminus \set{N}$ is open in $\mathbb{S}^{n}$. On the other hand, $q^{-1}(N) = \partial{\overline{B}}^{n}$, so $q^{-1}(\mathbb{S}^{n}\smallsetminus \set{N}) = {\overline{B}}^{n} \smallsetminus \partial{\overline{B}}^{n} = B^{n}$. Moreover, the restriction of $q$ on $\mathbb{S}^{n}\smallsetminus \set{N}$ is injective, it follows that $\mathbb{S}^{n}\smallsetminus \set{N}$ is homeomorphic to $B^{n}$, hence $\mathbb{S}^{n}\smallsetminus \set{N}$ is a coordinate ball in $\mathbb{S}^{n}$. The closure of $\mathbb{S}^{n}\smallsetminus \set{N}$ is the entire $\mathbb{S}^{n}$.

	Thus $\mathbb{S}^{n}\smallsetminus \set{N}$ is a coordinate ball in $\mathbb{S}^{n}$ whose closure is equal to all of $\mathbb{S}^{n}$.
\end{proof}

\begin{exercise}{4.61}\label{exercise:4.61}
	Complete the proof of Proposition 4.60 by showing that $\mathscr{B}$ is a basis.
\end{exercise}

\begin{proof}
	Proposition 4.60. Every manifold has a countable basis of regular coordinate balls.

	Let $M$ be an $n$-manifold. Every point of $M$ has an Euclidean neighborhood (or coordinate domain), and these Euclidean neighborhoods cover $M$. Since $M$ is second countable, every open cover of $X$ has a countable subcover (This is Theorem 2.50). Let $\set{U_{i}}_{i\in\mathbb{N}}$ be a countable collection of such neighborhoods, and for each $U_{i}$, there is a homeomorphism $\varphi_{i}$ from $U_{i}$ onto an open subset $\hat{U}_{i} \subseteq \mathbb{R}^{n}$. For each $x \in \hat{U}_{i}$ there is a positive number $r(x)$ such that $B_{r(x)}(x) \subseteq \hat{U}_{i}$.

	For every $\hat{U}_{i}$, consider the open balls $B_{r}(x)$ where $B_{r}(x) \subseteq \hat{U}_{i}$, all coordinates of $x\in \hat{U}_{i}$ are rational, $r$ is a positive rational number strictly less than $r(x)$. From Lemma 4.59, $\varphi_{i}^{-1}(B_{r}(x))$ is a regular coordinate ball. Let $\mathscr{B}_{i}$ be the collection of those $\varphi_{i}^{-1}(B_{r}(x))$ then $\mathscr{B}_{i}$ is countable. Therefore $\mathscr{B} = \bigcup_{i\in\mathbb{N}}\mathscr{B}_{i}$ is countable.

	Every element of $\mathscr{B}$ is an open subset of $M$. Let $U$ be a nonempty open subset of $M$, then
	\begin{equation*}
		U = \bigcup_{i\in\mathbb{N}}(U\cap U_{i}).
	\end{equation*}

	Remind that the collection of open balls with rational radii and rational coordinates only is a basis for the Euclidean topology on $\mathbb{R}^{n}$. If $a\in U\cap U_{i}$ then the point $\varphi_{i}(a) \in \varphi_{i}(U_{i}\cap U)$ is contained in some $B_{r}(x)$ where $x$ has rational coordinates only and $0 < r < r(x)$ is rational. So $a\in \varphi_{i}^{-1}(B_{r}(x))$, which is an element of $\mathscr{B}$. Hence every point of $U$ admits an element of $\mathscr{B}$ as its neighborhood and this neighborhood is contained in $U$. Therefore $\mathscr{B}$ is a countable basis for $M$.

	Thus every manifold has a countable basis of regular coordinate balls.
\end{proof}

\begin{exercise}{4.62}
	Prove that every manifold with boundary has a countable basis consisting of regular coordinate balls and half-balls.
\end{exercise}

\begin{proof}
	First, we prove a result similar to Lemma 4.59 as follows: Let $M$ be an $n$-manifold with boundary. $B'\subseteq M$ is the (coordinate half-ball) domain of any boundary chart and $\varphi: B' \to B_{r'}(x)\cap \mathbb{H}^{n}$ where $x \in \partial\mathbb{H}^{n}$ is a homeomorphism, then $\varphi^{-1}(B_{r}(x) \cap \mathbb{H}^{n})$ is a regular coordinate half-ball whenever $0 < r < r'$.

	For every $0 < r < r'$, $B_{r}(x) \cap \mathbb{H}^{n}$ is an open subset of $B_{r'}(x) \cap \mathbb{H}^{n}$, so $B = \varphi^{-1}(B_{r}(x) \cap \mathbb{H}^{n})$ is a coordinate half-ball. Regard $\varphi^{-1}$ as a map from $\bar{B}_{r}(0) \cap \mathbb{H}^{n}$ to $M$, then $\varphi^{-1}$ is a continuous and closed map (because $\bar{B}_{r}(x) \cap \mathbb{H}^{n}$ is compact due to Heine-Borel's theorem, and $M$ is Hausdorff). By Problem~\ref{problem:2-6}, $\varphi^{-1}(\bar{B}_{r}(x) \cap \mathbb{H}^{n}) = \overline{B}$, hence $\varphi(\overline{B}) = \bar{B}_{r}(x) \cap \mathbb{H}^{n}$. Hence $\varphi^{-1}(B_{r}(x) \cap \mathbb{H}^{n})$ is a regular coordinate half-ball whenever $0 < r < r'$.

	An equivalence statement to the result we have just proved is given by replacing the point $x$ in $\partial\mathbb{H}^{n}$ by $0$, and this fits the definition of regular coordinate half-ball in the book. However, for convenience, we will use the definition given at the begining of this proof.

	Back to the proof for the main result. Let $M$ be an $n$-manifold with boundary. Every point of $M$ has an Euclidean neighborhood, so $M$ is covered by those Euclidean neighborhoods. Since $M$ is second countable then the open cover made of those Euclidean neighborhoods has a countable subcover. Let $\set{U_{i}}_{i\in\mathbb{N}}$ be an open cover of $M$ made of Euclidean neighborhoods, then for each $U_{i}$, there is a homeomorphism $\varphi_{i}$ from $U_{i}$ to an open subset $\hat{U}_{i}$ of $\mathbb{R}^{n}$ or $\mathbb{H}^{n}$. For each $x\in \hat{U}_{i}$, there is a positive number $r(x)$ such that
	\begin{equation*}
		\begin{cases}
			\text{if $x$ is in $\partial\mathbb{H}^{n}$ then $B_{r(x)} \cap \mathbb{H}^{n} \subseteq \hat{U}_{i}$,} \\
			\text{if $x$ is not in $\partial\mathbb{H}^{n}$ then $B_{r(x)} \subseteq \hat{U}_{i}$.}
		\end{cases}
	\end{equation*}

	For every $\hat{U}_{i}$, consider the following open sets: $B_{r}(x) \subseteq \hat{U}_{i}$ where $x \notin \partial\mathbb{H}^{n}$ (whose coordinates are all rational) and $r$ is a positive rational number strictly less than $r(x)$; $B_{r}(x) \cap \mathbb{H}^{n} \subseteq \hat{U}_{i}$ where $x \in \partial\mathbb{H}^{n}$ (whose coordinates are all rational) and $r$ is a positive rational number strictly less than $r(x)$. In the former case, $B = \varphi_{i}^{-1}(B_{r}(x))$ is a regular coordinate ball, due to Lemma 4.59; in the latter, $B = \varphi_{i}^{-1}(B_{r}(x) \cap \mathbb{H}^{n})$ is a regular coordinate half-ball, according to the first part of this proof. Denote by $\mathscr{B}_{i}$ the collection of those regular coordinate balls and regular coordinate half-balls, and $\mathscr{B} = \bigcup_{i\in\mathbb{N}}\mathscr{B}_{i}$.

	Since regular coordinate balls and regular coordinate half-balls are open subsets of $M$, every element of $\mathscr{B}$ is an open subset of $M$. Let $U$ be an open subset of $M$ then $U = \bigcup_{i\in\mathbb{N}}(U\cap U_{i})$. Remind that the collection of open ball $B_{r}(x)$ (where all coordinates of $x$ are rational, $r$ is a positive rational number) and open half-ball $B_{r}(x) \cap \mathbb{H}^{n}$ (where all coordinates of $x$ are rational, $x \in \partial\mathbb{H}^{n}$, $r$ is a positive rational number) is a basis for the subspace topology on $\mathbb{H}^{n}$. If $a \in U\cap U_{i}$ then the point $\varphi_{i}(a) \in \varphi_{i}(U\cap U_{i})$ is either in $\partial\mathbb{H}^{n}$ or $\operatorname{Int}\mathbb{H}^{n}$. In either case, there is an element of $\mathscr{B}_{i}$ containing $\varphi_{i}(a)$ and contained in $\varphi_{i}(U\cap U_{i})$. Therefore $\mathscr{B}$ is a countable basis for $M$.

	Thus every manifold with boundary has a countable basis of regular coordinate balls and half-balls.
\end{proof}

\section*{Local Compactness}\addcontentsline{toc}{section}{Local Compactness}

\begin{lemma}{4.65}
	Let $X$ be a locally compact Hausdorff space. If $x\in X$ and $U$ is any neighborhood of $x$, there exists a precompact neighborhood $V$ of $x$ such that $\overline{V} \subseteq U$.
\end{lemma}

\begin{proof}
	Since $X$ is a locally compact Hausdorff space, $x$ has a precompact neighborhood $W$. $\overline{W}$ is compact and $\overline{W}\smallsetminus U = \overline{W} \cap (X\smallsetminus U)$ is a closed subset of $\overline{W}$, so $\overline{W}\smallsetminus U$ is compact.

	In a Hausdorff space, disjoint compact subsets have disjoint neighborhoods, so there exist disjoint open subsets $Y, Y'\subseteq X$ such that $\set{x} \subseteq Y$ and $\overline{W}\smallsetminus U \subseteq Y'$. Define $V = W\cap Y$ then $V \subseteq W$, which implies $\overline{V} \subseteq \overline{W}$. Therefore $\overline{V}$ is compact (because it is a closed subset of the compact set $\overline{W}$). We will show that $\overline{V}\subseteq U$.

	$Y, Y'$ are disjoint and $Y, Y'\subseteq X$ so $Y\cup Y' \subseteq X$. Hence $V = W\cap Y \subseteq Y \subseteq X\smallsetminus Y'$, it follows that $\overline{V} \subseteq \overline{Y} \subseteq \overline{X\smallsetminus Y'} = X\smallsetminus Y'$. Hence $\overline{V} \subseteq \overline{W} \cap (X\smallsetminus Y') = \overline{W} \smallsetminus Y' \subseteq U$, so $V$ is a precompact neighborhood of $x$ such that $\overline{V} \subseteq U$.
\end{proof}

\begin{exercise}{4.67}
	Show that any finite product of locally compact spaces is locally compact.
\end{exercise}

\begin{proof}
	Let $X, Y$ be locally compact spaces and $\tuple{x, y}$ be a point of $X\times Y$. Because $X, Y$ are locally compact, there exist $U_{x}, K_{x} \subseteq X$ and $U_{y}, K_{y} \subseteq Y$ such that $U_{x}, U_{y}$ are open, $K_{x}, K_{y}$ are compact, and $x\in U_{x} \subseteq K_{x}, y\in U_{y} \subseteq K_{y}$. So $\tuple{x, y} \in U_{x}\times U_{y} \subseteq K_{x} \times K_{y}$, where $U_{x}\times U_{y}$ is a product open set and $K_{x}\times K_{y}$ is compact because the product of finitely many compact spaces is compact. Hence $X\times Y$ is locally compact.

	It follows from mathematical induction that any finite product of locally compact spaces is locally compact.
\end{proof}

\begin{exercise}{4.70}
	Prove Proposition 4.69: In a Baire space, every meager subset has dense complement.
\end{exercise}

\begin{quote}
	A subset $F$ of a topological space $X$ is said to be \textbf{nowhere dense} if $\overline{F}$ has a dense complement, and $F$ is said to be \textbf{meager} if it can be expressed as a union of countably many nowhere dense subsets.
\end{quote}

\begin{proof}
	Let $X$ be a Baire space and $A\subseteq X$ is a meager subset. The meager set $A$ can be expressed as a union of countable many nowhere dense subsets $\set{U_{i}}_{i\in\mathbb{N}}$. Because $U_{i}$ is nowhere dense, $X\smallsetminus\overline{U_{i}}$ is dense for every $i\in\mathbb{N}$. By De Morgan's law and $U_{i}\subseteq \overline{U_{i}}$ for every $i\in \mathbb{N}$, we obtain
	\begin{equation*}
		X\smallsetminus A = X\smallsetminus \left(\bigcup_{i\in\mathbb{N}}U_{i}\right) = \bigcap_{i\in\mathbb{N}}(X\smallsetminus U_{i}) \supseteq \bigcap_{i\in\mathbb{N}}(X\smallsetminus \overline{U_{i}})
	\end{equation*}

	Because $X$ is a Baire space and $X\smallsetminus \overline{U_{i}}$ is a dense open subset for every $i\in\mathbb{N}$, the intersection $\bigcap_{i\in\mathbb{N}} (X\smallsetminus \overline{U_{i}})$ is dense, from which we deduce that $X\smallsetminus A$ (which is a superset of the intersection) is dense. Thus every meager subset has a dense complement.
\end{proof}

\begin{example}{4.71}
	The solution set of any polynomial equation in two variables is nowhere dense in $\mathbb{R}^{2}$.

	There are points in the plane that satisfy no rational polynomial equation.
\end{example}

\begin{proof}
	Let $f \in \mathbb{R}[x, y]$ be a nonzero polynomial in two variables. $\set{0}$ is closed in $\mathbb{R}$ and the solution set of $p$ is $f^{-1}(0)$, which is closed because $f$ is continuous. Assume that the interior (which is an open set) of $f^{-1}(0)$ is nonempty, then there is an open disk of $\mathbb{R}^{2}$ on which $f$ vanishes. This implies that $f$ is the zero polynomial, which is a contradiction. Hence the interior of $f^{-1}(0)$ is empty, so it is nowhere dense.

	There are countably many polynomials with rational coefficients, and the union of their solution sets (they are nowhere dense) is a meager subset of $\mathbb{R}^{2}$, so the complement of this meager subset is dense in $\mathbb{R}^{2}$, which implies the existence of points satisfying no rational polynomial equation.
\end{proof}

\section*{Paracompactness}\addcontentsline{toc}{section}{Paracompactness}

\begin{exercise}{4.73}
	Suppose $\mathscr{A}$ is an \textit{open} cover of $X$ such that each element of $\mathscr{A}$ intersects only finitely many others. Show that $\mathscr{A}$ is locally finite. Give a counterexample to show that this need not be true when the elements of $\mathscr{A}$ are not open.
\end{exercise}

\begin{proof}
	Because $\mathscr{A}$ is an open cover of $X$, for every element of $x$, there exists $A\in\mathscr{A}$ such that $x\in A$. Moreover, $A$ is a neighborhood of $x$ and $A$ intersects at most finitely many other elements of $\mathscr{A}$. Therefore $\mathscr{A}$ is locally finite.

	Here is a counterexample: Consider the $(n+1)$-space $\mathbb{R}^{n+1}$ where $n > 1$. Define $\mathscr{A}$ to be the set of 1-dimensional subspaces of $\mathbb{R}^{n+1}$ then each element of $\mathscr{A}$ is not open and $\mathscr{A}$ covers $\mathbb{R}^{n+1}$. However, every neighborhood of the origin intersects every element of $\mathscr{A}$ and $\mathscr{A}$ has infinite elements.
\end{proof}

\subsection*{Normal Spaces}\addcontentsline{toc}{subsection}{Normal Spaces}

\begin{exercise}{4.78}
	Show that every compact Hausdorff space is normal.
\end{exercise}

\begin{exercise}{4.79}
	Show that every closed subspace of a normal space is normal.
\end{exercise}

\subsection*{Partition of Unity}\addcontentsline{toc}{subsection}{Partition of Unity}

\begin{exercise}{4.87}
	Show that every compact manifold with boundary is homeomorphic to a subset of some Euclidean space. [Hint: use the double.]
\end{exercise}

\section*{Proper Maps}\addcontentsline{toc}{section}{Proper Maps}

\section*{Problems}

\begin{problem}{4-1}\label{problem:4-1}
Show that for $n > 1$, $\mathbb{R}^{n}$ is not homeomorphic to any open subset of $\mathbb{R}$.
\end{problem}

\begin{proof}
	Let $n$ be a positive integer greater than $1$. Assume that $\mathbb{R}^{n}$ is homeomorphic to an open subset $U\subseteq \mathbb{R}$, then $U$ is nonempty and there is a homeomorphism $\varphi: \mathbb{R}^{n} \to U$. Let $p$ be a point of $U$. Since $U\subseteq \mathbb{R}$ is open, $U$ is an union of disjoint open intervals, and $x$ lies in one of those open intervals, denote such open interval by $\openinterval{a, b}$, then $\openinterval{a, b}\smallsetminus\set{p}$ is disconnected. Therefore $U\smallsetminus\set{p}$ is disconnected. Because $\varphi$ is a homeomorphism, $\varphi^{-1}(U\smallsetminus\set{p}) = \mathbb{R}^{n} \smallsetminus \set{\varphi^{-1}(p)}$.

	We will show that $\mathbb{R}^{n}\smallsetminus\set{0}$ is path-connected. Let $x, y$ be two points of $\mathbb{R}^{n}\smallsetminus\set{0}$. Because $n > 1$, there exists a nonzero vector $v\in \mathbb{R}^{n}$ such that $x, y$ are not multiples of $v$. Let $v_{x} = \frac{\abs{x}}{\abs{v}}v$ and $v_{y} = \frac{\abs{y}}{\abs{v}}v$. We will construct
	\begin{itemize}
		\item a path in $\mathbb{R}^{n}\smallsetminus\set{0}$ from $x$ to $v_{x}$

		      Note that $\abs{x} = \abs{v_{x}}$. $x = (x_{1}, \ldots, x_{n})$ and $v_{x} = (v_{x,1}, \ldots, v_{x,n})$.

		      There exist $\varphi_{x,1}, \ldots, \varphi_{x,n-1} \in \mathbb{R}$ such that
		      \begin{align*}
			      x_{1}   & = \abs{x}\cos(\varphi_{x,1})                                                 \\
			      x_{2}   & = \abs{x}\sin(\varphi_{x,1})\cos(\varphi_{x,2})                              \\
			      \cdots  &                                                                              \\
			      x_{n-1} & = \abs{x}\sin(\varphi_{x,1})\cdots\sin(\varphi_{x,n-2})\cos(\varphi_{x,n-1}) \\
			      x_{n}   & = \abs{x}\sin(\varphi_{x,1})\cdots\sin(\varphi_{x,n-2})\sin(\varphi_{x,n-1})
		      \end{align*}

		      Also there exist $\varphi_{v_{x},1}, \ldots, \varphi_{v_{x},n-1} \in \mathbb{R}$ such that
		      \begin{align*}
			      v_{x,1}   & = \abs{x}\cos(\varphi_{v_{x},1})                                                         \\
			      v_{x,2}   & = \abs{x}\sin(\varphi_{v_{x},1})\cos(\varphi_{v_{x},2})                                  \\
			      \cdots    &                                                                                          \\
			      v_{x,n-1} & = \abs{x}\sin(\varphi_{v_{x},1})\cdots\sin(\varphi_{v_{x},n-2})\cos(\varphi_{v_{x},n-1}) \\
			      v_{x,n}   & = \abs{x}\sin(\varphi_{v_{x},1})\cdots\sin(\varphi_{v_{x},n-2})\sin(\varphi_{v_{x},n-1})
		      \end{align*}

		      The maps $f_{i}: \closedinterval{0, 1} \to \mathbb{R}$ given by
		      \begin{equation*}
			      f_{i}(t) = \abs{x}\sin((1-t)\varphi_{x,1} + t\varphi_{v_{x},1})\cdots \sin((1-t)\varphi_{x,i-1} + t\varphi_{v_{x},i-1})\cos((1-t)\varphi_{x,i} + t\varphi_{v_{x},i})
		      \end{equation*}

		      if $i < n$ and
		      \begin{equation*}
			      f_{n}(t) = \abs{x}\sin((1-t)\varphi_{x,1} + t\varphi_{v_{x},1})\cdots \sin((1-t)\varphi_{x,n-1} + t\varphi_{v_{x},n-1})
		      \end{equation*}

		      are continuous. So ${(f_{1}(t))}^{2} + \cdots + {(f_{n}(t))}^{2} = \abs{x}^{2} \ne 0$ for every $t\in \closedinterval{0,1}$. Hence the map $f_{x}: \closedinterval{0, 1} \to \mathbb{R}^{n}\smallsetminus\set{0}$ given by
		      \begin{equation*}
			      f_{x}(t) = (f_{1}(t), \ldots, f_{n}(t))
		      \end{equation*}

		      is continuous, due to the characteristic property of product topology. Hence $f$ is a path in $\mathbb{R}^{n}\smallsetminus\set{0}$ from $x$ to $v_{x}$.
		\item a path in $\mathbb{R}^{n}\smallsetminus\set{0}$ from $v_{x}$ to $v_{y}$

		      The map $f: \closedinterval{0, 1} \to \mathbb{R}^{n}\smallsetminus\set{0}$ given by
		      \begin{equation*}
			      f(t) = (1 - t)v_{x} + tv_{y}
		      \end{equation*}

		      is continuous (this map is well-defined because the line segment connecting $v_{x}$ and $v_{y}$ lies entirely in $\mathbb{R}^{n}\smallsetminus\set{0}$) so there is a path in $\mathbb{R}^{n}$ from $v_{x}$ to $v_{y}$.
		\item a path in $\mathbb{R}^{n}\smallsetminus\set{0}$ from $v_{y}$ to $y$

		      Similar to the first contruction, we can construct a path $f_{y}$ in $\mathbb{R}^{n}\smallsetminus\set{0}$ from $v_{y}$ to $y$.
	\end{itemize}

	From these constructions, we deduce that there are continuous maps $g_{x}: \closedinterval{0, \frac{1}{3}} \to \mathbb{R}^{n}\smallsetminus\set{0}$ such that $g_{x}(0) = x$ and $g_{x}(1/3) = v_{x}$, $g: \closedinterval{\frac{1}{3}, \frac{2}{3}} \to \mathbb{R}^{n}\smallsetminus\set{0}$ such that $g(1/3) = v_{x}$ and $g(2/3) = v_{y}$, $g_{y}: \closedinterval{\frac{2}{3}, 1} \to \mathbb{R}^{n}\smallsetminus\set{0}$ such that $g_{y}(2/3) = v_{y}$ and $g_{y}(1) = y$. By the gluing lemma, there is a unique continuous map $f: \closedinterval{0, 1} \to \mathbb{R}^{n}\setminus \set{0}$ such that $f\vert_{\closedinterval{0, \frac{1}{3}}} = g_{x}$, $f\vert_{\closedinterval{\frac{1}{3}, \frac{2}{3}}} = g$, $f\vert_{\closedinterval{\frac{2}{3}, 1}} = g_{y}$. Hence there is a path in $\mathbb{R}^{n}\smallsetminus\set{0}$ from $x$ to $y$.

	Back to the set $\mathbb{R}^{n}\smallsetminus\set{\varphi^{-1}(p)}$. For every $c, d \in \mathbb{R}^{n}\smallsetminus\set{\varphi^{-1}(p)}$, there is a path in $\mathbb{R}^{n}\smallsetminus\set{0}$ from $c - \varphi^{-1}(p)$ to $d - \varphi^{-1}(p)$, so there is a path in $\mathbb{R}^{n}\smallsetminus\set{\varphi^{-1}(p)}$ from $c$ to $d$. Therefore $\mathbb{R}^{n}\smallsetminus\set{\varphi^{-1}(p)}$ is path-connected. Since $\varphi$ is a homeomorphism
	\begin{equation*}
		U\smallsetminus\set{p} = \varphi(\varphi^{-1}(U\smallsetminus\set{p})) = \varphi(\mathbb{R}^{n}\smallsetminus\set{\varphi^{-1}(p)})
	\end{equation*}

	is also path-connected, which is a contradiction because $U\smallsetminus\set{p}$ is disconnected.

	Thus for $n > 1$, $\mathbb{R}^{n}$ is not homeomorphic to any open subset $U\subseteq \mathbb{R}$.
\end{proof}

\begin{problem}{4-2}\label{problem:4-2}
\textsc{Invariance of Dimension, 1-Dimensional Case:} Prove that a nonempty topological space cannot be both a 1-manifold and an $n$-manifold for some $n > 1$.
\end{problem}

\begin{proof}
	Let $M$ be a nonempty $n$-manifold where $n > 1$. Assume that $M$ is also a 1-manifold. Let $x$ be a point of $M$. Because $M$ is an $n$-manifold and a 1-manifold, $x$ has a neighborhood $U$ which admits a homeomorphism $\varphi: U\to \mathbb{R}^{n}$ and a neighborhood $V$ which admits a homeomorphism $\psi: V\to \mathbb{R}$. $U\cap V$ is nonempty because it contains $x$ and it is open. The restrictions $\varphi\vert_{U\cap V}: U\cap V \to \varphi(U\cap V)$ and $\psi\vert_{U\cap V}: U\cap V \to \psi(U\cap V)$ are also a homemorphisms. Since $\varphi(U\cap V)$ is open (because a homemorphism is an open map), there is an open $n$-ball $B^{n}_{r}(\varphi(x)) \subseteq \varphi(U\cap V)$. Denote by $W$ the preimage $\varphi^{-1}(B^{n}_{r}(\varphi(x)))$ then $W\subseteq U\cap V$. The restrictions $\varphi\vert_{W}: W \to \varphi(W) = B^{n}_{r}(\varphi(x))$ and $\psi\vert_{W}: W \to \psi(W)$ are also homeomophism, so the open $n$-ball $B^{n}_{r}(\varphi(x))$ and $\psi(W)$ are homeomorphic. On the other hand, every open $n$-ball is homeomorphic to $\mathbb{R}^{n}$ and $\mathbb{R}^{n}$ is not homeomorphic to $\psi(W)$ (which is an open subset of $\mathbb{R}$) according to Problem~\ref{problem:4-1}, hence the contradiction. Thus a nonempty topological space cannot be both a 1-manifold and an $n$-manifold for some $n > 1$.
\end{proof}

\begin{problem}{4-3}
\textsc{Invariance of the Boundary, 1-Dimensional Case:} Suppose $M$ is a 1-dimensional manifold with boundary. Show that a point of $M$ cannot be both a boundary point and an interior point.
\end{problem}

\begin{proof}
	Firstly, we prove that $\mathbb{R}$ and $\mathbb{H}$ are not homeomorphic. Assume that there is a homeomorphism $f: \mathbb{H} \to \mathbb{R}$. $\mathbb{H}\smallsetminus\set{0}$ is path-connected, however, $\mathbb{R}\smallsetminus\set{f(0)}$ is not path-connected, which is a contradiction because a homeomorphism preserves path-connectedness. Hence $\mathbb{R}$ and $\mathbb{H}$ are not homeomorphic.

	Assume that $M$ has a point $x$ which is both a boundary point and an interior point. Because $x$ is a boundary point, $x$ is in the domain of a boundary chart $(U, \varphi)$ where $\varphi(x) \in \partial\mathbb{H}$ (which implies $\varphi(x) = 0$). Because $x$ is an interior point, $x$ is in the domain of an interior chart $(V, \psi)$. $U\cap V$ is a neighborhood of $x$ and the restrictions $\varphi\vert_{U\cap V}: U\cap V \to \varphi(U\cap V)$, $\psi\vert_{U\cap V}: U\cap V \to \psi(U\cap V)$. Since $\varphi(U\cap V)$ is open and $\varphi(x) = 0$, there exists $r > 0$ such that $\halfopenright{0, r} \subseteq \varphi(U\cap V)$.

	Denote $W = \varphi^{-1}(\halfopenright{0, r})$ then $W \subseteq U\cap V$. It follows that $\halfopenright{0, r}$ and $W$ are homeomorphic, $W$ and $\psi(W)$ are homeomorphic. On the other hand, $W$ is open (because $\halfopenright{0, a} \subseteq \mathbb{H}$ is open) and $\varphi$ is continuous, so $\psi(W) \subseteq \mathbb{R}$ is open (because $\psi$ is a homeomorphism, hence an open map), so $\halfopenright{0, r} \subseteq \mathbb{H}$ is homeomorphic to an open subset $A\subseteq \mathbb{R}$.

	Since $\halfopenright{0, r}$ is connected, $A$ is also connected. $A$ is a connected and open subset of $\mathbb{R}$ so $A$ is an open interval. $\halfopenright{0, r}$ is homeomorphic to $\mathbb{H}$, an open interval is homeomorphic to $\mathbb{R}$, hence $\mathbb{R}$ and $\mathbb{H}$ are homeomorphic, which is a contradiction.

	Thus a point of a 1-dimensional manifold with boundary cannot be both a boundary point and an interior point.
\end{proof}

\begin{problem}{4-4}
Show that the following topological spaces are not manifolds
\begin{enumerate}[label={(\alph*)}]
	\item the union of the $x$-axis and the $y$-axis in $\mathbb{R}^{2}$
	\item the conical surface $C\subseteq \mathbb{R}^{3}$ defined by
	      \begin{equation*}
		      C = \set{(x,y,z) : z^{2} = x^{2} + y^{2}}
	      \end{equation*}
\end{enumerate}
\end{problem}

\begin{proof}
	\begin{enumerate}[label={(\alph*)}]
		\item Assume that $M = (\mathbb{R}\times\set{0}) \cup (\set{0}\times\mathbb{R}) \subseteq \mathbb{R}^{2}$ is an $n$-manifold for some positive integer $n$.

		      Let $x$ be a nonzero real number. From the definition of manifold, $x$ has a neighborhood $U$ which is homeomorphic to an open subset of $\mathbb{R}^{n}$. On the other hand, a basis for the topology on $M$ is obtained by taking the intersection of $M$ and open balls in $\mathbb{R}^{2}$, so there is an open ball $B_{r}(\tuple{x,0})$ such that $\tuple{x,0} \in B_{r}(\tuple{x,0}) \cap M \subseteq U$. Let $\varepsilon$ be a positive number such that $\varepsilon < \min\set{r, \abs{x}}$ then
		      \begin{equation*}
			      \tuple{x, 0} \in \openinterval{x - \varepsilon, x + \varepsilon} \times \set{0} \subseteq B_{r}(\tuple{x,0}) \cap M \subseteq U.
		      \end{equation*}

		      $\openinterval{x - \varepsilon, x + \varepsilon} \times \set{0}$ is a neighborhood of $(x, 0)$ in $M$ and it is homeomorphic to an open subset of $\mathbb{R}$ and an open subset of $\mathbb{R}^{n}$. From Problem~\ref{problem:4-2}, we deduce that $n = 1$.

		      Because $M$ is a 1-manifold, $\tuple{0,0}$ has a neighborhood $V$ which admits a homeomorphism $\varphi: V \to \mathbb{R}$. There is an open ball $B_{r}(\tuple{0,0})$ in $\mathbb{R}^{2}$ such that
		      \begin{equation*}
			      \tuple{0, 0} \in B_{r}(\tuple{0, 0}) \cap M \subseteq M.
		      \end{equation*}

		      Denote $B_{r}(\tuple{0, 0}) \cap M$ by $W$ then the restriction $\varphi\vert_{W}: W \to \varphi(W)$ is also a homeomorphism. $W$ is path-connected because it is the union of path-connected sets with a point in common (the origin)
		      \begin{equation*}
			      W = (\openinterval{-r, r} \times\set{0}) \cup (\set{0} \times \openinterval{-r, r}).
		      \end{equation*}

		      so $\varphi(W) \subseteq \mathbb{R}$ is path-connected, hence it is an interval. $W\smallsetminus\set{0}$ has four path components, namely
		      \begin{equation*}
			      \openinterval{0, r}\times\set{0};\quad \openinterval{-r, 0}\times\set{0};\quad \set{0}\times\openinterval{0, r};\quad \set{0}\times\openinterval{-r, 0}
		      \end{equation*}

		      but $\varphi(W\smallsetminus\set{0}) \subseteq \mathbb{R}$ has two path components (because it is an open interval minus a point), which is a contradiction.

		      Thus $M$ is not a manifold.
		\item Assume that $C$ is an $n$-manifold for some positive integer $n$.

		      Every point on $C$ other than $(0, 0, 0)$ is of the form $(r\cos\theta, r\sin\theta, r)$ for some $r\ne 0$. Consider a point $(r\cos\theta, r\sin\theta, r)$ where $r > 0$. The set $H = \set{ \tuple{x, y, z} \in \mathbb{R}^{3} : z > 0 }$ is open in $\mathbb{R}^{3}$, so $H \cap C$ is open in $C$ (using the subspace topology). In fact, $H\cap C$ is the set of points on $C$ with positive $z$-ordinate. The map $f: H\cap C \to \mathbb{R}^{2}$ given by $f(x, y, z) = (x, y)$ is a homeomorphism, so $(r\cos\theta, r\sin\theta, r)$ has a neighborhood that is homeomorphic to $\mathbb{R}^{2}$. From Problem~\ref{problem:4-2} we deduce that $n > 1$.

		      $\tuple{0, 0, 0}$ has a neighborhood $U \subseteq C$ which is homeomorphic to $\mathbb{R}^{n}$ with the coordinate map $\varphi$. There exists $r > 0$ such that
		      \begin{equation*}
			      \tuple{0,0,0} \in B_{r}(\tuple{0,0,0}) \cap C \subseteq U \cap C
		      \end{equation*}

		      because the set of open balls is a basis for the Euclidean topology on $\mathbb{R}^{3}$ (and taking the intersection with a subset of $\mathbb{R}^{3}$, we obtain a basis for the subspace topology).

		      Denote $B_{r}(\tuple{0,0,0}) \cap C$ by $V$. $V$ is connected. $V\smallsetminus\set{\tuple{0,0,0}}$ is disconnected since it has two components, namely
		      \begin{equation*}
			      \begin{split}
				      \set{\tuple{(x, y, z)} \in \mathbb{R}^{3} : 0 < z < r} \cap C \\
				      \set{\tuple{(x, y, z)} \in \mathbb{R}^{3} : -r < z < 0} \cap C
			      \end{split}
		      \end{equation*}

		      and $\varphi(V \smallsetminus \set{\tuple{0,0,0}})$ is therefore disconnected. However, because $\varphi(V)$ is connected, $\varphi(V \smallsetminus \set{\tuple{0,0,0}}) = \varphi(V) \smallsetminus\set{\varphi(\tuple{0,0,0})}$. A connected open subset in $\mathbb{R}^{n}$ (where $n > 1$) minus a point is still connected, hence a contradiction.

		      Thus $C$ is not a manifold.
	\end{enumerate}
\end{proof}

\begin{problem}{4-5}
Let $M = \mathbb{S}^{1}\times\mathbb{R}$, and let $A = \mathbb{S}^{1} \times \set{0}$. Show that the space $M/A$ obtained by collapsing $A$ to a point is homeomorphic to the space $C$ of Problem 4-4 (b), and thus is Hausdorff and second countable but not locally Euclidean.
\end{problem}

\begin{proof}[Unrigorous Proof]
	The map $f: \mathbb{S}^{1}\times\mathbb{R} \to C$ given by
	\begin{equation*}
		f(e^{\iota\theta}, r) = (r\cos\theta, r\sin\theta, r)
	\end{equation*}

	is a quotient map (I don't have a rigorous proof for this yet). Moreover, the quotient map $q: M\to M/A$ and $f$ have the same identification, hence $M/A$ and $C$ are homeomorphism, according to the uniqueness of quotient space. Thus $C$ is Hausdorff, second countable but not locally Euclidean.
\end{proof}

\begin{problem}{4-6}
Like Problem~\ref{problem:2-22}, this problem constructs a space that is locally Euclidean and Hausdorff but not second countable. Unlike that example, however, this one is connected.
\begin{enumerate}[label={(\alph*)}]
	\item Recall that a totally ordered set is said to be well ordered if every nonempty subset has a smallest element. Show that the well-ordering theorem implies that there exists an uncountable well-ordered set $Y$ such that for every $y_{0}\in Y$, there are only countably many $y\in Y$ such that $y < y_{0}$.
	\item Now let $\mathscr{R} = Y \times \halfopenright{0,1}$, with the \textbf{dictionary order}: this means that $(y_{1}, s_{1}) < (y_{2}, s_{2})$ if either $y_{1} < y_{2}$, or $y_{1} = y_{2}$ and $s_{1} < s_{2}$. With the order topology, $\mathscr{R}$ is called the \textbf{long ray}. The \textbf{long line} $\mathscr{L}$ is the wedge sum $\mathscr{R}\vee \mathscr{R}$ obtained by identifying both copies of $(y_{0}, 0)$ with each other, where $y_{0}$ is the smallest element in $Y$. Show that $\mathscr{L}$ is locally Euclidean, Hausdorff, and first countable, but not second countable.
	\item Show that $\mathscr{L}$ is path-connected.
\end{enumerate}
\end{problem}

\begin{proof}
	Unsolved.
	\begin{enumerate}[label={(\alph*)}]
		\item This assumes the well-ordering theorem.

		      Let $X$ be an uncountable well-ordered set. If for every element of $X$, there is only countably many elements strictly less than it, then we are done. If this is not the case, the subset $X_{0}\subseteq X$ of elements $x$ such that there are uncountably many elements strictly less than $x$, is nonempty. Due to the well-ordering theorem, $X_{0}$ has a smallest element $x_{0}$. Let $Y$ be the subset of $X$ containing element strictly less than $x_{0}$ then $Y$ is uncountable and has the desired property (otherwise, it contradicts the minimality of $x_{0}$).
		\item
		\item
	\end{enumerate}
\end{proof}

\begin{note}[Characterizations of Local Connectedness]\label{note:characterizations-of-local-connectedness}
	Let $X$ be a topological space. The following statements are equivalent
	\begin{enumerate}[label={(\alph*)}]
		\item $X$ is locally connected.
		\item Every open subset of $X$ is locally connected.
		\item All components of every open subset of $X$ are open in $X$.
		\item For every $x\in X$, every neighborhood of $x$ contains a connected open set containing $x$.
		\item Every point of $X$ has a connected neighborhood basis (local basis).
	\end{enumerate}
\end{note}

\begin{proof}
	$(a)\implies (b)$ $X$ is locally connected then $X$ has a basis $\mathscr{B}$ of connected open sets. Let $U$ be an open subset of $X$ and $\mathscr{B}_{U} = \set{ B\in \mathscr{B} : B\subseteq U }$. If $x\in U$ and $V$ is a neighborhood of $x$ in $U$, then $V$ is also open in $X$, so there is $B\in\mathscr{B}$ such that $x\in B\subseteq V\subseteq U$, which implies $B\in\mathscr{B}_{U}$. Hence $\mathscr{B}_{U}$ is a basis for the subspace topology on $U$, which consists of connected open subsets of $U$. Therefore every open subset of $X$ is locally connected.

	$(b)\implies (a)$ $X$ is an open subset of $X$ so $X$ is locally connected.

	$(a)\implies (c)$ Let $U$ be an open subset of $X$, then $U$ is locally connected ($(a) \implies (b)$). Because $U$ is locally connected, every component of $U$ is open in $U$. Therefore every component of $U$ is open in $X$.

	$(c)\implies (a)$ Let $x$ be a point of $X$. For every neighborhood $U$ of $x$, all components of $U$ are open in $X$, let $C_{U,x}$ be the component of $U$ containing $x$. Therefore the collection of $C_{U,x}$ is a connected neighborhood basis of $X$, so $X$ is locally connected.

	$(a)\implies (d)$ Let $x$ be a point of $X$ and $U$ be a neighborhood of $x$. Since $X$ has a basis $\mathscr{B}$ consisting of connected open sets, there exists $B\in\mathscr{B}$ such that $x \in B\subseteq U$. Hence every neighborhood of $x$ contains a connected open set containing $x$.

	$(d)\implies (c)$ Let $U$ be an open subset of $X$ and $C$ be a component of $U$. Let $x$ be a point of $C$ then there is a connected open set $V$ such that $x\in V\subseteq U$. Since $x\in V, x\in C$ and $V, C$ are connected subsets of $U$, $V\cup C$ is a connected subset of $U$. From the maximality of $C$, it follows that $V\subseteq C$, which means there is a neighborhood of $x$ contained in $C$. This is true for every $x\in C$ so $C$ is open in $U$, hence open in $X$ (because $U\subseteq X$ is open). Hence all components of every open subset of $X$ are open in $X$.

	$(a)\implies (e)$ Let $\mathscr{B}$ be a basis of $X$ consisting connected open sets. Let $x$ be a point of $X$ and $\mathscr{B}_{x} = \set{ B\in\mathscr{B}: x\in B }$. For every neighborhood $U$ of $x$, there exists $B\in\mathscr{B}$ such that $x\in B\subseteq U$, which implies $B\in \mathscr{B}_{x}$. Hence $\mathscr{B}_{x}$ is a connected neighborhood basis of $x$.

	$(e)\implies (a)$ Since every point of $X$ has a connected neighborhood basis, the union of these neighborhood bases gives a basis for the topology on $X$, which contains connected open sets.
\end{proof}

\begin{note}[Characterizations of Local Path-Connectedness]\label{note:characterizations-of-local-path-connectedness}
	Let $X$ be a topological space. The following statements are equivalent
	\begin{enumerate}[label={(\alph*)}]
		\item $X$ is locally path-connected.
		\item Every open subset of $X$ is locally path-connected.
		\item All path components of every open subset of $X$ are open in $X$.
		\item For every $x\in X$, every neighborhood of $x$ contains a path-connected open set containing $x$.
		\item Every point of $X$ has a path-connected neighborhood basis (local basis).
	\end{enumerate}
\end{note}

\begin{proof}
	$(a)\implies (b)$ $X$ is locally path-connected then $X$ has a basis $\mathscr{B}$ of path-connected open sets. Let $U$ be an open subset of $X$ and $\mathscr{B}_{U} = \set{ B\in \mathscr{B} : B\subseteq U }$. If $x\in U$ and $V$ is a neighborhood of $x$ in $U$, then $V$ is also open in $X$, so there is $B\in\mathscr{B}$ such that $x\in B\subseteq V\subseteq U$, which implies $B\in\mathscr{B}_{U}$. Hence $\mathscr{B}_{U}$ is a basis for the subspace topology on $U$, which consists of path-connected open subsets of $U$. Therefore every open subset of $X$ is locally path-connected.

	$(b)\implies (a)$ $X$ is an open subset of $X$ so $X$ is locally connected.

	$(a)\implies (c)$ $X$ is locally path-connected so there is a basis $\mathscr{B}$ for the topology on $X$ consisting of path-connected open sets. Let $U$ be an open subset of $X$ and $C$ be a path component of $U$. For every $x\in C$, there exists $B\in\mathscr{B}$ such that $x\in B \subseteq U$. Since $C$ and $B$ have the point $x$ in common and they are path-connected, $C\cup B$ is a path-connected subset of $U$. Because of the maximality of $C$, $C\cup B = C$, so $B\subseteq C$. Hence every point of $C$ has a path-connected neighborhood contained in $C$, so $C$ is open in $X$. Therefore all path components of every open subset of $X$ are open in $X$.

	$(c)\implies (a)$ Let $x$ be a point of $X$. For every neighborhood $U$ of $x$, all path components of $U$ are open in $X$, let $P_{U,x}$ be the path component of $U$ containing $x$. The collection of $P_{U,x}$ ($x$ varies on $X$, every neighborhood $U$ of $x$) is then a basis for $X$, which consists of path-connected open sets of $X$. Hence $X$ is locally path-connected.

	$(a)\implies (d)$ Let $x$ be a point of $X$ and $U$ be a neighborhood of $x$. Since $X$ has a basis $\mathscr{B}$ consisting of path-connected open sets, there exists $B\in\mathscr{B}$ such that $x \in B\subseteq U$. Hence every neighborhood of $x$ contains a path-connected open set containing $x$.

	$(d)\implies (c)$ Let $U$ be an open subset of $X$ and $P$ be a path component of $U$. Let $x$ be a point of $P$, then there is a path-connected open set $V$ such that $x\in V\subseteq U$. Since $x\in V, x\in P$ and $V, P$ are connected subsets of $U$, $V\cup P$ is a connected subset of $U$. From the maximality of $P$, it follows that $V\subseteq P$, which means there is a neighborhood of $x$ contained in $P$. This is true for every $x\in P$ so $P$ is open in $U$, hence open in $X$ (because $U\subseteq X$ is open). Hence all path components of every open subset of $X$ are open in $X$.

	$(a)\implies (e)$ Let $\mathscr{B}$ be a basis of $X$ consisting connected open sets. Let $x$ be a point of $X$ and $\mathscr{B}_{x} = \set{ B\in\mathscr{B}: x\in B }$. For every neighborhood $U$ of $x$, there exists $B\in\mathscr{B}$ such that $x\in B\subseteq U$, which implies $B\in \mathscr{B}_{x}$. Hence $\mathscr{B}_{x}$ is a path-connected neighborhood basis of $x$.

	$(e)\implies (a)$ Since every point of $X$ has a path-connected neighborhood basis, the union of these neighborhood bases gives a basis for the topology on $X$, which contains path-connected open sets.
\end{proof}

\begin{note}[Product of Local Connected and Local Path-Connected Spaces]
	\begin{enumerate}[label={(\alph*)}]
		\item The product of finitely many local connected spaces is locally connected.
		\item The product of finitely many local path-connected spaces is locally path-connected.
	\end{enumerate}
\end{note}

\begin{proof}
	\begin{enumerate}[label={(\alph*)}]
		\item Suppose $X, Y$ are locally connected spaces.

		      Let $\mathscr{B}_{X}$ be a basis for the topology on $X$ consisting of connected open sets of $X$, $\mathscr{B}_{Y}$ be a basis for the topology on $Y$ consisting of connected open sets of $Y$.

		      For every $U\in \mathscr{B}_{X}, V\in \mathscr{B}_{Y}$, $U\times V$ is connected (because the product of two connected spaces is connected). Hence $\mathscr{B} = \set{ U\times V : U\in \mathscr{B}_{X}, V\in \mathscr{B}_{Y} }$ is a basis for the product topology on $X\times Y$ consisting of connected product open sets. Therefore $X\times Y$ is locally connected.

		      By mathematical induction, we conclude that the product of finitely many connected spaces is locally connected.
		\item The proof for local path-connectedness is similar.
	\end{enumerate}
\end{proof}

\begin{problem}{4-7}\label{problem:4-7}
Let $q: X\to Y$ be a quotient map. Show that
\begin{itemize}
	\item if $X$ is locally connected then $Y$ is locally connected,
	\item if $X$ is locally path-connected then $Y$ is locally path-connected,
	\item if $q$ is open and $X$ is locally compact, then $Y$ is locally compact.
\end{itemize}
\end{problem}

\begin{proof}
	A continuous map maps connected sets to connected sets, path-connected sets to path-connected sets, and compact sets to compact sets.

	\textbf{$X$ is locally connected.}

	Let $V$ be an open subset of $Y$, $y$ be a point of $V$, and $C_{y}$ be the component of $V$ containing $y$. We will show that $C_{y}$ is open in $Y$.

	Let $x$ be a point of $q^{-1}(C_{y})$. $q(x) \in C_{y} \subseteq V$ so $x \in q^{-1}(V)$, which is open because $q$ is continuous. Because $X$ is locally connected, $x$ has a connected neighborhood $U_{x}$ such that $x\in U_{x} \subseteq q^{-1}(V)$. Since continuous maps map connected sets to connected sets, $q(U_{x})$ is connected. Connected sets $q(U_{x})$ and $C_{y}$ have the point $q(x)$ is common so $q(U_{x}) \cup C_{y}$ is connected. On the other hand, $q(U_{x})$ and $C_{y}$ are subsets of $V$, and $C_{y}$ is a maximal connected set of $V$, so $q(U_{x}) \cup C_{y} = C_{y}$. Therefore $q(U_{x}) \subseteq C_{y}$, so $U_{x} \subseteq q^{-1}(q(U_{x})) \subseteq q^{-1}(C_{y})$, which means $U_{x}$ is a neighborhood of $x$ contained in $q^{-1}(C_{y})$.

	Because of the arbitrariness of $x$, we deduce that $q^{-1}(C_{y})$ is open. According to the definition of quotient map, $C_{y}$ is open, hence all components of every open subset of $Y$ is open in $Y$. From Note~\ref{note:characterizations-of-local-connectedness} (c), we conclude that $Y$ is locally connected.

	\textbf{$X$ is locally path-connected.}

	Let $V$ be an open subset of $Y$, $y$ be a point of $V$, and $P_{y}$ be the path component of $V$ containing $y$. We will show that $P_{y}$ is open in $Y$ and use the characterization in Note~\ref{note:characterizations-of-local-path-connectedness} (c).

	Let $x$ be a point of $q^{-1}(P_{y})$. $q(x) \in P_{y} \subseteq V$ so $x \in q^{-1}(V)$. The preimage $q^{-1}(V)$ is open because $q$ is continuous. Since $X$ is locally path-connected and $q^{-1}(V)$ is a neighborhood of $x$, there is a path-connected open set $U_{x}$ such that $x \in U_{x} \subseteq q^{-1}(V)$. Continuous maps map path-connected sets to path-connected sets so $q(U_{x})$ is path-connected. $q(U_{x})$ and $P_{y}$ are path-connected and have the point $q(x)$ in common so $q(U_{x}) \cup P_{y}$ is path-connected. Because of the maximality of $P_{y}$ (it is a path component), $q(U_{x}) \subseteq P_{y}$. Therefore $x \in U_{x} \subseteq q^{-1}(q(U_{x})) \subseteq q^{-1}(P_{y})$, which means $U_{x}$ is a neighborhood of $x$ contained in $q^{-1}(P_{y})$.

	Together with the arbitrariness of $x$, we deduce that $q^{-1}(P_{y})$ is open, so $P_{y}$ is open, since $q$ is a quotient map. Hence all path components of every open subset of $Y$ is open in $Y$, this implies $Y$ is locally path-connected.

	\textbf{$q$ is open and $X$ is locally compact.}

	Let $y$ be an element of $Y$ then there exists $x\in X$ such that $q(x) = y$. Since $X$ is locally compact, there is an open subset $U$ and a compact subset $K$ of $X$ such that $x \in U \subseteq K$. Because $q$ is continuous, $q(K)$ is a compact subset of $Y$. Because $q$ is an open map, $q(U)$ is an open subset of $Y$. Therefore $y = q(x) \in q(U) \subseteq q(K)$. Together with the arbitrariness of $y$, we conclude that $Y$ is locally compact.
\end{proof}

\begin{problem}{4-8}\label{problem:4-8}
Show that a locally connected topological space is homeomorphic to the disjoint union of its components.
\end{problem}

\begin{proof}
	Let $X$ be a locally connected topological space and $\mathscr{C}$ be the collection of its components. Denote by $Y$ the space $\coprod_{C\in\mathscr{C}}C$ with the disjoint union topology.

	Consider the map $f: X\to Y$ where $f(x)$ is defined to be $\iota_{C}(x)$ where $C$ is the component containing $x$. $f$ is well-defined because the components of a topological space give a partition of the given space. $f$ is bijective by definition.

	Let $U$ be an open subset of $X$, then $U = \bigcup_{C\in\mathscr{C}}(U\cap C)$ and $\iota_{C}(U\cap C)$ is open in $C$. Therefore $f(U)$ is open in $Y$. Hence $f$ is an open map.

	Let $V$ be an open subset of $Y$, then $V\cap C \subseteq C$ is open for every $C\in\mathscr{C}$. The preimage of $V$ is
	\begin{equation*}
		f^{-1}(V) = f^{-1}\left( \coprod_{C\in\mathscr{C}} (V\cap C) \right) = \bigcup_{C\in\mathscr{C}}\iota_{C}^{-1}(V\cap C).
	\end{equation*}

	$\iota_{C}^{-1}(V\cap C)$ is open in $C$ for every $C\in\mathscr{C}$. On the other hand, \textbf{$X$ is locally connected so every component of $X$ is open}, so $C$ is an open subset of $X$ for every $C\in\mathscr{C}$. From Exercise~\ref{exercise:3.6} (Proposition 3.5), it follows that $\iota_{C}^{-1}(V\cap C)$ is open in $X$. Hence $f^{-1}(V)$ is open in $X$. Therefore $f$ is continuous.

	Thus $f$ is a homeomorphism, which implies that $X$ is homeomorphic to the disjoint union of its components.
\end{proof}

\begin{problem}{4-9}
Show that every $n$-manifold is homeomorphic to a disjoint union of countably many connected $n$-manifolds, and every $n$-manifold with boundary is homeomorphic to a disjoint union of countably many connected $n$-manifolds with (possibly empty) boundaries.
\end{problem}

\begin{proof}
	Every manifold (with or without boundary) is locally path-connected (hence locally connected).

	Suppose $M$ is an $n$-manifold (with or without boundary). From Problem~\ref{problem:4-8}, $M$ is locally connected, so $M$ is homeomorphic to the disjoint union of its components. Because $M$ is locally connected, every component of $M$ is open. Assume for the sake of contradiction that $M$ has uncountably many components. $M$ is second countable so it has a countable basis $\mathscr{B}$. Each component $C$ of $M$ is open and nonempty so it contains an element $B_{C}$ of $\mathscr{B}$. Since the components of $M$ are disjoint and uncountably many, it follows that $\mathscr{B}$ is uncountably infinite, which is a contradiction. Hence $M$ has countably many components.

	Let $C$ be a component of $M$ and $x$ be an element of $C$.

	\textbf{$M$ is an $n$-manifold without boundary.}

	Because $M$ is an $n$-manifold, $x$ is contained in some coordinate ball $B$. Moreover, every coordinate ball of an $n$-manifold is connected (because it is homeomorphic to an open ball of $\mathbb{R}^{n}$) so $B$ is contained in a single component of $M$. Since $x\in B, x\in C$, then $B \subseteq C$, so $C$ is locally Euclidean of dimension $n$. On the other hand, $C\subseteq M$ is second countable and Hausdorff (with the subspace topology) so $C$ is a connected $n$-manifold. Therefore $M$ is homeomorphic to a disjoint union of countably many connected $n$-manifolds.

	\textbf{$M$ is an $n$-manifold with boundary.}

	Because $M$ is an $n$-manifold with boundary, $x$ is contained in some regular coordinate ball or half-ball $B$. Every regular coordinate ball (or half-ball) of an $n$-manifold is connected, so $B$ is connected and is contained in a single component. Since $x\in B, x\in C$, then $B\subseteq C$, so $C$ is locally Euclidean of dimension $n$. On the other hand, $C\subseteq M$ is second countable and Hausdorff (with the subspace topology) so $C$ is a connected $n$-manifold. Therefore $M$ is homeomorphic to a disjoint union of countably many connected $n$-manifolds with (possibly empty) boundaries.
\end{proof}

\begin{problem}{4-10}
Let $S$ be the square $I\times I$ with the order topology generated by the dictionary order.
\begin{enumerate}[label={(\alph*)}]
	\item Show that $S$ has the least upper bound property.
	\item Show that $S$ is connected.
	\item Show that $S$ is locally connected, but not locally path-connected.
\end{enumerate}
\end{problem}

\begin{proof}
	Denote by $\pi_{1}, \pi_{2}$ the canonical maps from $I\times I$ to $I$ given by $\pi_{1}(x, y) = x$ and $\pi_{2}(x, y) = y$. $I$ has the least upper bound property.
	\begin{enumerate}[label={(\alph*)}]
		\item Let $A$ be a nonempty subset of $S$. $\pi_{1}(A)$ is a nonempty subset of $I$ so it has a least upper bound $x_{0}$.

		      If $x_{0} \notin \pi_{1}(A)$ then $\tuple{x_{0}, 0}$ is an upper bound of $A$. If $\tuple{x, y} < \tuple{x_{0}, 0}$ then $x < x_{0}$ so there is $x' \in \pi_{1}(A)$ such that $x < x' < x_{0}$, which means any element of $S$ less than $\tuple{x_{0}, 0}$ is not an upper bound of $A$. So $\tuple{x_{0}, 0}$ is the least upper bound of $A$.

		      If $x_{0} \in \pi_{1}(A)$, define $B = \set{ \tuple{x, y} \in A : x = x_{0} }$ then $B$ is nonempty. $\pi_{2}(B)$ is a nonempty subset of $I$ so it has a least upper bound $y_{0}$. If $\tuple{x, y} < \tuple{x_{0}, y_{0}}$ then either $x = x_{0}$ and $y < y_{0}$ or $x < x_{0}$. In case $x = x_{0}$ and $y < y_{0}$, there is $y' \in \pi_{2}(B)$ such that $y < y'$, so $\tuple{x, y} < \tuple{x_{0}, y'} \leq \tuple{x_{0}, y_{0}}$. In case $x < x_{0}$, for every $y' \in \pi_{2}(B)$, $\tuple{x, y} < \tuple{x_{0}, y'} \leq \tuple{x_{0}, y_{0}}$. In either case, $\tuple{x, y}$ is not an upper bound. So $\tuple{x_{0}, y_{0}}$ is the least upper bound of $A$.

		      Hence $S$ has the least upper bound property.
		\item We will prove a stronger result, which is a generalization for Proposition 4.11: A subset $I$ of a topological space $X$ where
		      \begin{itemize}
			      \item $X$ is a linear continuum
			      \item $X$ is endowed with the order topology
			      \item $X$ has the least upper bound property
		      \end{itemize}

		      is connected if and only if $I$ is a singleton or an interval.

		      The proof for this result is similar to that of Proposition 4.11. In fact, the three properties of $X$ listed above are precisely the properties of $\mathbb{R}$ which are used in the proof of Proposition 4.11 in the book. However, we write a proof in details as follows.

		      A basis for the order topology on $X$ is the collection of open intervals of $X$.

		      If $I$ is a singleton then it is connected, so assume that $I$ has at least two elements.

		      Suppose $I$ is an interval. Assume for the sake of contradiction that $I$ is not connected then there are open subsets $U, V\subseteq X$ such that $U\cap I$ and $V\cap I$ disconnect $I$. Choose $a \in U\cap I$ and $b\in V\cap I$ and assume (interchanging $U, V$ if necessary) that $a < b$. Since $I$ is an interval, the closed interval $\closedinterval{a, b}$ is contained in $I$. Because $U, V$ are open, there exist $a'\in U\cap I$, $b'\in V\cap I$, such that $a < a', b' < b$ and $\halfopenright{a, a'} \subseteq U\cap I$ and $\halfopenleft{b', b}\subseteq V\cap I$. From this, we deduce that $a' \leq b'$ (since otherwise, $\halfopenright{a, a'}$ and $\halfopenleft{b', b}$ are not disjoint, so $U$ and $V$ are not disjoint.)

		      Let $c = \sup (U\cap \closedinterval{a, b})$. Since $\halfopenright{a, a'} \subseteq U\cap I \subseteq U$ and $\halfopenright{a, b'} \subseteq \closedinterval{a, b}$, it follows that $\halfopenright{a, a'} \subseteq U\cap \closedinterval{a, b}$, so $a' = \sup\halfopenright{a, a'} \leq \sup (U\cap \closedinterval{a, b}) = c$. Assume that $c > b'$ then $b'$ is not an upper bound of $U\cap\closedinterval{a, b}$, which means there exists $d \in U\cap\closedinterval{a, b}$ such that $d > b'$, which implies $U\cap I$ and $V\cap I$ are not disjoint (because they both contain $d$), which is a contradiction, so $c\leq b'$. Hence $c \in \closedinterval{a', b'} \subseteq \openinterval{a, b} \subseteq \closedinterval{a, b} \subseteq I$.

		      If $c\in U$ then $c\in U\cap\closedinterval{a, b}$, there is a neighborhood $c\in \openinterval{c_{1}, c_{2}} \subseteq U$, so $c\in \openinterval{c_{1}, c_{2}} \cap \closedinterval{a, b} \subseteq U\cap\closedinterval{a, b}$, this contradicts $c$ being the least upper bound of $U\cap\closedinterval{a, b}$.

		      Otherwise, $c\in V$, then there is a neighborhood $c\in \openinterval{c_{1}, c_{2}} \subseteq V$, so $c\in \openinterval{c_{1}, c_{2}} \cap I \subseteq V\cap I$, which means $\openinterval{c_{1}, c_{2}}$ is disjoint from $U\cap I$ (because $U\cap I$ and $V\cap I$ are disjoint). Again, this contradicts the definition of $c$ because $\closedinterval{c_{1}, c_{2}}$ (a neighborhood of the supremum of $U\cap \closedinterval{a, b}$) must intersect $U\cap \closedinterval{a, b}$, hence intersects $U\cap I$.

		      Hence the assumption is false, so $I$ is connected.

		      Conversely, suppose that $I$ is not an interval, so there exist $a < c < b$ such that $a, b\in I$ but $c\notin I$. The sets $\openinterval{-\infty, c}\cap I$ and $\openinterval{c, \infty}\cap I$ disconnects $I$ so $I$ is not connected.

		      Back to the problem. $S$ is the interval $\closedinterval{\tuple{0,0}, \tuple{1,1}}$ so it is connected.
		\item Because the collection of open intervals of $S$ is a basis for the order topology on $S$ and every open interval of $S$ is connected, it follows that $S$ is locally connected.

		      Assume that $S$ is path-connected, so there is a continuous map $f: \closedinterval{0, 1} \to S$ such that $f(0) = \tuple{0,0}$ and $f(1) = \tuple{1,1}$. $f(\closedinterval{0,1})$ is connected because the image of a connected set under a continuous map. $f(\closedinterval{0,1})$ is connected so it is an interval, according to the proof in part (b) and since it contains $\tuple{0, 0}$ and $\tuple{1, 1}$, we conclude that $f(\closedinterval{0, 1})$ contains every point of $S$.

		      For each $x\in I$, $U_{x} = f^{-1}(\set{x} \times \openinterval{0,1})$ is a nonempty open subset of $\closedinterval{0,1}$, because $f$ is continuous and $\set{x} \times \openinterval{0, 1} = \openinterval{\tuple{x, 0}, \tuple{x, 1}}$ is an open interval (which is an open subset) in $S$. For each $x\in I$, choose a rational number $q_{x} \in U_{x}$. Because the sets $U_{x}$ are pairwise disjoint, the map $x\mapsto q_{x}$ is an injection from $I$ into $\mathbb{Q}\cap I$. This means the cardinality of $I$ is less than or equal to the cardinality of $\mathbb{Q}\cap I$ but this is a contradiction because $I$ is uncountable and $\mathbb{Q}\cap I$ is countable. Therefore the assumption is false and $S$ is not path-connected.

		      Assume that $S$ is locally path-connected. In a locally path-connected space, connectedness and path-connectedness are equivalent, so $S$ is path-connected, which is a contradiction. Hence $S$ is not locally path-connected.
	\end{enumerate}
\end{proof}

\begin{problem}{4-11}
Let $X$ be a topological space, and let $CX$ be the cone on $X$ (see Example 3.53).
\begin{enumerate}[label={(\alph*)}]
	\item Show that $CX$ is path-connected.
	\item Show that $CX$ is locally connected if and only if $X$ is, and locally path-connected if and only if $X$ is.
\end{enumerate}
\end{problem}

\begin{proof}
	$CX = (X \times I)/(X \times\set{0})$. Denote by $q$ the quotient map $X\times I \to (X \times I)/(X \times\set{0})$.

	\begin{enumerate}[label={(\alph*)}]
		\item $q(\tuple{y, 0})$ and $q(\tuple{y, a})$ are path-connected for every $a\in I$, because the map $f: I \to CX$ given by $f(t) = q(\tuple{y, t\cdot a})$ is continuous since $f = q\circ g$ where $g: I \to X\times I$, $g(t) = \tuple{y, t\cdot a}$. Hence $f$ is continuous and $f(0) = q(\tuple{y, 0})$ and $f(1) = q(\tuple{y, a})$, so there is a path from $q(\tuple{y, 0})$ to $q(\tuple{y, a})$.

		      Let $q(\tuple{x, a})$ and $q(\tuple{y, b})$ be two arbitrary points of $CX$. According to the previous paragraph, there is a path from $q(\tuple{x, a})$ to $q(\tuple{x, 0}) = q(\tuple{y, 0})$ and a path from $q(\tuple{y, 0})$ to $q(\tuple{y, b})$, so there is a path from $q(\tuple{x, a})$ and $q(\tuple{y, b})$.

		      Hence $CX$ is path-connected.
		\item Suppose $X$ is locally {(path-)}connected then $X\times I$ is locally {(path-)}connected, since $I$ is locally {(path-)}connected and the product of two local {(path-)}connected spaces is locally {(path-)}connected. From Problem~\ref{problem:4-7}, we conclude that $CX = (X\times I)/(X\times\set{0})$ is locally {(path-)}connected.

		      Conversely, suppose $CX$ is locally {(path-)}connected. Let $x\in X$ and $U$ be a neighborhood of $x$. The product set $U\times\halfopenleft{0, 1}$ is open in $X\times I$ and saturated (any two points of $U\times\halfopenleft{0, 1}$ are not identified by $q$, since it doesn't intersect $X\times\set{0}$), so $q(U\times \halfopenleft{0,1})$ is open in $CX$. Because $CX$ is locally {(path-)}connected, there is a {(path-)}connected open set $V_{x}$ such that $q(\tuple{x, 1}) \in V_{x} \subseteq q(U\times \halfopenleft{0,1})$. Therefore $\tuple{x, 1} \in q^{-1}(V_{x}) \subseteq q^{-1}(q(U\times \halfopenleft{0, 1})) = U\times\halfopenleft{0,1}$. The set $q^{-1}(V_{x})$ is saturated (by definition) and open in $X\times I$ (since $q$ is continuous).

		      Since the restriction of a quotient map to a saturated open set is still a quotient map, it follows that $q\vert_{q^{-1}(V_{x})}: q^{-1}(V_{x}) \to V_{x}$ is a quotient map. On the other hand, any two points of $q^{-1}(V_{x})$ are not identified by $q$ (since its superset $U\times\halfopenleft{0,1}$ has this property) so $q\vert_{q^{-1}(V_{x})}$ is injective. Every injective quotient map is a homeomorphism, so $q\vert_{q^{-1}(V_{x})}$ is a homeomorphism. Because of this homeomorphism $V_{x}$ being {(path-)}connected implies $q^{-1}(V_{x})$ being {(path-)}connected. The canonical projection $\pi_{1}: X\times I \to X$ is open and continuous, so $\pi_{1}(q^{-1}(V_{x}))$ is a {(path-)}connected open subset of $X$. Hence $x \in \pi_{1}(q^{-1}(V_{x})) \subseteq \pi_{1}(U\times\halfopenleft{0,1}) = U$, from which we conclude that $X$ is locally {(path-)}connected.
	\end{enumerate}
\end{proof}

\begin{problem}{4-12}
Suppose $X$ is a topological space and $S\subseteq X$ is a subset that is both open and closed in $X$. Show that $S$ is a union of components of $X$.
\end{problem}

\begin{proof}
	If $S$ is empty then the statement is true. Suppose that $S$ is nonempty, then $S$ intersects some component $C$ of $X$.

	Assume that $C\smallsetminus S$ is nonempty. We have $C\smallsetminus S = C\cap (X\smallsetminus S)$ and $C = (C\cap S) \cup (C\cap (X\smallsetminus S))$ where $C\cap S$ is nonempty because $S$ intersects $C$, and $C\smallsetminus S$ is nonempty by assumption. On the other hand, $C\cap S$ is open in $C$ (due to the definition of subspace topology and $S\subseteq X$ is open) and $C\cap (X\smallsetminus S)$ is open in $C$ (due to the  definition of subspace topology and $X\smallsetminus S \subseteq X$ is open). So $C\cap S$ and $C\cap (X\smallsetminus S)$ disconnect $C$, which is a contradiction since $C$ is a component.

	Hence $C\smallsetminus S$ is empty and it follows that $C\subseteq S$. Therefore any component of $X$ that intersects $S$ is contained in $S$, thus $S$ is a union of components of $X$.
\end{proof}

\chapter{Eigenvalues and Eigenvectors}

\section{Invariant Subspaces}

% chapter5:sectionA:exercise1
\begin{exercise}
    Suppose $T\in\lmap{V}$ and $U$ is a subspace of $V$.
    \begin{enumerate}[label={(\alph*)}]
        \item Prove that if $U\subseteq \kernel{T}$, then $U$ is invariant under $T$.
        \item Prove that if $\range{T}\subseteq U$, then $U$ is invariant under $T$.
    \end{enumerate}
\end{exercise}

\begin{proof}
    I skip this exercise.
\end{proof}
\newpage

% chapter5:sectionA:exercise2
\begin{exercise}
    Suppose that $T\in\lmap{V}$ and $V_{1}, \ldots, V_{m}$ are subspaces of $V$ invariant under $T$. Prove that $V_{1} + \cdots + V_{m}$ is invariant under $T$.
\end{exercise}

\begin{proof}
    I skip this exercise.
\end{proof}
\newpage

% chapter5:sectionA:exercise3
\begin{exercise}
    Suppose $T\in\lmap{V}$. Prove that the intersection of every collection of subspaces of $V$ invariant under $T$ is invariant under $T$.
\end{exercise}

\begin{proof}
    I skip this exercise.
\end{proof}
\newpage

% chapter5:sectionA:exercise4
\begin{exercise}
    Prove or give a counterexample: If $V$ is finite-dimensional and $U$ is a subspace of $V$ that is invariant under every operator on $V$, then $U = \{0\}$ or $U = V$.
\end{exercise}

\begin{proof}
    The answer is affirmative.

    If $U = \{0\}$ or $U = V$, then $U$ is invariant under every operator on $V$.

    Let $U$ be a subspace of $V$ such that $U\ne \{0\}$ and $U\ne V$ and invariant under $T$.

    Let $u_{1}, \ldots, u_{m}$ be a basis of $U$. Extend this list to obtain a basis of $V$ and let it be
    \[
        u_{1}, \ldots, u_{m}, v_{1}, \ldots, v_{n}.
    \]

    I define the operator $T$ on $V$ as follows: $Tu_{1} = v_{1}$, $Tu_{k} = u_{k}$ for $k = 2,\ldots, m$, and $Tv_{i} = v_{i}$ for $i = 1,\ldots, n$. Then $U$ is not invariant under $T$.

    Hence if $V$ is finite-dimensional and $U$ is a subspace of $V$ that is invariant under every operator on $V$, then $U = \{0\}$ or $U = V$.
\end{proof}
\newpage

% chapter5:sectionA:exercise5
\begin{exercise}
    Suppose $T\in\lmap{\mathbb{R}^{2}}$ is defined by $T(x, y) = (-3y, x)$. Find the eigenvalues of $T$.
\end{exercise}

\begin{proof}
    Suppose $\lambda$ is an eigenvalue of $T$, then there exists $(x, y)$ such that $(-3y, x) = (\lambda x, \lambda y)$.
    \[
        -3x = -3\lambda y = \lambda^{2}x
    \]

    it follows that $x(\lambda^{2} + 3) = 0$ so $x = 0$. Because $x = \lambda y$, we deduce that $\lambda y = 0$. However $y\ne 0$ because $(x, y)$ is an eigenvector, so $\lambda = 0$. Therefore $x = y = 0$.

    Hence $T$ has no (real) eigenvalues.
\end{proof}
\newpage

% chapter5:sectionA:exercise6
\begin{exercise}
    Define $T\in\lmap{\mathbb{F}^{2}}$ by $T(w, z) = (z, w)$. Find all eigenvalues and eigenvectors of $T$.
\end{exercise}

\begin{proof}
    The eigenvalues of $T$ are $1$ and $-1$.

    The eigenvectors of $T$ with respect to $1$ are of the form $(z, z)$.

    The eigenvectors of $T$ with respect to $-1$ are of the form $(z, -z)$.
\end{proof}
\newpage

% chapter5:sectionA:exercise7
\begin{exercise}
    Define $T\in\lmap{\mathbb{F}^{3}}$ by $T(z_{1}, z_{2}, z_{3}) = (2z_{2}, 0, 5z_{3})$. Find all eigenvalues and eigenvectors of $T$.
\end{exercise}

\begin{proof}
    The only eigenvalues of $T$ are $0$ and $5$.

    The eigenvectors of $T$ with respect to $0$ are of the form $(z, 0, 0)$.

    The eigenvectors of $T$ with respect to $0$ are of the form $(0, 0, z)$.
\end{proof}
\newpage

% chapter5:sectionA:exercise8
\begin{exercise}
    Suppose $P\in\lmap{V}$ is such that $P^{2} = P$. Prove that if $\lambda$ is an eigenvalue of $P$, then $\lambda = 0$ or $\lambda = 1$.
\end{exercise}

\begin{proof}
    If $\lambda$ is an eigenvalue of $P$, then for every eigenvector $v$ of $P$ with respect to $\lambda$, we have
    \[
        \lambda v = Pv = P^{2}v = \lambda^{2}v
    \]

    Hence $\lambda = \lambda^{2}$, which implies $\lambda = 0$ or $\lambda = 1$.
\end{proof}
\newpage

% chapter5:sectionA:exercise9
\begin{exercise}
    Define $T: \mathscr{P}(\mathbb{R})\to \mathscr{P}(\mathbb{R})$ by $Tp = p'$. Find all eigenvalues and eigenvectors of $T$.
\end{exercise}

\begin{proof}
    If $\deg P > 0$, then $\deg p' = (\deg p) - 1$. So every nonconstant polynomial is not an eigenvector of $T$.

    If $\deg P = 0$, then $p' = 0 = 0p$. So every nonzero constant polynomial is an eigenvector of $T$ with respect to the eigenvalue $0$.
\end{proof}
\newpage

% chapter5:sectionA:exercise10
\begin{exercise}
    Define $T\in\lmap{\mathscr{P}(\mathbb{R})}$ by $(Tp)(x) = xp'(x)$ for all $x\in \mathbb{R}$. Find all eigenvalues and eigenvectors of $T$.
\end{exercise}

The original problem is $T\in\lmap{\mathscr{P}_{4}(\mathbb{R})}$

\begin{proof}
    Let $\lambda$ be an eigenvalue of $T$ and $p(x) = a_{0} + a_{1}x + \cdots + a_{n}x^{n}$ an eigenvector corresponding to $\lambda$.
    \[
        \begin{split}
            xp'(x) = a_{1}x + 2a_{2}x^{2} + \cdots + na_{n}x^{n} \\
            \lambda p(x) = \lambda a_{0} + \lambda a_{1}x + \cdots + \lambda a_{n}x^{n}.
        \end{split}
    \]

    By comparing the coefficients, we obtain $0 = \lambda a_{0}$, $a_{1} = \lambda a_{1}$, \ldots, $na_{n} = \lambda a_{n}$.

    So $\lambda = 0$ and $p$ is a nonzero constant polynomial, or $\lambda = n$ and $p(x) = x^{n}$.

    Hence the eigenvalues of $T$ are nonnegative integer $n$ and the corresponding eigenvectors are of the form $kx^{n}$ (where $k$ is a nonzero constant in $\mathbb{F}$).
\end{proof}
\newpage

% chapter5:sectionA:exercise11
\begin{exercise}\label{chapter5:sectionA:exercise11}
    Suppose $V$ is finite-dimensional, $T\in\lmap{V}$, and $\alpha\in\mathbb{F}$. Prove that there exists $\delta > 0$ such that $T - \lambda I$ is invertible for all $\lambda\in\mathbb{F}$ such that $0 < \abs{\alpha - \lambda} < \delta$.
\end{exercise}

\begin{proof}
    $T$ has at most $(\dim V)$ eigenvalues. Let $\lambda_{1}, \ldots, \lambda_{m}$ be the distinct eigenvalues of $T$.

    If $\alpha$ is not an eigenvalue of $T$, I choose
    \[
        \delta = \min\left\{ \abs{\alpha - \lambda_{i}}: 1\leq i\leq m \right\}.
    \]

    Then for all $\lambda$ such that $0 < \abs{\alpha - \lambda} < \delta$, $\lambda$ is not an eigenvalue of $T$.

    If $\alpha = \lambda_{k}$ for some $k$ in $1, \ldots, m$, I choose
    \[
        \delta = \min\left\{ \abs{\alpha - \lambda_{i}}: 1\leq i\leq m\wedge i\ne k \right\}.
    \]

    Then for all $\lambda$ such that $0 < \abs{\alpha - \lambda} < \delta$, $\lambda$ is not an eigenvalue of $T$.

    On the other hand, $\lambda\in\mathbb{F}$ is not an eigenvalue of $T$ if and only if $T - \lambda I$ is invertible.

    Hence there exists $\delta > 0$ such that $T - \lambda I$ is invertible for all $\lambda\in\mathbb{F}$ such that $0 < \abs{\alpha - \lambda} < \delta$.
\end{proof}
\newpage

% chapter5:sectionA:exercise12
\begin{exercise}
    Suppose $V = U\oplus W$, where $U$ and $W$ are nonzero subspaces of $V$. Define $P\in\lmap{V}$ by $P(u + w) = u$ for each $u\in U$ and each $w\in W$. Find all eigenvalues and eigenvectors of $P$.
\end{exercise}

\begin{proof}
    Let $\lambda$ be an eigenvalue of $P$ and $u + w$ be an eigenvector corresponding to $\lambda$ (where $u\in U$, $w\in W$), then $P(u + w) = \lambda(u + w)$. According to the definition of $P$, $\lambda (u + w) = u$, so $(\lambda - 1)u + \lambda w = 0$.

    Because $U\cap W = \{0\}$, it follows that $(\lambda - 1)u = 0$ and $\lambda w = 0$. Therefore $\lambda = 1$ or $\lambda = 0$.

    Hence the eigenvalues of $P$ are $1$ and $0$. The eigenvectors corresponding to $1$ is $ku$ (where $k\in\mathbb{F}$ and $k\ne 0$). The eigenvectors corresponding to $0$ is $kw$ (where $k\in\mathbb{F}$ and $k\ne 0$).
\end{proof}
\newpage

% chapter5:sectionA:exercise13
\begin{exercise}
    Suppose $T\in\lmap{V}$. Suppose $S\in\lmap{V}$ is invertible.
    \begin{enumerate}[label={(\alph*)}]
        \item Prove that $T$ and $S^{-1}TS$ have the same eigenvalues.
        \item What is the relationship between the eigenvectors of $T$ and the eigenvectors of $S^{-1}TS$?
    \end{enumerate}
\end{exercise}

\begin{proof}
    \begin{enumerate}[label={(\alph*)}]
        \item If $\lambda$ is an eigenvalue of $T$.

              Let $v$ be an eigenvector corresponding to $\lambda$. Since $S$ is invertible, there exists $w\in V$ such that $Sw = v$. Therefore
              \[
                  (S^{-1}TS)(w) = (S^{-1}T)(v) = S^{-1}(\lambda v) = \lambda w.
              \]

              Hence $\lambda$ is also an eigenvalue of $S^{-1}TS$.

              If $\lambda$ is an eigenvalue of $S^{-1}TS$.

              Let $w$ be an eigenvector corresponding to $\lambda$, then $(S^{-1}TS)(w) = \lambda w$. Let $v = Sw$. Then apply $S$ to $(S^{-1}TS)(w)$ and $\lambda w$, we obtain that $(TS)(w) = S(\lambda w)$ and
              \[
                  Tv = T(Sw) = S(\lambda w) = \lambda Sw = \lambda v.
              \]

              Hence $\lambda$ is also an eigenvalue of $T$.

              Thus $T$ and $S^{-1}TS$ have the same eigenvalues.
        \item Let $A$ be the set of eigenvectors of $T$ with respect to an eigenvalue $\lambda$. Let $B$ be the set of eigenvectors of $S^{-1}TS$ with respect to the eigenvalue $\lambda$.

              There is a bijection from $A$ onto $B$ defined by $v\mapsto S^{-1}v$. In other word, $v$ is an eigenvector of $T$ corresponding to $\lambda$ if and only if $S^{-1}v$ is an eigenvector of $S^{-1}TS$ corresponding to $\lambda$.
    \end{enumerate}
\end{proof}
\newpage

% chapter5:sectionA:exercise14
\begin{exercise}
    Give an example of an operator on $\mathbb{R}^{4}$ that has no (real) eigenvalues.
\end{exercise}

\begin{proof}
    I define $T\in\lmap{\mathbb{R}^{4}}$ as follows:
    \[
        T(x_{1}, x_{2}, x_{3}, x_{4}) = (-x_{2}, x_{1}, -x_{4}, x_{3}).
    \]

    If $\lambda$ is an eigenvalue of $T$ and $(x_{1}, x_{2}, x_{3}, x_{4})$ is an eigenvector corresponding to $\lambda$, then
    \[
        (-x_{2}, x_{1}, -x_{4}, x_{3}) = \lambda (x_{1}, x_{2}, x_{3}, x_{4}).
    \]

    Therefore $x_{2} = -\lambda x_{1} = -\lambda^{2}x_{2}$ and $x_{4} = -\lambda x_{3} = -\lambda^{2}x_{4}$. Because $\lambda^{2} + 1\ne 0$, it follows that $x_{2} = 0$ and $x_{4} = 0$. So $x_{1} = 0$ and $x_{3} = 0$. This contradicts $(x_{1}, x_{2}, x_{3}, x_{4})$ being an eigenvector.

    Hence $T$ has no (real) eigenvalues.
\end{proof}
\newpage

% chapter5:sectionA:exercise15
\begin{exercise}\label{chapter5:sectionA:exercise15}
    Suppose $V$ is finite-dimensional, $T\in\lmap{V}$, and $\lambda\in\mathbb{F}$. Show that $\lambda$ is an eigenvalue of $T$ if and only if $\lambda$ is an eigenvalue of the dual operator $T'\in\lmap{V'}$.
\end{exercise}

\begin{proof}
    The range of a linear map and the range of its dual map have the same dimension. $T' - \lambda I'$ is the dual map of $T - \lambda I$. According to the fundamental theorem of linear maps
    \begin{align*}
        \dim\kernel{(T - \lambda I)} & = \dim V - \dim\range{(T - \lambda I)}    \\
                                     & = \dim V' - \dim\range{(T' - \lambda I')} \\
                                     & = \dim\kernel{(T' - \lambda I')}.
    \end{align*}

    $\lambda$ is an eigenvalue of $T$ if and only if $\kernel{(T - \lambda I)}\ne \{0\}$. $\lambda$ is an eigenvalue of $T'$ if and only if $\kernel{(T' - \lambda I')}\ne \{0\}$. Therefore $\lambda$ is an eigenvalue of $T$ if and only if $\lambda$ is an eigenvalue of $T'$.
\end{proof}
\newpage

% chapter5:sectionA:exercise16
\begin{exercise}
    Suppose $v_{1}, \ldots, v_{n}$ is a basis of $V$ and $T\in\lmap{V}$. Prove that if $\lambda$ is an eigenvalue of $T$, then
    \[
        \abs{\lambda}\leq n\max\left\{ \abs{{\mathcal{M}(T)}_{j,k}}: 1\leq j, k\leq n \right\},
    \]

    where ${\mathcal{M}(T)}_{j,k}$ denotes the entry in row $j$, column $k$ of the matrix of $T$ with respect to the basis $v_{1}, \ldots, v_{n}$.
\end{exercise}

\begin{proof}
    Let $A = \mathcal{M}(T)$ and $v = x_{1}v_{1} + \cdots + x_{n}v_{n}$ be an eigenvector of $T$ with respect to $\lambda$, then
    \[
        A\begin{pmatrix}x_{1} \\ \vdots \\ x_{n}\end{pmatrix} = \begin{pmatrix}
            A_{1,1}x_{1} + \cdots + A_{1,n}x_{n} \\
            \vdots                               \\
            A_{n,1}x_{1} + \cdots + A_{n,n}x_{n}
        \end{pmatrix} = \begin{pmatrix}
            \lambda x_{1} \\ \vdots \\ \lambda x_{n}
        \end{pmatrix}
    \]

    By Cauchy-Schwarz inequality
    \begin{align*}
        \abs{\lambda}^{2}\abs{x_{1}}^{2} + \cdots + \abs{\lambda}^{2}\abs{x_{n}}^{2} & = \sum^{n}_{i=1}{\abs{A_{i,1}x_{1} + \cdots + A_{i,n}x_{n}}}^{2}                                                \\
                                                                                     & \leq \sum^{n}_{i=1}(\abs{x_{1}}^{2} + \cdots + \abs{x_{n}}^{2})(\abs{A_{i,1}}^{2} + \cdots + \abs{A_{i,n}}^{2}) \\
                                                                                     & = (\abs{x_{1}}^{2} + \cdots + \abs{x_{n}}^{2})\sum_{j,k}\abs{A_{j,k}}^{2}                                       \\
                                                                                     & \leq n^{2}(\abs{x_{1}}^{2} + \cdots + \abs{x_{n}}^{2})\max\left\{\abs{A_{j,k}}^{2}: 1\leq j, k\leq n\right\}
    \end{align*}

    Because $\abs{x_{1}}^{2} + \cdots + \abs{x_{n}}^{2}\ne 0$, we deduce that
    \[
        \lambda^{2}\leq n^{2}\max\left\{\abs{A_{j,k}}^{2}: 1\leq j, k\leq n\right\}.
    \]

    Hence
    \[
        \abs{\lambda}\leq n\max\left\{\abs{A_{j,k}}: 1\leq j, k\leq n\right\}.\qedhere
    \]
\end{proof}
\newpage

% chapter5:sectionA:exercise17
\begin{exercise}
    Suppose $\mathbb{F} = \mathbb{R}$, $T\in\lmap{V}$, and $\lambda\in\mathbb{R}$. Prove that $\lambda$ is an eigenvalue of $T$ if and only if $\lambda$ is an eigenvalue of the complexification of $T_{\mathbb{C}}$.
\end{exercise}

\begin{proof}
    $(\Rightarrow)$ $\lambda$ is an eigenvalue of $T$.

    Let $v$ be an eigenvector of $T$ corresponding to $\lambda$. Then according to the definition of $T_{\mathbb{C}}$
    \[
        T_{\mathbb{C}}(v + \iota v) = Tv + \iota Tv = \lambda v + \iota \lambda v.
    \]

    Hence $\lambda$ is also an eigenvalue of $T_{\mathbb{C}}$.

    $(\Rightarrow)$ $\lambda$ is an eigenvalue of $T_{\mathbb{C}}$.

    Let $u + \iota v$ be an eigenvector of $T_{\mathbb{C}}$ corresponding to $\lambda$. Because $\lambda\in\mathbb{R}$ and due to the definition of eigenvalue and eigenvector
    \[
        \begin{split}
            T_{\mathbb{C}}(u + \iota v) = Tu + \iota Tv, \\
            T_{\mathbb{C}}(u + \iota v) = \lambda u + \iota \lambda v.
        \end{split}
    \]

    Hence $Tu = \lambda u$ and $Tv = \lambda v$. Because $u + \iota v\ne 0 + \iota 0$, it follows that at least one vector in the list $u, v$ is nonzero. Hence $\lambda$ is also an eigenvalue of $T$.
\end{proof}
\newpage

% chapter5:sectionA:exercise18
\begin{exercise}
    Suppose $\mathbb{F} = \mathbb{R}$, $T\in\lmap{V}$, and $\lambda\in\mathbb{C}$. Prove that $\lambda$ is an eigenvalue of the complexification $T_{\mathbb{C}}$ if and only if $\conj{\lambda}$ is an eigenvalue of $T_{\mathbb{C}}$.
\end{exercise}

\begin{proof}
    Let $a = \operatorname{Re}\lambda$ and $b = \operatorname{Im}\lambda$.

    $(\Rightarrow)$ $\lambda$ is an eigenvalue of $T_{\mathbb{C}}$.

    Let $u + \iota v$ be an eigenvector of $T_{\mathbb{C}}$ corresponding to $\lambda$, then
    \[
        Tu + \iota Tv = T_{\mathbb{C}}(u + \iota v) = \lambda (u + \iota v) = (a + b\iota)(u + \iota v) = (au - bv) + \iota(av + bu).
    \]

    So $Tu = au - bv$ and $Tv = av + bu$. Then
    \[
        T_{\mathbb{C}}(u - \iota v) = Tu - \iota Tv = (au - bv) - \iota (av + bu) = (a - b\iota)(u - \iota v) = \conj{\lambda}(u - \iota v).
    \]

    Moreover, $u + \iota v\ne 0 + \iota 0$ so $u - \iota v\ne 0 + \iota 0$. Hence $\conj{\lambda}$ is an eigenvalue of $T_{\mathbb{C}}$.

    $(\Leftarrow)$ $\conj{\lambda}$ is an eigenvalue of $T_{\mathbb{C}}$.

    According to the previous part, $\conj{\conj{\lambda}}$ is an eigenvalue of $T_{\mathbb{C}}$. Hence $\lambda$ is an eigenvalue of $T_{\mathbb{C}}$.

    \bigskip

    Thus $\lambda$ is an eigenvalue of $T_{\mathbb{C}}$ if and only if $\conj{\lambda}$ is an eigenvalue of $T_{\mathbb{C}}$.
\end{proof}
\newpage

% chapter5:sectionA:exercise19
\begin{exercise}
    Show that the forward shift operator $T\in\lmap{\mathbb{F}^{\infty}}$ defined by
    \[
        T(z_{1}, z_{2}, \ldots) = (0, z_{1}, z_{2}, \ldots)
    \]

    has no eigenvalues.
\end{exercise}

\begin{proof}
    Assume $\lambda$ is an eigenvalue of $T$. Let $(z_{1}, z_{2}, \ldots)$ be an eigenvector of $T$ corresponding to $\lambda$.
    \[
        T(z_{1}, z_{2}, \ldots) = (\lambda z_{1}, \lambda z_{2}, \ldots).
    \]

    Identify $(\lambda z_{1}, \lambda z_{2}, \ldots)$ and $(0, z_{1}, z_{2}, \ldots)$, we obtain $\lambda z_{1} = 0$ and $\lambda z_{n+1} = z_{n}$. If $\lambda\ne 0$, then $z_{n} = 0$ for every positive integer $n$. If $\lambda = 0$, then $z_{n} = 0$ for every positive integer $n$. This contradicts $(z_{1}, z_{2}, \ldots)$ being an eigenvector.

    Thus $T$ has no eigenvalues.
\end{proof}
\newpage

% chapter5:sectionA:exercise20
\begin{exercise}
    Define the backward shift operator $S\in\lmap{\mathbb{F}^{\infty}}$ by
    \[
        S(z_{1}, z_{2}, z_{3}, \ldots) = (z_{2}, z_{3}, \ldots)
    \]

    \begin{enumerate}[label={(\alph*)}]
        \item Show that every element of $\mathbb{F}$ is an eigenvalue of $S$.
        \item Find all eigenvectors of $S$.
    \end{enumerate}
\end{exercise}

\begin{proof}
    \begin{enumerate}[label={(\alph*)}]
        \item Let $\lambda$ be an element of $\mathbb{F}$, then
              \[
                  S(1, \lambda, \lambda^{2}, \ldots) = (\lambda, \lambda^{2}, \ldots) = \lambda (1, \lambda, \lambda^{2}, \ldots)
              \]

              So $\lambda$ is an eigenvalue of $S$.
        \item Let $\lambda$ be an eigenvalue of $S$. Let $(z_{1}, z_{2}, z_{3}, \ldots)$ be an eigenvector of $S$ corresponding to $\lambda$. Then
              \[
                  (z_{2}, z_{3}, \ldots) = (\lambda z_{1}, \lambda z_{2}, \ldots)
              \]

              So $z_{n} = \lambda^{n-1}z_{1}$ for every positive integer $n > 1$. Because $(z_{1}, z_{2}, z_{3}, \ldots)$ is nonzero, it follows that $z_{1}\ne 0$. Hence the eigenvectors of $S$ corresponding to the eigenvalue $\lambda$ are of the form
              \[
                  (z_{1}, \lambda z_{1}, \lambda^{2}z_{1}, \ldots)
              \]

              where $z_{1}\ne 0$.\qedhere
    \end{enumerate}
\end{proof}
\newpage

% chapter5:sectionA:exercise21
\begin{exercise}
    Suppose $T\in\lmap{V}$ is invertible.
    \begin{enumerate}[label={(\alph*)}]
        \item Suppose $\lambda\in\mathbb{F}$ with $\lambda \ne 0$. Prove that $\lambda$ is an eigenvalue of $T$ if and only if $\frac{1}{\lambda}$ is an eigenvalue of $T^{-1}$.
        \item Prove that $T$ and $T^{-1}$ have the same eigenvectors.
    \end{enumerate}
\end{exercise}

\begin{proof}
    \begin{enumerate}[label={(\alph*)}]
        \item If $\lambda$ is an eigenvalue of $T$, then there exists a nonzero vector $v$ such that $Tv = \lambda v$. So $v = T^{-1}(\lambda v)$, and it follows that $T^{-1}v = \frac{1}{\lambda}v$. Hence $\frac{1}{\lambda}$ is an eigenvalue of $T^{-1}$.

              If $\frac{1}{\lambda}$ is an eigenvalue of $T^{-1}$, then $\frac{1}{1/\lambda} = \lambda$ is an eigenvalue of ${(T^{-1})}^{-1} = T$.
        \item If $v$ is an eigenvector of $T$ with respect to the eigenvalue $\lambda$, then $v$ is also an eigenvector of $T^{-1}$ with respect to the eigenvalue $\frac{1}{\lambda}$.

              If $v$ is an eigenvector of $T^{-1}$ with respect to the eigenvalue $\frac{1}{\lambda}$, then $v$ is also an eigenvector of $T$ with respect to the eigenvalue $\lambda$.

              Hence $T$ and $T^{-1}$ have the same eigenvectors.\qedhere
    \end{enumerate}
\end{proof}
\newpage

% chapter5:sectionA:exercise22
\begin{exercise}
    Suppose $T \in \lmap{V}$ and there exist nonzero vectors $u$ and $w$ in $V$ such that
    \[
        Tu = 3w \qquad\text{and}\qquad Tw = 3u.
    \]

    Prove that $3$ or $-3$ is an eigenvalue of $T$.
\end{exercise}

\begin{proof}
    $T^{2}u = T(Tu) = T(3w) = 9u$. So $(T^{2} - 9I)(u) = 0$, it follows that
    \[
        (T - 3I)(T + 3I)(u) = 0\qquad (T + 3I)(T - 3I)(u) = 0.
    \]

    Assume that $(T + 3I)(u)\ne 0$ and $(T - 3I)(u)\ne 0$. Then let $v_{1} = (T + 3I)(u)$ and $v_{2} = (T - 3I)(u)$. Therefore $(T - 3I)(v_{1}) = 0$ and $(T + 3I)(v_{2}) = 0$. So $\kernel{(T - 3I)}\ne \{0\}$ and $\kernel{(T + 3I)}\ne \{0\}$, which is a contradiction.

    Hence  $(T + 3I)(u) = 0$ or $(T - 3I)(u) = 0$. Thus $3$ or $-3$ is an eigenvalue of $T$.
\end{proof}
\newpage

% chapter5:sectionA:exercise23
\begin{exercise}
    Suppose $V$ is finite-dimensional and $S, T \in \lmap{V}$. Prove that $ST$ and $TS$ have the same eigenvalues.
\end{exercise}

\begin{proof}
    Assume $\lambda$ is a nonzero eigenvalue of $ST$ and $v$ is an eigenvector of $ST$ corresponding to $\lambda$, then $(ST)(v) = \lambda v$, and $(TS)(Tv) = \lambda Tv$. Because $\lambda$ and $v$ are nonzero, $S(Tv) = \lambda v$ is nonzero, so $Tv\ne 0$. Hence $\lambda$ is an eigenvalue of $TS$.

    Assume $\lambda$ is a nonzero eigenvalue of $TS$ and $v$ is an eigenvector of $TS$ corresponding to $\lambda$, then $(TS)(v) = \lambda v$, and $(ST)(Sv) = \lambda Sv$. Because $\lambda$ and $v$ are nonzero, $T(Sv) = \lambda v$ is nonzero, so $Sv\ne 0$. Hence $\lambda$ is an eigenvalue of $ST$.

    According to Exercise~\ref{chapter3:sectionD:exercise11}, $ST$ is invertible iff $S, T$ are invertible, and $TS$ is invertible iff $S, T$ are invertible. Therefore $ST$ is not invertible iff $TS$ is not invertible. Equivalently, $(ST - 0I)$ is not invertible iff $(TS - 0I)$ is not invertible. So $0$ is an eigenvalue of $ST$ iff $0$ is an eigenvalue of $TS$.

    Hence $ST$ and $TS$ have the same eigenvalues.
\end{proof}
\newpage

% chapter5:sectionA:exercise24
\begin{exercise}
    Suppose $A$ is an $n$-by-$n$ matrix with entries in $\mathbb{F}$. Define $T\in\lmap{\mathbb{F}^{n}}$ by $Tx = Ax$, where elements of $\mathbb{F}^{n}$ are thought of as $n$-by-$1$ column vectors.
    \begin{enumerate}[label={(\alph*)}]
        \item Suppose the sum of the entries in each row of $A$ equals $1$. Prove that $1$ is an eigenvalue of $T$.
        \item Suppose the sum of the entries in each column of $A$ equals $1$. Prove that $1$ is an eigenvalue of $T$.
    \end{enumerate}
\end{exercise}

\begin{proof}
    \begin{enumerate}[label={(\alph*)}]
        \item \[
                  \begin{pmatrix}
                      A_{1,1} & \cdots & A_{1,n} \\
                      \vdots  & \ddots & \vdots  \\
                      A_{n,1} & \cdots & A_{n,n}
                  \end{pmatrix}
                  \begin{pmatrix}
                      1      \\
                      \vdots \\
                      1
                  \end{pmatrix}
                  = \begin{pmatrix}
                      A_{1,1} + \cdots + A_{1,n} \\
                      \vdots                     \\
                      A_{n,1} + \cdots + A_{n,n}
                  \end{pmatrix}
                  = \begin{pmatrix}
                      1      \\
                      \vdots \\
                      1
                  \end{pmatrix}
              \]

              So $1$ is an eigenvalue of $T$.
        \item Let $T'$ be the dual map of $T$, then $T'x = A^{T}x$. The sum of the entries in each column of $A$ equals $1$ so the sum of entries in each row of $A^{T}$ equals $1$. Due to (a), it follows that $1$ is an eigenvalue of $T'$. According to Exercise~\ref{chapter5:sectionA:exercise15}, $1$ is also an eigenvalue of $T$.
    \end{enumerate}
\end{proof}
\newpage

% chapter5:sectionA:exercise25
\begin{exercise}\label{chapter5:sectionA:exercise25}
    Suppose $T \in \lmap{V}$ and $u, w$ are eigenvectors of $T$ such that $u + w$ is also an eigenvector of $T$. Prove that $u$ and $w$ are eigenvectors of $T$ corresponding to the same eigenvalue.
\end{exercise}

\begin{proof}
    Assume that $u$ and $w$ are eigenvectors of $T$ corresponding to eigenvalues $\lambda_{1}$ and $\lambda_{2}$, respectively.

    Because $u + w$ is also an eigenvector of $T$, there exists $\lambda\in\mathbb{F}$ such that $T(u + w) = \lambda(u + w)$. So
    \[
        \lambda(u + w) = T(u + w) = Tu + Tw = \lambda_{1}u + \lambda_{2}w.
    \]

    Therefore $(\lambda - \lambda_{1})u + (\lambda - \lambda_{2})w = 0$.

    If $\lambda - \lambda_{1}$ and $\lambda - \lambda_{2}$ are both zero, then $u$ and $w$ correspond to the same eigenvalue.

    If If $\lambda - \lambda_{1}$ and $\lambda - \lambda_{2}$ are not both zero, then $u$ and $w$ are linearly dependent, which means at least one vector is a scalar multiple of the other. Therefore they correspond to the same eigenvalue.

    Hence $u$ and $w$ are eigenvectors of $T$ corresponding to the same eigenvalue.
\end{proof}
\newpage

% chapter5:sectionA:exercise26
\begin{exercise}\label{chapter5:sectionA:exercise26}
    Suppose $T \in \lmap{V}$ is such that every nonzero vector in $V$ is an eigenvector of $T$. Prove that $T$ is a scalar multiple of the identity operator.
\end{exercise}

\begin{proof}
    According to Exercise~\ref{chapter5:sectionA:exercise25}, for every nonzero vector $v$, there exists $\lambda\in\mathbb{F}$ such that $Tv = \lambda v$. Of course, $T0 = \lambda 0$.

    Hence $T$ is a scalar multiple of the identity operator.
\end{proof}
\newpage

% chapter5:sectionA:exercise27
\begin{exercise}
    Suppose that $V$ is finite-dimensional and $k \in \{1, \ldots, \dim V - 1\}$. Suppose $T \in \lmap{V}$ is such that every subspace of $V$ of dimension $k$ is invariant under $T$. Prove that $T$ is a scalar multiple of the identity operator.
\end{exercise}

\begin{proof}
    Let $n = \dim V$.

    The statement is true for $k = 1$, which is Exercise~\ref{chapter5:sectionA:exercise26}.

    Assume that the statement is true for $k = p$, where $1\leq p < \dim V - 1$.

    Let $U$ be a subspace of $V$ of dimension $p$. Let $v_{1}, \ldots, v_{p}$ be a basis of $U$. Extend this list to create a basis of $V$ and let it be
    \[
        v_{1}, \ldots, v_{p}, \ldots, v_{n}
    \]

    The subspaces $\operatorname{span}(v_{1}, \ldots, v_{p}, v_{p+1})$ and $\operatorname{span}(v_{1}, \ldots, v_{p}, v_{p+2})$ are invariant under $T$. On the other hand
    \[
        \operatorname{span}(v_{1}, \ldots, v_{p}, v_{p+1})\cap \operatorname{span}(v_{1}, \ldots, v_{p}, v_{p+2}) = \operatorname{span}(v_{1}, \ldots, v_{p}) = U.
    \]

    If $u\in U$, then $Tu$ is in $\operatorname{span}(v_{1}, \ldots, v_{p}, v_{p+1})$ and $\operatorname{span}(v_{1}, \ldots, v_{p}, v_{p+2})$ so $Tu\in U$, which means $U$ is invariant under $T$. According to the induction hypothesis, $T$ is a scalar multiple of the identity operator.

    Thus, due to the principle of mathematical induction, if every subspace of $V$ of dimension $k$ (where $1\leq k < \dim V - 1$) is invariant under $T$, then $T$ is a scalar multiple of the identity operator.
\end{proof}
\newpage

% chapter5:sectionA:exercise28
\begin{exercise}
    Suppose $V$ is finite-dimensional and $T\in \lmap{V}$. Prove that $T$ has at most $1 + \dim\range{T}$ distinct eigenvalues.
\end{exercise}

\begin{proof}
    Let $\lambda_{1}, \ldots, \lambda_{m}$ be the distinct eigenvalues of $T$. Let $v_{1}, \ldots, v_{m}$ be eigenvectors corresponding to $\lambda_{1}, \ldots, \lambda_{m}$. Then $v_{1}, \ldots, v_{m}$ is linearly independent.

    If $\lambda_{1}, \ldots, \lambda_{m}$ are nonzero, then
    \begin{align*}
        m & = \dim\operatorname{span}(v_{1}, \ldots, v_{m})                       \\
          & = \dim\operatorname{span}(\lambda_{1}v_{1}, \ldots, \lambda_{m}v_{m}) \\
          & = \dim\operatorname{span}(Tv_{1}, \ldots, Tv_{m})                     \\
          & \leq \dim\range{T} < 1 + \dim\range{T}.
    \end{align*}

    If there is one eigenvalue equals $0$, I assume it is $\lambda_{1}$, then
    \begin{align*}
        m & = \dim\operatorname{span}(v_{1}, \ldots, v_{m})                           \\
          & = 1 + \dim\operatorname{span}(v_{2}, \ldots, v_{m})                       \\
          & = 1 + \dim\operatorname{span}(\lambda_{2}v_{2}, \ldots, \lambda_{m}v_{m}) \\
          & = 1 + \dim\operatorname{span}(Tv_{2}, \ldots, Tv_{m})                     \\
          & \leq 1 + \dim\range{T}.
    \end{align*}

    Thus $T$ has at most $1 + \dim\range{T}$ distinct eigenvalues.
\end{proof}
\newpage

% chapter5:sectionA:exercise29
\begin{exercise}
    Suppose $T\in\lmap{\mathbb{R}^{3}}$ and $-4$, $5$ and $\sqrt{7}$ are eigenvalues of $T$. Prove that there exists $x\in\mathbb{R}^{3}$ such that $Tx - 9x = (-4, 5, \sqrt{7})$.
\end{exercise}

\begin{proof}
    There exist nonzero vectors $v_{1}, v_{2}, v_{3}$ which are eigenvectors of $T$ correspond to  $-4$, $5$ and $\sqrt{7}$, respectively. Moreover, $v_{1}, v_{2}, v_{3}$ is linearly independent, so $v_{1}, v_{2}, v_{3}$ is a basis of $\mathbb{R}^{3}$. So there exists $b_{1}, b_{2}, b_{3}\in\mathbb{R}$ such that
    \[
        b_{1}v_{1} + b_{2}v_{2} + b_{3}v_{3} = (-4, 5, \sqrt{7}).
    \]

    $Tv_{1} = -4v_{1}$, $Tv_{2} = 5v_{2}$, $Tv_{3} = \sqrt{7}v_{3}$. Let $x = a_{1}v_{1} + a_{2}v_{2} + a_{3}v_{3}$ where $a_{1}, a_{2}, a_{3}\in\mathbb{R}$.
    \begin{align*}
        Tx - 9x & = (-4a_{1})v_{1} + (5a_{2})v_{2} + (\sqrt{7}a_{3})v_{3} - (9a_{1})v_{1} - (9a_{2})v_{2} - (9a_{3})v_{3} \\
                & = (-13a_{1})v_{1} + (-4a_{2})v_{2} + (\sqrt{7} - 9)a_{3}v_{3}
    \end{align*}

    Let $a_{1} = \frac{b_{1}}{-13}$, $a_{2} = \frac{b_{2}}{-4}$, $a_{3} = \frac{b_{3}}{\sqrt{7} - 9}$, then $Tx - 9x = (-4, 5, \sqrt{7})$.
\end{proof}
\newpage

% chapter5:sectionA:exercise30
\begin{exercise}
    Suppose $T\in\lmap{V}$ and $(T - 2I)(T - 3I)(T - 4I) = 0$. Suppose $\lambda$ is an eigenvalue of $T$. Prove that $\lambda = 2$ or $\lambda = 3$ or $\lambda = 4$.
\end{exercise}

\begin{proof}
    Let $v$ be an eigenvector of $T$ corresponding to $\lambda$.
    \begin{align*}
        (T - 2I)(T - 3I)(T - 4I)(v) & = (T - 2I)(T - 3I)(Tv - 4v)                                          \\
                                    & = (T - 2I)(T - 3I)((\lambda - 4)v)                                   \\
                                    & = (T - 2I)((\lambda - 4)\lambda v - 3(\lambda - 4)v)                 \\
                                    & = (T - 2I)((\lambda - 3)(\lambda - 4)v)                              \\
                                    & = \lambda (\lambda - 3)(\lambda - 4)v - 2(\lambda - 3)(\lambda - 4)v \\
                                    & = (\lambda - 2)(\lambda - 3)(\lambda - 4)v.
    \end{align*}

    Because $(T - 2I)(T - 3I)(T - 4I) = 0$ and $v$ is nonzero, it follows that
    \[
        (\lambda - 2)(\lambda - 3)(\lambda - 4) = 0.
    \]

    So $\lambda = 2$ or $\lambda = 3$ or $\lambda = 4$.
\end{proof}
\newpage

% chapter5:sectionA:exercise31
\begin{exercise}
    Given an example of $T\in\lmap{\mathbb{R}^{2}}$ such that $T^{4} = -I$.
\end{exercise}

\begin{proof}
    Let $T(x_{1}, x_{2}) = \left(\frac{x_{1}}{\sqrt{2}} + \frac{x_{2}}{\sqrt{2}}, \frac{-x_{1}}{\sqrt{2}} + \frac{x_{2}}{\sqrt{2}}\right)$.
    \begin{align*}
        T^{2}(x_{1}, x_{2}) & = (x_{2}, -x_{1}),                         \\
        T^{4}(x_{1}, x_{2}) & = T^{2}(x_{2}, -x_{1}) = (-x_{1}, -x_{2}).
    \end{align*}

    So $T^{4} = -I$.
\end{proof}
\newpage

% chapter5:sectionA:exercise32
\begin{exercise}
    Suppose $T\in\lmap{V}$ has no eigenvalues and $T^{4} = I$. Prove that $T^{2} = -I$.
\end{exercise}

\begin{proof}
    $T^{4} - I = 0$ implies $(T^{2} - I)(T^{2} + I) = 0$. Then either $T^{2} = I$ or $T^{2} = -I$. If $T^{2} = I$, then $(T - I)(T + I) = 0$, which implies $1$ or $-1$ is an eigenvalue of $T$. However, since $T$ has no eigenvalues, $T^{2}\ne I$. Hence $T^{2} = -I$.
\end{proof}
\newpage

% chapter5:sectionA:exercise33
\begin{exercise}
    Suppose $T\in\lmap{V}$ and $m$ is a positive integer.
    \begin{enumerate}[label={(\alph*)}]
        \item Prove that $T$ is injective if and only if $T^{m}$ is injective.
        \item Prove that $T$ is surjective if and only if $T^{m}$ is surjective.
    \end{enumerate}
\end{exercise}

\begin{proof}
    I prove these using mathematical induction and the results: $ST$ is injective implies $T$ is injective; $ST$ is surjective implies $S$ is surjective; composition of two injections is an injection; composition of two surjections is a surjection.

    \begin{enumerate}[label={(\alph*)}]
        \item When $m = 1$, the statement is true.

              Assume when $m = n$, the statement is true.

              If $T^{n+1}$ is injective, then $T^{n}T$ is injective, so $T$ is injective.

              If $T$ is injective, then $T^{n}$ is injective (induction hypothesis). So $T^{n+1} = T^{n}T$ is injective.

              Thus, according to the principle of mathematical induction, for every positive integer $m$, $T$ is injective if and only if $T^{m}$ is injective.
        \item When $m = 1$, the statement is true.

              Assume when $m = n$, the statement is true.

              If $T^{n+1}$ is surjective, then $TT^{n}$ is surjective, so $T$ is surjective.

              If $T$ is surjective, then $T^{n}$ is surjective (induction hypothesis). So $T^{n+1} = T^{n}T$ is surjective.

              Thus, according to the principle of mathematical induction, for every positive integer $m$, $T$ is surjective if and only if $T^{m}$ is surjective.
    \end{enumerate}
\end{proof}
\newpage

% chapter5:sectionA:exercise34
\begin{exercise}
    Suppose $V$ is finite-dimensional and $v_{1}, \ldots, v_{m} \in V$. Prove that the list $v_{1} , \ldots, v_{m}$ is linearly independent if and only if there exists $T \in \lmap{V}$ such that $v_{1} , \ldots, v_{m}$ are eigenvectors of $T$ corresponding to distinct eigenvalues.
\end{exercise}

In this exercise, $\mathbb{F}$ must have infinite elements.

\begin{proof}
    If there exists a linear map $T\in\lmap{V}$ such that $v_{1} , \ldots, v_{m}$ are eigenvectors of $T$ corresponding to distinct eigenvalues, then these vectors are linearly independent.

    If the list $v_{1} , \ldots, v_{m}$ is linearly independent, we can add vectors to this list to obtain a basis of $V$, and let such a basis be
    \[
        v_{1} , \ldots, v_{m}, v_{m+1}, \ldots, v_{m+n}
    \]

    Let $\lambda_{1}, \ldots, \lambda_{m}$ be distinct elements of $\mathbb{F}$. I define the linear map $T\in\lmap{V}$ as follows: $Tv_{i} = \lambda_{i}v_{i}$ for $i = 1, \ldots, m$ and $Tv_{i} = 0$ for $i > m$. So $v_{1} , \ldots, v_{m}$ are eigenvectors of $T$ corresponding to distinct eigenvalues $\lambda_{1}, \ldots, \lambda_{m}$.
\end{proof}
\newpage

% chapter5:sectionA:exercise35
\begin{exercise}
    Suppose that $\lambda_{1}, \ldots, \lambda_{n}$ is a list of distinct real numbers. Prove that the list $e^{\lambda_{1}x} , \ldots, e^{\lambda_{n}x}$ is linearly independent in the vector space of real-valued functions on $\mathbb{R}$.
\end{exercise}

\begin{proof}
    Let $D$ be the linear operator on the vector space of differentiable real-valued functions on $\mathbb{R}$ defined by $Df = f'$. Because $De^{\lambda_{i}x} = \lambda_{i}e^{\lambda_{i}x}$ so $e^{\lambda_{i}x}$ is an eigenvector of $D$ corresponding to the eigenvalue $\lambda_{i}$ for $i = 1, \ldots, n$.

    Because $\lambda_{1}, \ldots, \lambda_{n}$ are pairwise distinct, it follows that $e^{\lambda_{1}x} , \ldots, e^{\lambda_{n}x}$ is linearly independent in the vector space of differentiable real-valued functions on $\mathbb{R}$, and also in the vector space of real-valued functions on $\mathbb{R}$.
\end{proof}
\newpage

% chapter5:sectionA:exercise36
\begin{exercise}
    Suppose that $\lambda_{1}, \ldots, \lambda_{n}$ is a list of distinct positive numbers. Prove that the list $\cos{(\lambda_{1}x)} , \ldots, \cos{(\lambda_{n}x)}$ is linearly independent in the vector space of real-valued functions on $\mathbb{R}$.
\end{exercise}

\begin{proof}
    Let $D$ be the linear operator on the vector space of twice differentiable real-valued functions on $\mathbb{R}$ defined by $Df = f'$. Because $D^{2}(\cos(\lambda_{i}x)) = -\lambda_{i}^{2}\cos(\lambda_{i}x)$, $\cos(\lambda_{i}x)$ is an eigenvalue of $D^{2}$ corresponding to the eigenvalue $\lambda_{i}^{2}$ for $i = 1, \ldots, n$.

    Because $\lambda_{1}^{2}, \ldots, \lambda_{n}^{2}$ are pairwise distinct, it follows that $\cos{(\lambda_{1}x)} , \ldots, \cos{(\lambda_{n}x)}$ is linearly independent in the vector space of twice differentiable real-valued functions on $\mathbb{R}$, and also in the vector space of real-valued functions on $\mathbb{R}$.
\end{proof}
\newpage

% chapter5:sectionA:exercise37
\begin{exercise}
    Suppose $V$ is finite-dimensional and $T \in \lmap{V}$. Define $A \in \lmap{\lmap{V}}$ by
    \[
        \mathcal{A}(S) = TS
    \]

    for each $S\in\lmap{V}$. Prove that the set of eigenvalues of $T$ equals the set of eigenvalues of $\mathcal{A}$.
\end{exercise}

\begin{proof}
    Let $\lambda$ be an eigenvalue of $T$ and $v$ be a corresponding eigenvector. Let $v_{1}, \ldots, v_{n}$ be a basis of $V$. I define the linear map $S\in\lmap{V}$ as follows: $Sv_{i} = v$ for $i = 1, \ldots, n$. Then
    \[
        (TS)(v_{i}) = Tv = \lambda v = \lambda Sv_{i}
    \]

    for $i = 1, \ldots, n$. Hence $TS = \lambda S$, which means $\lambda$ is an eigenvalue of $\mathcal{A}$.

    \bigskip
    Let $\lambda$ be an eigenvalue of $\mathcal{A}$ and $S$ be a corresponding eigenvector. Then $(TS)(v) = \lambda Sv$ for all $v\in V$. Therefore $T(Sv) = \lambda Sv$ for all $v\in V$. Because $S\ne 0$, then there exists $v_{0}\in V$ such that $Sv_{0}\ne 0$. So $T(Sv_{0}) = \lambda Sv_{0}$ and we conclude that $\lambda$ is an eigenvalue of $T$.

    Thus the set of eigenvalues of $T$ equals the set of eigenvalues of $\mathcal{A}$.
\end{proof}
\newpage

% chapter5:sectionA:exercise38
\begin{exercise}
    Suppose $V$ is finite-dimensional, $T\in\lmap{V}$, and $U$ is a subspace of $V$ invariant under $T$. The \textit{quotient operator} $T/U\in\lmap{V/U}$ is defined by
    \[
        (T/U)(v + U) = Tv + U
    \]

    for each $v\in V$.
    \begin{enumerate}[label={(\alph*)}]
        \item Show that the definition of $T/U$ makes sense (which requires using the
              condition that $U$ is invariant under $T$) and show that $T/U$ is an operator
              on $V/U$.
        \item Show that each eigenvalue of $T/U$ is an eigenvalue of $T$.
    \end{enumerate}
\end{exercise}

\begin{proof}
    \begin{enumerate}[label={(\alph*)}]
        \item Assume $v + U = w + U$, then $v - w\in U$. Because $U$ is invariant under $T$, $Tv - Tw\in U$, so $Tv + U = Tw + U$. So
              \[
                  (T/U)(v + U) = Tv + U = Tw + U = (T/U)(w + U).
              \]

              Hence the definition of $T/U$ make sense ($T/U$ is well-defined).

              Moreover, $T/U$ is a linear map. So $T/U$ is an operator on $V/U$.
        \item Assume $\lambda$ is an eigenvalue of $T/U$ and $v + U$ be a corresponding eigenvector. According to the definition of eigenvector, $v\notin U$. Then $(T/U)(v + U) = Tv + U = \lambda v + U$, which implies $Tv - \lambda v \in U$.

              Let $Tv - \lambda v = u$. For $u_{1}\in U$,
              \[
                  T(v + u_{1}) = Tv + Tu_{1} = \lambda v + u + Tu_{1}.
              \]

              Because $v\notin U$ and $u_{1}\in U$, it follows that $v + u_{1}\notin U$, so $v + u_{1}$ is nonzero.

              If the restriction on $U$ of $T - \lambda I$ is not invertible, then $T - \lambda I$ is also not invertible. So $\lambda$ is an eigenvalue of $T$.

              If the restriction on $U$ of $T - \lambda I$ is invertible, then there exists $u_{1}$ such that $-u = (T - \lambda I)(u_{1})$. Therefore $T(v + u_{1}) = \lambda v + \lambda u_{1} = \lambda (v + u_{1})$. So $\lambda$ is an eigenvalue of $T$.

              Hence each eigenvalue of $T/U$ is an eigenvalue of $T$.
    \end{enumerate}
\end{proof}
\newpage

% chapter5:sectionA:exercise39
\begin{exercise}
    Suppose $V$ is finite-dimensional and $T \in \lmap{V}$. Prove that $T$ has an eigenvalue if and only if there exists a subspace of $V$ of dimension $\dim V - 1$ that is invariant under $T$.
\end{exercise}

\begin{proof}
    $(\Rightarrow)$ $T$ has an eigenvalue $\lambda$.

    If $\lambda  = 0$. Let $Tv_{1}, \ldots, Tv_{m}$ be a basis of $\range{T}$, then $v_{1}, \ldots, v_{m}$ is linearly independent. Let $v$ be a vector in $V$, then there exist $a_{1}, \ldots, a_{m}\in\mathbb{F}$ such that
    \[
        Tv = a_{1}Tv_{1} + \cdots + a_{m}Tv_{m} = T(a_{1}v_{1} + \cdots + a_{m}v_{m}).
    \]

    So $v - (a_{1}v_{1} + \cdots + a_{m}v_{m})$ is in $\kernel{T}$. Let $u_{1}, \ldots, u_{p}$ be a basis of $\kernel{T}$.
    \[
        \kernel{T}\cap\operatorname{span}(v_{1}, \ldots, v_{m}) = \{ 0 \}.
    \]

    Because $\lambda = 0$, it follows that $\kernel{T}\ne \{ 0 \}$. So $p\geq 1$. The subspace
    \[
        \operatorname{span}(v_{1}, \ldots, v_{m})\oplus\operatorname{span}(u_{1}, \ldots, u_{p-1})
    \]

    is an invariant subspace of dimension $\dim V - 1$ under $T$.

    If $\lambda\ne 0$. Let $v_{1}$ be an eigenvector of $T$ corresponding to $\lambda$ and let $v_{1}, \ldots, v_{n}$ be a basis of $V$. Then the matrix of $T$ with respect to this basis is
    \[
        \begin{pmatrix}
            \lambda & A_{1,2} & \cdots & A_{1,n} \\
            0       & A_{2,2} & \cdots & A_{2,n} \\
            \vdots  & \vdots  &        & \vdots  \\
            0       & A_{n,2} & \cdots & A_{n,n}
        \end{pmatrix}.
    \]

    I choose $w_{1} = v_{1}$, $w_{k} = v_{k} - \lambda^{-1}A_{1,k}v_{1}$ for $1 < k\leq n$. Then $w_{1}, \ldots, w_{n}$ is a basis of $V$ and the matrix of $T$ with respect to this basis is
    \[
        \begin{pmatrix}
            \lambda & 0       & \cdots & 0       \\
            0       & A_{2,2} & \cdots & A_{2,n} \\
            \vdots  & \vdots  &        & \vdots  \\
            0       & A_{n,2} & \cdots & A_{n,n}
        \end{pmatrix}.
    \]

    Then $\operatorname{span}(w_{2}, \ldots, w_{n})$ is a subspace of $V$ of dimension $\dim V - 1$, which is invariant under $T$.

    $(\Leftarrow)$ There exists a subspace of $V$ of dimension $\dim V - 1$ that is invariant under $T$.

    Let $n = \dim V$. Let $U$ be such subspace and $v_{1}, \ldots, v_{n-1}$ be a basis of $U$. This list can be extended to become a basis of $V$, let such a basis be $v_{1}, \ldots, v_{n}$. Let $A$ be the matrix of $T$ with respect to this basis, then
    \[
        A = \begin{pmatrix}
            A_{1,1}   & \cdots & A_{1,n-1}   & A_{1,n}   \\
            \vdots    &        & \vdots      & \vdots    \\
            A_{n-1,1} & \cdots & A_{n-1,n-1} & A_{n-1,n} \\
            0         & \cdots & 0           & A_{n,n}
        \end{pmatrix}.
    \]

    $A - A_{n,n}I$ is the matrix of $T - A_{n,n}I$ with respect to $v_{1}, \ldots, v_{n}$. $A - A_{n,n}I$ is not invertible because its $n$th row is zero, hence $T - A_{n,n}I$ is not invertible. Therefore $A_{n,n}$ is an eigenvalue of $T$. So $T$ has an eigenvalue.
\end{proof}
\newpage

% chapter5:sectionA:exercise40
\begin{exercise}
    Suppose $S,T\in\lmap{V}$ and $S$ is invertible. Suppose $p\in\mathscr{P}(\mathbb{F})$ is a polynomial. Prove that
    \[
        p(STS^{-1}) = Sp(T)S^{-1}.
    \]
\end{exercise}

\begin{proof}
    Let $p(x) = a_{0} + a_{1}x + \cdots + a_{n}x^{n}$. We have
    \[
        {(STS^{-1})}^{k} = ST^{k}S^{-1}
    \]

    for every nonnegative integer $k$, so
    \begin{align*}
        p(STS^{-1}) & = a_{0}(SIS^{-1}) + a_{1}(STS^{-1}) + \cdots + a_{n}(ST^{n}S^{-1}) \\
                    & = S(a_{0}I + a_{1}T + \cdots + a_{n}T^{n})S^{-1}                   \\
                    & = Sp(T)S^{-1}.\qedhere
    \end{align*}
\end{proof}
\newpage

% chapter5:sectionA:exercise41
\begin{exercise}
    Suppose $T\in\lmap{V}$ and $U$ is a subspace of $V$ invariant under $T$. Prove that $U$ is invariant under $p(T)$ for every polynomial $p\in\mathscr{P}(\mathbb{F})$.
\end{exercise}

\begin{proof}
    Because $U$ is invariant under $T$ and $U$ is invariant under $I$, then $U$ is invariant under $T^{m}$ for every nonnegative integer $m$. Therefore $U$ is invariant under $p(T)$ for every polynomial $p\in\mathscr{P}(\mathbb{F})$.
\end{proof}
\newpage

% chapter5:sectionA:exercise42
\begin{exercise}\label{chapter5:sectionA:exercise42}
    Define $T\in\lmap{\mathbb{F}^{n}}$ by $T(x_{1}, x_{2}, x_{3}, \ldots, x_{n}) = (x_{1}, 2x_{2}, 3x_{3}, \ldots, nx_{n})$.
    \begin{enumerate}[label={(\alph*)}]
        \item Find all eigenvalues and eigenvectors of $T$.
        \item Find all subspaces of $\mathbb{F}^{n}$ that are invariant under $T$.
    \end{enumerate}
\end{exercise}

\begin{proof}
    \begin{enumerate}[label={(\alph*)}]
        \item Assume $\lambda$ is an eigenvalue of $T$ and let $(x_{1}, x_{2}, x_{3}, \ldots, x_{n})$ be a corresponding eigenvector. Then
              \[
                  (\lambda x_{1}, \lambda x_{2}, \lambda x_{3}, \ldots, \lambda x_{n}) = (x_{1}, 2x_{2}, 3x_{3}, \ldots, nx_{n})
              \]

              Hence $(\lambda - k)x_{k} = 0$ for $k = 1,\ldots, n$. So $\lambda = 1$, or $\lambda = 2$, \ldots, or $\lambda = n$.

              The eigenvectors corresponding to $\lambda = k$ is a scalar multiple (nonzero) of $e_{k}$ (where $e_{1}, \ldots, e_{n}$ is the standard basis of $\mathbb{F}^{n}$).
        \item Find all subspaces of $\mathbb{F}^{n}$ that are invariant under $T$.

              I will prove the following statement: If $V$ is a subspace of $\mathbb{F}^{n}$ that is invariant under $T$ and $(x_{1}, \ldots, x_{n})\in V$, then $x_{1}e_{1}, \ldots, x_{n}e_{n}\in V$.

              If $(x_{1}, \ldots, x_{n})\in V$ then for every $k = 1, \ldots, n$
              \[
                  (0, \ldots, 0, (k-1)!x_{k}, \ldots, n(n-1)\cdots (n - k + 1)x_{n})\in V.
              \]

              Therefore $(0, \ldots, 0, (n-1)!x_{n})\in V$ and $x_{n}e_{n}\in V$. So $(x_{1}, \ldots, x_{n-1}, 0)\in V$. Similarly, $x_{k}e_{k}\in V$ for $k = n-1, \ldots, 1$.

              Let $i_{1}, \ldots, i_{k}$ be the positive integers in $\{ 1, \ldots, n \}$ such that there exists an element in $V$ where the $i_{j}$th slot is nonzero. According to the statement that we have just proved, $V$ is the span of $e_{i_{1}}$, \ldots, $e_{i_{k}}$.

              Hence all subspaces of $\mathbb{F}^{n}$ that are invariant under $T$ are spans of vectors in the list $e_{1}, \ldots, e_{n}$.
    \end{enumerate}
\end{proof}
\newpage

% chapter5:sectionA:exercise43
\begin{exercise}
    Suppose that $V$ is finite-dimensional, $\dim V > 1$, and $T\in\lmap{V}$. Prove that $\{ p(T): p\in\mathscr{P}(\mathbb{F}) \}\ne \lmap{V}$.
\end{exercise}

\begin{proof}
    This result follows Exercise~\ref{chapter5:sectionB:exercise19}.
    \bigskip

    Or using the result: $T$ commutes with all operators on $V$ if and only if $T$ is a scalar multiple of the identity operator.

    If $T$ is a scalar multiple of the identity operator then $\{ p(T): p\in\mathscr{P}(\mathbb{F}) \}$ consists of operators which are scalar multiple of the identity operator, so it is not $\lmap{V}$.

    If $T$ is not a scalar multiple of the identity operator, then there exists an operator $S\in\lmap{V}$ such that $ST\ne TS$ (this is true for $\dim V > 1$). Therefore $S$ is not in $\{ p(T): p\in\mathscr{P}(\mathbb{F}) \}$. Therefore the set is not $\lmap{V}$.

    Hence $\{ p(T): p\in\mathscr{P}(\mathbb{F}) \}\ne \lmap{V}$.
\end{proof}
\newpage

\section{The Minimal Polynomial}

% chapter5:sectionB:exercise1
\begin{exercise}
    Suppose $T\in \lmap{V}$. Prove that $9$ is an eigenvalue of $T^{2}$ if and only if $3$ or $-3$ is an eigenvalue of $T$.
\end{exercise}

\begin{proof}
    If $9$ is an eigenvalue of $T^{2}$ then there exists a nonzero vector $v$ such that $(T^{2} - 9I)(v) = 0$, so $(T - 3I)(T + 3I)(v) = 0$. Then $3$ or $-3$ is an eigenvalue of $T$ (otherwise, $(T - 3I)(T + 3I)(v) \ne 0$).

    And if $3$ is an eigenvalue of $T$, then there exists a nonzero vector $v_{1}$ such that $(T^{2} - 9I)(v_{1}) = (T + 3I)(T - 3I)(v_{1}) = 0$. If $-3$ is an eigenvalue of $T$, then there exists a nonzero vector $v_{2}$ such that $(T^{2} - 9I)(v_{2}) = (T - 3I)(T + 3I)(v_{2}) = 0$. Hence $9$ is an eigenvalue of $T^{2}$.
\end{proof}
\newpage

% chapter5:sectionB:exercise2
\begin{exercise}
    Suppose $V$ is a complex vector space and $T\in\lmap{V}$ has no eigenvalues. Prove that every subspace of $V$ invariant under $T$ is either $\{0\}$ or infinite-dimensional.
\end{exercise}

\begin{proof}
    Let $U$ be a subspace of $V$ such that $U$ is invariant under $T$.

    If $U$ is finite-dimensional and $\dim U > 0$, then $T\vert_{U}$ has an eigenvalue.

    So if $T$ has no eigenvalues, then $T\vert_{U}$ also has no eigenvalues, therefore either $\dim U = 0$ or $U$ is infinite-dimensional.
\end{proof}
\newpage

% chapter5:sectionB:exercise3
\begin{exercise}
    Suppose $n$ is a positive integer and $T\in\lmap{\mathbb{F}^{n}}$ is defined by
    \[
        T(x_{1}, \ldots, x_{n}) = (x_{1} + \cdots + x_{n}, \ldots, x_{1} + \cdots + x_{n}).
    \]

    \begin{enumerate}[label={(\alph*)}]
        \item Find all eigenvalues and eigenvectors of $T$.
        \item Find the minimal polynomial of $T$.
    \end{enumerate}
\end{exercise}

\begin{proof}
    \begin{enumerate}[label={(\alph*)}]
        \item Let $\lambda$ be an eigenvalue of $T$ and $(x_{1}, \ldots, x_{n})$ be a corresponding eigenvector. Then
              \[
                  (x_{1} + \cdots + x_{n}, \ldots, x_{1} + \cdots + x_{n}) = (\lambda x_{1}, \ldots, \lambda x_{n}).
              \]

              It follows that $\lambda x_{1} = \cdots = \lambda x_{n}$. Then either $\lambda = 0$ or $x_{1} = \cdots = x_{n}$. Hence all eigenvalues of $T$ are $0$ and $n$. The eigenvectors of $T$ corresponding to $0$ are nonzero vectors of the vector space
              \[
                  \{ (x_{1}, \ldots, x_{n}): x_{1} + \cdots + x_{n} = 0 \}.
              \]

              The eigenvectors of $T$ corresponding to $n$ are nonzero vectors of the vector space
              \[
                  \operatorname{span}((1, \ldots, 1)).
              \]
        \item Because $T$ has two different eigenvalues, it follows the the degree of the minimal polynomial of $T$ is at least $2$.
              \[
                  T^{2}(x_{1}, \ldots, x_{n}) = (n(x_{1} + \cdots + x_{n}), \ldots, n(x_{1} + \cdots + x_{n})) = nT(x_{1}, \ldots, x_{n}).
              \]

              So $T^{2} - nT = 0$. Hence $x^{2} - nx$ is the minimal polynomial of $T$.
    \end{enumerate}
\end{proof}
\newpage

% chapter5:sectionB:exercise4
\begin{exercise}\label{chapter5:sectionB:exercise4}
    Suppose $\mathbb{F} = \mathbb{C}$, $T\in\lmap{V}$, $p\in\mathscr{P}(\mathbb{C})$, and $\alpha\in\mathbb{C}$. Prove that $\alpha$ is an eigenvalue of $p(T)$ if and only if $\alpha = p(\lambda)$ for some eigenvalue $\lambda$ of $T$.
\end{exercise}

\begin{proof}
    If $\lambda$ is an eigenvalue of $T$ and $v$ is an eigenvector of $T$ corresponding to $\lambda$, then
    \begin{align*}
        (p(T))(v) = (a_{0}I + a_{1}T + \cdots + a_{n}T^{n})(v) & = a_{0}v + a_{1}\lambda v + \cdots + a_{n}\lambda^{n}v  \\
                                                               & = (a_{0} + a_{1}\lambda + \cdots + a_{n}\lambda^{n})(v) \\
                                                               & = p(\lambda)v                                           \\
                                                               & = \alpha v.
    \end{align*}

    Hence $\alpha$ is an eigenvalue of $p(T)$.
    \bigskip

    Let $p$ be a polynomial of degree $n$ and $v$ be an eigenvector of $p(T)$ corresponding to $\alpha$.
    \[
        (p(T) - \alpha I)(v) = 0.
    \]

    According to the fundamental theorem of algebra, $p(z) - \alpha$ can be rewritten in the following form
    \[
        p(z) - \alpha = c(z - z_{1})\cdots (z - z_{n})
    \]

    So $c(T - z_{1}I)\cdots (T - z_{n}I)(v) = 0$. If none of $z_{1}, \ldots, z_{n}$ is an eigenvalue of $T$, then $c(T - z_{1}I)\cdots (T - z_{n}I)(v) \ne 0$. Therefore there is $\lambda$ in the list $z_{1}, \ldots, z_{n}$ such that $\lambda$ is an eigenvalue of $T$, this implies $p(\lambda) - \alpha = 0$. Hence $\alpha = p(\lambda)$ for some eigenvalue $\lambda$ of $T$.
\end{proof}
\newpage

% chapter5:sectionB:exercise5
\begin{exercise}
    Give an example of an operator on $\mathbb{R}^{2}$ that shows the result in Exercise 4 does not hold if $\mathbb{C}$ is replaced with $\mathbb{R}$.
\end{exercise}

\begin{proof}
    I define $T\in\lmap{\mathbb{R}^{2}}$ as follows: $T(x, y) = (-y, x)$, this linear operator $T$ has no eigenvalue. However, $T^{2}(x, y) = (-x, -y)$, which means $-1$ is an eigenvalue of $T^{2}$.
\end{proof}
\newpage

% chapter5:sectionB:exercise6
\begin{exercise}
    Suppose $T \in \lmap{\mathbb{F}^{2}}$ is defined by $T(w, z) = (-z, w)$. Find the minimal polynomial of $T$.
\end{exercise}

\begin{proof}
    $T^{2}(w, z) = T(-z, w) = (-w, -z)$. $p(x) = x^{2} + 1$ is the minimal polynomial of $T$.
\end{proof}
\newpage

% chapter5:sectionB:exercise7
\begin{exercise}
    \begin{enumerate}[label={(\alph*)}]
        \item Give an example of $S, T\in\lmap{\mathbb{F}^{2}}$ such that the minimal polynomial of $ST$ does not equal the minimal polynomial of $TS$.
        \item Suppose $V$ is finite-dimensional and $S,T \in \lmap{V}$. Prove that if at least one of $S, T$ is invertible, then the minimal polynomial of $ST$ equals the minimal polynomial of $TS$.
    \end{enumerate}
\end{exercise}

\begin{proof}
    \begin{enumerate}[label={(\alph*)}]
        \item I choose $S(x, y) = (0, y)$ and $T(x, y) = (0, x)$.
              \[
                  (ST)(x, y) = S(0, x) = (0, x)\qquad (TS)(x, y) = T(0, y) = (0, 0)
              \]

              The minimal polynomial of $ST$ is $p(x) = x^{2}$, meanwhile the minimal polynomial of $TS$ is $p(x) = x$.
        \item Let $p, q$ be the minimal polynomials of $ST, TS$, respectively.

              If $S$ is invertible, then
              \[
                  \begin{split}
                      0 = p(ST) = S^{-1}p(ST)S = p(S^{-1}STS) = p(TS), \\
                      0 = q(TS) = Sq(TS)S^{-1} = q(STSS^{-1}) = q(ST).
                  \end{split}
              \]

              so $\deg p\geq \deg q$ and $\deg q\geq \deg p$. Due to the uniqueness of the minimal polynomial, we deduce that $p = q$.

              If $T$ is invertible, then
              \[
                  \begin{split}
                      0 = p(ST) = Tp(ST)T^{-1} = p(TSTT^{-1}) = p(TS), \\
                      0 = q(TS) = T^{-1}q(TS)T = q(T^{-1}TST) = q(ST).
                  \end{split}
              \]

              so $\deg p\geq \deg q$ and $\deg q\geq \deg p$. Due to the uniqueness of the minimal polynomial, we deduce that $p = q$.

              Hence if at least one of $S, T$ is invertible, then the minimal polynomial of $ST$ equals the minimal polynomial of $TS$.
    \end{enumerate}
\end{proof}
\newpage

% chapter5:sectionB:exercise8
\begin{exercise}
    Suppose $T\in\lmap{\mathbb{R}^{2}}$ is the operator of counterclockwise rotation by $1^\circ$. Find the minimal polynomial of $T$.
\end{exercise}

\begin{proof}
    According to the definition of $T$,
    \[
        T(x, y) = (x\cos 1^{\circ} + y \sin 1^{\circ}, -x\sin 1^{\circ} + y\cos 1^{\circ}).
    \]
    \begin{align*}
        T^{2}(x, y) & = (x\cos 2^{\circ} + y\sin 2^{\circ}, -x\sin 2^{\circ} + y\cos 2^{\circ})
    \end{align*}

    So
    \begin{align*}
        T^{2}(1, 0) & = (\cos 2^{\circ}, -\sin 2^{\circ})                          \\
                    & = (2\cos^{2} 1^{\circ} - 1, -2\sin 1^{\circ}\cos 1^{\circ})  \\
                    & = 2\cos 1^{\circ}(\cos 1^{\circ}, -\sin 1^{\circ}) + (-1, 0) \\
                    & = 2\cos 1^{\circ} T(1, 0) - I(1, 0).
    \end{align*}

    The minimal polynomial of $T$ is $p(x) = x^{2} - 2x\cos 1^{\circ} + 1$.
\end{proof}
\newpage

% chapter5:sectionB:exercise9
\begin{exercise}
    Suppose $T\in\lmap{V}$ is such that with respect to some basis of $V$, all entries of the matrix of $T$ are rational numbers. Explain why all coefficients of the minimal polynomial of $T$ are rational numbers.
\end{exercise}

\begin{proof}
    Assume that all entries of the matrix of $T$ with respect to the basis $v_{1}, \ldots, v_{n}$ of $V$ are rational numbers. Let
    \[
        a_{0} + a_{1}z + \cdots + a_{m-1}z^{m-1} + z^{m}
    \]

    be the minimal polynomial of $T$. $x_{0}I + x_{1}T + \cdots + x_{m-1}T^{m-1} + T^{m} = 0$ if and only if $x_{0}v_{k} + x_{1}Tv_{k} + \cdots + x_{m-1}T^{m-1}v_{k} + T^{m}v_{k} = 0$ for all $k = 1,\ldots, m$. $T^{i}v_{k}$ can be rewritten as a linear combination of $v_{1}, \ldots, v_{m}$. So from  $x_{0}v_{k} + x_{1}Tv_{k} + \cdots + x_{m-1}T^{m-1}v_{k} + T^{m}v_{k} = 0$ for all $k = 1,\ldots, m$, we obtain a system of linear equations of $m$ unknowns $x_{0}, x_{1}, \ldots, x_{m-1}$ where all coefficients are rational numbers. The solutions of this system of linear equations are the coefficients of the minimal polynomial of $T$. This solution is also unique due to the uniqueness of minimal polynomial of $T$. The solution also must comprise of rational numbers because all coefficients are rational numbers. Hence $a_{0}, a_{1}, \ldots, a_{m-1}$ are rational numbers.
\end{proof}
\newpage

% chapter5:sectionB:exercise10
\begin{exercise}
    Suppose $V$ is finite-dimensional, $T\in\lmap{V}$, and $v\in V$. Prove that
    \[
        \operatorname{span}(v, Tv, \ldots, T^{m}v) = \operatorname{span}(v, Tv, \ldots, T^{\dim V - 1}v)
    \]

    for all integers $m\geq \dim V - 1$.
\end{exercise}

\begin{proof}
    Let $p$ be the minimal polynomial of $T$, then $\deg T\leq \dim V$.

    I give a proof using mathematical induction on $m$.

    When $m = \dim V - 1$, $\operatorname{span}(v, Tv, \ldots, T^{m}v) = \operatorname{span}(v, Tv, \ldots, T^{\dim V - 1}v)$.

    Assume $\operatorname{span}(v, Tv, \ldots, T^{m}v) = \operatorname{span}(v, Tv, \ldots, T^{\dim V - 1}v)$ for every positive integer $m$ such that $\dim V - 1\leq m < n$.

    Let $p(x) = a_{0} + a_{1}x + \cdots + a_{k-1}x^{k-1} + x^{k}$, then $k\leq \dim V$, $k\leq n$, and
    \[
        T^{n}v = -a_{k-1}T^{n-1}v - \cdots - a_{1}T^{n-k+1}v - a_{0}T^{n-k}v.
    \]

    According to the induction hypothesis, $T^{n-1}, \ldots, T^{n-k}$ are in $\operatorname{span}(v, Tv, \ldots, T^{\dim V - 1}v)$. Therefore $T^{n}v$ is also in $\operatorname{span}(v, Tv, \ldots, T^{\dim V - 1}v)$.

    Hence, due to the principle of mathematical induction
    \[
        \operatorname{span}(v, Tv, \ldots, T^{m}v) = \operatorname{span}(v, Tv, \ldots, T^{\dim V - 1}v)
    \]

    for all integers $m\geq \dim V - 1$.
\end{proof}
\newpage

% chapter5:sectionB:exercise11
\begin{exercise}
    Suppose $V$ is a two dimensional vector space, $T\in\lmap{V}$, and the matrix of $T$ with respect to some basis of $V$ is $\begin{pmatrix}a & c \\ b & d\end{pmatrix}$.
    \begin{enumerate}[label={(\alph*)}]
        \item Show that $T^{2} - (a + d)T + (ad - bc)I = 0$.
        \item Show that the minimal polynomial of $T$ equals
              \[
                  \begin{cases}
                      z - a                        & \text{if $b = c = 0$ and $a = d$}, \\
                      z^{2} - (a + d)z + (ad - bc) & \text{otherwise}
                  \end{cases}
              \]
    \end{enumerate}
\end{exercise}

\begin{proof}
    Let $v_{1}, v_{2}$ be the basis such that the matrix of $T$ with respect to this basis is $\begin{pmatrix}a & c \\ b & d\end{pmatrix}$.

    \begin{align*}
        T^{2}(x_{1}v_{1} + x_{2}v_{2}) & = T(x_{1}av_{1} + x_{1}bv_{2} + x_{2}cv_{1} + x_{2}dv_{2})                                                     \\
                                       & = (a + d)T(x_{1}v_{1} + x_{2}v_{2}) + T(-x_{1}dv_{1} + x_{1}bv_{2} + x_{2}cv_{1} - x_{2}av_{2})                \\
                                       & = (a + d)T(x_{1}v_{1} + x_{2}v_{2}) + (-x_{1}d + x_{2}c)(av_{1} + bv_{2}) + (x_{1}b - x_{2}a)(cv_{1} + dv_{2}) \\
                                       & = (a + d)T(x_{1}v_{1} + x_{2}v_{2}) + (bc - ad)(x_{1}v_{1} + x_{2}v_{2}).
    \end{align*}

    Hence $T^{2} - (a + d)T + (ad - bc)I = 0$.

    If $b = c = 0$ and $a = d$, then $z - a$ is the minimal polynomial of $T$.

    Otherwise, if the degree of the minimal polynomial of $T$ is $1$, then there exists $\lambda\in\mathbb{F}$ such that $T = \lambda I$. So $Tv_{1} = \lambda v_{1} = av_{1} + bv_{2}$ and $Tv_{2} = \lambda v_{2} = cv_{1} + dv_{2}$. Therefore $a = d = \lambda$ and $b = c = 0$. So the degree of the minimal polynomial of $T$ must be $2$. Therefore the minimal polynomial of $T$ is $z^{2} - (a + d)z + (ad - bc)$.
\end{proof}
\newpage

% chapter5:sectionB:exercise12
\begin{exercise}
    Define $T\in\lmap{\mathbb{F}^{n}}$ by $T(x_{1}, x_{2}, x_{3}, \ldots, x_{n}) = (x_{1}, 2x_{2}, 3x_{3}, \ldots, nx_{n})$. Find the minimal polynomial of $T$.
\end{exercise}

\begin{proof}
    According to Exercise~\ref{chapter5:sectionA:exercise42}, $T$ has $n$ distinct eigenvalues, which are $1, \ldots, n$.

    On the other hand, every eigenvalue of $T$ is a root of the minimal polynomial of $T$. Moreover, the degree of the minimal polynomial of $T$ does not exceed $n$. Therefore the minimal polynomial of $T$ is $(z - 1)(z - 2)\cdots (z - n)$.
\end{proof}
\newpage

% chapter5:sectionB:exercise13
\begin{exercise}
    Suppose $T\in\lmap{V}$ and $p\in\mathscr{P}(\mathbb{F})$. Prove that there exists a unique $r\in \mathscr{P}(\mathbb{F})$ such that $p(T) = r(T)$ and $\deg r$ is less than the degree of the minimal polynomial of $T$.
\end{exercise}

\begin{proof}
    Let $q$ be the minimal polynomial of $T$. By the Euclid division algorithm, there exist unique polynomials $s$, $r$ such that $p = sq + r$ and $\deg r < \deg q$. Therefore $p(T) = s(T)q(T) + r(T) = r(T)$.

    If $r$ and $r'$ are polynomials such that $p(T) = r(T)$, $p(T) = r'(T)$ and $\deg r, \deg r' < \deg q$, then $(r - r')(T) = 0$. However, $\deg (r - r') < \deg q$, so $r - r' = 0$ (because $q$ is the minimal polynomial of $T$), which means $r = r'$.

    Hence there exists a unique $r\in \mathscr{P}(\mathbb{F})$ such that $p(T) = r(T)$ and $\deg r$ is less than the degree of the minimal polynomial of $T$.
\end{proof}
\newpage

% chapter5:sectionB:exercise14
\begin{exercise}
    Suppose $V$ is finite-dimensional and $T\in\lmap{V}$ has minimal polynomial $4 + 5z - 6z^{2} - 7z^{3} + 2z^{4} + z^{5}$. Find the minimal polynomial of $T^{-1}$.
\end{exercise}

\begin{proof}
    Because the constant term of the minimal polynomial is nonzero, it follows that $T$ is invertible. Let $S = T^{-1}$. Let $a_{0} + a_{1}z + \cdots + z^{n}$ be the minimal polynomial of $T^{-1}$.
    \[
        4I + 5T - 6T^{2} - 7T^{3} + 2T^{4} + T^{5} = 0
    \]

    so
    \[
        0 = T^{-5}(4I + 5T - 6T^{2} - 7T^{3} + 2T^{4} + T^{5}) = 4S^{5} + 5S^{4} - 6S^{3} - 7S^{2} + 2S + I.
    \]

    Hence the minimal polynomial of $T^{-1}$ is a divisor of $1 + 2z - 7z^{2} - 6z^{3} + 5z^{4} + 4z^{5}$, and $n\leq 5$.
    \[
        a_{0}I + a_{1}S+ \cdots + a_{n-1}S^{n-1} + S^{n} = 0.
    \]

    So
    \[
        a_{0}T^{n} + a_{1}T^{n-1} + \cdots + a_{n-1}T + I = 0.
    \]

    It follows that $n\geq 5$. Hence $n = 5$.

    Thus the minimal polynomial of $T^{-1}$ is
    \[
        \frac{1}{4} + \frac{1}{2}z - \frac{7}{4}z^{2} - \frac{3}{2}z^{3} + \frac{5}{4}z^{4} + z^{5}.\qedhere
    \]
\end{proof}
\newpage

% chapter5:sectionB:exercise15
\begin{exercise}
    Suppose $V$ is a finite-dimensional complex vector space with $\dim V > 0$ and $T\in\lmap{V}$. Define $f: \mathbb{C}\to \mathbb{R}$ by
    \[
        f(\lambda) = \dim\range{(T - \lambda I)}.
    \]

    Prove that $f$ is not a continuous function.
\end{exercise}

\begin{proof}
    Because $\dim V > 0$ and $V$ is a complex vector space, then $T$ has an eigenvalue $\alpha$.

    According to Exercise~\ref{chapter5:sectionA:exercise11}, there exists a positive real number $\delta$ such that for all $\lambda$, $0 < \abs{\alpha - \lambda} < \delta$ implies $(T - \lambda I)$ is invertible. Hence
    \begin{align*}
        \lim\limits_{\lambda\to \alpha} f(\lambda) & = \lim\limits_{\lambda\to \alpha} \dim\range{(T - \lambda I)} \\
                                                   & = \dim V                                                      \\
                                                   & \ne \dim\range{(T - \alpha I)} = f(\alpha).
    \end{align*}

    Thus $f$ is not a continuous function. Moreover, all discontinuous points of $f$ are eigenvalues of $T$.
\end{proof}
\newpage

% chapter5:sectionB:exercise16
\begin{exercise}
    Suppose $a_{0}, \ldots, a_{n-1}\in\mathbb{F}$. Let $T$ be the operator on $\mathbb{F}^{n}$ whose matrix (with respect to the standard basis) is
    \[
        \begin{pmatrix}
            0 &   &        &        &   & -a_{0}   \\
            1 & 0 &        &        &   & -a_{1}   \\
              & 1 & \ddots &        &   & -a_{2}   \\
              &   &        & \ddots &   & \vdots   \\
              &   &        &        & 0 & -a_{n-2} \\
              &   &        &        & 1 & -a_{n-1}
        \end{pmatrix}
    \]

    Here all entries of the matrix are $0$ except for all 1's on the line under the
    diagonal and the entries in the last column (some of which might also be $0$). Show that the minimal polynomial of $T$ is the polynomial
    \[
        a_{0} + a_{1}z + \cdots + a_{n-1}z^{n-1} + z^{n}.
    \]
\end{exercise}

\begin{proof}
    Let $e_{1}, \ldots, e_{n}$ be the standard basis of $\mathbb{F}^{n}$.
    \begin{itemize}
        \item $Ie_{1} = e_{1}$.
        \item $Te_{1} = e_{2}$.
        \item $T^{2}e_{1} = Te_{2} = e_{3}$.
        \item \ldots
        \item $T^{n-1}e_{1} = Te_{n-1} = e_{n}$.
        \item $T^{n}e_{1} = Te_{n} = -a_{0}e_{1} - a_{1}e_{2} - \cdots - a_{n-1}e_{n-1}$.
    \end{itemize}

    Because $e_{1}, \ldots, e_{n}$ is linearly independent, $Ie_{1}, Te_{1}, \ldots, T^{n-1}e_{1}$ is linearly independent. So for every $b_{0}, b_{1}, \ldots, b_{n-1}$ such that they are not all zero, $(b_{0}I + b_{1}T + \cdots + b_{n-1}T^{n-1})(v_{1})\ne 0$. So the minimal polynomial of $T$ has degree greater than $(n - 1)$. On the other hand, the degree of the minimal polynomial of $T$ does not exceed $n$. Therefore the degree of the minimal polynomial of $T$ is $n$.

    $T^{n}e_{1} = Te_{n} = -a_{0}e_{1} - a_{1}e_{2} - \cdots - a_{n-1}e_{n-1}$, it follows that
    \[
        (a_{0}I + a_{1}T + \cdots + a_{n-1}T^{n-1} + T^{n})(v_{1}) = 0.
    \]

    Moreover, for $k = 2, \ldots, n$
    \begin{align*}
        (a_{0}I + a_{1}T + \cdots + a_{n-1}T^{n-1} + T^{n})(v_{k}) & = (a_{0}I + a_{1}T + \cdots + a_{n-1}T^{n-1} + T^{n})(T^{k-1})(v_{1}) \\
                                                                   & = (T^{k-1})(a_{0}I + a_{1}T + \cdots + a_{n-1}T^{n-1} + T^{n})(v_{1}) \\
                                                                   & = T^{k-1}(0)                                                          \\
                                                                   & = 0.
    \end{align*}

    Hence $a_{0}I + a_{1}T + \cdots + a_{n-1}T^{n-1} + T^{n} = 0$. Thus $a_{0} + a_{1}z + \cdots + a_{n-1}z^{n-1} + z^{n}$ is the minimal polynomial of $T$.
\end{proof}
\newpage

% chapter5:sectionB:exercise17
\begin{exercise}
    Suppose $V$ is finite-dimensional, $T \in \lmap{V}$, and $p$ is the minimal polynomial of $T$. Suppose $\lambda \in \mathbb{F}$. Show that the minimal polynomial of $T - \lambda I$ is the polynomial $q$ defined by $q(z) = p(z + \lambda)$.
\end{exercise}

\begin{proof}
    Let $r$ be the minimal polynomial of $T - \lambda I$. Then $r(T - \lambda I) = 0$.

    Assume $\deg r < \deg p$. I define $s\in\mathscr{P}(\mathbb{F})$ as follows: $s(z) = r(z - \lambda)$, then $s(T) = r(T - \lambda I) = 0$ and $s$ is monic. However $\deg s = \deg r < \deg p$, which contradicts $p$ being the minimal polynomial of $T$. Hence $\deg r\geq \deg p$.

    Let $q(z) = p(z + \lambda)$. $q$ is monic, $\deg q = \deg p$ and $q(T - \lambda I) = p(T) = 0$. Therefore $q$ is the minimal polynomial of $T - \lambda I$.
\end{proof}
\newpage

% chapter5:sectionB:exercise18
\begin{exercise}
    Suppose $V$ is finite-dimensional, $T \in \lmap{V}$, and $p$ is the minimal polynomial of $T$. Suppose $\lambda \in \mathbb{F}\setminus\{0\}$. Show that the minimal polynomial of $\lambda T$ is the polynomial $q$ defined by $q(z) = \lambda^{\deg p}p\left(\dfrac{z}{\lambda}\right)$.
\end{exercise}

\begin{proof}
    Let $r$ be the minimal polynomial of $\lambda T$. Then $r(\lambda T) = 0$.

    Assume $\deg r < \deg p$. I define $s\in\mathscr{P}(\mathbb{F})$ as follows: $s(z) = \dfrac{1}{\lambda^{\deg r}}r(\lambda z)$. $s$ is monic and $s(T) = 0$. This contradicts $p$ being the minimal polynomial of $T$. Hence $\deg r\geq \deg p$.

    Let $q(z) = \lambda^{\deg p}p\left(\dfrac{z}{\lambda}\right)$. $q$ is monic and $q(\lambda T) = 0$. Therefore $q$ is the minimal polynomial of $\lambda T$.
\end{proof}
\newpage

% chapter5:sectionB:exercise19
\begin{exercise}\label{chapter5:sectionB:exercise19}
    Suppose $V$ is finite-dimensional and $T\in\lmap{V}$. Let $\mathcal{E}$ be the subspace of $\lmap{V}$ defined by
    \[
        \mathcal{E} = \{ q(T): q\in\mathscr{P}(\mathbb{F}) \}.
    \]

    Prove that $\dim\mathcal{E}$ equals the degree of the minimal polynomial of $T$.
\end{exercise}

\begin{proof}
    Let $p$ be the minimal polynomial of $T$. Let $q(T)$ be an element of $\mathcal{E}$. According to the Euclidean division algorithm, there exist unique polynomials $s$, $r$ such that $\deg r < \deg p$ and $q = ps + r$. So $q(T) = p(T)s(T) + r(T) = r(T)$. Therefore
    \[
        \mathcal{E} = \{ q(T): q\in\mathscr{P}(\mathbb{F}) \} = \{ q(T): q\in\mathscr{P}_{(\deg p) - 1}(\mathbb{F}) \}
    \]

    $I, T, \ldots, T^{(\deg p) - 1}$ is linearly independent, since the degree of the minimal polynomial of $T$ is $\deg p$. On the other hand, if $q(T)\in \mathcal{E}$, there exists $r\in\mathscr{P}_{(\deg p) - 1}(\mathbb{F})$ such that $q(T) = r(T)$. Moreover, $r(T)$ is in $\operatorname{span}(I, T, \ldots, T^{(\deg p) - 1})$. Therefore $I, T, \ldots, T^{(\deg p) - 1}$ is a basis of $\mathcal{E}$. Thus $\dim \mathcal{E} = \deg p$.
\end{proof}
\newpage

% chapter5:sectionB:exercise20
\begin{exercise}
    Suppose $T \in \lmap{\mathbb{F}^{4}}$ is such that the eigenvalues of $T$ are $3, 5, 8$. Prove that ${(T - 3I)}^{2}{(T - 5I)}^{2}{(T - 8I)}^{2} = 0$.
\end{exercise}

\begin{proof}
    Let $p$ be the minimal polynomial of $T$. Because the eigenvalues of $T$ are $3, 5, 8$, these are also the roots of $p$. So
    \[
        p(z) = (z - 3)(z - 5)(z - 8)q(z)
    \]

    where $q$ is a monic polynomial and $\deg q\leq 1$ (because $3\leq \deg p\leq 4$). Therefore, either $q(z) = 1$, or $q(z) = z - 3$, or $q(z) = z - 5$, or $q(z) = z - 8$ (because the roots of the minimal polynomial of $T$ are the eigenvalues of $T$). In every of these cases, we can conclude ${(T - 3I)}^{2}{(T - 5I)}^{2}{(T - 8I)}^{2} = 0$.
\end{proof}
\newpage

% chapter5:sectionB:exercise21
\begin{exercise}
    Suppose $V$ is finite-dimensional and $T \in \lmap{V}$. Prove that the minimal
    polynomial of $T$ has degree at most $1 + \dim \range{T}$.
\end{exercise}

\begin{proof}
    Let $p$ be the minimal polynomial of $T$ and
    \[
        p(z) = a_{0} + a_{1}z + \cdots + a_{n-1}z^{n-1} + z^{n}.
    \]

    $\range{T}$ is invariant under $T$. Let $q$ be the minimal polynomial of $T\vert_{\range{T}}$.

    Let $s(z) = z\times q(z)$. For all $v\in V$, $s(T)v = q(T)(Tv)$. On the other hand, $Tv\in \range{T}$, so
    \[
        q(T)(Tv) = q(T\vert_{\range{T}})(Tv) = 0.
    \]

    So $s$ is a polynomial multiple of $p$. Hence
    \[
        \deg p \leq \deg s = 1 + \deg q \leq 1 + \dim\range{T}.
    \]

    Thus $\deg p\leq 1 + \dim\range{T}$.
\end{proof}
\newpage

% chapter5:sectionB:exercise22
\begin{exercise}
    Suppose $V$ is finite-dimensional and $T\in\lmap{V}$. Prove that $T$ is invertible if and only if $I\in\operatorname{span}(T, T^{2}, \ldots, T^{\dim V})$.
\end{exercise}

\begin{proof}
    Let the minimal polynomial of $T$ be $p(z) = a_{0} + a_{1}z + \cdots + a_{n-1}z^{n-1} + z^{n}$.

    If $T$ is invertible, then $a_{0}\ne 0$, and
    \[
        I = -a_{0}^{-1}(a_{1}T + \cdots + a_{n-1}T^{n-1} + T^{n}).
    \]

    So $I\in\operatorname{span}(T, T^{2}, \ldots, T^{n})\subseteq \operatorname{span}(T, T^{2}, \ldots, T^{\dim V})$.

    \bigskip
    If $I\in\operatorname{span}(T, T^{2}, \ldots, T^{\dim V})$, then there exist $c_{1}, \ldots, c_{\dim V}$ in $\mathbb{F}$ such that
    \[
        I = c_{1}T + \cdots + c_{\dim V}T^{\dim V}
    \]

    So the polynomial $q(z) = 1 - c_{1}z - \cdots - c_{\dim V}z^{\dim V}$ is a multiple of $p$. Assume $T$ is not invertible, then $0$ is a root of $p$. So $0$ is also a root of $q$, but this is a contradiction because $q(0) = 1$. Hence the assumption is false, and we conclude that $T$ is invertible.

    Thus $T$ is invertible if and only if $I\in\operatorname{span}(T, T^{2}, \ldots, T^{\dim V})$.
\end{proof}
\newpage

% chapter5:sectionB:exercise23
\begin{exercise}
    Suppose $V$ is finite-dimensional and $T\in\lmap{V}$. Let $n = \dim V$. Prove that if $v\in V$, then $\operatorname{span}(v, Tv, \ldots, T^{n-1}v)$ is invariant under $T$.
\end{exercise}

\begin{proof}
    Let $p$ be the minimal polynomial of $T$ and
    \[
        p(z) = a_{0} + a_{1}z + \cdots + a_{n-1}z^{n-1} + z^{n}.
    \]

    Then $T^{n}v = -a_{0}v - a_{1}Tv - \cdots - a_{n-1}T^{n-1}v$. Hence
    \[
        Tv, T(Tv), \ldots, T(T^{n-1}v) \in \operatorname{span}(v, Tv, \ldots, T^{n-1}v).
    \]

    Therefore $\operatorname{span}(v, Tv, \ldots, T^{n-1}v)$ is invariant under $T$.
\end{proof}
\newpage

% chapter5:sectionB:exercise24
\begin{exercise}
    Suppose $V$ is a finite-dimensional complex vector space. Suppose $T\in\lmap{V}$ is such that $5$ and $6$ are eigenvalues of $T$ and that $T$ has no other eigenvalues. Prove that ${(T - 5I)}^{\dim V - 1}{(T - 6I)}^{\dim V - 1} = 0$.
\end{exercise}

\begin{proof}
    Let $p$ be the minimal polynomial of $T$ and $\lambda$ be a root of $p$. Then there exists a polynomial $q$ such that $p(z) = (z - \lambda)q(z)$.
    \[
        0 = p(T) = (T - \lambda I)q(T)
    \]

    $q(T)\ne 0$ because $\deg q < \deg p$, $\deg q\ne 0$ and $p$ is the minimal polynomial of $T$. So there exists a vector $v$ in $V$ such that $q(T)v\ne 0$, then $0 = p(T)v = (T - \lambda I)(q(T)v)$. So $\lambda$ is an eigenvalue of $T$.

    According to the hypothesis, $\lambda$ is either $5$ or $6$. From this and the fundamental theorem of algebra, we have
    \[
        p(z) = {(z - 5)}^{m}{(z - 6)}^{n}
    \]

    where $m, n\geq 1$ and $m + n \leq \dim V$. Hence ${(z - 5)}^{\dim V - 1}{(z - 6)}^{\dim V - 1}$ is a multiple of $p$. Thus
    \[
        {(T - 5I)}^{\dim V - 1}{(T - 6I)}^{\dim V - 1} = 0.\qedhere
    \]
\end{proof}
\newpage

% chapter5:sectionB:exercise25
\begin{exercise}\label{chapter5:sectionB:exercise25}
    Suppose $V$ is finite-dimensional, $T\in\lmap{V}$, and $U$ is a subspace of $V$ that is invariant under $T$.
    \begin{enumerate}[label={(\alph*)}]
        \item Prove that the minimal polynomial of $T$ is a polynomial multiple of the minimal polynomial of the quotient operator $T/U$.
        \item Prove that
              \[
                  \text{(minimal polynomial of $T\vert_{U}$)}\times\text{(minimal polynomial of $T/U$)}
              \]

              is a polynomial multiple of the minimal polynomial of $T$.
    \end{enumerate}
\end{exercise}

\begin{proof}
    \begin{enumerate}[label={(\alph*)}]
        \item Let $p$ be the minimal polynomial of $T$, then $p(T) = 0$.

              For every nonnegative integer $k$
              \[
                  {(T/U)}^{k}(v + U) = T^{k}v + U.
              \]

              Therefore
              \[
                  p(T/U)(v + U) = p(T)v + U = 0 + U = 0.
              \]

              So $p(T/U) = 0$. Hence $p$ is a polynomial multiple of the minimal polynomial of the quotient operator $T/U$.
        \item Let $q_{1}$ be the minimal polynomial of $T\vert_{U}$ and $q_{2}$ be the minimal polynomial of $T/U$.

              For all $v\in V$, $q_{2}(T/U)v = q_{2}(T)v + U$. However, $q_{2}(T/U) = 0$, so $q_{2}(T)v + U = 0 + U$, which means $q_{2}(T)v\in U$. Therefore
              \[
                  q_{1}(T)q_{2}(T)(v) = q_{1}(T)(q_{2}(T)v) = q_{1}(T\vert_{U})(q_{2}(T)v) = 0.
              \]

              Hence $(q_{1}q_{2})(T) = 0$. Thus $q_{1}q_{2}$ is a polynomial multiple of the minimal polynomial of $T$.
    \end{enumerate}
\end{proof}
\newpage

% chapter5:sectionB:exercise26
\begin{exercise}
    Suppose $V$ is finite-dimensional, $T\in\lmap{V}$, and $U$ is a subspace of $V$ that is invariant under $T$. Prove that the set of eigenvalues of $T$ equals the union of the set of eigenvalues of $T\vert_{U}$ and the set of eigenvalues of $T/U$.
\end{exercise}

\begin{proof}
    Let $\mu_{T}$, $\mu_{T\vert_{U}}$, $\mu_{T/U}$ be minimal polynomials of $T$, $T\vert_{U}$, $T/U$, respectively. We have
    \begin{enumerate}[label={(\arabic*)}]
        \item $\mu_{T}$ is a polynomial multiple of $\mu_{T\vert_U}$.
        \item $\mu_{T}$ is a polynomial multiple of $\mu_{T/U}$.
        \item $\mu_{T\vert_{U}}\times \mu_{T/U}$ is a polynomial multiple of $\mu_{T}$.
    \end{enumerate}

    According to (1) and (2), if $\lambda$ is an eigenvalue of $T\vert_{U}$ or $T/U$, then $\lambda$ is also an eigenvalue of $T$.

    According to (3), if $\lambda$ is an eigenvalue of $T$, then $\lambda$ is an eigenvalue of $T\vert_{U}$ or $T/U$.
\end{proof}
\newpage

% chapter5:sectionB:exercise27
\begin{exercise}
    Suppose $\mathbb{F} = \mathbb{R}$, $V$ is finite-dimensional, and $T\in\lmap{V}$. Prove that the minimal polynomial of $T_{\mathbb{C}}$ equals the minimal polynomial of $T$.
\end{exercise}

\begin{proof}
    Let $p$ be the minimal polynomial of $T$, $q$ be the minimal polynomial of $T_{\mathbb{C}}$. For all $u, v\in V$ and every nonnegative integer $k$
    \[
        T_{\mathbb{C}}^{k}(u + \iota v) = T^{k}u + \iota T^{k}v.
    \]

    So
    \[
        p(T_{\mathbb{C}})(u + \iota v) = p(T)u + \iota p(T)v = 0 + \iota 0.
    \]

    Hence $p$ is a polynomial multiple of $q$. On the other hand
    \[
        0 = q(T_{\mathbb{C}})(u + \iota v) = q(T)u + \iota q(T)v.
    \]

    So $q(T) = 0$. It follows that $q$ is a polynomial multiple of $p$.

    Also, $p, q$ are monic polynomials. Hence $p = q$.
\end{proof}
\newpage

% chapter5:sectionB:exercise28
\begin{exercise}\label{chapter5:sectionB:exercise28}
    Suppose $V$ is finite-dimensional and $T \in \lmap{V}$. Prove that the minimal polynomial of $T' \in \lmap{V'}$ equals the minimal polynomial of $T$.
\end{exercise}

\begin{proof}
    Since $(ST)' = T'S'$, then the dual map of $T^{n}$ is ${(T')}^{n}$, for every nonnegative integer $n$. For every $p\in\mathscr{P}(\mathbb{F})$, every $\varphi\in V'$, we have
    \[
        p(T')(\varphi) = \varphi\circ p(T).
    \]

    If $p$ is the minimal polynomial of $T$, then $p$ is a polynomial multiple of the minimal polynomial of $T'$.

    If $p$ is the minimal polynomial of $T'$, then $\varphi\circ p(T) = 0$ for all $\varphi\in V'$. Assume $p(T)\ne 0$, then there exists $v\in V$ such that $p(T)v = v_{1}\ne 0$. Let $v_{1}, \ldots, v_{n}$ be a basis of $V$ and $\varphi_{1}, \ldots, \varphi_{n}$ be the dual basis of $V$, then $\varphi_{1}(v_{1}) = 1\ne 0$, which is a contradiction to $\varphi\circ p(T) = 0$. Hence $p(T) = 0$. So $p$ is a polynomial multiple of the minimal polynomial of $T$.

    Thus the minimal polynomial of $T' \in \lmap{V'}$ equals the minimal polynomial of $T$.
\end{proof}
\newpage

% chapter5:sectionB:exercise29
\begin{exercise}
    Show that every operator on a finite-dimensional vector space of dimension at least two has an invariant subspace of dimension two.
\end{exercise}

\begin{proof}
    Unsolved.
\end{proof}
\newpage

\section{Upper-Triangular Matrices}

% chapter5:sectionC:exercise1
\begin{exercise}
    Prove or give a counterexample: If $T \in \lmap{V}$ and $T^{2}$ has an upper-triangular matrix with respect to some basis of $V$, then $T$ has an upper-triangular matrix with respect to some basis of $V$.
\end{exercise}

\begin{proof}
    I give a counterexample.

    Let $V = \mathbb{R}^{2}$ and $T(x, y) = (-y, x)$. Then $T^{2} = -I$, $T^{2}$ has an upper-triangular matrix with respect to any basis of $V$. However, the minimal polynomial of $T$ is $p(z) = z^{2} + 1$, which is not a product of monic polynomial of degree $1$. Therefore with respect to any basis of $V$, the matrix of $T$ is not an upper-triangular matrix.
\end{proof}
\newpage

% chapter5:sectionC:exercise2
\begin{exercise}\label{chapter5:sectionC:exercise2}
    Suppose $A$ and $B$ are upper-triangular matrices of the same size, with $\alpha_{1} , \ldots, \alpha_{n}$ on the diagonal of $A$ and $\beta_{1} , \ldots, \beta_{n}$ on the diagonal of $B$.
    \begin{enumerate}[label={(\alph*)}]
        \item Show that $A + B$ is an upper-triangular matrix with $\alpha_{1} + \beta_{1}, \ldots, \alpha_{n} + \beta_{n}$ on the diagonal.
        \item Show that $AB$ is an upper-triangular matrix with $\alpha_{1}\beta_{1}, \ldots, \alpha_{n}\beta_{n}$ on the diagonal.
    \end{enumerate}
\end{exercise}

\begin{proof}
    \begin{enumerate}[label={(\alph*)}]
        \item ${(A + B)}_{j,k} = A_{j,k} + B_{j,k}$. If $j > k$, then $A_{j,k} + B_{j,k} = 0$, and it follows that ${(A + B)}_{j,k} = 0$. Hence $A + B$ is an upper-triangular matrix.

              If $j = k$, then $A_{j,j} = \alpha_{j}$ and $B_{j,j} = \beta_{j}$. So ${(A + B)}_{j,j} = \alpha_{j} + \beta_{j}$.

              Thus $A + B$ is an upper-triangular matrix with $\alpha_{1} + \beta_{1}, \ldots, \alpha_{n} + \beta_{n}$ on the diagonal.
        \item If $j > k$
              \begin{align*}
                  {(AB)}_{j,k} & = \sum^{n}_{r=1}A_{j,r}B_{r,k}                                                                                                       \\
                               & = \sum^{j}_{r=1}A_{j,r}B_{r,k} + \sum^{n}_{r=j+1}A_{j,r}B_{r,k}                                                                      \\
                               & = 0 + 0 = 0                                                     & \text{($B_{r,k} = 0$ for $r\leq j$ and $A_{j,r} = 0$ for $r > j$)}
              \end{align*}

              If $j = k$
              \begin{align*}
                  {(AB)}_{j,j} & = \sum^{n}_{r=1}A_{j,r}B_{r,j}                                                     \\
                               & = \sum^{j-1}_{r=1}A_{j,r}B_{r,j} + A_{j,j}B_{j,j} + \sum^{n}_{r=j+1}A_{j,r}B_{r,j} \\
                               & = 0 + \alpha_{j}\beta_{j} + 0                                                      \\
                               & = \alpha_{j}\beta_{j}.
              \end{align*}

              Thus $AB$ is an upper-triangular matrix with $\alpha_{1}\beta_{1}, \ldots, \alpha_{n}\beta_{n}$ on the diagonal.
    \end{enumerate}
\end{proof}
\newpage

% chapter5:sectionC:exercise3
\begin{exercise}
    Suppose $T \in \lmap{V}$ is invertible and $v_{1}, \ldots, v_{n}$ is a basis of $V$ with respect to which the matrix of $T$ is upper triangular, with $\lambda_{1} , \ldots, \lambda_{n}$ on the diagonal. Show that the matrix of $T^{-1}$ is also upper triangular with respect to the basis $v_{1}, \ldots, v_{n}$, with
    \[
        \frac{1}{\lambda_{1}}, \ldots, \frac{1}{\lambda_{n}}
    \]

    on the diagonal.
\end{exercise}

\begin{proof}
    Let $A = \mathcal{M}(T, (v_{1}, \ldots, v_{n}))$.

    Using mathematical induction, I will show that $T^{-1}v_{k}\in\operatorname{span}(v_{1}, \ldots, v_{k})$ for each $k = 1,\ldots, n$.

    $Tv_{1} = \lambda_{1}v_{1}$ and $T$ is invertible so $T^{-1}v_{1} = \dfrac{1}{\lambda_{1}}v_{1}$.

    Assume $T^{-1}v_{k}\in\operatorname{span}(v_{1}, \ldots, v_{k})$ for $1\leq k < n$. $(T - \lambda_{k+1}I)(v_{k+1}) \in \operatorname{span}(v_{1}, \ldots, v_{k})$ and $\operatorname{span}(v_{1}, \ldots, v_{k})$ is invariant under $T^{-1}$, so
    \[
        \frac{-1}{\lambda_{k}}T^{-1}(T - \lambda_{k+1}I)(v_{k+1}) \in \operatorname{span}(v_{1}, \ldots, v_{k}).
    \]

    So $(T^{-1} - \frac{1}{\lambda_{k+1}}I)(v_{k+1}) \in \operatorname{span}(v_{1}, \ldots, v_{k})$. Therefore $T^{-1}v_{k+1}\in \operatorname{span}(v_{1}, \ldots, v_{k+1})$.

    By the principle of mathematical induction, we conclude that $T^{-1}v_{k}\in\operatorname{span}(v_{1}, \ldots, v_{k})$ for each $k = 1,\ldots, n$. So the matrix of $T^{-1}$ with respect to $v_{1}, \ldots, v_{n}$ is upper-triangular.

    Moreover, the product of the matrix of $T$ and $T^{-1}$ with respect to $v_{1}, \ldots, v_{n}$ is the identity matrix. By Exercise~\ref{chapter5:sectionC:exercise2}, the matrix $T^{-1}$ with respect to $v_{1}, \ldots, v_{n}$ is upper-triangular with
    \[
        \frac{1}{\lambda_{1}}, \ldots, \frac{1}{\lambda_{n}}
    \]

    on the diagonal.
\end{proof}
\newpage

% chapter5:sectionC:exercise4
\begin{exercise}
    Give an example of an operator whose matrix with respect to some basis contains only $0$'s on the diagonal, but the operator is invertible.
\end{exercise}

\begin{proof}
    On $\mathbb{F}^{n}$ where $n > 1$, we define the linear operator $T$ as follows: $Te_{1} = e_{2}$, $Te_{2} = e_{3}$, \ldots, $Te_{n-1} = e_{n}$, $Te_{n} = e_{1}$. The minimal polynomial of $T$ is $z^{n} - 1$, of which constant term is nonzero. Therefore $T$ is invertible. On the other hand, the matrix of $T$ with respect to the standard basis $e_{1}, \ldots, e_{n}$ contains only $0$'s on the diagonal.
\end{proof}
\newpage

% chapter5:sectionC:exercise5
\begin{exercise}
    Give an example of an operator whose matrix with respect to some basis contains only nonzero numbers on the diagonal, but the operator is not invertible.
\end{exercise}

\begin{proof}
    On $\mathbb{F}^{n}$ where $n > 1$, we define the linear operator $T$ as follows:
    \[
        T(x_{1}, \ldots, x_{n}) = (x_{1} + \cdots + x_{n}, \ldots, x_{1} + \cdots + x_{n}).
    \]

    The minimal polynomial of $T$ is $T^{2} - nT$. All entries of the matrix of $T$ with respect to the standard basis are $1$. $T$ is not invertible because the constant term of the minimal polynomial $T^{2} - nT$ is $0$.
\end{proof}
\newpage

% chapter5:sectionC:exercise6
\begin{exercise}
    Suppose $\mathbb{F} = \mathbb{C}$, $V$ is finite-dimensional, and $T \in \lmap{V}$. Prove that if $k \in \{1, \ldots, \dim V\}$, then $V$ has a $k$-dimensional subspace invariant under $T$.
\end{exercise}

\begin{proof}
    According to 5.47, $T$ has an upper-triangular matrix with respect to some basis $v_{1}, \ldots, v_{n}$ of $V$.

    Therefore, for every $k \in \{1, \ldots, \dim V\}$, $\operatorname{span}(v_{1}, \ldots, v_{k})$ is invariant under $T$. So for every $k \in \{1, \ldots, \dim V\}$, $V$ has a $k$-dimensional subspace invariant under $T$.
\end{proof}
\newpage

% chapter5:sectionC:exercise7
\begin{exercise}
    Suppose $V$ is finite-dimensional, $T\in\lmap{V}$, and $v\in V$.
    \begin{enumerate}[label={(\alph*)}]
        \item Prove that there exists a unique monic polynomial $p_{v}$ of smallest degree such that $p_{v}(T)v = 0$.
        \item Prove that the minimal polynomial of $T$ is a polynomial multiple of $p_{v}$.
    \end{enumerate}
\end{exercise}

\begin{proof}
    Let $p$ be the minimal polynomial of $T$.
    \begin{enumerate}[label={(\alph*)}]
        \item $p$ is monic and $p(T)v = 0$. So there exists a monic polynomial $p_{v}$ of smallest degree such that $p_{v}(T)v = 0$.

              Assume $q$ is a monic polynomial such that $q(T)v = 0$ and $\deg q = \deg p$. By Euclidean division algorithm, there exist unique polynomials $s$, $r$ such that $q = sp_{v} + r$ and $\deg r < \deg p_{v}$. So
              \[
                  0 = q(T)v = s(T)(p_{v}(T)v) + r(v) = r(v)
              \]

              Because $\deg r < \deg p_{v}$, it follows that $r = 0$. Hence $q = sp_{v}$. $s$ must be monic and $\deg s = 0$ so $s = 1$. Therefore $q = p_{v}$. Thus there exists a unique monic polynomial $p_{v}$ of smallest degree such that $p_{v}(T)v = 0$.
        \item By Euclidean division algorithm, there exist unique polynomials $s$, $r$ such that $p = sp_{v} + r$ and $\deg r < \deg p_{v}$.
              \[
                  0 = p(T)v = s(T)(p_{v}(T)v) + r(T)v = r(T)v.
              \]

              Because $\deg r < \deg p_{v}$, it follows that $\deg r = 0$. Hence $p$ is a polynomial multiple of $p_{v}$.
    \end{enumerate}
\end{proof}
\newpage

% chapter5:sectionC:exercise8
\begin{exercise}
    Suppose $V$ is finite-dimensional, $T \in \lmap{V}$, and there exists a nonzero vector $v \in V$ such that $T^{2}v + 2Tv = -2v$.
    \begin{enumerate}[label={(\alph*)}]
        \item Prove that if $\mathbb{F} = \mathbb{R}$, then there does not exist a basis of $V$ with respect to which $T$ has an upper-triangular matrix.
        \item Prove that if $\mathbb{F} = \mathbb{C}$ and $A$ is an upper-triangular matrix that equals the matrix of $T$ with respect to some basis of $V$, then $-1 + \iota$ or $-1 - \iota$ appears on the diagonal of $A$.
    \end{enumerate}
\end{exercise}

\begin{proof}
    \begin{enumerate}[label={(\alph*)}]
        \item Assume $T$ has an eigenvalue $\lambda$. Let $w$ be an eigenvector of $T$ corresponding to $\lambda$, then
              \[
                  -2w = T^{2}w + 2Tw = \lambda^{2}v + 2\lambda w.
              \]

              So $(\lambda^{2} + 2\lambda + 2)v = 0$. It follows that $\lambda^{2} + 2\lambda + 2 = 0$. However, there is no real number $\lambda$ such that $\lambda^{2} + 2\lambda + 2 = 0$ so the assumption is false. Hence $T$ has no eigenvalue. Thus there does not exist a basis of $V$ with respect to which $T$ has an upper-triangular matrix.
        \item The polynomial $p$ where $p(z) = z^{2} + 2z + 2$ satisfies $p(T) = 0$, so $p$ is a polynomial multiple of the minimal polynomial of $T$. The two roots of $p$ are $-1+\iota$ and $-1-\iota$.

              So the minimal polynomial of $T$ is of the form
              \[
                  {(z + 1 - \iota)}^{m}{(z + 1 + \iota)}^{n}
              \]

              where $m$ and $n$ are not both $0$ and $m + n\leq 2$. Hence $-1 + \iota$ or $-1-\iota$ appears on the diagonal of $A$.
    \end{enumerate}
\end{proof}
\newpage

% chapter5:sectionC:exercise9
\begin{exercise}
    Suppose $B$ is a square matrix with complex entries. Prove that there exists an invertible square matrix $A$ with complex entries such that $A^{-1}BA$ is an upper-triangular matrix.
\end{exercise}

\begin{proof}
    Let $n$ be the number of columns of $B$.

    I define the linear operator $T$ on $\mathbb{C}^{n}$ as follows: $Tx = Bx$ for all $x\in\mathbb{C}^{n,1}$.

    There exists a basis $v_{1}, \ldots, v_{n}$ of $\mathbb{C}^{n}$ such that the matrix of $T$ with respect to this basis is an upper-triangular matrix. Let
    \[
        A = \mathcal{M}(I, (v_{1}, \ldots, v_{n}), (e_{1}, \ldots, e_{n}))
    \]

    where $e_{1}, \ldots, e_{n}$ is the standard basis of $\mathbb{C}^{n}$. By the change-of-basis formula,
    \[
        \mathcal{M}(T, (v_{1}, \ldots, v_{n})) = A^{-1}BA.
    \]

    $\mathcal{M}(T, (v_{1}, \ldots, v_{n}))$ is an upper-triangular matrix.
\end{proof}
\newpage

% chapter5:sectionC:exercise10
\begin{exercise}\label{chapter5:sectionC:exercise10}
    Suppose $T\in\lmap{V}$ and $v_{1}, \ldots, v_{n}$ is a basis of $V$. Show that the following are equivalent.
    \begin{enumerate}[label={(\alph*)}]
        \item The matrix of $T$ with respect to $v_{1}, \ldots, v_{n}$ is lower triangular.
        \item $\operatorname{span}(v_{k}, \ldots, v_{n})$ is invariant under $T$ for each $k = 1, \ldots, n$.
        \item $Tv_{k}\in \operatorname{span}(v_{k}, \ldots, v_{n})$ for each $k = 1, \ldots, n$.
    \end{enumerate}
\end{exercise}

\begin{proof}
    $(a) \Rightarrow (b)$ Since the matrix of $T$ with respect to $v_{1}, \ldots, v_{n}$ is lower triangular, then for each $k\in 1,\ldots, n$, for each $j\in k,\ldots, n$, $Tv_{j}\in \operatorname{span}(v_{k}, \ldots, v_{n})$. Therefore $\operatorname{span}(v_{k}, \ldots, v_{n})$ is invariant under $T$.

    $(b) \Rightarrow (c)$ For each $k = 1,\ldots, n$, $\operatorname{span}(v_{k}, \ldots, v_{n})$ is invariant under $T$, then $Tv_{k}\in \operatorname{span}(v_{k}, \ldots, v_{n})$.

    $(c) \Rightarrow (a)$ Let $A$ be the matrix of $T$ with respect to $v_{1}, \ldots, v_{n}$. Since $Tv_{k}\in \operatorname{span}(v_{k}, \ldots, v_{n})$, then $A_{j,k} = 0$ if $j < k$. Therefore $A$ is a lower-triangular matrix.
\end{proof}
\newpage

% chapter5:sectionC:exercise11
\begin{exercise}
    Suppose $\mathbb{F} = \mathbb{C}$ and $V$ is finite-dimensional. Prove that if $T\in\lmap{V}$, then there exists a basis of $V$ with respect to which $T$ has a lower-triangular matrix.
\end{exercise}

\begin{proof}
    Because $T$ is a linear operator on a complex vector space, there exists a basis $v_{1}, \ldots, v_{n}$ with respect to which the matrix of $T$ is upper triangular.

    Let $w_{1} = v_{n}, \ldots, w_{n} = v_{1}$ then $w_{1}, \ldots, w_{n}$ is a basis of $V$. Because $Tv_{k}\in \operatorname{span}(v_{1}, \ldots, v_{k})$ for each $k = 1,\ldots, n$, then $Tw_{k}\in \operatorname{span}(w_{k}, \ldots, w_{n})$ for each $k = 1,\ldots, n$.

    By Exercise~\ref{chapter5:sectionC:exercise10}, it follows that the matrix of $T$ with respect to $w_{1}, \ldots, w_{n}$ is a lower-triangular matrix.
\end{proof}
\newpage

% chapter5:sectionC:exercise12
\begin{exercise}
    Suppose $V$ is finite-dimensional, $T\in\lmap{V}$ has an upper-triangular matrix with respect to some basis of $V$, and $U$ is a subspace of $V$ that is invariant under $T$.
    \begin{enumerate}[label={(\alph*)}]
        \item Prove that $T\vert_{U}$ has an upper-triangular matrix with respect to some basis of $U$.
        \item Prove that the quotient operator $T/U$ has an upper-triangular matrix with respect to some basis of $V/U$.
    \end{enumerate}
\end{exercise}

\begin{proof}
    Let $p$ be the minimal polynomial of $T$. Because $T$ has an upper-triangular matrix with respect to some basis of $V$, then $p$ is a product of monic polynomials of degree $1$.
    \begin{enumerate}[label={(\alph*)}]
        \item The polynomial of $T$ is a polynomial multiple of the minimal polynomial of $T\vert_{U}$. Therefore the minimal polynomial of $T\vert_{U}$ is also a product of monic polynomials of degree $1$. Hence $T\vert_{U}$ has an upper-triangular matrix with respect to some basis of $U$.
        \item The polynomial of $T$ is a polynomial multiple of the minimal polynomial of $T/U$. Therefore the minimal polynomial of $T/U$ is also a product of monic polynomials of degree $1$. Hence $T/U$ has an upper-triangular matrix with respect to some basis of $U$.
    \end{enumerate}
\end{proof}
\newpage

% chapter5:sectionC:exercise13
\begin{exercise}
    Suppose $V$ is finite-dimensional and $T\in \lmap{V}$. Suppose there exists a subspace $U$ of $V$ that is invariant under $T$ such that $T\vert_{U}$ has an upper-triangular matrix with respect to some basis of $U$ and also $T/U$ has an upper-triangular matrix with respect to some basis of $V/U$. Prove that $T$ has an upper-triangular matrix with respect to some basis of $V$.
\end{exercise}

\begin{proof}
    Let $\mu_{T}$, $\mu_{T\vert_{U}}$, $\mu_{T/U}$ be minimal polynomials of $T$, $T\vert_{U}$, $T/U$, respectively.

    $T\vert_{U}$ has an upper-triangular matrix with respect to some basis of $U$, so $\mu_{T\vert_{U}}$ is a product of monic polynomial of degree $1$.

    $T/U$ has an upper-triangular matrix with respect to some basis of $U$, so $\mu_{T/U}$ is a product of monic polynomial of degree $1$.

    By Exercise~\ref{chapter5:sectionB:exercise25}, $\mu_{T\vert_{U}}\times \mu_{T/U}$ is a polynomial multiple of $\mu_{T}$.

    Therefore $\mu_{T}$ is a product of monic polynomial of degree $1$. Equivalently, $T$ has an upper-triangular matrix with respect to some basis of $V$.
\end{proof}
\newpage

% chapter5:sectionC:exercise14
\begin{exercise}
    Suppose $V$ is finite-dimensional and $T \in \lmap{V}$. Prove that $T$ has an upper-triangular matrix with respect to some basis of $V$ if and only if the dual operator $T'$ has an upper-triangular matrix with respect to some basis of the dual space $V'$.
\end{exercise}

\begin{proof}
    By Exercise~\ref{chapter5:sectionB:exercise28}, the minimal polynomials of $T$ and its dual map $T'$ are equal.

    On the other hand, $T$ has an upper-triangular matrix with respect to some basis of $V$ if and only if the minimal polynomial of $T$ is a product of monic polynomials of degree $1$ AND $T'$ has an upper-triangular matrix with respect to some basis of $V'$ if and only if the minimal polynomial of $T'$ is a product of monic polynomials of degree $1$.

    Thus $T$ has an upper-triangular matrix with respect to some basis of $V$ if and only if the dual operator $T'$ has an upper-triangular matrix with respect to some basis of the dual space $V'$.
\end{proof}
\newpage

\section{Diagonalizable Operators}

% chapter5:sectionD:exercise1
\begin{exercise}
    Suppose $V$ is a finite-dimensional complex vector space and $T\in\lmap{V}$.
    \begin{enumerate}[label={(\alph*)}]
        \item Prove that if $T^{4} = I$, then $T$ is diagonalizable.
        \item Prove that if $T^{4} = T$, then $T$ is diagonalizable.
        \item Give an example of an operator $T\in\lmap{\mathbb{C}^{2}}$ such that $T^{4} = T^{2}$ and $T$ is not diagonalizable.
    \end{enumerate}
\end{exercise}

\begin{proof}
    \begin{enumerate}[label={(\alph*)}]
        \item $z^{4} - 1$ is a polynomial multiple of the minimal polynomial of $T$. $z^{4} - 1$ has four distinct zeros in $\mathbb{C}$ (they are $1$, $-1$, $\iota$, $-\iota$), so the minimal polynomial of $T$ has no multiple zero. Hence $T$ is diagonalizable.
        \item $z^{4} - z$ is a polynomial multiple of the minimal polynomial of $T$. $z^{4} - z$ has four distinct zeros in $\mathbb{C}$ (they are $0$, $1$, $\frac{-1-\sqrt{3}\iota}{2}$, $\frac{-1+\sqrt{3}\iota}{2}$), so the minimal polynomial of $T$ has no multiple zero. Hence $T$ is diagonalizable.
        \item I define the linear operator $T\in\lmap{\mathbb{C}^{2}}$ as follows:
              \[
                  T(z_{1}, z_{2}) = (z_{2}, 0).
              \]

              The minimal polynomial of $T$ is $z^{2}$, so $T^{4} = T^{2}$. However, since the minimal polynomial of $T$ has multiple root, it follows that $T$ is not diagonalizable.
    \end{enumerate}
\end{proof}
\newpage

% chapter5:sectionD:exercise2
\begin{exercise}\label{chapter5:sectionD:exercise2}
    Suppose $T\in\lmap{V}$ has a diagonal matrix $A$ with respect to some basis of $V$. Prove that if $\lambda\in\mathbb{F}$, then $\lambda$ appears on the diagonal of $A$ precisely $\dim E(\lambda, T)$ times.
\end{exercise}

\begin{proof}
    Let $v_{1}, \ldots, v_{n}$ be a basis with respect to which $T$ has a diagonal matrix $A$.

    If $\lambda$ is not an eigenvalue of $T$, then $\lambda$ does not appear on the diagonal of $A$, which means it appears $0 = \dim E(\lambda, T)$ times on the diagonal of $A$.

    If $\lambda$ is an eigenvalue of $T$, then $\lambda$ appears on the diagonal of $A$. Without loss of generality, assume that $v_{1}, \ldots, v_{m}$ are precisely the vectors in the list $v_{1}, \ldots, v_{n}$ that are also eigenvectors of $T$ corresponding to $\lambda$. Let $v$ be a vector in $E(\lambda, T)$. There exist scalars $x_{1}, \ldots, x_{n}$ such that
    \[
        v = x_{1}v_{1} + \cdots + x_{m}v_{m} + x_{m+1}v_{m+1} + \cdots + x_{n}v_{n}.
    \]

    Since $v_{m+1}, \ldots, v_{n}$ are also eigenvectors of $T$ but not corresponding to $\lambda$, there exist scalars $\lambda_{m+1}, \ldots, \lambda$ that are not equal to $\lambda$ such that $Tv_{m+1} = \lambda_{m+1} v_{m+1}$, \ldots, $Tv_{n} = \lambda_{n}v_{n}$. From $Tv = \lambda v$, we deduce that
    \begin{align*}
        T(x_{1}v_{1} + \cdots + x_{m}v_{m} + x_{m+1}v_{m+1} + \cdots + x_{n}v_{n})
         & = \lambda (x_{1}v_{1} + \cdots + x_{m}v_{m}) + \lambda (x_{m+1}v_{m+1} + \cdots + x_{n}v_{n}).
    \end{align*}
    \begin{multline*}
        T(x_{1}v_{1} + \cdots + x_{m}v_{m} + x_{m+1}v_{m+1} + \cdots + x_{n}v_{n}) \\
        = x_{1}Tv_{1} + \cdots + x_{m}Tv_{m} + x_{m+1}Tv_{m+1} + \cdots + x_{n}Tv_{n} \\
        = \lambda (x_{1}v_{1} + \cdots + x_{m}v_{m}) + (\lambda_{m+1}x_{m+1}v_{m+1} + \cdots + \lambda_{n}x_{n}v_{n}).
    \end{multline*}

    Hence $(\lambda - \lambda_{k})x_{k} = 0$ for each $k\in\{ m+1, \ldots, n \}$. However, $\lambda \ne \lambda_{k}$ for each $k\in\{ m+1, \ldots, n \}$ so $x_{k} = 0$ for each $k\in\{ m+1, \ldots, n \}$.

    Therefore $v_{1}, \ldots, v_{m}$ is an independent list that spans $E(\lambda, T)$, so $m = \dim E(\lambda, T)$. This means $\lambda$ appears $\dim E(\lambda, T)$ on the diagonal of $A$.

    Thus $\lambda$ appears on the diagonal of $A$ precisely $\dim E(\lambda, T)$ times.
\end{proof}
\newpage

% chapter5:sectionD:exercise3
\begin{exercise}\label{chapter5:sectionD:exercise3}
    Suppose $V$ is finite-dimensional and $T \in \lmap{V}$. Prove that if the operator $T$ is diagonalizable, then $V = \kernel{T}\oplus\range{T}$.
\end{exercise}

\begin{proof}
    Let $v_{1}, \ldots, v_{n}$ be a basis with respect to which $T$ has a diagonal matrix, and $Tv_{k} = \lambda_{k}v_{k}$ for each $k\in \{ 1, \ldots, n \}$.

    $\kernel{T} = \kernel{(T - 0I)} = E(0, T)$, let $m = \dim E(0, T)$. By Exercise~\ref{chapter5:sectionD:exercise2}, $0$ appears $m$ times on the diagonal of the matrix of $T$ with respect to $v_{1}, \ldots, v_{n}$. Without loss of generality, let $Tv_{k} = \lambda_{k}v_{k}$ for each $k\in \{ 1, \ldots, n \}$, where $\lambda_{k} = 0$ for each $k\in\{ 1, \ldots, m \}$, and $\lambda_{k} \ne 0$ for each $k\in\{ m+1, \ldots, n \}$. Also by Exercise~\ref{chapter5:sectionD:exercise2}, $v_{1}, \ldots, v_{m}$ is a basis of $\kernel{T}$.
    \begin{align*}
        T(x_{1}v_{1} + \cdots + x_{n}v_{n}) & = x_{1}Tv_{1} + \cdots + x_{m}Tv_{m} + x_{m+1}Tv_{m+1} + \cdots + x_{n}Tv_{n} \\
                                            & = x_{m+1}Tv_{m+1} + \cdots + x_{n}Tv_{n}                                      \\
                                            & = x_{m+1}\lambda_{m+1}v_{m+1} + \cdots + x_{n}\lambda_{n}v_{n}.
    \end{align*}

    So $v_{m+1}, \ldots, v_{n}$ spans $T$ and $Tv_{m+1}, \ldots, Tv_{n}$ spans $\range{T}$. Suppose
    \[
        a_{m+1}Tv_{m+1} + \cdots + a_{n}Tv_{n} = 0.
    \]

    Then $T(a_{m+1}v_{m+1} + \cdots + a_{n}v_{n}) = 0$ and $a_{m+1}v_{m+1} + \cdots + a_{n}v_{n}\in\kernel{T}$. Since $v_{1}, \ldots, v_{m}$ is a basis of $\kernel{T}$, there exist scalars $b_{1}, \ldots, b_{m}$ such that
    \[
        a_{m+1}v_{m+1} + \cdots + a_{n}v_{n} = b_{1}v_{1} + \cdots + b_{m}v_{m}.
    \]

    Because $v_{1}, \ldots, v_{n}$ is a basis of $V$, it follows that $a_{m+1} = \cdots = a_{n} = 0$ and $b_{1} = \cdots = b_{m} = 0$. So $Tv_{m+1}, \ldots, Tv_{n}$ is a basis of $\range{T}$, $v_{m+1}, \ldots, v_{n}$ is a basis of $\range{T}$, and $\kernel{T}\cap\range{T} = \{0\}$.

    On the other hand
    \[
        x_{1}v_{1} + \cdots + x_{n}v_{n} = \underbrace{x_{1}v_{1} + \cdots + x_{m}v_{m}}_{\in\kernel{T}} + \underbrace{x_{m+1}v_{m+1} + \cdots + x_{n}v_{n}}_{\in\range{T}}
    \]

    so $V = \kernel{T} + \range{T}$. Thus $V = \kernel{T}\oplus\range{T}$.
\end{proof}
\newpage

% chapter5:sectionD:exercise4
\begin{exercise}
    Suppose $V$ is finite-dimensional and $T\in\lmap{V}$. Prove that the following are equivalent.
    \begin{enumerate}[label={(\alph*)}]
        \item $V = \kernel{T}\oplus \range{T}$.
        \item $V = \kernel{T} + \range{T}$.
        \item $\kernel{T}\cap \range{T} = \{0\}$.
    \end{enumerate}
\end{exercise}

\begin{proof}
    (a) implies (b) and (c) due to the definition of direct sum.

    According to the fundamental theorem of linear maps,
    \[
        \dim V = \dim\kernel{T} + \dim\range{T}.
    \]

    Moreover,
    \[
        \dim V = \dim\kernel{T} + \dim\range{T} = \dim (\kernel{T} + \range{T}) - \dim (\kernel{T}\cap\range{T}).
    \]

    Therefore (b) implies (c) and (c) implies (a).
\end{proof}
\newpage

% chapter5:sectionD:exercise5
\begin{exercise}
    Suppose $V$ is a finite-dimensional complex vector space and $T$. Prove that $T$ is diagonalizable if and only if
    \[
        V = \kernel{(T - \lambda I)} \oplus \range{(T - \lambda I)}
    \]

    for every $\lambda\in\mathbb{C}$.
\end{exercise}

\begin{proof}
    If $T$ is diagonalizable, then there exists a basis $v_{1}, \ldots, v_{n}$ of $V$ with respect to which $T$ has a diagonal matrix $A$. Then the matrix of $T - \lambda I$ with respect to $v_{1}, \ldots, v_{n}$ is $A - \lambda I$, which is a diagonal matrix. So $T - \lambda I$ is diagonalizable. By Exercise~\ref{chapter5:sectionD:exercise3}, $V = \kernel{(T - \lambda I)}\oplus \range{(T - \lambda I)}$ for all $\lambda\in \mathbb{C}$.

    To prove the implication of the other direction, I use mathematical induction on $\dim V$. Assume the statement is true for every complex vector space of dimension less than $\dim V$.

    Let $\lambda$ be an eigenvalue of $T$, then $\kernel{(T - \lambda I)}\ne \{0\}$ and $\range{(T - \lambda I)}$ is a proper subspace of $V$. Since $\range{(T - \lambda I)}$ is invariant under $T$ and $\range{(T - \lambda I)}$ is a proper subspace of $V$ so according to the induction hypothesis, the restriction of $T$ on $\range{(T - \lambda I)}$ is diagonalizable.

    So there exists a basis of $\range{(T - \lambda I)}$ consisting of eigenvectors of the restriction of $T$ on $\range{(T - \lambda I)}$. These vectors are also eigenvectors of $T$, let them be $v_{1}, \ldots, v_{m}$. Let $v_{m+1}, \ldots, v_{m+n}$ be a basis of $\kernel{(T - \lambda I)}$.

    Because $V = \kernel{(T - \lambda I)} \oplus \range{(T - \lambda I)}$, we conclude that $v_{1}, \ldots, v_{m}, v_{m+1}, \ldots, v_{m+n}$ is a basis of $V$. Further, these vectors are eigenvectors of $T$, so $T$ is diagonalizable.

    According to the principle of mathematical induction, if $V = \kernel{(T - \lambda I)} \oplus \range{(T - \lambda I)}$ for every $\lambda\in\mathbb{C}$, then $T$ is diagonalizable.
\end{proof}
\newpage

% chapter5:sectionD:exercise6
\begin{exercise}
    Suppose $T\in\lmap{\mathbb{F}^{5}}$ and $\dim E(8, T) = 4$. Prove that $T - 2I$ or $T - 6I$ is invertible.
\end{exercise}

\begin{proof}
    $z - 8$ is not the minimal polynomial of $T$, because if so, $T = 8I$ and it follows that $\dim E(8, T) = 5\ne 4$.

    On the other hand, the minimal polynomial of $T$ is a multiple polynomial of $z - 8$ and has at least another zero other than $8$. Let $8, \lambda_{1}, \ldots, \lambda_{n}$ be the distinct roots of the minimal polynomial of $T$, then if $n > 1$
    \[
        \dim E(8, T) + \dim (\lambda_{1}, T) + \cdots + \dim (\lambda_{n}, T) > \dim E(8, T) + 1 = 5.
    \]

    Meanwhile,
    \[
        \dim E(8, T) + \dim (\lambda_{1}, T) + \cdots + \dim (\lambda_{n}, T) \leq \dim \mathbb{F}^{5} = 5.
    \]

    So the assumption is false. Hence $p$ has exactly one zero other than $8$, let it be $\lambda$. $\lambda$ cannot be both $2$ and $6$, so $T - 2I$ or $T - 6I$ is invertible.
\end{proof}
\newpage

% chapter5:sectionD:exercise7
\begin{exercise}
    Suppose $T\in\lmap{V}$ is invertible. Prove that
    \[
        E(\lambda, T) = E\left(\frac{1}{\lambda}, T^{-1}\right)
    \]

    for every $\lambda\in\mathbb{F}$ with $\lambda\ne 0$.
\end{exercise}

\begin{proof}
    \begin{align*}
        v\in E(\lambda, T) & \Longleftrightarrow Tv = \lambda v                                \\
                           & \Longleftrightarrow v = T^{-1}(\lambda v)                         \\
                           & \Longleftrightarrow \frac{1}{\lambda}v = T^{-1}v                  \\
                           & \Longleftrightarrow v\in E\left(\frac{1}{\lambda}, T^{-1}\right).
    \end{align*}

    Thus $E(\lambda, T) = E\left(\frac{1}{\lambda}, T^{-1}\right)$.
\end{proof}
\newpage

% chapter5:sectionD:exercise8
\begin{exercise}
    Suppose $V$ is finite-dimensional and $T\in\lmap{V}$. Let $\lambda_{1}, \ldots, \lambda_{m}$ denote the distinct nonzero eigenvalues of $T$. Prove that
    \[
        \dim E(\lambda_{1}, T) + \cdots + \dim E(\lambda_{m}, T)\leq \dim\range{T}.
    \]
\end{exercise}

\begin{proof}
    Let $v$ be a vector in $E(\lambda_{1}, T) + \cdots + E(\lambda_{m}, T)$, then there exist vectors $v_{k}\in E(\lambda_{k}, T)$ for each $k\in\{ 1, \ldots, m \}$ such that
    \[
        v = v_{1} + \cdots + v_{m}.
    \]

    On the other hand
    \begin{align*}
        v & = v_{1} + \cdots + v_{m}                                                                 \\
          & = \frac{1}{\lambda_{1}}\lambda_{1}v_{1} + \cdots + \frac{1}{\lambda_{m}}\lambda_{1}v_{m} \\
          & =  \frac{1}{\lambda_{1}}Tv_{1} + \cdots + \frac{1}{\lambda_{m}}Tv_{m}                    \\
          & = T\left(\frac{1}{\lambda_{1}}v_{1} + \cdots + \frac{1}{\lambda_{m}}v_{m}\right).
    \end{align*}

    So $v\in \range{T}$, therefore $E(\lambda_{1}, T) + \cdots + E(\lambda_{m}, T)$ is a subspace of $\range{T}$.

    $\lambda_{1}, \ldots, \lambda_{m}$ are distinct eigenvalues of $T$, so the sum
    \[
        E(\lambda_{1}, T) + \cdots + E(\lambda_{m}, T)
    \]

    is a direct sum. Thus
    \[
        \dim E(\lambda_{1}, T) + \cdots + \dim E(\lambda_{m}, T) = \dim(E(\lambda_{1}, T) + \cdots + E(\lambda_{m}, T))\leq \dim\range{T}.\qedhere
    \]
\end{proof}
\newpage

% chapter5:sectionD:exercise9
\begin{exercise}
    Suppose $R, T\in \lmap{\mathbb{F}^{3}}$ each have $2, 6, 7$ as eigenvalues. Prove that there exists an invertible operator $S\in\lmap{\mathbb{F}^{3}}$ such that $R = S^{-1}TS$.
\end{exercise}

\begin{proof}
    $R, T\in \lmap{\mathbb{F}^{3}}$ each have $2, 6, 7$ as eigenvalues. So $R$ and $T$ are diagonalizable.

    Let $v_{1}, v_{2}, v_{3}$ be eigenvectors of $R$ corresponding to $2, 6, 7$, respectively. Let $w_{1}, w_{2}, w_{3}$ be eigenvectors of $T$ corresponding to $2, 6, 7$, respectively. Then $v_{1}, v_{2}, v_{3}$ and $w_{1}, w_{2}, w_{3}$ are two bases of $\mathbb{F}^{3}$. Denote $\lambda_{1} = 2$, $\lambda_{2} = 6$, $\lambda_{3} = 7$.

    I define the operator $S\in\lmap{\mathbb{F}^{3}}$ as follows: $Sw_{1} = v_{1}$, $Sw_{2} = v_{2}$, $Sw_{3} = v_{1}$. $S$ is invertible. For each $i\in\{1, 2, 3\}$,
    \[
        (S^{-1}TS)(w_{i}) = (S^{-1}T)(v_{i}) = S^{-1}(\lambda_{i}v_{i}) = \lambda_{i}S^{-1}v_{i} = \lambda_{i}w_{i} = Rw_{i}.
    \]

    Hence $R = S^{-1}TS$.
\end{proof}
\newpage

% chapter5:sectionD:exercise10
\begin{exercise}
    Find $R, T\in \lmap{\mathbb{F}^{4}}$ such that $R$ and $T$ each have $2, 6, 7$ as eigenvalues, $R$ and $T$ have no other eigenvalues, and there does not exist an invertible operator $S\in\lmap{\mathbb{F}^{4}}$ such that $R = S^{-1}TS$.
\end{exercise}

\begin{proof}
    Let
    \[
        R(z_{1}, z_{2}, z_{3}, z_{4}) = (2z_{1}, 2z_{2}, 6z_{3}, 7z_{4})
    \]

    then $R$ is diagonalizable and the minimal polynomial of $R$ is $(z - 2)(z - 6)(z - 7)$.

    Let $Se_{1} = e_{2}$, $Se_{2} = e_{3}$, $Se_{3} = e_{4}$, $Se_{4} = -168e_{1} + 220e_{2} - 98e_{3} + 17e_{4}$. The matrix of $S$ with respect to the standard basis of $\mathbb{F}^{4}$ is
    \[
        \begin{pmatrix}
            0 & 0 & 0 & -168 \\
            1 & 0 & 0 & 220  \\
            0 & 1 & 0 & -98  \\
            0 & 0 & 1 & 17
        \end{pmatrix}
    \]

    so the minimal polynomial of $S$ is
    \[
        z^{4} - 17z^{3} + 98z^{2} - 220z + 168 = {(z - 2)}^{2}(z - 6)(z - 7).
    \]

    Let $p$ denote the minimal polynomial of $R$. Assume there exists an invertible operator $S\in\lmap{\mathbb{F}^{4}}$ such that $R = S^{-1}TS$, then
    \[
        p(T) = p(SRS^{-1}) = Sp(R)S^{-1} = 0.
    \]

    So $p$ is a polynomial multiple of ${(z - 2)}^{2}(z - 6)(z - 7)$, which is a contradiction. So there does not exist an invertible operator $S\in\lmap{\mathbb{F}^{4}}$ such that $R = S^{-1}TS$ for the given operators $R$ and $T$.
\end{proof}
\newpage

% chapter5:sectionD:exercise11
\begin{exercise}
    Find $T\in\lmap{\mathbb{C}^{3}}$ such that $6$ and $7$ are eigenvalues of $T$ and such that $T$ does not have a diagonal matrix with respect to any basis of $\mathbb{C}^{3}$.
\end{exercise}

\begin{proof}
    I define the operator $T$ on $\mathbb{C}^{3}$ as follows: $Te_{1} = e_{2}$, $Te_{2} = e_{3}$, $Te_{3} = 252e_{1} - 120e_{2} + 19e_{3}$.

    The matrix of $T$ with respect to the standard basis is
    \[
        \begin{pmatrix}
            0 & 0 & 252  \\
            1 & 0 & -120 \\
            0 & 1 & 19
        \end{pmatrix}
    \]

    which is also the companion matrix of the polynomial
    \[
        z^{3} - 19z^{2} + 120z - 252 = {(z - 6)}^{2}(z - 7).
    \]

    This is also the minimal polynomial of $T$. Therefore $T$ is not diagonalizable (because it has a double root).
\end{proof}
\newpage

% chapter5:sectionD:exercise12
\begin{exercise}
    Suppose $T\in\lmap{\mathbb{C}^{3}}$ is such that $6$ and $7$ are eigenvalues of $T$. Furthermore, suppose $T$ does not have a diagonal matrix with respect to any basis of $\mathbb{C}^{3}$. Prove that there exists $(z_{1}, z_{2}, z_{3}) \in \mathbb{C}^{3}$ such that
    \[
        T(z_{1}, z_{2}, z_{3}) = (6 + 8z_{1}, 7 + 8z_{2}, 13 + 8z_{3}).
    \]
\end{exercise}

\begin{proof}
    $T$ does not have a diagonal matrix with respect to any basis of $\mathbb{C}^{3}$, so $T$ does not have any eigenvalue other than $6$ and $7$ (otherwise, $T$ has three distinct eigenvalues, which makes $T$ diagonalizable). So $8$ is not an eigenvalue of $T$. Therefore $T - 8I$ is invertible. So there exists $(z_{1}, z_{2}, z_{3}) \in \mathbb{C}^{3}$ such that $(T - 8I)(z_{1}, z_{2}, z_{3}) = (6, 7, 13)$. Equivalently, there exists $(z_{1}, z_{2}, z_{3}) \in \mathbb{C}^{3}$ such that $T(z_{1}, z_{2}, z_{3}) = (6 + 8z_{1}, 7 + 8z_{2}, 13 + 8z_{3})$.
\end{proof}
\newpage

% chapter5:sectionD:exercise13
\begin{exercise}
    Suppose $A$ is a diagonal matrix with distinct entries on the diagonal and $B$ is a matrix of the same size as $B$. Show that $AB = BA$ if and only if $B$ is a diagonal matrix.
\end{exercise}

\begin{proof}
    Let $n$ be the number of rows of $A$, $B$. $AB = BA$ if and only if for all pairs $(j, k)$ such that $1\leq j, k\leq n$,
    \[
        \sum^{n}_{r=1}A_{j,r}B_{r,k} = \sum^{n}_{r=1}B_{j,r}A_{r,k}.
    \]

    Since $A$ is a diagonal matrix, then the statement is equivalent to
    \[
        A_{j,j}B_{j,k} = B_{j,k}A_{k,k}
    \]

    for all pairs $(j, k)$ such that $1\leq j, k\leq n$.

    Since the entries on the diagonal of $A$ are distinct, the statement is equivalent to $B_{j,k} = 0$ for all pairs $(j, k)$ such that $1\leq j, k\leq n$ and $j\ne k$. Equivalently, this means $B$ is a diagonal matrix.
\end{proof}
\newpage

% chapter5:sectionD:exercise14
\begin{exercise}
    \begin{enumerate}[label={(\alph*)}]
        \item Give an example of a finite-dimensional complex vector space and an
              operator $T$ on that vector space such that $T^{2}$ is diagonalizable but $T$ is not diagonalizable.
        \item Suppose $\mathbb{F} = \mathbb{C}$, $k$ is a positive integer, and $T\in\lmap{V}$ is invertible. Prove that $T$ is diagonalizable if and only if $T^{k}$ is diagonalizable.
    \end{enumerate}
\end{exercise}

\begin{proof}
    \begin{enumerate}[label={(\alph*)}]
        \item I define the linear operator $T$ on $\mathbb{C}^{2}$ as follows: $T(z_{1}, z_{2}) = (0, z_{1})$. Due to this definition, $T^{2} = 0$, so $T^{2}$ is diagonalizable. However, $T$ is not diagonalizable because the minimal polynomial of $T$ is $z^{2}$, which is not a product of different monic polynomials of degree $1$.
        \item If $T$ is diagonalizable, then there exists a basis $v_{1}, \ldots, v_{n}$ of $V$ such that the matrix of $T$ with respect to this basis is a diagonal matrix. Let $Tv_{i} = \lambda_{i}v_{i}$, where $\lambda_{i}$ is the $i$th entry on the diagonal of the matrix of $T$. Then $T^{k}v_{i} = \lambda_{i}^{k}v_{i}$. Therefore the matrix of $T^{k}$ with respect to the basis $v_{1}, \ldots, v_{n}$ is a diagonal matrix, and the $i$th entry on the diagonal of this matrix is $\lambda_{i}^{k}$.

              Assume that the statement is true for all positive integer less than $m$. If $T^{k}$ is diagonalizable, then the minimal polynomial of $T^{k}$ is
              \[
                  (z - \lambda_{1})\cdots (z - \lambda_{m})
              \]

              where $\lambda_{1}, \ldots, \lambda_{m}$ are distinct nonzero complex numbers. Each nonzero complex number $\lambda_{i}$ has $k$ $k$-roots, let them be $\lambda_{i,1}, \ldots, \lambda_{i,k}$. A $k$-root of $\lambda_{i}$ is not equal to a $k$-root of $\lambda_{j}$ for all $i\ne j$ (this is proved by contradiction). Hence
              \begin{align*}
                  (z^{k} - \lambda_{1})\cdots (z^{k} - \lambda_{m}) & = \prod^{m}_{i=1}(z^{k} - \lambda_{i})                           \\
                                                                    & = \prod^{m}_{i=1}\left(\prod^{k}_{j=1}(z - \lambda_{i,j})\right)
              \end{align*}

              Hence
              \[
                  \prod^{m}_{i=1}\left(\prod^{k}_{j=1}(T - \lambda_{i,j}I)\right) = 0.
              \]

              So this polynomial is a product of distinct monic polynomial of degree $1$ and is a multiple polynomial of the minimal polynomial of $T$. Therefore the minimal polynomial of $T$ is also a product of distinct monic polynomial of degree $1$. Therefore $T$ is diagonalizable.
    \end{enumerate}
\end{proof}
\newpage

% chapter5:sectionD:exercise15
\begin{exercise}
    Suppose $V$ is a finite-dimensional complex vector space, $T\in\lmap{V}$, and $p$ is the minimal polynomial of $T$. Prove that the following are equivalent.
    \begin{enumerate}[label={(\alph*)}]
        \item $T$ is diagonalizable.
        \item There does not exist $\lambda\in\mathbb{C}$ such that $p$ is a polynomial multiple of ${(z - \lambda)}^{2}$.
        \item $p$ and its derivative $p'$ have no zeros in common.
        \item The greatest common divisor of $p$ and $p'$ is the constant polynomial $1$.
    \end{enumerate}
\end{exercise}

\begin{proof}
    $T$ is diagonalizable if and only if $p$ is the product of distinct monic polynomial of degree $1$. Together with the second version of the fundamental theorem of algebra, it follows that (a) and (b) are equivalent.

    (b) and (c) are equivalent due to Exercise~\ref{chapter4:sectionA:exercise8}.

    Now I will prove that (c) and (d) are equivalent.

    If (c) is true, assume that the greatest common disisor of $p$ and $p'$ is a nonconstant polynomial, then $p$ and $p'$ have at least one zero in common. So the assumption is false. Therefore (d) is true. So (c) implies (d).

    If (d) is true, then there exist polynomials $s$ and $r$ such that
    \[
        sp + rp' = 1.
    \]

    From $sp + rp' = 1$, we deduce that every zero of $p$ is not a zero of $p'$ and vice versa, so $p$ and $p'$ have no zeros in common. Hence (d) implies (c).

    Thus (c) and (d) are equivalent.
\end{proof}
\newpage

% chapter5:sectionD:exercise16
\begin{exercise}
    Suppose that $T\in\lmap{V}$ is diagonalizable. Let $\lambda_{1}, \ldots, \lambda_{m}$ denote the distinct eigenvalues of $T$. Prove that a subspace $U$ of $V$ is invariant under $T$ if and only if there exist subspaces $U_{1}, \ldots, U_{m}$ of $V$ such that $U_{k}\subseteq E(\lambda_{k}, T)$ for each $k$ and $U = U_{1}\oplus \cdots \oplus U_{m}$.
\end{exercise}

\begin{proof}
    If there exist subspaces $U_{1}, \ldots, U_{m}$ of $V$ such that $U_{k}\subseteq E(\lambda_{k}, T)$ for each $k$ and $U = U_{1}\oplus \cdots \oplus U_{m}$, then for every $u\in U$, there exist $v_{1}\in U_{1}, \ldots, v_{m}\in U_{m}$ such that
    \[
        u = v_{1} + \cdots + v_{m}
    \]

    and $Tu = Tv_{1} + \cdots + Tv_{m} = \lambda_{1}v_{1} + \cdots + \lambda_{m}v_{m}\in U_{1}\oplus \cdots \oplus U_{m} = U$. Therefore $U$ is invariant under $T$.

    \bigskip

    If $U$ is invariant under $T$. Since $T$ is diagonalizable, then so is $T\vert_{U}$. Let $\alpha_{1}, \ldots, \alpha_{p}$ be the distinct eigenvalues of $T\vert_{U}$, then $\{ \alpha_{1}, \ldots, \alpha_{p} \}\subseteq \{ \lambda_{1}, \ldots, \lambda_{n} \}$.

    Let $U_{k} = E(\lambda_{k}, T\vert_{U})$ for each $k\in\{ 1, \ldots, n \}$, then
    \begin{align*}
        U & = E(\alpha_{1}, T\vert_{U})\oplus\cdots\oplus E(\alpha_{p}, T\vert_{U})   \\
          & = E(\lambda_{1}, T\vert_{U})\oplus\cdots\oplus E(\lambda_{m}, T\vert_{U}) \\
          & = U_{1}\oplus\cdots \oplus U_{m}.
    \end{align*}
\end{proof}
\newpage

% chapter5:sectionD:exercise17
\begin{exercise}
    Suppose $V$ is finite-dimensional. Prove that $\lmap{V}$ has a basis consisting of diagonalizable operators.
\end{exercise}

\begin{proof}
    Let $v_{1}, \ldots, v_{n}$ be a basis of $V$.

    Let $E_{i,j}$ be the linear map such that
    \[
        E_{i,j}v_{k} = \begin{cases}
            v_{i} & \text{if $k = j$} \\
            0     & \text{otherwise}
        \end{cases}
    \]

    Then the list $E_{i,j}$ ($1\leq i, j\leq n$) is a basis of $\lmap{V}$. Replace this list by $E_{1,1}, \ldots, E_{n,n}$, $E_{i,i} + E_{i,j}$ ($1\leq i, j\leq n$, $i\ne j$), then this list is still a basis of $\lmap{V}$.

    If $i\ne j$,
    \begin{align*}
        (E_{i,i} + E_{i,j})(E_{i,i} + E_{i,j}) & = E_{i,i}E_{i,i} + E_{i,j}E_{i,i} + E_{i,i}E_{i,j} + E_{i,j}E_{i,j} \\
                                               & = E_{i,i} + 0 + E_{i,j} + 0                                         \\
                                               & = E_{i,i} + E_{i,j}
    \end{align*}

    So $z^{2} - z = z(z - 1)$ is the minimal polynomial of $E_{i,i} + E_{i,j}$. Therefore $E_{i,i} + E_{i,j}$ is diagonalizable for all $i\ne j$.

    Thus the list $E_{1,1}, \ldots, E_{n,n}$, $E_{i,i} + E_{i,j}$ ($1\leq i, j\leq n$, $i\ne j$) consisting of diagonalizable operators and is a basis of $\lmap{V}$.
\end{proof}
\newpage

% chapter5:sectionD:exercise18
\begin{exercise}
    Suppose that $T \in \lmap{V}$ is diagonalizable and $U$ is a subspace of $V$ that is invariant under $T$. Prove that the quotient operator $T/U$ is a diagonalizable operator on $V/U$.
\end{exercise}

\begin{proof}
    $T$ is diagonalizable so the minimal polynomial of $T$ is a product of distinct monic polynomial of degree $1$. On the other hand, the minimal polynomial of $T$ is a polynomial multiple of the minimal polynomial of $T/U$. Therefore the minimal polynomial of $T/U$ is also a product of distinct monic polynomial of degree $1$. Thus $T/U$ is a diagonalizable operator on $V/U$.
\end{proof}
\newpage

% chapter5:sectionD:exercise19
\begin{exercise}
    Prove or give a counterexample: If $T \in \lmap{V}$ and there exists a subspace $U$ of $V$ that is invariant under $T$ such that $T\vert_{U}$ and $T/U$ are both diagonalizable, then $T$ is diagonalizable.
\end{exercise}

\begin{proof}
    I give a counterexample.

    On $V = \mathbb{F}^{2}$, I define the linear operator $T$ as follows: $T(x, y) = (y, 0)$. The minimal polynomial of $T$ is $z^{2} = 0$, so $T$ is not diagonalizable.

    $\kernel{T} = \operatorname{span}((1, 0))$. Let $U = \kernel{T}$ then $U$ is invariant under $T$. $T\vert_{U} = 0$ so $T\vert_{U}$ is diagonalizable.

    $V/U = \operatorname{span}((0, 1) + U)$, so
    \[
        T/U((0, 1) + U) = T(0, 1) + U = (1, 0) + U = 0 + U.
    \]

    Therefore $T/U = 0$, which means $T/U$ is diagonalizable.
\end{proof}
\newpage

% chapter5:sectionD:exercise20
\begin{exercise}
    Suppose $V$ is finite-dimensional and $T \in \lmap{V}$. Prove that $T$ is diagonalizable if and only if the dual operator $T'$ is diagonalizable.
\end{exercise}

\begin{proof}
    A linear operator is diagonalizable if and only if its minimal polynomial is a product of distinct monic polynomial of degree $1$. By Exercise~\ref{chapter5:sectionB:exercise28}, the minimal polynomials of $T$ and its dual map $T'$ are equal. Thus $T$ is diagonalizable if and only if the dual operator $T'$ is diagonalizable.
\end{proof}
\newpage

% chapter5:sectionD:exercise21
\begin{exercise}
    The \textit{Fibonacci sequence} $F_{0}, F_{1}, F_{2}, \ldots$ is defined by
    \[
        F_{0} = 0, F_{1} = 1, \text{ and } F_{n} = F_{n-2} + F_{n-1} \text{ for $n\geq 2$}.
    \]

    Define $T\in\lmap{\mathbb{R}^{2}}$ by $T(x, y) = (y, x + y)$.

    \begin{enumerate}[label={(\alph*)}]
        \item Show that $T^{n}(0, 1) = (F_{n}, F_{n+1})$ for each nonnegative integer $n$.
        \item Find the eigenvalues of $T$.
        \item Find a basis of $\mathbb{R}^{2}$ consisting of eigenvectors of $T$.
        \item Use the solution to (c) to compute $T^{n}(0, 1)$. Conclude that
              \[
                  F_{n} = \frac{1}{\sqrt{5}}\left[{\left(\frac{1 + \sqrt{5}}{2}\right)}^{n} - {\left(\frac{1 - \sqrt{5}}{2}\right)}^{n}\right]
              \]

              for each nonnegative integer $n$.
        \item Use (d) to conclude that if $n$ is a nonnegative integer, then the Fibonacci number $F_{n}$ is the integer that is closest to
              \[
                  \frac{1}{\sqrt{5}}{\left(\frac{1 + \sqrt{5}}{2}\right)}^{n}.
              \]
    \end{enumerate}
\end{exercise}

\begin{proof}
    \begin{enumerate}[label={(\alph*)}]
        \item I give a proof using mathematical induction.

              $T^{0}(0, 1) = (0, 1) = (F_{0}, F_{1})$.

              Assume $T^{n}(0, 1) = (F_{n}, F_{n+1})$, then according to the induction hypothesis
              \[
                  T^{n+1}(0, 1) = T(F_{n}, F_{n+1}) = (F_{n+1}, F_{n} + F_{n+1}) = (F_{n+1}, F_{n+2}).
              \]

              Thus, due to the principle of mathematical induction $T^{n}(0, 1) = (F_{n}, F_{n+1})$ for each nonnegative integer $n$.
        \item Let $\lambda$ be an eigenvalue of $T$ and $(x, y)$ be a corresponding eigenvector.
              \[
                  T(x, y) = (y, x + y) = (\lambda x, \lambda y).
              \]

              Therefore $\lambda y^{2} = \lambda x(x + y)$. If $\lambda = 0$ then $x = y = 0$, which contradicts the definition of eigenvector. So $\lambda\ne 0$ and it follows that $x(x + y) = y^{2}$. $x = 0$ if and only if $y = 0$, therefore $x, y\ne 0$. From $x(x + y) = y^{2}$, we deduce that $\frac{y}{x}$ is a root of the quadratic equation $t^{2} - t - 1 = 0$. The two roots of this equation are $\frac{1 + \sqrt{5}}{2}$ and $\frac{1 - \sqrt{5}}{2}$.

              Hence the eigenvalues of $T$ are $\frac{1 + \sqrt{5}}{2}$ and $\frac{1 - \sqrt{5}}{2}$.
        \item A basis of $\mathbb{R}^{2}$ consisting of eigenvectors of $T$ is
              \[
                  \left( 1, \frac{1+\sqrt{5}}{2} \right), \left( 1, \frac{1-\sqrt{5}}{2} \right).
              \]
        \item \begin{align*}
                  T^{n}(0, 1) & = \frac{1}{\sqrt{5}}T^{n}\left(1, \frac{1+\sqrt{5}}{2}\right) - \frac{1}{\sqrt{5}}T^{n}\left(1, \frac{1-\sqrt{5}}{2}\right)                                                                                                                                   \\
                              & = \frac{1}{\sqrt{5}}{\left(\frac{1+\sqrt{5}}{2}\right)}^{n}\left(1, \frac{1+\sqrt{5}}{2}\right) - \frac{1}{\sqrt{5}}{\left(\frac{1-\sqrt{5}}{2}\right)}^{n}\left(1, \frac{1-\sqrt{5}}{2}\right)                                                               \\
                              & = \left(\frac{1}{\sqrt{5}}\left[{\left(\frac{1 + \sqrt{5}}{2}\right)}^{n} - {\left(\frac{1 - \sqrt{5}}{2}\right)}^{n}\right], \frac{1}{\sqrt{5}}\left[{\left(\frac{1 + \sqrt{5}}{2}\right)}^{n+1} - {\left(\frac{1 - \sqrt{5}}{2}\right)}^{n+1}\right]\right)
              \end{align*}

              Hence
              \[
                  F_{n} = \frac{1}{\sqrt{5}}\left[{\left(\frac{1 + \sqrt{5}}{2}\right)}^{n} - {\left(\frac{1 - \sqrt{5}}{2}\right)}^{n}\right]
              \]

              for each nonnegative integer $n$.
        \item \[
                  \abs{F_{n} - \frac{1}{\sqrt{5}}{\left(\frac{1 + \sqrt{5}}{2}\right)}^{n}} = \abs{\frac{1}{\sqrt{5}}{\left(\frac{1 - \sqrt{5}}{2}\right)}^{n}}\leq \frac{1}{\sqrt{5}} < \frac{1}{2}.
              \]

              So $F_{n}$ is the integer that is closest to
              \[
                  \frac{1}{\sqrt{5}}{\left(\frac{1 + \sqrt{5}}{2}\right)}^{n}.
              \]
    \end{enumerate}
\end{proof}
\newpage

% chapter5:sectionD:exercise22
\begin{exercise}
    Suppose $T\in\lmap{V}$ and $A$ is an $n$-by-$n$ matrix that is the matrix of $T$ with respect to some basis of $V$. Prove that if
    \[
        \abs{A_{j,j}} > \sum^{n}_{\substack{k=1\\ k\ne j}}\abs{A_{j,k}}
    \]

    for each $j\in \{ 1, \ldots, n \}$, then $T$ is invertible.
\end{exercise}

\begin{proof}
    Assume that $T$ is not invertible, then there exists a nonzero vector $v$ such that $Tv = 0$. So $0$ is an eigenvalue of $T$.

    According to the Gershgorin disk theorem, there exists $j$ in $\{ 1, \ldots, n \}$ such that
    \[
        \abs{0 - A_{j,j}} < \sum^{n}_{\substack{k=1 \\ k\ne j}}\abs{A_{j,k}}.
    \]

    But this is a contradiction, since
    \[
        \abs{A_{j,j}} > \sum^{n}_{\substack{k=1\\ k\ne j}}\abs{A_{j,k}}
    \]

    for each $j\in \{ 1, \ldots, n \}$. Therefore the assumption is false, and thus $T$ is invertible.
\end{proof}
\newpage

% chapter5:sectionD:exercise23
\begin{exercise}
    Suppose the definition of the Gershgorin disks is changed so that the radius of the $k$th disk is the sum of the absolute values of the entries in column (instead of row) $k$ of $A$, excluding the diagonal entry. Show that the Gershgorin disk theorem (5.67) still holds with this changed definition.
\end{exercise}

\begin{proof}
    Assume that the linear operator $T$ has matrix $A$ with respect to a basis $v_{1}, \ldots, v_{n}$ of $V$, then the matrix of the dual map $T'$ of $T$ has matrix $B = A^{\top}$ with respect to the dual basis of $v_{1}, \ldots, v_{n}$.

    Let $\lambda$ be an eigenvalue of $T$, then $\lambda$ is also an eigenvalue of $T'$. According to the Gershgorin disk theorem, there exists $j$ in $\{ 1, \ldots, n \}$ such that
    \[
        \lambda \in \left\{ z : \abs{z - B_{j,j}} < \sum^{n}_{\substack{k=1\\k\ne j}} \abs{B_{j,k}} \right\}.
    \]

    Equivalently
    \[
        \lambda\in \left\{ z : \abs{z - A_{j,j}} < \sum^{n}_{\substack{k=1\\k\ne j}} \abs{A_{k,j}} \right\}.
    \]

    Thus the Gershgorin disk theorem where the radius of the $k$th disk is the sum of the absolute values of the entries in column $k$ of $A$, excluding the diagonal entry still holds.
\end{proof}
\newpage

\section{Commuting Operators}

% chapter5:sectionE:exercise1
\begin{exercise}
    Give an example of two commuting operators $S, T$ on $\mathbb{F}^{4}$ such that there is a subspace of $\mathbb{F}^{4}$ that is invariant under $S$ but not under $T$ and there is a subspace of $\mathbb{F}^{4}$ that is invariant under $T$ but not under $S$.
\end{exercise}

\begin{proof}
    I define $S$ and $T$ as follows:
    \begin{align*}
        S(x_{1}, x_{2}, x_{3}, x_{4}) & = (x_{1}, x_{1} + x_{2}, x_{1} + x_{2} + x_{3}, x_{1} + x_{2} + x_{3} + x_{4}), \\
        T(x_{1}, x_{2}, x_{3}, x_{4}) & = (x_{1}, x_{2} - x_{1}, x_{3} - x_{2}, x_{4} - x_{3}).
    \end{align*}

    Hence $ST = TS = I$. Let
    \[
        U = \{ (x_{1}, x_{2}, 0, 0): x_{1}, x_{2}\in\mathbb{F} \}\qquad V = \{ (0, 0, x_{3}, x_{4}): x_{3}, x_{4}\in\mathbb{F} \}.
    \]

    $U$ is invariant under $S$ but is not invariant under $T$. $V$ is invariant under $T$ but is not invariant under $S$.
\end{proof}
\newpage

% chapter5:sectionE:exercise2
\begin{exercise}
    Suppose $\mathcal{E}$ is a subset of $\lmap{V}$ and every element of $\mathcal{E}$ is diagonalizable. Prove that there exists a basis of $V$ with respect to which every element of $\mathcal{E}$ has a diagonal matrix if and only if every pair of elements of $\mathcal{E}$ commutes.
\end{exercise}

\begin{proof}
    Because every element of $\mathcal{E}$ is diagonalizable, then each of them has an eigenvalue.

    If there exists a basis of $V$ with respect to which every element of $\mathcal{E}$ has a diagonal matrix, then every pair of elements of $\mathcal{E}$ commutes because any two diagonal matrices commute.

    To prove the implication of the other direction, I give a proof using mathematical induction on $\dim V$.

    If $\dim V = 1$, every pair of elements of $\mathcal{E}$ commute, and all elements of $\mathcal{E}$ are simultaneously diagonalizable.

    Let $n$ be a positive integer greater than $1$. Assume that the statement is true for every vector space of dimension less than $n$. Let $\dim V = n$.

    If each element of $\mathcal{E}$ has precisely one eigenvalue, then each of them is a multiple of a scalar and the identity operator. So every pair of elements of $\mathcal{E}$ commute, and all elements of $\mathcal{E}$ are simultaneously diagonalizable with respect to any basis of $V$.

    Otherwise, there exists an operator $T\in\mathcal{E}$ such that $T$ have at least two eigenvalues. Let $\lambda_{1}, \ldots, \lambda_{m}$ be the distinct eigenvalues of $T$, then $m\geq 2$. Because $T$ is diagonalizable,
    \[
        V = E(\lambda_{1}, T)\oplus \cdots \oplus E(\lambda_{m}, T).
    \]

    By the induction hypothesis, for each $i\in \{ 1, \ldots, m \}$, for all $S\in\mathcal{E}$, there exists a basis of $E(\lambda_{i}, T)$ consisting of eigenvectors of $S\vert_{E(\lambda_{i}, T)}$. Putting all these bases together gives a basis of $V$ with respect to which every element of $\mathcal{E}$ has a diagonal matrix.

    So due to the principle of mathematical induction, if every pair of elements of $\mathcal{E}$ commutes, then there exists a basis of $V$ with respect to which every element of $\mathcal{E}$ has a diagonal matrix.
\end{proof}
\newpage

% chapter5:sectionE:exercise3
\begin{exercise}\label{chapter5:sectionE:exercise3}
    Suppose $S, T\in\lmap{V}$ are such that $ST = TS$. Suppose $p\in \mathscr{P}(\mathbb{F})$.
    \begin{enumerate}[label={(\alph*)}]
        \item Prove that $\kernel{p(S)}$ is invariant under $T$.
        \item Prove that $\range{p(S)}$ is invariant under $T$.
    \end{enumerate}
\end{exercise}

\begin{proof}
    \begin{enumerate}[label={(\alph*)}]
        \item Let $v\in \kernel{p(S)}$. Then $p(S)v = 0$. Because $S$ and $T$ commute, it follows that $p(S)$ and $T$ commute.
              \[
                  p(S)(Tv) = T(p(S)v) = 0.
              \]

              So $Tv\in\kernel{p(S)}$. Thus $\kernel{p(S)}$ is invariant under $T$.
        \item Let $w\in \range{p(S)}$. Then there exists $v\in V$ such that $p(S)v = w$. Because $S$ and $T$ commute, it follows that $p(S)$ and $T$ commute.
              \[
                  Tw = T(p(S)v) = p(S)(Tv).
              \]

              Hence $Tw\in \range{p(S)}$. Thus $\range{p(S)}$ is invariant under $T$.
    \end{enumerate}
\end{proof}
\newpage

% chapter5:sectionE:exercise4
\begin{exercise}
    Prove or give a counterexample: If $A$ is a diagonal matrix and $B$ is an upper-triangular matrix of the same size as $A$, then $A$ and $B$ commute.
\end{exercise}

\begin{proof}
    I give a counterexample.
    \[
        A = \begin{pmatrix}
            1 & 0 \\
            0 & 0
        \end{pmatrix}\qquad
        B = \begin{pmatrix}
            1 & 1 \\
            0 & 0
        \end{pmatrix}
    \]

    Then
    \[
        AB = \begin{pmatrix}
            1 & 1 \\
            0 & 0
        \end{pmatrix}\qquad
        BA = \begin{pmatrix}
            1 & 0 \\
            0 & 0
        \end{pmatrix}.
    \]
\end{proof}
\newpage

% chapter5:sectionE:exercise5
\begin{exercise}
    Prove that a pair of operators on a finite-dimensional vector space commute if and only if their dual operators commute.
\end{exercise}

\begin{proof}
    Let $S$ and $T$ be two linear operators on a finite-dimensional vector space $V$. Let $v_{1}, \ldots, v_{n}$ be a basis of $V$ and $\varphi_{1}, \ldots, \varphi_{n}$ is the dual basis.

    Let
    \[
        A = \mathcal{M}(S, (v_{1}, \ldots, v_{n}))\qquad
        B = \mathcal{M}(T, (v_{1}, \ldots, v_{n}))
    \]

    then
    \[
        A^{\top} = \mathcal{M}(S', (\varphi_{1}, \ldots, \varphi_{n}))\qquad
        B^{\top} = \mathcal{M}(T', (\varphi_{1}, \ldots, \varphi_{n}))
    \]

    $S$ and $T$ commute if and only if $A$ and $B$ commute. $S'$ and $T'$ commute if and only if $A^{\top}$ and $B^{\top}$ commute. $A$ and $B$ commute if and only if $A^{\top}$ and $B^{\top}$ commute, which is deduced from
    \[
        {(AB)}^{\top} = B^{\top}A^{\top}\qquad {(BA)}^{\top} = A^{\top}B^{\top}.
    \]

    Thus $S$ and $T$ commute if and only if $S'$ and $T'$ commute.
\end{proof}
\newpage

% chapter5:sectionE:exercise6
\begin{exercise}
    Suppose $V$ is a finite-dimensional complex vector space and $S, T\in\lmap{V}$ commute. Prove that there exist $\alpha, \lambda\in\mathbb{C}$ such that
    \[
        \range{(S - \alpha I)} + \range{(T - \lambda I)} \ne V.
    \]
\end{exercise}

\begin{proof}
    Unsolved.
\end{proof}
\newpage

% chapter5:sectionE:exercise7
\begin{exercise}
    Suppose $V$ is a complex vector space, $S \in \lmap{V}$ is diagonalizable, and $T\in\lmap{V}$ commutes with $S$. Prove that there is a basis of $V$ such that $S$ has a diagonal matrix with respect to this basis and $T$ has an upper-triangular matrix with respect to this basis.
\end{exercise}

\begin{proof}
    Let $\lambda_{1}, \ldots, \lambda_{m}$ be the distinct eigenvalues of $S$. Because $S$ is diagonalizable, it follows that
    \[
        V = E(\lambda_{1}, S) \oplus \cdots \oplus E(\lambda_{m}, S).
    \]

    For each $i\in\{1, \ldots, m\}$, $S\vert_{E(\lambda_{i}, S)}$ and $T\vert_{E(\lambda_{i}, S)}$ commute, so there exists a basis of $E(\lambda_{i}, S)$ with respect to which $S$ and $T$ have upper-triangular matrix. Putting all these bases gives a basis of $V$ with respect to which $S$ and $T$ have upper-triangular matrix. Moreover, these vectors are eigenvectors of $S$, so the matrix of $S$ with respect to this basis is not just upper-triangular matrix but also diagonal.

    Thus there exists a basis of $V$ such that $S$ has a diagonal matrix and $T$ has an upper-triangular matrix with respect to this basis.
\end{proof}
\newpage

% chapter5:sectionE:exercise8
\begin{exercise}
    Suppose $m = 3$ in Example 5.72 and $D_{x}$, $D_{y}$ are the commuting partial
    differentiation operators on $\mathscr{P}_{3}(\mathbb{R}^{2})$ from that example. Find a basis of $\mathscr{P}_{3}(\mathbb{R}^{2})$ with respect to which $D_{x}$ and $D_{y}$ each have an upper-triangular matrix.
\end{exercise}

\begin{proof}
    I choose the following basis
    \[
        1, x, y, x^{2}, y^{2}, xy, x^{2}y, xy^{2}, x^{3}, y^{3}.
    \]

    \[
        \mathcal{M}(D_{x}) = \begin{pmatrix}
            0 & 0 & 0 & 0 & 0 & 0 & 0 & 0 & 0 & 0 \\
            0 & 1 & 0 & 2 & 0 & 0 & 0 & 0 & 0 & 0 \\
            0 & 0 & 0 & 0 & 0 & 1 & 0 & 0 & 0 & 0 \\
            0 & 0 & 0 & 0 & 0 & 0 & 0 & 0 & 3 & 0 \\
            0 & 0 & 0 & 0 & 0 & 0 & 0 & 1 & 0 & 0 \\
            0 & 0 & 0 & 0 & 0 & 0 & 2 & 0 & 0 & 0 \\
            0 & 0 & 0 & 0 & 0 & 0 & 0 & 0 & 0 & 0 \\
            0 & 0 & 0 & 0 & 0 & 0 & 0 & 0 & 0 & 0 \\
            0 & 0 & 0 & 0 & 0 & 0 & 0 & 0 & 0 & 0 \\
            0 & 0 & 0 & 0 & 0 & 0 & 0 & 0 & 0 & 0
        \end{pmatrix}
    \]

    \[
        \mathcal{M}(D_{y}) = \begin{pmatrix}
            0 & 0 & 0 & 0 & 0 & 0 & 0 & 0 & 0 & 0 \\
            0 & 0 & 1 & 0 & 0 & 1 & 0 & 0 & 0 & 0 \\
            0 & 0 & 0 & 0 & 2 & 0 & 0 & 0 & 0 & 0 \\
            0 & 0 & 0 & 0 & 0 & 0 & 1 & 0 & 0 & 0 \\
            0 & 0 & 0 & 0 & 0 & 0 & 0 & 0 & 0 & 3 \\
            0 & 0 & 0 & 0 & 0 & 0 & 0 & 2 & 0 & 0 \\
            0 & 0 & 0 & 0 & 0 & 0 & 0 & 0 & 0 & 0 \\
            0 & 0 & 0 & 0 & 0 & 0 & 0 & 0 & 0 & 0 \\
            0 & 0 & 0 & 0 & 0 & 0 & 0 & 0 & 0 & 0 \\
            0 & 0 & 0 & 0 & 0 & 0 & 0 & 0 & 0 & 0
        \end{pmatrix}
    \]

    The matrices of $D_{x}$ and $D_{y}$ with respect to this basis are upper triangular.
\end{proof}
\newpage

% chapter5:sectionE:exercise9
\begin{exercise}\label{chapter5:sectionE:exercise9}
    Suppose $V$ is a finite-dimensional nonzero complex vector space. Suppose that $E \subseteq \lmap{V}$ is such that $S$ and $T$ commute for all $S, T\in \mathcal{E}$.
    \begin{enumerate}[label={(\alph*)}]
        \item Prove that there is a vector in $V$ that is an eigenvector for every element of $\mathcal{E}$.
        \item Prove that there is basis of $V$ with respect to which every element of $\mathcal{E}$ has an upper-triangular matrix.
    \end{enumerate}
\end{exercise}

\begin{proof}
    Unsolved.
\end{proof}
\newpage

% chapter5:sectionE:exercise10
\begin{exercise}
    Give an example of two commuting operators $S$, $T$ on a finite-dimensional real vector space such that $S + T$ has a eigenvalue that does not equal an eigenvalue of $S$ plus an eigenvalue of $T$ and $ST$ has a eigenvalue that does not equal an eigenvalue of $S$ times an eigenvalue of $T$.
\end{exercise}

\begin{proof}
    I define $S, T$ on $\lmap{\mathbb{R}^{2}}$ as follows:
    \[
        S(x, y) = (y, -x)\qquad T(x, y) = (-y, x)
    \]

    Then
    \begin{align*}
        (ST)(x, y) & = S(-y, x) = (x, y) \\
        (TS)(x, y) & = T(y, -x) = (x, y)
    \end{align*}

    so $ST = TS = I$. $S$ and $T$ have no (real) eigenvalue. On the other hand, $S + T = 0$ and $ST = I$. $0$ is the only eigenvalue of $S + T$, $1$ is the only eigenvalue of $ST$.
\end{proof}
\newpage

\chapter{Identification Topology; Weak Topology}

\section{Identification Topology}

\begin{problem}{VI.1.1}\label{problem:VI.1.1}
Reversing the situation treated in the text, let \(X\) be a set, \( (Y, \mathscr{T}) \) a space, and \( p: X \to Y \) a surjective map. Prove:
\begin{enumerate}[label={(\alph*)}]
	\item \( \mathscr{T}_{X} = \left\{ p^{-1}(U) \mid U \text{ open in } Y \right\} \) is a topology in \( X \).
	\item \( p: (X, \mathscr{T}_{X}) \to (Y, \mathscr{T}) \) is continuous, open, and closed.
\end{enumerate}
\end{problem}

\begin{proof}
	\begin{enumerate}[label={(\alph*)}]
		\item \( \mathscr{T}_{X} \) contains \( \varnothing, X \) as \( p^{-1}(\varnothing) = \varnothing \) and \( p^{-1}(Y) = X \).

		      If \( {\left\{ U_{\alpha} \right\}}_{\alpha\in\mathscr{A}} \) is a collection of open sets in \( Y \), then
		      \[
			      \bigcup_{\alpha\in\mathscr{A}} p^{-1}(U_{\alpha}) = p^{-1}\left(\bigcup_{\alpha\in\mathscr{A}} U_{\alpha}\right)
		      \]

		      so \( \mathscr{T}_{X} \) is closed under arbitrary union.

		      If \( U_{1}, \ldots, U_{n} \) are open sets in \( Y \) then
		      \[
			      \bigcap^{n}_{i=1} p^{-1}(U_{i}) = p^{-1}\left(\bigcap^{n}_{i=1} U_{i}\right)
		      \]

		      so \( \mathscr{T}_{X} \) is closed under finite intersection.

		      Hence \( \mathscr{T}_{X} \) is a topology in \( X \).
		\item For each open set \( U \) in \( Y \), \( p^{-1}(U) \in \mathscr{T}_{X} \) so \( p \) is continuous.

		      Let \( V \) be an open set in \( X \). Then there is an open set \( U \) in \( Y \) such that \( V = p^{-1}(U) \). Hence \( p(V) = pp^{-1}(U) = U \) because \( p \) is surjective. So \( p \) is an open map.

		      Let \( W \) be a closed set in \( X \) then \( X - W \) is open and there exists an open set \( U \) in \( Y \) such that \( X - W = p^{-1}(U) \). Therefore
		      \[
			      W = X - p^{-1}(U) = p^{-1}(Y) - p^{-1}(U) = p^{-1}(Y - U)
		      \]

		      which implies that \( p(W) = pp^{-1}(Y - U) = Y - U \), which is closed in \( Y \). So \( p \) is a closed map.

		      Thus \( p \) is a continuous, open, and closed map.
	\end{enumerate}
\end{proof}

\begin{problem}{VI.1.2}\label{problem:VI.1.2}
For each \( \alpha \in \mathscr{A} \), let \( p_{\alpha}: X_{\alpha} \to Y_{\alpha} \) be a continuous, open surjection. Show that \( \prod_{\alpha} p_{\alpha}: \prod_{\alpha} X_{\alpha} \to \prod_{\alpha} Y_{\alpha} \) is an identification.
\end{problem}

\begin{proof}
	For the sake of brevity, denote \( p = \prod_{\alpha} p_{\alpha} \). By definition, \( p \) is surjective.

	\( p_{Y_{\alpha}} \circ p \) is continuous for each projection \( p_{Y_{\alpha}}: \prod_{\alpha} Y_{\alpha} \to Y_{\alpha} \) so \( p \) is continuous.

	Let \( \prod_{\alpha} U_{\alpha} \) be a basic open set in \( \prod_{\alpha} X_{\alpha} \), which means \( U_{\alpha} = X_{\alpha} \) for all but finitely many \( \alpha \) and \( U_{\alpha} \) is open in \( X_{\alpha} \) for every \( \alpha \). Because \( p_{\alpha} \) is an open surjection for each \( \alpha \), the image
	\[
		p\left( \prod_{\alpha} U_{\alpha} \right) = \prod_{\alpha} p_{\alpha}(U_{\alpha})
	\]

	is open in \( \prod_{\alpha} Y_{\alpha} \) as \( p_{\alpha}(U_{\alpha}) \) is open in \( Y_{\alpha} \) and \( p_{\alpha}(U_{\alpha}) = Y_{\alpha} \) for all but finitely many \( \alpha \). Hence \( p \) is an open map.

	\( p \) is a continuous, open surjection so \( p \) is an identification.
\end{proof}

\begin{problem}{VI.1.3}
Let \( X \) be a space and \( A \subset X \) a subspace. Assume that there exists a continuous \( r: X \to A \) such that \( r\vert_{A} = 1_{A} \) (such a map is called a \textit{retraction} of \(X\) onto \(A\)). Show that \( r \) is an identification.
\end{problem}

\begin{proof}
	By definition, \( r \) is continuous and surjective. Let \( f: A \xhookrightarrow{} X \) be the inclusion map.

	\( f \) is continuous and \( r \circ f = 1_{A} \) so \( r \) is an identification.
\end{proof}

\begin{problem}{VI.1.4}\label{problem:VI.1.4}
Let \( X \) be any set. Given any family \( \left\{ (Y_{\alpha}, \mathscr{T}_{\alpha}), f_{\alpha} \mid \alpha \in \mathscr{A} \right\} \) of spaces and maps \( f_{\alpha}: X \to Y_{\alpha} \), the ``projective limit topology of \(X\) determined by this family'' is \( \bigvee_{\alpha} f_{\alpha}^{-1}(\mathscr{T}_{\alpha}) \) (see Problem~\ref{problem:III.3.8}). Prove:
\begin{enumerate}[label={(\alph*)}]
	\item If \( j: X \to \prod_{\alpha} Y_{\alpha} \) is the map \( j(x) = \left\{ f_{\alpha}(x) \right\} \), then \( \bigvee_{\alpha} f_{\alpha}^{-1}(\mathscr{T}_{\alpha}) \) is the topology in \(X\) determined by \(j\) as in Problem~\ref{problem:VI.1.1}.
	\item If whenever \( x \ne x^{\prime} \), there is some index \( \alpha \) such that \( f_{\alpha}(x) \ne f_{\alpha}(x^{\prime}) \), then \( j \) is an embedding.
\end{enumerate}
\end{problem}

\begin{proof}
	\begin{enumerate}[label={(\alph*)}]
		\item Let \( \prod_{\alpha} U_{\alpha} \) be a subbasic open set in \( \prod_{\alpha} Y_{\alpha} \) then \( U_{\alpha} = Y_{\alpha} \) for every \( \alpha \) but one \( \beta \in \mathscr{A} \).
		      \[
			      j^{-1}\left( \prod_{\alpha} U_{\alpha} \right) = \bigcap_{\alpha} f_{\alpha}^{-1}(U_{\alpha}) = f_{\beta}^{-1}(U_{\beta}) \in \bigvee_{\alpha} f_{\alpha}^{-1}(\mathscr{T}_{\alpha})
		      \]

		      Hence \( j \) is continuous, which means if \( j^{-1}(U) \) is open whenever \( U \subset \prod_{\alpha} Y_{\alpha} \) is open.

		      Let \( V \) be an open set in \( X \). According to the definition of the topology \( \bigvee_{\alpha} f_{\alpha}^{-1}(\mathscr{T}_{\alpha}) \), \( V \) can be written as a union of finite intersection of elements in \( \bigcup_{\alpha} f_{\alpha}^{-1}(\mathscr{T}_{\alpha}) \), which means
		      \[
			      V = \bigcup_{i\in I} V_{i}
		      \]

		      where each \( V_{i} \) is a finite intersection of elements in \( \bigcup_{\alpha} f_{\alpha}^{-1}(\mathscr{T}_{\alpha}) \).
		      \[
			      V_{i} = \bigcap^{n_{i}}_{k=1} f_{\alpha_{k}}^{-1}(U_{\alpha_{k}}) = \bigcap^{n_{i}}_{k=1} j^{-1}\left( U_{\alpha_{k}} \times \prod_{\alpha \ne \alpha_{k}} Y_{\alpha} \right) = j^{-1}\left( \bigcap^{n_{i}}_{k=1} U_{\alpha_{k}} \times \prod_{\alpha \ne \alpha_{k}} Y_{\alpha} \right) = j^{-1}(W_{i})
		      \]

		      where \( U_{\alpha_{k}} \) is open in \( Y_{\alpha_{k}} \). So
		      \[
			      V = \bigcup_{i\in I} j^{-1}(W_{i}) = j^{-1}\left( \bigcup_{i\in I} W_{i} \right)
		      \]

		      which means \( V \) is the preimage of an open set in \( \prod_{\alpha} Y_{\alpha} \).

		      Thus \( \bigvee_{\alpha} f_{\alpha}^{-1}(\mathscr{T}_{\alpha}) \) is the same as the topology in \( X \) determined by \( j \) as in Problem~\ref{problem:VI.1.1}.
		\item According to Problem~\ref{problem:VI.1.1}, \( j \) is continuous, open, and closed.

		      Whenever \( x \ne x^{\prime} \), there is some index \( \alpha \) such that \( f_{\alpha}(x) \ne f_{\alpha}(x^{\prime}) \), then \( j(x) \ne j(x^{\prime}) \), which implies \( j \) is injective.

		      A continuous, open, injective map is an embedding so \( j \) is an embedding.
	\end{enumerate}
\end{proof}

\section{Subspaces}

\begin{problem}{VI.2.1}
Let \(X\) have the projective limit topology (Problem~\ref{problem:VI.1.4}) determined by
\[
	\left\{ Y_{\alpha}, f_{\alpha} \mid \alpha \in \mathscr{A} \right\}
\]

and let \( A \subset X \). Prove: The subspace topology of \(A\) is the projective limit topology determined by the maps \( f_{\alpha}\vert_{A} \).
\end{problem}

\begin{proof}
	The projective limit topology on \( A \) determined by the maps \( f_{\alpha}\vert_{A} \) has subbasis
	\[
		\bigcup_{\alpha} {(f_{\alpha}\vert_{A})}^{-1}(\mathscr{T}_{\alpha})
	\]

	Let \( V \) be an open set in \( A \) (with the projective limit topology) then
	\[
		V = \bigcup_{i\in I} V_{i}
	\]

	in which each \( V_{i} \) is the intersection of finitely many elements of \( \bigcup_{\alpha} {(f_{\alpha}\vert_{A})}^{-1}(\mathscr{T}_{\alpha}) \). So there exist \( \alpha_{i_{1}}, \ldots, \alpha_{i_{n(i)}} \in \mathscr{A} \) such that
	\[
		V_{i} = \bigcap^{n(i)}_{k=1} {(f_{\alpha_{k}}\vert_{A})}^{-1}(U_{\alpha_{k}})
	\]

	Hence
	\begingroup
	\allowdisplaybreaks%
	\begin{align*}
		V_{i} & = \bigcap^{n(i)}_{k=1} (A \cap f_{\alpha_{k}}^{-1}(U_{\alpha_{k}}))                                                             \\
		      & = A \cap \bigcap^{n(i)}_{k=1} f_{\alpha_{k}}^{-1}(U_{\alpha_{k}})                                                               \\
		      & = A \cap \bigcap^{n(i)}_{k=1} j^{-1}\left( U_{\alpha_{k}} \times \prod_{\alpha \ne \alpha_{k}} Y_{\alpha} \right)               \\
		      & = A \cap j^{-1}\left( \bigcap^{n(i)}_{k=1} \left( U_{\alpha_{k}} \times \prod_{\alpha\ne\alpha_{k}} Y_{\alpha} \right) \right).
	\end{align*}
	\endgroup

	Therefore
	\begingroup
	\allowdisplaybreaks%
	\begin{align*}
		V & = A \cap \bigcup_{i\in I} j^{-1}\left( \bigcap^{n(i)}_{k=1} \left( U_{\alpha_{k}} \times \prod_{\alpha\ne\alpha_{k}} Y_{\alpha} \right) \right) \\
		  & = A \cap j^{-1}\left( \bigcup_{i\in J} \bigcap^{n(i)}_{k=1} \left( U_{\alpha_{k}} \times \prod_{\alpha\ne\alpha_{k}} Y_{\alpha} \right) \right)
	\end{align*}
	\endgroup

	Hence \( V \) is in the subspace topology of \( A \).

	Conversely, one can show that if \( V \) is in the subspace topology of \( A \), then \( V \) is also in the projective limit topology on \( A \) determinded by the maps \( f_{\alpha}\vert_{A} \).

	Thus the projective limit topology on \( A \) determinded by the maps \( f_{\alpha}\vert_{A} \) and the subspace topology on \( A \) coincide.
\end{proof}

\section{General Theorems}

\begin{problem}{VI.3.1}
Let \( p: X \to Y \) be a continuous open (or closed) surjection, and assume that each fiber \( p^{-1}(y) \) is connected. For any \( F \subset Y \), show that \( F \) is connected if and only if \( p^{-1}(F) \) is connected.
\end{problem}

\begin{proof}
	By Proposition 2.1, \( p\vert_{p^{-1}(F)}: p^{-1}(F) \to F \) is an identification because \( p \) is an identification which is also an open (or closed) map. Denote \( q = p\vert_{p^{-1}(F)} \).

	If \( p^{-1}(F) \) is connected then \( F = p(p^{-1}(F)) \) is connected, as \( p \) is a continuous surjection.

	If \( p^{-1}(F) \) is not connected then there is a continuous surjection \( h: p^{-1}(F) \to 2 \). As each fiber of \( q \) (each fiber of \(q \) is a fiber of \(p\)) is connected, the restriction of \( h \) to each fiber is a constant map. Therefore \( hq^{-1}: F \to 2 \) is a continuous surjection, according to the transgression property, which means \( F \) is not connected.
\end{proof}

\begin{problem}{VI.3.2}
Let \( X \) have the projective limit topology \( \mathscr{T} \) determined by the family
\[
	\left\{ (Y_{\alpha}, \mathscr{T}_{\alpha}), f_{\alpha} \mid \alpha \in \mathscr{A} \right\}
\]

Assume that each \( \mathscr{T}_{\alpha} \) is the projective limit topology determined by a family
\[
	\left\{ (Z_{\alpha, \beta}, \mathscr{T}_{\alpha,\beta}), g_{\alpha,\beta} \mid \beta \in \mathscr{B} \right\}.
\]

Prove: \( \mathscr{T} \) is the projective limit topology determined by
\[
	\left\{ (Z_{\alpha,\beta}, \mathscr{T}_{\alpha,\beta}), g_{\alpha,\beta} \circ f_{\alpha} \mid (\alpha, \beta) \in \mathscr{A} \times \mathscr{B} \right\}.
\]
\end{problem}

\begin{proof}
	Denote by \( \widetilde{\mathscr{T}} \) the projective limit topology determined by
	\[
		\left\{ (Z_{\alpha,\beta}, \mathscr{T}_{\alpha,\beta}), g_{\alpha,\beta} \circ f_{\alpha} \mid (\alpha, \beta) \in \mathscr{A} \times \mathscr{B} \right\}.
	\]

	Let \( h: X \to \prod_{\alpha} Y_{\alpha} \) be the map \( h(x) = {\left\{ f_{\alpha}(x) \right\}}_{\alpha} \) then
	\[
		\mathscr{T} = \left\{ h^{-1}(U) \mid U \text{ open in } \prod_{\alpha}Y_{\alpha} \right\}
	\]

	according to Problem~\ref{problem:VI.1.4}.

	For each \( \alpha \), let \( h_{\alpha}: Y_{\alpha} \to \prod_{\beta} Z_{\alpha,\beta} \) be the map \( h_{\alpha}(x) = {\left\{ g_{\alpha,\beta}(x) \right\}}_{\beta} \) then
	\[
		\mathscr{T}_{\alpha} = \left\{ h_{\alpha}^{-1}(U) \mid U \text{ open in } \prod_{\beta} Z_{\alpha,\beta} \right\}
	\]

	according to Problem~\ref{problem:VI.1.4}.

	Let \( \ell: (X, \widetilde{\mathscr{T}}) \to \prod_{\alpha,\beta} Z_{\alpha,\beta} \) be the map \( \ell(x) = {\left\{ g_{\alpha,\beta}(f_{\alpha}(x)) \right\}}_{\alpha,\beta} \) then
	\[
		\widetilde{\mathscr{T}} = \left\{ \ell^{-1}(U) \mid U \text{ open in } \prod_{\alpha,\beta} Z_{\alpha,\beta} \right\}
	\]

	according to Problem~\ref{problem:VI.1.4}.

	Note that \( f_{\alpha} = p_{\alpha} \circ h \) and \( g_{\alpha,\beta} = p_{\alpha,\beta} \circ h_{\alpha} \) in which \( p_{\alpha}: \prod_{\alpha} Y_{\alpha} \to Y_{\alpha} \) and \( p_{\alpha,\beta}: \prod_{\beta} Z_{\alpha,\beta} \to Z_{\alpha,\beta} \) are projection maps. Denote by \( q_{\alpha,\beta} \) the projection map \( \prod_{\alpha,\beta} W_{\alpha,\beta} \to W_{\alpha,\beta} \).
	\[
		\begin{tikzcd}
			&& {\prod_{\alpha} Y_{\alpha}} \\
			\\
			X && {Y_{\alpha}} && {\prod_{\beta}Z_{\alpha,\beta}} && {Z_{\alpha,\beta}}
			\arrow["{p_{\alpha}}", from=1-3, to=3-3]
			\arrow["h", from=3-1, to=1-3]
			\arrow["{f_{\alpha}}"', from=3-1, to=3-3]
			\arrow["{h_{\alpha}}"', from=3-3, to=3-5]
			\arrow["{g_{\alpha,\beta}}"', bend right, from=3-3, to=3-7]
			\arrow["{p_{\alpha,\beta}}"', from=3-5, to=3-7]
		\end{tikzcd}
	\]

	\[
		\begin{tikzcd}
			X && {\prod_{\alpha,\beta} Z_{\alpha,\beta}} && {Z_{\alpha,\beta}}
			\arrow["\ell", from=1-1, to=1-3]
			\arrow["{g_{\alpha,\beta} \circ f_{\alpha}}"', bend right, from=1-1, to=1-5]
			\arrow["{q_{\alpha,\beta}}", from=1-3, to=1-5]
		\end{tikzcd}
	\]

	Let \( U \in \mathscr{T} \) then there exists \( V \) open in \( \prod_{\alpha} Y_{\alpha} \) such that \( U = h^{-1}(V) \) (see Problem~\ref{problem:VI.1.4} and~\ref{problem:VI.1.1}). One can write \( V \) in terms of subbasic elements as follows
	\[
		V = \bigcup_{i\in I} \bigcap^{n(i)}_{k=1} p_{\alpha_{k}}^{-1}(V_{\alpha_{k}})
	\]

	in which \( V_{\alpha_{k}} \) is open in \( Y_{\alpha_{k}} \).

	As \( \mathscr{T}_{\alpha} \) is the projective limit topology on \( Y_{\alpha} \) determined by the maps \( g_{\alpha,\beta}: Y_{\alpha} \to Z_{\alpha,\beta} \), there is an open set \( W_{\alpha_{k}} \) in \( \prod_{\beta} Z_{\alpha_{k},\beta} \) such that \( V_{\alpha_{k}} = h_{\alpha}^{-1}(W_{\alpha_{k}}) \). The open set \( W_{\alpha_{k}} \) can be written in terms of subbasic elements as follows
	\[
		W_{\alpha_{k}} = \bigcup_{j \in J} \bigcap^{n(j)}_{r=1} p_{\alpha_{k},\beta_{r}}^{-1}(W_{\alpha_{k}, \beta_{r}})
	\]

	in which \( W_{\alpha_{k}, \beta_{r}} \) is open in \( Z_{\alpha_{k}, \beta_{r}} \).
	\begingroup
	\allowdisplaybreaks%
	\begin{align*}
		V             & = \bigcup_{i\in I} \bigcap^{n(i)}_{k=1} p_{\alpha_{k}}^{-1}(V_{\alpha_{k}})                                                                                                                         \\
		              & = \bigcup_{i\in I} \bigcap^{n(i)}_{k=1} p_{\alpha_{k}}^{-1}\left( h^{-1}_{\alpha_{k}}(W_{\alpha_{k}}) \right)                                                                                       \\
		              & = \bigcup_{i\in I} \bigcap^{n(i)}_{k=1} {(h_{\alpha_{k}} \circ p_{\alpha_{k}})}^{-1}(W_{\alpha_{k}})                                                                                                \\
		              & = \bigcup_{i\in I} \bigcap^{n(i)}_{k=1} {(h_{\alpha_{k}} \circ p_{\alpha_{k}})}^{-1} \left( \bigcup_{j \in J} \bigcap^{n(j)}_{r=1} p_{\alpha_{k},\beta_{r}}^{-1}(W_{\alpha_{k}, \beta_{r}}) \right) \\
		              & = \bigcup_{i\in I} \bigcap^{n(i)}_{k=1} \bigcup_{j\in J} \bigcap^{n(j)}_{r=1} {(p_{\alpha_{k},\beta_{r}} \circ h_{\alpha_{k}} \circ p_{\alpha_{k}})}^{-1}(W_{\alpha_{k},\beta_{r}})                 \\
		U = h^{-1}(V) & = \bigcup_{i\in I} \bigcap^{n(i)}_{k=1} \bigcup_{j\in J} \bigcap^{n(j)}_{r=1} {(p_{\alpha_{k},\beta_{r}} \circ h_{\alpha_{k}} \circ p_{\alpha_{k}} \circ h)}^{-1}(W_{\alpha_{k},\beta_{r}})         \\
		              & = \bigcup_{i\in I} \bigcap^{n(i)}_{k=1} \bigcup_{j\in J} \bigcap^{n(j)}_{r=1} {(g_{\alpha_{k},\beta_{r}} \circ f_{\alpha_{k}})}^{-1}(W_{\alpha_{k},\beta_{r}})                                      \\
		              & = \bigcup_{i\in I} \bigcap^{n(i)}_{k=1} \bigcup_{j\in J} \bigcap^{n(j)}_{r=1} {(q_{\alpha_{k},\beta_{r}} \circ \ell)}^{-1}(W_{\alpha_{k},\beta_{r}})                                                \\
		              & = \bigcup_{i\in I} \bigcap^{n(i)}_{k=1} \bigcup_{j\in J} \bigcap^{n(j)}_{r=1} \ell^{-1}q_{\alpha_{k},\beta_{r}}^{-1}(W_{\alpha_{k},\beta_{r}})                                                      \\
		              & = \ell^{-1}\left( \bigcup_{i\in I} \bigcap^{n(i)}_{k=1} \bigcup_{j\in J} \bigcap^{n(j)}_{r=1} q_{\alpha_{k},\beta_{r}}^{-1}(W_{\alpha_{k},\beta_{r}}) \right) \in \widetilde{\mathscr{T}}
	\end{align*}
	\endgroup

	Hence \( U \in \widetilde{\mathscr{T}} \), which means \( \mathscr{T} \subset \widetilde{\mathscr{T}} \).

	\bigskip
	Conversely, let \( U \in \widetilde{\mathscr{T}} \) then there exists \( W \) open in \( \prod_{\alpha,\beta} Z_{\alpha,\beta} \) such that \( U = \ell^{-1}(W) \).

	\( W \) can be written in terms of subbasic elements.
	\begingroup
	\allowdisplaybreaks%
	\begin{align*}
		U & = \ell^{-1}(W) = \ell^{-1}\left( \bigcup_{i\in I}\bigcap^{n(i)}_{r=1} q^{-1}_{\alpha_{r},\beta_{r}}(W_{\alpha_{r},\beta_{r}}) \right) \\
		  & = \bigcup_{i\in I}\bigcap^{n(i)}_{r=1} \ell^{-1}q^{-1}_{\alpha_{r},\beta_{r}}(W_{\alpha_{r},\beta_{r}})                               \\
		  & = \bigcup_{i\in I}\bigcap^{n(i)}_{r=1} {(q_{\alpha_{r},\beta_{r}}\circ \ell)}^{-1}(W_{\alpha_{r},\beta_{r}})                          \\
		  & = \bigcup_{i\in I}\bigcap^{n(i)}_{r=1} {(g_{\alpha_{r},\beta_{r}}\circ f_{\alpha_{r}})}^{-1}(W_{\alpha_{r},\beta_{r}})                \\
		  & = \bigcup_{i\in I}\bigcap^{n(i)}_{r=1} f_{\alpha_{r}}^{-1}(g_{\alpha_{r},\beta_{r}}^{-1}(W_{\alpha_{r},\beta_{r}})) \in \mathscr{T}
	\end{align*}
	\endgroup

	so \( \widetilde{\mathscr{T}} \subset \mathscr{T} \).

	Thus \( \mathscr{T} = \widetilde{\mathscr{T}} \).
\end{proof}

\begin{problem}{VI.3.3}
Let \(X\) have the projective limit topology determined by \( \left\{ Y_{\alpha}, f_{\alpha} \mid \alpha \in \mathscr{A} \right\} \). Prove: \( f: Z \to X \) is continuous if and only if each \( f_{\alpha} \circ f \) is continuous.
\end{problem}

\begin{proof}
	For each \( \alpha \), the map \( f_{\alpha}: X \to Y_{\alpha} \) is continuous.

	If \( f \) is continuous then each \( f_{\alpha} \circ f \) is continuous.

	Conversely, assume that each \( f_{\alpha} \circ f \) is continuous. Let \( U \) be an open set in \( X \).
	\[
		\bigcup_{\alpha} f_{\alpha}^{-1}(\mathscr{T}_{\alpha})
	\]

	is a subbasis for the projective limit topology on \( X \). Therefore \( U \) can be written as
	\[
		U = \bigcup_{\gamma} \bigcap^{n(\gamma)}_{k=1} f_{\gamma,k}^{-1}(U_{\gamma,k})
	\]

	in which \( U_{\gamma,k} \) is open in \( Y_{\gamma,k} \) so
	\begingroup
	\allowdisplaybreaks%
	\begin{align*}
		f^{-1}(U) & = f^{-1}\left( \bigcup_{\gamma} \bigcap^{n(\gamma)}_{k=1} f_{\gamma,k}^{-1}(U_{\gamma,k}) \right) \\
		          & = \bigcup_{\gamma} \bigcap^{n(\gamma)}_{k=1} f^{-1}(f_{\gamma,k}^{-1}(U_{\gamma,k}))              \\
		          & = \bigcup_{\gamma} \bigcap^{n(\gamma)}_{k=1} {(f_{\gamma,k} \circ f)}^{-1}(U_{\gamma,k})
	\end{align*}
	\endgroup

	\( {(f_{\gamma,k} \circ f)}^{-1}(U_{\gamma,k}) \) is open as \( U_{\gamma,k} \) is open in \( Y_{\gamma,k} \) and \( f_{\gamma,k} \circ f \) is continuous. Hence \( f^{-1}(U) \) is open (finite intersection then arbitrary union), so \( f \) is continuous.

	Thus \( f \) is continuous if and only if each \( f_{\alpha} \circ f \) is continuous.
\end{proof}

\section{Spaces with Equivalence Relations}

\begin{problem}{VI.4.1}
Let \( p: X \to X/R \) be an open (or closed) map, and \( B \subset X/R \) any subset. Show that \( B \) is homeomorphic to \( p^{-1}(B)/R_{0} \), where \( R_{0} \) is the restriction of \( R \) on \( p^{-1}(B) \).
\end{problem}

\begin{proof}
	Let \( q = p\vert_{p^{-1}(B)}: p^{-1}(B) \to B \) and \( r: p^{-1}(B) \to p^{-1}(B)/R_{0} \).

	\( p \) is an open (or closed) identification so \( q \) is an identification. Also, \( r \) is an identification.
	\[
		\begin{tikzcd}
			{p^{-1}(B)} && B \\
			\\
			{p^{-1}(B)/R_{0}}
			\arrow["q", from=1-1, to=1-3]
			\arrow["r"', from=1-1, to=3-1]
			\arrow["{rq^{-1}}", from=1-3, to=3-1]
			\arrow["{qr^{-1}}"{description}, shift left=3, from=3-1, to=1-3]
		\end{tikzcd}
	\]

	\( q \) is constant on each fiber of \( r \) and \( r \) is constant on each fiber of \( q \). Therefore the induced maps \( qr^{-1} \) and \( rq^{-1} \) are continuous. On the other hand, \( qr^{-1} \) and \( rq^{-1} \) are inverses of each other so they are homeomorphisms, which implies \( B \cong p^{-1}(B)/R_{0} \).
\end{proof}

\begin{problem}{VI.4.2}
Give an example showing that if \( A \subset X \) is not open or closed, then \( X - A \) need not be homeomorphic to the complement of \( [A] \) in \( X/A \).
\end{problem}

\begin{proof}
	Let \( X = \mathbb{R} \) and \( A = \mathbb{Q} \) then \( A \) is not open or closed in \( X \).

	The restriction of \( q: X \to X/A \) to \( q\vert_{X - A}: X - A \to X/A - \left\{ [A] \right\} \) is a continuous bijection. Let \( O = \openinterval{0, 1} \cap (\mathbb{R} - \mathbb{Q}) = \openinterval{0, 1} \cap (X - A) \) then \( O \) is open in \( X - A \). We will show that \( q(O) \) is not open in \( q(X - A) \).

	Assume that \( q(O) \) is open in \( q(X - A) \) then there exists an open set \( U \subset X/A \) such that \( q(O) = q(X - A) \cap U \).

	If \( U \) doesn't contain \( [A] \) then the preimage \( q^{-1}(U) \) doesn't contain any rational numbers, hence not open, which is a contradiction since \( U \) is open in \( X/A \) and \( q \) is continuous. Therefore \( U \) contains \( [A] \), so \( q^{-1}(U) \) contains \( \mathbb{Q} \), and
	\[
		q^{-1}(U) = q^{-1}(\left\{ [A] \right\} \cup q(O)) = q^{-1}(\left\{ [A] \right\}) \cup q^{-1}(q(O)) = \mathbb{Q} \cup O
	\]

	which is not open in \( X = \mathbb{R} \). This is a contradiction as \( q^{-1}(U) \) is simutaneously open and non-open.

	Thus \( q \) is not an open map, so it is not a homeomorphism.
\end{proof}

\begin{problem}{VI.4.3}
These problems will be much easier after studying compactness. (Use now Problem~\ref{problem:III.9.1}.)
\begin{enumerate}[label={(\alph*)}]
	\item In \( I^{2} \), let \( (0, y) \sim (1, y) \). Show \( I^{2}/R \cong \) the cylinder \( S^{1} \times I \).
	\item In \( I^{2} \), let \( (0, y) \sim (1, 1 - y) \). Show \( I^{2}/R \cong \) M\"{o}bius band.
	\item In \( I^{2} \), let \( \operatorname{Fr}(I^{2}) \sim (0, 0) \). Show \( I^{2}/R \cong S^{2} \).
	\item In \( I^{2} \), let \( (0, y) \sim (1, y), (x, 0) \sim (x, 1) \). Show \( I^{2}/R \cong \) the torus \( S^{1} \times S^{1} \).
\end{enumerate}
\end{problem}

\begin{proof}
	I have no idea other than using compactness and the closed map lemma.
\end{proof}

\begin{problem}{VI.4.4}
Let \( R \) be an equivalence relation in \( X \). For each \( A \subset X \) define \( C(A) = \left\{ x \in X \mid \exists a \in A: x R a \right\} \). Show that \( p(U) \) is open in \( X/R \) if and only if \( C(U) \) is open in \( X \).
\end{problem}

\begin{proof}
	According to the definition of \( C \), \( C(U) = p^{-1}(p(U)) \). Due to the definition of identification maps, \( p(U) \) is open in \( X/R \) if and only if \( p^{-1}(p(U)) \) is open in \( X \).

	Thus \( p(U) \) is open in \( X/R \) if and only if \( C(U) \) is open in \( X \).
\end{proof}

\begin{problem}{VI.4.5}
Let \( R, S \) be two equivalence relations in \( X \), and such that \( S \subset R \) (see Problem~\ref{problem:I.7.6}). Prove that \( (X/S)/(R/S) \cong X/R \).
\end{problem}

\begin{proof}
	Let \( p_{R}: X \to X/R \), \( p_{S}: X \to X/S \), and \( p: X/S \to (X/S)/(R/S) \) be projection maps.

	According to the transgression property, there is an induced continuous map \( q = p_{R}p_{S}^{-1} \) such that \( q \circ p_{S} = p_{R} \). According to Chapter I, section 7, there exists a unique map \( q \) that commutes the following diagram.
	\[
		\begin{tikzcd}
			X && X \\
			\\
			{X/S} && {X/R}
			\arrow["1", from=1-1, to=1-3]
			\arrow["{p_{S}}"', from=1-1, to=3-1]
			\arrow["{p_{R}}", from=1-3, to=3-3]
			\arrow["q"', from=3-1, to=3-3]
		\end{tikzcd}
	\]

	For each subset \( U \subset X/R \), \( p_{R}^{-1}(U) \) is \( p_{S} \)-saturated because \( S \subset R \). If \( U \) is open then \( p_{R}^{-1}(U) \) is open in \( X \) and \( p_{S} \)-saturated so \( p_{S}(p_{R}^{-1}(U)) \) is open in \( X/S \). Conversely, if \( q^{-1}(U) \) is open in \( X/S \) then \( p_{S}^{-1}(q^{-1}(U)) = {(q \circ p_{S})}^{-1}(U) = p_{R}^{-1}(U) \) is open in \( X \), which means \( U \) is open in \( X/R \). Hence \( q \) is an identification.

	According to the definition of \( R/S \) (see Problem~\ref{problem:I.7.5} and~\ref{problem:I.7.6})
	\[
		p(Sa) = p(Sb) \iff (Sa)R/S(Sb) \iff q(Sa) = q(Sb)
	\]

	which means \( q \) is constant on each fiber of \( p \) and \( p \) is constant on each fiber of \( q \)
	\[
		\begin{tikzcd}
			{X/S} && {X/R} \\
			\\
			{(X/S)/(R/S)}
			\arrow["q", from=1-1, to=1-3]
			\arrow["p"', from=1-1, to=3-1]
			\arrow["\cong"', from=3-1, to=1-3]
		\end{tikzcd}
	\]

	so \( (X/S)/(R/S) \cong X/R \).
\end{proof}

\begin{problem}{VI.4.6}
Let \( 0 \) be the origin in \( E^{3} \). In \( E^{3} - \left\{ 0 \right\} \), define \( x R y \) if \( x \) and \( y \) lie on a line through the origin. Show that \( R \) is an equivalence relation; \( (E^{3} - \left\{ 0 \right\})/R \) is called the projective plane \( P^{2} \). Call \textit{line} in \( P^{2} \) any set \( A \) such that \( p^{-1}(A) \) is a plane in \( E^{3} - \left\{ 0 \right\} \) going through the origin. Show that a line in \( P^{2} \) is homeomorphic to \( S^{1} \).
\end{problem}

\begin{proof}
	We will show that \( p: E^{3} - \left\{ 0 \right\} \to (E^{3} - \left\{ 0 \right\})/R \) an open map. Let \( U \) be an open set in \( E^{3} - \left\{ 0 \right\} \). The image \( p(U) \) is open in \( P^{2} \) if and only if \( p^{-1}(p(U)) \) is open in \( E^{3} - \left\{ 0 \right\} \).
	\[
		p^{-1}(p(U)) = \bigcup_{t \in E^{1} - \left\{ 0 \right\}} \left\{ tu \mid u \in U \right\}
	\]

	is open as \( \left\{ tu \mid u \in U \right\} \) is homeomorphic to \( U \) for each \( t \in E^{1} - \left\{ 0 \right\} \). Hence \( p \) is an open map.

	Let \( A \) be any line in \( P^{2} \) then \( p^{-1}(A) \) is a plane in \( E^{3} - \left\{ 0 \right\} \) going through the origin. Because \( p \) is an open map, \( p\vert_{p^{-1}(A)}: p^{-1}(A) \to A \) is an identification.

	Let's define \( f: E^{3} - \left\{ 0 \right\} \to E^{3} - \left\{ 0 \right\} \) by \( f(x) = x/\left\vert x \right\vert \) is continuous and open, so \( f: p^{-1}(A) \to f(p^{-1}(A)) \) is an identification. Moreover, \( f(p^{-1}(A)) \cong S^{1} \). The map \( g: S^{1} \to S^{1}/\left\{ x \sim -x \right\} \) is an open continuous surjection.

	The identifications \( g \circ (f\vert_{p^{-1}(A)}) \) and \( p\vert_{p^{-1}(A)} \) are constant on each fiber of the other map so \( A \) and \( S^{1}/\left\{ x \sim -x \right\} \) are homeomorphic.

	The homeomorphism of \( S^{1} \) and \( S^{1}/\left\{ x \sim -x \right\} \) follows from the map \( z \to z^{2} \).

	Thus \( A \cong S^{1} \).
\end{proof}

\begin{problem}{VI.4.7}
Let \( V^{2} = \left\{ x \in E^{2} \mid \left\vert x \right\vert \le 1 \right\} \). Generate an equivalence relation by \( x R y \) if \( \left\vert x \right\vert = \left\vert y \right\vert = 1 \), and \( x, y \) are diametrically opposite. Show \( V^{2}/R \) is homeomorphic to \( P^{2} \).
\end{problem}

\begin{proof}
	\( P^{2} \) is homeomorphic to \( S^{2}/\left\{ x \sim -x \right\} \).

	Let \( US^{2} \) be the upper hemisphere (\( x_{3} \ge 0 \)) then \( V^{2} \cong US^{2} \) and \( US^{2}/\left\{ x \sim -x \right\} \cong V^{2}/R \).

	Thus \( P^{2} \cong S^{2}/\left\{ x \sim -x \right\} \cong US^{2}/\left\{ x \sim -x \right\} \cong V^{2}/R \).
\end{proof}

\section{Cones and Suspensions}

\begin{quotation}
	The following two problems show that: The cone functor \( T \) takes closed embeddings to closed embeddings but doesn't necessarily take embeddings to embeddings.
\end{quotation}

\begin{problem}{VI.5.1}
If \( A \subset X \) is closed, prove that \( TA \) is homeomorphic to a closed subspace of \( TX \).
\end{problem}

\begin{proof}
	Denote \( q: X \times I \to TX \) and \( p: A \times I \to TA \).

	We will show that \( q(A \times I) \) is closed in \( TX \).
	\[
		q^{-1}(q(A \times I)) = (A \times I) \cup (X \times \left\{ 1 \right\})
	\]

	and \( A \times I, X \times \left\{ 1 \right\} \) are closed in \( X \times I \) so \( q(A \times I) \) is closed in \( TX \).

	Let \( p(a, t) \) be a point in \( TA \) then \( q(p^{-1}(p(a, t))) = \left\{ q(a, t) \right\} \), hence we can define a map \( f: TA \to q(A \times I) \) by \( f(p(a, t)) = q(a, t) \). This map \( f \) is bijective and \( f \circ p = q, p = f^{-1} \circ q \). As \( f^{-1} \circ q \) is continuous, we deduce that \( f^{-1} \) is continuous (Theorem 3.1). Besides, \( q \) is constant on each fiber of \( p \) so \( f \) is continuous (Theorem 3.2 on Transgression). Therefore \( f \) is a homeomorphism.

	Thus \( TA \) is homeomorphic to \( q(A\times I) \), which is a closed subspace of \( TX \).
\end{proof}

\begin{problem}{VI.5.2}
Let \( i: \operatorname{Int}(I) \to I \) be the inclusion map. Show that the map \( Ti: T[\operatorname{Int}(I)] \to TI \) is not an embedding.
\end{problem}

\begin{proof}
	This idea is from \href{https://math.stackexchange.com/a/4879627}{Thorgott on Math StackExchange}.

	\( Ti \) is injective and continuous.

	I don't want to make confusion so I introduce new notations: \( X = I \) and \( U = \operatorname{Int}(I) \).

	\( p: X \times I \to TX \) and \( q: U \times I \to TU \) are identification maps that induce the cones.

	The set \( A = \left\{ (s, t): s < t, s \in U, t \in I \right\} \) is open in \( U \times I \) and \( q \)-saturated (because \( A \) contains \( U \times \left\{ 1 \right\} \)) so \( B = q(A) \) is open in \( TU \). We will show that \( Ti(B) \) is not open in \( Ti(TU) \).

	Assume that \( Ti(B) \) is open in \( Ti(TU) \) then there exists an open set \( W \subset TX \) such that \( Ti(B) = Ti(TU) \cap W \).
	\[
		q^{-1}(Ti(B)) = (U \times I) \cap q^{-1}(W) = (U \times I) \cap p^{-1}(W).
	\]

	\( p^{-1}(W) \) is open in \( X \times I \) and contains \( X \times \left\{ 1 \right\} \) so \( p^{-1}(W) \) contains \( U \times \halfopenleft{1 - \varepsilon, 1} \) for some \( \varepsilon \in \openinterval{0, 1} \). Therefore \( q^{-1}(Ti(B)) \) contains \( U \times \halfopenleft{1 - \varepsilon, 1} \) and \( A \) contains \( U \times \halfopenleft{1 - \varepsilon, 1} \), which contradicts the definition of \( A \).

	Thus \( Ti \) is not an embedding.
\end{proof}

\section{Attaching of Spaces}

\section{The Relation \(K(f)\) of Continuous Maps}

\section{Weak Topologies}

\chapter{Non-Euclidean Geometry}

\chapter{Baire Spaces and Dimension Theory}

\section{Baire Spaces}

\section{A Nowhere-Differentiable Function}

\section{Introduction to Dimension Theory}

\chapter{Multilinear Algebra and Determinants}

\section{Bilinear Forms and Quadratic Forms}

\section{Alternating Multilinear Forms}

\section{Determinants}

\section{Tensor Products}


\chapter{Tychonoff's Theorem}

\section{The Product Topology For All Products}



\section{Zorn's Lemma}



\section{Tychonoff's Theorem}



\section{Stone-Czech Compacitification}




\chapter{Representations of Clifford Algebras}

\chapter{Stone-Weierstra{\ss} Theorem}

\section{The Weierstra{\ss} Approximation Theorem}



\section{The tone-Weierstra{\ss} Theorem}




\chapter{The space $C(Y)$}

\chapter{Elements of Vector Analysis and Field Theory}

\chapter{*Integration of Differential Forms on Manifolds}

\chapter{Uniform Convergence and the Basic Operations of Analysis on Series and Families of Functions}

\chapter{Topology of $E_{n}$}

\chapter{Fourier Series and the Fourier Transform}

\chapter{Path spaces; $H$-spaces}

\chapter{Fiber spaces}


\end{document}
