% chktex-file 44
\chapter{Linear Equations}

\section{Fields (no exercises)}

\section{Systems of Linear Equations}

\begin{exercise}
    Prove that the set of complex numbers of the form $a + b\sqrt{2}$, where $a, b$ are rational numbers, is a subfield of $\mathbb{C}$.
\end{exercise}

\begin{proof}
    Denote the set by $\mathbb{Q}(\sqrt{2})$. The addition and multiplication of $\mathbb{Q}(\sqrt{2})$ are defined as follows (these are usual addition and multiplication):
    \[
        \begin{split}
            (a_{1} + b_{1}\sqrt{2}) + (a_{2} + b_{2}\sqrt{2}) = (a_{1} + a_{2}) + (b_{1} + b_{2})\sqrt{2} \\
            (a_{1} + b_{1}\sqrt{2}) \cdot (a_{2} + b_{2}\sqrt{2}) = (a_{1}a_{2} + 2b_{1}b_{2}) + (a_{1}b_{2} + b_{1}a_{2})\sqrt{2}
        \end{split}
    \]

    \begin{enumerate}[label = (\arabic*)]
        \item Addition is associative.
              \begin{align*}
                    & \left((a_{1} + b_{1}\sqrt{2}) + (a_{2} + b_{2}\sqrt{2})\right) + (a_{3} + b_{3}\sqrt{2})                    \\
                  = & \left((a_{1} + a_{2}) + (b_{1} + b_{2})\sqrt{2}\right) + (a_{3} + b_{3}\sqrt{2})                            \\
                  = & \left(\left((a_{1} + a_{2}) + a_{3}\right)\right) + \left(\left(b_{1} + b_{2}\right) + b_{3}\right)\sqrt{2} \\
                  = & \left(a_{1} + \left(a_{2} + a_{3}\right)\right) + \left(b_{1} + \left(b_{2} + b_{3}\right)\right)\sqrt{2}   \\
                  = & (a_{1} + b_{1}\sqrt{2}) + \left((a_{2} + a_{3}) + (b_{2} + b_{3})\sqrt{2}\right)                            \\
                  = & (a_{1} + b_{1}\sqrt{2}) + \left((a_{2} + b_{2}\sqrt{2}) + (a_{3} + b_{3}\sqrt{2})\right)
              \end{align*}
        \item Addition has identity element.
              \begin{align*}
                  (a + b\sqrt{2}) + 0 = (a + b\sqrt{2}) + (0 + 0\sqrt{2}) = (a + 0) + (b + 0)\sqrt{2} = a + b\sqrt{2} \\
                  0 + (a + b\sqrt{2}) = (0 + 0\sqrt{2}) + (a + b\sqrt{2}) = (0 + a) + (0 + b)\sqrt{2} = a + b\sqrt{2}
              \end{align*}
        \item Each element has additive inverse.
              \begin{align*}
                  (a + b\sqrt{2}) + ((-a) + (-b)\sqrt{2}) & = (a + (-a)) + (b + (-b))\sqrt{2} = 0 + 0\sqrt{2} = 0 \\
                  ((-a) + (-b)\sqrt{2}) + (a + b\sqrt{2}) & = ((-a) + a) + ((-b) + b)\sqrt{2} = 0 + 0\sqrt{2} = 0
              \end{align*}
        \item Addition is commutative.
              \begin{align*}
                  (a_{1} + b_{1}\sqrt{2}) + (a_{2} + b_{2}\sqrt{2}) & = (a_{1} + a_{2}) + (b_{1} + b_{2})\sqrt{2}         \\
                                                                    & = (a_{2} + a_{1}) + (b_{2} + b_{1})\sqrt{2}         \\
                                                                    & = (a_{2} + b_{2}\sqrt{2}) + (a_{1} + b_{1}\sqrt{2})
              \end{align*}
        \item Multiplication is associative.
              \begin{align*}
                    & \left((a_{1} + b_{1}\sqrt{2})\cdot (a_{2} + b_{2}\sqrt{2})\right)\cdot (a_{3} + b_{3}\sqrt{2})                                                                \\
                  = & \left((a_{1}a_{2} + 2b_{1}b_{2}) + (a_{1}b_{2} + b_{1}a_{2})\sqrt{2}\right)\cdot(a_{3} + b_{3}\sqrt{2})                                                       \\
                  = & (a_{1}a_{2}a_{3} + 2b_{1}b_{2}a_{3} + 2a_{1}b_{2}b_{3} + 2b_{1}a_{2}b_{3}) + (a_{1}a_{2}b_{3} + b_{1}a_{2}a_{3} + a_{1}b_{2}a_{3} + 2b_{1}b_{2}b_{3})\sqrt{2}
              \end{align*}
              \begin{align*}
                    & (a_{1} + b_{1}\sqrt{2})\cdot\left((a_{2} + b_{2}\sqrt{2})\cdot(a_{3} + b_{3}\sqrt{2})\right)                                                                  \\
                  = & (a_{1} + b_{1}\sqrt{2})\cdot\left( (a_{2}a_{3} + 2b_{2}b_{3}) + (a_{2}b_{3} + b_{2}a_{3})\sqrt{2} \right)                                                     \\
                  = & (a_{1}a_{2}a_{3} + 2a_{1}b_{2}b_{3} + 2b_{1}a_{2}b_{3} + 2b_{1}b_{2}a_{3}) + (a_{1}a_{2}b_{3} + a_{1}b_{2}a_{3} + b_{1}a_{2}a_{3} + 2b_{1}b_{2}b_{3})\sqrt{2}
              \end{align*}
        \item Multiplication is distributive over addition.
              \begin{align*}
                    & (a_{1} + b_{1}\sqrt{2})\cdot \left((a_{2} + b_{2}\sqrt{2}) + (a_{3} + b_{3}\sqrt{2})\right)                                                               \\
                  = & (a_{1} + b_{1}\sqrt{2})\cdot\left((a_{2} + a_{3}) + (b_{2} + b_{3})\sqrt{2}\right)                                                                        \\
                  = & (a_{1}a_{2} + a_{1}a_{3} + 2b_{1}(b_{2} + b_{3})) + (a_{1}b_{2} + a_{1}b_{3} + b_{1}a_{2} + b_{1}a_{3})\sqrt{2}                                           \\
                  = & \left((a_{1}a_{2} + 2b_{1}b_{2}) + (a_{1}b_{2} + b_{1}a_{2})\sqrt{2}\right) + \left((a_{1}a_{3} + 2b_{1}b_{3}) + (a_{1}b_{3} + b_{1}a_{3})\sqrt{2}\right) \\
                  = & (a_{1} + b_{1}\sqrt{2})\cdot (a_{2} + b_{2}\sqrt{2}) + (a_{1} + b_{1}\sqrt{2})\cdot (a_{3} + b_{3}\sqrt{2})
              \end{align*}
              \begin{align*}
                    & \left((a_{1} + b_{1}\sqrt{2}) + (a_{2} + b_{2}\sqrt{2})\right)\cdot (a_{3} + b_{3}\sqrt{2})                                                               \\
                  = & \left((a_{1} + a_{2}) + (b_{1} + b_{2})\sqrt{2}\right)\cdot (a_{3} + b_{3}\sqrt{2})                                                                       \\
                  = & \left(a_{1}a_{3} + a_{2}a_{3} + 2b_{1}b_{3} + 2b_{2}b_{3}\right) + (a_{1}b_{3} + a_{2}b_{3} + b_{1}a_{3} + b_{2}a_{3})\sqrt{2}                            \\
                  = & \left((a_{1}a_{3} + 2b_{1}b_{3}) + (a_{1}b_{3} + b_{1}a_{3})\sqrt{2}\right) + \left((a_{2}a_{3} + 2b_{2}b_{3}) + (a_{2}b_{3} + b_{2}a_{3})\sqrt{2}\right) \\
                  = & (a_{1} + b_{1}\sqrt{2})\cdot (a_{3} + b_{3}\sqrt{2}) + (a_{2} + b_{2}\sqrt{2})\cdot (a_{3} + b_{3}\sqrt{2})
              \end{align*}
        \item Multiplication has identity element
              \begin{align*}
                  (a + b\sqrt{2})\cdot 1 & = (a + b\sqrt{2})\cdot (1 + 0\sqrt{2}) = a + b\sqrt{2} \\
                  1\cdot (a + b\sqrt{2}) & = (1 + 0\sqrt{2})\cdot (a + b\sqrt{2}) = a + b\sqrt{2}
              \end{align*}
        \item Multiplicative is commutative.
              \begin{align*}
                  (a_{1} + b_{1}\sqrt{2})\cdot (a_{2} + b_{2}\sqrt{2}) & = (a_{1}a_{2} + 2b_{1}b_{2}) + (a_{1}b_{2} + b_{1}a_{2})\sqrt{2} \\
                                                                       & = (a_{2}a_{1} + 2b_{2}b_{1}) + (a_{2}b_{1} + b_{2}a_{1})\sqrt{2} \\
                                                                       & = (a_{2} + b_{2}\sqrt{2})\cdot (a_{1} + b_{1}\sqrt{2})
              \end{align*}
        \item Each non-zero element has multiplicative inverse.

              $a + b\sqrt{2}$ is zero element if and only if $a = b = 2$ (since $\sqrt{2}$ is irrational).

              \begin{align*}
                  (a + b\sqrt{2})\cdot \left( \frac{a}{a^{2} - 2b^{2}} + \frac{(-b)}{a^{2} - 2b^{2}}\sqrt{2} \right) & = \frac{a^{2} - 2b^{2}}{a^{2} - 2b^{2}} + \frac{a(-b) + ba}{a^{2} - 2b^{2}}\sqrt{2} = 1 \\
                  \left( \frac{a}{a^{2} - 2b^{2}} + \frac{(-b)}{a^{2} - 2b^{2}}\sqrt{2} \right)\cdot (a + b\sqrt{2}) & = \frac{a^{2} - 2b^{2}}{a^{2} - 2b^{2}} + \frac{ab + (-b)a}{a^{2} - 2b^{2}}\sqrt{2} = 1
              \end{align*}
    \end{enumerate}
\end{proof}

Let $\mathbb{F}$ be the field of complex numbers. Are the following two systems of linear equations equivalent? If so, express each equation in each system as a linear combination of the equations in the other system.

\begin{exercise}
    \[
        \begin{cases}
            x_{1} - x_{2} = 0 \\
            2x_{1} + x_{2} = 0
        \end{cases}
        \qquad
        \begin{cases}
            3x_{1} + x_{2} = 0 \\
            x_{1} + x_{2} = 0
        \end{cases}
    \]
\end{exercise}

\begin{proof}
    The two systems are equivalent.
    \[
        \begin{cases}
            3x_{1} + x_{2} = \frac{1}{3}(x_{1} - x_{2}) + \frac{4}{3}(2x_{1} + x_{2}) \\
            x_{1} + x_{2} =  \frac{-1}{3}(x_{1} - x_{2}) + \frac{2}{3}(2x_{1} + x_{2})
        \end{cases}
    \]
    \[
        \begin{cases}
            x_{1} - x_{2} = (3x_{1} + x_{2}) - 2(x_{1} + x_{2}) \\
            2x_{1} + x_{2} = \frac{1}{2}(3x_{1} + x_{2}) + \frac{1}{2}(x_{1} + x_{2})
        \end{cases}
    \]
\end{proof}

\begin{exercise}
    \[
        \begin{cases}
            -x_{1} + x_{2} + 4x_{3} = 0 \\
            x_{1} + 3x_{2} + 8x_{3} = 0 \\
            \frac{1}{2}x_{1} + x_{2} + \frac{5}{2}x_{3} = 0
        \end{cases}
        \qquad
        \begin{cases}
            x_{1} - x_{3} = 0 \\
            x_{2} + 3x_{3} = 0
        \end{cases}
    \]
\end{exercise}

\begin{proof}
    The two systems are equivalent.
    \[
        \begin{cases}
            -x_{1} + x_{2} + 4x_{3} = (-1)(x_{1} - x_{3}) + (x_{2} + 3x_{3}) \\
            x_{1} + 3x_{2} + 8x_{3} = (x_{1} - x_{3}) + 3(x_{2} + 3x_{3})    \\
            \frac{1}{2}x_{1} + x_{2} + \frac{5}{2}x_{3} = \frac{1}{2}(x_{1} - x_{3}) + (x_{2} + 3x_{3})
        \end{cases}
    \]
    \[
        \begin{cases}
            x_{1} - x_{3} = \frac{-2}{3}(-x_{1} + x_{2} + 4x_{3}) + \frac{2}{3}(\frac{1}{2}x_{1} + x_{2} + \frac{5}{2}x_{3}) \\
            x_{2} + 3x_{3} = \frac{1}{4}(-x_{1} + x_{2} + 4x_{3}) + \frac{1}{4}(x_{1} + 3x_{2} + 8x_{3})
        \end{cases}
    \]
\end{proof}

\begin{exercise}
    \[
        \begin{cases}
            2x_{1} + (-1 + i)x_{2} + x_{4} = 0 \\
            3x_{2} - 2i x_{3} + 5x_{4} = 0
        \end{cases}
        \qquad
        \begin{cases}
            \left(1 + \frac{i}{2}\right)x_{1} + 8x_{2} - i x_{3} - x_{4} = 0 \\
            \frac{2}{3}x_{1} - \frac{1}{2}x_{2} + x_{3} + 7x_{4} = 0
        \end{cases}
    \]
\end{exercise}

\begin{proof}
    Suppose that $\frac{2}{3}x_{1} - \frac{1}{2}x_{2} + x_{3} + 7x_{4}$ is a linear combination of $2x_{1} + (-1 + i)x_{2} + x_{4}$ and $3x_{2} - 2i x_{3} + 5x_{4}$, then there exists $a$ and $b$ such that
    \[
        \frac{2}{3}x_{1} - \frac{1}{2}x_{2} + x_{3} + 7x_{4} = a(2x_{1} + (-1 + i)x_{2} + x_{4}) + b(3x_{2} - 2i x_{3} + 5x_{4})
    \]

    By identifying coefficients
    \[
        \begin{cases}
            \frac{2}{3} = 2a         \\
            1 = -2i b                \\
            -\frac{1}{2} = (-1 + i)a \\
            7 = a + 5b
        \end{cases}
        \Longrightarrow
        \begin{cases}
            a = \frac{1}{3} \\
            a = \frac{1 + i}{4}
        \end{cases}
    \]

    So $\frac{2}{3}x_{1} - \frac{1}{2}x_{2} + x_{3} + 7x_{4}$ is NOT a linear combination of $2x_{1} + (-1 + i)x_{2} + x_{4}$ and $3x_{2} - 2i x_{3} + 5x_{4}$. Thus the two systems are inequivalent.
\end{proof}

\begin{exercise}
    Let $F$ be a set which contains exactly two elements, 0 and 1. Define an addition and multiplication by the tables:
    \[
        \begin{array}{c|cc}
            + & 0 & 1 \\
            \hline
            0 & 0 & 1 \\
            1 & 1 & 0
        \end{array}
        \qquad
        \begin{array}{c|cc}
            \cdot & 0 & 1 \\
            \hline
            0     & 0 & 0 \\
            1     & 0 & 1
        \end{array}
    \]

    Verify that the set $F$, together with these two operations, is a field.
\end{exercise}

\begin{proof}
    \begin{enumerate}[label = (\arabic*)]
        \item Addition is associative.
              \[
                  \begin{split}
                      &(0 + 0) + 0 = 0 + 0 = 0 + (0 + 0) \\
                      &(0 + 0) + 1 = 0 + 1 = 1 = 0 + 1 = 0 + (0 + 1) \\
                      &(0 + 1) + 0 = 1 + 0 = 1 = 0 + 1 = 0 + (1 + 0) \\
                      &(0 + 1) + 1 = 1 + 1 = 0 = 0 + 0 = 0 + (1 + 1) \\
                      &(1 + 0) + 0 = 1 + 0 = 1 + (0 + 0) \\
                      &(1 + 0) + 1 = 1 + 1 = 1 + (0 + 1) \\
                      &(1 + 1) + 0 = 0 + 0 = 0 = 1 + 1 = 1 + (1 + 0) \\
                      &(1 + 1) + 1 = 0 + 1 = 1 = 1 + 0 = 1 + (1 + 1)
                  \end{split}
              \]
        \item Addition has identity element.
              \[
                  \begin{split}
                      0 + 0 = 0 \\
                      1 + 0 = 1 = 0 + 1
                  \end{split}
              \]
        \item Each element has additive inverse.
              \[
                  \begin{split}
                      0 + 0 = 0 \\
                      1 + 1 = 0
                  \end{split}
              \]
        \item Addition is commutative.
              \[
                  \begin{split}
                      0 + 0 = 0 \\
                      1 + 1 = 0 \\
                      0 + 1 = 1 = 1 + 0
                  \end{split}
              \]
        \item Multiplication is associative.
              \[
                  \begin{split}
                      &(0 \cdot 0) \cdot 0 = 0 \cdot 0 = 0 \cdot (0 \cdot 0) \\
                      &(0 \cdot 0) \cdot 1 = 0 \cdot 1 = 0 = 0 \cdot 0 = 0 \cdot (0 \cdot 1) \\
                      &(0 \cdot 1) \cdot 0 = 0 \cdot 0 = 0 \cdot (1 \cdot 0) \\
                      &(0 \cdot 1) \cdot 1 = 0 \cdot 1 = 0 \cdot (1 \cdot 1) \\
                      &(1 \cdot 0) \cdot 0 = 0 \cdot 0 = 0 = 1 \cdot 0 = 1 \cdot (0 \cdot 0) \\
                      &(1 \cdot 0) \cdot 1 = 0 \cdot 1 = 0 = 1 \cdot 0 = 1 \cdot (0 \cdot 1) \\
                      &(1 \cdot 1) \cdot 0 = 1 \cdot 0 = 0 = 1 \cdot 0 = 1 \cdot (1 \cdot 0) \\
                      &(1 \cdot 1) \cdot 1 = 1 \cdot 1 = 1 \cdot (1 \cdot 1)
                  \end{split}
              \]
        \item Multiplication is distributive over addition.
              \[
                  \begin{split}
                      &0\cdot(0 + 0) = 0\cdot 0 = 0 = 0 + 0 = 0\cdot 0 + 0\cdot 0 \\
                      &0\cdot(0 + 1) = 0\cdot 1 = 0 = 0 + 0 = 0\cdot 0 + 0\cdot 1 \\
                      &0\cdot(1 + 0) = 0\cdot 1 = 0 = 0 + 0 = 0\cdot 1 + 0\cdot 0 \\
                      &0\cdot(1 + 1) = 0\cdot 0 = 0 = 0 + 0 = 0\cdot 1 + 0\cdot 1 \\
                      &1\cdot(0 + 0) = 1\cdot 0 = 0 = 0 + 0 = 1\cdot 0 + 1\cdot 0 \\
                      &1\cdot(0 + 1) = 1\cdot 1 = 1 = 0 + 1 = 1\cdot 0 + 1\cdot 1 \\
                      &1\cdot(1 + 0) = 1\cdot 1 = 1 = 1 + 0 = 1\cdot 1 + 1\cdot 0 \\
                      &1\cdot(1 + 1) = 1\cdot 0 = 0 = 1 + 1 = 1\cdot 1 + 1\cdot 1
                  \end{split}
              \]
              \[
                  \begin{split}
                      &(0 + 0)\cdot 0 = 0\cdot 0 = 0 = 0 + 0 = 0\cdot 0 + 0\cdot 0 \\
                      &(0 + 0)\cdot 1 = 0\cdot 1 = 0 = 0 + 0 = 0\cdot 1 + 0\cdot 1 \\
                      &(0 + 1)\cdot 0 = 1\cdot 0 = 0 = 0 + 0 = 0\cdot 0 + 1\cdot 0 \\
                      &(0 + 1)\cdot 1 = 1\cdot 1 = 1 = 0 + 1 = 0\cdot 1 + 1\cdot 1 \\
                      &(1 + 0)\cdot 0 = 1\cdot 0 = 0 = 0 + 0 = 1\cdot 0 + 0\cdot 0 \\
                      &(1 + 0)\cdot 1 = 1\cdot 1 = 1 = 1 + 0 = 1\cdot 1 + 0\cdot 1 \\
                      &(1 + 1)\cdot 0 = 0\cdot 0 = 0 = 0 + 0 = 1\cdot 0 + 1\cdot 0 \\
                      &(1 + 1)\cdot 1 = 0\cdot 1 = 0 = 1 + 1 = 1\cdot 1 + 1\cdot 1
                  \end{split}
              \]
        \item Multiplication has identity element
              \[
                  \begin{split}
                      &0\cdot 1 = 0 = 1\cdot 0 \\
                      &1\cdot 1 = 1
                  \end{split}
              \]
        \item Multiplication is commutative.
              \[
                  \begin{split}
                      &0\cdot 0 = 0 \\
                      &1\cdot 1 = 1 \\
                      &0\cdot 1 = 1\cdot 0 = 0
                  \end{split}
              \]
        \item Each non-zero element has multiplicative inverse.
              \[
                  1\cdot 1 = 1
              \]
    \end{enumerate}
\end{proof}

\begin{exercise}
    Prove that if two homogeneous systems of linear equations in two unknowns have the same solutions, then they are equivalent.
\end{exercise}

\begin{proof}
    A homogeneous linear equation always has trivial solution, which contains only zero.

    A homogeneous linear equation is called trivial if and only if all of its coefficients are zero.

    \begin{lemma}\label{lemma:exercise:solution-to-homogeneous-linear-equation}
        Let $(a_{1}, a_{2})\ne (0, 0)$, then solutions to $a_{1}x_{1} + a_{2}x_{2} = 0$ are of the form $(k\cdot t_{1}, k\cdot t_{2})$, where $(t_{1}, t_{2})$ is solution other than $(0, 0)$ and $k$ is a scalar.
    \end{lemma}
    \begin{proof}[Proof of the Lemma]
        Let $t_{1} = -a_{2}$ and $t_{2} = a_{1}$, then $(t_{1}, t_{2})$ is a solution other than $(0, 0)$ of the equation. Without loss of generality, suppose that $t_{1}\ne 0$, then $a_{2}\ne 0$.

        Let $(y_{1}, y_{2})$ be a solution to $a_{1}x_{1} + a_{2}x_{2} = 0$. Since $t_{1}\ne 0$, then there exists a scalar $k$ such that $k\cdot t_{1} = y_{1}$.
        \begin{align*}
                            & (a_{1}y_{1} + a_{2}y_{2}) - (a_{1}kt_{1} + a_{2}kt_{2}) = 0  \\
            \Leftrightarrow & (a_{1}kt_{1} + a_{2}y_{2}) - (a_{1}kt_{1} + a_{2}kt_{2}) = 0 \\
            \Leftrightarrow & a_{2}y_{2} - a_{2}kt_{2} = 0                                 \\
            \Leftrightarrow & y_{2} = k\cdot t_{2} \qquad\text{(since $a_{2}\ne 0$)}
        \end{align*}
        Hence $y_{1} = kt_{1}, y_{2} = kt_{2}$.
    \end{proof}

    \begin{lemma}\label{lemma:exercise:equivalent-homogeneous-linear-equations}
        Two equations $a_{1}x_{1} + a_{2}x_{2} = 0$ and $b_{1}x_{1} + b_{2}x_{2} = 0$ have the same solution if and only if there exists two non-zero scalars $a, b$ such that
        \[
            \begin{split}
                a(a_{1}x_{1} + a_{2}x_{2}) = b_{1}x_{1} + b_{2}x_{2} \\
                b(b_{1}x_{1} + b_{2}x_{2}) = a_{1}x_{1} + a_{2}x_{2}
            \end{split}
        \]
    \end{lemma}
    \begin{proof}[Proof of the Lemma]
        ($\Rightarrow$) If there exists such scalars, then solutions to $a_{1}x_{1} + a_{2}x_{2} = 0$ are solutions to $b_{1}x_{1} + b_{2}x_{2} = 0$ and vice versa.

        ($\Leftarrow$) A homogeneous linear equation always has non-trivial solution (solution other than zeroes).

        Let's consider two cases
        \begin{enumerate}[label = \textbf{Case \arabic*.}, itemindent=1cm]
            \item Any pair of scalars is a solution to both systems.
                  This means $a_{1} = a_{2} = b_{1} = b_{2} = 0$. Hence
                  \[
                      \begin{split}
                          1\cdot (a_{1}x_{1} + a_{2}x_{2}) = b_{1}x_{1} + b_{2}x_{2} \\
                          1\cdot (b_{1}x_{1} + b_{2}x_{2}) = a_{1}x_{1} + a_{2}x_{2}
                      \end{split}
                  \]
            \item (Due to the previous lemma) Otherwise, solutions to both system are of the form $(k\cdot t_{1}, k\cdot t_{2})$, where $(t_{1}, t_{2})\ne (0, 0)$.

                  $(-a_{2}, a_{1})$ is a non-trivial solution to $a_{1}x_{1} + a_{2}x_{2} = 0$.

                  $(-b_{2}, b_{1})$ is a non-trivial solution to $b_{1}x_{1} + b_{2}x_{2} = 0$.

                  $(t_{1}, t_{2})$ is a non-trivial solution to both equations, then there exists non-zero scalars $k, \ell$ such that
                  \[
                      \begin{split}
                          -a_{2} = kt_{1}, a_{1} = kt_{2} \\
                          -b_{2} = \ell t_{1}, b_{1} = \ell t_{2}
                      \end{split}
                  \]
                  So
                  \[
                      \begin{split}
                          a_{1} = k\ell^{-1}b_{1}, a_{2} = k\ell^{-1}b_{2} \\
                          b_{1} = k^{-1}\ell a_{1}, b_{2} = k^{-1}\ell a_{2}
                      \end{split}
                  \]
                  which implies
                  \[
                      \begin{split}
                          a_{1}x_{1} + a_{2}x_{2} = k\ell^{-1}(b_{1}x_{1} + b_{2}x_{2}) \\
                          b_{1}x_{1} + b_{2}x_{2} = k^{-1}\ell(a_{1}x_{1} + a_{2}x_{2})
                      \end{split}
                  \]
                  Hence two systems are equivalent.\qedhere
        \end{enumerate}
    \end{proof}

    \begin{lemma}\label{lemma:exercise:linear-combination-of-inequivalent-linear-equations}
        Two homogeneous linear equations $a_{1}x_{1} + a_{2}x_{2} = 0$ and $b_{1}x_{1} + b_{2}x_{2} = 0$ are inequivalent and non-trivial.

        Then any homogeneous linear equation in two unknowns $x_{1}, x_{2}$ is a linear combination of these two equations.
    \end{lemma}
    \begin{proof}
        Let's $c_{1}x_{1} + c_{2}x_{2} = 0$ be a homogeneous linear equation in two unknowns $x_{1}, x_{2}$.

        \[
            t_{1} = \frac{c_{1}b_{2} - c_{2}b_{1}}{a_{1}b_{2} - a_{2}b_{1}}\qquad t_{2} = \frac{a_{1}c_{2} - a_{2}c_{1}}{a_{1}b_{2} - a_{2}b_{1}}
        \]
        then $c_{1}x_{1} + c_{2}x_{2} = t_{1}(a_{1}x_{1} + a_{2}x_{2}) + t_{2}(b_{1}x_{1} + b_{2}x_{2})$.
    \end{proof}

    Let the two systems be
    \[
        \begin{cases}
            A_{11}x_{1} + A_{12}x_{2} = 0 \\
            A_{21}x_{1} + A_{22}x_{2} = 0 \\
            \vdots                        \\
            A_{m1}x_{1} + A_{m2}x_{2} = 0
        \end{cases}
        \quad\text{and}\quad
        \begin{cases}
            B_{11}x_{1} + B_{12}x_{2} = 0 \\
            B_{21}x_{1} + B_{22}x_{2} = 0 \\
            \vdots                        \\
            B_{n1}x_{1} + B_{n2}x_{2} = 0
        \end{cases}
    \]
    and suppose that (without loss of generality) $(A_{i1}, A_{i2})\ne (0, 0)$ and $(B_{j1}, B_{j2})\ne (0, 0)$.

    \begin{enumerate}[label = \textbf{Case \arabic*.}, itemindent=1cm]
        \item Any pairs of scalars is a solution to both systems.

              This implies that all coefficients of the two systems are zero, hence the two systems are equivalent.
        \item Solutions of the two systems are of the form $(kp_{1}, kp_{2})$, where $(p_{1}, p_{2})\ne (0, 0)$.

              Let $c_{1}, c_{2}, \ldots c_{m}$ be $m$ scalars
              \[
                  (c_{1}A_{11} + c_{2}A_{21} + \cdots + c_{m}A_{m1})x_{1} + (c_{1}A_{12} + c_{2}A_{22} + \cdots + c_{m}A_{m2})x_{2}
              \]
              is a linear combination of $A_{i1}x_{1} + A_{i2}x_{2}$ such that
              \[
                  (c_{1}A_{11} + c_{2}A_{21} + \cdots + c_{m}A_{m1}, c_{1}A_{12} + c_{2}A_{22} + \cdots + c_{m}A_{m2})\ne (0, 0).
              \]
              Solutions to $B_{i1}x_{1} + B_{i2}x_{2} = 0$ are solutions to
              \[
                  (c_{1}A_{11} + c_{2}A_{21} + \cdots + c_{m}A_{m1})x_{1} + (c_{1}A_{12} + c_{2}A_{22} + \cdots + c_{m}A_{m2})x_{2} = 0
              \]
              Then there exists non-zero scalar $k_{i}$ such that
              \[
                  \begin{split}
                      k_{i}B_{i1} = c_{1}A_{11} + c_{2}A_{21} + \cdots + c_{m}A_{m1} \\
                      k_{i}B_{i2} = c_{1}A_{12} + c_{2}A_{22} + \cdots + c_{m}A_{m2}
                  \end{split}
                  \qquad\Longrightarrow\qquad
                  \begin{split}
                      B_{i1} = k_{i}^{-1}\left(c_{1}A_{11} + c_{2}A_{21} + \cdots + c_{m}A_{m1}\right) \\
                      B_{i2} = k_{i}^{-1}\left(c_{1}A_{12} + c_{2}A_{22} + \cdots + c_{m}A_{m2}\right)
                  \end{split}
              \]
              Therefore $B_{i1}x_{1} + B_{i2}x_{2}$ is a linear combination of linear equations in the first system.

              Analogously, $A_{i1}x_{1} + B_{i2}x_{2}$ is a linear combination of linear equations in the second system.

        \item The only solution to both systems is $(0, 0)$.

              The assumption implies that
              \begin{itemize}
                  \item Within the first system, there are two linear equations which are non-trivial and inequivalent.

                        According to Lemma~\ref{lemma:exercise:linear-combination-of-inequivalent-linear-equations}, each equation of the second system is a linear combination of the two inequivalent equations of the first.
                  \item Within the second system, there are two linear equations which are non-trivial and inequivalent.

                        According to Lemma~\ref{lemma:exercise:linear-combination-of-inequivalent-linear-equations}, each equation of the first system is a linear combination of the two inequivalent equations of the second.
              \end{itemize}
    \end{enumerate}

    In conclusion, the two systems are equivalent.
\end{proof}

\begin{exercise}
    Prove that each subfield of the field of complex numbers contains every rational number.
\end{exercise}

\begin{proof}
    Each subfield of the field of complex numbers contains $0$ and $1$.

    Then $n = \underbrace{1 + 1 + \cdots + 1}_{n}$ belongs to the subfield, for any natural number $n$. Consequently, $-n$ belongs to the subfield.

    Let $q$ be a non-zero integer, $p$ be an integer, then $p, q$ belongs to the subfield.

    According to Theorem~\ref{thm:inverse-elements-of-subfield}, $q^{-1}$ belongs to the subfield, then $\frac{p}{q} = pq^{-1}$ belongs to the subfield.

    Thus, each subfield of $\mathbb{C}$ contains every rational number.
\end{proof}

\begin{exercise}
    Prove that each field of characteristic zero contains a copy of the rational number field.
\end{exercise}

\begin{note}
    The term ``copy'' is used informally. A copy of the rational number field is a field which is isomorphic to the rational number field. It means, there exists an isomorphism (isomorphism mapping) $f: \mathbb{Q}\to\mathbb{F}$ such that
    \[
        f(x) + f(y) = f(x + y)\qquad f(x)f(y) = f(xy).
    \]
    Such isomorphism is called a \textit{field isomorphism}.

    If there exists an isomorphism between two fields, then the two are called \textit{isomorphic}.
\end{note}

\begin{proof}
    Denote the field by $\mathbb{F}$.

    $\mathbb{F}$ has two distinct elements $0_{\mathbb{F}}$ and $1_{\mathbb{F}}$, which are the additive identity and multiplicative identity, respectively.

    Since $\text{Char}(\mathbb{F}) = 0$, then for any pair of distinct positive integers $(p, q)$
    \[
        \underbrace{1_{\mathbb{F}} + 1_{\mathbb{F}} + \cdots + 1_{\mathbb{F}}}_{p} \ne \underbrace{1_{\mathbb{F}} + 1_{\mathbb{F}} + \cdots + 1_{\mathbb{F}}}_{q}
    \]

    We construct a mapping as follows:
    \[
        \begin{split}
            f:&\quad\mathbb{Q} \to \mathbb{F} \\
            f:&\quad 0 \mapsto 0_{\mathbb{F}} \\
            &\quad 1 \mapsto 1_{\mathbb{F}} \\
            &\quad n \mapsto \underbrace{1_{\mathbb{F}} + 1_{\mathbb{F}} + \cdots + 1_{\mathbb{F}}}_{n} \\
            &\quad -n \mapsto \underbrace{(-1_{\mathbb{F}}) + (-1_{\mathbb{F}}) + \cdots + (-1_{\mathbb{F}})}_{n} \\
            &\quad \frac{p}{q} \mapsto \frac{\underbrace{1_{\mathbb{F}} + 1_{\mathbb{F}} + \cdots + 1_{\mathbb{F}}}_{p}}{\underbrace{1_{\mathbb{F}} + 1_{\mathbb{F}} + \cdots + 1_{\mathbb{F}}}_{q}} \\
            &\quad \frac{-p}{q} \mapsto \frac{\underbrace{(-1_{\mathbb{F}}) + (-1_{\mathbb{F}}) + \cdots + (-1_{\mathbb{F}})}_{p}}{\underbrace{1_{\mathbb{F}} + 1_{\mathbb{F}} + \cdots + 1_{\mathbb{F}}}_{q}}
        \end{split}
    \]

    $f$ is a monomorphism (different ``inputs'' will be mapped to different ``outputs''). Hence $\mathbb{Q}$ and $f(\mathbb{Q})$ are isomorphic. By the definition of $f$, $f(\mathbb{Q})\subseteq\mathbb{F}$.

    Hence, each field of characteristic zero contains a subfield which is isomorphic to the field of the rational numbers.
\end{proof}

\section{Matrices and Elementary Row Operations}

\setcounter{exercise}{0}

We use the following elementary row operations
\begin{enumerate}[label={(\arabic*)\ =}]
    \item Multiply a row with a non-zero scalar.
    \item Let $c$ be a scalar. Add $c$ times $s$-th row to $r$-th row, where $r\ne s$.
    \item Swap two rows.
\end{enumerate}

\begin{exercise}
    Find all solutions to the system of equations
    \[
        \begin{cases}
            (1 - i)x_{1} - i x_{2} = 0, \\
            2x_{1} + (1 - i) x_{2} = 0.
        \end{cases}
    \]
\end{exercise}

\begin{proof}
    \begingroup{}
    \allowdisplaybreaks{}
    \begin{align*}
        \begin{bmatrix}
            1 - i & -i    \\
            2     & 1 - i
        \end{bmatrix}
        \stackrel{(1)}{\rightarrow}
        \begin{bmatrix}
            2 & (1 + i)(-i) \\
            2 & 1 - i       \\
        \end{bmatrix}
        \stackrel{(2)}{\rightarrow}
        \begin{bmatrix}
            2 & 1 - i \\
            0 & 0
        \end{bmatrix}
        \stackrel{(1)}{\rightarrow}
        \begin{bmatrix}
            1 & \frac{1 - i}{2} \\
            0 & 0
        \end{bmatrix}
    \end{align*}
    \endgroup{}

    So the original system of linear equations is equivalent to
    \[
        x_{1} - \frac{i - 1}{2}x_{2} = 0.
    \]

    Thus all solutions of the system of linear equations are
    \[
        (x_{1}, x_{2}) = \left( \frac{(i - 1)c}{2}, c \right).\qedhere
    \]
\end{proof}

\begin{exercise}
    If
    \[
        A =
        \begin{bmatrix}
            3 & -1 & 2 \\
            2 & 1  & 1 \\
            1 & -3 & 0
        \end{bmatrix}
    \]

    find all solutions of $AX = 0$ by row-reducing $A$.
\end{exercise}

\begin{proof}
    \begingroup{}
    \allowdisplaybreaks{}
    \begin{multline*}
        \begin{bmatrix}
            3 & -1 & 2 \\
            2 & 1  & 1 \\
            1 & -3 & 0
        \end{bmatrix}
        \stackrel{(2)}{\rightarrow}
        \begin{bmatrix}
            0 & 8  & 2 \\
            0 & 7  & 1 \\
            1 & -3 & 0
        \end{bmatrix} \\
        \stackrel{(1)}{\rightarrow}
        \begin{bmatrix}
            0 & 1  & \frac{1}{4} \\
            0 & 7  & 1           \\
            1 & -3 & 0
        \end{bmatrix}
        \stackrel{(2)}{\rightarrow}
        \begin{bmatrix}
            0 & 1 & \frac{1}{4}  \\
            0 & 0 & \frac{-3}{4} \\
            1 & 0 & \frac{3}{4}
        \end{bmatrix} \\
        \stackrel{(2)}{\rightarrow}
        \begin{bmatrix}
            0 & 1 & 0            \\
            0 & 0 & \frac{-3}{4} \\
            1 & 0 & 0
        \end{bmatrix}
        \stackrel{(1)}{\rightarrow}
        \begin{bmatrix}
            0 & 1 & 0 \\
            0 & 0 & 1 \\
            1 & 0 & 0
        \end{bmatrix}
    \end{multline*}
    \endgroup{}

    Hence $AX = 0$ is equivalent to $\begin{bmatrix}0 & 1 & 0 \\ 0 & 0 & 1 \\ 1 & 0 & 0\end{bmatrix}\begin{bmatrix}x_{1} \\ x_{2} \\ x_{3}\end{bmatrix} = \begin{bmatrix}0 \\ 0 \\ 0\end{bmatrix}$.

    Thus, the solution of $AX = 0$ is $(x_{1}, x_{2}, x_{3}) = (0, 0, 0)$.
\end{proof}

\begin{exercise}
    If
    \[
        A =
        \begin{bmatrix}
            6  & -4 & 0  \\
            4  & -2 & 0  \\
            -1 & 0  & -3
        \end{bmatrix}
    \]

    find all solutions of $AX = 2X$ and all solutions of $AX = 3X$. (The symbol $cX$ denotes the matrix each entry of which is $c$ times the corresponding entry of $X$.)
\end{exercise}

\begin{proof}
    $AX = 2X$ is equivalent to
    \begin{align*}
        \begin{bmatrix}
            6  & -4 & 0  \\
            4  & -2 & 0  \\
            -1 & 0  & -3
        \end{bmatrix}
        \begin{bmatrix}
            x_{1} \\ x_{2} \\ x_{3}
        \end{bmatrix}
        =
        \begin{bmatrix}
            2x_{1} \\ 2x_{2} \\ 2x_{3}
        \end{bmatrix}
        \Longleftrightarrow
        \begin{bmatrix}
            6x_{1} + (-4)x_{2} \\ 4x_{1} + (-2)x_{2} \\ (-1)x_{1} + (-3)x_{3}
        \end{bmatrix}
        =
        \begin{bmatrix}
            2x_{1} \\ 2x_{2} \\ 2x_{3}
        \end{bmatrix}
        \Longleftrightarrow
        \begin{bmatrix}
            4x_{1} + (-4)x_{2} \\
            4x_{1} + (-4)x_{2} \\
            (-1)x_{1} + (-5)x_{3}
        \end{bmatrix}
        =
        \begin{bmatrix}
            0 \\ 0 \\ 0
        \end{bmatrix}.
    \end{align*}

    Thus, all solutions of $AX = 2X$ are $(x_{1}, x_{2}, x_{3}) = (c, c, -5c)$.

    \bigskip\hrule\bigskip

    $AX = 3X$ is equivalent to
    \begin{align*}
        \begin{bmatrix}
            6  & -4 & 0  \\
            4  & -2 & 0  \\
            -1 & 0  & -3
        \end{bmatrix}
        \begin{bmatrix}
            x_{1} \\ x_{2} \\ x_{3}
        \end{bmatrix}
        =
        \begin{bmatrix}
            3x_{1} \\ 3x_{2} \\ 3x_{3}
        \end{bmatrix}
        \Longleftrightarrow
        \begin{bmatrix}
            6x_{1} + (-4)x_{2} \\ 4x_{1} + (-2)x_{2} \\ (-1)x_{1} + (-3)x_{3}
        \end{bmatrix}
        =
        \begin{bmatrix}
            3x_{1} \\ 3x_{2} \\ 3x_{3}
        \end{bmatrix}
        \Longleftrightarrow
        \begin{bmatrix}
            3x_{1} + (-4)x_{2} \\
            4x_{1} + (-5)x_{2} \\
            (-1)x_{1} + (-6)x_{3}
        \end{bmatrix}
        =
        \begin{bmatrix}
            0 \\ 0 \\ 0
        \end{bmatrix}.
    \end{align*}

    Thus, all solutions of $AX = 3X$ are $(x_{1}, x_{2}, x_{3}) = (0, 0, 0)$.
\end{proof}

\begin{exercise}
    Find a row-reduced matrix which is row-equivalent to
    \[
        A =
        \begin{bmatrix}
            i & -(1 + i) & 0  \\
            1 & -2       & 1  \\
            1 & 2i       & -1
        \end{bmatrix}.
    \]
\end{exercise}

\begin{proof}
    \begingroup{}
    \allowdisplaybreaks{}
    \begin{align*}
        \begin{bmatrix}
            i & -(1 + i) & 0  \\
            1 & -2       & 1  \\
            1 & 2i       & -1
        \end{bmatrix}
        \stackrel{(2)}{\rightarrow}
        \begin{bmatrix}
            0 & 1 - i   & i  \\
            0 & -2 - 2i & -2 \\
            1 & 2i      & -1
        \end{bmatrix}
        \stackrel{(1)}{\rightarrow}
        \begin{bmatrix}
            0 & 2     & i - 1 \\
            0 & 1 + i & 1     \\
            1 & 2i    & -1
        \end{bmatrix}
        \stackrel{(1)}{\rightarrow}
        \begin{bmatrix}
            0 & 1  & \frac{i - 1}{2} \\
            0 & 1  & \frac{1 - i}{2} \\
            1 & 2i & -1
        \end{bmatrix} \\
        \stackrel{(2)}{\rightarrow}
        \begin{bmatrix}
            0 & 0  & i - 1           \\
            0 & 1  & \frac{1 - i}{2} \\
            1 & 2i & -1
        \end{bmatrix}
        \stackrel{(1)}{\rightarrow}
        \begin{bmatrix}
            0 & 0  & 1               \\
            0 & 1  & \frac{1 - i}{2} \\
            1 & 2i & -1
        \end{bmatrix}
        \stackrel{(2)}{\rightarrow}
        \begin{bmatrix}
            0 & 0  & 1 \\
            0 & 1  & 0 \\
            1 & 2i & 0
        \end{bmatrix}
        \stackrel{(2)}{\rightarrow}
        \begin{bmatrix}
            0 & 0 & 1 \\
            0 & 1 & 0 \\
            1 & 0 & 0
        \end{bmatrix}.
    \end{align*}
    \endgroup{}
\end{proof}

\begin{exercise}
    Prove that the following two matrices are not row-equivalent.
    \[
        \begin{bmatrix}
            2 & 0  & 0 \\
            a & -1 & 0 \\
            b & c  & 3
        \end{bmatrix},\quad
        \begin{bmatrix}
            1  & 1 & 2  \\
            -2 & 0 & -1 \\
            1  & 3 & 5
        \end{bmatrix}.
    \]
\end{exercise}

\begin{proof}
    Let's find a row-reduced matrix which is row-equivalent to the second matrix.
    \[
        \begin{bmatrix}
            1  & 1 & 2  \\
            -2 & 0 & -1 \\
            1  & 3 & 5
        \end{bmatrix}
        \stackrel{(2)}{\rightarrow}
        \begin{bmatrix}
            1 & 1 & 2 \\
            0 & 2 & 3 \\
            0 & 2 & 3
        \end{bmatrix}
        \stackrel{(2)}{\rightarrow}
        \begin{bmatrix}
            1 & 1 & 2 \\
            0 & 2 & 3 \\
            0 & 0 & 0
        \end{bmatrix}
        \stackrel{(2)}{\rightarrow}
        \begin{bmatrix}
            1 & 1 & 2           \\
            0 & 1 & \frac{3}{2} \\
            0 & 0 & 0
        \end{bmatrix}
        \stackrel{(2)}{\rightarrow}
        \begin{bmatrix}
            1 & 0 & \frac{1}{2} \\
            0 & 1 & \frac{3}{2} \\
            0 & 0 & 0
        \end{bmatrix}
    \]

    Since $(2, 0, 0)$ is NOT a linear combination of $(1, 0, \frac{1}{2})$ and $(0, 1, \frac{3}{2})$, the two matrices are not row-equivalent.
\end{proof}

\begin{exercise}
    Let
    \[
        A =
        \begin{bmatrix}
            a & b \\
            c & d
        \end{bmatrix}
    \]

    be a $2\times 2$ matrix with complex entries. Suppose that $A$ is row-reduced and also that $a + b + c + d = 0$. Prove that there are exactly three such matrices.
\end{exercise}

\begin{proof}
    The zero matrix $\begin{bmatrix}0 & 0 \\ 0 & 0\end{bmatrix}$ is one such matrix.

    Suppose that $A$ is not the zero matrix, then $a = 1$ or $c = 1$.

    If $a = 1$, then $c = 0$ (since $A$ is row-reduced). Furthermore
    \begin{itemize}
        \item $d = 0$. It follows that $b = -1$ (since $a + b + c + d = 0$).
        \item $d = 1$. It follows that $b = 0$. But $a + b + c + d\ne 0$.
    \end{itemize}

    If $c = 1$, then $a = 0$ (since $A$ is row-reduced). Furthermore
    \begin{itemize}
        \item $b = 0$. It follows that $d = -1$ (since $a + b + c + d = 0$).
        \item $b = 1$. It follows that $d = 0$. But $a + b + c + d\ne 0$.
    \end{itemize}

    Thus, the following three matrices are the only that satisfy the conditions
    \[
        \begin{bmatrix}
            0 & 0 \\
            0 & 0
        \end{bmatrix},\quad
        \begin{bmatrix}
            1 & -1 \\
            0 & 0
        \end{bmatrix},\quad
        \begin{bmatrix}
            0 & 0  \\
            1 & -1
        \end{bmatrix}.
    \]
\end{proof}

\begin{exercise}
    Prove that the interchange of two rows of a matrix can be accomplished by a finite sequence of elementary row operations of the other two types.
\end{exercise}

\begin{proof}
    If two rows of a matrix are identical, then after zero elementary row operation, we ``swap them''.

    Otherwise, two rows are not identical. Let's denote them by $r$ and $s$.

    \begin{itemize}
        \item By the 1st type of operation, we obtain $(r, s) \to (-r, s)$.
        \item By the 2nd type of operation, we obtain $(-r, s) \to (-r+s, s)$.
        \item By the 2nd type of operation, we obtain $(-r+s, s) \to (-r+s, s - (-r+s)) = (-r+s, r)$.
        \item By the 2nd type of operation, we obtain $(-r+s, r) \to (-r+s+r, r) = (s, r)$.
    \end{itemize}

    Hence, we swap two distinct rows by using only the 1st and 2nd elementary row operation.
\end{proof}

\begin{exercise}
    Consider the system of equations $AX = 0$ where
    \[
        A =
        \begin{bmatrix}
            a & b \\
            c & d
        \end{bmatrix}
    \]

    is a $2\times 2$ matrix over the field $\mathbb{F}$. Prove the following.
    \begin{enumerate}[label={(\alph*)}]
        \item If every entry of $A$ is 0, then every pair $(x_{1}, x_{2})$ is a solution of $AX = 0$.
        \item If $ad - bc \ne 0$, the system $AX = 0$ has the only trivial solution $x_{1} = x_{2} = 0$.
        \item If $ad - bc = 0$ and some entry of $A$ is different from 0, then there is a solution $({x}^{0}_{1}, {x}^{0}_{2})$ such that $(x_{1}, x_{2})$ is a solution if and only if there exists a scalar $y$ such that $x_{1} = y{x}^{0}_{1}, x_{2} = y{x}^{0}_{2}$.
    \end{enumerate}
\end{exercise}

\begin{proof}
    \begin{enumerate}[label={(\alph*)}]
        \item $a = b = c = d = 0$, then $AX = 0$ is equivalent to
              \[
                  \begin{cases}
                      0x_{1} + 0x_{2} = 0, \\
                      0x_{1} + 0x_{2} = 0.
                  \end{cases}
              \]

              Hence every pair $(x_{1}, x_{2})$ is a solution of $AX = 0$.
        \item $AX = 0$ is equivalent to
              \[
                  \begin{cases}
                      a{x}_{1} + b{x}_{2} = 0 \\
                      c{x}_{1} + d{x}_{2} = 0
                  \end{cases}.
              \]

              there are these two particular linear combinations
              \[
                  \begin{cases}
                      d(a{x}_{1} + b{x}_{2}) - b(c{x}_{1} + d{x}_{2}) = 0 \\
                      c(a{x}_{1} + b{x}_{2}) - a(c{x}_{1} + d{x}_{2}) = 0
                  \end{cases}
                  \Longleftrightarrow
                  \begin{cases}
                      (ad - bc)x_{1} = 0 \\
                      (bc - ad)x_{2} = 0
                  \end{cases}.
              \]

              Since $ad - bc\ne 0$, we obtain that $x_{1} = x_{2} = 0$. This is the only solution.
        \item Without loss of generality, suppose that $a\ne 0$. Since $ad = bc$,
              \[
                  \begin{bmatrix}
                      a & b \\
                      c & d
                  \end{bmatrix}
                  \stackrel{(1)}{\rightarrow}
                  \begin{bmatrix}
                      a  & b  \\
                      ac & ad
                  \end{bmatrix}
                  \rightarrow
                  \begin{bmatrix}
                      a  & b  \\
                      ac & bc
                  \end{bmatrix}
                  \stackrel{(2)}{\rightarrow}
                  \begin{bmatrix}
                      a & b \\
                      0 & 0
                  \end{bmatrix}
              \]

              Since $a\ne 0$, then $AX = 0$ has a non trivial solution $({x}^{0}_{1}, {x}^{0}_{2}) = (-b, a)$.

              Suppose that $(x_{1}, x_{2})$ is a solution of $AX = 0$. Since $a\ne 0$, then there exists $y$ such that $x_{2} = ay = y{x}^{0}_{2}$.
              \begin{align*}
                  0 & = a{x}_{1} + b{x}_{2}                                   \\
                    & = a({x}_{1} - y{x}^{0}_{1}) + b({x}_{2} - y{x}^{0}_{2}) \\
                    & = a({x}_{1} - y{x}^{0}_{1}) + 0                         \\
                    & = a({x}_{1} - y{x}^{0}_{1}).
              \end{align*}

              Since $a\ne 0$, ${x}_{1} = y{x}^{0}_{1}$.

              Hence, there exists a scalar $y$ such that $x_{1} = y{x}^{0}_{1}, x_{2} = y{x}^{0}_{2}$, where $({x}^{0}_{1}, {x}^{0}_{2})$ is a non-trivial solution of $AX = 0$ and $(x_{1}, y_{1})$ is an arbitrary solution of $AX = 0$.
    \end{enumerate}
\end{proof}

\section{Row-Reduced Echelon Matrices}

\setcounter{exercise}{0}

\begin{exercise}
    Find all solutions to the following system of equations by row-reducing the coefficient matrix:
    \[
        \begin{array}{ccccccc}
            \frac{1}{3}{x}_{1}  & + & 2{x}_{2} & - & 6{x}_{3}         & = & 0  \\
            -4{x}_{1}           &   &          & + & 5{x}_{3}         & = & 0  \\
            -3{x}_{1}           & + & 6{x}_{2} & - & 13{x}_{3}        & = & 0  \\
            -\frac{7}{3}{x}_{1} & + & 2{x}_{2} & - & \frac{8}{3}x_{3} & = & 0.
        \end{array}
    \]
\end{exercise}

\begin{proof}
    The system of linear equations is equivalent to
    \[
        \begin{bmatrix}
            \frac{1}{3}  & 2 & -6           \\
            -4           & 0 & 5            \\
            -3           & 6 & -13          \\
            -\frac{7}{3} & 2 & -\frac{8}{3}
        \end{bmatrix}
        \begin{bmatrix}
            x_{1} \\
            x_{2} \\
            x_{3}
        \end{bmatrix}
        =
        \begin{bmatrix}
            0 \\
            0 \\
            0
        \end{bmatrix}.
    \]

    Let's perform elementary-row operation to obtain a row-reduce echelon form.
    \begingroup{}
    \allowdisplaybreaks{}
    \begin{align*}
        \begin{bmatrix}
            \frac{1}{3}  & 2 & -6           \\
            -4           & 0 & 5            \\
            -3           & 6 & -13          \\
            -\frac{7}{3} & 2 & -\frac{8}{3}
        \end{bmatrix}
        \stackrel{(1)}{\rightarrow}
        \begin{bmatrix}
            1            & 6 & -18          \\
            -4           & 0 & 5            \\
            -3           & 6 & -13          \\
            -\frac{7}{3} & 2 & -\frac{8}{3}
        \end{bmatrix}
        \stackrel{(2)}{\rightarrow}
        \begin{bmatrix}
            1 & 6  & -18            \\
            0 & 24 & -67            \\
            0 & 24 & -67            \\
            0 & 16 & -\frac{134}{3}
        \end{bmatrix}
        \stackrel{(2)}{\rightarrow}
        \begin{bmatrix}
            1 & 6  & -18 \\
            0 & 24 & -67 \\
            0 & 24 & -67 \\
            0 & 24 & -67
        \end{bmatrix} \\
        \stackrel{(2)}{\rightarrow}
        \begin{bmatrix}
            1 & 6  & -18 \\
            0 & 24 & -67 \\
            0 & 0  & 0   \\
            0 & 0  & 0
        \end{bmatrix}
        \stackrel{(2)}{\rightarrow}
        \begin{bmatrix}
            1 & 0  & -\frac{5}{4} \\
            0 & 24 & -67          \\
            0 & 0  & 0            \\
            0 & 0  & 0
        \end{bmatrix}
        \stackrel{(1)}{\rightarrow}
        \begin{bmatrix}
            1 & 0 & -\frac{5}{4}   \\
            0 & 1 & -\frac{67}{24} \\
            0 & 0 & 0              \\
            0 & 0 & 0
        \end{bmatrix}
    \end{align*}
    \endgroup{}

    Thus, all solutions are of the form $(\frac{5}{4}c, \frac{67}{24}c, c)$.
\end{proof}

\begin{exercise}
    Find a row-reduced echelon matrix which is row-equivalent to
    \[
        A =
        \begin{bmatrix}
            1 & -i    \\
            2 & 2     \\
            i & 1 + i
        \end{bmatrix}.
    \]

    What are the solutions of $AX = 0$.
\end{exercise}

\begin{proof}
    \begingroup{}
    \allowdisplaybreaks{}
    \begin{align*}
        \begin{bmatrix}
            1   & -i{}    \\
            2   & 2       \\
            i{} & 1 + i{}
        \end{bmatrix}
        \stackrel{(1)}{\rightarrow}
        \begin{bmatrix}
            1   & -{i}    \\
            1   & 1       \\
            {i} & 1 + {i}
        \end{bmatrix}
        \stackrel{(2)}{\rightarrow}
        \begin{bmatrix}
            1 & -{i} \\
            1 & 1    \\
            0 & 1
        \end{bmatrix}
        \stackrel{(2)}{\rightarrow}
        \begin{bmatrix}
            1 & -{i}    \\
            0 & 1 + {i} \\
            0 & 1
        \end{bmatrix}
        \stackrel{(1)}{\rightarrow}
        \begin{bmatrix}
            1 & -{i} \\
            0 & 1    \\
            0 & 0
        \end{bmatrix}
        \stackrel{(2)}{\rightarrow}
        \begin{bmatrix}
            1 & 0 \\
            0 & 1 \\
            0 & 0
        \end{bmatrix}.
    \end{align*}
    \endgroup{}

    So the row-reduced echelon form of $A$ is
    \[
        \begin{bmatrix}
            1 & 0 \\
            0 & 1 \\
            0 & 0
        \end{bmatrix}
    \]

    The solution of $AX = 0$ is $x_{1} = x_{2} = 0$.
\end{proof}

\begin{exercise}
    Describe explicitly all $2\times 2$ row-reduced echelon matrices.
\end{exercise}

\begin{proof}
    One of the row-reduced echelon matrix of this kind is the $2\times 2$ zero matrix.

    If a $2\times 2$ row-reduced echelon matrix is non-zero, then there are the following possibilities
    \begin{itemize}
        \item The second row is zero.

              In this case, the 1st non-zero entry of the 1st row is either $a_{11}$ or $a_{12}$.

              If $a_{11}$ is non-zero, then $a_{11} = 1$ and $a_{12}$ can have any value. If $a_{11}$ is zero, then $a_{12} = 1$.
        \item The second row is non-zero.

              In this case, $a_{11} = 1$, $a_{12} = 0$, $a_{21} = 0$, $a_{22} = 0$.
    \end{itemize}

    Thus, all $2\times 2$ row-reduced echelon matrices are
    \begin{itemize}
        \item The zero matrix.
        \item $\begin{bmatrix}1 & c \\ 0 & 0\end{bmatrix}$, where $c$ is any element of the field $\mathbb{F}$.
        \item $\begin{bmatrix}0 & 1 \\ 0 & 0\end{bmatrix}$.
        \item The identity matrix.
    \end{itemize}
\end{proof}

\begin{exercise}
    Consider the system of equations
    \begingroup{}
    \allowdisplaybreaks{}
    \begin{align*}
        \begin{array}{ccccccc}
            x_{1}    & - & x_{2}    & + & 2{x}_{3} & = & 1  \\
            2{x}_{1} &   &          & + & 2x_{3}   & = & 1  \\
            x_{1}    & - & 3{x}_{2} & + & 4{x}_{3} & = & 2.
        \end{array}
    \end{align*}
    \endgroup{}

    Does this system have a solution? If so, describe explicitly all solutions.
\end{exercise}

\begin{proof}
    \begingroup{}
    \allowdisplaybreaks{}
    \begin{align*}
        \left[\begin{array}{ccc|c}
                      1 & -1 & 2 & 1 \\
                      2 & 0  & 2 & 1 \\
                      1 & -3 & 4 & 2
                  \end{array}\right]
        \stackrel{(1)}{\rightarrow}
        \left[\begin{array}{ccc|c}
                      1 & -1 & 2 & 1           \\
                      1 & 0  & 1 & \frac{1}{2} \\
                      1 & -3 & 4 & 2
                  \end{array}\right]
        \stackrel{(2)}{\rightarrow}
        \left[\begin{array}{ccc|c}
                      1 & -1 & 2  & 1            \\
                      0 & 1  & -1 & \frac{-1}{2} \\
                      0 & -2 & 2  & 1
                  \end{array}\right]
        \stackrel{(2)}{\rightarrow}
        \left[\begin{array}{ccc|c}
                      1 & -1 & 2  & 1            \\
                      0 & 1  & -1 & \frac{-1}{2} \\
                      0 & 0  & 0  & 0
                  \end{array}\right]
        \stackrel{(2)}{\rightarrow}
        \left[\begin{array}{ccc|c}
                      1 & 0 & 1  & \frac{1}{2}  \\
                      0 & 1 & -1 & \frac{-1}{2} \\
                      0 & 0 & 0  & 0
                  \end{array}\right]
    \end{align*}
    \endgroup{}

    So the system does has a solution.

    All solutions are of the form $(\frac{1}{2} - c, \frac{-1}{2} + c , c)$.
\end{proof}

\begin{exercise}
    Give an example of a system of two linear equations in two unknowns which has no solution.
\end{exercise}

\begin{proof}
    An example is
    \[
        \begin{array}{ccccc}
            x_{1} & + & x_{2} & = & 1  \\
            x_{1} & + & x_{2} & = & 0.
        \end{array}\qedhere
    \]
\end{proof}

\begin{exercise}
    Show that the system
    \[
        \begin{array}{cccccccccc}
            x_{1} & - & 2{x}_{2} & + & x_{3}    & + & 2{x}_{4} & = & 1 \\
            x_{1} & + & x_{2}    & - & x_{3}    & + & x_{4}    & = & 2 \\
            x_{1} & + & 7{x}_{2} & - & 5{x}_{3} & - & x_{4}    & = & 3
        \end{array}
    \]

    has no solution.
\end{exercise}

\begin{proof}
    \begingroup{}
    \allowdisplaybreaks{}
    \begin{align*}
        \left[\begin{array}{cccc|c}
                      1 & -2 & 1  & 2  & 1 \\
                      1 & 1  & -1 & 1  & 2 \\
                      1 & 7  & -5 & -1 & 3
                  \end{array}\right]
        \stackrel{(2)}{\rightarrow}
        \left[\begin{array}{cccc|c}
                      1 & -2 & 1  & 2  & 1 \\
                      0 & 3  & -2 & -1 & 1 \\
                      0 & 9  & -6 & -3 & 2
                  \end{array}\right]
        \stackrel{(2)}{\rightarrow}
        \left[\begin{array}{cccc|c}
                      1 & -2 & 1  & 2  & 1  \\
                      0 & 3  & -2 & -1 & 1  \\
                      0 & 0  & 0  & 0  & -1
                  \end{array}\right] \\
        \stackrel{(1)}{\rightarrow}
        \left[\begin{array}{cccc|c}
                      1 & -2 & 1            & 2            & 1           \\
                      0 & 1  & \frac{-2}{3} & \frac{-1}{3} & \frac{1}{3} \\
                      0 & 0  & 0            & 0            & -1
                  \end{array}\right]
        \stackrel{(2)}{\rightarrow}
        \left[\begin{array}{cccc|c}
                      1 & 0 & \frac{-1}{3} & \frac{4}{3}  & \frac{5}{3} \\
                      0 & 1 & \frac{-2}{3} & \frac{-1}{3} & \frac{1}{3} \\
                      0 & 0 & 0            & 0            & -1
                  \end{array}\right].
    \end{align*}
    \endgroup{}

    The last row of the coefficient matrix is zero, but the last row of the augmented matrix is non-zero. Therefore, the system has no solution.
\end{proof}

\begin{exercise}
    Find all solutions of
    \[
        \begin{array}{ccccccccccc}
            2{x_{1}} & - & 3{x_{2}} & - & 7{x_{3}} & + & 5{x_{4}} & + & 2{x_{5}} & = & -2  \\
            x_{1}    & - & 2{x_{2}} & - & 4{x_{3}} & + & 3{x_{4}} & + & x_{5}    & = & -2  \\
            2{x_{1}} &   &          & - & 4{x_{3}} & + & 2{x_{4}} & + & x_{5}    & = & 3   \\
            x_{1}    & - & 5{x_{2}} & - & 7{x_{3}} & + & 6{x_{4}} & + & 2{x_{5}} & = & -7.
        \end{array}
    \]
\end{exercise}

\begin{proof}
    \begingroup{}
    \allowdisplaybreaks{}
    \begin{align*}
        \left[\begin{array}{ccccc|c}
                      2 & -3 & -7 & 5 & 2 & -2 \\
                      1 & -2 & -4 & 3 & 1 & -2 \\
                      2 & 0  & -4 & 2 & 1 & 3  \\
                      1 & -5 & -7 & 6 & 2 & -7
                  \end{array}\right]
        \stackrel{(3)}{\rightarrow}
        \left[\begin{array}{ccccc|c}
                      1 & -2 & -4 & 3 & 1 & -2 \\
                      1 & -5 & -7 & 6 & 2 & -7 \\
                      2 & -3 & -7 & 5 & 2 & -2 \\
                      2 & 0  & -4 & 2 & 1 & 3
                  \end{array}\right]
        \stackrel{(2)}{\rightarrow}
        \left[\begin{array}{ccccc|c}
                      1 & -2 & -4 & 3  & 1  & -2 \\
                      0 & -3 & -3 & 3  & 1  & -5 \\
                      0 & 1  & 1  & -1 & 0  & 2  \\
                      0 & 4  & 4  & -4 & -1 & 7
                  \end{array}\right] \\
        \stackrel{(3)}{\rightarrow}
        \left[\begin{array}{ccccc|c}
                      1 & -2 & -4 & 3  & 1  & -2 \\
                      0 & 1  & 1  & -1 & 0  & 2  \\
                      0 & -3 & -3 & 3  & 1  & -5 \\
                      0 & 4  & 4  & -4 & -1 & 7
                  \end{array}\right]
        \stackrel{(3)}{\rightarrow}
        \left[\begin{array}{ccccc|c}
                      1 & 0 & -2 & 1  & 1  & 2  \\
                      0 & 1 & 1  & -1 & 0  & 2  \\
                      0 & 0 & 0  & 0  & 1  & 1  \\
                      0 & 0 & 0  & 0  & -1 & -1
                  \end{array}\right]
        \stackrel{(3)}{\rightarrow}
        \left[\begin{array}{ccccc|c}
                      1 & 0 & -2 & 1  & 0 & 1 \\
                      0 & 1 & 1  & -1 & 0 & 2 \\
                      0 & 0 & 0  & 0  & 1 & 1 \\
                      0 & 0 & 0  & 0  & 0 & 0
                  \end{array}\right].
    \end{align*}
    \endgroup{}

    Thus, all solutions are $(x_{1}, x_{2}, x_{3}, x_{4}, x_{5}) = (1 - b + 2a, 2 + b - a , a, b, 1)$.
\end{proof}

\begin{exercise}
    Let
    \[
        A =
        \begin{bmatrix}
            3 & -1 & 2 \\
            2 & 1  & 1 \\
            1 & -3 & 0
        \end{bmatrix}.
    \]

    For which triples $(y_{1}, y_{2}, y_{3})$ does the system $AX = Y$ have a solution.
\end{exercise}

\begin{proof}
    \begingroup{}
    \allowdisplaybreaks{}
    \begin{align*}
        \left[\begin{array}{ccc|c}
                      3 & -1 & 2 & y_{1} \\
                      2 & 1  & 1 & y_{2} \\
                      1 & -3 & 0 & y_{3}
                  \end{array}\right]
        \stackrel{(3)}{\rightarrow}
        \left[\begin{array}{ccc|c}
                      1 & -3 & 0 & y_{3} \\
                      2 & 1  & 1 & y_{2} \\
                      3 & -1 & 2 & y_{1}
                  \end{array}\right]
        \stackrel{(2)}{\rightarrow}
        \left[\begin{array}{ccc|c}
                      1 & -3 & 0 & y_{3}            \\
                      0 & 7  & 1 & y_{2} - 2{y_{3}} \\
                      0 & 8  & 2 & y_{1} - 3{y_{3}}
                  \end{array}\right]
        \stackrel{(1)}{\rightarrow}
        \left[\begin{array}{ccc|c}
                      1 & -3 & 0           & y_{3}                                \\
                      0 & 1  & \frac{1}{7} & \frac{y_{2}}{7} - \frac{2{y_{3}}}{7} \\
                      0 & 1  & \frac{1}{4} & \frac{y_{1}}{8} - \frac{3{y_{3}}}{8}
                  \end{array}\right]                                                     \\
        \stackrel{(2)}{\rightarrow}
        \left[\begin{array}{ccc|c}
                      1 & -3 & 0            & y_{3}                                                   \\
                      0 & 1  & \frac{1}{7}  & \frac{y_{2}}{7} - \frac{2{y_{3}}}{7}                    \\
                      0 & 0  & \frac{3}{28} & \frac{y_{1}}{8} - \frac{y_{2}}{7} - \frac{5{y_{3}}}{56}\end{array}\right]
        \stackrel{(2)}{\rightarrow}
        \left[\begin{array}{ccc|c}
                      1 & -3 & 0           & y_{3}                                                        \\
                      0 & 1  & \frac{1}{7} & \frac{y_{2}}{7} - \frac{2{y_{3}}}{7}                         \\
                      0 & 0  & 1           & \frac{7{y_{1}}}{6} - \frac{4{y_{2}}}{3} - \frac{5{y_{3}}}{6}\end{array}\right]
        \stackrel{(2)}{\rightarrow}
        \left[\begin{array}{ccc|c}
                      1 & -3 & 0 & y_{3}                                                        \\
                      0 & 1  & 0 & -\frac{y_{1}}{6} + \frac{y_{2}}{3} - \frac{y_{3}}{6}         \\
                      0 & 0  & 1 & \frac{7{y_{1}}}{6} - \frac{4{y_{2}}}{3} - \frac{5{y_{3}}}{6}\end{array}\right] \\
        \stackrel{(2)}{\rightarrow}
        \left[\begin{array}{ccc|c}
                      1 & 0 & 0 & -\frac{y_{1}}{2} + y_{2} + \frac{y_{3}}{2}                   \\
                      0 & 1 & 0 & -\frac{y_{1}}{6} + \frac{y_{2}}{3} - \frac{y_{3}}{6}         \\
                      0 & 0 & 1 & \frac{7{y_{1}}}{6} - \frac{4{y_{2}}}{3} - \frac{5{y_{3}}}{6}\end{array}\right].
    \end{align*}
    \endgroup{}

    Hence the system has a solution for any triples $(y_{1}, y_{2}, y_{3})$.
\end{proof}

\begin{exercise}
    Let
    \[
        A =
        \begin{bmatrix}
            3  & -6 & 2 & -1 \\
            -2 & 4  & 1 & 3  \\
            0  & 0  & 1 & 1  \\
            1  & -2 & 1 & 0
        \end{bmatrix}.
    \]

    For which $(y_{1}, y_{2}, y_{3}, y_{4})$ does the system of equations $AX = Y$ have a solution?
\end{exercise}

\begin{proof}
    I am going to perform elementary-row operations on the augmented matrix.
    \begingroup{}
    \allowdisplaybreaks{}
    \begin{align*}
        \left[\begin{array}{cccc|c}
                      3  & -6 & 2 & -1 & y_{1} \\
                      -2 & 4  & 1 & 3  & y_{2} \\
                      0  & 0  & 1 & 1  & y_{3} \\
                      1  & -2 & 1 & 0  & y_{4}
                  \end{array}\right]
        \stackrel{(3)}{\rightarrow}
        \left[\begin{array}{cccc|c}
                      1  & -2 & 1 & 0  & y_{4} \\
                      -2 & 4  & 1 & 3  & y_{2} \\
                      3  & -6 & 2 & -1 & y_{1} \\
                      0  & 0  & 1 & 1  & y_{3}
                  \end{array}\right]
        \stackrel{(2)}{\rightarrow}
        \left[\begin{array}{cccc|c}
                      1  & -2 & 1 & 0 & y_{4}         \\
                      -2 & 4  & 1 & 3 & y_{2}         \\
                      1  & -2 & 3 & 2 & y_{1} + y_{2} \\
                      0  & 0  & 1 & 1 & y_{3}
                  \end{array}\right]                                                                       \\
        \stackrel{(2)}{\rightarrow}
        \left[\begin{array}{cccc|c}
                      1 & -2 & 1 & 0 & y_{4}                 \\
                      0 & 0  & 3 & 3 & y_{2} + 2{y_{4}}      \\
                      0 & 0  & 2 & 2 & y_{1} + y_{2} - y_{4} \\
                      0 & 0  & 1 & 1 & y_{3}
                  \end{array}\right]
        \stackrel{(2)}{\rightarrow}
        \left[\begin{array}{cccc|c}
                      1 & -2 & 1 & 0 & y_{4}                            \\
                      0 & 0  & 0 & 0 & y_{2} + 2{y_{4}} - 3{y_{3}}      \\
                      0 & 0  & 0 & 0 & y_{1} + y_{2} - 2{y_{3}} - y_{4} \\
                      0 & 0  & 1 & 1 & y_{3}
                  \end{array}\right]
        \stackrel{(3)}{\rightarrow}
        \left[\begin{array}{cccc|c}
                      1 & -2 & 1 & 0 & y_{4}                            \\
                      0 & 0  & 1 & 1 & y_{3}                            \\
                      0 & 0  & 0 & 0 & y_{2} + 2{y_{4}} - 3{y_{3}}      \\
                      0 & 0  & 0 & 0 & y_{1} + y_{2} - 2{y_{3}} - y_{4}
                  \end{array}\right] \\
        \stackrel{(2)}{\rightarrow}
        \left[\begin{array}{cccc|c}
                      1 & -2 & 0 & -1 & y_{4} - y_{3}                    \\
                      0 & 0  & 1 & 1  & y_{3}                            \\
                      0 & 0  & 0 & 0  & y_{2} - 3{y_{3}} + 2{y_{4}}      \\
                      0 & 0  & 0 & 0  & y_{1} + y_{2} - 2{y_{3}} - y_{4}
                  \end{array}\right]
    \end{align*}
    \endgroup{}

    Hence the system $AX = Y$ has a solution if and only if $y_{2} - 3{y_{3}} + 2{y_{4}} = y_{1} + y_{2} - 2{y_{3}} - y_{4} = 0$.
\end{proof}

\begin{exercise}
    Suppose $R$ and $R'$ are $m\times n$ row-reduced echelon matrices and that the systems $RX = 0$ and $R'X = 0$ have exactly the same solutions. Prove that $R = R'$.
\end{exercise}

\begin{proof}
    If $R$ is a zero matrix, then any $n$-tuple will be a solution of $RX = 0$. Otherwise, there exists an $n$-tuple which is not a solution of $RX = 0$ (suppose that in $i$-th row, the entry $a_{ij}$ is non-zero, then $(\ldots, x_{j}, \ldots) = (0, \ldots, 1, \ldots, 0)$ is not a solution of $RX = 0$).

    Hence, if $RX = 0$ and $R'X = 0$ have exactly the same solutions, then either they are both zero matrices, or non-zero matrices.

    $R = {(r_{i,j})}_{m\times n}$, $R' = {(r'_{i,j})}_{m\times n}$.

    Without loss of generality, suppose that the two matrices are non-zero and $R$ has $r$ non-zero rows, $R'$ has $r'$ non-zero rows.

    Before continuing, I would like to make two definitions (independent columns and dependent columns). All solutions of $RX = 0$ are obtained by
    \begin{itemize}
        \item assign arbitrary values to unknowns on the columns which do not contain a leading non-zero entry (let's call these independent columns),
        \item compute the rest of the unknowns (correspond to the rest of the columns, let's call these dependent columns) from the system $RX = 0$ (this is straightforward since $R$ is of row-reduced echelon form).
    \end{itemize}

    \begin{enumerate}[label={\textbf{Step \arabic*.}},itemindent=0.5cm]
        \item Prove that the leading non-zero entries of the 1st row of $R$ and the 1st row of $R'$ have the same column index.

              Let $A_{1,j_{1}}$ be the leading non-zero entry of the 1st row of $R$, $B_{1,k_{1}}$ be the leading non-zero entry of the 1st row of $R'$.

              Assume that $j_{1} < k_{1}$.

              In the matrix $R$, $j_{1}$-th column is a dependent column. Meanwhile, in matrix $R'$, $j_{1}$-th column is an independent column (due to the assumption). Therefore, we can pick a solution of $R'X = 0$ which is not a solution of $RX = 0$. This contradicts the hypothesis that the two systems have exactly the same solutions.

              Similarly, the assumption $j_{1} > k_{1}$ also leads to contradiction.

              Therefore $j_{1} = k_{1}$.
        \item Prove that the leading non-zero entries of the $i$-th row of $R$ and the $i$-th row of $R'$ have the same column index ($i \le \min\{ r, r' \}$).

              We apply the same method as the previous step to the 2nd row, then the 3rd row, etc until $\min\{ r, r' \}$.
        \item Prove that $r = r'$.

              Without loss of generality, assume that $r < r'$. Then $R$ has more independent columns than $R'$.

              Let the index of the column that contains the leading non-zero entry of the $r$-th row of $R$ and $R'$ be $x$.

              On the one hand, in $R'$, the column that contains the leading non-zero entry of the $(r+1)$-th row of $R'$ is a dependent column and has index $y$. Since $R'$ is of row-reduced echelon form, then $x < y$.

              On the other hand, because $x < y$, the $y$-th column of $R$ is an independent column.

              So we can pick a solution of $RX = 0$ which is not a solution of $R'X = 0$. This contradicts the hypothesis that the two systems have exactly the same solutions.

              Similarly, the assumption $r > r'$ also leads to contradiction.

              Therefore $r = r'$.
        \item Prove that $R = R'$.

              Due to the two previous steps, if $j$-th column of $R$ is independent then $j$-th column of $R'$ is also independent and vice versa.

              Let the indices of independent columns be $\ell_{1}, \ell_{2}, \ldots, \ell_{n-r}$. Since $x_{\ell_{1}}, x_{\ell_{2}}, \ldots, x_{\ell_{n-r}}$ can have any values and $RX = 0, R'X = 0$ have exactly the same solutions, then.

              Let $(x_{1}, x_{2}, \ldots, x_{n})$ be a common solution of $RX = 0$ and $R'X = 0$. Notice that $x_{\ell_{1}}, x_{\ell_{2}}, \ldots, x_{\ell_{n-r}}$ can have any values. Subtract the $i$-th equation of $R'X = 0$ from the $i$-th equation of $RX = 0$ (the unknowns corresponding to dependent columns will be eliminated, thanks to the 1st and 2nd step), we obtain that
              \[
                  \sum^{n-r}_{k=1}(r_{i,\ell_{k}} - r'_{i,\ell_{k}}) x_{\ell_{k}} = 0\qquad\forall x_{\ell_{1}}, x_{\ell_{2}}, \ldots, x_{\ell_{n-r}}.
              \]

              Pick solutions where $x_{\ell_{k}}$ is $1$ and the rest of the independent unknowns (corresponding to independent columns) are $0$, we obtain that $r_{i,\ell_{k}} = r'_{i,\ell_{k}}$ for every $i$ and $\ell_{k}$.

              Hence $R = R'$.\qedhere
    \end{enumerate}
\end{proof}

\section{Matrix Multiplication}

\setcounter{exercise}{0}

\begin{exercise}
    Let
    \[
        A = \begin{bmatrix}
            2 & -1 & 1 \\
            1 & 2  & 1
        \end{bmatrix},\quad
        B = \begin{bmatrix}
            3 \\
            1 \\
            -1
        \end{bmatrix},\quad
        C = \begin{bmatrix}
            1 & -1
        \end{bmatrix}.
    \]

    Compute $ABC$ and $CAB$.
\end{exercise}

\begin{proof}
    $ABC = (AB)C$ and $CAB = C(AB)$.
    \[
        AB = \begin{bmatrix}
            4 \\
            4
        \end{bmatrix}
    \]

    \[
        (AB)C = \begin{bmatrix}
            4 & -4 \\
            4 & -4
        \end{bmatrix},\qquad
        C(AB) = \begin{bmatrix} 0 \end{bmatrix}.
    \]
\end{proof}

\begin{exercise}
    Let
    \[
        A = \begin{bmatrix}
            1 & -1 & 1 \\
            2 & 0  & 1 \\
            3 & 0  & 1
        \end{bmatrix},\quad
        B = \begin{bmatrix}
            2 & -2 \\
            1 & 3  \\
            4 & 4
        \end{bmatrix}.
    \]

    Verify directly that $A(AB) = {A}^{2}B$.
\end{exercise}

\begin{proof}
    \[
        AB = \begin{bmatrix}
            5  & -1 \\
            8  & 0  \\
            10 & -2
        \end{bmatrix}
        \qquad
        A(AB) = \begin{bmatrix}
            7  & -3 \\
            20 & -4 \\
            25 & -5
        \end{bmatrix}.
    \]

    \[
        {A}^{2} = \begin{bmatrix}
            2 & -1 & 1 \\
            5 & -2 & 3 \\
            6 & -3 & 4
        \end{bmatrix}
        \qquad
        {A}^{2}B = \begin{bmatrix}
            7  & -3 \\
            20 & -4 \\
            25 & -5
        \end{bmatrix}.
    \]

    Thus $A(AB) = {A}^{2}B$.
\end{proof}

\begin{exercise}
    Find two different $2\times 2$ matrices $A$ such that ${A}^{2} = 0$ but $A\ne 0$.
\end{exercise}

\begin{proof}
    \[
        {\begin{bmatrix}
                    A_{11} & A_{12} \\
                    A_{21} & A_{22}
                \end{bmatrix}}^{2}
        =
        \begin{bmatrix}
            {A_{11}}^{2} + A_{12}A_{21} & A_{12}(A_{11} + A_{22})     \\
            A_{21}(A_{11} + A_{22})     & {A_{22}}^{2} + A_{12}A_{21}
        \end{bmatrix}.
    \]

    Two possible matrices are
    \[
        \begin{bmatrix}
            1  & 1  \\
            -1 & -1
        \end{bmatrix}\qquad
        \begin{bmatrix}
            -1 & -1 \\
            1  & 1
        \end{bmatrix}.
    \]
\end{proof}

\begin{exercise}
    For the matrix $A$ of Exercise 2, find elementary matrices $E_{1}, E_{2}, \ldots, E_{k}$ such that
    \[
        E_{k}\cdots E_{2}E_{1}A = I.
    \]
\end{exercise}

\begin{proof}
    Perform elementary-row operations consecutively.

    Add $(-2)$ times the 1st row to the 2nd row.
    \[
        \underbrace{\begin{bmatrix}
                1  & 0 & 0 \\
                -2 & 1 & 0 \\
                0  & 0 & 1
            \end{bmatrix}}_{E_{1}}
        \begin{bmatrix}
            1 & -1 & 1 \\
            2 & 0  & 1 \\
            3 & 0  & 1
        \end{bmatrix}
        =
        \begin{bmatrix}
            1 & -1 & 1  \\
            0 & 2  & -1 \\
            3 & 0  & 1
        \end{bmatrix}.
    \]

    Add $(-3)$ times the 1st row to the 3rd row
    \[
        \underbrace{\begin{bmatrix}
                1  & 0 & 0 \\
                0  & 1 & 0 \\
                -3 & 0 & 1
            \end{bmatrix}}_{E_{2}}
        \begin{bmatrix}
            1 & -1 & 1  \\
            0 & 2  & -1 \\
            3 & 0  & 1
        \end{bmatrix}
        =
        \begin{bmatrix}
            1 & -1 & 1  \\
            0 & 2  & -1 \\
            0 & 3  & -2
        \end{bmatrix}.
    \]

    Multiply the 2nd row by $\frac{1}{2}$
    \[
        \underbrace{\begin{bmatrix}
                1 & 0           & 0 \\
                0 & \frac{1}{2} & 0 \\
                0 & 0           & 1
            \end{bmatrix}}_{E_{3}}
        \begin{bmatrix}
            1 & -1 & 1  \\
            0 & 2  & -1 \\
            0 & 3  & -2
        \end{bmatrix}
        =
        \begin{bmatrix}
            1 & -1 & 1            \\
            0 & 1  & \frac{-1}{2} \\
            0 & 3  & -2
        \end{bmatrix}.
    \]

    Add $(-3)$ times the 2nd row to the 3rd row
    \[
        \underbrace{\begin{bmatrix}
                1 & 0  & 0 \\
                0 & 1  & 0 \\
                0 & -3 & 1
            \end{bmatrix}}_{E_{4}}
        \begin{bmatrix}
            1 & -1 & 1            \\
            0 & 1  & \frac{-1}{2} \\
            0 & 3  & -2
        \end{bmatrix}
        =
        \begin{bmatrix}
            1 & -1 & 1            \\
            0 & 1  & \frac{-1}{2} \\
            0 & 0  & \frac{-1}{2}
        \end{bmatrix}.
    \]

    Multiply the 3rd row by $-2$
    \[
        \underbrace{\begin{bmatrix}
                1 & 0 & 0  \\
                0 & 1 & 0  \\
                0 & 0 & -2
            \end{bmatrix}}_{E_{5}}
        \begin{bmatrix}
            1 & -1 & 1            \\
            0 & 1  & \frac{-1}{2} \\
            0 & 0  & \frac{-1}{2}
        \end{bmatrix}
        =
        \begin{bmatrix}
            1 & -1 & 1            \\
            0 & 1  & \frac{-1}{2} \\
            0 & 0  & 1
        \end{bmatrix}.
    \]

    Add $\frac{1}{2}$ times the 3rd row to the 2nd row
    \[
        \underbrace{\begin{bmatrix}
                1 & 0 & 0           \\
                0 & 1 & \frac{1}{2} \\
                0 & 0 & 1
            \end{bmatrix}}_{E_{6}}
        \begin{bmatrix}
            1 & -1 & 1            \\
            0 & 1  & \frac{-1}{2} \\
            0 & 0  & 1
        \end{bmatrix}
        =
        \begin{bmatrix}
            1 & -1 & 1 \\
            0 & 1  & 0 \\
            0 & 0  & 1
        \end{bmatrix}.
    \]

    Add $(-1)$ times the 3rd row to the 1st row
    \[
        \underbrace{\begin{bmatrix}
                1 & 0 & -1 \\
                0 & 1 & 0  \\
                0 & 0 & 1
            \end{bmatrix}}_{E_{7}}
        \begin{bmatrix}
            1 & -1 & 1 \\
            0 & 1  & 0 \\
            0 & 0  & 1
        \end{bmatrix}
        =
        \begin{bmatrix}
            1 & -1 & 0 \\
            0 & 1  & 0 \\
            0 & 0  & 1
        \end{bmatrix}.
    \]

    Add the 2nd row to the 1st row
    \[
        \underbrace{\begin{bmatrix}
                1 & 1 & 0 \\
                0 & 1 & 0 \\
                0 & 0 & 1
            \end{bmatrix}}_{E_{8}}
        \begin{bmatrix}
            1 & -1 & 0 \\
            0 & 1  & 0 \\
            0 & 0  & 1
        \end{bmatrix}
        =
        \begin{bmatrix}
            1 & 0 & 0 \\
            0 & 1 & 0 \\
            0 & 0 & 1
        \end{bmatrix} = I.
    \]
\end{proof}

\begin{exercise}
    Let
    \[
        A = \begin{bmatrix}
            1 & -1 \\
            2 & 2  \\
            1 & 0
        \end{bmatrix},
        \quad
        B = \begin{bmatrix}
            3  & 1 \\
            -4 & 4
        \end{bmatrix}.
    \]

    Is there a matrix $C$ such that $CA = B$?
\end{exercise}

\begin{proof}
    Let's find such a matrix (2 rows, 3 columns).

    \[
        \begin{bmatrix}
            C_{11} & C_{12} & C_{13} \\
            C_{21} & C_{22} & C_{23}
        \end{bmatrix}
        \begin{bmatrix}
            1 & -1 \\
            2 & 2  \\
            1 & 0
        \end{bmatrix}
        =
        \begin{bmatrix}
            C_{11} + 2{C_{12}} + C_{13} & -{C_{11}} + 2{C_{12}} \\
            C_{21} + 2{C_{22}} + C_{23} & -{C_{21}} + 2{C_{22}}
        \end{bmatrix}
    \]

    Solving for all entries of $C$ comes down to solving the following two systems
    \[
        \begin{cases}
            C_{11} + 2{C_{12}} + C_{13} = 3 \\
            -{C_{11}} + 2{C_{12}} = 1
        \end{cases}
        \qquad
        \begin{cases}
            C_{21} + 2{C_{22}} + C_{23} = -4 \\
            -{C_{21}} + 2{C_{22}} = 4
        \end{cases}
    \]

    The two systems are equivalent to
    \[
        \begin{cases}
            C_{11} + 2{C_{12}} + C_{13} = 3 \\
            {C_{12}} + \frac{1}{4}C_{13} = 1
        \end{cases}
        \qquad
        \begin{cases}
            C_{21} + 2{C_{22}} + C_{23} = -4 \\
            {C_{22}} + \frac{1}{4}C_{23} = 0
        \end{cases}.
    \]

    The two systems have solutions:
    \[
        (C_{11}, C_{12}, C_{13}) = (1 - \frac{1}{2}a, 1 - \frac{1}{4}a, a)\qquad (C_{21}, C_{22}, C_{23}) = (-4 - \frac{1}{2}b, \frac{-1}{4}b, b),
    \]

    where $a, b$ are arbitrary scalars.

    Thus $C$ is of the form
    \[
        \begin{bmatrix}
            1 - \frac{1}{2}a  & 1 - \frac{1}{4}a & a \\
            -4 - \frac{1}{2}b & \frac{-1}{4}b    & b
        \end{bmatrix}.
    \]
\end{proof}

\begin{exercise}
    Let $A$ be an $m\times n$ matrix and $B$ an $n\times k$ matrix. Show that the columns of $C = AB$ are linear combinations of the columns of $A$. If $\alpha_{1},\ldots,\alpha_{n}$ are the columns of $A$ and $\gamma_{1},\ldots,\gamma_{k}$ are the columns of $C$, then
    \[
        \gamma_{i} = \sum^{n}_{r=1} B_{ri}\alpha_{r}.
    \]
\end{exercise}

\begin{proof}
    $C$ is a $m\times k$ matrix.
    \[
        C_{ji} = \sum^{n}_{r=1} A_{jr}B_{ri} = \sum^{n}_{r=1} B_{ri}A_{jr}
    \]

    Apply for all $j\in \{ 1, 2,\ldots n \}$, we obtain that
    \[
        \gamma_{i} =
        \begin{bmatrix}
            C_{1i} \\
            C_{2i} \\
            \vdots \\
            C_{mi}
        \end{bmatrix}
        = \begin{bmatrix}
            \sum^{n}_{r=1}B_{ri}A_{1r} \\
            \sum^{n}_{r=1}B_{ri}A_{2r} \\
            \vdots                     \\
            \sum^{n}_{r=1}B_{ri}A_{mr}
        \end{bmatrix}
        = \sum^{n}_{r=1}B_{ri}\begin{bmatrix}
            A_{1r} \\
            A_{2r} \\
            \vdots \\
            A_{mr}
        \end{bmatrix}
        = \sum^{n}_{r=1}B_{ri}\alpha_{r}.
    \]

    The equality implies that each column of $C$ is a linear combination of the columns of $A$.
\end{proof}

\begin{exercise}
    Let $A$ and $B$ be $n\times n$ matrices such that $AB = I$. Prove that $BA = I$.
\end{exercise}

\begin{proof}
    \textbf{Lemma.} Let $X, Y, Z$ be $n\times n$ matrices such that $XY = ZX = 0$. Prove that if $X$ is column-equivalent to $I$ then $Y = 0$, if $X$ is row-equivalent to $I$ then $Z = 0$.

    \textit{Proof of the lemma.} The columns of $XY$ are linear combinations of the columns of $X$. Since $X$ and $I$ are column-equivalent, there exists elementary-column operations\footnote{Similar to elementary-row operation, but does not appear in the textbook, please read your/my notes for details.} (each corresponds to an elementary matrix) $E_{1}, E_{2}, \ldots, E_{k}$ such that $IE_{1}E_{2}\cdots E_{k} = X$. It follows that $E_{1}E_{2}\cdots E_{k}Y = XY = 0$, which implies that $0$ and $Y$ are row-equivalent. Therefore, $Y = 0$.

    The rows of $ZX$ are linear combinations of the rows of $X$. Since $X$ and $I$ are row-equivalent, there exists elementary-row operations (each corresponds to an elementary matrix) $E_{1}, E_{2}, \ldots, E_{\ell}$ such that $E_{\ell}\cdots E_{2}E_{1}I = X$. It follows that $ZE_{\ell}\cdots E_{2}E_{1} = ZX = 0$, which implies that $0$ and $Z$ are column-equivalent. Therefore, $Z = 0$.

    Proof of the lemma is done.

    Back to the exercise.

    $AB = I$ implies that $B$ and $I$ are row-equivalent, $A$ and $I$ are column-equivalent.

    There exists elementary matrices $E_{1}, E_{2}, \ldots, E_{k}$ such that
    \[
        (E_{k}\cdots E_{2}E_{1})B = I.
    \]

    Together with $AB = I$, we obtain that $(A - E_{k}\cdots E_{2}E_{1})B = 0$. According to the lemma, $A = E_{k}\cdots E_{2}E_{1}$.

    Each elementary-row operation has an inverse operation, so for each elementary matrix $E$, there exists an elementary matrix $E'$ such that $EE' = E'E = I$.

    Let $E'_{i}$ be an matrix such that $E_{i}E'_{i} = E_{i}E'_{i} = I$, then $E_{k}\cdots (E_{2}(E_{1}E'_{1})E'_{2})\cdots E'_{k} = I$. Therefore, $A(B - E'_{1}E'_{2}\cdots E'_{k}) = 0$. According to the lemma, $B = E'_{1}E'_{2}\cdots E'_{k}$.

    Hence $BA = E'_{1}(E'_{2}\cdots (E'_{k}E_{k})\cdots E_{2}) E_{1} = I$.
\end{proof}

\begin{exercise}
    Let
    \[
        C = \begin{bmatrix}
            C_{11} & C_{12} \\
            C_{21} & C_{22}
        \end{bmatrix}
    \]

    be a $2\times 2$ matrix. We inquire when it is possible to find $2\times 2$ matrices $A$ and $B$ such that $C = AB - BA$. Prove that such matrices can be found if and only if $C_{11} + C_{22} = 0$.
\end{exercise}

\begin{proof}
    If there are matrices $A$ and $B$ such that $C = AB - BA$.

    Let
    \[
        A = \begin{bmatrix}
            A_{11} & A_{12} \\
            A_{21} & A_{22}
        \end{bmatrix}
        \qquad
        B = \begin{bmatrix}
            B_{11} & B_{12} \\
            B_{21} & B_{22}
        \end{bmatrix}.
    \]

    We have
    \[
        AB = \begin{bmatrix}
            A_{11}B_{11} + A_{12}B_{21} & A_{11}B_{12} + A_{12}B_{22} \\
            A_{21}B_{11} + A_{22}B_{21} & A_{21}B_{12} + A_{22}B_{22}
        \end{bmatrix}
        \qquad
        BA = \begin{bmatrix}
            B_{11}A_{11} + B_{12}A_{21} & B_{11}A_{12} + B_{12}A_{22} \\
            B_{21}A_{11} + B_{22}A_{21} & B_{21}A_{12} + B_{22}A_{22}
        \end{bmatrix}
    \]

    from which we obtain that
    \[
        C = AB - BA =
        \begin{bmatrix}
            A_{12}B_{21} - B_{12}A_{21}                                 & A_{11}B_{12} + A_{12}B_{22} - (B_{11}A_{12} + B_{12}A_{22}) \\
            A_{21}B_{11} + A_{22}B_{21} - (B_{21}A_{11} + B_{22}A_{21}) & A_{21}B_{12} - B_{21}A_{12}
        \end{bmatrix}.
    \]

    $C_{11} + C_{22} = A_{12}B_{21} - B_{12}A_{21} + A_{21}B_{12} - B_{21}A_{12} = 0$.

    \hrulefill{}

    If $C_{11} + C_{22} = 0$.

    We find $A$, $B$ (their entries) such that $AB - BA = C$.

    From $AB - BA = C$, we obtain the following system of equations $(\dagger)$:
    \[
        \begin{split}
            A_{12}B_{21} - B_{12}A_{21} = C_{11} = -{C_{22}}                \\
            B_{12}\underbrace{(A_{11} - A_{22})}_{x_{1}} + (-{A_{12}})\underbrace{(B_{11} - B_{22})}_{x_{2}} = C_{12} \\
            (-{B_{21}})\underbrace{(A_{11} - A_{22})}_{x_{1}} + A_{21}\underbrace{(B_{11} - B_{22})}_{x_{2}} = C_{21}
        \end{split}
    \]

    \begin{enumerate}[label={\textbf{Case \arabic*.}},itemindent=1.2cm]
        \item $C_{11} = -{C_{22}} = 0$.

              Choose $B_{12} = -A_{12} = C_{12}$ and $-B_{21} = A_{21} = C_{21}$. Then the last two equations become
              \[
                  \begin{split}
                      C_{12}\underbrace{(A_{11} - A_{22})}_{x_{1}} + C_{12}\underbrace{(B_{11} - B_{22})}_{x_{2}} = C_{12} \\
                      C_{21}\underbrace{(A_{11} - A_{22})}_{x_{1}} + C_{21}\underbrace{(B_{11} - B_{22})}_{x_{2}} = C_{21}
                  \end{split}.
              \]

              This system have solutions (consider two cases: $C_{12} = C_{21} = 0$ and at least one of them are non-zero).
        \item $C_{11} = -{C_{22}} \ne 0$.

              Then
              \[
                  A_{11} - A_{22} = \dfrac{C_{12}A_{21} + C_{21}A_{12}}{B_{12}A_{21} - A_{12}B_{21}}
                  \qquad
                  B_{11} - B_{22} = \dfrac{B_{12}C_{21} + B_{21}C_{12}}{B_{12}A_{21} - A_{12}B_{21}}
              \]

              So $(\dagger)$ have solutions.
    \end{enumerate}

    In conclusion, $(\dagger)$ have solutions as long as $C_{11} + C_{22} = 0$.

    \hrulefill{}

    Thus, there exists $2\times 2$ matrices $A$ and $B$ such that $AB - BA = C$ if and only if $C_{11} + C_{22} = 0$.
\end{proof}

\section{Invertible Matrices}

\setcounter{exercise}{0}

\begin{exercise}
    Let
    \[
        A = \begin{bmatrix}
            1  & 2  & 1 & 0 \\
            -1 & 0  & 3 & 5 \\
            1  & -2 & 1 & 1
        \end{bmatrix}.
    \]

    Find a row-reduced echelon form matrix $R$ which is row-equivalent to $A$ and an invertible $3\times 3$ matrix $P$ such that $R = PA$.
\end{exercise}

\begin{proof}
    We begin with the following
    \[
        \begin{array}{cccc|ccc}
            1  & 2  & 1 & 0 & 1 & 0 & 0 \\
            -1 & 0  & 3 & 5 & 0 & 1 & 0 \\
            1  & -2 & 1 & 1 & 0 & 0 & 1
        \end{array}
    \]

    Add the 1st row to the 2nd row
    \[
        \begin{array}{cccc|ccc}
            1 & 2  & 1 & 0 & 1 & 0 & 0 \\
            0 & 2  & 4 & 5 & 1 & 1 & 0 \\
            1 & -2 & 1 & 1 & 0 & 0 & 1
        \end{array}
    \]

    Add (-1) times the 1st row to the 3rd row
    \[
        \begin{array}{cccc|ccc}
            1 & 2  & 1 & 0 & 1  & 0 & 0 \\
            0 & 2  & 4 & 5 & 1  & 1 & 0 \\
            0 & -4 & 0 & 1 & -1 & 0 & 1
        \end{array}
    \]

    Add 2 times the 2nd row to the 3rd row
    \[
        \begin{array}{cccc|ccc}
            1 & 2 & 1 & 0  & 1 & 0 & 0 \\
            0 & 2 & 4 & 5  & 1 & 1 & 0 \\
            0 & 0 & 8 & 11 & 1 & 2 & 1
        \end{array}
    \]

    Multiply the 3rd row by $\frac{1}{8}$
    \[
        \begin{array}{cccc|ccc}
            1 & 2 & 1 & 0            & 1           & 0           & 0           \\
            0 & 2 & 4 & 5            & 1           & 1           & 0           \\
            0 & 0 & 1 & \frac{11}{8} & \frac{1}{8} & \frac{1}{4} & \frac{1}{8}
        \end{array}
    \]

    Add (-4) times the 3rd row to the 2nd row
    \[
        \begin{array}{cccc|ccc}
            1 & 2 & 1 & 0            & 1           & 0           & 0            \\
            0 & 2 & 0 & \frac{-1}{2} & \frac{1}{2} & 0           & \frac{-1}{2} \\
            0 & 0 & 1 & \frac{11}{8} & \frac{1}{8} & \frac{1}{4} & \frac{1}{8}
        \end{array}
    \]

    Add (-1) times the 3rd row to the 1st row
    \[
        \begin{array}{cccc|ccc}
            1 & 2 & 0 & \frac{-11}{8} & \frac{7}{8} & \frac{-1}{4} & \frac{-1}{8} \\
            0 & 2 & 0 & \frac{-1}{2}  & \frac{1}{2} & 0            & \frac{-1}{2} \\
            0 & 0 & 1 & \frac{11}{8}  & \frac{1}{8} & \frac{1}{4}  & \frac{1}{8}
        \end{array}
    \]

    Multiply the 2nd row by $\frac{1}{2}$
    \[
        \begin{array}{cccc|ccc}
            1 & 2 & 0 & \frac{-11}{8} & \frac{7}{8} & \frac{-1}{4} & \frac{-1}{8} \\
            0 & 1 & 0 & \frac{-1}{4}  & \frac{1}{4} & 0            & \frac{-1}{4} \\
            0 & 0 & 1 & \frac{11}{8}  & \frac{1}{8} & \frac{1}{4}  & \frac{1}{8}
        \end{array}
    \]

    Add (-2) times the 2nd row to the 1st row
    \[
        \begin{array}{cccc|ccc}
            1 & 0 & 0 & \frac{-7}{8} & \frac{3}{8} & \frac{-1}{4} & \frac{3}{8}  \\
            0 & 1 & 0 & \frac{-1}{4} & \frac{1}{4} & 0            & \frac{-1}{4} \\
            0 & 0 & 1 & \frac{11}{8} & \frac{1}{8} & \frac{1}{4}  & \frac{1}{8}
        \end{array}
    \]

    Hence $R = PA$, where
    \[
        R = \begin{bmatrix}
            1 & 0 & 0 & \frac{-7}{8} \\
            0 & 1 & 0 & \frac{-1}{4} \\
            0 & 0 & 1 & \frac{11}{8}
        \end{bmatrix}
        \qquad
        P = \begin{bmatrix}
            \frac{3}{8} & \frac{-1}{4} & \frac{3}{8}  \\
            \frac{1}{4} & 0            & \frac{-1}{4} \\
            \frac{1}{8} & \frac{1}{4}  & \frac{1}{8}
        \end{bmatrix}.
    \]
\end{proof}

\begin{exercise}
    Do Exercise 1, but with
    \[
        A = \begin{bmatrix}
            2 & 0  & i  \\
            1 & -3 & -i \\
            i & 1  & 1
        \end{bmatrix}.
    \]
\end{exercise}

\begin{proof}
    We begin with the following matrix
    \[
        \begin{array}{ccc|ccc}
            2 & 0  & i  & 1 & 0 & 0 \\
            1 & -3 & -i & 0 & 1 & 0 \\
            i & 1  & 1  & 0 & 0 & 1
        \end{array}.
    \]

    Multiply the 1st row by $\frac{1}{2}$, multiply the 3rd row by $-i$
    \[
        \begin{array}{ccc|ccc}
            1 & 0  & \frac{1}{2}i & \frac{1}{2} & 0 & 0  \\
            1 & -3 & -i           & 0           & 1 & 0  \\
            1 & -i & -i           & 0           & 0 & -i
        \end{array}.
    \]

    Add $(-1)$ times the 1st row to the 2nd row, add $(-1)$ times the 1st row to the 3rd row
    \[
        \begin{array}{ccc|ccc}
            1 & 0  & \frac{1}{2}i  & \frac{1}{2}  & 0 & 0  \\
            0 & -3 & \frac{-3}{2}i & \frac{-1}{2} & 1 & 0  \\
            0 & -i & \frac{-3}{2}i & \frac{-1}{2} & 0 & -i
        \end{array}.
    \]

    Multiply the 2nd row by $\frac{-1}{3}$, multiply the 3rd row by $i$
    \[
        \begin{array}{ccc|ccc}
            1 & 0 & \frac{1}{2}i & \frac{1}{2}   & 0            & 0 \\
            0 & 1 & \frac{1}{2}i & \frac{1}{6}   & \frac{-1}{3} & 0 \\
            0 & 1 & \frac{3}{2}  & \frac{-1}{2}i & 0            & 1
        \end{array}.
    \]

    Add $(-1)$ times the 2nd row to the 3rd row
    \[
        \begin{array}{ccc|ccc}
            1 & 0 & \frac{1}{2}i               & \frac{1}{2}                  & 0            & 0 \\
            0 & 1 & \frac{1}{2}i               & \frac{1}{6}                  & \frac{-1}{3} & 0 \\
            0 & 0 & \frac{3}{2} - \frac{1}{2}i & \frac{-1}{6} + \frac{-1}{2}i & \frac{1}{3}  & 1
        \end{array}.
    \]

    Multiply the 3rd row by $\frac{3}{5} + \frac{1}{5}i$
    \[
        \begin{array}{ccc|ccc}
            1 & 0 & \frac{1}{2}i & \frac{1}{2}   & 0                           & 0                          \\
            0 & 1 & \frac{1}{2}i & \frac{1}{6}   & \frac{-1}{3}                & 0                          \\
            0 & 0 & 1            & \frac{-1}{3}i & \frac{1}{5} + \frac{1}{15}i & \frac{3}{5} + \frac{1}{5}i
        \end{array}.
    \]

    Add $\frac{-1}{2}i$ times the 3rd row to the 2nd row, add $\frac{-1}{2}i$ times the 3rd row to the 2nd row
    \[
        \begin{array}{ccc|ccc}
            1 & 0 & 0 & \frac{1}{3}   & \frac{1}{30} + \frac{-1}{10}i  & \frac{1}{10} + \frac{-3}{10}i \\
            0 & 1 & 0 & 0             & \frac{-3}{10} + \frac{-1}{10}i & \frac{1}{10} + \frac{-3}{10}i \\
            0 & 0 & 1 & \frac{-1}{3}i & \frac{1}{5} + \frac{1}{15}i    & \frac{3}{5} + \frac{1}{5}i
        \end{array}.
    \]

    Thus
    \[
        \begin{bmatrix}
            1 & 0 & 0 \\
            0 & 1 & 0 \\
            0 & 0 & 1
        \end{bmatrix}
        =
        \begin{bmatrix}
            \frac{1}{3}   & \frac{1}{30} + \frac{-1}{10}i  & \frac{1}{10} + \frac{-3}{10}i \\
            0             & \frac{-3}{10} + \frac{-1}{10}i & \frac{1}{10} + \frac{-3}{10}i \\
            \frac{-1}{3}i & \frac{1}{5} + \frac{1}{15}i    & \frac{3}{5} + \frac{1}{5}i
        \end{bmatrix}
        \begin{bmatrix}
            2 & 0  & i  \\
            1 & -3 & -i \\
            i & 1  & 1
        \end{bmatrix}.
    \]
\end{proof}

\begin{exercise}
    For each of the two matrices
    \[
        \begin{bmatrix}
            2 & 5  & -1 \\
            4 & -1 & 2  \\
            6 & 4  & 1
        \end{bmatrix}
        \quad
        \begin{bmatrix}
            1 & -1 & 2  \\
            3 & 2  & 4  \\
            0 & 1  & -2
        \end{bmatrix}
    \]

    use elementary row operations to discover whether it is invertible, and to find the inverse in case it is.
\end{exercise}

\begin{proof}
    \[
        \begin{array}{ccc|ccc}
            2 & 5  & -1 & 1 & 0 & 0 \\
            4 & -1 & 2  & 0 & 1 & 0 \\
            6 & 4  & 1  & 0 & 0 & 1
        \end{array}
    \]

    Add $(-2)$ times the 1st row to the 2nd row, add $(-3)$ times the 1st row to the 3rd row
    \[
        \begin{array}{ccc|ccc}
            2 & 5   & -1 & 1  & 0 & 0 \\
            0 & -11 & 4  & -2 & 1 & 0 \\
            0 & -11 & 4  & -3 & 0 & 1
        \end{array}
    \]

    Add $(-1)$ times the 2nd row to the 3rd row
    \[
        \begin{array}{ccc|ccc}
            2 & 5   & -1 & 1  & 0  & 0 \\
            0 & -11 & 4  & -2 & 1  & 0 \\
            0 & 0   & 0  & -1 & -1 & 1
        \end{array}
    \]

    Multiply the 1st row by $\frac{1}{2}$, multiply the 2nd row by $\frac{-1}{11}$
    \[
        \begin{array}{ccc|ccc}
            1 & \frac{5}{2} & \frac{-1}{2}  & \frac{1}{2}  & 0             & 0 \\
            0 & 1           & \frac{-4}{11} & \frac{2}{11} & \frac{-1}{11} & 0 \\
            0 & 0           & 0             & -1           & -1            & 1
        \end{array}
    \]

    Add $\frac{-5}{2}$ times the 2nd row to the 1st row
    \[
        \begin{array}{ccc|ccc}
            1 & 0 & \frac{9}{22}  & \frac{1}{22} & \frac{5}{22}  & 0 \\
            0 & 1 & \frac{-4}{11} & \frac{2}{11} & \frac{-1}{11} & 0 \\
            0 & 0 & 0             & -1           & -1            & 1
        \end{array}
    \]

    Hence $A$ is not invertible.

    \hrulefill{}
    \[
        \begin{array}{ccc|ccc}
            1 & -1 & 2  & 1 & 0 & 0 \\
            3 & 2  & 4  & 0 & 1 & 0 \\
            0 & 1  & -2 & 0 & 0 & 1
        \end{array}
    \]

    Add $(-3)$ times the 1st row to the 2nd row
    \[
        \begin{array}{ccc|ccc}
            1 & -1 & 2  & 1  & 0 & 0 \\
            0 & 5  & -2 & -3 & 1 & 0 \\
            0 & 1  & -2 & 0  & 0 & 1
        \end{array}
    \]

    Swap the 2nd row and the 3rd row
    \[
        \begin{array}{ccc|ccc}
            1 & -1 & 2  & 1  & 0 & 0 \\
            0 & 1  & -2 & 0  & 0 & 1 \\
            0 & 5  & -2 & -3 & 1 & 0
        \end{array}
    \]

    Add $(-5)$ times the 2nd row to the 3rd row
    \[
        \begin{array}{ccc|ccc}
            1 & -1 & 2  & 1  & 0 & 0  \\
            0 & 1  & -2 & 0  & 0 & 1  \\
            0 & 0  & 8  & -3 & 1 & -5
        \end{array}
    \]

    Multiply the 3rd row by $\frac{1}{8}$, add the 2nd row to the 1st row
    \[
        \begin{array}{ccc|ccc}
            1 & 0 & 0  & 1            & 0           & 1            \\
            0 & 1 & -2 & 0            & 0           & 1            \\
            0 & 0 & 1  & \frac{-3}{8} & \frac{1}{8} & \frac{-5}{8}
        \end{array}
    \]

    Add $2$ times the 3rd row to the 2nd row
    \[
        \begin{array}{ccc|ccc}
            1 & 0 & 0 & 1            & 0           & 1            \\
            0 & 1 & 0 & \frac{-3}{4} & \frac{1}{4} & \frac{-1}{4} \\
            0 & 0 & 1 & \frac{-3}{8} & \frac{1}{8} & \frac{-5}{8}
        \end{array}
    \]

    Hence $B$ is invertible.
\end{proof}

\begin{exercise}
    Let
    \[
        A = \begin{bmatrix}
            5 & 0 & 0 \\
            1 & 5 & 0 \\
            0 & 1 & 5
        \end{bmatrix}
    \]

    For which $X$ does there exist a scalar $c$ such that $AX = cX$?
\end{exercise}

\begin{proof}
    $X = \begin{bmatrix} x_{1} \\ x_{2} \\ x_{3} \end{bmatrix}$
    \[
        AX = \begin{bmatrix}
            5x_{1}         \\
            x_{1} + 5x_{2} \\
            x_{2} + 5x_{3}
        \end{bmatrix}
        \qquad
        cX = \begin{bmatrix}
            cx_{1} \\
            cx_{2} \\
            cx_{3}
        \end{bmatrix}
    \]

    $AX = cX$ is equivalent to
    \[
        \begin{array}{cccccc}
            (5-c)x_{1} &   &            &   &            & = 0 \\
            x_{1}      & + & (5-c)x_{2} &   &            & = 0 \\
                       &   & x_{2}      & + & (5-c)x_{3} & = 0
        \end{array}
    \]

    Swap the 1st row and the 2nd row
    \[
        \begin{array}{cccccc}
            x_{1}      & + & (5-c)x_{2} &   &            & = 0 \\
            (5-c)x_{1} &   &            &   &            & = 0 \\
                       &   & x_{2}      & + & (5-c)x_{3} & = 0
        \end{array}
    \]

    Swap the 2nd row and the 3rd row
    \[
        \begin{array}{cccccc}
            x_{1}      & + & (5-c)x_{2} &   &            & = 0 \\
                       &   & x_{2}      & + & (5-c)x_{3} & = 0 \\
            (5-c)x_{1} &   &            &   &            & = 0
        \end{array}
    \]

    Add $(c-5)$ times the 1st row to the 3rd row
    \[
        \begin{array}{cccccc}
            x_{1} & + & (5-c)x_{2}        &   &            & = 0 \\
                  &   & x_{2}             & + & (5-c)x_{3} & = 0 \\
                  &   & -{(5-c)}^{2}x_{2} &   &            & = 0
        \end{array}
    \]

    Add ${(5-c)}^{2}$ times the 2nd row to the 3rd row
    \[
        \begin{array}{cccccc}
            x_{1} & + & (5-c)x_{2} &   &                  & = 0 \\
                  &   & x_{2}      & + & (5-c)x_{3}       & = 0 \\
                  &   &            &   & {(5-c)}^{3}x_{3} & = 0
        \end{array}
    \]

    If $c\ne 5$, $x_{1} = x_{2} = x_{3} = 0$.

    If $c = 5$, $x_{1} = x_{2} = 0$.

    Thus if $X = \begin{bmatrix}0 \\ 0 \\ t\end{bmatrix}$ ($t$ is an arbitrary scalar), there exists scalar $c$ such that $AX = cX$.
\end{proof}

\begin{exercise}
    Discover whether
    \[
        A = \begin{bmatrix}
            1 & 2 & 3 & 4 \\
            0 & 2 & 3 & 4 \\
            0 & 0 & 3 & 4 \\
            0 & 0 & 0 & 4
        \end{bmatrix}.
    \]

    is invertible, and find $A^{-1}$ if it exists.
\end{exercise}

\begin{proof}
    We consider the following matrix
    \[
        \begin{array}{cccc|cccc}
            1 & 2 & 3 & 4 & 1 & 0 & 0 & 0 \\
            0 & 2 & 3 & 4 & 0 & 1 & 0 & 0 \\
            0 & 0 & 3 & 4 & 0 & 0 & 1 & 0 \\
            0 & 0 & 0 & 4 & 0 & 0 & 0 & 1
        \end{array}.
    \]

    Add $(-1)$ times the 2nd row to the 1st row
    \[
        \begin{array}{cccc|cccc}
            1 & 0 & 0 & 0 & 1 & -1 & 0 & 0 \\
            0 & 2 & 3 & 4 & 0 & 1  & 0 & 0 \\
            0 & 0 & 3 & 4 & 0 & 0  & 1 & 0 \\
            0 & 0 & 0 & 4 & 0 & 0  & 0 & 1
        \end{array}.
    \]

    Add $(-1)$ times the 3rd row to the 2nd row
    \[
        \begin{array}{cccc|cccc}
            1 & 0 & 0 & 0 & 1 & -1 & 0  & 0 \\
            0 & 2 & 0 & 0 & 0 & 1  & -1 & 0 \\
            0 & 0 & 3 & 4 & 0 & 0  & 1  & 0 \\
            0 & 0 & 0 & 4 & 0 & 0  & 0  & 1
        \end{array}.
    \]

    Add $(-1)$ times the 4th row to the 3rd row
    \[
        \begin{array}{cccc|cccc}
            1 & 0 & 0 & 0 & 1 & -1 & 0  & 0  \\
            0 & 2 & 0 & 0 & 0 & 1  & -1 & 0  \\
            0 & 0 & 3 & 0 & 0 & 0  & 1  & -1 \\
            0 & 0 & 0 & 4 & 0 & 0  & 0  & 1
        \end{array}.
    \]

    Multiply the 2nd row by $\frac{1}{2}$, multiply the 3rd row by $\frac{1}{3}$, multiply the 4th row by $\frac{1}{4}$
    \[
        \begin{array}{cccc|cccc}
            1 & 0 & 0 & 0 & 1 & -1          & 0            & 0            \\
            0 & 1 & 0 & 0 & 0 & \frac{1}{2} & \frac{-1}{2} & 0            \\
            0 & 0 & 1 & 0 & 0 & 0           & \frac{1}{3}  & \frac{-1}{3} \\
            0 & 0 & 0 & 1 & 0 & 0           & 0            & \frac{1}{4}
        \end{array}.
    \]

    Thus $A$ is invertible and
    \[
        A^{-1} =
        \begin{bmatrix}
            1 & -1          & 0            & 0            \\
            0 & \frac{1}{2} & \frac{-1}{2} & 0            \\
            0 & 0           & \frac{1}{3}  & \frac{-1}{3} \\
            0 & 0           & 0            & \frac{1}{4}
        \end{bmatrix}.
    \]
\end{proof}

\begin{exercise}
    Suppose that $A$ is a $2\times 1$ matrix and $B$ is a $1\times 2$ matrix. Prove that $C = AB$ is not invertible.
\end{exercise}

\begin{proof}
    Let
    \[
        A = \begin{bmatrix}
            a_{1} \\
            a_{2}
        \end{bmatrix}
        \qquad
        B = \begin{bmatrix}
            b_{1} & b_{2}
        \end{bmatrix}.
    \]

    Then
    \[
        C = AB = \begin{bmatrix}
            a_{1}b_{1} & a_{1}b_{2} \\
            a_{2}b_{1} & a_{2}b_{2}
        \end{bmatrix}.
    \]

    If $a_{1} = a_{2} = 0$, $C$ is the $2\times 2$ zero matrix, which is not invertible.

    If $a_{1}$ and $a_{2}$ are not simultaneously zero
    \[
        a_{2}\begin{bmatrix} a_{1}b_{1} & a_{1}b_{2} \end{bmatrix}
        -
        a_{1}\begin{bmatrix} a_{2}b_{1} & a_{2}b_{2} \end{bmatrix}
        = 0.
    \]

    So the row-reduced echelon form of $C$ has one zero row, which implies that $C$ is not invertible.
\end{proof}

\begin{exercise}
    Let $A$ be an $n\times n$ matrix. Prove the following two statements:
    \begin{enumerate}[label={(\alph*)}]
        \item If $A$ is invertible and $AB = 0$ for some $n\times n$ matrix $B$, then $B = 0$.
        \item If $A$ is not invertible, then there exists an $n\times n$ matrix $B$ such that $AB = 0$ but $B\ne 0$.
    \end{enumerate}
\end{exercise}

\begin{proof}
    \begin{enumerate}[label={(\alph*)}]
        \item If $A$ is invertible, $A^{-1}(AB) = 0$. On the other hand, $A^{-1}A = I$, so $B = 0$.
        \item If $A$ is not invertible, the system of linear equations $AX = 0$ have non-trivial solution.

              Let $X = \begin{bmatrix} t_{1} \\ t_{2} \\ \vdots \\ t_{n} \end{bmatrix}$ be such non-trivial solution.

              Let
              \[
                  B = \begin{bmatrix}
                      t_{1}  & t_{1}  & \cdots & t_{1}  \\
                      t_{2}  & t_{2}  & \cdots & t_{2}  \\
                      \vdots & \vdots & \ddots & \vdots \\
                      t_{n}  & t_{n}  & \cdots & t_{n}
                  \end{bmatrix}
              \]

              then $B$ is not a zero matrix and $AB = 0$.
    \end{enumerate}
\end{proof}

\begin{exercise}
    Let
    \[
        A = \begin{bmatrix}
            a & b \\
            c & d
        \end{bmatrix}
    \]

    Prove, using elementary row operations, that $A$ is invertible if and only if $(ad - bc)\ne 0$.
\end{exercise}

\begin{proof}
    $(\Rightarrow)$. $(ad - bc)\ne 0$.

    $a$ and $c$ are not simultaneously zero. Without loss of generality, suppose that $a\ne 0$. We perform elementary row operations on the following matrix
    \[
        \begin{array}{cc|cc}
            a & b & 1 & 0 \\
            c & d & 0 & 1
        \end{array}.
    \]

    Add $(-ca^{-1})$ times the 1st row to the 2nd row
    \[
        \begin{array}{cc|cc}
            a & b            & 1        & 0 \\
            0 & d - bca^{-1} & -ca^{-1} & 1
        \end{array}.
    \]

    Multiply the 1st row by $a^{-1}$, multiply the 2nd row by $\frac{a}{ad-bc}$
    \[
        \begin{array}{cc|cc}
            1 & ba^{-1} & a^{-1}           & 0               \\
            0 & 1       & \frac{-c}{ad-bc} & \frac{a}{ad-bc}
        \end{array}
    \]

    Add $(-ba^{-1})$ times the 2nd row to the 1st row
    \[
        \begin{array}{cc|cc}
            1 & 0 & \frac{d}{ad-bc}  & \frac{-b}{ad-bc} \\
            0 & 1 & \frac{-c}{ad-bc} & \frac{a}{ad-bc}
        \end{array}
    \]

    So $A$ is row-equivalent to the $2\times 2$ matrix, which means $A$ is invertible.

    $(\Leftarrow)$ $A$ is invertible.

    $A$ is invertible implies that $AX = 0$ has trivial solution only. Let's consider this system of linear equations
    \[
        \begin{split}
            ax_{1} + bx_{2} = 0 \\
            cx_{1} + dx_{2} = 0
        \end{split}
    \]

    If $a = c = 0$ then $AX = 0$ has non-trivial solution (for example $x_{1} = 1$, $x_{2} = 0$). So $a$ and $c$ are not simultaneously zero. Without loss of generality, suppose that $a\ne 0$.
    \[
        \begin{bmatrix}
            a & b
            c & d
        \end{bmatrix}
    \]

    Multiply the 1st row by $a^{-1}$
    \[
        \begin{bmatrix}
            1 & ba^{-1} \\
            c & d
        \end{bmatrix}
    \]

    Add $(-c)$ times the 1st row to the 2nd row
    \[
        \begin{bmatrix}
            1 & ba^{-1}      \\
            0 & d - bca^{-1}
        \end{bmatrix}
    \]

    If $ad - bc = 0$, then $A$ is row-equivalent to $\begin{bmatrix}1 & ba^{-1} \\ 0 & 0\end{bmatrix}$, which is not row-equivalent to the $2\times 2$ identity matrix.

    Hence $(ad - bc)\ne 0$.
\end{proof}

\begin{exercise}
    An $n\times n$ matrix $A$ is called upper-triangular if $A_{ij} = 0$ for $i > j$, that is, if every entry below the main diagonal is $0$. Prove that an upper-triangular (square) matrix is invertible if and only if every entry on its main diagonal is different from $0$.
\end{exercise}

\begin{proof}
    $(\Rightarrow)$ Every entry on the main diagonal is different from $0$.
    \[
        \begin{bmatrix}
            a_{11} & a_{12} & \cdots & a_{1n} \\
            0      & a_{22} & \cdots & a_{2n} \\
            \vdots & \vdots & \ddots & \vdots \\
            0      & 0      & \cdots & a_{nn}
        \end{bmatrix}
    \]

    Multiply the $k$-th row by ${a_{k}}^{-1}$
    \[
        \begin{bmatrix}
            1      & a_{12}{a_{11}}^{-1} & \cdots & a_{1n}{a_{11}}^{-1} \\
            0      & 1                   & \cdots & a_{2n}{a_{22}}^{-1} \\
            \vdots & \vdots              & \ddots & \vdots              \\
            0      & 0                   & \cdots & 1
        \end{bmatrix}
    \]

    Add $(-a_{kn}{a_{kk}}^{-1})$ times the $n$-th row to the $k$-th row ($1\le k < n$)
    \[
        \begin{bmatrix}
            1      & a_{12}{a_{11}}^{-1} & \cdots & a_{1(n-1)}{a_{11}}^{-1} & 0      \\
            0      & 1                   & \cdots & a_{2(n-1)}{a_{22}}^{-1} & 0      \\
            \vdots & \vdots              & \ddots & \vdots                  & \vdots \\
            0      & 0                   & \cdots & 1                       & 0      \\
            0      & 0                   & \cdots & 0                       & 1
        \end{bmatrix}
    \]

    Then we do the following consecutively
    \begin{itemize}
        \item add $(-a_{k(n-1)}{a_{kk}}^{-1})$ times the $(n-1)$-th row to the $k$-th row ($1\le k < n-1$),
        \item add $(-a_{k(n-2)}{a_{kk}}^{-1})$ times the $(n-2)$-th row to the $k$-th row ($1\le k < n-2$),
        \item $\ldots$
        \item add $(-a_{k2}{a_{22}}^{-1})$ times the 2nd row to the 1st row.
    \end{itemize}

    After the last elementary row operation, we obtain the $n\times n$ identity matrix. Hence $A$ is row-equivalent to the $n\times n$ identity matrix, which means $A$ is invertible.

    \bigskip

    $(\Leftarrow)$ $A$ is invertible.

    Let's consider the system of linear equations $AX = 0$. Since $A$ is invertible, this system have trivial solution only.

    If $a_{nn} = 0$, then for every $n\times n$ matrix $B$, the $n$-th row of $AB$ is zero, which means $A$ is not invertible. So $a_{nn}\ne 0$ and $x_{n} = 0$.

    If $a_{(n-1)(n-1)} = 0$, the following system of linear equations
    \[
        \begin{array}{cccccccc}
            a_{11}x_{1} & + & a_{12}x_{2} & + & \cdots & + & a_{1(n-1)}x_{n-1}     & = 0 \\
                        &   & a_{22}x_{2} & + & \cdots & + & a_{2(n-1)}x_{n-1}     & = 0 \\
            \cdots                                                                       \\
                        &   &             &   &        &   & a_{(n-1)(n-1)}x_{n-1} & = 0
        \end{array}
    \]

    has non-trivial solution, and $AX = 0$ have non-trivial solution. So $a_{(n-1)(n-1)}\ne 0$ and $x_{n-1} = x_{n} = 0$.

    If $a_{kk} = 0$, the following system of linear equations
    \[
        \begin{array}{cccccc}
            a_{11}x_{1} & + & \cdots & + & a_{1k}x_{k} & = 0 \\
                        &   & \cdots & + & a_{2k}x_{k} & = 0 \\
            \cdots                                           \\
                        &   &        &   & a_{kk}x_{k} & = 0
        \end{array}
    \]
    has non-trivial solution, and $AX = 0$ have non-trivial solution. So $a_{kk}\ne 0$.

    Hence $a_{kk}\ne 0$ for every integer $k$ such that $1\le k \le n$. Thus every entry on the main diagonal of $A$ is different from $0$.
\end{proof}

\begin{exercise}
    Prove the following generalization of Exercise 6. If $A$ is an $m\times n$ matrix, $B$ is an $n\times m$ matrix and $n < m$, then $AB$ is not invertible.
\end{exercise}

\begin{proof}
    Since $B$ is not a square matrix, the system of linear equations $BX = 0$ have non-trivial solution.

    Every row of $AB$ is a linear combination of the rows of $B$, so every solution of $BX = 0$ ($m$ unknowns) is also a solution of $(AB)X = 0$ ($m$ unknowns, since $AB$ is a $m\times m$ matrix).

    Because $BX = 0$ have non-trivial solution, then so does $(AB)X = 0$. Therefore $AB$ is not invertible.
\end{proof}

\begin{exercise}
    Let $A$ be an $m\times n$ matrix. Show that by means of a finite number of elementary row and/or column operations one can pass from it to a matrix $R$ which is both `row-reduced echelon' and `column-reduced echelon', i.e., $R_{ij} = 0$ if $i \ne j$, $R_{ii} = 1$ if $1\le i\le r$, $R_{ii} = 0$ if $i > r$. Show that $R = PAQ$, where P is an invertible $m\times m$ matrix and $Q$ is an invertible $n\times n$ matrix.
\end{exercise}

\begin{proof}
    Theorems on which this proof is based are provided in my note.

    Let $B$ be the row-reduced echelon matrix which is row-equivalent to $A$, then there exists an invertible matrix $P$ such that $B = PA$. Let $R$ be the column-reduced echelon matrix which is column-equivalent to $B$, then there exists an invertible matrix $Q$ such that $R = BQ$.

    Hence $R$ is both row-reduced echelon and column-reduced echelon. $R = BQ = (PA)Q = PAQ$.
\end{proof}

\begin{exercise}
    Let
    \[
        A = \begin{bmatrix}
            1           & \frac{1}{2}   & \cdots & \frac{1}{n}    \\
            \frac{1}{2} & \frac{1}{3}   & \cdots & \frac{1}{n+1}  \\
            \vdots      & \vdots        & \ddots & \vdots         \\
            \frac{1}{n} & \frac{1}{n+1} & \cdots & \frac{1}{2n-1}
        \end{bmatrix}.
    \]

    Prove that $A$ is invertible and all entries of $A^{-1}$ are integers.
\end{exercise}

\begin{proof}
    % unsolved
    We perform elementary (row, column) operations on the following matrix
    \[
        \begin{array}{ccccc|ccccc}
            1           & \frac{1}{2}   & \frac{1}{3}   & \cdots & \frac{1}{n}    & 1      & 0      & 0      & \cdots & 0      \\
            \frac{1}{2} & \frac{1}{3}   & \frac{1}{4}   & \cdots & \frac{1}{n+1}  & 0      & 1      & 0      & \cdots & 0      \\
            \frac{1}{3} & \frac{1}{4}   & \frac{1}{5}   & \cdots & \frac{1}{n+2}  & 0      & 0      & 1      & \cdots & 0      \\
            \vdots      & \vdots        & \vdots        & \ddots & \vdots         & \vdots & \vdots & \vdots & \ddots & \vdots \\
            \frac{1}{n} & \frac{1}{n+1} & \frac{1}{n+2} & \cdots & \frac{1}{2n-1} & 0      & 0      & 0      & \cdots & 1      \\
            \hline
            1           & 0             & 0             & \cdots & 1                                                           \\
            0           & 1             & 0             & \cdots & 0                                                           \\
            0           & 0             & 1             & \cdots & 0                                                           \\
            \vdots      & \vdots        & \vdots        & \ddots & \vdots                                                      \\
            0           & 0             & 0             & \cdots & 1
        \end{array}
    \]
\end{proof}
