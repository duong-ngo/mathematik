% chktex-file 44
\chapter{Linear Equations}

\section{Fields (no exercises)}

\section{Systems of Linear Equations}

\begin{exercise}
    Prove that the set of complex numbers of the form $a + b\sqrt{2}$, where $a, b$ are rational numbers, is a subfield of $\mathbb{C}$.
\end{exercise}

\begin{proof}
    Denote the set by $\mathbb{Q}(\sqrt{2})$. The addition and multiplication of $\mathbb{Q}(\sqrt{2})$ are defined as follows (these are usual addition and multiplication):
    \[
        \begin{split}
            (a_{1} + b_{1}\sqrt{2}) + (a_{2} + b_{2}\sqrt{2}) = (a_{1} + a_{2}) + (b_{1} + b_{2})\sqrt{2} \\
            (a_{1} + b_{1}\sqrt{2}) \cdot (a_{2} + b_{2}\sqrt{2}) = (a_{1}a_{2} + 2b_{1}b_{2}) + (a_{1}b_{2} + b_{1}a_{2})\sqrt{2}
        \end{split}
    \]

    \begin{enumerate}[label = (\arabic*)]
        \item Addition is associative.
              \begin{align*}
                    & \left((a_{1} + b_{1}\sqrt{2}) + (a_{2} + b_{2}\sqrt{2})\right) + (a_{3} + b_{3}\sqrt{2})                    \\
                  = & \left((a_{1} + a_{2}) + (b_{1} + b_{2})\sqrt{2}\right) + (a_{3} + b_{3}\sqrt{2})                            \\
                  = & \left(\left((a_{1} + a_{2}) + a_{3}\right)\right) + \left(\left(b_{1} + b_{2}\right) + b_{3}\right)\sqrt{2} \\
                  = & \left(a_{1} + \left(a_{2} + a_{3}\right)\right) + \left(b_{1} + \left(b_{2} + b_{3}\right)\right)\sqrt{2}   \\
                  = & (a_{1} + b_{1}\sqrt{2}) + \left((a_{2} + a_{3}) + (b_{2} + b_{3})\sqrt{2}\right)                            \\
                  = & (a_{1} + b_{1}\sqrt{2}) + \left((a_{2} + b_{2}\sqrt{2}) + (a_{3} + b_{3}\sqrt{2})\right)
              \end{align*}
        \item Addition has identity element.
              \begin{align*}
                  (a + b\sqrt{2}) + 0 = (a + b\sqrt{2}) + (0 + 0\sqrt{2}) = (a + 0) + (b + 0)\sqrt{2} = a + b\sqrt{2} \\
                  0 + (a + b\sqrt{2}) = (0 + 0\sqrt{2}) + (a + b\sqrt{2}) = (0 + a) + (0 + b)\sqrt{2} = a + b\sqrt{2}
              \end{align*}
        \item Each element has additive inverse.
              \begin{align*}
                  (a + b\sqrt{2}) + ((-a) + (-b)\sqrt{2}) & = (a + (-a)) + (b + (-b))\sqrt{2} = 0 + 0\sqrt{2} = 0 \\
                  ((-a) + (-b)\sqrt{2}) + (a + b\sqrt{2}) & = ((-a) + a) + ((-b) + b)\sqrt{2} = 0 + 0\sqrt{2} = 0
              \end{align*}
        \item Addition is commutative.
              \begin{align*}
                  (a_{1} + b_{1}\sqrt{2}) + (a_{2} + b_{2}\sqrt{2}) & = (a_{1} + a_{2}) + (b_{1} + b_{2})\sqrt{2}         \\
                                                                    & = (a_{2} + a_{1}) + (b_{2} + b_{1})\sqrt{2}         \\
                                                                    & = (a_{2} + b_{2}\sqrt{2}) + (a_{1} + b_{1}\sqrt{2})
              \end{align*}
        \item Multiplication is associative.
              \begin{align*}
                    & \left((a_{1} + b_{1}\sqrt{2})\cdot (a_{2} + b_{2}\sqrt{2})\right)\cdot (a_{3} + b_{3}\sqrt{2})                                                                \\
                  = & \left((a_{1}a_{2} + 2b_{1}b_{2}) + (a_{1}b_{2} + b_{1}a_{2})\sqrt{2}\right)\cdot(a_{3} + b_{3}\sqrt{2})                                                       \\
                  = & (a_{1}a_{2}a_{3} + 2b_{1}b_{2}a_{3} + 2a_{1}b_{2}b_{3} + 2b_{1}a_{2}b_{3}) + (a_{1}a_{2}b_{3} + b_{1}a_{2}a_{3} + a_{1}b_{2}a_{3} + 2b_{1}b_{2}b_{3})\sqrt{2}
              \end{align*}
              \begin{align*}
                    & (a_{1} + b_{1}\sqrt{2})\cdot\left((a_{2} + b_{2}\sqrt{2})\cdot(a_{3} + b_{3}\sqrt{2})\right)                                                                  \\
                  = & (a_{1} + b_{1}\sqrt{2})\cdot\left( (a_{2}a_{3} + 2b_{2}b_{3}) + (a_{2}b_{3} + b_{2}a_{3})\sqrt{2} \right)                                                     \\
                  = & (a_{1}a_{2}a_{3} + 2a_{1}b_{2}b_{3} + 2b_{1}a_{2}b_{3} + 2b_{1}b_{2}a_{3}) + (a_{1}a_{2}b_{3} + a_{1}b_{2}a_{3} + b_{1}a_{2}a_{3} + 2b_{1}b_{2}b_{3})\sqrt{2}
              \end{align*}
        \item Multiplication is distributive over addition.
              \begin{align*}
                    & (a_{1} + b_{1}\sqrt{2})\cdot \left((a_{2} + b_{2}\sqrt{2}) + (a_{3} + b_{3}\sqrt{2})\right)                                                               \\
                  = & (a_{1} + b_{1}\sqrt{2})\cdot\left((a_{2} + a_{3}) + (b_{2} + b_{3})\sqrt{2}\right)                                                                        \\
                  = & (a_{1}a_{2} + a_{1}a_{3} + 2b_{1}(b_{2} + b_{3})) + (a_{1}b_{2} + a_{1}b_{3} + b_{1}a_{2} + b_{1}a_{3})\sqrt{2}                                           \\
                  = & \left((a_{1}a_{2} + 2b_{1}b_{2}) + (a_{1}b_{2} + b_{1}a_{2})\sqrt{2}\right) + \left((a_{1}a_{3} + 2b_{1}b_{3}) + (a_{1}b_{3} + b_{1}a_{3})\sqrt{2}\right) \\
                  = & (a_{1} + b_{1}\sqrt{2})\cdot (a_{2} + b_{2}\sqrt{2}) + (a_{1} + b_{1}\sqrt{2})\cdot (a_{3} + b_{3}\sqrt{2})
              \end{align*}
              \begin{align*}
                    & \left((a_{1} + b_{1}\sqrt{2}) + (a_{2} + b_{2}\sqrt{2})\right)\cdot (a_{3} + b_{3}\sqrt{2})                                                               \\
                  = & \left((a_{1} + a_{2}) + (b_{1} + b_{2})\sqrt{2}\right)\cdot (a_{3} + b_{3}\sqrt{2})                                                                       \\
                  = & \left(a_{1}a_{3} + a_{2}a_{3} + 2b_{1}b_{3} + 2b_{2}b_{3}\right) + (a_{1}b_{3} + a_{2}b_{3} + b_{1}a_{3} + b_{2}a_{3})\sqrt{2}                            \\
                  = & \left((a_{1}a_{3} + 2b_{1}b_{3}) + (a_{1}b_{3} + b_{1}a_{3})\sqrt{2}\right) + \left((a_{2}a_{3} + 2b_{2}b_{3}) + (a_{2}b_{3} + b_{2}a_{3})\sqrt{2}\right) \\
                  = & (a_{1} + b_{1}\sqrt{2})\cdot (a_{3} + b_{3}\sqrt{2}) + (a_{2} + b_{2}\sqrt{2})\cdot (a_{3} + b_{3}\sqrt{2})
              \end{align*}
        \item Multiplication has identity element
              \begin{align*}
                  (a + b\sqrt{2})\cdot 1 & = (a + b\sqrt{2})\cdot (1 + 0\sqrt{2}) = a + b\sqrt{2} \\
                  1\cdot (a + b\sqrt{2}) & = (1 + 0\sqrt{2})\cdot (a + b\sqrt{2}) = a + b\sqrt{2}
              \end{align*}
        \item Multiplicative is commutative.
              \begin{align*}
                  (a_{1} + b_{1}\sqrt{2})\cdot (a_{2} + b_{2}\sqrt{2}) & = (a_{1}a_{2} + 2b_{1}b_{2}) + (a_{1}b_{2} + b_{1}a_{2})\sqrt{2} \\
                                                                       & = (a_{2}a_{1} + 2b_{2}b_{1}) + (a_{2}b_{1} + b_{2}a_{1})\sqrt{2} \\
                                                                       & = (a_{2} + b_{2}\sqrt{2})\cdot (a_{1} + b_{1}\sqrt{2})
              \end{align*}
        \item Each non-zero element has multiplicative inverse.

              $a + b\sqrt{2}$ is zero element if and only if $a = b = 2$ (since $\sqrt{2}$ is irrational).

              \begin{align*}
                  (a + b\sqrt{2})\cdot \left( \frac{a}{a^{2} - 2b^{2}} + \frac{(-b)}{a^{2} - 2b^{2}}\sqrt{2} \right) & = \frac{a^{2} - 2b^{2}}{a^{2} - 2b^{2}} + \frac{a(-b) + ba}{a^{2} - 2b^{2}}\sqrt{2} = 1 \\
                  \left( \frac{a}{a^{2} - 2b^{2}} + \frac{(-b)}{a^{2} - 2b^{2}}\sqrt{2} \right)\cdot (a + b\sqrt{2}) & = \frac{a^{2} - 2b^{2}}{a^{2} - 2b^{2}} + \frac{ab + (-b)a}{a^{2} - 2b^{2}}\sqrt{2} = 1
              \end{align*}
    \end{enumerate}
\end{proof}

Let $\mathbb{F}$ be the field of complex numbers. Are the following two systems of linear equations equivalent? If so, express each equation in each system as a linear combination of the equations in the other system.

\begin{exercise}
    \[
        \begin{cases}
            x_{1} - x_{2} = 0 \\
            2x_{1} + x_{2} = 0
        \end{cases}
        \qquad
        \begin{cases}
            3x_{1} + x_{2} = 0 \\
            x_{1} + x_{2} = 0
        \end{cases}
    \]
\end{exercise}

\begin{proof}
    The two systems are equivalent.
    \[
        \begin{cases}
            3x_{1} + x_{2} = \frac{1}{3}(x_{1} - x_{2}) + \frac{4}{3}(2x_{1} + x_{2}) \\
            x_{1} + x_{2} =  \frac{-1}{3}(x_{1} - x_{2}) + \frac{2}{3}(2x_{1} + x_{2})
        \end{cases}
    \]
    \[
        \begin{cases}
            x_{1} - x_{2} = (3x_{1} + x_{2}) - 2(x_{1} + x_{2}) \\
            2x_{1} + x_{2} = \frac{1}{2}(3x_{1} + x_{2}) + \frac{1}{2}(x_{1} + x_{2})
        \end{cases}
    \]
\end{proof}

\begin{exercise}
    \[
        \begin{cases}
            -x_{1} + x_{2} + 4x_{3} = 0 \\
            x_{1} + 3x_{2} + 8x_{3} = 0 \\
            \frac{1}{2}x_{1} + x_{2} + \frac{5}{2}x_{3} = 0
        \end{cases}
        \qquad
        \begin{cases}
            x_{1} - x_{3} = 0 \\
            x_{2} + 3x_{3} = 0
        \end{cases}
    \]
\end{exercise}

\begin{proof}
    The two systems are equivalent.
    \[
        \begin{cases}
            -x_{1} + x_{2} + 4x_{3} = (-1)(x_{1} - x_{3}) + (x_{2} + 3x_{3}) \\
            x_{1} + 3x_{2} + 8x_{3} = (x_{1} - x_{3}) + 3(x_{2} + 3x_{3})    \\
            \frac{1}{2}x_{1} + x_{2} + \frac{5}{2}x_{3} = \frac{1}{2}(x_{1} - x_{3}) + (x_{2} + 3x_{3})
        \end{cases}
    \]
    \[
        \begin{cases}
            x_{1} - x_{3} = \frac{-2}{3}(-x_{1} + x_{2} + 4x_{3}) + \frac{2}{3}(\frac{1}{2}x_{1} + x_{2} + \frac{5}{2}x_{3}) \\
            x_{2} + 3x_{3} = \frac{1}{4}(-x_{1} + x_{2} + 4x_{3}) + \frac{1}{4}(x_{1} + 3x_{2} + 8x_{3})
        \end{cases}
    \]
\end{proof}

\begin{exercise}
    \[
        \begin{cases}
            2x_{1} + (-1 + \iota)x_{2} + x_{4} = 0 \\
            3x_{2} - 2\iota x_{3} + 5x_{4} = 0
        \end{cases}
        \qquad
        \begin{cases}
            \left(1 + \frac{\iota}{2}\right)x_{1} + 8x_{2} - \iota x_{3} - x_{4} = 0 \\
            \frac{2}{3}x_{1} - \frac{1}{2}x_{2} + x_{3} + 7x_{4} = 0
        \end{cases}
    \]
\end{exercise}

\begin{proof}
    Suppose that $\frac{2}{3}x_{1} - \frac{1}{2}x_{2} + x_{3} + 7x_{4}$ is a linear combination of $2x_{1} + (-1 + \iota)x_{2} + x_{4}$ and $3x_{2} - 2\iota x_{3} + 5x_{4}$, then there exists $a$ and $b$ such that
    \[
        \frac{2}{3}x_{1} - \frac{1}{2}x_{2} + x_{3} + 7x_{4} = a(2x_{1} + (-1 + \iota)x_{2} + x_{4}) + b(3x_{2} - 2\iota x_{3} + 5x_{4})
    \]

    By identifying coefficients
    \[
        \begin{cases}
            \frac{2}{3} = 2a             \\
            1 = -2\iota b                \\
            -\frac{1}{2} = (-1 + \iota)a \\
            7 = a + 5b
        \end{cases}
        \Longrightarrow
        \begin{cases}
            a = \frac{1}{3} \\
            a = \frac{1 + \iota}{4}
        \end{cases}
    \]

    So $\frac{2}{3}x_{1} - \frac{1}{2}x_{2} + x_{3} + 7x_{4}$ is NOT a linear combination of $2x_{1} + (-1 + \iota)x_{2} + x_{4}$ and $3x_{2} - 2\iota x_{3} + 5x_{4}$. Thus the two systems are inequivalent.
\end{proof}

\begin{exercise}
    Let $F$ be a set which contains exactly two elements, 0 and 1. Define an addition and multiplication by the tables:
    \[
        \begin{array}{c|cc}
            + & 0 & 1 \\
            \hline
            0 & 0 & 1 \\
            1 & 1 & 0
        \end{array}
        \qquad
        \begin{array}{c|cc}
            \cdot & 0 & 1 \\
            \hline
            0     & 0 & 0 \\
            1     & 0 & 1
        \end{array}
    \]

    Verify that the set $F$, together with these two operations, is a field.
\end{exercise}

\begin{proof}
    \begin{enumerate}[label = (\arabic*)]
        \item Addition is associative.
              \[
                  \begin{split}
                      &(0 + 0) + 0 = 0 + 0 = 0 + (0 + 0) \\
                      &(0 + 0) + 1 = 0 + 1 = 1 = 0 + 1 = 0 + (0 + 1) \\
                      &(0 + 1) + 0 = 1 + 0 = 1 = 0 + 1 = 0 + (1 + 0) \\
                      &(0 + 1) + 1 = 1 + 1 = 0 = 0 + 0 = 0 + (1 + 1) \\
                      &(1 + 0) + 0 = 1 + 0 = 1 + (0 + 0) \\
                      &(1 + 0) + 1 = 1 + 1 = 1 + (0 + 1) \\
                      &(1 + 1) + 0 = 0 + 0 = 0 = 1 + 1 = 1 + (1 + 0) \\
                      &(1 + 1) + 1 = 0 + 1 = 1 = 1 + 0 = 1 + (1 + 1)
                  \end{split}
              \]
        \item Addition has identity element.
              \[
                  \begin{split}
                      0 + 0 = 0 \\
                      1 + 0 = 1 = 0 + 1
                  \end{split}
              \]
        \item Each element has additive inverse.
              \[
                  \begin{split}
                      0 + 0 = 0 \\
                      1 + 1 = 0
                  \end{split}
              \]
        \item Addition is commutative.
              \[
                  \begin{split}
                      0 + 0 = 0 \\
                      1 + 1 = 0 \\
                      0 + 1 = 1 = 1 + 0
                  \end{split}
              \]
        \item Multiplication is associative.
              \[
                  \begin{split}
                      &(0 \cdot 0) \cdot 0 = 0 \cdot 0 = 0 \cdot (0 \cdot 0) \\
                      &(0 \cdot 0) \cdot 1 = 0 \cdot 1 = 0 = 0 \cdot 0 = 0 \cdot (0 \cdot 1) \\
                      &(0 \cdot 1) \cdot 0 = 0 \cdot 0 = 0 \cdot (1 \cdot 0) \\
                      &(0 \cdot 1) \cdot 1 = 0 \cdot 1 = 0 \cdot (1 \cdot 1) \\
                      &(1 \cdot 0) \cdot 0 = 0 \cdot 0 = 0 = 1 \cdot 0 = 1 \cdot (0 \cdot 0) \\
                      &(1 \cdot 0) \cdot 1 = 0 \cdot 1 = 0 = 1 \cdot 0 = 1 \cdot (0 \cdot 1) \\
                      &(1 \cdot 1) \cdot 0 = 1 \cdot 0 = 0 = 1 \cdot 0 = 1 \cdot (1 \cdot 0) \\
                      &(1 \cdot 1) \cdot 1 = 1 \cdot 1 = 1 \cdot (1 \cdot 1)
                  \end{split}
              \]
        \item Multiplication is distributive over addition.
              \[
                  \begin{split}
                      &0\cdot(0 + 0) = 0\cdot 0 = 0 = 0 + 0 = 0\cdot 0 + 0\cdot 0 \\
                      &0\cdot(0 + 1) = 0\cdot 1 = 0 = 0 + 0 = 0\cdot 0 + 0\cdot 1 \\
                      &0\cdot(1 + 0) = 0\cdot 1 = 0 = 0 + 0 = 0\cdot 1 + 0\cdot 0 \\
                      &0\cdot(1 + 1) = 0\cdot 0 = 0 = 0 + 0 = 0\cdot 1 + 0\cdot 1 \\
                      &1\cdot(0 + 0) = 1\cdot 0 = 0 = 0 + 0 = 1\cdot 0 + 1\cdot 0 \\
                      &1\cdot(0 + 1) = 1\cdot 1 = 1 = 0 + 1 = 1\cdot 0 + 1\cdot 1 \\
                      &1\cdot(1 + 0) = 1\cdot 1 = 1 = 1 + 0 = 1\cdot 1 + 1\cdot 0 \\
                      &1\cdot(1 + 1) = 1\cdot 0 = 0 = 1 + 1 = 1\cdot 1 + 1\cdot 1
                  \end{split}
              \]
              \[
                  \begin{split}
                      &(0 + 0)\cdot 0 = 0\cdot 0 = 0 = 0 + 0 = 0\cdot 0 + 0\cdot 0 \\
                      &(0 + 0)\cdot 1 = 0\cdot 1 = 0 = 0 + 0 = 0\cdot 1 + 0\cdot 1 \\
                      &(0 + 1)\cdot 0 = 1\cdot 0 = 0 = 0 + 0 = 0\cdot 0 + 1\cdot 0 \\
                      &(0 + 1)\cdot 1 = 1\cdot 1 = 1 = 0 + 1 = 0\cdot 1 + 1\cdot 1 \\
                      &(1 + 0)\cdot 0 = 1\cdot 0 = 0 = 0 + 0 = 1\cdot 0 + 0\cdot 0 \\
                      &(1 + 0)\cdot 1 = 1\cdot 1 = 1 = 1 + 0 = 1\cdot 1 + 0\cdot 1 \\
                      &(1 + 1)\cdot 0 = 0\cdot 0 = 0 = 0 + 0 = 1\cdot 0 + 1\cdot 0 \\
                      &(1 + 1)\cdot 1 = 0\cdot 1 = 0 = 1 + 1 = 1\cdot 1 + 1\cdot 1
                  \end{split}
              \]
        \item Multiplication has identity element
              \[
                  \begin{split}
                      &0\cdot 1 = 0 = 1\cdot 0 \\
                      &1\cdot 1 = 1
                  \end{split}
              \]
        \item Multiplication is commutative.
              \[
                  \begin{split}
                      &0\cdot 0 = 0 \\
                      &1\cdot 1 = 1 \\
                      &0\cdot 1 = 1\cdot 0 = 0
                  \end{split}
              \]
        \item Each non-zero element has multiplicative inverse.
              \[
                  1\cdot 1 = 1
              \]
    \end{enumerate}
\end{proof}

\begin{exercise}
    Prove that if two homogeneous systems of linear equations in two unknowns have the same solutions, then they are equivalent.
\end{exercise}

\begin{proof}
    A homogeneous linear equation always has trivial solution, which contains only zero.

    A homogeneous linear equation is called trivial if and only if all of its coefficients are zero.

    \begin{lemma}\label{lemma:exercise:solution-to-homogeneous-linear-equation}
        Let $(a_{1}, a_{2})\ne (0, 0)$, then solutions to $a_{1}x_{1} + a_{2}x_{2} = 0$ are of the form $(k\cdot t_{1}, k\cdot t_{2})$, where $(t_{1}, t_{2})$ is solution other than $(0, 0)$ and $k$ is a scalar.
    \end{lemma}
    \begin{proof}[Proof of the Lemma]
        Let $t_{1} = -a_{2}$ and $t_{2} = a_{1}$, then $(t_{1}, t_{2})$ is a solution other than $(0, 0)$ of the equation. Without loss of generality, suppose that $t_{1}\ne 0$, then $a_{2}\ne 0$.

        Let $(y_{1}, y_{2})$ be a solution to $a_{1}x_{1} + a_{2}x_{2} = 0$. Since $t_{1}\ne 0$, then there exists a scalar $k$ such that $k\cdot t_{1} = y_{1}$.
        \begin{align*}
                            & (a_{1}y_{1} + a_{2}y_{2}) - (a_{1}kt_{1} + a_{2}kt_{2}) = 0  \\
            \Leftrightarrow & (a_{1}kt_{1} + a_{2}y_{2}) - (a_{1}kt_{1} + a_{2}kt_{2}) = 0 \\
            \Leftrightarrow & a_{2}y_{2} - a_{2}kt_{2} = 0                                 \\
            \Leftrightarrow & y_{2} = k\cdot t_{2} \qquad\text{(since $a_{2}\ne 0$)}
        \end{align*}
        Hence $y_{1} = kt_{1}, y_{2} = kt_{2}$.
    \end{proof}

    \begin{lemma}\label{lemma:exercise:equivalent-homogeneous-linear-equations}
        Two equations $a_{1}x_{1} + a_{2}x_{2} = 0$ and $b_{1}x_{1} + b_{2}x_{2} = 0$ have the same solution if and only if there exists two non-zero scalars $a, b$ such that
        \[
            \begin{split}
                a(a_{1}x_{1} + a_{2}x_{2}) = b_{1}x_{1} + b_{2}x_{2} \\
                b(b_{1}x_{1} + b_{2}x_{2}) = a_{1}x_{1} + a_{2}x_{2}
            \end{split}
        \]
    \end{lemma}
    \begin{proof}[Proof of the Lemma]
        ($\Rightarrow$) If there exists such scalars, then solutions to $a_{1}x_{1} + a_{2}x_{2} = 0$ are solutions to $b_{1}x_{1} + b_{2}x_{2} = 0$ and vice versa.

        ($\Leftarrow$) A homogeneous linear equation always has non-trivial solution (solution other than zeroes).

        Let's consider two cases
        \begin{enumerate}[label = \textbf{Case \arabic*.}, itemindent=1cm]
            \item Any pair of scalars is a solution to both systems.
                  This means $a_{1} = a_{2} = b_{1} = b_{2} = 0$. Hence
                  \[
                      \begin{split}
                          1\cdot (a_{1}x_{1} + a_{2}x_{2}) = b_{1}x_{1} + b_{2}x_{2} \\
                          1\cdot (b_{1}x_{1} + b_{2}x_{2}) = a_{1}x_{1} + a_{2}x_{2}
                      \end{split}
                  \]
            \item (Due to the previous lemma) Otherwise, solutions to both system are of the form $(k\cdot t_{1}, k\cdot t_{2})$, where $(t_{1}, t_{2})\ne (0, 0)$.

                  $(-a_{2}, a_{1})$ is a non-trivial solution to $a_{1}x_{1} + a_{2}x_{2} = 0$.

                  $(-b_{2}, b_{1})$ is a non-trivial solution to $b_{1}x_{1} + b_{2}x_{2} = 0$.

                  $(t_{1}, t_{2})$ is a non-trivial solution to both equations, then there exists non-zero scalars $k, \ell$ such that
                  \[
                      \begin{split}
                          -a_{2} = kt_{1}, a_{1} = kt_{2} \\
                          -b_{2} = \ell t_{1}, b_{1} = \ell t_{2}
                      \end{split}
                  \]
                  So
                  \[
                      \begin{split}
                          a_{1} = k\ell^{-1}b_{1}, a_{2} = k\ell^{-1}b_{2} \\
                          b_{1} = k^{-1}\ell a_{1}, b_{2} = k^{-1}\ell a_{2}
                      \end{split}
                  \]
                  which implies
                  \[
                      \begin{split}
                          a_{1}x_{1} + a_{2}x_{2} = k\ell^{-1}(b_{1}x_{1} + b_{2}x_{2}) \\
                          b_{1}x_{1} + b_{2}x_{2} = k^{-1}\ell(a_{1}x_{1} + a_{2}x_{2})
                      \end{split}
                  \]
                  Hence two systems are equivalent.\qedhere
        \end{enumerate}
    \end{proof}

    \begin{lemma}\label{lemma:exercise:linear-combination-of-inequivalent-linear-equations}
        Two homogeneous linear equations $a_{1}x_{1} + a_{2}x_{2} = 0$ and $b_{1}x_{1} + b_{2}x_{2} = 0$ are inequavalent and non-trivial.

        Then any homogeneous linear equation in two unknowns $x_{1}, x_{2}$ is a linear combination of these two equations.
    \end{lemma}
    \begin{proof}
        Let's $c_{1}x_{1} + c_{2}x_{2} = 0$ be a homogeneous linear equation in two unknowns $x_{1}, x_{2}$.

        \[
            t_{1} = \frac{c_{1}b_{2} - c_{2}b_{1}}{a_{1}b_{2} - a_{2}b_{1}}\qquad t_{2} = \frac{a_{1}c_{2} - a_{2}c_{1}}{a_{1}b_{2} - a_{2}b_{1}}
        \]
        then $c_{1}x_{1} + c_{2}x_{2} = t_{1}(a_{1}x_{1} + a_{2}x_{2}) + t_{2}(b_{1}x_{1} + b_{2}x_{2})$.
    \end{proof}

    Let the two systems be
    \[
        \begin{cases}
            A_{11}x_{1} + A_{12}x_{2} = 0 \\
            A_{21}x_{1} + A_{22}x_{2} = 0 \\
            \vdots                        \\
            A_{m1}x_{1} + A_{m2}x_{2} = 0
        \end{cases}
        \quad\text{and}\quad
        \begin{cases}
            B_{11}x_{1} + B_{12}x_{2} = 0 \\
            B_{21}x_{1} + B_{22}x_{2} = 0 \\
            \vdots                        \\
            B_{n1}x_{1} + B_{n2}x_{2} = 0
        \end{cases}
    \]
    and suppose that (without loss of generality) $(A_{i1}, A_{i2})\ne (0, 0)$ and $(B_{j1}, B_{j2})\ne (0, 0)$.

    \begin{enumerate}[label = \textbf{Case \arabic*.}, itemindent=1cm]
        \item Any pairs of scalars is a solution to both systems.

              This implies that all coefficients of the two systems are zero, hence the two systems are equivalent.
        \item Solutions of the two systems are of the form $(kp_{1}, kp_{2})$, where $(p_{1}, p_{2})\ne (0, 0)$.

              Let $c_{1}, c_{2}, \ldots c_{m}$ be $m$ scalars
              \[
                  (c_{1}A_{11} + c_{2}A_{21} + \cdots + c_{m}A_{m1})x_{1} + (c_{1}A_{12} + c_{2}A_{22} + \cdots + c_{m}A_{m2})x_{2}
              \]
              is a linear combination of $A_{i1}x_{1} + A_{i2}x_{2}$ such that
              \[
                  (c_{1}A_{11} + c_{2}A_{21} + \cdots + c_{m}A_{m1}, c_{1}A_{12} + c_{2}A_{22} + \cdots + c_{m}A_{m2})\ne (0, 0).
              \]
              Solutions to $B_{i1}x_{1} + B_{i2}x_{2} = 0$ are solutions to
              \[
                  (c_{1}A_{11} + c_{2}A_{21} + \cdots + c_{m}A_{m1})x_{1} + (c_{1}A_{12} + c_{2}A_{22} + \cdots + c_{m}A_{m2})x_{2} = 0
              \]
              Then there exists non-zero scalar $k_{i}$ such that
              \[
                  \begin{split}
                      k_{i}B_{i1} = c_{1}A_{11} + c_{2}A_{21} + \cdots + c_{m}A_{m1} \\
                      k_{i}B_{i2} = c_{1}A_{12} + c_{2}A_{22} + \cdots + c_{m}A_{m2}
                  \end{split}
                  \qquad\Longrightarrow\qquad
                  \begin{split}
                      B_{i1} = k_{i}^{-1}\left(c_{1}A_{11} + c_{2}A_{21} + \cdots + c_{m}A_{m1}\right) \\
                      B_{i2} = k_{i}^{-1}\left(c_{1}A_{12} + c_{2}A_{22} + \cdots + c_{m}A_{m2}\right)
                  \end{split}
              \]
              Therefore $B_{i1}x_{1} + B_{i2}x_{2}$ is a linear combination of linear equations in the first system.

              Analogously, $A_{i1}x_{1} + B_{i2}x_{2}$ is a linear combination of linear equations in the second system.

        \item The only solution to both systems is $(0, 0)$.

              The assumption implies that
              \begin{itemize}
                  \item Within the first system, there are two linear equations which are non-trivial and inequivalent.

                        According to Lemma~\ref{lemma:exercise:linear-combination-of-inequivalent-linear-equations}, each equation of the second system is a linear combination of the two inequivalent equations of the first.
                  \item Within the second system, there are two linear equations which are non-trivial and inequivalent.

                        According to Lemma~\ref{lemma:exercise:linear-combination-of-inequivalent-linear-equations}, each equation of the first system is a linear combination of the two inequivalent equations of the second.
              \end{itemize}
    \end{enumerate}

    In conclusion, the two systems are equivalent.
\end{proof}

\begin{exercise}
    Prove that each subfield of the field of complex numbers contains every rational number.
\end{exercise}

\begin{proof}
    Each subfield of the field of complex numbers contains $0$ and $1$.

    Then $n = \underbrace{1 + 1 + \cdots + 1}_{n}$ belongs to the subfield, for any natural number $n$. Consequently, $-n$ belongs to the subfield.

    Let $q$ be a non-zero integer, $p$ be an integer, then $p, q$ belongs to the subfield.

    According to Theorem~\ref{thm:inverse-elements-of-subfield}, $q^{-1}$ belongs to the subfield, then $\frac{p}{q} = pq^{-1}$ belongs to the subfield.

    Thus, each subfield of $\mathbb{C}$ contains every rational number.
\end{proof}

\begin{exercise}
    Prove that each field of characteristic zero contains a copy of the rational number field.
\end{exercise}

\begin{note}
    The term ``copy'' is used informally. A copy of the rational number field is a field which is isomorphic to the rational number field. It means, there exists an isomorphism (isomorphism mapping) $f: \mathbb{Q}\to\mathbb{F}$ such that
    \[
        f(x) + f(y) = f(x + y)\qquad f(x)f(y) = f(xy).
    \]
    Such isomorphism is called a \textit{field isomorphism}.

    If there exists an isomorphism between two fields, then the two are called \textit{isomorphic}.
\end{note}

\begin{proof}
    Denote the field by $\mathbb{F}$.

    $\mathbb{F}$ has two distinct elements $0_{\mathbb{F}}$ and $1_{\mathbb{F}}$, which are the additive identity and multiplicative identity, respectively.

    Since $\text{Char}(\mathbb{F}) = 0$, then for any pair of distinct positive integers $(p, q)$
    \[
        \underbrace{1_{\mathbb{F}} + 1_{\mathbb{F}} + \cdots + 1_{\mathbb{F}}}_{p} \ne \underbrace{1_{\mathbb{F}} + 1_{\mathbb{F}} + \cdots + 1_{\mathbb{F}}}_{q}
    \]

    We construct a mapping as follows:
    \[
        \begin{split}
            f:&\quad\mathbb{Q} \to \mathbb{F} \\
            f:&\quad 0 \mapsto 0_{\mathbb{F}} \\
            &\quad 1 \mapsto 1_{\mathbb{F}} \\
            &\quad n \mapsto \underbrace{1_{\mathbb{F}} + 1_{\mathbb{F}} + \cdots + 1_{\mathbb{F}}}_{n} \\
            &\quad -n \mapsto \underbrace{(-1_{\mathbb{F}}) + (-1_{\mathbb{F}}) + \cdots + (-1_{\mathbb{F}})}_{n} \\
            &\quad \frac{p}{q} \mapsto \frac{\underbrace{1_{\mathbb{F}} + 1_{\mathbb{F}} + \cdots + 1_{\mathbb{F}}}_{p}}{\underbrace{1_{\mathbb{F}} + 1_{\mathbb{F}} + \cdots + 1_{\mathbb{F}}}_{q}} \\
            &\quad \frac{-p}{q} \mapsto \frac{\underbrace{(-1_{\mathbb{F}}) + (-1_{\mathbb{F}}) + \cdots + (-1_{\mathbb{F}})}_{p}}{\underbrace{1_{\mathbb{F}} + 1_{\mathbb{F}} + \cdots + 1_{\mathbb{F}}}_{q}}
        \end{split}
    \]

    $f$ is a monomorphism (different ``inputs'' will be mapped to different ``outputs''). Hence $\mathbb{Q}$ and $f(\mathbb{Q})$ are isomorphic. By the definition of $f$, $f(\mathbb{Q})\subseteq\mathbb{F}$.

    Hence, each field of characteristic zero contains a subfield which is isomorphic to the field of the rational numbers.
\end{proof}

\section{Matrices and Elementary Row Operations}

\setcounter{exercise}{0}

We use the following elementary row operations
\begin{enumerate}[label={(\arabic*)}]
    \item Multiply a row with a non-zero scalar.
    \item Let $c$ be a scalar. Add $c$ times $s$-th row to $r$-th row, where $r\ne s$.
    \item Swap two rows.
\end{enumerate}

\begin{exercise}
    Find all solutions to the system of equations
    \[
        \begin{cases}
            (1 - \iota)x_{1} - \iota x_{2} = 0, \\
            2x_{1} + (1 - \iota) x_{2} = 0.
        \end{cases}
    \]
\end{exercise}

\begin{proof}
    \begingroup{}
    \allowdisplaybreaks{}
    \begin{align*}
        \begin{bmatrix}
            1 - \iota & -\iota    \\
            2         & 1 - \iota
        \end{bmatrix}
        \stackrel{(1)}{\rightarrow}
        \begin{bmatrix}
            2 & (1 + \iota)(-\iota) \\
            2 & 1 - \iota           \\
        \end{bmatrix}
        \stackrel{(2)}{\rightarrow}
        \begin{bmatrix}
            2 & 1 - \iota \\
            0 & 0
        \end{bmatrix}
        \stackrel{(1)}{\rightarrow}
        \begin{bmatrix}
            1 & \frac{1 - \iota}{2} \\
            0 & 0
        \end{bmatrix}
    \end{align*}
    \endgroup{}

    So the original system of linear equations is equivalent to
    \[
        x_{1} - \frac{\iota - 1}{2}x_{2} = 0.
    \]

    Thus all solutions of the system of linear equations are
    \[
        (x_{1}, x_{2}) = \left( \frac{(\iota - 1)c}{2}, c \right).\qedhere
    \]
\end{proof}

\begin{exercise}
    If
    \[
        A =
        \begin{bmatrix}
            3 & -1 & 2 \\
            2 & 1  & 1 \\
            1 & -3 & 0
        \end{bmatrix}
    \]

    find all solutions of $AX = 0$ by row-reducing $A$.
\end{exercise}

\begin{proof}
    \begingroup{}
    \allowdisplaybreaks{}
    \begin{multline*}
        \begin{bmatrix}
            3 & -1 & 2 \\
            2 & 1  & 1 \\
            1 & -3 & 0
        \end{bmatrix}
        \stackrel{(2)}{\rightarrow}
        \begin{bmatrix}
            0 & 8  & 2 \\
            0 & 7  & 1 \\
            1 & -3 & 0
        \end{bmatrix} \\
        \stackrel{(1)}{\rightarrow}
        \begin{bmatrix}
            0 & 1  & \frac{1}{4} \\
            0 & 7  & 1           \\
            1 & -3 & 0
        \end{bmatrix}
        \stackrel{(2)}{\rightarrow}
        \begin{bmatrix}
            0 & 1 & \frac{1}{4}  \\
            0 & 0 & \frac{-3}{4} \\
            1 & 0 & \frac{3}{4}
        \end{bmatrix} \\
        \stackrel{(2)}{\rightarrow}
        \begin{bmatrix}
            0 & 1 & 0            \\
            0 & 0 & \frac{-3}{4} \\
            1 & 0 & 0
        \end{bmatrix}
        \stackrel{(1)}{\rightarrow}
        \begin{bmatrix}
            0 & 1 & 0 \\
            0 & 0 & 1 \\
            1 & 0 & 0
        \end{bmatrix}
    \end{multline*}
    \endgroup{}

    Hence $AX = 0$ is equivalent to $\begin{bmatrix}0 & 1 & 0 \\ 0 & 0 & 1 \\ 1 & 0 & 0\end{bmatrix}\begin{bmatrix}x_{1} \\ x_{2} \\ x_{3}\end{bmatrix} = \begin{bmatrix}0 \\ 0 \\ 0\end{bmatrix}$.

    Thus, the solution of $AX = 0$ is $(x_{1}, x_{2}, x_{3}) = (0, 0, 0)$.
\end{proof}

\begin{exercise}
    If
    \[
        A =
        \begin{bmatrix}
            6  & -4 & 0  \\
            4  & -2 & 0  \\
            -1 & 0  & -3
        \end{bmatrix}
    \]

    find all solutions of $AX = 2X$ and all solutions of $AX = 3X$. (The symbol $cX$ denotes the matrix each entry of which is $c$ times the corresponding entry of $X$.)
\end{exercise}

\begin{proof}
    $AX = 2X$ is equivalent to
    \begin{align*}
        \begin{bmatrix}
            6  & -4 & 0  \\
            4  & -2 & 0  \\
            -1 & 0  & -3
        \end{bmatrix}
        \begin{bmatrix}
            x_{1} \\ x_{2} \\ x_{3}
        \end{bmatrix}
        =
        \begin{bmatrix}
            2x_{1} \\ 2x_{2} \\ 2x_{3}
        \end{bmatrix}
        \Longleftrightarrow
        \begin{bmatrix}
            6x_{1} + (-4)x_{2} \\ 4x_{1} + (-2)x_{2} \\ (-1)x_{1} + (-3)x_{3}
        \end{bmatrix}
        =
        \begin{bmatrix}
            2x_{1} \\ 2x_{2} \\ 2x_{3}
        \end{bmatrix}
        \Longleftrightarrow
        \begin{bmatrix}
            4x_{1} + (-4)x_{2} \\
            4x_{1} + (-4)x_{2} \\
            (-1)x_{1} + (-5)x_{3}
        \end{bmatrix}
        =
        \begin{bmatrix}
            0 \\ 0 \\ 0
        \end{bmatrix}.
    \end{align*}

    Thus, all solutions of $AX = 2X$ are $(x_{1}, x_{2}, x_{3}) = (c, c, -5c)$.

    \bigskip\hrule\bigskip

    $AX = 3X$ is equivalent to
    \begin{align*}
        \begin{bmatrix}
            6  & -4 & 0  \\
            4  & -2 & 0  \\
            -1 & 0  & -3
        \end{bmatrix}
        \begin{bmatrix}
            x_{1} \\ x_{2} \\ x_{3}
        \end{bmatrix}
        =
        \begin{bmatrix}
            3x_{1} \\ 3x_{2} \\ 3x_{3}
        \end{bmatrix}
        \Longleftrightarrow
        \begin{bmatrix}
            6x_{1} + (-4)x_{2} \\ 4x_{1} + (-2)x_{2} \\ (-1)x_{1} + (-3)x_{3}
        \end{bmatrix}
        =
        \begin{bmatrix}
            3x_{1} \\ 3x_{2} \\ 3x_{3}
        \end{bmatrix}
        \Longleftrightarrow
        \begin{bmatrix}
            3x_{1} + (-4)x_{2} \\
            4x_{1} + (-5)x_{2} \\
            (-1)x_{1} + (-6)x_{3}
        \end{bmatrix}
        =
        \begin{bmatrix}
            0 \\ 0 \\ 0
        \end{bmatrix}.
    \end{align*}

    Thus, all solutions of $AX = 3X$ are $(x_{1}, x_{2}, x_{3}) = (0, 0, 0)$.
\end{proof}

\begin{exercise}
    Find a row-reduced matrix which is row-equivalent to
    \[
        A =
        \begin{bmatrix}
            \iota & -(1 + \iota) & 0  \\
            1     & -2           & 1  \\
            1     & 2\iota       & -1
        \end{bmatrix}.
    \]
\end{exercise}

\begin{proof}
    \begingroup{}
    \allowdisplaybreaks{}
    \begin{align*}
        \begin{bmatrix}
            \iota & -(1 + \iota) & 0  \\
            1     & -2           & 1  \\
            1     & 2\iota       & -1
        \end{bmatrix}
        \stackrel{(2)}{\rightarrow}
        \begin{bmatrix}
            0 & 1 - \iota   & \iota \\
            0 & -2 - 2\iota & -2    \\
            1 & 2\iota      & -1
        \end{bmatrix}
        \stackrel{(1)}{\rightarrow}
        \begin{bmatrix}
            0 & 2         & \iota - 1 \\
            0 & 1 + \iota & 1         \\
            1 & 2\iota    & -1
        \end{bmatrix}
        \stackrel{(1)}{\rightarrow}
        \begin{bmatrix}
            0 & 1      & \frac{\iota - 1}{2} \\
            0 & 1      & \frac{1 - \iota}{2} \\
            1 & 2\iota & -1
        \end{bmatrix} \\
        \stackrel{(2)}{\rightarrow}
        \begin{bmatrix}
            0 & 0      & \iota - 1           \\
            0 & 1      & \frac{1 - \iota}{2} \\
            1 & 2\iota & -1
        \end{bmatrix}
        \stackrel{(1)}{\rightarrow}
        \begin{bmatrix}
            0 & 0      & 1                   \\
            0 & 1      & \frac{1 - \iota}{2} \\
            1 & 2\iota & -1
        \end{bmatrix}
        \stackrel{(2)}{\rightarrow}
        \begin{bmatrix}
            0 & 0      & 1 \\
            0 & 1      & 0 \\
            1 & 2\iota & 0
        \end{bmatrix}
        \stackrel{(2)}{\rightarrow}
        \begin{bmatrix}
            0 & 0 & 1 \\
            0 & 1 & 0 \\
            1 & 0 & 0
        \end{bmatrix}.
    \end{align*}
    \endgroup{}
\end{proof}

\begin{exercise}
    Prove that the following two matrices are not row-equivalent.
    \[
        \begin{bmatrix}
            2 & 0  & 0 \\
            a & -1 & 0 \\
            b & c  & 3
        \end{bmatrix},\quad
        \begin{bmatrix}
            1  & 1 & 2  \\
            -2 & 0 & -1 \\
            1  & 3 & 5
        \end{bmatrix}.
    \]
\end{exercise}

\begin{proof}
    Let's find a row-reduced matrix which is row-equivalent to the second matrix.
    \[
        \begin{bmatrix}
            1  & 1 & 2  \\
            -2 & 0 & -1 \\
            1  & 3 & 5
        \end{bmatrix}
        \stackrel{(2)}{\rightarrow}
        \begin{bmatrix}
            1 & 1 & 2 \\
            0 & 2 & 3 \\
            0 & 2 & 3
        \end{bmatrix}
        \stackrel{(2)}{\rightarrow}
        \begin{bmatrix}
            1 & 1 & 2 \\
            0 & 2 & 3 \\
            0 & 0 & 0
        \end{bmatrix}
        \stackrel{(2)}{\rightarrow}
        \begin{bmatrix}
            1 & 1 & 2           \\
            0 & 1 & \frac{3}{2} \\
            0 & 0 & 0
        \end{bmatrix}
        \stackrel{(2)}{\rightarrow}
        \begin{bmatrix}
            1 & 0 & \frac{1}{2} \\
            0 & 1 & \frac{3}{2} \\
            0 & 0 & 0
        \end{bmatrix}
    \]

    Since $(2, 0, 0)$ is NOT a linear combination of $(1, 0, \frac{1}{2})$ and $(0, 1, \frac{3}{2})$, the two matrices are not row-equivalent.
\end{proof}

\begin{exercise}
    Let
    \[
        A =
        \begin{bmatrix}
            a & b \\
            c & d
        \end{bmatrix}
    \]

    be a $2\times 2$ matrix with complex entries. Suppose that $A$ is row-reduced and also that $a + b + c + d = 0$. Prove that there are exactly three such matrices.
\end{exercise}

\begin{proof}
    The zero matrix $\begin{bmatrix}0 & 0 \\ 0 & 0\end{bmatrix}$ is one such matrix.

    Suppose that $A$ is not the zero matrix, then $a = 1$ or $c = 1$.

    If $a = 1$, then $c = 0$ (since $A$ is row-reduced). Furthermore
    \begin{itemize}
        \item $d = 0$. It follows that $b = -1$ (since $a + b + c + d = 0$).
        \item $d = 1$. It follows that $b = 0$. But $a + b + c + d\ne 0$.
    \end{itemize}

    If $c = 1$, then $a = 0$ (since $A$ is row-reduced). Furthermore
    \begin{itemize}
        \item $b = 0$. It follows that $d = -1$ (since $a + b + c + d = 0$).
        \item $b = 1$. It follows that $d = 0$. But $a + b + c + d\ne 0$.
    \end{itemize}

    Thus, the following three matrices are the only that satisfy the conditions
    \[
        \begin{bmatrix}
            0 & 0 \\
            0 & 0
        \end{bmatrix},\quad
        \begin{bmatrix}
            1 & -1 \\
            0 & 0
        \end{bmatrix},\quad
        \begin{bmatrix}
            0 & 0  \\
            1 & -1
        \end{bmatrix}.
    \]
\end{proof}

\begin{exercise}
    Prove that the interchange of two rows of a matrix can be accomplished by a finite sequence of elementary row operations of the other two types.
\end{exercise}

\begin{proof}
    If two rows of a matrix are identical, then after zero elementary row operation, we ``swap them''.

    Otherwise, two rows are not identical. Let's denote them by $r$ and $s$.

    \begin{itemize}
        \item By the 1st type of operation, we obtain $(r, s) \to (-r, s)$.
        \item By the 2nd type of operation, we obtain $(-r, s) \to (-r+s, s)$.
        \item By the 2nd type of operation, we obtain $(-r+s, s) \to (-r+s, s - (-r+s)) = (-r+s, r)$.
        \item By the 2nd type of operation, we obtain $(-r+s, r) \to (-r+s+r, r) = (s, r)$.
    \end{itemize}

    Hence, we swap two distinct rows by using only the 1st and 2nd elementary row operation.
\end{proof}

\begin{exercise}
    Consider the system of equations $AX = 0$ where
    \[
        A =
        \begin{bmatrix}
            a & b \\
            c & d
        \end{bmatrix}
    \]

    is a $2\times 2$ matrix over the field $\mathbb{F}$. Prove the following.
    \begin{enumerate}[label={(\alph*)}]
        \item If every entry of $A$ is 0, then every pair $(x_{1}, x_{2})$ is a solution of $AX = 0$.
        \item If $ad - bc \ne 0$, the system $AX = 0$ has the only trivial solution $x_{1} = x_{2} = 0$.
        \item If $ad - bc = 0$ and some entry of $A$ is different from 0, then there is a solution $({x}^{0}_{1}, {x}^{0}_{2})$ such that $(x_{1}, x_{2})$ is a solution if and only if there exists a scalar $y$ such that $x_{1} = y{x}^{0}_{1}, x_{2} = y{x}^{0}_{2}$.
    \end{enumerate}
\end{exercise}

\begin{proof}
    \begin{enumerate}[label={(\alph*)}]
        \item $a = b = c = d = 0$, then $AX = 0$ is equivalent to
              \[
                  \begin{cases}
                      0x_{1} + 0x_{2} = 0, \\
                      0x_{1} + 0x_{2} = 0.
                  \end{cases}
              \]

              Hence every pair $(x_{1}, x_{2})$ is a solution of $AX = 0$.
        \item $AX = 0$ is equivalent to
              \[
                  \begin{cases}
                      a{x}_{1} + b{x}_{2} = 0 \\
                      c{x}_{1} + d{x}_{2} = 0
                  \end{cases}.
              \]

              there are these two particular linear combinations
              \[
                  \begin{cases}
                      d(a{x}_{1} + b{x}_{2}) - b(c{x}_{1} + d{x}_{2}) = 0 \\
                      c(a{x}_{1} + b{x}_{2}) - a(c{x}_{1} + d{x}_{2}) = 0
                  \end{cases}
                  \Longleftrightarrow
                  \begin{cases}
                      (ad - bc)x_{1} = 0 \\
                      (bc - ad)x_{2} = 0
                  \end{cases}.
              \]

              Since $ad - bc\ne 0$, we obtain that $x_{1} = x_{2} = 0$. This is the only solution.
        \item Without loss of generality, suppose that $a\ne 0$. Since $ad = bc$,
              \[
                  \begin{bmatrix}
                      a & b \\
                      c & d
                  \end{bmatrix}
                  \stackrel{(1)}{\rightarrow}
                  \begin{bmatrix}
                      a  & b  \\
                      ac & ad
                  \end{bmatrix}
                  \rightarrow
                  \begin{bmatrix}
                      a  & b  \\
                      ac & bc
                  \end{bmatrix}
                  \stackrel{(2)}{\rightarrow}
                  \begin{bmatrix}
                      a & b \\
                      0 & 0
                  \end{bmatrix}
              \]

              Since $a\ne 0$, then $AX = 0$ has a non trivial solution $({x}^{0}_{1}, {x}^{0}_{2}) = (-b, a)$.

              Suppose that $(x_{1}, x_{2})$ is a solution of $AX = 0$. Since $a\ne 0$, then there exists $y$ such that $x_{2} = ay = y{x}^{0}_{2}$.
              \begin{align*}
                  0 & = a{x}_{1} + b{x}_{2}                                   \\
                    & = a({x}_{1} - y{x}^{0}_{1}) + b({x}_{2} - y{x}^{0}_{2}) \\
                    & = a({x}_{1} - y{x}^{0}_{1}) + 0                         \\
                    & = a({x}_{1} - y{x}^{0}_{1}).
              \end{align*}

              Since $a\ne 0$, ${x}_{1} = y{x}^{0}_{1}$.

              Hence, there exists a scalar $y$ such that $x_{1} = y{x}^{0}_{1}, x_{2} = y{x}^{0}_{2}$, where $({x}^{0}_{1}, {x}^{0}_{2})$ is a non-trivial solution of $AX = 0$ and $(x_{1}, y_{1})$ is an arbitrary solution of $AX = 0$.
    \end{enumerate}
\end{proof}

\section{Row-Reduced Echelon Matrices}

\section{Matrix Multiplication}

\section{Invertible Matrices}

