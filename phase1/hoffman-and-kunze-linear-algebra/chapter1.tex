% chktex-file 44
\chapter{Linear Equations}

\section{Fields}

\subsection{Definition and Properties}

The authors assumed that readers are familiar with the elementary algebras of real and complex numbers. Let's denote the set of real numbers or complex numbers by $\mathbb{F}$. Two operations, namely, addition and multiplication, have the following properties, from which one can deduce a lot of algebraic properties of real and complex numbers.

\begin{enumerate}[label = (\arabic*)]
	\item Addition is associative.
	      \[
		      (x + y) + z = x + (y + z)\qquad\forall x, y, z\in\mathbb{F}
	      \]
	\item Identity element of addition.

	      There exists an element $0$ (zero) in $\mathbb{F}$ such that
	      \[
		      x + 0 = 0 + x = x\qquad\forall x\in\mathbb{F}
	      \]
	\item Inverse element in terms of addition.

	      For each $x$ in $\mathbb{F}$, there exists an element $(-x)$ in $\mathbb{F}$ such that
	      \[
		      x + (-x) = (-x) + x = 0
	      \]
	\item Addition is commutative.
	      \[
		      x + y = y + x\qquad\forall x, y\in\mathbb{F}
	      \]
	\item Multiplication is associative.
	      \[
		      (x\cdot y)\cdot z = x\cdot (y\cdot z)\qquad\forall x, y, z\in\mathbb{F}
	      \]
	\item Multiplication is distributive over addition.
	      \[
		      \begin{split}
			      z\cdot (x + y) = z\cdot x + z\cdot y\qquad\forall x, y, z\in\mathbb{F} \\
			      (x + y)\cdot z = x\cdot z + y\cdot z\qquad\forall x, y, z\in\mathbb{F}
		      \end{split}
	      \]
	\item Identity element of multiplication.

	      There exists an element $1$ (one) in $\mathbb{F}$ such that
	      \[
		      x\cdot 1 = 1\cdot x = x\qquad\forall x\in\mathbb{F}
	      \]
	\item Multiplication is commutative.
	      \[
		      x\cdot y = y\cdot x\qquad\forall x, y\in\mathbb{F}
	      \]
	\item Inverse element in terms of multiplication.

	      For each non-zero element $x$ in $\mathbb{F}$, there exists an element $x^{-1}$ in $\mathbb{F}$ such that
	      \[
		      x\cdot x^{-1} = x^{-1}\cdot x = 1
	      \]
\end{enumerate}

\begin{note}
	The numbers themselves are not useful. Things start to be interesting when they admit some kinds of ``rules''. In the cases of real numbers and complex numbers, some of those rules are the above properties. It is even more interesting that one can prove a lot of more properties from these rules \textit{only}.

	Here comes an abstract point of view. It is painful and took me a lot of time to accept.

	There are objects other than real numbers and complex numbers (together with addition and multiplication) and they still possess nine of the above properties. Mathematicians love to expand their knowledge from a minimal list of axioms (or rules). From the nineteenth century, mathematicians came up with a revolution idea: the objects that satisfy a list of rules \textit{need not to be defined}, they could be anything, one can just focus on the rules and do the mathematics works. In nine of the above properties, everything is merely symbol, including ``$x$'', ``$y$'', ``$z$'', even the plus sign ``$+$'', the dot sign ``$\cdot$'', zero ``$0$'', one ``$1$'', ``$(-x)$'', ``$x^{-1}$''. Properties are the meanings that are given to the two operations.

	Thinking of numbers as symbols allows us to do a lot of things. Symbols allow us to forget about numbers and the redundant, to focus on ``the minimal list of axioms'', which are the essentials. Symbols are \textit{abstractions}. Abstractions allow generalizing. By using symbols, we are able to prove many (not all) properties of any objects that satisfy the rules.

	By accepting that point of view, I was able to accept the following definition and go on with linear algebra and abstract algebra.

	The other words for ``rule'' are ``axiom'' and ``law''.
\end{note}

\begin{definition}[Field]
	A set $\mathbb{F}$ equipped with two operations\footnote{In this case, ``Operation'' is also known as \textit{the law of composition}, for it maps two elements to another.} called \textit{addition} (denoted by plus sign ``$+$'') and \textit{multiplication} (denoted by dot sign ``$\cdot$'')
	\[
		\begin{split}
			+: \mathbb{F}\times\mathbb{F}\to\mathbb{F} \\
			\cdot: \mathbb{F}\times\mathbb{F}\to\mathbb{F}
		\end{split}
	\]
	is called \textit{a field} if these two operations satisfy
	\begin{enumerate}[label = (\arabic*)]
		\item Addition is associative.\label{field:addition-associative}
		      \[
			      (\forall x, y, z\in\mathbb{F}): (x + y) + z = x + (y + z)
		      \]
		\item Addition has an identity element called zero.\label{field:additive-identity}
		      \[
			      (\exists\,0\in\mathbb{F})(\forall x\in\mathbb{F}): x + 0 = 0 + x = x
		      \]
		\item Each element has an additive inverse.\label{field:additive-inverse}
		      \[
			      (\forall x\in\mathbb{F})(\exists (-x)\in\mathbb{F}): x + (-x) = (-x) + x = 0
		      \]
		\item Addition is commutative.\label{field:addition-commutative}
		      \[
			      (\forall x,y\in\mathbb{F}): x + y = y + x
		      \]
		\item Multiplication is associative.\label{field:multiplication-association}
		      \[
			      (\forall x, y, z\in\mathbb{F}): (x\cdot y)\cdot z = x\cdot (y\cdot z)
		      \]
		\item Multiplication is distributive (on both sides) over addition.\label{field:distributive}
		      \[
			      (\forall x, y, z\in\mathbb{F}):
			      \begin{cases}
				      z\cdot (x + y) = z\cdot x + z\cdot y \\
				      (x + y)\cdot z = x\cdot z + y\cdot z
			      \end{cases}
		      \]
		\item Multiplication has an identity element called one, or unit.\label{field:multiplication-identity}
		      \[
			      (\exists\,1\in\mathbb{F})(\forall x\in\mathbb{F}): x\cdot 1 = 1\cdot x = x
		      \]
		\item Multiplication is commutative.\label{field:multiplication-commutative}
		      \[
			      (\forall x, y\in\mathbb{F}): x\cdot y = y\cdot x
		      \]
		\item Each non-zero element has a multiplicative inverse.\label{field:multiplicative-inverse}
		      \[
			      (\forall x\in\mathbb{F} \wedge x\ne 0)(\exists x^{-1}\in\mathbb{F}): x\cdot x^{-1} = x^{-1}\cdot x = 1
		      \]
	\end{enumerate}
\end{definition}

\begin{note}
	The zero and one element of a field are different. Therefore, a field must have at least two distinct elements.

	In the book, elements of a field are also called \textit{scalars}.

	I listed the commutative laws after the associative ones. I did this for good reason. Commutative is much more simple and familiar than associative, but it is more ``rare'' in other algebraic structures like groups and rings. For now, you don't have to dwell on this.
\end{note}

\begin{definition}[Substraction and Division]
	Within a field
	\begin{enumerate}
		\item Subtraction is adding with the additive inverse. Formally
		      \[
			      x - y = x + (-y)
		      \]
		\item Division by a non-zero element is multiplying with its multiplicative inverse. Formally
		      \[
			      x / y = x\cdot y^{-1}
		      \]
	\end{enumerate}
\end{definition}

Proof of the following theorem demonstrates how to use the axioms to derive other properties.

\begin{theorem}\label{thm:field-properties}
	In a field $\mathbb{F}$
	\begin{enumerate}[label = (\roman*)]
		\item Zero element is unique.
		\item Each element has unique additive inverse.
		\item One (unit) element is unique.
		\item Each non-zero element has unique multiplicative inverse.
		\item Multiplying an arbitrary element with zero will produce zero.
		\item $xy = 0$ implies that $x = 0$ or $y = 0$, where $x$ and $y$ are elements of $\mathbb{F}$.
	\end{enumerate}
\end{theorem}

\begin{proof}
	\begin{enumerate}[label = (\roman*)]
		\item Suppose that $0$ and $0'$ are zero elements, then by axiom~\ref{field:additive-identity}
		      \[
			      0 = 0 + 0' = 0'
		      \]
		      which means the two zero elements are identical.
		\item Suppose that $x_{1}, x_{2}$ are additive inverses of $x$. Then by axioms~\ref{field:addition-associative} and~\ref{field:additive-identity}
		      \[
			      x_{1} = x_{1} + 0 = x_{1} + (x + x_{2}) = (x_{1} + x) + x_{2} = 0 + x_{2} = x_{2}
		      \]
		      which means the two additive inverses are identical.
		\item Suppose that $1$ and $1'$ are one elements, then by axiom~\ref{field:multiplication-identity}
		      \[
			      1 = 1\cdot 1' = 1'
		      \]
		      which means the two one elements are identical.
		\item Suppose that $x_{1}, x_{2}$ are multiplicative inverse of the non-zero element $x$. Then by axioms~\ref{field:multiplication-association} and~\ref{field:multiplication-identity}
		      \[
			      x_{1} = x_{1}\cdot 1 = x_{1}\cdot (x\cdot x_{2}) = (x_{1}\cdot x)\cdot x_{2} = 1\cdot x_{2} = x_{2}
		      \]
		\item By axiom~\ref{field:distributive}
		      \[
			      0x = (0 + 0)x = 0x + 0x
		      \]
		      By axiom~\ref{field:additive-inverse}
		      \[
			      0x - 0x = 0x + 0x - 0x
		      \]
		      It follows that $0x = 0$.
		\item If $x$ is other than zero, then by axiom~\ref{field:multiplicative-inverse}, there exists $x^{-1}$ such that $x\cdot x^{-1} = x{^-1}x = 1$. Since $xy = 0$, then
		      \begin{align*}
			                           & x^{-1}\cdot (xy) = x^{-1}\cdot 0 \\
			      \Leftrightarrow\quad & (x^{-1}\cdot x)\cdot y = 0       \\
			      \Leftrightarrow\quad & 1\cdot y = 0                     \\
			      \Leftrightarrow\quad & y = 0.
		      \end{align*}
		      Similarly, if $y$ is other than zero, then $x = 0$.

		      Hence $xy = 0$ implies that $x = 0$ or $y = 0$.
	\end{enumerate}
\end{proof}

\begin{note}
	Part (v) and (vi) of Theorem~\ref{thm:field-properties} surprised me at first. If no one had shown me how to derive it from the axioms, I would have been taken it for granted. The lesson here is don't be like me.
\end{note}

Now let's consider some examples and non-exampples of field.

\begin{example}
	\begin{enumerate}[label = (\alph*)]
		\item The set of real numbers $\mathbb{R}$, the set of complex numbers $\mathbb{C}$ (with the usual addition and multiplication) are fields.
		\item The set of natural numbers $\mathbb{N}$ (with the usual addition and multiplication) is not a field, since elements other than zero do not have additive inverses.
		\item The set of integer numbers $\mathbb{Z}$ (with the usual addition and multiplication) is not a field, since non-zero integers other than $\pm 1$ do not have multiplicative inverses.
		\item The set of rational numbers $\mathbb{Q}$ (with the usual addition and multiplication) is a field.
	\end{enumerate}
\end{example}

\subsection{Subfield}

Let's take a look at the examples of rational numbers, real numbers, and complex numbers: $\mathbb{Q}\subseteq\mathbb{R}\subseteq\mathbb{C}$. This remark leads to the definition of \textit{subfield}.

\begin{definition}[Subfield]
	$\mathbb{G}$ is a subfield of $\mathbb{F}$ if and only if $\mathbb{G}$ is a field and a subset of $\mathbb{F}$, where the operations of $\mathbb{G}$ are those of $\mathbb{F}$ but restricted to $\mathbb{G}$.
\end{definition}

With the field axioms and the definition of subfield, we can obtain further information about subfield.

\begin{theorem}
	Let $\mathbb{G}$ be a subfield of $\mathbb{F}$. Then the two share the same zero element and unit element.
\end{theorem}

\begin{proof}
	$\mathbb{G}$ is a subfield of $\mathbb{F}$. Then $\mathbb{G}$ has at least two distinct elements, according to the note in the previous subsection. Therefore, from the elements of $\mathbb{G}$, we can pick a non-zero element, let's denote it by $x$.

	Let $0_{\mathbb{G}}$ and $1_{\mathbb{G}}$ be the zero and unit elements of $\mathbb{G}$, respectively. $0_{\mathbb{F}}$ and $1_{\mathbb{F}}$ are those of $\mathbb{F}$. Notice that $0_{\mathbb{G}}$ and $1_{\mathbb{G}}$ are also elements of $\mathbb{F}$, since $\mathbb{G}$ is a subset of $\mathbb{F}$.

	$(-x)$ is the additive inverse of $x$ within $\mathbb{F}$.

	Apply axioms (1) (2) (3)
	\begin{align*}
		x + 0_{\mathbb{G}}                          & = x = x + 0_{\mathbb{F}}          \\
		\Rightarrow (-x) + (x + 0_{\mathbb{G}})     & = (-x) + (x + 0_{\mathbb{F}})     \\
		\Rightarrow ((-x) + x) + 0_{\mathbb{G}}     & = ((-x) + x) + 0_{\mathbb{F}}     \\
		\Rightarrow 0_{\mathbb{F}} + 0_{\mathbb{G}} & = 0_{\mathbb{F}} + 0_{\mathbb{F}} \\
		\Rightarrow 0_{\mathbb{G}} = 0_{\mathbb{F}}.
	\end{align*}
	Hence the two fields share the same zero element. So $x$ is also a non-zero element of $\mathbb{F}$, which means it has multiplicative inverse. $x^{-1}$ is the multiplicative inverse of $x$ within $\mathbb{F}$.

	Apply axioms (5) (7) (9)
	\begin{align*}
		x\cdot 1_{\mathbb{G}}                           & = x = x\cdot 1_{\mathbb{F}}           \\
		\Rightarrow x^{-1}\cdot (x\cdot 1_{\mathbb{G}}) & = x^{-1}\cdot (x\cdot 1_{\mathbb{F}}) \\
		\Rightarrow (x^{-1}\cdot x)\cdot 1_{\mathbb{G}} & = (x^{-1}\cdot x)\cdot 1_{\mathbb{F}} \\
		\Rightarrow 1_{\mathbb{F}}\cdot 1_{\mathbb{G}}  & = 1_{\mathbb{F}}\cdot 1_{\mathbb{F}}  \\
		\Rightarrow 1_{\mathbb{G}}                      & = 1_{\mathbb{F}}.
	\end{align*}

	Hence the two fields share the same unit element.
\end{proof}

\begin{theorem}\label{thm:inverse-elements-of-subfield}
	Let $\mathbb{G}$ be a subfield of $\mathbb{F}$. The additive inverse of an element within $\mathbb{G}$ is within $\mathbb{G}$, the multiplicative inverse of a non-zero element within $\mathbb{G}$ is within $\mathbb{G}$.
\end{theorem}

\begin{proof}
	The additive inverse of zero element is itself, so it belongs to $\mathbb{G}$.

	Let $x$ be a non-zero element of $\mathbb{F}$. Let ${(-x)}_{\mathbb{F}}$, ${(-x)}_{\mathbb{G}}$ be the additive inverses of $x$ within $\mathbb{F}$, $\mathbb{G}$ respectively.

	Since $\mathbb{G}$ and $\mathbb{F}$ share the same zero element, then by axioms (1) (2) (3)
	\begin{align*}
		{(-x)}_{\mathbb{G}} & = {(-x)}_{\mathbb{G}} + 0                         \\
		                    & = {(-x)}_{\mathbb{G}} + (x + {(-x)}_{\mathbb{F}}) \\
		                    & = ({(-x)}_{\mathbb{G}} + x) + {(-x)}_{\mathbb{F}} \\
		                    & = 0 + {(-x)}_{\mathbb{F}}                         \\
		                    & = {(-x)}_{\mathbb{F}}
	\end{align*}

	Let ${x^{-1}}_{\mathbb{F}}$, ${x^{-1}}_{\mathbb{G}}$ be the multiplicative inverses of $x$ within $\mathbb{F}$, $\mathbb{G}$ respectively.

	Since $\mathbb{G}$ and $\mathbb{F}$ share the same unit element, then by axioms (5) (7) (9)
	\begin{align*}
		{x^{-1}}_{\mathbb{G}} & = {x^{-1}}_{\mathbb{G}}\cdot 1                              \\
		                      & = {x^{-1}}_{\mathbb{G}}\cdot (x\cdot {x^{-1}}_{\mathbb{F}}) \\
		                      & = ({x^{-1}}_{\mathbb{G}}\cdot x)\cdot {x^{-1}}_{\mathbb{F}} \\
		                      & = 1\cdot {x^{-1}}_{\mathbb{F}}                              \\
		                      & = {x^{-1}}_{\mathbb{F}}
	\end{align*}
	Thus, the results follow.
\end{proof}

\subsection{Characteristic}

Let's begin with another example of field. Furthermore, it is a finite field $-$ it is called so because it has a finite number of elements.

\begin{theorem}
	The set of integers modulo $n$ ($n > 1$), with addition and multiplication modulo $n$, is a field if and only if $n$ is a prime.
\end{theorem}

The set of integers modulo $n$ is denoted by $\mathbb{Z}/n\mathbb{Z}$.

\begin{proof}
	$\mathbb{Z}/n\mathbb{Z}$, together with addition and multiplication modulo $n$ satisfies (1) $-$ (8).

	$(\Rightarrow)$ Let $a$ be an integer which is not a multiple of $n$, then $a$ and $n$ are coprime. According to Bezout's theorem, there exists integers $x$ and $y$ such that $ax + ny = 1$. Therefore, $ax\equiv 1\pmod{n}$, which indicates that $a$ has an multiplicative inverse. Hence $\mathbb{Z}/n\mathbb{Z}$ with addition and multiplication modulo $n$ is a field (it satisfy (9)).

	$(\Leftarrow)$ Let $a$ be an integer which is not a multiple of $n$. Since $\mathbb{Z}/n\mathbb{Z}$ is a field, then there exists an integer $x$ such that $ax\equiv 1\pmod{n}$, or equivalently, $ax - 1$ is a multiple of $n$. In other words, there exists an integer $y$ such that $ax - ny = 1$. The equality implies that $a$ and $n$ are coprime. So every integer, which is not a multiple of $n$, is coprime with $n$. This means $n$ is a prime.

	In conclusion, $\mathbb{Z}/n\mathbb{Z}$ with addition and multiplication modulo $n$ is a field iff $n$ is a prime.
\end{proof}

If $p$ is a prime, then the field $\mathbb{Z}/p\mathbb{Z}$ is denoted by $\mathbb{F}_{p}$ (to emphasize that it is a field). In $\mathbb{F}_{p}$, there is a phenomenon
\[
	\underbrace{1 + 1 + \cdots + 1}_{p} = 0 \qquad \text{more specific}\qquad \underbrace{1 + 1 + \cdots + 1}_{p} \equiv 0\pmod{p}
\]

Informally, it means we can add up $1$'s successively and end up with $0$ after a finite number of 1's. However, this phenomenon does not happen in the cases of rational, real, or complex numbers.

\begin{definition}[Characteristic of a field]
	Given a field $\mathbb{F}$, there are two possibilities
	\begin{itemize}
		\item There exists a positive integer $n$ such that if one add up $1$'s $n$ times, then the result will be $0$. The least of such positive numbers is called \textit{the characteristic of $\mathbb{F}$}.
		\item If such positive integer does not exist, we said $\mathbb{F}$ has characteristic zero.
	\end{itemize}
\end{definition}

\begin{note}
	In abstract algebra, the characteristic is defined for rings, which is more general than fields, in the same manner.

	The characteristic of a field cannot be $1$, since zero element and unit element of a field are distinct.

	Characterisitic of $\mathbb{F}$ is denoted by $\text{Char}(\mathbb{F})$.
\end{note}

\begin{theorem}
	Characteristic of a field is either $0$, or a prime.
\end{theorem}

\begin{proof}
	We will prove that if the characteristic of a field $\mathbb{F}$ is other than $0$, then it must be a prime.

	Suppose that the characteristic of $\mathbb{F}$ is a composite $n = ab$, where $1 < a, b < n$.

	According to the definition of characteristic: $\underbrace{1 + 1 + \cdots + 1}_{n} = 0$.

	Let $x = \underbrace{1 + 1 + \cdots + 1}_{a}, y = \underbrace{1 + 1 + \cdots + 1}_{b}$, then $xy = 0$. To be more details

	\begin{align*}
		xy & = x\cdot (\underbrace{1 + 1 + \cdots + 1}_{b})                                                                                  \\
		   & = \underbrace{x + x + \cdots + x}_{b}                                                                                           \\
		   & = \underbrace{\underbrace{1 + \cdots + 1}_{a} + \underbrace{1 + \cdots + 1}_{a} + \cdots + \underbrace{1 + \cdots + 1}_{a}}_{b} \\
		   & = \underbrace{1 + 1 + \cdots + 1}_{n}                                                                                           \\
		   & = 0.
	\end{align*}

	$xy = 0$ implies that $x = 0$ or $y = 0$. This is a contradiction to the definition of characteristic, since $n$ is the least positive integer such that $\underbrace{1 + 1 + \cdots + 1}_{n} = 0$.

	Thus if $\text{Char}(\mathbb{F})$ is other than zero, then it must be a prime.
\end{proof}

\begin{theorem}
	A field and its subfields have the same characteristic.
\end{theorem}

\begin{proof}
	Let's denote the field by $\mathbb{F}$. Let $\mathbb{G}$ be a subfield of $\mathbb{F}$.

	Since $\mathbb{F}$ and $\mathbb{G}$ share the same unit element, then $\underbrace{1 + 1 + \cdots + 1}_{n}$ belongs to $\mathbb{G}$ and $\mathbb{F}$ for any positive integer $n$.

	Hence it follows that $\text{Char}(\mathbb{F}) = \text{Char}(\mathbb{G})$.
\end{proof}

The exercises in the book assume the field that we are working with is a subfield of $\mathbb{C}$, unless it is explicitly stated.

\section{System of Linear Equations}

A fundamental problem in Linear Algebra is solving linear equations. We consider the problem of finding $n$ scalars of a field $\mathbb{F}$, which satisfy the conditions

\begin{equation}
	\begin{split}
		A_{11}x_{1} + A_{12}x_{2} + \cdots + A_{1n}x_{n} = y_{1}  \\
		A_{21}x_{1} + A_{22}x_{2} + \cdots + A_{2n}x_{n} = y_{2}  \\
		\vdots \\
		A_{m1}x_{1} + A_{m2}x_{2} + \cdots + A_{mn}x_{n} = y_{m}
	\end{split}\label{eq:general-system-of-linear-equations}
\end{equation}

where $y_{1}, y_{2}, \ldots, y_{m}$ and $A_{ij} (1\le i\le m, 1\le j\le n)$ are given scalars of $\mathbb{F}$.

(\ref{eq:general-system-of-linear-equations}) is called a \textbf{system of $m$ linear equations in $n$ unknowns}. An $n$-tuple $(x_{1}, \ldots, x_{n})$ of scalars of $\mathbb{F}$ which satisfies each of the linear equation is called a \textbf{solution} of the system. Particularly, if $y_{1} = y_{2} = \cdots = y_{m}$, people say that the system is \textbf{homogeneous}, each of the equation of the system is homogeneous.

The technique for finding the solutions of a system of linear equations is elimination (read the original text for an example).

We have a technique for solving any specific system of linear equations. But we wish to formalize it so that we may understand the essence of the technique and solve the systems in an organized manner.

For the general system (\ref{eq:general-system-of-linear-equations}), we choose $m$ scalars $c_{1}, \ldots, c_{m}$, multiply $j^{th}$ equation by $c_{j}$ then add up the equations. We obtain the following equation
\begin{multline*}
	(c_{1}A_{11} + c_{2}A_{21} + \cdots + c_{m}A_{m1})x_{1} + (c_{2}A_{12} + c_{2}A_{22} + \cdots + c_{m}A_{m2})x_{2} + \cdots + (c_{1}A_{1n} + c_{2}A_{2n} + \cdots + c_{m}A_{mn})x_{n} \\
	= c_{1}y_{1} + c_{2}y_{2} + \cdots + c_{m}y_{m}.
\end{multline*}
which is called a \textbf{linear combination} of the equations of (\ref{eq:general-system-of-linear-equations}). Notice that any solution to a system of linear equations is also solution to the linear combination of its equations. This is the fundamental idea of elimination. If we have another system of linear equations
\begin{equation}
	\begin{split}
		B_{11}x_{1} + B_{12}x_{2} + \cdots + B_{1n}x_{n} = z_{1}  \\
		B_{21}x_{1} + B_{22}x_{2} + \cdots + B_{2n}x_{n} = z_{2}  \\
		\vdots \\
		B_{m1}x_{1} + B_{m2}x_{2} + \cdots + B_{mn}x_{n} = z_{m}
	\end{split}\label{eq:new-system-of-linear-equations}
\end{equation}
such that each of its equation is a linear combination of the equations in (\ref{eq:general-system-of-linear-equations}). Then every solution of (\ref{eq:general-system-of-linear-equations}) is also a solution of the new system, but the converse can be false, which means there might be a solution of (\ref{eq:new-system-of-linear-equations}) which is not a solution of (\ref{eq:general-system-of-linear-equations}). However, the converse holds true if each equation of (\ref{eq:general-system-of-linear-equations}) is a linear combinations of equations in (\ref{eq:new-system-of-linear-equations}). From this remark, we say two systems of linear equations are \textbf{equivalent} if every equation in each system is a linear combination of the equations in the other system, and we also obtain the following result.

\begin{theorem}
	Equivalent systems of linear equations have exactly the same solutions.
\end{theorem}

\section*{Exercises}\addcontentsline{toc}{section}{Exercises}

\begin{exercise}
	Prove that the set of complex numbers of the form $a + b\sqrt{2}$, where $a, b$ are rational numbers, is a subfield of $\mathbb{C}$.
\end{exercise}

\begin{proof}
	Denote the set by $\mathbb{Q}(\sqrt{2})$. The addition and multiplication of $\mathbb{Q}(\sqrt{2})$ are defined as follows (these are usual addition and multiplication):
	\[
		\begin{split}
			(a_{1} + b_{1}\sqrt{2}) + (a_{2} + b_{2}\sqrt{2}) = (a_{1} + a_{2}) + (b_{1} + b_{2})\sqrt{2} \\
			(a_{1} + b_{1}\sqrt{2}) \cdot (a_{2} + b_{2}\sqrt{2}) = (a_{1}a_{2} + 2b_{1}b_{2}) + (a_{1}b_{2} + b_{1}a_{2})\sqrt{2}
		\end{split}
	\]

	\begin{enumerate}[label = (\arabic*)]
		\item Addition is associative.
		      \begin{align*}
			        & \left((a_{1} + b_{1}\sqrt{2}) + (a_{2} + b_{2}\sqrt{2})\right) + (a_{3} + b_{3}\sqrt{2})                    \\
			      = & \left((a_{1} + a_{2}) + (b_{1} + b_{2})\sqrt{2}\right) + (a_{3} + b_{3}\sqrt{2})                            \\
			      = & \left(\left((a_{1} + a_{2}) + a_{3}\right)\right) + \left(\left(b_{1} + b_{2}\right) + b_{3}\right)\sqrt{2} \\
			      = & \left(a_{1} + \left(a_{2} + a_{3}\right)\right) + \left(b_{1} + \left(b_{2} + b_{3}\right)\right)\sqrt{2}   \\
			      = & (a_{1} + b_{1}\sqrt{2}) + \left((a_{2} + a_{3}) + (b_{2} + b_{3})\sqrt{2}\right)                            \\
			      = & (a_{1} + b_{1}\sqrt{2}) + \left((a_{2} + b_{2}\sqrt{2}) + (a_{3} + b_{3}\sqrt{2})\right)
		      \end{align*}
		\item Addition has identity element.
		      \begin{align*}
			      (a + b\sqrt{2}) + 0 = (a + b\sqrt{2}) + (0 + 0\sqrt{2}) = (a + 0) + (b + 0)\sqrt{2} = a + b\sqrt{2} \\
			      0 + (a + b\sqrt{2}) = (0 + 0\sqrt{2}) + (a + b\sqrt{2}) = (0 + a) + (0 + b)\sqrt{2} = a + b\sqrt{2}
		      \end{align*}
		\item Each element has additive inverse.
		      \begin{align*}
			      (a + b\sqrt{2}) + ((-a) + (-b)\sqrt{2}) & = (a + (-a)) + (b + (-b))\sqrt{2} = 0 + 0\sqrt{2} = 0 \\
			      ((-a) + (-b)\sqrt{2}) + (a + b\sqrt{2}) & = ((-a) + a) + ((-b) + b)\sqrt{2} = 0 + 0\sqrt{2} = 0
		      \end{align*}
		\item Addition is commutative.
		      \begin{align*}
			      (a_{1} + b_{1}\sqrt{2}) + (a_{2} + b_{2}\sqrt{2}) & = (a_{1} + a_{2}) + (b_{1} + b_{2})\sqrt{2}         \\
			                                                        & = (a_{2} + a_{1}) + (b_{2} + b_{1})\sqrt{2}         \\
			                                                        & = (a_{2} + b_{2}\sqrt{2}) + (a_{1} + b_{1}\sqrt{2})
		      \end{align*}
		\item Multiplication is associative.
		      \begin{align*}
			        & \left((a_{1} + b_{1}\sqrt{2})\cdot (a_{2} + b_{2}\sqrt{2})\right)\cdot (a_{3} + b_{3}\sqrt{2})                                                                \\
			      = & \left((a_{1}a_{2} + 2b_{1}b_{2}) + (a_{1}b_{2} + b_{1}a_{2})\sqrt{2}\right)\cdot(a_{3} + b_{3}\sqrt{2})                                                       \\
			      = & (a_{1}a_{2}a_{3} + 2b_{1}b_{2}a_{3} + 2a_{1}b_{2}b_{3} + 2b_{1}a_{2}b_{3}) + (a_{1}a_{2}b_{3} + b_{1}a_{2}a_{3} + a_{1}b_{2}a_{3} + 2b_{1}b_{2}b_{3})\sqrt{2}
		      \end{align*}
		      \begin{align*}
			        & (a_{1} + b_{1}\sqrt{2})\cdot\left((a_{2} + b_{2}\sqrt{2})\cdot(a_{3} + b_{3}\sqrt{2})\right)                                                                  \\
			      = & (a_{1} + b_{1}\sqrt{2})\cdot\left( (a_{2}a_{3} + 2b_{2}b_{3}) + (a_{2}b_{3} + b_{2}a_{3})\sqrt{2} \right)                                                     \\
			      = & (a_{1}a_{2}a_{3} + 2a_{1}b_{2}b_{3} + 2b_{1}a_{2}b_{3} + 2b_{1}b_{2}a_{3}) + (a_{1}a_{2}b_{3} + a_{1}b_{2}a_{3} + b_{1}a_{2}a_{3} + 2b_{1}b_{2}b_{3})\sqrt{2}
		      \end{align*}
		\item Multiplication is distributive over addition.
		      \begin{align*}
			        & (a_{1} + b_{1}\sqrt{2})\cdot \left((a_{2} + b_{2}\sqrt{2}) + (a_{3} + b_{3}\sqrt{2})\right)                                                               \\
			      = & (a_{1} + b_{1}\sqrt{2})\cdot\left((a_{2} + a_{3}) + (b_{2} + b_{3})\sqrt{2}\right)                                                                        \\
			      = & (a_{1}a_{2} + a_{1}a_{3} + 2b_{1}(b_{2} + b_{3})) + (a_{1}b_{2} + a_{1}b_{3} + b_{1}a_{2} + b_{1}a_{3})\sqrt{2}                                           \\
			      = & \left((a_{1}a_{2} + 2b_{1}b_{2}) + (a_{1}b_{2} + b_{1}a_{2})\sqrt{2}\right) + \left((a_{1}a_{3} + 2b_{1}b_{3}) + (a_{1}b_{3} + b_{1}a_{3})\sqrt{2}\right) \\
			      = & (a_{1} + b_{1}\sqrt{2})\cdot (a_{2} + b_{2}\sqrt{2}) + (a_{1} + b_{1}\sqrt{2})\cdot (a_{3} + b_{3}\sqrt{2})
		      \end{align*}
		      \begin{align*}
			        & \left((a_{1} + b_{1}\sqrt{2}) + (a_{2} + b_{2}\sqrt{2})\right)\cdot (a_{3} + b_{3}\sqrt{2})                                                               \\
			      = & \left((a_{1} + a_{2}) + (b_{1} + b_{2})\sqrt{2}\right)\cdot (a_{3} + b_{3}\sqrt{2})                                                                       \\
			      = & \left(a_{1}a_{3} + a_{2}a_{3} + 2b_{1}b_{3} + 2b_{2}b_{3}\right) + (a_{1}b_{3} + a_{2}b_{3} + b_{1}a_{3} + b_{2}a_{3})\sqrt{2}                            \\
			      = & \left((a_{1}a_{3} + 2b_{1}b_{3}) + (a_{1}b_{3} + b_{1}a_{3})\sqrt{2}\right) + \left((a_{2}a_{3} + 2b_{2}b_{3}) + (a_{2}b_{3} + b_{2}a_{3})\sqrt{2}\right) \\
			      = & (a_{1} + b_{1}\sqrt{2})\cdot (a_{3} + b_{3}\sqrt{2}) + (a_{2} + b_{2}\sqrt{2})\cdot (a_{3} + b_{3}\sqrt{2})
		      \end{align*}
		\item Multiplication has identity element
		      \begin{align*}
			      (a + b\sqrt{2})\cdot 1 & = (a + b\sqrt{2})\cdot (1 + 0\sqrt{2}) = a + b\sqrt{2} \\
			      1\cdot (a + b\sqrt{2}) & = (1 + 0\sqrt{2})\cdot (a + b\sqrt{2}) = a + b\sqrt{2}
		      \end{align*}
		\item Multiplicative is commutative.
		      \begin{align*}
			      (a_{1} + b_{1}\sqrt{2})\cdot (a_{2} + b_{2}\sqrt{2}) & = (a_{1}a_{2} + 2b_{1}b_{2}) + (a_{1}b_{2} + b_{1}a_{2})\sqrt{2} \\
			                                                           & = (a_{2}a_{1} + 2b_{2}b_{1}) + (a_{2}b_{1} + b_{2}a_{1})\sqrt{2} \\
			                                                           & = (a_{2} + b_{2}\sqrt{2})\cdot (a_{1} + b_{1}\sqrt{2})
		      \end{align*}
		\item Each non-zero element has multiplicative inverse.

		      $a + b\sqrt{2}$ is zero element if and only if $a = b = 2$ (since $\sqrt{2}$ is irrational).

		      \begin{align*}
			      (a + b\sqrt{2})\cdot \left( \frac{a}{a^{2} - 2b^{2}} + \frac{(-b)}{a^{2} - 2b^{2}}\sqrt{2} \right) & = \frac{a^{2} - 2b^{2}}{a^{2} - 2b^{2}} + \frac{a(-b) + ba}{a^{2} - 2b^{2}}\sqrt{2} = 1 \\
			      \left( \frac{a}{a^{2} - 2b^{2}} + \frac{(-b)}{a^{2} - 2b^{2}}\sqrt{2} \right)\cdot (a + b\sqrt{2}) & = \frac{a^{2} - 2b^{2}}{a^{2} - 2b^{2}} + \frac{ab + (-b)a}{a^{2} - 2b^{2}}\sqrt{2} = 1
		      \end{align*}
	\end{enumerate}
\end{proof}

Let $\mathbb{F}$ be the field of complex numbers. Are the following two systems of linear equations equivalent? If so, express each equation in each system as a linear combination of the equations in the other system.

\begin{exercise}
	\[
		\begin{cases}
			x_{1} - x_{2} = 0 \\
			2x_{1} + x_{2} = 0
		\end{cases}
		\qquad
		\begin{cases}
			3x_{1} + x_{2} = 0 \\
			x_{1} + x_{2} = 0
		\end{cases}
	\]
\end{exercise}

\begin{proof}
	The two systems are equivalent.
	\[
		\begin{cases}
			3x_{1} + x_{2} = \frac{1}{3}(x_{1} - x_{2}) + \frac{4}{3}(2x_{1} + x_{2}) \\
			x_{1} + x_{2} =  \frac{-1}{3}(x_{1} - x_{2}) + \frac{2}{3}(2x_{1} + x_{2})
		\end{cases}
	\]
	\[
		\begin{cases}
			x_{1} - x_{2} = (3x_{1} + x_{2}) - 2(x_{1} + x_{2}) \\
			2x_{1} + x_{2} = \frac{1}{2}(3x_{1} + x_{2}) + \frac{1}{2}(x_{1} + x_{2})
		\end{cases}
	\]
\end{proof}

\begin{exercise}
	\[
		\begin{cases}
			-x_{1} + x_{2} + 4x_{3} = 0 \\
			x_{1} + 3x_{2} + 8x_{3} = 0 \\
			\frac{1}{2}x_{1} + x_{2} + \frac{5}{2}x_{3} = 0
		\end{cases}
		\qquad
		\begin{cases}
			x_{1} - x_{3} = 0 \\
			x_{2} + 3x_{3} = 0
		\end{cases}
	\]
\end{exercise}

\begin{proof}
	The two systems are equivalent.
	\[
		\begin{cases}
			-x_{1} + x_{2} + 4x_{3} = (-1)(x_{1} - x_{3}) + (x_{2} + 3x_{3}) \\
			x_{1} + 3x_{2} + 8x_{3} = (x_{1} - x_{3}) + 3(x_{2} + 3x_{3})    \\
			\frac{1}{2}x_{1} + x_{2} + \frac{5}{2}x_{3} = \frac{1}{2}(x_{1} - x_{3}) + (x_{2} + 3x_{3})
		\end{cases}
	\]
	\[
		\begin{cases}
			x_{1} - x_{3} = \frac{-2}{3}(-x_{1} + x_{2} + 4x_{3}) + \frac{2}{3}(\frac{1}{2}x_{1} + x_{2} + \frac{5}{2}x_{3}) \\
			x_{2} + 3x_{3} = \frac{1}{4}(-x_{1} + x_{2} + 4x_{3}) + \frac{1}{4}(x_{1} + 3x_{2} + 8x_{3})
		\end{cases}
	\]
\end{proof}

\begin{exercise}
	\[
		\begin{cases}
			2x_{1} + (-1 + \iota)x_{2} + x_{4} = 0 \\
			3x_{2} - 2\iota x_{3} + 5x_{4} = 0
		\end{cases}
		\qquad
		\begin{cases}
			\left(1 + \frac{\iota}{2}\right)x_{1} + 8x_{2} - \iota x_{3} - x_{4} = 0 \\
			\frac{2}{3}x_{1} - \frac{1}{2}x_{2} + x_{3} + 7x_{4} = 0
		\end{cases}
	\]
\end{exercise}

\begin{proof}
	Suppose that $\frac{2}{3}x_{1} - \frac{1}{2}x_{2} + x_{3} + 7x_{4}$ is a linear combination of $2x_{1} + (-1 + \iota)x_{2} + x_{4}$ and $3x_{2} - 2\iota x_{3} + 5x_{4}$, then there exists $a$ and $b$ such that
	\[
		\frac{2}{3}x_{1} - \frac{1}{2}x_{2} + x_{3} + 7x_{4} = a(2x_{1} + (-1 + \iota)x_{2} + x_{4}) + b(3x_{2} - 2\iota x_{3} + 5x_{4})
	\]
	By identifying coefficients
	\[
		\begin{cases}
			\frac{2}{3} = 2a             \\
			1 = -2\iota b                \\
			-\frac{1}{2} = (-1 + \iota)a \\
			7 = a + 5b
		\end{cases}
		\Longrightarrow
		\begin{cases}
			a = \frac{1}{3} \\
			a = \frac{1 + \iota}{4}
		\end{cases}
	\]
	So $\frac{2}{3}x_{1} - \frac{1}{2}x_{2} + x_{3} + 7x_{4}$ is NOT a linear combination of $2x_{1} + (-1 + \iota)x_{2} + x_{4}$ and $3x_{2} - 2\iota x_{3} + 5x_{4}$. Thus the two systems are inequivalent.
\end{proof}

\begin{exercise}
	Let $F$ be a set which contains exactly two elements, 0 and 1. Define an addition and multiplication by the tables:
	\[
		\begin{array}{c|cc}
			+ & 0 & 1 \\
			\hline
			0 & 0 & 1 \\
			1 & 1 & 0
		\end{array}
		\qquad
		\begin{array}{c|cc}
			\cdot & 0 & 1 \\
			\hline
			0     & 0 & 0 \\
			1     & 0 & 1
		\end{array}
	\]
	Verify that the set $F$, together with these two operations, is a field.
\end{exercise}

\begin{proof}
	\begin{enumerate}[label = (\arabic*)]
		\item Addition is associative.
		      \[
			      \begin{split}
				      &(0 + 0) + 0 = 0 + 0 = 0 + (0 + 0) \\
				      &(0 + 0) + 1 = 0 + 1 = 1 = 0 + 1 = 0 + (0 + 1) \\
				      &(0 + 1) + 0 = 1 + 0 = 1 = 0 + 1 = 0 + (1 + 0) \\
				      &(0 + 1) + 1 = 1 + 1 = 0 = 0 + 0 = 0 + (1 + 1) \\
				      &(1 + 0) + 0 = 1 + 0 = 1 + (0 + 0) \\
				      &(1 + 0) + 1 = 1 + 1 = 1 + (0 + 1) \\
				      &(1 + 1) + 0 = 0 + 0 = 0 = 1 + 1 = 1 + (1 + 0) \\
				      &(1 + 1) + 1 = 0 + 1 = 1 = 1 + 0 = 1 + (1 + 1)
			      \end{split}
		      \]
		\item Addition has identity element.
		      \[
			      \begin{split}
				      0 + 0 = 0 \\
				      1 + 0 = 1 = 0 + 1
			      \end{split}
		      \]
		\item Each element has additive inverse.
		      \[
			      \begin{split}
				      0 + 0 = 0 \\
				      1 + 1 = 0
			      \end{split}
		      \]
		\item Addition is commutative.
		      \[
			      \begin{split}
				      0 + 0 = 0 \\
				      1 + 1 = 0 \\
				      0 + 1 = 1 = 1 + 0
			      \end{split}
		      \]
		\item Multiplication is associative.
		      \[
			      \begin{split}
				      &(0 \cdot 0) \cdot 0 = 0 \cdot 0 = 0 \cdot (0 \cdot 0) \\
				      &(0 \cdot 0) \cdot 1 = 0 \cdot 1 = 0 = 0 \cdot 0 = 0 \cdot (0 \cdot 1) \\
				      &(0 \cdot 1) \cdot 0 = 0 \cdot 0 = 0 \cdot (1 \cdot 0) \\
				      &(0 \cdot 1) \cdot 1 = 0 \cdot 1 = 0 \cdot (1 \cdot 1) \\
				      &(1 \cdot 0) \cdot 0 = 0 \cdot 0 = 0 = 1 \cdot 0 = 1 \cdot (0 \cdot 0) \\
				      &(1 \cdot 0) \cdot 1 = 0 \cdot 1 = 0 = 1 \cdot 0 = 1 \cdot (0 \cdot 1) \\
				      &(1 \cdot 1) \cdot 0 = 1 \cdot 0 = 0 = 1 \cdot 0 = 1 \cdot (1 \cdot 0) \\
				      &(1 \cdot 1) \cdot 1 = 1 \cdot 1 = 1 \cdot (1 \cdot 1)
			      \end{split}
		      \]
		\item Multiplication is distributive over addition.
		      \[
			      \begin{split}
				      &0\cdot(0 + 0) = 0\cdot 0 = 0 = 0 + 0 = 0\cdot 0 + 0\cdot 0 \\
				      &0\cdot(0 + 1) = 0\cdot 1 = 0 = 0 + 0 = 0\cdot 0 + 0\cdot 1 \\
				      &0\cdot(1 + 0) = 0\cdot 1 = 0 = 0 + 0 = 0\cdot 1 + 0\cdot 0 \\
				      &0\cdot(1 + 1) = 0\cdot 0 = 0 = 0 + 0 = 0\cdot 1 + 0\cdot 1 \\
				      &1\cdot(0 + 0) = 1\cdot 0 = 0 = 0 + 0 = 1\cdot 0 + 1\cdot 0 \\
				      &1\cdot(0 + 1) = 1\cdot 1 = 1 = 0 + 1 = 1\cdot 0 + 1\cdot 1 \\
				      &1\cdot(1 + 0) = 1\cdot 1 = 1 = 1 + 0 = 1\cdot 1 + 1\cdot 0 \\
				      &1\cdot(1 + 1) = 1\cdot 0 = 0 = 1 + 1 = 1\cdot 1 + 1\cdot 1
			      \end{split}
		      \]
		      \[
			      \begin{split}
				      &(0 + 0)\cdot 0 = 0\cdot 0 = 0 = 0 + 0 = 0\cdot 0 + 0\cdot 0 \\
				      &(0 + 0)\cdot 1 = 0\cdot 1 = 0 = 0 + 0 = 0\cdot 1 + 0\cdot 1 \\
				      &(0 + 1)\cdot 0 = 1\cdot 0 = 0 = 0 + 0 = 0\cdot 0 + 1\cdot 0 \\
				      &(0 + 1)\cdot 1 = 1\cdot 1 = 1 = 0 + 1 = 0\cdot 1 + 1\cdot 1 \\
				      &(1 + 0)\cdot 0 = 1\cdot 0 = 0 = 0 + 0 = 1\cdot 0 + 0\cdot 0 \\
				      &(1 + 0)\cdot 1 = 1\cdot 1 = 1 = 1 + 0 = 1\cdot 1 + 0\cdot 1 \\
				      &(1 + 1)\cdot 0 = 0\cdot 0 = 0 = 0 + 0 = 1\cdot 0 + 1\cdot 0 \\
				      &(1 + 1)\cdot 1 = 0\cdot 1 = 0 = 1 + 1 = 1\cdot 1 + 1\cdot 1
			      \end{split}
		      \]
		\item Multiplication has identity element
		      \[
			      \begin{split}
				      &0\cdot 1 = 0 = 1\cdot 0 \\
				      &1\cdot 1 = 1
			      \end{split}
		      \]
		\item Multiplication is commutative.
		      \[
			      \begin{split}
				      &0\cdot 0 = 0 \\
				      &1\cdot 1 = 1 \\
				      &0\cdot 1 = 1\cdot 0 = 0
			      \end{split}
		      \]
		\item Each non-zero element has multiplicative inverse.
		      \[
			      1\cdot 1 = 1
		      \]
	\end{enumerate}
\end{proof}

\begin{exercise}
	Prove that if two homogeneous systems of linear equations in two unknowns have the same solutions, then they are equivalent.
\end{exercise}

\begin{proof}
	A homogeneous linear equation always has trivial solution, which contains only zero.

	A homogeneous linear equation is called trivial if and only if all of its coefficients are zero.

	\begin{lemma}\label{lemma:exercise:solution-to-homogeneous-linear-equation}
		Let $(a_{1}, a_{2})\ne (0, 0)$, then solutions to $a_{1}x_{1} + a_{2}x_{2} = 0$ are of the form $(k\cdot t_{1}, k\cdot t_{2})$, where $(t_{1}, t_{2})$ is solution other than $(0, 0)$ and $k$ is a scalar.
	\end{lemma}
	\begin{proof}[Proof of the Lemma]
		Let $t_{1} = -a_{2}$ and $t_{2} = a_{1}$, then $(t_{1}, t_{2})$ is a solution other than $(0, 0)$ of the equation. Without loss of generality, suppose that $t_{1}\ne 0$, then $a_{2}\ne 0$.

		Let $(y_{1}, y_{2})$ be a solution to $a_{1}x_{1} + a_{2}x_{2} = 0$. Since $t_{1}\ne 0$, then there exists a scalar $k$ such that $k\cdot t_{1} = y_{1}$.
		\begin{align*}
			                & (a_{1}y_{1} + a_{2}y_{2}) - (a_{1}kt_{1} + a_{2}kt_{2}) = 0  \\
			\Leftrightarrow & (a_{1}kt_{1} + a_{2}y_{2}) - (a_{1}kt_{1} + a_{2}kt_{2}) = 0 \\
			\Leftrightarrow & a_{2}y_{2} - a_{2}kt_{2} = 0                                 \\
			\Leftrightarrow & y_{2} = k\cdot t_{2} \qquad\text{(since $a_{2}\ne 0$)}
		\end{align*}
		Hence $y_{1} = kt_{1}, y_{2} = kt_{2}$.
	\end{proof}

	\begin{lemma}\label{lemma:exercise:equivalent-homogeneous-linear-equations}
		Two equations $a_{1}x_{1} + a_{2}x_{2} = 0$ and $b_{1}x_{1} + b_{2}x_{2} = 0$ have the same solution if and only if there exists two non-zero scalars $a, b$ such that
		\[
			\begin{split}
				a(a_{1}x_{1} + a_{2}x_{2}) = b_{1}x_{1} + b_{2}x_{2} \\
				b(b_{1}x_{1} + b_{2}x_{2}) = a_{1}x_{1} + a_{2}x_{2}
			\end{split}
		\]
	\end{lemma}
	\begin{proof}[Proof of the Lemma]
		($\Rightarrow$) If there exists such scalars, then solutions to $a_{1}x_{1} + a_{2}x_{2} = 0$ are solutions to $b_{1}x_{1} + b_{2}x_{2} = 0$ and vice versa.

		($\Leftarrow$) A homogeneous linear equation always has non-trivial solution (solution other than zeroes).

		Let's consider two cases
		\begin{enumerate}[label = \textbf{Case \arabic*.}, itemindent=1cm]
			\item Any pair of scalars is a solution to both systems.
			      This means $a_{1} = a_{2} = b_{1} = b_{2} = 0$. Hence
			      \[
				      \begin{split}
					      1\cdot (a_{1}x_{1} + a_{2}x_{2}) = b_{1}x_{1} + b_{2}x_{2} \\
					      1\cdot (b_{1}x_{1} + b_{2}x_{2}) = a_{1}x_{1} + a_{2}x_{2}
				      \end{split}
			      \]
			\item (Due to the previous lemma) Otherwise, solutions to both system are of the form $(k\cdot t_{1}, k\cdot t_{2})$, where $(t_{1}, t_{2})\ne (0, 0)$.

			      $(-a_{2}, a_{1})$ is a non-trivial solution to $a_{1}x_{1} + a_{2}x_{2} = 0$.

			      $(-b_{2}, b_{1})$ is a non-trivial solution to $b_{1}x_{1} + b_{2}x_{2} = 0$.

			      $(t_{1}, t_{2})$ is a non-trivial solution to both equations, then there exists non-zero scalars $k, \ell$ such that
			      \[
				      \begin{split}
					      -a_{2} = kt_{1}, a_{1} = kt_{2} \\
					      -b_{2} = \ell t_{1}, b_{1} = \ell t_{2}
				      \end{split}
			      \]
			      So
			      \[
				      \begin{split}
					      a_{1} = k\ell^{-1}b_{1}, a_{2} = k\ell^{-1}b_{2} \\
					      b_{1} = k^{-1}\ell a_{1}, b_{2} = k^{-1}\ell a_{2}
				      \end{split}
			      \]
			      which implies
			      \[
				      \begin{split}
					      a_{1}x_{1} + a_{2}x_{2} = k\ell^{-1}(b_{1}x_{1} + b_{2}x_{2}) \\
					      b_{1}x_{1} + b_{2}x_{2} = k^{-1}\ell(a_{1}x_{1} + a_{2}x_{2})
				      \end{split}
			      \]
			      Hence two systems are equivalent.\qedhere
		\end{enumerate}
	\end{proof}

	\begin{lemma}\label{lemma:exercise:linear-combination-of-inequivalent-linear-equations}
		Two homogeneous linear equations $a_{1}x_{1} + a_{2}x_{2} = 0$ and $b_{1}x_{1} + b_{2}x_{2} = 0$ are inequavalent and non-trivial.

		Then any homogeneous linear equation in two unknowns $x_{1}, x_{2}$ is a linear combination of these two equations.
	\end{lemma}
	\begin{proof}
		Let's $c_{1}x_{1} + c_{2}x_{2} = 0$ be a homogeneous linear equation in two unknowns $x_{1}, x_{2}$.

		\[
			t_{1} = \frac{c_{1}b_{2} - c_{2}b_{1}}{a_{1}b_{2} - a_{2}b_{1}}\qquad t_{2} = \frac{a_{1}c_{2} - a_{2}c_{1}}{a_{1}b_{2} - a_{2}b_{1}}
		\]
		then $c_{1}x_{1} + c_{2}x_{2} = t_{1}(a_{1}x_{1} + a_{2}x_{2}) + t_{2}(b_{1}x_{1} + b_{2}x_{2})$.
	\end{proof}

	Let the two systems be
	\[
		\begin{cases}
			A_{11}x_{1} + A_{12}x_{2} = 0 \\
			A_{21}x_{1} + A_{22}x_{2} = 0 \\
			\vdots                        \\
			A_{m1}x_{1} + A_{m2}x_{2} = 0
		\end{cases}
		\quad\text{and}\quad
		\begin{cases}
			B_{11}x_{1} + B_{12}x_{2} = 0 \\
			B_{21}x_{1} + B_{22}x_{2} = 0 \\
			\vdots                        \\
			B_{n1}x_{1} + B_{n2}x_{2} = 0
		\end{cases}
	\]
	and suppose that (without loss of generality) $(A_{i1}, A_{i2})\ne (0, 0)$ and $(B_{j1}, B_{j2})\ne (0, 0)$.

	\begin{enumerate}[label = \textbf{Case \arabic*.}, itemindent=1cm]
		\item Any pairs of scalars is a solution to both systems.

		      This implies that all coefficients of the two systems are zero, hence the two systems are equivalent.
		\item Solutions of the two systems are of the form $(kp_{1}, kp_{2})$, where $(p_{1}, p_{2})\ne (0, 0)$.

		      Let $c_{1}, c_{2}, \ldots c_{m}$ be $m$ scalars
		      \[
			      (c_{1}A_{11} + c_{2}A_{21} + \cdots + c_{m}A_{m1})x_{1} + (c_{1}A_{12} + c_{2}A_{22} + \cdots + c_{m}A_{m2})x_{2}
		      \]
		      is a linear combination of $A_{i1}x_{1} + A_{i2}x_{2}$ such that
		      \[
			      (c_{1}A_{11} + c_{2}A_{21} + \cdots + c_{m}A_{m1}, c_{1}A_{12} + c_{2}A_{22} + \cdots + c_{m}A_{m2})\ne (0, 0).
		      \]
		      Solutions to $B_{i1}x_{1} + B_{i2}x_{2} = 0$ are solutions to
		      \[
			      (c_{1}A_{11} + c_{2}A_{21} + \cdots + c_{m}A_{m1})x_{1} + (c_{1}A_{12} + c_{2}A_{22} + \cdots + c_{m}A_{m2})x_{2} = 0
		      \]
		      Then there exists non-zero scalar $k_{i}$ such that
		      \[
			      \begin{split}
				      k_{i}B_{i1} = c_{1}A_{11} + c_{2}A_{21} + \cdots + c_{m}A_{m1} \\
				      k_{i}B_{i2} = c_{1}A_{12} + c_{2}A_{22} + \cdots + c_{m}A_{m2}
			      \end{split}
			      \qquad\Longrightarrow\qquad
			      \begin{split}
				      B_{i1} = k_{i}^{-1}\left(c_{1}A_{11} + c_{2}A_{21} + \cdots + c_{m}A_{m1}\right) \\
				      B_{i2} = k_{i}^{-1}\left(c_{1}A_{12} + c_{2}A_{22} + \cdots + c_{m}A_{m2}\right)
			      \end{split}
		      \]
		      Therefore $B_{i1}x_{1} + B_{i2}x_{2}$ is a linear combination of linear equations in the first system.

		      Analogously, $A_{i1}x_{1} + B_{i2}x_{2}$ is a linear combination of linear equations in the second system.

		\item The only solution to both systems is $(0, 0)$.

		      The assumption implies that
		      \begin{itemize}
			      \item Within the first system, there are two linear equations which are non-trivial and inequivalent.

			            According to Lemma~\ref{lemma:exercise:linear-combination-of-inequivalent-linear-equations}, each equation of the second system is a linear combination of the two inequivalent equations of the first.
			      \item Within the second system, there are two linear equations which are non-trivial and inequivalent.

			            According to Lemma~\ref{lemma:exercise:linear-combination-of-inequivalent-linear-equations}, each equation of the first system is a linear combination of the two inequivalent equations of the second.
		      \end{itemize}
	\end{enumerate}

	In conclusion, the two systems are equivalent.
\end{proof}

\begin{exercise}
	Prove that each subfield of the field of complex numbers contains every rational number.
\end{exercise}

\begin{proof}
	Each subfield of the field of complex numbers contains $0$ and $1$.

	Then $n = \underbrace{1 + 1 + \cdots + 1}_{n}$ belongs to the subfield, for any natural number $n$. Consequently, $-n$ belongs to the subfield.

	Let $q$ be a non-zero integer, $p$ be an integer, then $p, q$ belongs to the subfield.

	According to Theorem~\ref{thm:inverse-elements-of-subfield}, $q^{-1}$ belongs to the subfield, then $\frac{p}{q} = pq^{-1}$ belongs to the subfield.

	Thus, the subfield contains every rational number.
\end{proof}

\begin{exercise}
	Prove that each field of characteristic zero contains a copy of the rational number field.
\end{exercise}

\begin{note}
	The term ``copy'' is used informally. A copy of the rational number field is a field which is isomorphic to the rational number field. It means, there exists a bijection $f: \mathbb{Q}\to\mathbb{F}$ such that
	\[
		f(x) + f(y) = f(x + y)\qquad f(x)f(y) = f(xy).
	\]
	Such bijection is called a \textit{field isomorphism}.
\end{note}

\begin{proof}
	Denote the field by $\mathbb{F}$.

	$\mathbb{F}$ has two distinct elements $0_{\mathbb{F}}$ and $1_{\mathbb{F}}$, which are the additive identity and multiplicative identity, respectively.

	Since $\text{Char}(\mathbb{F}) = 0$, then for any pair of distinct positive integers $(p, q)$
	\[
		\underbrace{1_{\mathbb{F}} + 1_{\mathbb{F}} + \cdots + 1_{\mathbb{F}}}_{p} \ne \underbrace{1_{\mathbb{F}} + 1_{\mathbb{F}} + \cdots + 1_{\mathbb{F}}}_{q}
	\]

	We construct an isomorphism as follows:
	\[
		\begin{split}
			f:&\quad 0 \mapsto 0_{\mathbb{F}} \\
			&\quad 1 \mapsto 1_{\mathbb{F}} \\
			&\quad n \mapsto \underbrace{1_{\mathbb{F}} + 1_{\mathbb{F}} + \cdots + 1_{\mathbb{F}}}_{n} \\
			&\quad -n \mapsto \underbrace{(-1_{\mathbb{F}}) + (-1_{\mathbb{F}}) + \cdots + (-1_{\mathbb{F}})}_{n} \\
			&\quad \frac{p}{q} \mapsto \frac{\underbrace{1_{\mathbb{F}} + 1_{\mathbb{F}} + \cdots + 1_{\mathbb{F}}}_{p}}{\underbrace{1_{\mathbb{F}} + 1_{\mathbb{F}} + \cdots + 1_{\mathbb{F}}}_{q}} \\
			&\quad \frac{-p}{q} \mapsto \frac{\underbrace{(-1_{\mathbb{F}}) + (-1_{\mathbb{F}}) + \cdots + (-1_{\mathbb{F}})}_{p}}{\underbrace{1_{\mathbb{F}} + 1_{\mathbb{F}} + \cdots + 1_{\mathbb{F}}}_{q}}
		\end{split}
	\]

    Hence, $\mathbb{F}$ contains a subfield which is isomorphic to the field of the rational numbers.
\end{proof}

\section{Matrices and Elementary Row Operations}

\section{Row-Reduced Echelon Matrices}

\section*{Exercises}\addcontentsline{toc}{section}{Exercises}

\section{Matrix Multiplication}

\section{Invertible Matrices}

\section*{Exercises}\addcontentsline{toc}{section}{Exercises}
