\chapter{Riemann Integration}

\section{Review: Riemann Integral}

% chapter1:sectionA:exercise1
\begin{exercise}\label{chapter1:sectionA:exercise1}
    Suppose $f: \closedinterval{a,b}\to\mathbb{R}$ is a bounded function such that
    \[
        L(f, P, \closedinterval{a, b}) = U(f, P, \closedinterval{a, b})
    \]

    for some partition $P$ of $\closedinterval{a,b}$. Prove that $f$ is a constant function on $\closedinterval{a, b}$.
\end{exercise}

\begin{proof}
    Let $P$ be the list $a = x_{0} < x_{1} < \cdots < x_{n} = b$.
    \begin{align*}
        L(f, P, \closedinterval{a, b}) & = \sum^{n}_{i=1}(x_{i} - x_{i-1})\inf\limits_{\closedinterval{x_{i-1}, x_{i}}}f, \\
        U(f, P, \closedinterval{a, b}) & = \sum^{n}_{i=1}(x_{i} - x_{i-1})\sup\limits_{\closedinterval{x_{i-1}, x_{i}}}f.
    \end{align*}

    If there exists a closed interval $\closedinterval{x_{i-1}, x_{i}}$ such that
    \[
        \inf\limits_{\closedinterval{x_{i-1}, x_{i}}}f < \sup\limits_{\closedinterval{x_{i-1}, x_{i}}}f
    \]

    then $L(f, P, \closedinterval{a,b}) < U(f, P, \closedinterval{a, b})$. Therefore, for every closed interval $\closedinterval{x_{i-1}, x_{i}}$ (where $i=1,\ldots,n$),
    \[
        \inf\limits_{\closedinterval{x_{i-1}, x_{i}}}f = \sup\limits_{\closedinterval{x_{i-1}, x_{i}}}f.
    \]

    Hence $f\vert_{\closedinterval{x_{i-1}, x_{i}}}$ is a constant function for every $i = 1,\ldots,n$. Thus $f$ is a constant function on $\closedinterval{a,b}$.
\end{proof}
\newpage

% chapter1:sectionA:exercise2
\begin{exercise}\label{chapter1:sectionA:exercise2}
    Suppose $a\leq s < t\leq b$. Define $f: \closedinterval{a, b}\to\mathbb{R}$ by
    \[
        f(x) = \begin{cases}
            1 & \text{if $s < x < t$}, \\
            0 & \text{otherwise.}
        \end{cases}
    \]

    Prove that $f$ is Riemann integrable on $\closedinterval{a, b}$ and that $\int^{b}_{a}f = t - s$.
\end{exercise}

\begin{proof}
    $P$ is the partition $a = x_{0} < x_{1} < \cdots < x_{n} = b$. Let $i$ be the smallest positive integer such that $s\in\closedinterval{x_{i-1}, x_{i}}$ and $j$ be the largest positive integer such that $t\in\closedinterval{x_{j-1}, x_{j}}$. Then
    \[
        \begin{split}
            L(f, P, \closedinterval{a, b}) = \sum^{n}_{k=1}(x_{k} - x_{k-1})\inf\limits_{\closedinterval{x_{k-1}, x_{k}}}f = \sum^{j-1}_{k=i+1}(x_{k} - x_{k-1})\inf\limits_{\closedinterval{x_{k-1}, x_{k}}}f = x_{j-1} - x_{i} \\
            U(f, P, \closedinterval{a, b}) = \sum^{n}_{k=1}(x_{k} - x_{k-1})\sup\limits_{\closedinterval{x_{k-1}, x_{k}}}f = \sum^{j}_{k=i}(x_{k} - x_{k-1})\sup\limits_{\closedinterval{x_{k-1}, x_{k}}}f  = x_{j} - x_{i-1}
        \end{split}
    \]

    Moreover,
    \[
        \begin{split}
            \sup\limits_{P}(x_{j-1} - x_{i}) = t - s \\
            \inf\limits_{P}(x_{j} - x_{i-1}) = t - s
        \end{split}
    \]

    because $x_{j-1} - x_{i} < t - s < x_{j} - x_{i-1}$ and we can choose partitions such that $\closedinterval{x_{j-1}, x_{j}}$ and $\closedinterval{x_{i-1}, x_{i}}$ are arbitrarily small.

    so $L(f, \closedinterval{a, b}) = U(f, \closedinterval{a, b}) = t - s$. Thus $f$ is Rieman integrable on $\closedinterval{a, b}$ and $\int^{b}_{a}f = t - s$.
\end{proof}
\newpage

% chapter1:sectionA:exercise3
\begin{exercise}\label{chapter1:sectionA:exercise3}
    Suppose $f: \closedinterval{a, b}\to \mathbb{R}$ is a bounded function. Prove that $f$ is Riemann integrable if and only if for each $\varepsilon > 0$, there exists a partition $P$ of $\closedinterval{a, b}$ such that
    \[
        U(f, P, \closedinterval{a, b}) - L(f, P, \closedinterval{a, b}) < \varepsilon.
    \]
\end{exercise}

\begin{proof}
    Suppose $f$ is Riemann integrable.

    Because
    \[
        \begin{split}
            \sup\limits_{P} L(f, P, \closedinterval{a, b}) = L(f, \closedinterval{a, b}), \\
            \inf\limits_{P} U(f, P, \closedinterval{a, b}) = U(f, \closedinterval{a, b})
        \end{split}
    \]

    then for every $\varepsilon > 0$, there exist partitions $P_{1}$ and $P_{2}$ of the closed interval $\closedinterval{a, b}$ such that
    \[
        \begin{split}
            L(f, \closedinterval{a, b}) - L(f, P_{1}, \closedinterval{a, b}) < \frac{\varepsilon}{2} \\
            U(f, P_{2}, \closedinterval{a, b}) - U(f, \closedinterval{a, b}) < \frac{\varepsilon}{2}.
        \end{split}
    \]

    Let $P$ be the partition on $\closedinterval{a, b}$ whose points are obtained by merging points of $P_{1}$ and $P_{2}$, then
    \[
        \begin{split}
            L(f, \closedinterval{a, b}) - L(f, P, \closedinterval{a, b}) < L(f, \closedinterval{a, b}) - L(f, P_{1}, \closedinterval{a, b}) < \frac{\varepsilon}{2}, \\
            U(f, P, \closedinterval{a, b}) - U(f, \closedinterval{a, b}) < U(f, P_{2}, \closedinterval{a, b}) - U(f, \closedinterval{a, b}) < \frac{\varepsilon}{2}.
        \end{split}
    \]

    Moreover, because $f$ is Riemann integrable, then $L(f, \closedinterval{a, b}) = U(f, \closedinterval{a, b})$. Hence
    \[
        U(f, P, \closedinterval{a, b}) - L(f, P, \closedinterval{a, b}) < \frac{\varepsilon}{2} + \frac{\varepsilon}{2} = \varepsilon.
    \]

    Thus for each $\varepsilon > 0$, there exists a partition $P$ of $\closedinterval{a, b}$ such that
    \[
        U(f, P, \closedinterval{a,b}) - L(f, P, \closedinterval{a,b}) < \varepsilon.
    \]

    \bigskip
    Suppose for each $\varepsilon > 0$, there exists a partition $P$ of $\closedinterval{a, b}$ such that
    \[
        U(f, P, \closedinterval{a,b}) - L(f, P, \closedinterval{a,b}) < \varepsilon.
    \]

    Because $f$ is bounded, there exist $L(f, \closedinterval{a, b})$ and $U(f, \closedinterval{a, b})$.
    Then for each $\varepsilon > 0$, there exists a partition $P$ of $\closedinterval{a, b}$ such that
    \[
        U(f, P, \closedinterval{a,b}) - L(f, P, \closedinterval{a,b}) < \varepsilon.
    \]

    and it follows that
    \[
        U(f, \closedinterval{a, b}) - L(f, \closedinterval{a, b}) \leq U(f, P, \closedinterval{a, b}) - L(f, P, \closedinterval{a, b}) < \varepsilon.
    \]

    Hence for each $\varepsilon > 0$, $0\leq U(f, \closedinterval{a, b}) - L(f, \closedinterval{a, b}) < \varepsilon$. Therefore $U(f, \closedinterval{a, b}) = L(f, \closedinterval{a, b})$. Thus $f$ is Riemann integrable on $\closedinterval{a, b}$.
\end{proof}
\newpage

% chapter1:sectionA:exercise4
\begin{exercise}\label{chapter1:sectionA:exercise4}
    Suppose $f, g: \closedinterval{a, b}\to\mathbb{R}$ are Riemann integrable. Prove that $f + g$ is Riemann integrable on $\closedinterval{a, b}$ and
    \[
        \int^{b}_{a}(f + g) = \int^{b}_{a}f + \int^{b}_{a}g.
    \]
\end{exercise}

\begin{proof}
    For each partition $P$ on $\closedinterval{a, b}$,
    \begin{align*}
        L(f + g, P, \closedinterval{a, b}) & = \sum^{n}_{i=1}(x_{i} - x_{i-1})\inf\limits_{\closedinterval{x_{i-1}, x_{i}}}(f + g)                                                                           \\
                                           & \geq \sum^{n}_{i=1}(x_{i} - x_{i-1})\left(\inf\limits_{\closedinterval{x_{i-1}, x_{i}}}f + \inf\limits_{\closedinterval{x_{i-1}, x_{i}}}g\right)                \\
                                           & = \sum^{n}_{i=1}(x_{i} - x_{i-1})\inf\limits_{\closedinterval{x_{i-1}, x_{i}}}f + \sum^{n}_{i=1}(x_{i} - x_{i-1})\inf\limits_{\closedinterval{x_{i-1}, x_{i}}}g \\
                                           & = L(f, P, \closedinterval{a, b}) + L(g, P, \closedinterval{a, b})                                                                                               \\
        U(f + g, P, \closedinterval{a, b}) & = \sum^{n}_{i=1}(x_{i} - x_{i-1})\sup_{\closedinterval{x_{i-1}, x_{i}}}(f + g)                                                                                  \\
                                           & \leq \sum^{n}_{i=1}(x_{i} - x_{i-1})\left(\sup_{\closedinterval{x_{i-1}, x_{i}}}f + \sup_{\closedinterval{x_{i-1}, x_{i}}}g\right)                              \\
                                           & = \sum^{n}_{i=1}(x_{i} - x_{i-1})\sup_{\closedinterval{x_{i-1}, x_{i}}}f + \sum^{n}_{i=1}(x_{i} - x_{i-1})\sup_{\closedinterval{x_{i-1}, x_{i}}}g               \\
                                           & = U(f, P, \closedinterval{a, b}) + U(g, P, \closedinterval{a, b}).
    \end{align*}

    Take supremum and infimum, we obtain
    \[
        \begin{split}
            L(f, \closedinterval{a, b}) + L(g, \closedinterval{a, b}) \leq L(f + g, \closedinterval{a, b}) \leq U(f + g, \closedinterval{a, b})\leq U(f, \closedinterval{a, b}) + U(g, \closedinterval{a, b})
        \end{split}
    \]

    and $L(f, \closedinterval{a, b}) = U(f, \closedinterval{a, b})$, $L(g, \closedinterval{a, b}) = U(f, \closedinterval{a, b})$ since $f, g$ are Riemann integrable, it follows that
    \[
        L(f+g, \closedinterval{a, b}) = U(f+g, \closedinterval{a, b}) = \int^{b}_{a}f + \int^{b}_{a}g.
    \]

    Hence $f + g$ is Riemann integrable on $\closedinterval{a, b}$ and
    \[
        \int^{b}_{a}(f + g) = \int^{b}_{a}f + \int^{b}_{a}g.\qedhere
    \]
\end{proof}
\newpage

% chapter1:sectionA:exercise5
\begin{exercise}\label{chapter1:sectionA:exercise5}
    Suppose $f: \closedinterval{a, b}\to\mathbb{R}$ is Riemann integrable. Prove that the function $-f$ is Riemann integrable on $\closedinterval{a, b}$ and
    \[
        \int^{b}_{a}(-f) = -\int^{b}_{a}f.
    \]
\end{exercise}

\begin{proof}
    For every partition $P$ on $\closedinterval{a, b}$,
    \[
        \begin{split}
            L(-f, P, \closedinterval{a, b}) = \sum^{n}_{i=1}(x_{i} - x_{i-1})\inf\limits_{\closedinterval{x_{i-1}, x_{i}}}(-f) = \sum^{n}_{i=1}(x_{i} - x_{i-1})\left(-\sup\limits_{\closedinterval{x_{i-1}, x_{i}}}f\right) = -U(f, P, \closedinterval{a, b}), \\
            U(-f, P, \closedinterval{a, b}) = \sum^{n}_{i=1}(x_{i} - x_{i-1})\sup\limits_{\closedinterval{x_{i-1}, x_{i}}}(-f) = \sum^{n}_{i=1}(x_{i} - x_{i-1})\left(-\inf\limits_{\closedinterval{x_{i-1}, x_{i}}}f\right) = -L(f, P, \closedinterval{a, b}).
        \end{split}
    \]

    Take supremum and infimum
    \[
        \begin{split}
            L(-f, \closedinterval{a, b}) = \sup\limits_{P}L(-f, P, \closedinterval{a, b}) = \sup\limits_{P}\left(-U(f, P, \closedinterval{a, b})\right) = -\inf\limits_{P} U(f, P, \closedinterval{a, b}) = -U(f, \closedinterval{a,b}), \\
            U(-f, \closedinterval{a, b}) = \inf\limits_{P}U(-f, P, \closedinterval{a, b}) = \inf\limits_{P}\left(-L(f, P, \closedinterval{a, b})\right) = -\sup\limits_{P}L(f, P, \closedinterval{a, b}) = -L(f, \closedinterval{a, b}).
        \end{split}
    \]

    Because $f$ is Riemann integrable, $U(f, \closedinterval{a, b}) = L(f, \closedinterval{a, b})$. Therefore
    \[
        L(-f, \closedinterval{a, b}) = U(-f, \closedinterval{a, b}).
    \]

    Thus $-f$ is Riemann integrable on $\closedinterval{a, b}$ and
    \[
        \int^{b}_{a}(-f) = -\int^{b}_{a}f.\qedhere
    \]
\end{proof}
\newpage

% chapter1:sectionA:exercise6
\begin{exercise}\label{chapter1:sectionA:exercise6}
    Suppose $f: \closedinterval{a,b}\to\mathbb{R}$ is Riemann integrable. Suppose $g: \closedinterval{a, b}\to\mathbb{R}$ is a function such that $g(x) = f(x)$ for all except finitely many $x\in\closedinterval{a, b}$. Prove that $g$ is Riemann integrable on $\closedinterval{a, b}$ and
    \[
        \int^{b}_{a}g = \int^{b}_{a}f.
    \]
\end{exercise}

\begin{proof}
    Let $c_{1} < \cdots < c_{m}$ be the points on $\closedinterval{a, b}$ such that $g(x)\ne f(x)$.

    Let $\varepsilon > 0$ and $P$ be a partition $a = x_{0} < x_{1} < \cdots < x_{n} = b$ on $\closedinterval{a, b}$ such that
    \begin{itemize}
        \item $c_{1}, \ldots, c_{m}$ belong to different \textbf{open} intervals $\openinterval{x_{i_{k}-1}, x_{i_{k}}}\subsetneq\closedinterval{x_{i_{k} - 1}, x_{i_{k}}}$.
        \item for every $k = 1,\ldots,m$,
              \[
                  \begin{split}
                      x_{i_{k}} - x_{i_{k} - 1} < \frac{\varepsilon}{m \left(\abs{\sup\limits_{\closedinterval{x_{i_{k} - 1}, x_{i_{k}}}}(g - f)} + 1\right)} \\
                      x_{i_{k}} - x_{i_{k} - 1} < \frac{\varepsilon}{m \left(\abs{\inf\limits_{\closedinterval{x_{i_{k} - 1}, x_{i_{k}}}}(g - f)} + 1\right)}
                  \end{split}
              \]
    \end{itemize}

    Therefore
    \begin{align*}
        \abs{U(g - f, P, \closedinterval{a, b})} & = \abs{\sum^{m}_{k=1}(x_{i_{k}} - x_{i_{k} - 1})\sup\limits_{\closedinterval{x_{i_{k} - 1}, x_{i_{k}}}}(g - f)}    \\
                                                 & \leq \sum^{m}_{k=1}(x_{i_{k}} - x_{i_{k} - 1})\abs{\sup\limits_{\closedinterval{x_{i_{k} - 1}, x_{i_{k}}}}(g - f)} \\
                                                 & < \sum^{m}_{k=1}\frac{\varepsilon}{m} = \varepsilon,                                                               \\
        \abs{L(g - f, P, \closedinterval{a, b})} & = \abs{\sum^{m}_{k=1}(x_{i_{k}} - x_{i_{k} - 1})\inf\limits_{\closedinterval{x_{i_{k} - 1}, x_{i_{k}}}}(g - f)}    \\
                                                 & \leq \sum^{m}_{k=1}(x_{i_{k}} - x_{i_{k} - 1})\abs{\inf\limits_{\closedinterval{x_{i_{k} - 1}, x_{i_{k}}}}(g - f)} \\
                                                 & < \sum^{m}_{k=1}\frac{\varepsilon}{m} = \varepsilon.
    \end{align*}

    So
    \[
        \begin{split}
            U(g - f, \closedinterval{a, b}) = \inf\limits_{P}U(g - f, P, \closedinterval{a, b}) = 0 \\
            L(g - f, \closedinterval{a, b}) = \sup\limits_{P}L(g - f, P, \closedinterval{a, b}) = 0
        \end{split}
    \]

    therefore $g - f$ is Riemann integrable on $\closedinterval{a, b}$ and $\int^{b}_{a}(g - f) = 0$. By Exercise~\ref{chapter1:sectionA:exercise4}, $g$ is Riemann integrable on $\closedinterval{a, b}$ and
    \[
        \int^{b}_{a}g = \int^{b}_{a}(g - f) + \int^{b}_{a}f = \int^{b}_{a}f.\qedhere
    \]
\end{proof}
\newpage

% chapter1:sectionA:exercise7
\begin{exercise}\label{chapter1:sectionA:exercise7}
    Suppose $f: \closedinterval{a,b}\to\mathbb{R}$ is a bounded function. For $n\in\mathbb{Z}^{+}$, let $P_{n}$ denote the partition that divides $\closedinterval{a, b}$ into $2^{n}$ intervals of equals size. Prove that
    \[
        L(f, \closedinterval{a, b}) = \lim\limits_{n\to+\infty} L(f, P_{n}, \closedinterval{a, b})\quad\text{and}\quad U(f, \closedinterval{a, b}) = \lim\limits_{n\to+\infty} U(f, P_{n}, \closedinterval{a, b}).
    \]
\end{exercise}

\begin{proof}
    Because the points in $P_{n}$ are also the points in $P_{n+1}$, so
    \[
        \begin{split}
            L(f, P_{n}, \closedinterval{a, b})\leq L(f, P_{n+1}, \closedinterval{a,b}), \\
            U(f, P_{n}, \closedinterval{a, b})\geq U(f, P_{n+1}, \closedinterval{a, b}).
        \end{split}
    \]

    On the other hand, the numerical sequence ${(L(f, P_{n}, \closedinterval{a, b}))}_{n\geq 1}$ is increasing and bounded above by $L(f, \closedinterval{a, b})$ so it converges to its supremum.

    The numerical sequence ${(U(f, P_{n}, \closedinterval{a, b}))}_{n\geq 1}$ is decreasing and bounded below by $U(f, \closedinterval{a, b})$ so it converges to its infimum.

    Let $L = \inf\limits_{n}L(f, P_{n}, \closedinterval{a, b})$ and $U = \sup\limits_{n}U(f, P_{n}, \closedinterval{a, b})$, then $L\leq L(f, \closedinterval{a, b})$ and $U(f, \closedinterval{a, b})\leq U$.

    For every $\varepsilon > 0$, there exist partitions $P'$ and $P''$ such that
    \[
        \begin{split}
            L(f, \closedinterval{a, b}) - L(f, P', \closedinterval{a, b}) < \varepsilon \\
            U(f, P'', \closedinterval{a, b}) - U(f, \closedinterval{a, b}) < \varepsilon
        \end{split}
    \]

    and there exist $N$ and $M$ such that for every $n > \max\{ M, N \}$
    \[
        \begin{split}
            L - L(f, P_{n}, \closedinterval{a, b}) < \varepsilon, \\
            U - U(f, P_{n}, \closedinterval{a, b}) < \varepsilon.
        \end{split}
    \]

    Let $Q' = P'\cup P_{n}$ and $Q'' = P''\cup P_{n}$, then
    \[
        \begin{split}
            L(f, \closedinterval{a, b}) - L(f, Q', \closedinterval{a, b}) < \varepsilon,  \\
            L - L(f, Q', \closedinterval{a, b}) < \varepsilon,                            \\
            U(f, Q'', \closedinterval{a, b}) - U(f, \closedinterval{a, b}) < \varepsilon, \\
            U(f, Q'', \closedinterval{a, b}) - U < \varepsilon.
        \end{split}
    \]

    According to the definition of supremum and infimum,
    \[
        \begin{split}
            L = \sup\limits_{P}L(f, P, \closedinterval{a, b}) = L(f, \closedinterval{a, b}), \\
            U = \inf\limits_{P}U(f, P, \closedinterval{a, b}) = U(f, \closedinterval{a, b}).
        \end{split}
    \]

    Thus
    \[
        \begin{split}
            L(f, \closedinterval{a, b}) = \sup\limits_{n}L(f, P_{n}, \closedinterval{a, b}) = \lim\limits_{n\to\infty}L(f, P_{n}, \closedinterval{a, b}), \\
            U(f, \closedinterval{a, b}) = \inf\limits_{n}U(f, P_{n}, \closedinterval{a, b}) = \lim\limits_{n\to\infty}U(f, P_{n}, \closedinterval{a, b}).
        \end{split}
    \]
\end{proof}
\newpage

% chapter1:sectionA:exercise8
\begin{exercise}\label{chapter1:sectionA:exercise8}
    Suppose $f: \closedinterval{a, b}\to\mathbb{R}$ is Riemann integrable. Prove that
    \[
        \int^{b}_{a}f = \lim\limits_{n\to\infty}\frac{b-a}{n}\sum^{n}_{j=1}f\left(a + \frac{j(b-a)}{n}\right).
    \]
\end{exercise}

\noindent\textbf{Lemma 1.} Let $f: \closedinterval{a, b}\to\mathbb{R}$ be bounded. For every $\varepsilon > 0$, there exists $\delta > 0$ such that for all $P$ of $\closedinterval{a,b}$ with $\norm{P} < \delta$,
\[
    L(f, \closedinterval{a, b}) - L(f, P, \closedinterval{a, b}) < \varepsilon
    \qquad
    U(f, P, \closedinterval{a, b}) - U(f, \closedinterval{a, b}) < \varepsilon.
\]

\noindent Denote by $\norm{P}$ the length of the largest subintervals of the partition $P$.

\begin{proof}
    Let $\varepsilon > 0$. By the definition of supremum and infimum, there exist partitions $P_{\varepsilon, L}$ and $P_{\varepsilon, U}$ of $\closedinterval{a,b}$ such that
    \[
        \begin{split}
            0 \leq L(f, \closedinterval{a, b}) - L(f, P_{\varepsilon, L}, \closedinterval{a, b}) < \frac{\varepsilon}{2}, \\
            0 \leq U(f, P_{\varepsilon, U}, \closedinterval{a, b}) - U(f, \closedinterval{a, b}) < \frac{\varepsilon}{2}.
        \end{split}
    \]

    Let $P_{\varepsilon} = P_{\varepsilon, L}\cup P_{\varepsilon, U}$, then
    \[
        \begin{split}
            0 \leq L(f, \closedinterval{a, b}) - L(f, P_{\varepsilon}, \closedinterval{a, b}) < \frac{\varepsilon}{2}, \\
            0 \leq U(f, P_{\varepsilon}, \closedinterval{a, b}) - U(f, \closedinterval{a, b}) < \frac{\varepsilon}{2}.
        \end{split}
    \]

    Because $f$ is bounded, there exists $M$ such that $\abs{f(x)}\leq M$ for all $x\in\closedinterval{a, b}$. Let $N_{\varepsilon}$ be the number of closed intervals of $P_{\varepsilon}$. Let $\delta$ be a positive number such that $\delta < \varepsilon/4MN_{\varepsilon}$.

    Let $P$ be a partition with $\norm{P} < \delta$, and $P^{*} = P\cup P_{\varepsilon}$.
    \begin{align*}
        0 \leq L(f, \closedinterval{a, b}) - L(f, P, \closedinterval{a, b}) & = L(f, \closedinterval{a, b}) - L(f, P^{*}, \closedinterval{a, b}) + L(f, P^{*}, \closedinterval{a, b}) - L(f, P, \closedinterval{a, b}) \\
                                                                            & < \frac{\varepsilon}{2} + L(f, P^{*}, \closedinterval{a, b}) - L(f, P, \closedinterval{a, b}),                                           \\
        0 \leq U(f, P, \closedinterval{a, b}) - U(f, \closedinterval{a, b}) & = U(f, P, \closedinterval{a, b}) - U(f, P^{*}, \closedinterval{a, b}) + U(f, P^{*}, \closedinterval{a, b}) - U(f, \closedinterval{a, b}) \\
                                                                            & = U(f, P, \closedinterval{a, b}) - U(f, P^{*}, \closedinterval{a, b}) + \frac{\varepsilon}{2}.
    \end{align*}

    $P = \{ a = x_{0} < x_{1} < \cdots < x_{n} = b \}$. For $1\leq i \leq n$, define $P^{*}_{i} = P^{*} \cap \closedinterval{x_{i-1}, x_{i}}$, then $P^{*}_{i}$ is a partition of $\closedinterval{x_{i-1}, x_{i}}$.
    \begingroup
    \allowdisplaybreaks{}
    \begin{align*}
        0 \leq L(f, P^{*}, \closedinterval{a,b}) - L(f, P, \closedinterval{a,b}) & = \sum^{n}_{i=1}L(f, P^{*}_{i}, \closedinterval{x_{i-1}, x_{i}}) - \sum^{n}_{i=1}(x_{i} - x_{i-1})\inf\limits_{\closedinterval{x_{i-1}, x_{i}}}f, \\
                                                                                 & = \sum^{n}_{i=1}\left(L(f, P^{*}_{i}, \closedinterval{x_{i-1}, x_{i}}) - (x_{i} - x_{i-1})\inf\limits_{\closedinterval{x_{i-1}, x_{i}}}f\right)   \\
        0 \leq U(f, P, \closedinterval{a,b}) - U(f, P^{*}, \closedinterval{a,b}) & =  \sum^{n}_{i=1}(x_{i} - x_{i-1})\sup\limits_{\closedinterval{x_{i-1}, x_{i}}}f - \sum^{n}_{i=1}U(f, P^{*}_{i}, \closedinterval{x_{i-1}, x_{i}}) \\
                                                                                 & = \sum^{n}_{i=1}\left((x_{i} - x_{i-1})\sup\limits_{\closedinterval{x_{i-1}, x_{i}}}f - U(f, P^{*}_{i}, \closedinterval{x_{i-1}, x_{i}})\right)
    \end{align*}
    \endgroup

    $P_{\varepsilon}$ and $P$ have at least two common partition points $a$ and $b$, so $P_{\varepsilon}$ has at most $N_{\varepsilon} - 1$ partition points that are not partition points of $P$. So there are at most $N_{\varepsilon} - 1$ indices $i$ such that $\closedinterval{x_{i-1}, x_{i}}$ contains at least one partition point of $P^{*}_{i}$ within the open interval $\openinterval{x_{i-1}, x_{i}}$. For all other indices, the terms in the two above sums are $0$ because it doesn't contain any partition points of $P^{*}_{i}$. For each $i$ such that the term is nonzero,
    \[
        \begin{split}
            0\leq L(f, P^{*}_{i}, \closedinterval{x_{i-1}, x_{i}}) - (x_{i} - x_{i-1})\inf\limits_{\closedinterval{x_{i-1}, x_{i}}}f < 2M(x_{i} - x_{i-1}) < 2M\delta, \\
            0\leq (x_{i} - x_{i-1})\sup\limits_{\closedinterval{x_{i-1}, x_{i}}}f - U(f, P^{*}_{i}, \closedinterval{x_{i-1}, x_{i}}) < 2M(x_{i} - x_{i-1}) < 2M\delta.
        \end{split}
    \]

    Because there are at most $N_{\varepsilon} - 1$ nonzero terms, so
    \[
        \begin{split}
            0 \leq L(f, P^{*}, \closedinterval{a,b}) - L(f, P, \closedinterval{a,b}) < 2MN_{\varepsilon}\delta < \frac{\varepsilon}{2}, \\
            0 \leq U(f, P, \closedinterval{a,b}) - U(f, P^{*}, \closedinterval{a,b}) < 2MN_{\varepsilon}\delta < \frac{\varepsilon}{2}.
        \end{split}
    \]

    Therefore
    \[
        \begin{split}
            0 \leq L(f, \closedinterval{a, b}) - L(f, P, \closedinterval{a, b}) < \varepsilon, \\
            0 \leq U(f, P, \closedinterval{a, b}) - U(f, \closedinterval{a, b}) < \varepsilon.
        \end{split}
    \]
\end{proof}

\noindent\textbf{Lemma 2 (not neccessary here).} Suppose $f: \closedinterval{a, b}\to\mathbb{R}$ is a bounded function. $f$ is Riemann integrable if and only if for every $\varepsilon > 0$, there exists $\delta > 0$ such that for every partition $P$ on $\closedinterval{P}$
\[
    \norm{P} < \delta \implies U(f, P, \closedinterval{a, b}) - L(f, P, \closedinterval{a, b}) < \varepsilon.
\]

\begin{proof}[Lemma 2's proof]
    Suppose $f$ is Riemann integrable, then $U(f, \closedinterval{a, b}) = L(f, \closedinterval{a, b})$.

    By Lemma 1, there exist $\delta > 0$ such that for every partition $P$ on $\closedinterval{a, b}$ with $\norm{P} < \delta$,
    \[
        \begin{split}
            0\leq L(f, \closedinterval{a, b}) - L(f, P, \closedinterval{a, b}) < \frac{\varepsilon}{2} \\
            0\leq U(f, P, \closedinterval{a, b}) - U(f, \closedinterval{a, b}) < \frac{\varepsilon}{2}
        \end{split}
    \]

    therefore
    \[
        U(f, P, \closedinterval{a, b}) - L(f, P, \closedinterval{a, b}) = U(f, P, \closedinterval{a, b}) - U(f, \closedinterval{a, b}) + L(f, \closedinterval{a, b}) - L(f, P, \closedinterval{a, b}) < \varepsilon.
    \]

    Hence for every $\varepsilon > 0$, there exists $\delta > 0$ such that for every partition $P$ on $\closedinterval{a, b}$ with $\norm{P} < \delta$,
    \[
        U(f, P, \closedinterval{a, b}) - L(f, P, \closedinterval{a, b}) < \varepsilon.
    \]

    \bigskip
    Suppose for every $\varepsilon > 0$, there exists $\delta > 0$ such that for every partition $P$ on $\closedinterval{a, b}$ with $\norm{P} < \delta$,
    \[
        U(f, P, \closedinterval{a, b}) - L(f, P, \closedinterval{a, b}) < \varepsilon.
    \]

    Because
    \[
        U(f, \closedinterval{a, b}) - L(f, \closedinterval{a, b}) \leq U(f, P, \closedinterval{a, b}) - L(f, P, \closedinterval{a, b})
    \]

    and $U(f, P, \closedinterval{a, b}) - L(f, P, \closedinterval{a, b})$ can be arbitrarily small according to the hypothesis. Therefore $U(f, \closedinterval{a, b}) - L(f, \closedinterval{a, b})$ is arbitrarily small. Thus $U(f, \closedinterval{a, b}) = L(f, \closedinterval{a, b})$, which implies $f$ is Riemann integrable.
\end{proof}

\begin{proof}
    By Lemma 1, for every $\varepsilon > 0$, there exists $\delta > 0$ such that for every partition $P$ on $\closedinterval{a, b}$ with $\norm{P} < \delta$,
    \[
        \begin{split}
            0\leq L(f, \closedinterval{a, b}) - L(f, P, \closedinterval{a, b}) < \frac{\varepsilon}{2}, \\
            0\leq U(f, P, \closedinterval{a, b}) - U(f, \closedinterval{a, b}) < \frac{\varepsilon}{2}.
        \end{split}
    \]

    For every positive integer $n$  such that $(b - a)/n < \delta$, let $P_{n}$ be a partition on $\closedinterval{a, b}$ that divides $\closedinterval{a, b}$ into $n$ equal closed intervals, then
    \[
        \begin{split}
            0\leq L(f, \closedinterval{a, b}) - L(f, P_{n}, \closedinterval{a, b}) < \frac{\varepsilon}{2}, \\
            0\leq U(f, P_{n}, \closedinterval{a, b}) - U(f, \closedinterval{a, b}) < \frac{\varepsilon}{2}.
        \end{split}
    \]

    Moreover,
    \begin{align*}
        \sum^{n}_{i=1}\frac{b-a}{n}f\left(a + \frac{j(b-a)}{n}\right) - \int^{b}_{a}f & \geq L(f, P_{n}, \closedinterval{a, b}) - U(f, \closedinterval{a,b})         \\
                                                                                      & \geq L(f, P_{n}, \closedinterval{a, b}) - U(f, P_{n}, \closedinterval{a,b})  \\
        \sum^{n}_{i=1}\frac{b-a}{n}f\left(a + \frac{j(b-a)}{n}\right) - \int^{b}_{a}f & \leq U(f, P_{n}, \closedinterval{a, b}) - L(f, \closedinterval{a, b})        \\
                                                                                      & \leq U(f, P_{n}, \closedinterval{a, b}) - L(f, P_{n}, \closedinterval{a, b})
    \end{align*}

    so
    \begin{align*}
        \abs{\sum^{n}_{i=1}\frac{b-a}{n}f\left(a + \frac{j(b-a)}{n}\right) - \int^{b}_{a}f} & \leq U(f, P_{n}, \closedinterval{a, b}) - L(f, P_{n}, \closedinterval{a, b})                                                          \\
                                                                                            & = U(f, P_{n}, \closedinterval{a, b}) - U(f, \closedinterval{a, b}) + L(f, \closedinterval{a, b}) - L(f, P_{n}, \closedinterval{a, b}) \\
                                                                                            & < \frac{\varepsilon}{2} + \frac{\varepsilon}{2} = \varepsilon.
    \end{align*}

    Thus
    \[
        \int^{b}_{a}f = \lim\limits_{n\to\infty}\frac{b-a}{n}\sum^{n}_{j=1}f\left(a + \frac{j(b-a)}{n}\right).\qedhere
    \]
\end{proof}
\newpage

% chapter1:sectionA:exercise9
\begin{exercise}\label{chapter1:sectionA:exercise9}
    Suppose $f: \closedinterval{a,b}\to\mathbb{R}$ is Riemann integrable. Prove that if $c, d\in\mathbb{R}$ and $a\leq c < d\leq b$, then $f$ is Riemann integrable on $\closedinterval{c, d}$.

        [To say that $f$ is Riemann integrable on $\closedinterval{c,d}$ means that $f$ with its domain restricted to $\closedinterval{c,d}$ is Riemann integrable.]
\end{exercise}

\begin{proof}
    Let $\varepsilon > 0$. By Exercise~\ref{chapter1:sectionA:exercise3}, there is a partition $P$ on $\closedinterval{a, b}$ such that
    \[
        U(f, P, \closedinterval{a, b}) - L(f, P, \closedinterval{a, b}) < \varepsilon.
    \]

    Let $P' = P\cup \{ c, d \}$, then
    \[
        \begin{split}
            U(f, P', \closedinterval{a, b}) \leq U(f, P, \closedinterval{a, b}), \\
            L(f, P, \closedinterval{a, b}) \leq L(f, P', \closedinterval{a, b}).
        \end{split}
    \]

    We make use of the following expression for subsequent arguments.
    \[
        U(f, P, \closedinterval{a, b}) - L(f, P, \closedinterval{a, b}) = \sum^{n}_{i=1}(x_{i} - x_{i-1})\left(\sup\limits_{\closedinterval{x_{i-1},x_{i}}}f - \inf\limits_{\closedinterval{x_{i-1},x_{i}}}f\right)
    \]

    Let $P^{*} = P'\cap \closedinterval{c, d}$, then
    \[
        U(f, P^{*}, \closedinterval{c, d}) - L(f, P^{*}, \closedinterval{c, d}) \leq U(f, P', \closedinterval{a, b}) - U(f, P', \closedinterval{a, b})
    \]

    because
    \begin{itemize}
        \item every term in the upper and lower sum expansion on the left appears in the upper and lower sum expansion on the right,
        \item the difference between the upper and lower term is nonnegative.
    \end{itemize}

    Moreover,
    \[
        U(f, P', \closedinterval{a, b}) - U(f, P', \closedinterval{a, b}) \leq U(f, P, \closedinterval{a, b}) - L(f, P, \closedinterval{a, b}) < \varepsilon.
    \]

    So
    \[
        U(f, P^{*}, \closedinterval{c, d}) - L(f, P^{*}, \closedinterval{c, d}) < \varepsilon.
    \]

    By Exercise~\ref{chapter1:sectionA:exercise3}, $f$ is Riemann integrable on $\closedinterval{c, d}$.
\end{proof}
\newpage

% chapter1:sectionA:exercise10
\begin{exercise}\label{chapter1:sectionA:exercise10}
    Suppose $f: \closedinterval{a,b}\to\mathbb{R}$ is a bounded function and $c\in\openinterval{a,b}$. Prove that $f$ is Riemann integrable on $\closedinterval{a,b}$ if and only if $f$ is Riemann integrable on $\closedinterval{a,c}$ and $f$ is Riemann integrable on $\closedinterval{c,b}$. Furthermore, prove that if these conditions hold, then
    \[
        \int^{b}_{a}f = \int^{c}_{a}f + \int^{b}_{c}f.
    \]
\end{exercise}

\noindent\textbf{Lemma 1.} Suppose $f: \closedinterval{a,b}\to\mathbb{R}$ is a bounded function and $c\in\openinterval{a,b}$, then
\[
    \begin{split}
        U(f, \closedinterval{a, b}) = U(f, \closedinterval{a, c}) + U(f, \closedinterval{c, b}), \\
        L(f, \closedinterval{a, b}) = L(f, \closedinterval{a, c}) + L(f, \closedinterval{c, b}).
    \end{split}
\]

\begin{proof}[Lemma 1's proof]
    For every partition $P$ on $\closedinterval{a, b}$.
    \begin{align*}
        U(f, P, \closedinterval{a, b}) & \geq U(f, P\cup\{c\}, \closedinterval{a, b})                                                                                              \\
                                       & = U(f, (P\cup\{c\})\cap\closedinterval{a, c}, \closedinterval{a, c}) + U(f, (P\cup\{c\})\cap\closedinterval{c, b}, \closedinterval{c, b}) \\
                                       & \geq U(f, \closedinterval{a, c}) + U(f, \closedinterval{c, b}).
    \end{align*}

    So $U(f, \closedinterval{a, c}) + U(f, \closedinterval{c, b})$ is a lower bound of $U(f, P, \closedinterval{a, b})$ (as $P$ varies). According to the definition of infimum
    \[
        U(f, \closedinterval{a, b})\geq U(f, \closedinterval{a, c}) + U(f, \closedinterval{c, b}).
    \]

    On the other hand, for every partition $P_{1}$ on $\closedinterval{a, b}$ and partition $P_{2}$ on $\closedinterval{c, b}$,
    \begin{align*}
        U(f, P_{1}, \closedinterval{a, c}) + U(f, P_{2}, \closedinterval{c, b}) = U(f, P_{1}\cup P_{2}, \closedinterval{a, b}) \geq U(f, \closedinterval{a, b}).
    \end{align*}

    So $U(f, \closedinterval{a, b})$ is a lower bound of $U(f, P_{1}, \closedinterval{a, c}) + U(f, P_{2}, \closedinterval{c, b})$ (as $P_{1}, P_{2}$ vary).

    The infimum of $U(f, P_{1}, \closedinterval{a, c}) + U(f, P_{2}, \closedinterval{c, b})$ (as $P_{1}, P_{2}$ vary) is $U(f, \closedinterval{a, c}) + U(f, \closedinterval{c, b})$.

    Therefore
    \[
        U(f, \closedinterval{a, b})\leq U(f, \closedinterval{a, c}) + U(f, \closedinterval{c, b}).
    \]

    Hence $U(f, \closedinterval{a, b}) = U(f, \closedinterval{a, c}) + U(f, \closedinterval{c, b})$.

    Analogously, $L(f, \closedinterval{a, b}) = L(f, \closedinterval{a, c}) + L(f, \closedinterval{c, b})$.
\end{proof}

\begin{proof}
    Suppose $f$ is Riemann integrable on $\closedinterval{a, b}$. According to Exercise~\ref{chapter1:sectionA:exercise9}, $f$ is Riemann integrable on $\closedinterval{a, c}$ and $\closedinterval{c, b}$.

    Suppose $f$ is Riemann integrable on $\closedinterval{a, c}$ and $\closedinterval{c, b}$. Let $\varepsilon > 0$. By Exercise~\ref{chapter1:sectionA:exercise3}, there exists partition $P_{1}$ on $\closedinterval{a, c}$ and partition $P_{2}$ on $\closedinterval{c, b}$ such that
    \[
        \begin{split}
            U(f, P_{1}, \closedinterval{a, c}) - L(f, P_{1}, \closedinterval{a, c}) < \frac{\varepsilon}{2}, \\
            U(f, P_{2}, \closedinterval{c, b}) - L(f, P_{2}, \closedinterval{c, b}) < \frac{\varepsilon}{2}.
        \end{split}
    \]

    $P = P_{1}\cup P_{2}$ is a partition on $\closedinterval{a, b}$ and
    \begin{align*}
        U(f, P, \closedinterval{a, b}) - L(f, P, \closedinterval{a, b}) & = U(f, P_{1}, \closedinterval{a, c}) + U(f, P_{2}, \closedinterval{c, b}) - L(f, P_{1}, \closedinterval{a, c}) - L(f, P_{2}, \closedinterval{c, b})  \\
                                                                        & = U(f, P_{1}, \closedinterval{a, c}) - L(f, P_{1}, \closedinterval{a, c}) +  U(f, P_{2}, \closedinterval{c, b}) - L(f, P_{2}, \closedinterval{c, b}) \\
                                                                        & < \frac{\varepsilon}{2} + \frac{\varepsilon}{2} = \varepsilon.
    \end{align*}

    By Exercise~\ref{chapter1:sectionA:exercise3}, we conclude that $f$ is Riemann integrable on $\closedinterval{a, b}$.

    \bigskip
    Suppose these conditions hold. By Lemma 1 and the definition of Riemann integrablility
    \[
        \int^{b}_{a}f = U(f, \closedinterval{a, b}) = U(f, \closedinterval{a, c}) + U(f, \closedinterval{c, b}) = \int^{c}_{a}f + \int^{b}_{c}f.\qedhere
    \]
\end{proof}
\newpage

% chapter1:sectionA:exercise11
\begin{exercise}\label{chapter1:sectionA:exercise11}
    Suppose $f: \closedinterval{a, b}\to\mathbb{R}$ is Riemann integrable. Define $F: \closedinterval{a,b}\to\mathbb{R}$ by
    \[
        F(t) = \begin{cases}
            0             & \text{if $t = a$}                    \\
            \int^{t}_{a}f & \text{if $t\in\halfopenleft{a, b}$.}
        \end{cases}
    \]

    Prove that $F$ is continuous on $\closedinterval{a,b}$.
\end{exercise}

\begin{proof}
    Let $c, d\in\closedinterval{a, b}$ such that $c < d$.

    If $c = a$, then
    \[
        F(d) - F(c) = F(d) = \int^{d}_{a}f = \int^{d}_{c}f.
    \]

    If $c > a$, then by Exercise~\ref{chapter1:sectionA:exercise10}
    \[
        F(d) - F(c) = \int^{d}_{a}f - \int^{c}_{a}f = \int^{d}_{c}f.
    \]

    Hence for every $c < d$ and $c, d\in\closedinterval{a, b}$,
    \[
        F(d) - F(c) = \int^{d}_{c}f.
    \]


    $f$ is Riemann integrable so $f$ is bounded. Let $M = \sup\limits_{\closedinterval{a, b}}f$ and $m = \inf\limits_{\closedinterval{a, b}}f$ and $A = \max\{ \abs{M}, \abs{m} \}$.

    Let $x, y\in\closedinterval{a, b}$.

    If $x > y$, let $P$ be a partition on $\closedinterval{y, x}$.
    \begin{align*}
        F(x) - F(y) & = \int^{x}_{y}f \leq U(f, P, \closedinterval{y, x}) \leq M(x - y) \leq A(x - y),  \\
        F(x) - F(y) & = \int^{x}_{y}f \geq L(f, P, \closedinterval{y, x}) \geq m(x - y) \geq -A(x - y).
    \end{align*}

    If $x < y$, let $P$ be a partition on $\closedinterval{x, y}$.
    \begin{align*}
        F(x) - F(y) & = -\int^{y}_{x}f \geq -U(f, P, \closedinterval{x, y}) \geq -M(y - x) \geq -A(y - x), \\
        F(x) - F(y) & = -\int^{t}_{c}f \leq -L(f, P, \closedinterval{x, y}) \leq -m(y - x) \leq -A(y - x).
    \end{align*}

    Therefore $\abs{F(x) - F(y)} \leq A\abs{x - y}$ for all $x, y\in\closedinterval{a, b}$. Thus $F$ is Lipschitz continuous on $\closedinterval{a, b}$, which means $F$ is continuous on $\closedinterval{a, b}$.
\end{proof}
\newpage

% chapter1:sectionA:exercise12
\begin{exercise}\label{chapter1:sectionA:exercise12}
    Suppose $f: \closedinterval{a, b}\to\mathbb{R}$ is Riemann integrable. Prove that $\abs{f}$ is Riemann integrable and that
    \[
        \abs{\int^{b}_{a}f} \leq \int^{b}_{a}\abs{f}.
    \]
\end{exercise}

\begin{proof}
    Let $\varepsilon > 0$. By Exercise~\ref{chapter1:sectionA:exercise3}, there exists a partition $P = \{ x_{0}, x_{1}, \ldots, x_{n} \}$ on $\closedinterval{a, b}$ such that
    \[
        U(f, P, \closedinterval{a, b}) - L(f, P, \closedinterval{a, b}) < \varepsilon.
    \]

    Moreover,
    \begin{align*}
        U(\abs{f}, P, \closedinterval{a, b}) - L(\abs{f}, P, \closedinterval{a, b}) = \sum^{n}_{i=1}(x_{i} - x_{i-1})\left(\sup\limits_{\closedinterval{x_{i-1}, x_{i}}}\abs{f} - \inf\limits_{\closedinterval{x_{i-1}, x_{i}}}\abs{f}\right)
    \end{align*}

    For every $x, y\in\closedinterval{x_{i-1}, x_{i}}$
    \[
        \abs{\abs{f(x)} - \abs{f(y)}} \leq \abs{f(x) - f(y)} \leq \sup\limits_{\closedinterval{x_{i-1}, x_{i}}}f - \inf\limits_{\closedinterval{x_{i-1}, x_{i}}}f.
    \]

    So $\sup\limits_{\closedinterval{x_{i-1}, x_{i}}}f - \inf\limits_{\closedinterval{x_{i-1}, x_{i}}}f$ is an upper bound of $\abs{\abs{f(x)} - \abs{f(y)}}$, as $x, y\in\closedinterval{x_{i-1}, x_{i}}$. By taking supremum, we obtain
    \[
        0\leq \sup\limits_{\closedinterval{x_{i-1}, x_{i}}}\abs{f} - \inf\limits_{\closedinterval{x_{i-1}, x_{i}}}\abs{f} = \sup\limits_{x, y\in\closedinterval{x_{i-1},x_{i}}}\abs{\abs{f(x)} - \abs{f(y)}} \leq \sup\limits_{\closedinterval{x_{i-1}, x_{i}}}f - \inf\limits_{\closedinterval{x_{i-1}, x_{i}}}f.
    \]

    Therefore
    \begin{align*}
        U(\abs{f}, P, \closedinterval{a, b}) - L(\abs{f}, P, \closedinterval{a, b}) & = \sum^{n}_{i=1}(x_{i} - x_{i-1})\left(\sup\limits_{\closedinterval{x_{i-1}, x_{i}}}\abs{f} - \inf\limits_{\closedinterval{x_{i-1}, x_{i}}}\abs{f}\right) \\
                                                                                    & \leq \sum^{n}_{i=1}(x_{i} - x_{i-1})\left(\sup\limits_{\closedinterval{x_{i-1}, x_{i}}}f - \inf\limits_{\closedinterval{x_{i-1}, x_{i}}}f\right)          \\
                                                                                    & = U(f, P, \closedinterval{a, b}) - L(f, P, \closedinterval{a, b})                                                                                         \\
                                                                                    & < \varepsilon.
    \end{align*}

    By Exercise~\ref{chapter1:sectionA:exercise3}, we conclude that $\abs{f}$ is Riemann integrable on $\closedinterval{a, b}$.

    \bigskip
    For every partition $P = \{ x_{0}, x_{1}, \ldots, x_{n} \}$ on $\closedinterval{a, b}$. By the triangle's inequality
    \begin{align*}
        \abs{U(f, P, \closedinterval{a, b})} & = \abs{\sum^{n}_{i=1}(x_{i} - x_{i-1})\sup\limits_{\closedinterval{x_{i-1}, x_{i}}}f}    \\
                                             & \leq \sum^{n}_{i=1}(x_{i} - x_{i-1})\abs{\sup\limits_{\closedinterval{x_{i-1}, x_{i}}}f} \\
                                             & \leq \sum^{n}_{i=1}(x_{i} - x_{i-1})\sup\limits_{\closedinterval{x_{i-1},x_{i}}}\abs{f}  \\
                                             & = U(\abs{f}, P, \closedinterval{a, b}).
    \end{align*}

    On the other hand, by Exercise~\ref{chapter1:sectionA:exercise5}, $-f$ is Riemann integrable on $\closedinterval{a, b}$, $L(f, P, \closedinterval{a, b}) = -U(-f, P, \closedinterval{a, b})$, and by the above inequality,
    \begin{align*}
        \abs{L(f, P, \closedinterval{a, b})} = \abs{-U(-f, P, \closedinterval{a, b})} = \abs{U(-f, P, \closedinterval{a, b})} \leq U(\abs{f}, P, \closedinterval{a, b}).
    \end{align*}

    Together with $L(f, P, \closedinterval{a, b}) \leq \int^{b}_{a}f \leq U(f, P, \closedinterval{a, b})$, we get
    \[
        -U(\abs{f}, P, \closedinterval{a, b}) \leq L(f, P, \closedinterval{a, b}) \leq \int^{b}_{a}f \leq U(f, P, \closedinterval{a, b}) \leq U(\abs{f}, P, \closedinterval{a, b}).
    \]

    So
    \[
        \abs{\int^{b}_{a}f} \leq U(\abs{f}, P, \closedinterval{a, b}).
    \]

    Take infimum
    \[
        \abs{\int^{b}_{a}f} \leq \inf\limits_{P}U(\abs{f}, P, \closedinterval{a,b}) = U(\abs{f}, \closedinterval{a, b})
    \]

    and because $\abs{f}$ is Riemann integrable, $U(\abs{f}, \closedinterval{a, b}) = \int^{b}_{a}\abs{f}$, we conclude that
    \[
        \abs{\int^{b}_{a}f} \leq \int^{b}_{a}\abs{f}.\qedhere
    \]
\end{proof}
\newpage

% chapter1:sectionA:exercise13
\begin{exercise}\label{chapter1:sectionA:exercise13}
    Suppose $f: \closedinterval{a, b}\to\mathbb{R}$ is an increasing function, meaning that $c, d\in\closedinterval{a, b}$ with $c < d$ implies $f(c)\leq f(d)$. Prove that $f$ is Riemann integrable on $\closedinterval{a, b}$.
\end{exercise}

\begin{proof}
    $f$ is bounded, because $f(a) \leq f(x) \leq f(b)$ for every $x\in\closedinterval{a, b}$.

    Let $\varepsilon > 0$ and let $P$ be a partition on $\closedinterval{a, b}$ such that $\norm{P} < \frac{\varepsilon}{f(b) - f(a) + 1}$.
    \begin{align*}
        U(f, P, \closedinterval{a, b}) - L(f, P, \closedinterval{a, b}) & = \sum^{n}_{i=1}(x_{i} - x_{i-1})(\sup\limits_{\closedinterval{x_{i-1},x_{i}}}f - \inf\limits_{\closedinterval{x_{i-1},x_{i}}}f) \\
                                                                        & = \sum^{n}_{i=1}(x_{i} - x_{i-1})(f(x_{i}) - f(x_{i-1}))                                                                         \\
                                                                        & \leq \sum^{n}_{i=1}\frac{\varepsilon}{f(b) - f(a) + 1}(f(x_{i}) - f(x_{i-1}))                                                    \\
                                                                        & = \frac{\varepsilon}{f(b) - f(a) + 1}(f(b) - f(a)) < \varepsilon.
    \end{align*}

    By Exercise~\ref{chapter1:sectionA:exercise3}, we conclude that $f$ is Riemann integrable on $\closedinterval{a, b}$.
\end{proof}
\newpage

% chapter1:sectionA:exercise14
\begin{exercise}\label{chapter1:sectionA:exercise14}
    Suppose $f_{1}, f_{2}, \ldots$ is a sequence of Riemann integrable functions on $\closedinterval{a,b}$ such that $f_{1}, f_{2}, \ldots$ converges uniformly on $\closedinterval{a,b}$ to a function $f: \closedinterval{a,b}\to\mathbb{R}$. Prove that $f$ is Riemann integrable and
    \[
        \int^{b}_{a}f = \lim\limits_{n\to\infty}\int^{b}_{a}f_{n}.
    \]
\end{exercise}

\begin{proof}
    According to the definition of uniform convergence, for every $\varepsilon > 0$, there exists a positive integer $N_{\varepsilon}$ such that for every $n\geq N_{\varepsilon}$, for all $x\in \closedinterval{a, b}$,
    \[
        \abs{f_{n}(x) - f(x)} < \frac{\varepsilon}{3(b - a)}.
    \]

    Let $n$ be a positive integer such that $n\geq N_{\varepsilon}$.

    By Exercise~\ref{chapter1:sectionA:exercise3}, there exists a partition $P = \{ x_{0}, x_{1}, \ldots, x_{n} \}$ on $\closedinterval{a, b}$ such that
    \[
        U(f_{n}, P, \closedinterval{a, b}) - L(f_{n}, P, \closedinterval{a, b}) < \frac{\varepsilon}{3}.
    \]

    I will estimate $U(f, P, \closedinterval{a, b}) - L(f, P, \closedinterval{a, b})$.
    \begin{multline*}
        U(f, P, \closedinterval{a, b}) - L(f, P, \closedinterval{a, b}) = U(f, P, \closedinterval{a, b}) - U(f_{n}, P, \closedinterval{a, b}) \\
        + U(f_{n}, P, \closedinterval{a, b}) - L(f_{n}, P, \closedinterval{a, b}) \\
        + L(f_{n}, P, \closedinterval{a, b}) - L(f, P, \closedinterval{a, b}).
    \end{multline*}

    On $\closedinterval{x_{i-1}, x_{i}}$
    \begin{align*}
        \sup\limits_{\closedinterval{x_{i-1}, x_{i}}}f \leq \sup\limits_{\closedinterval{x_{i-1}, x_{i}}}(f - f_{n}) + \sup\limits_{\closedinterval{x_{i-1}, x_{i}}}f_{n} < \frac{\varepsilon}{3(b - a)} + \sup\limits_{\closedinterval{x_{i-1}, x_{i}}}f_{n}, \\
        \inf\limits_{\closedinterval{x_{i-1}, x_{i}}}f \geq \inf\limits_{\closedinterval{x_{i-1}, x_{i}}}(f - f_{n}) + \inf\limits_{\closedinterval{x_{i-1}, x_{i}}}f_{n} > \frac{-\varepsilon}{3(b - a)} + \inf\limits_{\closedinterval{x_{i-1}, x_{i}}}f_{n}.
    \end{align*}

    Summing to obtain the upper Riemann sum and lower Riemann sum, we obtain
    \begin{align*}
        U(f, P, \closedinterval{a, b}) < \frac{\varepsilon}{3} + U(f_{n}, P, \closedinterval{a, b}), \\
        L(f, P, \closedinterval{a, b}) > \frac{-\varepsilon}{3} + L(f_{n}, P, \closedinterval{a, b}).
    \end{align*}

    Hence
    \[
        U(f, P, \closedinterval{a, b}) - L(f, P, \closedinterval{a, b}) < \frac{\varepsilon}{3} + \frac{\varepsilon}{3} + \frac{\varepsilon}{3} = \varepsilon.
    \]

    So $f$ is Riemann integrable.

    Continue with the same $\varepsilon, N_{\varepsilon}, n\geq N_{\varepsilon}$ and partition $P$, we get
    \[
        \frac{-\varepsilon}{3} + L(f_{n}, P, \closedinterval{a, b}) < L(f, P, \closedinterval{a, b}) \leq \int^{b}_{a}f \leq U(f, P, \closedinterval{a, b}) < \frac{\varepsilon}{3} + U(f_{n}, P, \closedinterval{a, b})
    \]

    Moreover,
    \[
        \frac{-\varepsilon}{3} + L(f_{n}, P, \closedinterval{a, b}) \leq \frac{-\varepsilon}{3} + \int^{b}_{a}f_{n} < \int^{b}_{a}f_{n} < \frac{\varepsilon}{3} + \int^{b}_{a}f_{n} \leq \frac{\varepsilon}{3} + U(f_{n}, P, \closedinterval{a, b}).
    \]

    Therefore, for all $n\geq N_{\varepsilon}$
    \[
        \abs{\int^{b}_{a}f_{n} - \int^{b}_{a}f} \leq U(f_{n}, P, \closedinterval{a, b}) - L(f_{n}, P, \closedinterval{a, b}) + \frac{2\varepsilon}{3} < \varepsilon.
    \]

    So for every $\varepsilon > 0$, there exists a positive integer $N_{\varepsilon}$ such that for every $n\geq N_{\varepsilon}$,
    \[
        \abs{\int^{b}_{a}f_{n} - \int^{b}_{a}f} < \varepsilon.
    \]

    This means
    \[
        \int^{b}_{a}f = \lim\limits_{n\to\infty}\int^{b}_{a}f_{n}.\qedhere
    \]
\end{proof}
\newpage

\section{Riemann Integral Is Not Good Enough}

% chapter1:sectionB:exercise1
\begin{exercise}\label{chapter1:sectionB:exercise1}
    Define $f: \closedinterval{0, 1}\to \mathbb{R}$ as follows:
    \[
        f(a) = \begin{cases}
            0           & \text{if $a$ is irrational},                                                                                                             \\
            \frac{1}{n} & \parbox[t]{0.7\textwidth}{if $a$ is rational and $n$ is the smallest positive integer such that $a = \frac{m}{n}$ for some integer $m$.}
        \end{cases}
    \]

    Show that $f$ is Riemann integrable and compute $\int^{1}_{0}f$.
\end{exercise}

\begin{proof}
    $f$ is bounded, because $0\leq f(x)\leq 1$ for all $x\in\closedinterval{a, b}$.

    Let $\varepsilon > 0$ and $N$ is a positive integer such that $\frac{1}{N} < \frac{\varepsilon}{2}$. Consider the following finite set
    \[
        A = \left\{ 0, 1, \frac{1}{2}, \frac{1}{3}, \frac{2}{3}, \ldots, \frac{1}{N}, \ldots, \frac{N-1}{N} \right\}.
    \]

    If $x\in\closedinterval{0,1}$ and $x\notin A$ then
    \[
        f(x) \leq \frac{1}{N+1} < \frac{\varepsilon}{2}.
    \]

    Choose a partition $P = \{ x_{0}, x_{1}, \ldots, x_{n} \}$ such that $n$ is larger than $m$ and $\norm{P} < \frac{\varepsilon}{4m}$ where $m$ is the number of elements of $A$. Because every closed interval (that contains more than 1 element) contains an irrational number, $L(f, P, \closedinterval{a, b}) = 0$.

    For every $i\in\{ 1,\ldots, n \}$, let
    \[
        A_{i} = \closedinterval{x_{i-1}, x_{i}}\cap A.
    \]

    Consider the upper Riemann sum
    \[
        U(f, P, \closedinterval{a, b}) = \sum^{n}_{i=1}(x_{i} - x_{i-1})\sup\limits_{\closedinterval{x_{i-1},x_{i}}}f = \sum_{i: A_{i}\ne\varnothing}(x_{i} - x_{i-1})\sup\limits_{\closedinterval{x_{i-1},x_{i}}}f + \sum_{i: A_{i}=\varnothing}(x_{i} - x_{i-1})\sup\limits_{\closedinterval{x_{i-1},x_{i}}}f.
    \]

    The number of subintervals $\closedinterval{x_{i-1}, x_{i}}$ that contains an element of $A$ does not exceed $2m$ (because a closed interval has boundary and an element of $A$ can be at the boundary of one of these closed subintervals), so
    \[
        \sum_{i: A_{i}\ne\varnothing}(x_{i} - x_{i-1})\sup\limits_{\closedinterval{x_{i-1},x_{i}}}f \leq \sum_{i: A_{i}\ne\varnothing} 1\cdot \norm{P} \leq 2m\cdot 1\norm{P} < 2m \cdot \frac{\varepsilon}{4m} = \frac{\varepsilon}{2}.
    \]

    If $\closedinterval{x_{i-1}, x_{i}}\cap A = \varnothing$, then $\sup\limits_{\closedinterval{x_{i-1},x_{i}}}f\leq \frac{1}{N+1}$
    \[
        \sum_{i: A_{i}=\varnothing}(x_{i} - x_{i-1})\sup\limits_{\closedinterval{x_{i-1},x_{i}}}f \leq \sum_{i: A_{i}=\varnothing}(x_{i} - x_{i-1})\frac{1}{N+1} \leq (x_{n} - x_{0})\frac{1}{N+1} = \frac{1}{N+1} < \frac{\varepsilon}{2}.
    \]

    Therefore $U(f, P, \closedinterval{a, b}) - L(f, P, \closedinterval{a, b}) = U(f, P, \closedinterval{a, b}) < \varepsilon$. Hence $f$ is Riemann integrable. By the definition of Riemann integrablility,
    \[
        \int^{1}_{0}f = L(f, \closedinterval{a, b}) = \inf\limits_{P}L(f, P, \closedinterval{a, b}) = \inf\limits_{P}0 = 0.
    \]
\end{proof}
\newpage

% chapter1:sectionB:exercise2
\begin{exercise}\label{chapter1:sectionB:exercise2}
    Suppose $f: \closedinterval{a,b}\to\mathbb{R}$ is a bounded function. Prove that $f$ is Riemann integrable if and only if
    \[
        L(-f, \closedinterval{a,b}) = -L(f, \closedinterval{a,b}).
    \]
\end{exercise}

\begin{proof}
    For every partition $P$ on $\closedinterval{a, b}$
    \[
        L(-f, P, \closedinterval{a, b}) = -U(f, P, \closedinterval{a, b}).
    \]

    Therefore
    \begin{align*}
        L(-f, \closedinterval{a, b}) & = \sup\limits_{P}L(-f, P, \closedinterval{a, b})   \\
                                     & = \sup\limits_{P}(-U(f, P, \closedinterval{a, b})) \\
                                     & = -\inf\limits_{P}U(f, P, \closedinterval{a, b})   \\
                                     & = -U(f, \closedinterval{a, b}).
    \end{align*}

    $f$ is Riemann integrable if and only if $U(f, \closedinterval{a, b}) = L(f, \closedinterval{a, b})$. $U(f, \closedinterval{a, b}) = L(f, \closedinterval{a, b})$ if and only if $L(-f, \closedinterval{a, b}) = -L(f, \closedinterval{a, b})$.

    Thus $f$ is Riemann integrable if and only if $L(-f, \closedinterval{a, b}) = -L(f, \closedinterval{a, b})$.
\end{proof}
\newpage

% chapter1:sectionB:exercise3
\begin{exercise}\label{chapter1:sectionB:exercise3}
    Suppose $f, g: \closedinterval{a,b}\to\mathbb{R}$ are bounded functions. Prove that
    \[
        L(f, \closedinterval{a,b}) + L(g, \closedinterval{a,b}) \leq L(f + g, \closedinterval{a,b})
    \]

    and
    \[
        U(f + g, \closedinterval{a,b}) \leq U(f, \closedinterval{a,b}) + U(g, \closedinterval{a, b}).
    \]
\end{exercise}

\begin{proof}
    Let $P_{1}, P_{2}$ be two partitions on $\closedinterval{a, b}$.
    \begin{align*}
        L(f, P_{1}, \closedinterval{a, b}) + L(g, P_{2}, \closedinterval{a, b}) & \leq L(f, P_{1}\cup P_{2}, \closedinterval{a, b}) + L(g, P_{1}\cup P_{2}, \closedinterval{a, b}), \\
        U(f, P_{1}, \closedinterval{a, b}) + U(g, P_{2}, \closedinterval{a, b}) & \geq U(f, P_{1}\cup P_{2}, \closedinterval{a, b}) + U(g, P_{1}\cup P_{2}, \closedinterval{a, b}).
    \end{align*}

    We have
    \[
        \inf\limits_{\closedinterval{x_{i-1}, x_{i}}}f + \inf\limits_{\closedinterval{x_{i-1}, x_{i}}}g \leq \inf\limits_{\closedinterval{x_{i-1}, x_{i}}}(f + g).
    \]

    because $\inf\limits_{\closedinterval{x_{i-1}, x_{i}}}(f + g)$ is the greatest lower bound of $f(x) + g(x)$ for $x\in\closedinterval{x_{i-1}, x_{i}}$, meanwhile, $\inf\limits_{\closedinterval{x_{i-1}, x_{i}}}f + \inf\limits_{\closedinterval{x_{i-1}, x_{i}}}g$ is a lower bound.
    \[
        \sup\limits_{\closedinterval{x_{i-1}, x_{i}}}f + \sup\limits_{\closedinterval{x_{i-1}, x_{i}}}g \geq \sup\limits_{\closedinterval{x_{i-1}, x_{i}}}(f + g)
    \]

    because $\sup\limits_{\closedinterval{x_{i-1}, x_{i}}}(f + g)$ is the least upper bound of $f(x) + g(x)$ for $x\in\closedinterval{x_{i-1}, x_{i}}$, meanwhile, $\sup\limits_{\closedinterval{x_{i-1}, x_{i}}}f + \sup\limits_{\closedinterval{x_{i-1}, x_{i}}}g$ is an upper bound.

    Apply these results to the subintervals of $P_{1}\cup P_{2}$
    \begin{align*}
        L(f, P_{1}\cup P_{2}, \closedinterval{a, b}) + L(g, P_{1}\cup P_{2}, \closedinterval{a, b}) & \leq L(f + g, P_{1}\cup P_{2}, \closedinterval{a, b}) \leq L(f + g, \closedinterval{a, b}), \\
        U(f, P_{1}\cup P_{2}, \closedinterval{a, b}) + U(g, P_{1}\cup P_{2}, \closedinterval{a, b}) & \geq U(f + g, P_{1}\cup P_{2}, \closedinterval{a, b}) \geq U(f + g, \closedinterval{a, b}).
    \end{align*}

    $L(f + g, \closedinterval{a, b})$ is an upper bound of $L(f, P_{1}, \closedinterval{a, b}) + L(g, P_{2}, \closedinterval{a, b})$ as $P_{1}$ and $P_{2}$ change.

    $L(f, \closedinterval{a, b}) + L(g, \closedinterval{a, b})$ is the least upper bound of $L(f, P_{1}, \closedinterval{a, b}) + L(g, P_{2}, \closedinterval{a, b})$ as $P_{1}$ and $P_{2}$ change.

    $U(f + g, \closedinterval{a, b})$ is a lower bound of $U(f, P_{1}, \closedinterval{a, b}) + U(g, P_{2}, \closedinterval{a, b})$ as $P_{1}$ and $P_{2}$ change.

    $U(f, \closedinterval{a,b}) + U(g, \closedinterval{a, b})$ is the greatest lower bound of $U(f, P_{1}, \closedinterval{a, b}) + U(g, P_{2}, \closedinterval{a, b})$ as $P_{1}$ and $P_{2}$ change.

    Thus
    \begin{align*}
        L(f, \closedinterval{a,b}) + L(g, \closedinterval{a,b}) \leq L(f + g, \closedinterval{a,b}), \\
        U(f + g, \closedinterval{a,b}) \leq U(f, \closedinterval{a,b}) + U(g, \closedinterval{a, b}).
    \end{align*}
\end{proof}
\newpage

% chapter1:sectionB:exercise4
\begin{exercise}\label{chapter1:sectionB:exercise4}
    Given an example of bounded functions $f, g: \closedinterval{0, 1}\to \mathbb{R}$ such that
    \[
        L(f, \closedinterval{0,1}) + L(g, \closedinterval{0,1}) < L(f + g, \closedinterval{0, 1})
    \]

    and
    \[
        U(f + g, \closedinterval{0,1}) < U(f, \closedinterval{0,1}) + U(g, \closedinterval{0, 1}).
    \]
\end{exercise}

\begin{proof}
    Let $f, g: \closedinterval{0, 1}\to\mathbb{R}$ be the following functions
    \[
        \begin{split}
            f(x) = \begin{cases}
                       1 & \text{if $x$ is rational}   \\
                       0 & \text{if $x$ is irrational}
                   \end{cases} \\
            g(x) = \begin{cases}
                       0 & \text{if $x$ is rational}   \\
                       1 & \text{if $x$ is irrational}
                   \end{cases}
        \end{split}
    \]

    then $(f + g)(x) = 1$ for all $x\in\closedinterval{0, 1}$ so $f + g$ is Riemann integrable on $\closedinterval{0, 1}$ and
    \[
        U(f + g, \closedinterval{0, 1}) = L(f + g, \closedinterval{0, 1}) = 1.
    \]

    We have
    \[
        \begin{split}
            L(f, \closedinterval{0, 1}) = L(g, \closedinterval{0, 1}) = 0 \qquad L(f + g, \closedinterval{0, 1}) = 1, \\
            U(f, \closedinterval{0, 1}) = U(g, \closedinterval{0, 1}) = 1 \qquad U(f + g, \closedinterval{0, 1}) = 1.
        \end{split}
    \]

    Therefore
    \[
        \begin{split}
            L(f, \closedinterval{0, 1}) + L(g, \closedinterval{0, 1}) = 0 < 1 = L(f + g, \closedinterval{0, 1}), \\
            U(f, \closedinterval{0, 1}) + U(g, \closedinterval{0, 1}) = 2 > 1 = U(f + g, \closedinterval{0, 1}).
        \end{split}
    \]
\end{proof}
\newpage

% chapter1:sectionB:exercise5
\begin{exercise}\label{chapter1:sectionB:exercise5}
    Given an example of a sequence of continuous real-valued functions $f_{1}, f_{2}, \ldots$ on $\closedinterval{0, 1}$ and a continuous real-valued function $f$ on $\closedinterval{0, 1}$ such that
    \[
        f(x) = \lim\limits_{k\to\infty}f_{k}(x)
    \]

    for each $x\in\closedinterval{0,1}$ but
    \[
        \int^{1}_{0}f \ne \lim\limits_{k\to\infty}\int^{1}_{0}f_{k}.
    \]
\end{exercise}

\begin{proof}
    Consider the following sequence of conditions real-valued functions on $\closedinterval{0, 1}$
    \[
        f_{k}(x) = k^{2}x{(1 - x)}^{k}.
    \]

    Integrate by parts and use the fundamental theorem of calculus, we obtain
    \[
        \int^{1}_{0}f_{k} = \frac{k^{2}}{(k + 1)(k + 2)}
    \]

    On the other hand, for each $x\in\closedinterval{0, 1}$
    \[
        f(x) = \lim\limits_{k\to\infty}f_{k}(x) = 0
    \]

    so $f$ is continuous and $\int^{1}_{0}f = 0$.
    \[
        \int^{1}_{0}f = 0 \ne 1 = \lim\limits_{k\to\infty}\int^{1}_{0}f_{k}.
    \]
\end{proof}
\newpage
