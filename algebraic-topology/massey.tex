\documentclass[12pt]{book}

\usepackage[left=2cm,right=2cm,top=2.5cm,bottom=2.5cm]{geometry}
\usepackage[unicode=true,colorlinks=true,linkcolor=blue]{hyperref}

\usepackage{amsmath}
\usepackage{amsfonts}
\usepackage{amssymb}
\usepackage{amsthm}
\usepackage{amscd}
\usepackage{mathtools}
\usepackage{mathrsfs}
\usepackage{cases}

\usepackage{fancyhdr}
\usepackage{xcolor}
\usepackage{titlesec}
\usepackage{indentfirst}
\usepackage{chngcntr}
\usepackage{caption}
\usepackage{subcaption}
\usepackage{booktabs}
\usepackage{makecell}
\usepackage[inline]{enumitem}
\usepackage{setspace}
\usepackage{pgf,tikz}
\usepackage{tikz-cd}
\usepackage{pgfplots}
\usetikzlibrary{matrix}
\usetikzlibrary{arrows}
\usetikzlibrary{decorations.pathmorphing}
\usetikzlibrary{positioning}
\usetikzlibrary{calc}
\usetikzlibrary{intersections}
\usetikzlibrary{through}
\usetikzlibrary{math}
\usetikzlibrary{patterns}
\pgfplotsset{compat=1.15}

\setcounter{chapter}{0}
\setstretch{1.4142}
\theoremstyle{definition}
\newtheorem{innercustomgeneric}{\customgenericname}
\providecommand{\customgenericname}{}
\newcommand{\newcustomtheorem}[2]{%
  \newenvironment{#1}[1]
  {%
   \renewcommand\customgenericname{#2}%
   \renewcommand\theinnercustomgeneric{##1}%
   \innercustomgeneric%
  }
  {\endinnercustomgeneric}
}

\newcustomtheorem{theorem}{Theorem}
\newcustomtheorem{prop}{Proposition}
\newcustomtheorem{lemma}{Lemma}
\newcustomtheorem{example}{Example}
\newcustomtheorem{exercise}{Exercise}
\newcustomtheorem{problem}{Problem}
\newcustomtheorem{corollary}{Corollary}

\newtheorem{note}{Note}
\counterwithin{note}{chapter}

\captionsetup{labelfont={bf},labelsep=period}
\counterwithin{figure}{chapter}
\counterwithin{table}{chapter}

\newenvironment{sqcases}{%
    \matrix@check\sqcases\env@sqcases
}{%
    \endarray\right.%
}
\def\env@sqcases{%
\let\@ifnextchar\new@ifnextchar
\left\lbrack{}
\def\arraystretch{1.2}%
\array{@{}l@{\quad}l@{}}%
}
\renewcommand{\emptyset}{\varnothing}
\newcommand{\innerprod}[1]{\left\langle{#1}\right\rangle}
\newcommand{\anglebracket}[1]{\left\langle{#1}\right\rangle}
\newcommand{\abs}[1]{\left\vert{#1}\right\vert}
\newcommand{\openinterval}[1]{\left\rbrack{#1}\right\lbrack}
\newcommand{\closedinterval}[1]{\left\lbrack{#1}\right\rbrack}
\newcommand{\halfopenleft}[1]{\left\rbrack{#1}\right\rbrack}
\newcommand{\halfopenright}[1]{\left\lbrack{#1}\right\lbrack}
\newcommand{\set}[1]{\left\{{#1}\right\}}
\newcommand{\tuple}[1]{\left({#1}\right)}
\newcommand{\openball}[2]{B_{#1}\left({#2}\right)}
\newcommand{\closedball}[2]{\overline{B}_{#1}\left({#2}\right)}

\title{William S.\@ Massey's ``Algebraic Topology: An Introduction'': Notes, Exercises, and Problems}
\author{Ngo Quang Duong}
\date{\today}

\begin{document}

\maketitle

\tableofcontents

\chapter{Two-Dimensional Manifolds}

\chapter{The Fundamental Group}

\chapter{Free Groups and Free Products of Groups}

\chapter{Seifert and Van Kampen Theorem on the Fundamental Group of the Union of Two Spaces. Applications}

\chapter{Covering Spaces}

\chapter{The Fundamental Group and Covering Spaces of a Graph. Applications to Group Theory}

\chapter{The Fundamental Group of Higher Dimensional Spaces}

\chapter{Epilogue}

\appendix

\chapter{The Quotient Space or Identification Space Topology}

\begin{exercise}{5.1}
	Let \( X \) and \( Y \) be topological spaces, and let \( f: X \to Y, g: Y \to X \) be continuous maps such that \( fg \) is the identity map of \( Y \) onto itself. Prove the following statements:
	\begin{enumerate}[itemsep=0pt,label={(\alph*)}]
		\item \( f \) is onto and \( g \) is one-to-one.
		\item \( Y \) has the quotient topology determined by \( f \).
		\item \( g \) maps \( Y \) homeomorphically onto a subspace of \( X \) (i.e., \(Y\) has the subspace topology determined by \(g\)).
		\item If \( X \) is a Hausdorff space, then so is \( Y \).
	\end{enumerate}
\end{exercise}

\begin{quotation}
	The result in (d) means: If a continuous map \( f: X \to Y \) has a right continuous inverse and \( X \) is Hausdorff then \( Y \) is Hausdorff.
\end{quotation}

\begin{proof}
	\begin{enumerate}[itemsep=0pt,label={(\alph*)}]
		\item Let \( y \in Y \) and \( x = g(y) \) then \( f(x) = fg(y) = y \), which implies \( f \) is onto. If \( y_{1} \ne y_{2} \) then \( fg(y_{1}) \ne fg(y_{2}) \), which means \( g(y_{1}) \ne g(y_{2}) \), so \( g \) is one-to-one.
		\item If \( U \subseteq Y \) is open then \( f^{-1}(U) \) is open in \( X \) for \( f \) is continuous.

		      Conversely, if \( f^{-1}(U) \) is open in \( X \) then \( U = {(fg)}^{-1}(U) = g^{-1}(f^{-1}(U)) \) is open in \( Y \) as \( g \) is continuous.

		      Hence \( Y \) has the quotient topology determined by \( f \).
		\item Let \( S = g(Y) \) then \( S \subseteq X \).

		      If \( U \subseteq S \) is open then \( g^{-1}(U) \) is open in \( Y \) for \( g \) is continuous.

		      Conversely, if \( V \subseteq Y \) is open then \( f^{-1}(V) \) is open in \( X \). Therefore \( g(V) = f^{-1}(V) \cap S \), which means \( g(V) \) is open in \( S \).

		      Hence \( Y \) has the subspace topology determined by \( g \).
		\item Let \( y_{1}, y_{2} \) be two distinct points of \( Y \) then \( g(y_{1}) \ne g(y_{2}) \) as \( g \) is one-to-one. Since \( X \) is Hausdorff, there are disjoint open subsets \( U_{1}, U_{2} \subseteq X \) such that \( g(y_{1}) \subseteq U_{1}, g(y_{2}) \subseteq U_{2} \). Moreover, \( g^{-1}(U_{1}), g^{-1}(U_{2}) \) are neighborhoods of \( y_{1}, y_{2} \) and
		      \[
			      g^{-1}(U_{1}) \cap g^{-1}(U_{2}) = g^{-1}(U_{1} \cap U_{2}) = \varnothing.
		      \]

		      Hence \( Y \) is Hausdorff.
	\end{enumerate}
\end{proof}

\chapter{Permutation Groups or Transformation Groups}

\begin{exercise}{1.1}
	Assume that \( E \) is a left \( G \)-space. For any \( x \in E \) and \( g \in G \), define \( x \cdot g = (g^{-1}) \cdot x \).

	With this definition, prove that \( E \) is a right \( G \)-space.
\end{exercise}

\begin{proof}
	\( x \cdot 1 = {(1^{-1})} \cdot x = 1 \cdot x = x \) for every \( x \in E \).

	For every \( x \in E \) and \( g_{1}, g_{2} \in G \)
	\begingroup
	\allowdisplaybreaks%
	\begin{align*}
		x \cdot (g_{1}g_{2}) & = {(g_{1}g_{2})}^{-1} \cdot x = (g_{2}^{-1}g_{1}^{-1}) \cdot x = g_{2}^{-1} \cdot (g_{1}^{-1} \cdot x) \\
		                     & = (g_{1}^{-1} \cdot x) \cdot g_{2} = (x \cdot g_{1}) \cdot g_{2}.
	\end{align*}
	\endgroup

    Hence \( E \) is a right \( G \)-space.
\end{proof}


\end{document}
