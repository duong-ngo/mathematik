% chktex-file 8
\chapter{Metric Spaces}

\section{Metric Space}

\begin{exercise}\label{chapter1:section1:exercise1}
    Show that the real line is a metric space.
\end{exercise}

\begin{proof}
    $\abs{x - y}\geq 0$ for all $x, y\in\mathbb{R}$.

    $\abs{x - y} = \abs{y - x}$ for all $x, y\in\mathbb{R}$.

    $\abs{x - y} = 0$ iff $x - y = 0$ iff $x = y$.

    $\abs{x - y} = \abs{(x - z) + (z - y)}\leq \abs{x-z} + \abs{z-y}$ for all $x, y, z\in\mathbb{R}$.
\end{proof}

\begin{exercise}\label{chapter1:section1:exercise2}
    Does $d(x, y) = {(x - y)}^{2}$ define a metric on the set of all real numbers?
\end{exercise}

\begin{proof}
    No.

    Here is my counterexample: ${(2 - 0)}^{2} > {(2 - 1)}^{2} + {(1 - 0)}^{2}$.
\end{proof}

\begin{exercise}\label{chapter1:section1:exercise3}
    Show that $d(x, y) = \sqrt{\abs{x - y}}$ defines a metric on the set of all real numbers.
\end{exercise}

\begin{proof}
    $\sqrt{\abs{x - y}}\geq 0$ for all $x, y\in\mathbb{R}$.

    $\sqrt{\abs{x - y}} = \sqrt{\abs{y - x}}$ for all $x, y\in\mathbb{R}$.

    $\sqrt{\abs{x - y}} = 0$ iff $x = y$.

    For all $x, y, z\in\mathbb{R}$
    \[
        \abs{x - y}\leq \abs{x - z} + \abs{z - y} \leq \abs{x - z} + \abs{z - y} + 2\sqrt{\abs{x - z}}\sqrt{\abs{z - y}}
    \]

    so $\sqrt{\abs{x - y}}\leq \sqrt{\abs{x - z}} + \sqrt{\abs{z - y}}$.
\end{proof}

\begin{exercise}\label{chapter1:section1:exercise4}
    Find all metrics on a set $X$ consisting of two points. Consisting of one point.
\end{exercise}

\begin{proof}
    If $X$ consists of two points, let $X = \{ a, b \}$.

    I define $d_{c}(x, y) = c$ if $x\ne y$ and $0$ if $x = y$, where $c$ is a positive real number. Then $d_{c}$ is indeed a metric. Now let $d$ be a metric on $X$, then $d(a, a) = d(b, b) = 0$ and $d(a, b) = r$ where $r$ is a real number, then $d = d_{r}$. Hence all metrics on $X$ are $d_{c}$, where
    \[
        d_{c}(x, y) = \begin{cases}
            c & \text{if $x\ne y$} \\
            0 & \text{if $x = y$}
        \end{cases}
    \]

    If $X$ consists of one point, then all possible metrics are $d: X\times X\to \mathbb{R}$ such that $d(x, x) = 0$.
\end{proof}

\begin{exercise}\label{chapter1:section1:exercise5}
    Let $d$ be a metric on $X$. Determine all constants $k$ such that (i) $kd$, (ii) $d + k$ is a metric on $X$.
\end{exercise}

\begin{proof}
    \begin{enumerate}[label={(\roman*)}]
        \item If $X$ consists of only one element, then $kd$ is a metric for all $k\in\mathbb{R}$.

              If $X$ consists of more than one element, then $kd$ is a metric if and only if $k$ is a positive real number.
        \item If $k > 0$ then $d + k$ is not a metric, because $d(x, y) + k$ is never $0$ if $k > 0$.

              If $k < 0$ then $d + k$ is not a metric, because $d(x, y) + k$ can be negative.

              If $k = 0$ then $d + k$ is a metric.
    \end{enumerate}
\end{proof}

\begin{exercise}\label{chapter1:section1:exercise6}
    Show that $d$ in $1.1-6$ satisfies the triangle inequality.
\end{exercise}

\begin{proof}
    $X$ is the set of all bounded sequences of complex numbers, any $x\in X$ is of the form $x = (\xi_{1}, \xi_{2}, \ldots)$. $d$ is a metric on $X$ defined by
    \[
        d(x, y) = \sup_{j\in\mathbb{N}}\abs{\xi_{j} - \eta_{j}}.
    \]

    Let $x, y, z$ be bounded sequences of complex numbers. $x = {(\xi_{n})}$, $y = {(\eta_{n})}$, $z = (\zeta_{n})$.

    For every $n\in\mathbb{N}$, $\abs{\xi_{n} - \eta_{n}}\leq \abs{\xi_{n} - \zeta_{n}} + \abs{\zeta_{n} - \eta_{n}}$. On the other hand
    \[
        \abs{\xi_{n} - \zeta_{n}} + \abs{\zeta_{n} - \eta_{n}} \leq \sup_{j\in\mathbb{N}}\abs{\xi_{j} - \zeta_{j}} + \sup_{j\in\mathbb{N}}\abs{\zeta_{j} - \eta_{j}}.
    \]

    So $\sup_{j\in\mathbb{N}}\abs{\xi_{n} - \zeta_{n}} + \sup_{j\in\mathbb{N}}\abs{\zeta_{n} - \eta_{n}}$ is an upper bound of $\abs{\xi_{1} - \eta_{1}}, \abs{\xi_{2} - \eta_{2}}, \ldots$ Because supremum is the least upper bound, it follows that
    \[
        \sup_{j\in\mathbb{N}}\abs{\xi_{j} - \eta_{j}} \leq \sup_{j\in\mathbb{N}}\abs{\xi_{j} - \zeta_{j}} + \sup_{j\in\mathbb{N}}\abs{\zeta_{j} - \eta_{j}}.
    \]

    Thus $d(x, y)\leq d(x, z) + d(z, y)$.
\end{proof}

\begin{exercise}\label{chapter1:section1:exercise7}
    If $A$ is the subspace of $L^{\infty}$ consisting of all sequences of zeros and ones, what is the induced metric on $A$?
\end{exercise}

\begin{proof}
    The induced metric on $A$ is the discrete metric.
\end{proof}

\begin{exercise}\label{chapter1:section1:exercise8}
    Show that another metric $\widetilde{d}$ on the set $X$ in 1.1-7 is defined by
    \[
        \widetilde{d}(x, y) = \int^{b}_{a}\abs{x(t) - y(t)}dt.
    \]
\end{exercise}

\begin{proof}
    For all $x, y\in X$, $\widetilde{d}(x, y) = \int^{b}_{a}\abs{x(t) - y(t)}dt\geq 0$.

    $\widetilde{d}(x, y) = \int^{b}_{a}\abs{x(t) - y(t)}dt = 0$ if and only if $x(t) = y(t)$ for all $t\in [a, b]$, equivalently, $x = y$.

    Because $\abs{x(t) - y(t)} = \abs{y(t) - x(t)}$, $\widetilde{d}(x, y) = \widetilde{d}(y, x)$.

    For all $x, y, z\in X$,
    \begin{align*}
        \widetilde{d}(x, y) & = \int^{b}_{a}\abs{x(t) - y(t)}dt                                        \\
                            & \leq \int^{b}_{a}(\abs{x(t) - z(t)} + \abs{z(t) - y(t)})dt               \\
                            & \leq \int^{b}_{a}\abs{x(t) - z(t)}dt + \int^{b}_{a}{\abs{z(t) - y(t)}}dt \\
                            & = \widetilde{d}(x, z) + \widetilde{d}(z, y).\qedhere
    \end{align*}
\end{proof}

\begin{exercise}\label{chapter1:section1:exercise9}
    Show that $d$ in 1.1-8 is a metric.
\end{exercise}

\begin{proof}
    $d(x, y) = d(y, x)$ for all $x, y\in X$.

    $d(x, y)\geq 0$ for all $x, y\in X$ because $d(x, y)$ is either $0$ or $1$.

    $d(x, y) = 0$ if and only if $x = y$ by definition of $d$.

    For all $x, y, z\in X$,
    \begin{itemize}
        \item if $x = y$, $y = z$, then $d(x, y) = d(x, z) + d(z, y)$.
        \item if $x = y$, $y\ne z$, then $d(x, y) < d(x, z) + d(z, y)$.
        \item if $x\ne y$, and $z = x$ or $z = y$, then $d(x, y) = d(x, z) + d(z, y)$.
        \item if $x, y, z$ are pairwise distinct, then $d(x, y) < d(x, z) + d(z, y)$.
    \end{itemize}

    Hence $d(x, y)\leq d(x, z) + d(z, y)$.
\end{proof}

\begin{exercise}[Hamming distance]\label{chapter1:section1:exercise10}
    Let $X$ be the set of all $n$-tuples of zeros and ones. Show that $X$ consists of $2^{n}$ elements and a metric $d$ on $X$ is defined by $d(x, y) =$ number of places where $x$ and $y$ have different entries.
\end{exercise}

\begin{proof}
    The $k$th slot of an element of $X$ is either zero or one, and every element of $X$ has $n$ slots, so $X$ has $2^{n}$ elements.

    I will prove $d$ is a metric on $X$ by using mathematical induction. For each $n$, I use the notation $d_{n}$ instead of $d$.

    The statement is true for $n = 1$ because when $n = 1$, $d_{1}$ is the discrete metric.

    Assume the statement is true for $n = k\geq 1$.
    \begin{align*}
        d_{n+1}(x, y) & = d_{n}(x, y) + d_{1}(x_{n+1}, y_{n+1})                                                                            \\
                      & \leq d_{n}(x, z) + d_{n}(z, y) + d_{1}(x_{n+1}, y_{n+1})                           & \text{(induction hypothesis)} \\
                      & \leq d_{n}(x, z) + d_{n}(z, y) + d_{1}(x_{n+1}, z_{n+1}) + d_{1}(z_{n+1}, y_{n+1})                                 \\
                      & = d_{n+1}(x, z) + d_{n+1}(z, y).
    \end{align*}

    Hence for every $n$, $d_{n}$ is a metric on the set of all $n$-tuples of zeros and ones.
\end{proof}

\begin{exercise}\label{chapter1:section1:exercise11}
    Prove (1).
\end{exercise}

\begin{proof}
    For $n = 2$, $d(x_{1}, x_{2})\leq d(x_{1}, x_{2})$.

    Assume that $d(x_{1}, x_{k})\leq d(x_{1}, x_{2}) + \cdots + d(x_{k-1}, x_{k})$, then
    \begin{align*}
        d(x_{1}, x_{k+1}) & \leq d(x_{1}, x_{k}) + d(x_{k}, x_{k+1})                               \\
                          & \leq d(x_{1}, x_{2}) + \cdots + d(x_{k-1}, x_{k}) + d(x_{k}, x_{k+1}).
    \end{align*}

    According to the principle of mathematical induction
    \[
        d(x_{1}, x_{n}) \leq d(x_{1}, x_{2}) + \cdots + d(x_{n-1}, x_{n}).\qedhere
    \]
\end{proof}

\begin{exercise}[Triangle inequality]\label{chapter1:section1:exercise12}
    The triangle inequality has several useful consequences. For instance, using (1), show that
    \[
        \abs{d(x, y) - d(z, w)}\leq d(x, z) + d(y, w).
    \]
\end{exercise}

\begin{proof}
    Because
    \[
        d(x, y)\leq d(x, z) + d(z, w) + d(w, y)
    \]

    and
    \[
        d(z, w)\leq d(z, x) + d(x, y) + d(y, w)
    \]

    it follows that $\abs{d(x, y) - d(z, w)}\leq d(x, z) + d(y, w)$.
\end{proof}

\begin{exercise}\label{chapter1:section1:exercise13}
    Using the triangle inequality, show that
    \[
        \abs{d(x, z) - d(y, z)} \leq d(x, y).
    \]
\end{exercise}

\begin{proof}
    $d(x, z)\leq d(x, y) + d(y, z)$ and $d(y, z)\leq d(x, y) + d(x, z)$. Therefore $d(x, z) - d(y, z)\leq d(x, y)$ and $d(y, z) - d(x, z)\leq d(x, y)$. Hence
    \[
        \abs{d(x, z) - d(y, z)} \leq d(x, y).\qedhere
    \]
\end{proof}

\begin{exercise}[Axioms of a metric]\label{chapter1:section1:exercise14}
    (M1) to (M4) could be replaced by other axioms (without changing the definition). For instance, show that (M3) and (M4) could be obtain from (M2) and
    \[
        d(x, y)\leq d(z, x) + d(z, y).
    \]
\end{exercise}

\begin{proof}
    Suppose that (M1) (M2) (M4) hold. Then $d(x, y)\leq d(y, x) + d(y, y) = d(y, x)$ and $d(y, x)\leq d(x, y) + d(x, x) = d(x, y)$. So $d(x, y) = d(y, x)$.

    Finally, $d(x, y)\leq d(z, x) + d(z, y) = d(x, z) + d(z, y)$.
\end{proof}

\begin{exercise}\label{chapter1:section1:exercise15}
    Show that nonnegativity of a metric follows from (M2) to (M4).
\end{exercise}

\begin{proof}
    For all $x, y\in X$, $d(x, y) + d(y, x) \geq d(x, x) = 0$. So $d(x, y) + d(y, x) = 2d(x, y)\geq 0$ for all $x, y\in X$. Therefore $d(x, y)\geq 0$ for all $x, y\in X$.

    Hence nonnegativity of a metric follows from (M2) to (M4).
\end{proof}

\section{Further Examples of Metric Spaces}

\begin{exercise}\label{chapter1:section2:exercise1}
    Show that in 1.2-1 we can obtain another metric by replacing $1/2^{j}$ with $\mu_{j} > 0$ such that $\sum \mu_{j}$ converges.
\end{exercise}

\begin{proof}
    On the sequence space, we define
    \[
        d(x, y) = \sum^{\infty}_{j=1} \mu_{j}\frac{\abs{\xi_{j} - \eta_{j}}}{1 + \abs{\xi_{j} - \eta_{j}}}
    \]

    where $x = (\xi_{j})$ and $y = (\eta_{j})$ and $\mu_{j} > 0$, $\sum \mu_{j}$ converges.

    $d(x, y) = d(y, x)$. $d(x, y)\geq 0$ and $d(x, y) = 0$ if and only if $x = y$.
    \[
        \frac{\abs{\xi_{j} - \eta_{j}}}{1 + \abs{\xi_{j} - \eta_{j}}} \leq \frac{\abs{\xi_{j} - \zeta_{j}}}{1 + \abs{\xi_{j} - \zeta_{j}}} + \frac{\abs{\zeta_{j} - \eta_{j}}}{1 + \abs{\zeta_{j} - \eta_{j}}}
    \]

    so
    \[
        \mu_{j}\frac{\abs{\xi_{j} - \eta_{j}}}{1 + \abs{\xi_{j} - \eta_{j}}} \leq \mu_{j}\frac{\abs{\xi_{j} - \zeta_{j}}}{1 + \abs{\xi_{j} - \zeta_{j}}} + \mu_{j}\frac{\abs{\zeta_{j} - \eta_{j}}}{1 + \abs{\zeta_{j} - \eta_{j}}}.
    \]

    Hence $d(x, y)\leq d(x, z) + d(z, y)$. Thus $d$ is a metric on the sequence space.
\end{proof}

\begin{exercise}\label{chapter1:section2:exercise2}
    Using (6), show that the geometric mean of two positive numbers does not exceed the arithmetic mean.
\end{exercise}

\begin{proof}
    Let $a$ and $b$ be two positive numbers. By Young's inequality,
    \[
        \sqrt{ab} = \sqrt{a}\sqrt{b} \leq \frac{\sqrt{a^{2}}}{2} + \frac{\sqrt{b^{2}}}{2} = \frac{a + b}{2}.
    \]
\end{proof}

\begin{exercise}\label{chapter1:section2:exercise3}
    Show that the Cauchy-Schwarz inequality (11) implies
    \[
        {(\abs{\xi_{1}} + \cdots + \abs{\xi_{n}})}^{2} \leq n(\abs{\xi_{1}}^{2} + \cdots + \abs{\xi_{n}}^{2}).
    \]
\end{exercise}

\begin{proof}
    Apply Cauchy-Schwarz' inequality to $(1, \ldots, 1)$ and $(\xi_{1}, \ldots, \xi_{n})$, we have
    \[
        \abs{\xi_{1}} + \cdots + \abs{\xi_{n}} \leq \sqrt{n(\abs{\xi_{1}}^{2} + \cdots + \abs{\xi_{n}}^{2})}
    \]

    so ${(\abs{\xi_{1}} + \cdots + \abs{\xi_{n}})}^{2} \leq n(\abs{\xi_{1}}^{2} + \cdots + \abs{\xi_{n}}^{2})$.
\end{proof}

\begin{exercise}[Space $\ell^{p}$]\label{chapter1:section2:exercise4}
    Find a sequence which converges to $0$, but not in any space $\ell^{p}$, where $1\leq p < +\infty$.
\end{exercise}

\begin{proof}
    Let $x = (\xi_{n})$ where
    \[
        \xi_{n} = \frac{1}{\ln (1 + n)}.
    \]

    Moreover,
    \[
        \lim\limits_{n\to\infty}\frac{{(\ln (1 + n))}^{p}}{n} = 0
    \]

    for all $1\leq p < +\infty$. There exists $N$ such that ${(\ln(1+n))}^{p} < n$ for every $n > N$.
    \begin{align*}
        \sum^{\infty}_{n=1}\frac{1}{{(\ln(1+n))}^{p}} & = \sum^{N}_{n=1}\frac{1}{{(\ln(1+n))}^{p}} + \sum^{\infty}_{n=N+1}\frac{1}{{(\ln(1+n))}^{p}} \\
                                                      & > \sum^{N}_{n=1}\frac{1}{{(\ln(1+n))}^{p}} + \sum^{\infty}_{n=N+1}\frac{1}{n}                \\
                                                      & = \infty
    \end{align*}

    so $x\notin \ell^{p}$ for every $1\leq p < +\infty$.
\end{proof}

\begin{exercise}\label{chapter1:section2:exercise5}
    Find a sequence $x$ which is in $\ell^{p}$ with $p > 1$ but $x\notin\ell^{1}$.
\end{exercise}

\begin{proof}
    If $x = \left(1, \frac{1}{2}, \frac{1}{3}, \ldots\right)$, then $x$ is in $\ell^{p}$ with $p > 1$ but $x\notin\ell^{1}$.
\end{proof}

\begin{exercise}[Diameter, bounded set]\label{chapter1:section2:exercise6}
    The diameter $\delta(A)$ of a nonempty set $A$ in a metric space $(X, d)$ is defined to be
    \[
        \delta(A) = \sup\limits_{x,y\in A} d(x, y).
    \]

    $A$ is said to be bounded if $\delta(A) < \infty$. Show that $A\subset B$ implies $\delta(A) \leq \delta(B)$.
\end{exercise}

\begin{proof}
    For every $x, y\in A$, $d(x, y) \leq \delta(B)$. So $\delta(B)$ is an upper bound of $d(x, y)$. Therefore $\sup\limits_{x,y\in A} d(x, y)\leq \delta(B)$, hence $\delta(A)\leq \delta(B)$.
\end{proof}

\begin{exercise}\label{chapter1:section2:exercise7}
    Show that $\delta(A) = 0$ if and only if $A$ consists of a single point.
\end{exercise}

\begin{proof}
    Suppose $\delta(A) = 0$. Then $d(x, y) = 0$ for every $x, y\in A$ so $A$ consists of a single point.

    Conversely, if $A$ consists of a single point, then $\delta(A) = 0$.
\end{proof}

\begin{exercise}[Distance between sets]\label{chapter1:section2:exercise8}
    The distance $D(A, B)$ between two nonempty subsets $A$ and $B$ of a metric space $(X, d)$ is defined to be
    \[
        D(A, B) = \inf\limits_{\substack{a\in A \\ b\in B}}d(a, b).
    \]

    Show that $D$ does not define a metric on the power set of $X$.
\end{exercise}

\begin{proof}
    Let $A, B$ be two distinct subsets of $X$ such that $A\cap B\ne \varnothing$, then $D(A, B) = 0$. Therefore $D$ does not define a metric on the power set of $X$.
\end{proof}

\begin{exercise}\label{chapter1:section2:exercise9}
    If $A\cap B \ne \varnothing$, show that $D(A, B) = 0$. What about the converse?
\end{exercise}

\begin{proof}
    Let $x\in A\cap B$, then $\min\limits_{\substack{a\in A \\ b\in B}}d(a, b) = 0$, so $D(A, B) = 0$.

    Conversely, if $D(A, B) = 0$, $A\cap B$ is not necessarily nonempty. For example, let $X = \mathbb{R}$ and $d(x, y) = \abs{x - y}$,
    \[
        A = \left\{ 1, \frac{1}{2}, \frac{1}{3}, \ldots \right\}\qquad B = \left\{ -1, \frac{-1}{2}, \frac{-1}{3}, \ldots \right\}
    \]

    $D(A, B) = 0$ because for every $\varepsilon > 0$, there exists a positive integer $n$ such that
    \[
        \abs{\frac{1}{n} - \frac{-1}{n}} < \varepsilon
    \]

    however, $A\cap B = \varnothing$.
\end{proof}

\begin{exercise}\label{chapter1:section2:exercise10}
    The distance $D(x, B)$ from a point $x$ to a non-empty subset $B$ of $(X, d)$ is defined to be
    \[
        D(x, B) = \inf\limits_{b\in B} d(x, b).
    \]

    Show that for any $x, y\in X$,
    \[
        \abs{D(x, B) - D(y, B)}\leq d(x, y).
    \]
\end{exercise}

\begin{proof}
    For $b\in B$, $d(x, b)\leq d(x, y) + d(y, b)$ and $d(y, b)\leq d(x, y) + d(x, b)$.
    \[
        \begin{split}
            \inf\limits_{b\in B}(d(x, y) + d(y, b)) = d(x, y) + D(y, B) \\
            \inf\limits_{b\in B}(d(x, y) + d(x, b)) = d(x, y) + D(x, B)
        \end{split}
    \]

    so
    \[
        \begin{split}
            d(x, y) + D(y, B) \geq d(x, b)\geq D(x, B) \\
            d(x, y) + D(x, B) \geq d(y, b)\geq D(y, B)
        \end{split}
    \]

    Thus $\abs{D(x, B) - D(y, B)} \leq d(x, y)$.
\end{proof}

\begin{exercise}\label{chapter1:section2:exercise11}
    If $(X, d)$ is any metric space, show that another metric on $X$ is defined by
    \[
        \widetilde{d}(x, y) = \frac{d(x, y)}{1 + d(x, y)}
    \]

    and $X$ is bounded in the metric $\widetilde{d}$.
\end{exercise}

\begin{proof}
    For every $x, y\in X$,
    \[
        \widetilde{d}(x, y) = \frac{d(x, y)}{1 + d(x, y)} < 1
    \]

    so $\delta(X)\leq 1$. Thus $X$ is bounded in the metric $\widetilde{d}$.
\end{proof}

\begin{exercise}\label{chapter1:section2:exercise12}
    Show that the union of two bounded sets $A$ and $B$ in a metric space is a bounded set.
\end{exercise}

\begin{proof}
    Let $x, y\in A\cup B$ and $a\in A, b\in B$.

    If $x, y\in A$ then $d(x, y) \leq \delta(A)$. If $x, y\in B$ then $d(x, y) \leq \delta(B)$.

    If $x\in A$ and $y\in B$ then $d(x, y)\leq d(x, a) + d(a, b) + d(b, y) \leq \delta(A) + d(a, b) + \delta(B)$.

    Thus $d(x, y)\leq \delta(A) + d(a, b) + \delta(B)$ for all $x, y\in A\cup B$ so $A\cup B$ is bounded.
\end{proof}

\begin{exercise}[Product of metric spaces]\label{chapter1:section2:exercise13}
    The Cartesian product $X = X_{1} \times X_{2}$ of two metric spaces $(X_{1},  d_{1})$ and $(X_{2}, d_{2})$ can be made into a metric space $(X, d)$ in many ways. For instance, show that a metric $d$ is defined by
    \[
        d(x, y) = d_{1}(x_{1}, y_{1}) + d_{2}(x_{2}, y_{2}),
    \]

    where $x = (x_{1}, x_{2}), y = (y_{1}, y_{2})$.
\end{exercise}

\begin{proof}
    For every $x, y\in X$
    \[
        d(x, y) = d_{1}(x_{1}, y_{1}) + d_{2}(x_{2}, y_{2}) \geq 0 + 0 = 0.
    \]

    $d(x, y) = 0$ if and only if $x_{1} = y_{1}$ and $x_{2} = y_{2}$, equivalently, $x = y$.
    \[
        d(x, y) = d_{1}(x_{1}, y_{1}) + d_{2}(x_{2}, y_{2}) = d_{1}(y_{1}, x_{1}) + d_{2}(y_{2}, x_{2}) = d(y, x).
    \]

    For every $x, y, z\in X$
    \begin{align*}
        d(x, y) & = d_{1}(x_{1}, y_{1}) + d_{2}(x_{2}, y_{2})                                                    \\
                & \leq (d_{1}(x_{1}, z_{1}) + d_{1}(z_{1}, y_{1})) + (d_{2}(x_{2}, z_{2}) + d_{2}(z_{2}, y_{2})) \\
                & = d(x, z) + d(z, y).\qedhere
    \end{align*}
\end{proof}

\begin{exercise}\label{chapter1:section2:exercise14}
    Show that another metric on $X$ in Exercise~\ref{chapter1:section2:exercise13} is defined by
    \[
        \widetilde{d}(x, y) = \sqrt{{d_{1}(x_{1}, y_{1})}^{2} + {d_{2}(x_{2}, y_{2})}^{2}}.
    \]
\end{exercise}

\begin{proof}
    For every $x, y\in X$
    \[
        \widetilde{d}(x, y) = \sqrt{{d_{1}(x_{1}, y_{1})}^{2} + {d_{2}(x_{2}, y_{2})}^{2}} \geq 0.
    \]

    $d(x, y) = 0$ if and only if $x_{1} = y_{1}$ and $x_{2} = y_{2}$, equivalently, $x = y$.
    \[
        \widetilde{d}(x, y) = \sqrt{{d_{1}(x_{1}, y_{1})}^{2} + {d_{2}(x_{2}, y_{2})}^{2}} = \sqrt{{d_{1}(y_{1}, x_{1})}^{2} + {d_{2}(y_{2}, x_{2})}^{2}} = \widetilde{d}(y, x).
    \]

    For every $x, y, z\in X$
    \begin{align*}
        \widetilde{d}(x, y) & = \sqrt{{d_{1}(x_{1}, y_{1})}^{2} + {d_{2}(x_{2}, y_{2})}^{2}}                                                                   \\
                            & \leq \sqrt{{d_{1}(x_{1}, z_{1}) + d_{1}(z_{1}, y_{1})}^{2} + {d_{2}(x_{2}, z_{2}) + d_{2}(z_{2}, y_{2})}^{2}}                    \\
                            & \leq \sqrt{{d_{1}(x_{1}, z_{1})}^{2} + {d_{2}(x_{2}, z_{2})}^{2}} + \sqrt{{d_{1}(z_{1}, y_{1})}^{2} + {d_{2}(z_{2}, y_{2})}^{2}} \\
                            & = \widetilde{d}(x, z) + \widetilde{d}(z, y). \qedhere
    \end{align*}

    Thus $\widetilde{d}$ defines a metric on $X$.
\end{proof}

\begin{exercise}\label{chapter1:section2:exercise15}
    Show that a third metric on $X$ in Exercise~\ref{chapter1:section2:exercise13} is defined by
    \[
        \widetilde{\widetilde{d}}(x, y) = \max \{d_{1}(x_{1}, y_{1}), d_{2}(x_{2}, y_{2})\}
    \]
\end{exercise}

\begin{proof}
    For every $x, y\in X$
    \[
        \widetilde{\widetilde{d}}(x, y) = \sqrt{{d_{1}(x_{1}, y_{1})}^{2} + {d_{2}(x_{2}, y_{2})}^{2}} \geq 0.
    \]

    $d(x, y) = 0$ if and only if $x_{1} = y_{1}$ and $x_{2} = y_{2}$, equivalently, $x = y$.
    \[
        \widetilde{\widetilde{d}}(x, y) = \max \{d_{1}(x_{1}, y_{1}), d_{2}(x_{2}, y_{2})\} = \max \{d_{1}(y_{1}, x_{1}), d_{2}(y_{2}, x_{2})\} = \widetilde{\widetilde{d}}(y, x).
    \]

    For every $x, y, z\in X$
    \begin{align*}
        \widetilde{\widetilde{d}}(x, z) + \widetilde{\widetilde{d}}(z, y) & = \max \{d_{1}(x_{1}, z_{1}), d_{2}(x_{2}, z_{2})\} + \max \{d_{1}(z_{1}, y_{1}), d_{2}(z_{2}, y_{2})\} \\
                                                                          & \geq d_{1}(x_{1}, z_{1}) + d_{1}(z_{1}, y_{1}) \geq d_{1}(x_{1}, y_{1}),                                \\
        \widetilde{\widetilde{d}}(x, z) + \widetilde{\widetilde{d}}(z, y) & = \max \{d_{1}(x_{1}, z_{1}), d_{2}(x_{2}, z_{2})\} + \max \{d_{1}(z_{1}, y_{1}), d_{2}(z_{2}, y_{2})\} \\
                                                                          & \geq d_{2}(x_{2}, z_{2}) + d_{2}(z_{2}, y_{2}) \geq d_{2}(x_{2}, y_{2}).
    \end{align*}

    So $\widetilde{\widetilde{d}}(x, z) + \widetilde{\widetilde{d}}(z, y) \geq \widetilde{\widetilde{d}}(x, y)$. Thus $\widetilde{\widetilde{d}}$ defines a metric on $X$.
\end{proof}

\section{Open Set, Closed Set, Neighborhood}

\begin{exercise}\label{chapter1:section3:exercise1}
    Justify the terms ``open ball'' and ``closed ball'' by proving that (a) any open ball is an open set, (b) any closed ball is a closed set.
\end{exercise}

\begin{proof}
    \begin{enumerate}[label={(\alph*)}]
        \item Let $B(x_{0}; r)$ be an open ball in the metric space $(X, d)$ and $x$ be an element of $B(x_{0}; r)$.

              Then $d(x, x_{0}) < r$. Let $r_{1}$ be a positive real number that is less than $r - d(x, x_{0})$. Let $y$ be an element of $B(x; r_{1})$.
              \[
                  d(x_{0}, y) \leq d(x_{0}, x) + d(x, y) < d(x_{0}, x) + r_{1} = r.
              \]

              So $y$ is an element of $B(x_{0}; r)$, hence $B(x_{0}; r)$ contains $B(x; r_{1})$. Thus $B(x_{0}; r)$ is an open set.
        \item Let $\widetilde{B}(x_{0}; r)$ be a closed ball in the metric space $(X, d)$.

              Let $A = X - \widetilde{B}(x_{0}; r)$. For each $a\in A$, $d(x_{0}, a) > r$ and there exists a positive real number $r_{a}$ such that $r_{a} < d(x_{0}; a) - r$. For every $y\in B(a; r_{a})$
              \[
                  d(x_{0}, y) \geq \abs{d(x_{0}, a) - d(a, y)} \geq d(x_{0}, a) - d(a, y) > r_{a} + r - d(a, y) > r.
              \]

              So $y\in A$, hence $B(a; r_{a})\subset A$ for every $a\in A$. Therefore
              \[
                  A \subset \bigcup_{a\in A}B(a; r_{a})\qquad A \supset \bigcup_{a\in A}B(a; r_{a})
              \]

              so $A = \bigcup_{a\in A}B(a; r_{a})$. Thus $A$ is an open set in $X$ and $\widetilde{B}(x_{0}; r) = X - A$ is a closed set in $X$.
    \end{enumerate}
\end{proof}

\begin{exercise}\label{chapter1:section3:exercise2}
    What is an open ball $B(x_{0}; 1)$ on $\mathbb{R}$? In $\mathbb{C}$? In $C[a, b]$?
\end{exercise}

\begin{proof}
    The open ball $B(x_{0}; 1)$ on $\mathbb{R}$ is the open interval $\openinterval{x_{0} - 1, x_{0} + 1}$.

    The open ball $B(x_{0}; 1)$ on $\mathbb{C}$ is the open disk with center $x_{0}$ and radius $1$.

    The open ball $B(x_{0}; 1)$ in $C[a, b]$ is the set
    \[
        \left\{ f \in C[a, b] \mid \max_{t\in [a, b]}\abs{f(t) - x_{0}(t)} < 1 \right\}.
    \]
\end{proof}

\begin{exercise}\label{chapter1:section3:exercise3}
    Consider $C[0, 2\pi]$ and determine the smallest $r$ such that $y\in \widetilde{B}(x; r)$, where $x(t) = \sin t$ and $y(t) = \cos t$.
\end{exercise}

\begin{proof}
    The smallest $r$ is
    \[
        \max_{t\in [0,2\pi]}\abs{\sin t - \cos t} = \sqrt{2}
    \]

    where $\abs{\sin t - \cos t} = \sqrt{2}$ if $t = \frac{-\pi}{4}$.
\end{proof}

\begin{exercise}\label{chapter1:section3:exercise4}
    Show that any nonempty set $A\subset (X, d)$ is open if and only if it is a union of open balls.
\end{exercise}

\begin{proof}
    $(\Rightarrow)$ $A$ is a union of open balls.

    Because open balls are open sets, so $A$ is open.

    $(\Leftarrow)$ $A$ is open.

    Because $A$ is open, for each $a\in A$, there exists $r_{a} > 0$ such that $B(a; r_{a})\subset A$. So
    \[
        \bigcup_{a\in A}B(a; r_{a})\subset A.
    \]

    Moreover, for each $a\in A$, $a\in B(a; r_{a})\subset \bigcup_{a\in A}B(a; r_{a})$, so
    \[
        A\subset \bigcup_{a\in A}B(a; r_{a}).
    \]

    Hence $A = \bigcup_{a\in A}B(a; r_{a})$, equivalently, $A$ is a union of open balls.
\end{proof}

\begin{exercise}\label{chapter1:section3:exercise5}
    It is important to realize that certain sets may be open and closed at the same time.\ (a) Show that this is always the case for $X$ and $\varnothing$.\ (b) Show that in a discrete metric space $X$, every subset is open and closed.
\end{exercise}

\begin{proof}
    \begin{enumerate}[label={(\alph*)}]
        \item For every $x\in X$ and $r > 0$, $B(x; r)\subset X$. So $X$ is open and $\varnothing$ is closed.

              $\varnothing$ is the union of the empty list of open sets, so $\varnothing$ is open, and $X$ is closed.
        \item In a discrete metric space $X$, for every $x\in X$, $B(x; 1)$ is open and contains a single element. Therefore, for every subset $A$ of $X$
              \[
                  A = \bigcup_{a\in A}B(a; 1)
              \]

              which means every subset of $X$ is open. Consequently, every subset of $X$ is closed. Hence every subset of $X$ is open and closed.
    \end{enumerate}
\end{proof}

\begin{exercise}\label{chapter1:section3:exercise6}
    If $x_{0}$ is an accumulation point of a set $A \subset (X, d)$, show that any neighborhood of $x_{0}$ contains infinitely many points of $A$.
\end{exercise}

\begin{proof}
    Let $N$ be a neighborbood of $x_{0}$.

    By the definition of neighborhood, there exists a $\varepsilon_{0}$-neighborhood of $x_{0}$ which is contained in $N$. By the definition of accumulation point, there exists $x_{1}$ in the $\varepsilon_{0}$-neighborbood of $x_{0}$ such that $x_{1}\ne x_{0}$.

    Let $\varepsilon_{1}$ be a positive number such that $\varepsilon_{1} < d(x_{0}, x_{1})$. By the definition of accumulation point, there exists $x_{2}$ in the $\varepsilon_{1}$-neighborbood of $x_{0}$ such that $x_{2}\ne x_{0}$. Because $d(x_{0}, x_{2}) < \varepsilon_{1} < d(x_{0}, x_{1})$, $x_{2}\ne x_{1}$.

    Iteratively, we contruct a sequence $x_{1}, x_{2}, \ldots$ consisting of distinct points and they are all in the neighborbood $N$ of $x_{0}$. Because $N$ is arbitrary, we conclude that every neighborhood of $x_{0}$ contains infinitely many points of $A$.
\end{proof}

\begin{exercise}\label{chapter1:section3:exercise7}
    Describe the closure of each of the following subsets.\ (a) The integers on $\mathbb{R}$, (b) the rational numbers on $\mathbb{R}$, (c) the complex numbers with rational real and imaginary parts in $\mathbb{C}$, (d) the disk $\{ z \mid \abs{z} < 1 \}\subset \mathbb{C}$.
\end{exercise}

\begin{proof}
    \begin{enumerate}[label={(\alph*)}]
        \item The closure of $\mathbb{Z}$ in $\mathbb{R}$ is $\mathbb{Z}$.
        \item The closure of $\mathbb{Q}$ in $\mathbb{R}$ is $\mathbb{R}$.
        \item The closure of $\{ a + b\iota \mid a, b\in\mathbb{Q} \}$ is $\mathbb{C}$.
        \item The closure of the open disk $\{ z \mid \abs{z} < 1 \}\subset \mathbb{C}$ is the closed disk $\{ z \mid \abs{z}\leq 1 \}$.
    \end{enumerate}
\end{proof}

\begin{exercise}\label{chapter1:section3:exercise8}
    Show that the closure $\overline{B(x_{0}; r)}$ of an open ball $B(x_{0}; r)$ in a metric space can differ from the closed ball $\widetilde{B}(x_{0}; r)$.
\end{exercise}

\begin{proof}
    In a discrete metric space $X$ with more than one element, $\overline{B(x_{0}; 1)}$ consists of a single point, which is $x_{0}$. However, $\widetilde{B}(x_{0}; 1) = X \ne \{ x_{0} \} = \overline{B(x_{0}; 1)}$.
\end{proof}

\begin{exercise}\label{chapter1:section3:exercise9}
    Show that $A\subset \overline{A}$, $\overline{\overline{A}} = \overline{A}$, $\overline{A\cup B} = \overline{A}\cup\overline{B}$, $\overline{A\cap B} \subset \overline{A}\cap\overline{B}$.
\end{exercise}

\begin{proof}
    \begin{itemize}
        \item Because the closure of $A$ consists of elements of $A$ and accumulation points of $A$, it follows that $A\subset \overline{A}$.
        \item I will prove that $S\subset X$ is closed if and only if $S$ contains all of its accumulation points (actually, this part is no need here).

              Suppose $S$ is closed. Let $x$ be an accumulation point of $S$. Assume $x\in X\setminus S$. Because $S$ is closed, it follows that $X\setminus S$ is open. Because $X\setminus S$ is open, there exists $r > 0$ such that $B(x; r)\subset X\setminus S$. On the other hand, $x$ is an accumulation point of $S$, so $B(x; r)\cap S \ne \varnothing$, which contradicts $B(x; r)\subset X\setminus S$. Hence the assumption is false, which means $x\in S$. Thus $S$ contains all of its accumulation points.

              Suppose $S$ contains all of its accumulation points. Assume $S$ is not closed, so $X\setminus S$ is not open. So there exists $y\in X\setminus S$ such that for all $r > 0$, $B(y; r)\cap S\ne \varnothing$. This means $y$ is an accumulation point of $S$. But $y\in X\setminus S$ and $S$ contains all of its accumulation points, this is a contradiction. Thus $S$ is closed.

              And I prove that $S$ is closed if and only if $S = \overline{S}$.

              If $S = \overline{S}$ then $S$ is closed because $\overline{S}$ is closed. If $S$ is closed then $S$ contained all of its accumulation points, therefore $S = \overline{S}$.

              And here is the proof that $\overline{\overline{A}} = \overline{A}$.

              Let $x$ be an accumulation point of $\overline{A}$. Every neighborhood $N$ of $x$ contains a point $y\in\overline{A}$ other than $x$. If $y$ is an accumulation point of $A$ then $N$ (which is also a neighborhood of $y$) contains a point in $A$ other than $y$. If $y$ is not an accumulation point of $A$ then $y$ is in $A$ (because $y\in\overline{A}$). Therefore every neighborhood of $x$ contains a point in $A$ other than $x$. So $x$ is an accumulation point of $A$. Hence every accumulation point of $\overline{A}$ is an accumulation point of $A$. Thus $\overline{\overline{A}} = \overline{A}$.

              And I will prove that $\overline{S}$ is the smallest closed set containing $S$.

              $\overline{S}$ is closed. Let $R$ be a closed set containing $S$, then $R$ contains all of its accumulation points. Therefore $R$ contains all accumulation points of $S$. Thus $R$ contains $\overline{S}$, so $\overline{S}$ is the smallest closed set containing $S$.
        \item Accumulation points of $A$ are also accumulation points of $A\cup B$. Accumulation points of $B$ are also accumulation points of $A\cup B$. Therefore $\overline{A}\cup \overline{B}\subset \overline{A\cup B}$.

              $\overline{A}\cup \overline{B}$ is closed (finite union of closed sets is closed) and $\overline{A}\cup \overline{B}$ contains $A\cup B$. On the other hand, $\overline{A\cup B}$ is the smallest closed set containing $A\cup B$ so $\overline{A\cup B}\subset \overline{A}\cup \overline{B}$.

              Hence $\overline{A\cup B} = \overline{A}\cup \overline{B}$.
        \item Suppose $x\in \overline{A\cap B}$.

              If $x$ is an accumulation point of $A\cap B$, then every neighborhood of $x$ contains a point other than $x$ in $A\cap B$. So every neighborhood of $x$ contains a point other than $x$ in $A$ and $B$. Therefore $x$ is an accumulation point of $A$ and $B$, so $x\in \overline{A}\cap\overline{B}$.

              If $x$ is not an accumulation point of $A\cap B$ then $x\cap A\cap B$. So $x\in \overline{A}\cap \overline{B}$.

              Thus $\overline{A\cap B}\subset \overline{A}\cap\overline{B}$.
    \end{itemize}
\end{proof}

\begin{exercise}\label{chapter1:section3:exercise10}
    A point $x$ not belonging to a closed set $M\subset (X, d)$ always has a nonzero distance from $M$. To prove this, show that $x\in \overline{A}$ if and only if $D(x, A) = 0$; here $A$ is any nonempty subset of $X$.
\end{exercise}

\begin{proof}
    $(\Rightarrow)$ $x\in \overline{A}$.

    If $x\in A$, then $D(x, A) = 0$.

    If $x$ is an accumulation point of $A$ then for every $\varepsilon > 0$, $B(x; \varepsilon)$ contains a point in $A$ other than $x$. So $D(x, A) < \varepsilon$ for every $\varepsilon > 0$. Hence $D(x, A) = 0$.

    $(\Leftarrow)$ $D(x, A) = 0$.

    If $x\in A$ then $x\in \overline{A}$.

    If $x\notin A$. Let $N$ be a neighborbood of $x$, then there exists an open abll $B(x; r)$ contained in $N$. Because $D(x, A) = 0$ and $x\notin A$, there exists $a\in A$ such that $0 < d(x, a) < r$. So $a\in B(x; r) \subset N$. Hence $x$ is an accumulation point of $A$. Therefore $x\in \overline{A}$.

    Thus $x\in \overline{A}$ if and only if $D(x, A) = 0$.
\end{proof}

\begin{exercise}[Boundary]\label{chapter1:section3:exercise11}
    A boundary point $x$ of a set $A\subset (X, d)$ is a point of $X$ (which may or may not belong to $A$) such that every neighborhood of $x$ contains points of $A$ as well as points not belonging to $A$; and the boundary of $A$ is the set of all boundary points of $A$. Describe the boundary of (a) the intervals $\openinterval{-1, 1}$, $\halfopenleft{-1, 1}$, $\halfopenright{-1, 1}$ on $\mathbb{R}$; (b) the set of all rational numbers on $\mathbb{R}$; (c) the disks $\{ z\mid \abs{z} < 1 \}\subset \mathbb{C}$ and $\{ z\mid \abs{z} \leq 1 \}\subset \mathbb{C}$.
\end{exercise}

\begin{proof}
    \begin{enumerate}[label={(\alph*)}]
        \item The boundary of $\openinterval{-1, 1}$ is $\{ -1, 1 \}$.

              The boundary of $\halfopenleft{-1, 1}$ is $\{ -1, 1 \}$.

              The boundary of $\halfopenright{-1, 1}$ is $\{ -1, 1 \}$.
        \item The boundary of $\mathbb{Q}$ in $\mathbb{R}$ is $\mathbb{R}$ because every open ball in $\mathbb{R}$ contains both rational numbers and irrational numbers.
        \item The boundary of the disks $\{ z\mid \abs{z} < 1 \}\subset \mathbb{C}$ and $\{ z\mid \abs{z} \leq 1 \}\subset \mathbb{C}$ is $\{ z\mid \abs{z} = 1 \}\subset \mathbb{C}$.
    \end{enumerate}
\end{proof}

\begin{exercise}\label{chapter1:section3:exercise12}
    Show that $B[a, b]$, $a < b$, is not separable.
\end{exercise}

\begin{proof}
    Let $f$ be a bounded function on $B[a, b]$ where $f(x)$ is either $0$ or $1$ for every $x\in [a, b]$. The set of such functions $f$ is uncountable and the distance between different functions $f$ is $1$. Therefore $B(f; 1/3)$ are pairwise disjoint.

    Assume $M$ is dense in $B[a, b]$, then $B(f; 1/3)$ contains a function other than $f$ in $M$. Since the open balls $B(f; 1/3)$ are pairwise disjoint, it follows that $M$ is uncountable. Hence every dense set in $B[a, b]$ is uncountable. Thus $B[a, b]$ is not separable.
\end{proof}

\begin{exercise}\label{chapter1:section3:exercise13}
    Show that a metric space $X$ is separable if and only if $X$ has a countable subset $Y$ with the following property. For every $\varepsilon > 0$ and every $x \in X$ there is a $y\in Y$ such that $d(x, y) < \varepsilon$.
\end{exercise}

\begin{proof}
    $(\Rightarrow)$ $X$ is separable.

    Then there exists a countable subset $Y$ of $X$ such that $\overline{Y} = X$. For every $\varepsilon > 0$ and $x\in X$,
    \begin{itemize}
        \item if $x\in Y$ then let $y = x$, we have $d(x, y) = 0 < \varepsilon$,
        \item if $x\notin Y$ then $x$ is an accumulation point of $Y$, so $B(x; \varepsilon)$ contains an element $y$ of $Y$ that is other than $x$, therefore $d(x, y) < \varepsilon$.
    \end{itemize}

    $(\Leftarrow)$ $X$ has a countable subset $Y$, where for every $\varepsilon > 0$ and every $x \in X$ there is a $y\in Y$ such that $d(x, y) < \varepsilon$.

    Let $x$ be an element in $X$. Let $N$ be a neighborhood of $x$, so there exists an open ball $B(x;\varepsilon)$ contained in $N$. Moreover, $B(x; \varepsilon)$ contains an element $y$ of $Y$ and $y\ne x$. Therefore $N$ contains an element of $Y$ which is other than $x$. So $x$ is an accumulation point of $Y$. Hence $\overline{Y} = X$.

    Furthermore, $Y$ is a countable subset of $X$, so $X$ is separable.
\end{proof}

\begin{exercise}[Continuous mapping]\label{chapter1:section3:exercise14}
    Show that a mapping $T: X\to Y$ is continuous if and only if the inverse image of any closed set $M\subset Y$ is a closed set in $X$.
\end{exercise}

\begin{quote}
    The inverse image behaves nicely.
    \begin{align*}
        f^{-1}(\bigcup_{\alpha\in I}A_{\alpha}) & = \bigcup_{\alpha\in I}f^{-1}(A_{\alpha}) \\
        f^{-1}(\bigcap_{\alpha\in I}A_{\alpha}) & = \bigcap_{\alpha\in I}f^{-1}(A_{\alpha}) \\
        f^{-1}(A\setminus B)                    & = f^{-1}(A)\setminus f^{-1}(B)
    \end{align*}
\end{quote}

\begin{proof}
    $(\Rightarrow)$ $T$ is continuous.

    Let $M$ be a closed set in $Y$. $X\setminus T^{-1}(M) = T^{-1}(Y) \setminus T^{-1}(M) = T^{-1}(Y\setminus M)$. Because $Y\setminus M$ is open, then so is $T^{-1}(Y\setminus M)$. Therefore $X\setminus T^{-1}(M)$ is open. Thus $T^{-1}(M)$ is closed.

    $(\Leftarrow)$ The inverse image of any closed set $M\subset Y$ is a closed set in $X$.

    Let $A$ be an open set contained in $Y$, then
    \[
        T^{-1}(A) = T^{-1}(Y\setminus (Y\setminus A)) = T^{-1}(Y)\setminus T^{-1}(Y\setminus A) = X\setminus T^{-1}(Y\setminus A).
    \]

    Because $Y\setminus A$ is closed, so $T^{-1}(Y\setminus A)$ is also closed. Therefore $X\setminus T^{-1}(Y\setminus A)$ is open. So $T^{-1}(A)$ is open for every open set $A$ contained in $Y$. Thus $T$ is continuous.
\end{proof}

\begin{exercise}\label{chapter1:section3:exercise15}
    Show that the image of an open set under a continuous mapping need not be open.
\end{exercise}

\begin{proof}
    The function $f: \mathbb{R}\to \mathbb{R}$ defined by $f(x) = x^{2}$ is continuous. The image of the open set $\openinterval{-1, 1}$ is $\halfopenright{0, 1}$, which is not an open set.
\end{proof}

\section{Convergence, Cauchy Sequence, Completeness}

% chapter1:section4:exercise1
\begin{exercise}[Subsequence]\label{chapter1:section4:exercise1}
    If a sequence $(x_{n})$ in a metric space $X$ is convergent and has limit $x$, show that every subsequence $(x_{n_{k}})$ of $(x_{n})$ is convergent and has the same limit $x$.
\end{exercise}

\begin{proof}
    $x_{n}\to x$ so for every $\varepsilon > 0$, there exists $N$ such that $d(x_{n}, x) < \varepsilon$ if $n > N$.

    Therefore for every $\varepsilon > 0$, there exists $N$ such that $d(x_{n_{k}}, x) < \varepsilon$ if $n_{k} > N$. Hence $x_{n_{k}}\to x$.
\end{proof}

% chapter1:section4:exercise2
\begin{exercise}\label{chapter1:section4:exercise2}
    If $(x_{n})$ is Cauchy and has a convergent subsequence, say $x_{n_{k}}\to x$, show that $(x_{n})$ is convergent with the limit $x$.
\end{exercise}

\begin{proof}
    Let $\varepsilon > 0$.

    There exists $N$ such that $d(x_{m}, x_{n}) < \frac{\varepsilon}{2}$ if $m, n > N$.

    There exists $N'$ such that $d(x_{n_{k}}, x) < \frac{\varepsilon}{2}$ if $n_{k} > N'$.

    Let $N'' = \max\{ N, N' \}$. If $n > N''$ then for $n_{k} > N''$, we have
    \[
        d(x_{n}, x)\leq d(x_{n}, x_{n_{k}}) + d(x_{n_{k}}, x) < \frac{\varepsilon}{2} + \frac{\varepsilon}{2} = \varepsilon.
    \]

    So for every $\varepsilon > 0$, there exists $N''$ such that $d(x_{n}, x) < \varepsilon$ if $n > N''$. Thus $x_{n}\to x$.
\end{proof}

% chapter1:section4:exercise3
\begin{exercise}\label{chapter1:section4:exercise3}
    Show that $x_{n}\to x$ if and only if for every neighborhood $V$ of $x$ there is an integer $n_{0}$ such that $x_{n}\in V$ for all $n > n_{0}$.
\end{exercise}

\begin{proof}
    $(\Rightarrow)$ $x_{n}\to x$.

    Let $V$ be a neighborhood of $x$. Then there exists an open ball $\openinterval{x - \varepsilon, x + \varepsilon}$ contained in $V$. Moreover, for every $\varepsilon > 0$, there exists a positive integer $n_{0}$ such that $d(x_{n}, x) < \varepsilon$ if $n > n_{0}$. Thus there exists a positive integer $n_{0}$ such that $x_{n}\in \openinterval{x - \varepsilon, x + \varepsilon}\subset V$ if $n > n_{0}$.

    $(\Leftarrow)$ For every neighborhood $V$ of $x$ there is an integer $n_{0}$ such that $x_{n}\in V$ for all $n > n_{0}$.

    For every $\varepsilon > 0$, $B(x;\varepsilon) = \openinterval{x - \varepsilon, x + \varepsilon}$ is an open neighborbood of $x$, so there exists $n_{0}$ such that $d(x_{n}, x) < \varepsilon$ if $n > n_{0}$. Thus $x_{n}\to x$.
\end{proof}

% chapter1:section4:exercise4
\begin{exercise}[Boundedness]\label{chapter1:section4:exercise4}
    Show that a Cauchy sequence is bounded.
\end{exercise}

\begin{proof}
    Let $(x_{n})$ be a Cauchy sequence.

    There exists a positive integer $N$ such that $d(x_{m}, x_{n}) < 1$ if $m, n > N$. So

    If $m, n > N$ then $d(x_{n}, x_{m}) < 1$.

    If $m, n\leq N$ then $d(x_{n}, x_{m}) \leq \max\limits_{i,j\leq N}\{ d(x_{i}, x_{j}) \}$.

    If $m > N$ and $n\leq N$ then
    \[
        d(x_{n}, x_{m})\leq d(x_{n}, x_{N+1}) + d(x_{N+1}, x_{m}) < 1 + \max\{ d(x_{1}, x_{N+1}), \ldots, d(x_{n}, x_{N+1}) \}.
    \]

    Let $B$ be the maximum of the list
    \[
        1; \max\limits_{ i, j\leq N }\{ d(x_{i}, x_{j})\}; 1 + \max\{ d(x_{1}, x_{N+1}), \ldots, d(x_{n}, x_{N+1}) \}
    \]

    then $d(x_{n}, x_{m}) \leq B$ for every $m, n$. Hence $(x_{n})$ is bounded.
\end{proof}

% chapter1:section4:exercise5
\begin{exercise}\label{chapter1:section4:exercise5}
    Is boundedness of a sequence in a metric space sufficient for the sequence to be Cauchy? Convergent?
\end{exercise}

\begin{proof}
    No. Boundedness of a sequence in a metric space is not sufficient for a sequence to be Cauchy or convergent.

    For example, $x_{n} = {(-1)}^{n}$ is bounded but it is neither Cauchy or convergent.
\end{proof}

% chapter1:section4:exercise6
\begin{exercise}\label{chapter1:section4:exercise6}
    If $(x_{n})$ and $(y_{n})$ are Cauchy sequences in a metric space $(X, d)$, show that $(a_{n})$, where $a_{n} = d(x_{n}, y_{n})$, converges. Give illustrative examples.
\end{exercise}

\begin{proof}
    Let $\varepsilon > 0$.

    There exists $N$ such that $d(x_{m}, x_{n}) < \frac{\varepsilon}{2}$ if $m, n > N$.

    There exists $N'$ such that $d(y_{m}, y_{n}) < \frac{\varepsilon}{2}$ if $m, n > N'$.

    Let $N'' = \max\{ N, N' \}$. For every $m, n > N''$
    \begin{align*}
        \abs{a_{m} - a_{n}} & = \abs{d(x_{m}, y_{m}) - d(x_{n}, y_{n})}                      \\
                            & \leq d(x_{m}, x_{n}) + d(y_{m}, y_{n})                         \\
                            & < \frac{\varepsilon}{2} + \frac{\varepsilon}{2} = \varepsilon.
    \end{align*}

    So ${(a_{n})}$ is a Cauchy sequence in $\mathbb{R}$. Hence ${(a_{n})}$ is convergent, because $\mathbb{R}$ is a complete metric space.
\end{proof}

% chapter1:section4:exercise7
\begin{exercise}\label{chapter1:section4:exercise7}
    Give an indirect proof of Lemma 1.4-2(b).

    If $x_{n}\to x$ and $y_{n}\to y$ in $X$, then $d(x_{n}, y_{n}) \to d(x, y)$.
\end{exercise}

\begin{proof}
    Let $\varepsilon > 0$.

    There exists $N$ such that $d(x_{n}, x) < \frac{\varepsilon}{2}$ if $n > N$.

    There exists $N'$ such that $d(y_{n}, y) < \frac{\varepsilon}{2}$ if $n > N'$.

    Let $N'' = \max\{ N, N' \}$ then if $n > N''$
    \[
        \abs{d(x_{n}, y_{n}) - d(x, y)} \leq d(x_{n}, x) + d(y_{n}, y) < \frac{\varepsilon}{2} + \frac{\varepsilon}{2} = \varepsilon.
    \]

    So $d(x_{n}, y_{n})\to d(x, y)$.
\end{proof}

% chapter1:section4:exercise8
\begin{exercise}\label{chapter1:section4:exercise8}
    If $d_{1}$ and $d_{2}$ are metrics on the same set $X$ and there are positive numbers $a$ and $b$ such that for all $x, y\in X$,
    \[
        ad_{1}(x, y) \leq d_{2}(x, y) \leq bd_{1}(x, y),
    \]

    show that the Cauchy sequences in $(X, d_{1})$ and $(X, d_{2})$ are the same.
\end{exercise}

\begin{proof}
    Let $(x_{n})$ be a Cauchy sequence in $(X, d_{1})$.

    So for every $\varepsilon > 0$, there exists $N$ such that if $m, n > N$, then $d_{1}(x_{m}, x_{n}) < \varepsilon/b$.

    Hence  for every $\varepsilon > 0$, there exists $N$ such that if $m, n > N$, then $d_{2}(x_{m}, x_{n}) \leq bd_{1}(x_{m}, x_{n}) < b\cdot \varepsilon/b = \varepsilon$. Thus $(x_{n})$ is a Cauchy sequence in $(X, d_{2})$.

    \bigskip

    Let $(x_{n})$ be a Cauchy sequence in $(X, d_{2})$.

    So for every $\varepsilon > 0$, there exists $N$ such that if $m, n > N$, then $d_{2}(x_{m}, x_{n}) < a\varepsilon$.

    Hence for every $\varepsilon > 0$, there exists $N$ such that if $m, n > N$, then $d_{1}(x_{m}, x_{n}) \leq d_{2}(x_{m}, x_{n})/a < a\varepsilon/a = \varepsilon$. Thus $(x_{n})$ is a Cauchy sequence in $(X, d_{1})$.

    Therefore the Cauchy sequences in $(X, d_{1})$ and $(X, d_{2})$ are the same.
\end{proof}

% chapter1:section4:exercise9
\begin{exercise}\label{chapter1:section4:exercise9}
    Using Exercise~\ref{chapter1:section4:exercise8}, show that the metric spaces in Exercise~\ref{chapter1:section2:exercise13},~\ref{chapter1:section2:exercise14},~\ref{chapter1:section2:exercise15} have the same Cauchy sequences.
\end{exercise}

\begin{proof}
    For every $x, y$ in $X$
    \begin{align*}
        d(x, y) & = d_{1}(x_{1}, y_{1}) + d_{2}(x_{2}, y_{2}) \leq \sqrt{2}\sqrt{{(d_{1}(x_{1}, y_{1}))}^{2} + {(d_{2}(x_{2}, y_{2}))}^{2}} = \sqrt{2}\widetilde{d}(x, y), \\
        d(x, y) & = d_{1}(x_{1}, y_{1}) + d_{2}(x_{2}, y_{2}) \geq 1\cdot\sqrt{{(d_{1}(x_{1}, y_{1}))}^{2} + {(d_{2}(x_{2}, y_{2}))}^{2}} = 1\cdot \widetilde{d}(x, y),    \\
        d(x, y) & = d_{1}(x_{1}, y_{1}) + d_{2}(x_{2}, y_{2}) \leq 2\cdot\max\{ d_{1}(x_{1}, y_{1}), d_{2}(x_{2}, y_{2}) \} = 2\tilde{\tilde{d}}(x, y),                    \\
        d(x, y) & = d_{1}(x_{1}, y_{1}) + d_{2}(x_{2}, y_{2}) \geq 1\cdot \max\{ d_{1}(x_{1}, y_{1}), d_{2}(x_{2}, y_{2}) \} = \tilde{\tilde{d}}(x, y).
    \end{align*}

    Thus, by Exercise~\ref{chapter1:section4:exercise8}, $(X, d)$, $(X, \tilde{d})$, $(X, \tilde{\tilde{d}})$ have the same Cauchy sequences.
\end{proof}

% chapter1:section4:exercise10
\begin{exercise}\label{chapter1:section4:exercise10}
    Using the completeness of $\mathbb{R}$, prove completeness of $\mathbb{C}$.
\end{exercise}

\begin{proof}
    Let $(z_{n})$ be a Cauchy sequence in $\mathbb{C}$ and $z_{n} = x_{n} + \iota y_{n}$ where $x_{n}, y_{n}\in\mathbb{R}$.

    For every $\varepsilon > 0$, there exists $N$ such that if $m, n > N$, $\abs{z_{m} - z_{n}} < \varepsilon$, so
    \[
        \sqrt{\abs{x_{m} - x_{n}}^{2} + \abs{y_{m} - y_{n}}^{2}} < \varepsilon
    \]

    hence $\abs{x_{m} - x_{n}} < \varepsilon$ and $\abs{y_{m} - y_{n}} < \varepsilon$. Due to the completeness of $\mathbb{R}$, ${(x_{n})}$ and ${(y_{n})}$ are convergent.

    Let $x_{n} \to x$ and $y_{n} \to y$. For every $\varepsilon > 0$,
    \begin{itemize}
        \item there exists $N_{x}$ such that if $n > N_{x}$, $\abs{x_{n} - x} < \frac{\varepsilon^{2}}{2}$
        \item there exists $N_{y}$ such that if $n > N_{y}$, $\abs{y_{n} - y} < \frac{\varepsilon^{2}}{2}$
    \end{itemize}

    so if $n > N = \max\{ N_{x}, N_{y} \}$, then
    \[
        \abs{z_{n} - (x + \iota y)} = \sqrt{\abs{x_{n} - x}^{2} + \abs{y_{n} - y}^{2}} < \varepsilon.
    \]

    Hence $z_{n} \to x + \iota y$. Thus $\mathbb{C}$ is a complete metric space.
\end{proof}

\section{Examples. Completeness Proofs}

% chapter1:section5:exercise1
\begin{exercise}\label{chapter1:section5:exercise1}
    Let $a, b\in \mathbb{R}$ and $a < b$. Show that the open interval $\openinterval{a, b}$ is an incomplete subspace of $\mathbb{R}$, whereas the closed interval $\closedinterval{a, b}$ is complete.
\end{exercise}

\begin{proof}
    Let $x = (a + \frac{b-a}{2}, a + \frac{b-a}{2^{2}}, a + \frac{b-a}{2^{3}} \ldots)$. $x$ is a Cauchy sequence in $\openinterval{a, b}$, because for every $\varepsilon > 0$, if $n > N = 1 + \left\lfloor\frac{b - a}{\varepsilon}\right\rfloor$, and $p > 0$, then
    \[
        \abs{a + \frac{b-a}{2^{n}} - a - \frac{b-a}{2^{n+p}}} = \abs{\frac{b - a}{2^{n}} - \frac{b-a}{2^{n+p}}} \leq \frac{b-a}{2^{n+1}} \leq \frac{b-a}{n+1} < \varepsilon.
    \]

    But $x$ converges to $a$, which is not in $\openinterval{a, b}$, so $\openinterval{a, b}$ is an incomplete subspace of $\mathbb{R}$.

    The complement of $\closedinterval{a, b}$ in $\mathbb{R}$ is $\openinterval{-\infty, a}\cup \openinterval{b, +\infty}$, which is an open set. So $\closedinterval{a, b}$ is a closed set, therefore it contains all of its accumulation points. Let $x$ be a Cauchy sequence in $\closedinterval{a, b}$ and $y$ be its limit ($y$ exists because $\mathbb{R}$ is complete). Because $\closedinterval{a, b}$ contains all of its accumulation points, it follows that $y\in \closedinterval{a, b}$. Thus $\closedinterval{a, b}$ is complete.
\end{proof}

% chapter1:section5:exercise2
\begin{exercise}\label{chapter1:section5:exercise2}
    Let $X$ be the space of all ordered $n$-tuples $x = (\xi_{1}, \ldots, \xi_{n})$ of real numbers and
    \[
        d(x, y) = \max\limits_{j}\abs{\xi_{j} - \eta_{j}}
    \]

    where $y = {(\eta_{j})}$. Show that $(X, d)$ is complete.
\end{exercise}

\begin{proof}
    Let $(x_{n})$ be a Cauchy sequence in $X$. $x_{m} = (\xi_{1}^{(m)}, \ldots, \xi_{n}^{(m)})$. Let $\varepsilon > 0$, there exists a positive integer $N$ such that if $m, p > N$, then
    \[
        d(x_{m}, x_{p}) = \max_{j}\abs{\xi_{j}^{(n)} - \xi_{j}^{(p)}} < \varepsilon.
    \]

    So for every $j$, if $m, p > N$, then $\abs{\xi_{j}^{(n)} - \xi_{j}^{(p)}} < \varepsilon$. Therefore $(\xi_{j}^{(1)}, \xi_{j}^{(2)}, \ldots)$ is a Cauchy sequence. Moreover, $\mathbb{R}$ is complete, so $(\xi_{j}^{(1)}, \xi_{j}^{(2)}, \ldots)$ is convergent. Let $\xi_{j} = \lim\limits_{m\to\infty}\xi_{j}^{(m)}$ and $x = (\xi_{1}, \ldots, \xi_{n})$ so $x\in X$.

    $d(x_{m}, x_{p}) < \varepsilon$ for every $m, p > N$. Let $p\to\infty$, we obtain that $d(x_{m}, x)\leq \varepsilon$. Thus $x_{m}\to x$, and because $x_{m}$ is an arbitrary Cauchy sequence in $X$, we conclude that $(X, d)$ is complete.
\end{proof}

% chapter1:section5:exercise3
\begin{exercise}\label{chapter1:section5:exercise3}
    Let $M\subset \ell^{\infty}$ be the subspace consisting of all sequences $x = (\xi_{j})$ with at most finitely many nonzero terms. Find a Cauchy sequence in $M$ which does not converge in $M$, so that $M$ is not complete.
\end{exercise}

\begin{proof}
    Let $(x_{m})$ be the sequence that
    \[
        x_{m} = \left(1, \frac{1}{2}, \ldots, \frac{1}{m}, 0, \ldots\right).
    \]

    For every $\varepsilon > 0$, if $m, n > N = 1 + \left\lfloor\frac{1}{\varepsilon}\right\rfloor$ then
    \[
        d(x_{m}, x_{n}) = \frac{1}{\min\{ m, n \}} < \varepsilon.
    \]

    So $(x_{m})$ is a Cauchy sequence in $M$. The limit point of $(x_{m})$ is the sequence
    \[
        x = \left(1, \frac{1}{2}, \frac{1}{3}, \ldots\right)
    \]

    which is not in $M$. Hence $M$ is not complete.
\end{proof}

% chapter1:section5:exercise4
\begin{exercise}\label{chapter1:section5:exercise4}
    Show that $M$ in Exercise~\ref{chapter1:section5:exercise3} is not complete by applying Theorem 1.4-7.
\end{exercise}

\begin{proof}
    $x = (1, 1/2, 1/3, \ldots)$ is not an element of $M$. Let $\varepsilon > 0$ and I choose a positive integer $n$ such thtat $\frac{1}{n} < \varepsilon$. Let $y = (1, 1/2, 1/3, \ldots, \varepsilon, 0, \ldots)$ where $y^{(n)} = \varepsilon$ and $\frac{1}{n} < \varepsilon$, $y^{(m)} = \frac{1}{m}$ if $m < n$, $y^{(m)} = 0$ if $m > n$. This means $y\notin \ell^{\infty}\setminus M$. Then
    \[
        d(x, y) = \varepsilon - \frac{1}{n} < \varepsilon.
    \]

    Therefore $B(x; \varepsilon)$ is not contained in $\ell^{\infty}\setminus M$ for every $\varepsilon > 0$, so $\ell^{\infty}\setminus M$ is not open in $\ell^{\infty}$, and $M$ is not closed in $\ell^{\infty}$.

    Moreover, because $\ell^{\infty}$ is complete, so $M$ is complete if and only if $M$ is closed. Hence $M$ is not complete, because $M$ is not closed.
\end{proof}

% chapter1:section5:exercise5
\begin{exercise}\label{chapter1:section5:exercise5}
    Show that the set $X$ of all integers with metric $d$ defined by $d(m, n) = \abs{m - n}$ is a complete metric space.
\end{exercise}

\begin{proof}
    The complement of $\mathbb{Z}$ in $\mathbb{R}$ is
    \[
        \bigcup_{n\in\mathbb{Z}}\openinterval{n, n+1}
    \]

    so $\mathbb{R}\setminus\mathbb{Z}$ is open. Therefore $\mathbb{Z}$ is closed. Because $\mathbb{R}$ is complete and $\mathbb{Z}$ is closed in $\mathbb{R}$, it follows that $\mathbb{Z}$ is complete.
\end{proof}

% chapter1:section5:exercise6
\begin{exercise}\label{chapter1:section5:exercise6}
    Show that the set of all real numbers constitutes an incomplete metric space if we choose
    \[
        d(x, y) = \abs{\arctan x - \arctan y}
    \]
\end{exercise}

\begin{proof}
    Let $x = (1, 2, 3, \ldots)$.

    $\lim\limits_{n\to\infty}\arctan n = \frac{\pi}{2}$. Let $\varepsilon > 0$, then there exists $N$ such that if $n > N$, then
    \[
        \abs{\arctan n - \frac{\pi}{2}} < \frac{\varepsilon}{2}.
    \]

    So for every $m, n > N$,
    \[
        d(m, n) = \abs{\arctan m - \arctan n} \leq \abs{\arctan m - \frac{\pi}{2}} + \abs{\arctan n - \frac{\pi}{2}} < \frac{\varepsilon}{2} + \frac{\varepsilon}{2} = \varepsilon.
    \]

    Hence $x$ is a Cauchy sequence. Assume $x$ is convergent and its limit is $a$, then for every $\varepsilon > 0$, there exists $N'$ such that if $n > N'$, then $\abs{\arctan n - \arctan a} < \varepsilon$, so $\lim\limits_{n\to\infty}\arctan n = \arctan a$. On the other hand $\lim\limits_{n\to\infty}\arctan n = \frac{\pi}{2}$ so $\arctan a = \frac{\pi}{2}$. However, for every real number $a$, $\arctan a < \frac{\pi}{2}$, so the assumption is false. Thus $x$ is not convergent, and $\mathbb{R}$ is an incomplete metric space if we choose $d(x, y) = \abs{\arctan x - \arctan y}$.
\end{proof}


% chapter1:section5:exercise7
\begin{exercise}\label{chapter1:section5:exercise7}
    Let $X$ be the set of all positive integers and $d(m, n) = \abs{m^{-1} - n^{-1}}$. Show that $(X, d)$ is not complete.
\end{exercise}

\begin{proof}
    Let $x = (1, 2, 3, \ldots)$.

    For every $\varepsilon > 0$, let $N = 1 + \left\lfloor\frac{1}{\varepsilon}\right\rfloor$, then for every $m, n > N$,
    \[
        d(m, n) = \abs{\frac{1}{m} - \frac{1}{n}} < \frac{1}{\min\{m,n\}} < \varepsilon.
    \]

    So $x$ is a Cauchy sequence. Assume $p\in\mathbb{Z}^{+}$ is a limit of $x$, then for every $\varepsilon > 0$, there exists $N$ such that if $n > N$
    \[
        \abs{\frac{1}{n} - \frac{1}{p}} < \varepsilon.
    \]

    Moreover, $\lim\limits_{n\to\infty}\frac{1}{n} = 0$ so $\frac{1}{p} = 0$. This is a contradiction, since $p\in\mathbb{Z}^{+}$. Hence $x$ is not convergent. Thus $(X, d)$ is not complete.
\end{proof}


% chapter1:section5:exercise8
\begin{exercise}\label{chapter1:section5:exercise8}
    Show that the subspace $Y\subset C[a, b]$ consisting of all $x\in C[a, b]$ such that $x(a) = x(b)$ is complete.
\end{exercise}

\begin{proof}
    Let ${(x_{n})}$ be a Cauchy sequence in $Y$. Because $C[a, b]$ is complete, there exists a function $x\in C[a, b]$ such that $x_{n}\to x$.

    Therefore $x_{n}(a)\to x(a)$ and $x_{n}(b)\to x(b)$. Because $x_{n}(a) = x_{n}(b)$ for every positive integer $n$, we conclude that $x(a) = x(b)$, so $x\in Y$. Hence $Y$ is complete.
\end{proof}

% chapter1:section5:exercise9
\begin{exercise}\label{chapter1:section5:exercise9}
    In 1.5-5 we referred to the following theorem of calculus. If a sequence $(x_{m})$ of continuous functions on $\closedinterval{a, b}$ converges on $\closedinterval{a, b}$ and the convergence is uniform on $\closedinterval{a, b}$, then the limit function $x$ is continuous on $\closedinterval{a, b}$. Prove this theorem.
\end{exercise}

\begin{proof}
    Assume $x_{n}$ converges uniformly to $x$, then for every $\varepsilon > 0$, there exists $N$ such that for all $t\in \closedinterval{a, b}$ and $n > N$, $\abs{x_{n}(t) - x(t)} < \frac{\varepsilon}{3}$.

    Let $t_{0}\in \closedinterval{a, b}$. There exists $\delta$ such that if $0 < \abs{t - t_{0}} < \delta$, then $\abs{x_{n}(t) - x_{n}(t_{0})} < \frac{\varepsilon}{3}$.

    So if $0 < \abs{t - t_{0}} < \delta$
    \begin{align*}
        \abs{x(t) - x(t_{0})} & \leq \abs{x(t) - x_{N+1}(t)} + \abs{x_{N+1}(t) - x_{N+1}(t_{0})} + \abs{x_{N+1}(t_{0}) - x(t_{0})} \\
                              & < \frac{\varepsilon}{3} + \frac{\varepsilon}{3} + \frac{\varepsilon}{3} = \varepsilon.
    \end{align*}

    Hence $x$ is continuous at $t_{0}$, for every $t_{0}\in \closedinterval{a, b}$.
\end{proof}


% chapter1:section5:exercise10
\begin{exercise}[Discrete metric]\label{chapter1:section5:exercise10}
    Show that a discrete metric space is complete.
\end{exercise}

\begin{proof}
    Let $(x_{n})$ be a Cauchy sequence in a discrete metric space.

    There exists $N$ such that if $n, m > N$, then $d(x_{n}, x_{m}) < 1$. Let $\varepsilon < 1$, then $x_{N+1} = x_{N+2} = \cdots$, therefore for every $\varepsilon > 0$, $d(x_{n}, x_{N+1}) < \varepsilon$ for every $n > N$. So $(x_{n})$ is convergent.

    Thus every discrete metric space is complete.
\end{proof}

% chapter1:section5:exercise11
\begin{exercise}[Space $s$]\label{chapter1:section5:exercise11}
    Show that in the sequence space, we have $x_{n}\to x$ if and only if $\xi^{(n)}_{j}\to \xi_{j}$ for all $j = 1, 2, \ldots$ where $x_{n} = {(\xi_{j}^{(n)})}$ and $x = (\xi_{j})$.
\end{exercise}

\begin{proof}
    $(\Rightarrow)$ $x_{n}\to x$.

    Let $j$ be a positive integer. For every $\varepsilon > 0$, there exists $N$ such that if $n > N$,
    \[
        d(x_{n}, x) = \sum^{\infty}_{k=1}\frac{1}{2^{k}}\frac{\abs{\xi^{(n)}_{k} - \xi_{k}}}{1 + \abs{\xi^{(n)}_{k} - \xi_{k}}} < \frac{1}{2^{j}}\frac{\varepsilon}{1+\varepsilon}
    \]

    it follows that
    \[
        \frac{1}{2^{j}}\frac{\abs{\xi^{(n)}_{j} - \xi_{j}}}{1 + \abs{\xi^{(n)}_{j} - \xi_{j}}} \leq d(x_{n}, x) < \frac{1}{2^{j}}\frac{\varepsilon}{1 + \varepsilon}
    \]

    so
    \[
        \frac{\abs{\xi^{(n)}_{j} - \xi_{j}}}{1 + \abs{\xi^{(n)}_{j} - \xi_{j}}} < \frac{\varepsilon}{1 + \varepsilon}.
    \]

    Therefore $\abs{\xi^{(n)}_{j} - \xi_{j}} < \varepsilon$ if $n > N$. Hence $\xi^{(n)}_{j}\to \xi_{j}$ as $n\to\infty$ for every positive integer $j$.

    $(\Leftarrow)$ $\xi_{j}^{(n)}\to \xi_{j}$ as $n\to\infty$ for every positive integer $j$.

    Let $\varepsilon > 0$. There exists $M_{0}$ such that $\frac{1}{2^{M_{0}}} < \frac{\varepsilon}{2}$, so
    \[
        \sum_{k > M_{0}}\frac{1}{2^{k}}\frac{\abs{\xi^{(n)}_{k} - \xi_{k}}}{1 + \abs{\xi^{(n)}_{k} - \xi_{k}}} \leq \sum_{k > M_{0}}\frac{1}{2^{k}} = \frac{1}{2^{M_{0}}} < \frac{\varepsilon}{2}.
    \]

    There exists $M_{1}$ such that for each $k\leq M_{0}$, if $n\geq M_{1}$
    \[
        \abs{\xi^{(n)}_{k} - \xi_{k}} < \frac{\varepsilon}{2(M_{0} + 1)}
    \]

    Therefore if $n > \max\{ M_{0}, M_{1} \}$
    \begin{align*}
        d(x_{n}, x) & = \sum^{M_{0}}_{k=1}\frac{1}{2^{k}}\frac{\abs{\xi^{(n)}_{k} - \xi_{k}}}{1 + \abs{\xi^{(n)}_{k} - \xi_{k}}} + \sum_{k > M_{0}}\frac{1}{2^{k}}\frac{\abs{\xi^{(n)}_{k} - \xi_{k}}}{1 + \abs{\xi^{(n)}_{k} - \xi_{k}}} \\
                    & < \sum^{M_{0}}_{k=1}\frac{1}{2^{k}}\abs{\xi^{(n)}_{k} - \xi_{k}} + \frac{\varepsilon}{2}                                                                                                                            \\
                    & < \frac{\varepsilon}{2} + \frac{\varepsilon}{2}                                                                                                                                                                     \\
                    & = \varepsilon.
    \end{align*}

    Hence $x_{n}\to x$ as $n\to\infty$.
\end{proof}


% chapter1:section5:exercise12
\begin{exercise}\label{chapter1:section5:exercise12}
    Using Exercise~\ref{chapter1:section5:exercise11}, show that the sequence space $s$ in 1.2-1 is complete.
\end{exercise}

\begin{proof}
    Let ${(x_{n})}_{n}$ be a Cauchy sequence in the sequence space $s$.

    Let $j$ be a positive integer. For every $\varepsilon > 0$, there exists $N$ such that if $m, n > N$, then
    \[
        d(x_{m}, x_{n}) = \sum^{\infty}_{k=1}\frac{1}{2^{k}}\frac{\abs{\xi^{(m)}_{k} - \xi^{(n)}_{k}}}{1 + \abs{\xi^{(m)}_{k} - \xi^{(n)}_{k}}} < \frac{1}{2^{j}}\frac{\varepsilon}{1 + \varepsilon}
    \]

    so
    \[
        \frac{1}{2^{j}}\frac{\abs{\xi^{(m)}_{j} - \xi^{(n)}_{j}}}{1 + \abs{\xi^{(m)}_{j} - \xi^{(n)}_{j}}} \leq d(x_{m}, x_{n}) < \frac{1}{2^{j}}\frac{\varepsilon}{1 + \varepsilon}.
    \]

    Therefore
    \[
        \frac{\abs{\xi^{(m)}_{j} - \xi^{(n)}_{j}}}{1 + \abs{\xi^{(m)}_{j} - \xi^{(n)}_{j}}} < \frac{\varepsilon}{1 + \varepsilon}
    \]

    and we conclude that $\abs{\xi^{(m)}_{j} - \xi^{(n)}_{j}} < \varepsilon$. Hence for every $\varepsilon > 0$, there exists $N$ such that if $m, n > N$, $\abs{\xi^{(m)}_{j} - \xi^{(n)}_{j}} < \varepsilon$. So ${(\xi^{(n)}_{j})}_{n}$ is a Cauchy sequence. Moreover, $\mathbb{R}$ is complete, so ${(\xi^{(n)}_{j})}_{n}$ is convergent. Let $\xi^{(n)}_{j} \to \xi_{j}$ as $n\to\infty$.

    Let $x = (\xi_{1}, \xi_{2}, \ldots)$. According to Exercise~\ref{chapter1:section5:exercise12}, $x_{n}$ converges to $x$ as $n\to\infty$. Because $x$ is in the sequence space and ${(x_{n})}_{n}$ is an arbitary Cauchy sequence in the sequence space $s$, we conclude that $s$ is complete.
\end{proof}

% chapter1:section5:exercise13
\begin{exercise}\label{chapter1:section5:exercise13}
    Show that in 1.5-9, another Cauchy sequence is $(x_{n})$, where
    \[
        x_{n}(t) = \begin{cases}
            n        & \text{if $0\leq t\leq n^{-2}$} \\
            t^{-1/2} & \text{if $n^{-2}\leq t\leq 1$}
        \end{cases}
    \]
\end{exercise}

\begin{proof}
    For $\varepsilon > 0$, let $N = 1 + \left\lfloor\frac{1}{\varepsilon}\right\rfloor$, if $n > N$ and $p > 0$,
    \begin{align*}
        d(x_{n+p}, x_{n}) & = \int^{1}_{0}\abs{x_{n+p}(t) - x_{n}(t)}dt                                                                                                                             \\
                          & = \int^{1/{(n+p)}^{2}}_{0}\abs{x_{n+p}(t) - x_{n}(t)}dt + \int^{1/n^{2}}_{1/{(n+p)}^{2}}\abs{x_{n+p}(t) - x_{n}(t)}dt + \int^{1}_{1/n^{2}}\abs{x_{n+p}(t) - x_{n}(t)}dt \\
                          & = \int^{1/{(n+p)}^{2}}_{0}\abs{x_{n+p}(t) - x_{n}(t)}dt + \int^{1/n^{2}}_{1/{(n+p)}^{2}}\abs{x_{n+p}(t) - x_{n}(t)}dt                                                   \\
                          & =  \int^{1/{(n+p)}^{2}}_{0}pdt + \int^{1/n^{2}}_{1/{(n+p)}^{2}}\left(\frac{1}{\sqrt{t}} - n\right)dt                                                                    \\
                          & = \frac{p}{{(n+p)}^{2}} + \left(\frac{\sqrt{t}}{2} - nt\right)\Biggl{\vert}^{1/n^{2}}_{1/{{(n+p)}^{2}}}                                                                 \\
                          & = \frac{p}{{(n+p)}^{2}} + \frac{1}{2n} - \frac{1}{2(n+p)} - n\left(\frac{1}{n^{2}} - \frac{1}{{(n+p)}^{2}}\right)                                                       \\
                          & \leq \frac{1}{2n} + \frac{1}{2n} - \frac{1}{n} + \frac{n}{{(n+p)}^{2}}                                                                                                  \\
                          & = \frac{n}{{(n+p)}^{2}} < \frac{1}{n} < \varepsilon
    \end{align*}

    Hence ${(x_{n})}_{n}$ is a Cauchy sequence.
\end{proof}

% chapter1:section5:exercise14
\begin{exercise}\label{chapter1:section5:exercise14}
    Show that the Cauchy sequence in Exercise~\ref{chapter1:section5:exercise13} does not converge.
\end{exercise}

\begin{proof}
    Assume $x_{n}$ converges to a continuous function $x$ on $\closedinterval{0, 1}$ as $n\to\infty$.

    Then $x_{n}(0)\to x(0)$ as $n\to\infty$. So $n\to x(0)$ as $n\to\infty$. This is a contradiction because $x$ is continuous on $\closedinterval{0, 1}$.

    Hence $x_{n}$ does not converge.
\end{proof}

% chapter1:section5:exercise15
\begin{exercise}\label{chapter1:section5:exercise15}
    Let $X$ be the metric space of all real sequences $x = (\xi_{j})$ each of which has only finitely many nonzero terms, and $d(x, y) = \sum\abs{\xi_{j} - \eta_{j}}$, where $y = (\eta_{j})$. Note that this is a finite sum but the number of terms depends on $x$ and $y$. Show that $(x_{n})$ with $x_{n} = (\xi_{j}^{(n)})$,
    \[
        \xi_{j}^{(n)} = \begin{cases}
            j^{-2} & \text{for $j = 1, \ldots, n$} \\
            0      & \text{for $j > n$}
        \end{cases}
    \]

    is Cauchy but does not converge.
\end{exercise}

\begin{proof}
    For every $\varepsilon > 0$, let $N = 1 + \left\lfloor\frac{1}{2\varepsilon}\right\rfloor$, if $n > N$ and $p > 0$,
    \begin{align*}
        d(x_{n+p}, x_{n}) & = \frac{1}{{(n+1)}^{2}} + \cdots + \frac{1}{{(n+p)}^{2}} \\
                          & < \frac{p}{{(n+p)}^{2}}                                  \\
                          & \leq \frac{1}{2n} < \varepsilon.
    \end{align*}

    Hence ${(x_{n})}_{n}$ is a Cauchy sequence.

    Assume $x_{n}$ converges to $x$ as $n\to\infty$. $x = (\xi_{1}, \xi_{2}, \ldots)$.

    Let $j$ be a positive integer. If $\xi_{j} \ne j^{-2}$, then let $\varepsilon = \abs{\xi_{j} - j^{-2}}$, for every $n$, $d(x_{n}, x)\geq \abs{\xi_{j} - j^{-2}} = \varepsilon$. So there exists $\varepsilon > 0$ such that for all $N$ there exists $n > N$ such that $d(x_{n}, x)\geq \varepsilon$. This contradicts the convergence of ${(x_{n})}_{n}$, so $\xi_{j} = j^{-2}$ for every positive integer $j$. This means $x$ has infinitely many nonzero terms, so the assumption is false.

    Thus ${(x_{n})}_{n}$ does not converge.
\end{proof}

\section{Completion of Metric Spaces}

% chapter1:section6:exercise1
\begin{exercise}\label{chapter1:section6:exercise1}
    Show that if a subspace $Y$ of a metric space consists of finitely many points, then $Y$ is complete.
\end{exercise}

\begin{proof}
    $Y = \{ y_{1}, y_{2}, \ldots, y_{r} \}$.

    Let ${(x_{n})}_{n}$ be a Cauchy sequence in $Y$.

    For every $\varepsilon > 0$, there exists $N$ such that if $n, m > N$, $d(x_{m}, x_{n}) < \varepsilon$. Let $\varepsilon = \min\{ d(y_{i}, y_{j}) \mid 1\leq i,j\leq r \}$, we conclude that $x_{N+1} = x_{N+2} = \cdots$.

    Therefore $x_{n}$ converges to $x_{N+1}$. Thus $Y$ is complete.
\end{proof}

% chapter1:section6:exercise2
\begin{exercise}\label{chapter1:section6:exercise2}
    What is the completion of $(X, d)$, where $X$ is the set of all rational numbers and $d(x, y) = \abs{x - y}$?
\end{exercise}

\begin{proof}
    The completion in this case is $\mathbb{R}$.
\end{proof}


% chapter1:section6:exercise3
\begin{exercise}\label{chapter1:section6:exercise3}
    What is the completion of a discrete metric space $X$?
\end{exercise}

\begin{proof}
    Let ${(x_{n})}_{n}$ be a Cauchy sequence in $X$.

    For every $\varepsilon > 0$, there exists $N$ such that if $m, n > N$, $d(x_{m}, d_{n}) < \varepsilon$. Let $\varepsilon < 1$, then $x_{N+1} = x_{N+2} = \cdots$ (since $(X, d)$ is a discrete metric space).

    Therefore $x_{n}$ converges to $x_{N+1}$ as $n\to\infty$. Hence $(X, d)$ is complete, so the completion of a discrete metric space $X$ is itself.
\end{proof}

% chapter1:section6:exercise4
\begin{exercise}\label{chapter1:section6:exercise4}
    If $X_{1}$ and $X_{2}$ are isometric and $X_{1}$ is complete, show that $X_{2}$ is complete.
\end{exercise}

\begin{proof}
    Let $T$ be an isometry from $X_{1}$ and onto $X_{2}$.

    Let ${(y_{n})}_{n}$ be a Cauchy sequence in $X_{2}$. Because $T$ is bijective, there exists $x_{n}\in X_{1}$ such that $Tx_{n} = y_{n}$ for every positive integer $n$.

    For every $\varepsilon > 0$, there exists $N$ such that if $m, n > N$, then $d_{2}(y_{m}, y_{n}) < \varepsilon$. Moreover, $d_{1}(x_{m}, x_{n}) = d_{2}(Tx_{m}, Tx_{n}) = d_{2}(y_{m}, y_{n}) < \varepsilon$. Therefore ${(x_{n})}_{n}$ is a Cauchy sequence. Because $X_{1}$ is complete, ${(x_{n})}_{n}$ converges to $x\in X_{1}$.

    Because $x_{n}$ converges to $x\in X_{1}$, then for every $\varepsilon > 0$, there exists $N$ such that if $n > N$, $d_{1}(x_{n}, x) < \varepsilon$. So $d_{2}(y_{n}, Tx) = d_{2}(Tx_{n}, Tx) = d_{1}(x_{n}, x) < \varepsilon$. Hence ${(y_{n})}_{n}$ converges to $Tx\in X_{2}$. Thus $X_{2}$ is complete.
\end{proof}

% chapter1:section6:exercise5
\begin{exercise}[Homeomorphism]\label{chapter1:section6:exercise5}
    A homeomorphism is a continuous bijective mapping $T: X \to Y$ whose inverse is continuous; the metric spaces $X$ and $Y$ are then said to be homeomorphic.
    \begin{enumerate}[label={(\alph*)}]
        \item Show that if $X$ and $Y$ are isometric, they are homeomorphic.
        \item Illustrate with an example that a complete and an incomplete metric space may be homeomorphic.
    \end{enumerate}
\end{exercise}

\begin{proof}
    \begin{enumerate}[label={(\alph*)}]
        \item Let $T$ be an isometry from $X$ onto $Y$, then $T^{-1}$ is an isometry from $Y$ onto $X$.

              Let $N_{Y}$ be an open subset of $Y$ and $N_{X}$ is the inverse image of $N_{Y}$ under $T$.

              If $N_{Y} = \varnothing$ then $N_{X} = \varnothing$ and they are both open.

              If $N_{Y}$ is not empty, let $y_{0}$ be an element of $N_{Y}$. Because $N_{Y}$ is open, there exists an open ball of radius $\varepsilon$ with center $y_{0}$ contained in $N_{Y}$. Let $y\in B(y_{0}; \varepsilon)$, then
              \[
                  d_{X}(T^{-1}y, T^{-1}y_{0}) = d_{Y}(y, y_{0}) < \varepsilon
              \]

              so $T^{-1}y\in N_{X}$ for every $y\in B(y_{0}; \varepsilon)$ and $B(T^{-1}y_{0}; \varepsilon)$ is contained in $N_{X}$. Therefore $N_{X}$ is open in $X$. Hence $T$ is continuous.

              Analogously, $T^{-1}$ is also continuous. Thus $T: X\to Y$ is a bijective mapping whose inverse is continuous, so $X$ and $Y$ are homeomorphic.
        \item $\exp: \mathbb{R}\to \mathbb{R}_{>0}$ is a bijective function and its inverse $\ln$ is continuous.

              $\mathbb{R}$ is complete but $\mathbb{R}_{>0}$ is incomplete, where the metric is $d(x, y) = \abs{x - y}$.
    \end{enumerate}
\end{proof}

% chapter1:section6:exercise6
\begin{exercise}\label{chapter1:section6:exercise6}
    Show that $C[0, 1]$ and $C[a, b]$ are isometric.
\end{exercise}

\begin{proof}
    Let $T: C[0, 1]\to C[a, b]$ defined by $Tf = \overline{f}$, where
    \[
        \overline{f}(t) = f\left(\frac{t - a}{b - a}\right).
    \]

    $T$ is bijective because $S: C[a, b]\to C[0, 1]$ defined by
    \[
        (S\overline{f})(t) = \overline{f}(a + (b - a)t)
    \]

    satisfies $ST = \text{id}_{C[0,1]}$ and $TS = \text{id}_{C[a,b]}$. Moreover $f(t) = \overline{f}(a + t(b - a))$ for every $t\in \closedinterval{0, 1}$ and
    \begin{align*}
        d_{C[0,1]}(f, g) & = \max_{t\in[0,1]}\abs{f(t) - g(t)} = \max_{t\in [0,1]}\abs{\overline{f}(a + t(b - a)) - \overline{g}(a + t(b - a))}         \\
                         & = \max_{s\in [a, b]}\abs{\overline{f}(s) - \overline{g}(s)} = d_{C[a, b]}(\overline{f}, \overline{g}) = d_{C[a, b]}(Tf, Tg).
    \end{align*}

    Therefore $T$ is an isometry. Thus $C[0,1]$ and $C[a,b]$ are isometric.
\end{proof}

% chapter1:section6:exercise7
\begin{exercise}\label{chapter1:section6:exercise7}
    If $(X, d)$ is complete, show that $(X, \widetilde{d})$, where $\widetilde{d} = \frac{d}{1 + d}$, is complete.
\end{exercise}

\begin{proof}
    Let ${(x_{n})}_{n}$ be a Cauchy sequence in $(X, \widetilde{d})$. For every $\varepsilon > 0$, there exists $N$ such that if $m, n > N$,
    \[
        \widetilde{d}(x_{m}, x_{n}) = \frac{d(x_{m}, x_{n})}{1 + d(x_{m}, x_{n})} < \frac{\varepsilon}{1 + \varepsilon}
    \]

    it follows that $d(x_{m}, x_{n}) < \varepsilon$ if $m, n > N$. Hence ${(x_{n})}_{n}$ is a Cauchy sequence in $(X, d)$.

    Because $(X, d)$ is complete, ${(x_{n})}_{n}$ converges to $x$ in $(X, d)$. For every $\varepsilon > 0$, there exists $N$ such that if $n > N$,
    \[
        d(x_{n}, x) < \frac{\min\{1/2, \varepsilon\}}{1 - \min\{1/2, \varepsilon\}}
    \]

    so
    \[
        \widetilde{d}(x_{n}, x) < \min\{ 1/2, \varepsilon \}\leq \varepsilon.
    \]

    Hence ${(x_{n})}_{n}$ in $(X, \tilde{d})$ converges to $x$. Thus $(X, \widetilde{d})$ is complete.
\end{proof}

% chapter1:section6:exercise8
\begin{exercise}\label{chapter1:section6:exercise8}
    Show that in Exercise~\ref{chapter1:section6:exercise7}, completeness of $(X, \widetilde{d})$ implies completeness of $(X, d)$.
\end{exercise}

\begin{proof}
    Let ${(x_{n})}_{n}$ be a Cauchy sequence in $(X, d)$. For every $\varepsilon > 0$, there exists $N$ such that if $m, n > N$,
    \[
        d(x_{m}, x_{n}) = \frac{\widetilde{d}(x_{m}, x_{n})}{1 - \widetilde{d}(x_{m}, x_{n})} < \frac{\min\{1/2, \varepsilon\}}{1 - \min\{ 1/2, \varepsilon \}}
    \]

    it follows that $\widetilde{d}(x_{m}, x_{n}) < \min\{ 1/2, \varepsilon \} \leq\varepsilon$ if $m, n > N$. Hence ${(x_{n})}_{n}$ is a Cauchy sequence in $(X, \widetilde{d})$.

    Because $(X, \widetilde{d})$ is complete, ${(x_{n})}_{n}$ converges to $x$ in $(X, \widetilde{d})$. For every $\varepsilon$, there exists $N$ such that if $n > N$,
    \[
        \frac{d(x_{n}, x)}{1 + d(x_{n},x)} = \widetilde{d}(x_{n}, x) < \frac{\varepsilon}{1 + \varepsilon}
    \]

    so
    \[
        d(x_{n}, x) < \varepsilon.
    \]

    Hence ${(x_{n})}$ in $(X, d)$ converges to $x$. Thus $(X, d)$ is complete.
\end{proof}

% chapter1:section6:exercise9
\begin{exercise}\label{chapter1:section6:exercise9}
    If $\lim\limits_{n\to\infty} d(x_{n}, x'_{n}) = 0$ and $x_{n}\to \ell$, show that ${(x'_{n})}$ converges and has the limit $\ell$.
\end{exercise}

\begin{proof}
    Because $\lim\limits_{n\to\infty} d(x_{n}, x'_{n}) = 0$ and $x_{n}\to \ell$ as $n\to \infty$, then for every $\varepsilon$, there exist $N$ and $N'$ such that
    \begin{itemize}
        \item if $n > N$, $d(x_{n}, x'_{n}) < \varepsilon/2$,
        \item if $n > N'$, $d(x_{n}, \ell) < \varepsilon/2$.
    \end{itemize}

    So if $n > N'' = \max\{N, N'\}$, then
    \[
        d(x'_{n}, \ell)\leq d(x_{n}, x'_{n}) + d(x_{n}, \ell) < \varepsilon/2 + \varepsilon/2 = \varepsilon.
    \]

    Therefore ${(x'_{n})}_{n}$ converges and has the limit $\ell$.
\end{proof}

% chapter1:section6:exercise10
\begin{exercise}\label{chapter1:section6:exercise10}
    If ${(x_{n})}$ and ${(x'_{n})}$ are convergent sequences in a metric space $(X, d)$ and have the same limit $\ell$, show that $\lim\limits_{n\to\infty} d(x_{n}, x'_{n}) = 0$.
\end{exercise}

\begin{proof}
    For every $\varepsilon > 0$, there exists $N$ and $N'$ such that
    \begin{itemize}
        \item if $n > N$, then $d(x_{n}, \ell) < \varepsilon/2$
        \item if $n > N'$, then $d(x'_{n}, \ell) < \varepsilon/2$
    \end{itemize}

    so if $n > N'' = \max\{ N, N' \}$, then
    \[
        d(x_{n}, x'_{n})\leq d(x_{n}, \ell) + d(x'_{n},\ell) < \varepsilon/2 + \varepsilon/2 = \varepsilon.
    \]

    Therefore $\lim\limits_{n\to\infty} d(x_{n}, x'_{n}) = 0$.
\end{proof}

% chapter1:section6:exercise11
\begin{exercise}\label{chapter1:section6:exercise11}
    Show that $\lim\limits_{n\to\infty} d(x_{n}, x'_{n}) = 0$ defines an equivalence relation on the set of all Cauchy sequences of elements of $X$.
\end{exercise}

\begin{proof}
    For every Cauchy sequence ${(x_{n})}_{n}$, $d(x_{n}, x_{n}) = 0$ so $\lim\limits_{n\to\infty} d(x_{n}, x_{n}) = 0$. Hence the relation is reflexive.

    If two Cauchy sequences ${(x_{n})}_{n}$ and ${(y_{n})}_{n}$ satisfy $\lim\limits_{n\to\infty}d(x_{n}, y_{n}) = 0$ then $\lim\limits_{n\to\infty}d(y_{n}, x_{n}) = 0$ (because $d(x_{n}, y_{n}) = d(y_{n}, x_{n})$ for every positive integer $n$). Hence the relation is symmetric.

    If there Cauchy sequences ${(x_{n})}_{n}, {(y_{n})}_{n}, {(z_{n})}_{n}$ satisfy $\lim\limits_{n\to\infty}d(x_{n}, y_{n}) = 0$ and $\lim\limits_{n\to\infty}d(y_{n}, z_{n}) = 0$, then for every $\varepsilon$, there exist $N$ and $N'$ such that
    \begin{itemize}
        \item if $n > N$, $d(x_{n}, y_{n}) < \varepsilon/2$
        \item if $n > N'$, $d(y_{n}, z_{n}) < \varepsilon/2$
    \end{itemize}

    so if $n > N'' = \max\{ N, N' \}$, then
    \[
        d(x_{n}, z_{n}) \leq d(x_{n}, y_{n}) + d(y_{n}, z_{n}) < \varepsilon/2 + \varepsilon/2 = \varepsilon.
    \]

    Therefore $\lim\limits_{n\to\infty}d(x_{n}, z_{n}) = 0$. Hence the relation is transitive.

    Thus $\lim\limits_{n\to\infty}d(x_{n}, x'_{n}) = 0$ defines an equivalence relation on the set of all Cauchy sequences of elements of $X$.
\end{proof}

% chapter1:section6:exercise12
\begin{exercise}\label{chapter1:section6:exercise12}
    If $(x_{n})$ is Cauchy in $(X, d)$ and ${(x'_{n})}$ in $X$ satisfy $\lim\limits_{n\to\infty} d(x_{n}, x'_{n}) = 0$, show that ${(x'_{n})}$ is Cauchy in $X$.
\end{exercise}

\begin{proof}
    For every $\varepsilon > 0$, there exists $N_{1}, N_{2}$ such that
    \begin{itemize}
        \item if $n > N_{1}$, $d(x_{n}, x'_{n}) < \varepsilon/3$
        \item if $m, n > N_{2}$, $d(x_{m}, x_{n}) < \varepsilon/3$
    \end{itemize}

    so if $m, n > N = \max\{ N_{1}, N_{2} \}$ then
    \[
        d(x'_{m}, x'_{n})\leq d(x_{m}, x'_{m}) + d(x_{m}, x_{n}) + d(x_{n}, x'_{n}) < \varepsilon/3 + \varepsilon/3 + \varepsilon/3 = \varepsilon
    \]

    hence ${(x'_{n})}_{n}$ is Cauchy in $X$.
\end{proof}

% chapter1:section6:exercise13
\begin{exercise}[Pseudometric]\label{chapter1:section6:exercise13}
    A finite pseudometric on a set $X$ is a function $d: X \times X \to \mathbb{R}$ satisfying (M1), (M3), (M4), Section 1.1, and
    \[
        d(x, x) = 0.
    \]

    What is the difference between a metric and a pseudometric? Show that $d(x, y) = \abs{\xi_{1} - \eta_{1}}$ defines a pseudometric on the set of all ordered pairs of real numbers, where $x = (\xi_{1}, \xi_{2}), y = (\eta_{1}, \eta_{2})$.
\end{exercise}

\begin{proof}
    The difference between a metric and a pseudometric is that if $d$ is a metric, then $d(x, y) = 0$ if and only if $x = y$, but if $d$ is a pseudometric, $d(x, y) = 0$ does not imply $x = y$.

    $d(x, x) = \abs{\xi_{1} - \xi_{1}} = 0$. $d(x, y) = \abs{\xi_{1} - \eta_{1}} = \abs{\eta_{1} - \xi_{1}} = d(y, x)$. $d(x, y) = \abs{\xi_{1} - \eta_{1}}\leq \abs{\xi_{1} - \zeta_{1}} + \abs{\zeta_{1} - \eta_{1}} = d(x, z) + d(z, y)$. Therefore $d$ is a pseudometric.
\end{proof}

% chapter1:section6:exercise14
\begin{exercise}\label{chapter1:section6:exercise14}
    Does
    \[
        d(x, y) = \int^{b}_{a}\abs{x(t) - y(t)}dt
    \]

    define a metric or pseudometric on $X$ if $X$ is
    \begin{enumerate}[label={(\alph*)}]
        \item the set of all real-valued continuous functions on $\closedinterval{a,b}$,
        \item the set of all real-valued Riemann integrable functions on $\closedinterval{a, b}$?
    \end{enumerate}
\end{exercise}

\begin{proof}
    \begin{enumerate}[label={(\alph*)}]
        \item $d$ defines a metric on the set of all real-valued continuous functions on $\closedinterval{a, b}$.
        \item $d$ defines a pseudometric on the set of all real-valued continuous functions on $\closedinterval{a, b}$, for examples, $x(t) = 0$ for all $t\in\closedinterval{a, b}$, $y(t) = 0$ for all $t\in\halfopenleft{a, b}$ and $y(a) = 1$, then $d(x, y) = 0$.
    \end{enumerate}
\end{proof}

% chapter1:section6:exercise15
\begin{exercise}\label{chapter1:section6:exercise15}
    If $(X, d)$ is a pseudometric space, we call a set
    \[
        B(x_{0}; r) = \{ x\in X \mid d(x, x_{0}) < r \}
    \]

    where $r > 0$ an open ball in $X$ with center $x_{0}$ and radius $r$. What are open balls of radius $1$ in Exercise~\ref{chapter1:section6:exercise13}.
\end{exercise}

\begin{proof}
    In Exercise~\ref{chapter1:section6:exercise13}, $B(x; 1)$ is the following set
    \[
        \{ (\zeta_{1}, \zeta_{2})\in \mathbb{R}^{2} \mid \abs{\zeta_{1} - \xi_{1}} < 1 \}.
    \]

    On the $Oxy$ plane, the above set is represented as a strip.
\end{proof}
