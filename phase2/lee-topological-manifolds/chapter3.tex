% chktex-file 1
% chktex-file 8
\chapter{New Spaces from Old}

\section*{Subspaces}\addcontentsline{toc}{section}{Subspaces}

\begin{exercise}{3.1}
	Prove that $\mathscr{T}_{S}$ is a topology on $S$.
\end{exercise}

\begin{proof}
	$\varnothing = S\cap\varnothing$, $S = S\cap X$. Therefore $\varnothing, S$ are in $\mathscr{T}_{S}$.

	Suppose ${(U_{\alpha})}_{\alpha\in A}$ is a family of elements of $\mathscr{T}_{S}$. Then for every $\alpha\in A$, there is an open subset $V_{\alpha}$ of $X$ such that $U_{\alpha} = S\cap V_{\alpha}$.
	\[
		\bigcup_{\alpha\in A}U_{\alpha} = \bigcup_{\alpha\in A}(S\cap V_{\alpha}) = S\cap \left(\bigcup_{\alpha\in A}V_{\alpha}\right)
	\]

	is in $\mathscr{T}_{S}$ because $\bigcup_{\alpha\in A}V_{\alpha}$ is an open subset of $X$.

	Suppose $U_{1}, \ldots, U_{n}$ are in $\mathscr{T}_{S}$, then there exist open subsets $V_{1}, \ldots, V_{n}$ of $X$ such that $U_{i} = S\cap V_{i}$ for $i = 1,\ldots, n$.
	\[
		\bigcap^{n}_{i=1}U_{i} = \bigcap^{n}_{i=1}(S\cap V_{i}) = S\cap \left(\bigcap^{n}_{i=1}V_{i}\right)
	\]

	is in $\mathscr{T}_{S}$ because $\bigcap^{n}_{i=1}V_{i}$ is an open subset of $X$.

	Thus $\mathscr{T}_{S}$ is a topology on $S$.
\end{proof}

\begin{exercise}{3.2}\label{exercise:3.2}
	Suppose $S$ is a subspace of $X$. Prove that a subset $B\subseteq S$ is closed in $S$ if and only if it is equal to the intersection of $S$ with some closed subset of $X$.
\end{exercise}

\begin{proof}
	$(\Longrightarrow)$ Suppose $B$ is closed in $S$.

	Then $S\smallsetminus B$ is open in $S$, so there is an open subset $U$ of $X$ such that $S\smallsetminus B = S\cap U$.
	\[
		B = S\smallsetminus (S\smallsetminus B) = S\smallsetminus (S\cap U) = (S\cap X)\smallsetminus (S\cap U) = S\cap (X\smallsetminus U)
	\]

	so $B$ is the intersection of $S$ and the closed subset $S\smallsetminus U$ of $X$.

	$(\Longleftarrow)$ Suppose $B$ is equal to the intersection of $S$ with some closed subset $A$ of $X$.
	\[
		S\smallsetminus B = S\smallsetminus (S\cap A) = (S\cap X) \smallsetminus (S\cap A) = S\cap (X\smallsetminus A).
	\]

	Because $A$ is closed in $X$, $X\smallsetminus A$ is open in $X$. Therefore $S\smallsetminus B$ is open in $S$, it follows that $B$ is closed in $S$.
\end{proof}

\begin{exercise}{3.3}
	Let $M$ be a metric space, and let $S\subseteq M$ be any subset. Show that the subspace topology on $S$ is the same as the metric topology obtained by restricting the metric of $M$ to pairs of points in $S$.
\end{exercise}

\begin{proof}
	Let $d$ be the metric on $M$ and $d_{S}$ is the metric on $S$ obtained by restricting $d$ to $S$. By the definition of open balls, if $x\in S$, we have
	\[
		B^{(d_{S})}_{r}(x) = \{ y : d(x, y) < r \text{ and } y\in S \} = B^{(d)}_{r}(x) \cap S.
	\]

	Let $\mathscr{T}_{d_{S}}$ be the metric topology restricted to $S$ and $\mathscr{T}_{S}$ be the subspace topology on $S$.

	\begin{itemize}
		\item Prove that $\mathscr{T}_{d_{S}}\subseteq \mathscr{T}_{S}$.

		      Let $U$ be an open subset of $S$ with the metric topology generated by $d_{S}$. For every $x\in U$, there is $r_{x} > 0$ such that $B^{(d_{S})}_{r_{x}}(x)\subseteq U$. Therefore $U = \bigcup_{x\in U}B^{(d_{S})}_{r_{x}}(x)$. Moreover, $B^{(d_{S})}_{r_{x}}(x) = S\cap B^{(d)}_{r_{x}}(x)$ so $U = S\cap \bigcup_{x\in U}B^{(d)}_{r_{x}}(x)$. Let $V = \bigcup_{x\in U}B^{(d)}_{r_{x}}(x)$ then $V$ is an open subset of $M$ and $U = S\cap V$, which means $U$ is an open subset of $S$ with the subspace topology.

		      Therefore $\mathscr{T}_{d_{S}}\subseteq \mathscr{T}_{S}$.

		\item Prove that $\mathscr{T}_{S}\subseteq \mathscr{T}_{d_{S}}$.

		      Let $U$ be an open subset of $S$ with the subspace topology, then there is an open subset $V$ of $M$ such that $U = S\cap V$. For every $x\in U$ (which means $x\in V$), there is an open ball $B^{(d)}_{r_{x}}(x)$ such that $x\in B^{(d)}_{r_{x}}(x)\subseteq V$. Therefore $U = \bigcup_{x\in U}(S\cap B^{(d)}_{r_{x}}(x)) = \bigcup_{x\in U}B^{(d_{S})}_{r_{x}}(x)$. So $U$ is an open subset of $S$ with the metric topology generated by $d_{S}$.

		      Therefore $\mathscr{T}_{S}\subseteq\mathscr{T}_{d_{S}}$.
	\end{itemize}

	Thus $\mathscr{T}_{S} = \mathscr{T}_{d_{S}}$, which means the subspace topology on $S$ is the same as the metric topology with the metric obtained by restricting the metric of $M$ to $S$.
\end{proof}

\begin{exercise}{3.6}\label{exercise:3.6}
	Prove Proposition 3.5.

	Suppose $S$ is a subspace of the topological space $X$.
	\begin{enumerate}[label={(\alph*)}]
		\item If $U\subseteq S\subseteq X$, $U$ is open in $S$, and $S$ is open in $X$, then $U$ is open in $X$. The same is true with ``closed'' in place of ``open''.
		\item If $U$ is a subset of $S$ that is either open or closed in $X$, then it is also open or closed in $S$, respectively.
	\end{enumerate}
\end{exercise}

\begin{proof}
	\begin{enumerate}[label={(\alph*)}]
		\item Suppose $U$ is open in $S$, and $S$ is open in $X$.

		      Because $U$ is open in $S$, there is an open set $V$ in $X$ such that $U = S\cap V$. $S$ and $V$ are open in $X$, so $S\cap V$ is open in $X$, hence $U$ is open in $X$.

		      Suppose $U$ is closed in $S$, and $S$ is closed in $X$.

		      By Exercise~\ref{exercise:3.2}, there is a closed set $V$ in $X$ such that $U = S\cap V$. $S$ and $V$ are closed in $X$, so $S\cap V$ is closed in $X$, hence $U$ is closed in $X$.
		\item Suppose $U$ is open in $X$, then $U = S\cap U$, which implies $U$ is open in $S$, due to the definition of subspace topology.

		      Suppose $U$ is closed in $X$, then $U = S\cap U$, and by Exercise~\ref{exercise:3.2}, it follows that $U$ is closed in $S$.
	\end{enumerate}
\end{proof}

\begin{exercise}{3.7}\label{exercise:3.7}
	Suppose $X$ is a topological space and $U\subseteq S\subseteq X$.
	\begin{enumerate}[label={(\alph*)}]
		\item Show that the closure of $U$ in $S$ is equal to $\overline{U}\cap S$.
		\item Show that the interior of $U$ in $S$ contains $\operatorname{Int}U\cap S$; give an example to show that they might not be equal.
	\end{enumerate}
\end{exercise}

\begin{proof}
	\begin{enumerate}[label={(\alph*)}]
		\item Let $W$ be a closed set in $S$ that contains $U$, then there is a closed set $V$ of $X$ such that $W = S\cap V$. Therefore $V$ is a closed set of $X$ that contains $U$, so $\overline{U}\subseteq V$. Hence $\overline{U}\cap S\subseteq S\cap V = W$. On the other hand, $\overline{U}\cap S$ is closed in $S$ and $U\subseteq \overline{U}\cap S$, so $\overline{U}\cap S$ is the smallest closed set in $S$ that contains $U$.

		      Thus the closure of $U$ in $S$ is equal to $\overline{U}\cap S$.
		\item Since $\operatorname{Int} U\subseteq U\subseteq S$, $\operatorname{Int}U\cap S\subseteq U\cap S = U$. $\operatorname{Int}U$ is open in $X$, so $\operatorname{Int} U\cap S$ is open in $S$. Hence the interior of $U$ in $S$ contains $\operatorname{Int}U\cap S$ (because the interior of a given set is the largest open set that is contained in the given set).

		      Consider $\mathbb{R}$ with the Euclidean topology. $\closedinterval{0,1}$ is closed in $\mathbb{R}$. We have $\openinterval{0, 1} = \operatorname{Int}\closedinterval{0, 1}$. The interior of $\closedinterval{0,1}$ in $\closedinterval{0,1}$ is $\closedinterval{0,1}$, however, $\operatorname{Int}\closedinterval{0,1}\cap \closedinterval{0,1} = \openinterval{0,1}$ and $\openinterval{0,1}\subsetneq\closedinterval{0,1}$. This example shows that the interior of $U$ in $S$ might not be equal to $\operatorname{Int} U\cap S$.
	\end{enumerate}
\end{proof}

\begin{exercise}{3.12}
	Prove Proposition 3.11 (Other Properties of the Subspace Topology).

	Suppose $S$ is a subspace of the topological space $X$.
	\begin{enumerate}[label={(\alph*)}]
		\item If $R\subseteq S$ is a subspace of $S$, then $R$ is a subspace of $X$; in other words, the subspace topologies that $R$ inherits from $S$ and from $X$ agree.
		\item If $\mathscr{B}$ is a basis for the topology of $X$, then
		      \[
			      \mathscr{B}_{S} = \{ B\cap S: B\in\mathscr{B} \}
		      \]

		      is a basis for the topology of $S$.
		\item If ${(p_{n})}$ is a sequence of points of $S$ and $p\in S$, then $p_{n}\to p$ if and only if $p_{n}\to p$ in $X$.
		\item Every subspace of a Hausdorff space is Hausdorff.
		\item Every subspace of a first countable space is first countable.
		\item Every subspace of a second countable space is second countable.
	\end{enumerate}
\end{exercise}

\begin{proof}
	\begin{enumerate}[label={(\alph*)}]
		\item Let $U\subseteq R$.

		      Suppose $U$ is an open set in $R$ with the topology inherited from $S$. Then there is an open set $V$ in $S$ such that $U = R\cap V$, and there is an open set $W$ in $X$ such that $V = S\cap W$. Therefore $U = (R\cap S)\cap W = R\cap W$, which means $U$ is an open set in $R$ with the topology inherits from $X$.

		      Suppose $U$ is an open set in $R$ with the topology inherited from $X$. Then there is an open set $W$ in $X$ such that $U = R\cap W$. Therefore $U = R\cap W = (R\cap S)\cap W = R\cap (S\cap W)$. Since $S\cap W$ is an open set in $S$ with topology inherited from $X$, $R\cap (S\cap W)$ is an open set in $R$ with the topology inherited from $S$.

		      Hence the topologies that $R$ inherits from $S$ and from $X$ are the same.
		\item By the definition of subspace topology, every element of $\mathscr{B}_{S}$ is open in $S$.

		      Let $U$ be an open set in $S$, then there is an open set $V$ in $X$ such that $U = S\cap V$. Because $\mathscr{B}$ is a basis for the topology of $X$, there is a collection ${(B_{i})}_{i\in I}$ of elements of $\mathscr{B}$ such that $V = \bigcup_{i\in I}B_{i}$. Therefore $U = S\cap \bigcup_{i\in I}B_{i} = \bigcup_{i\in I}(B_{i}\cap S)$. Because $U$ is an arbitrary open set in $S$ and $U$ is the union of some elements of $\mathscr{B}_{S}$, it follows that $\mathscr{B}_{S}$ is a basis for the topology of $S$.
		\item Suppose $p_{i}\to p$ in $S$.

		      Let $V$ be a neighborhood of $p$ in $X$, then $U = S\cap V$ is a neighborhood of $p$ in $S$. Because $p_{i}\to p$ in $S$, there is $N\in\mathbb{N}$ such that $p_{i}\in U$ for all $i\geq N$. Therefore $p_{i}\in V$ for all $i\geq N$, which means $p_{i}\to p$ in $X$.

		      Suppose $p_{i}\to p$ in $X$.

		      Let $U$ be a neighborhood of $p$ in $S$, then there is an open set $V$ in $X$ such that $U = S\cap V$. Because $V$ is also a neighborhood of $p$ in $X$, and $p_{i}\to p$ in $X$, there is $N\in\mathbb{N}$ such that $p_{i}\in V$ for all $i\geq N$. By the hypothesis, $p_{i}\in S$ for all $i$, so $p_{i}\in S\cap V = U$ for all $i\geq N$. Therefore $p_{i}\to p$ in $S$.
		\item Suppose $X$ is a Hausdorff space and $S$ is a subspace of $X$.

		      Let $x, y$ be two distinct points of $S$. Because $x, y$ are also points of $X$, they are separated by some neighborhoods $V_{x}, V_{y}$ in $X$. $U_{x} = S\cap V_{x}$ is a neighborhood of $x$ in $S$, $U_{y} = S\cap V_{y}$ is a neighborhood of $y$ in $S$. $V_{x}$ and $V_{y}$ are disjoint, so are $U_{x}$ and $U_{y}$, hence $x$ and $y$ are separated by open sets in $S$. Thus $S$ is Hausdorff.
		\item Suppose $X$ is a first countable space and $S$ is a subspace of $X$.

		      Let $x$ be a point of $S$ and $U$ be a neighborhood of $x$ in $S$. Let $\mathscr{B}$ be a countable neighborhood basis of $x$ in $X$. Define $\mathscr{B}_{S} = \{ S\cap B : B\in\mathscr{B} \}$. Because $U$ is open in $S$, there is an open set $V$ in $X$ such that $U = S\cap V$. Since $X$ is first countable, there is $B\in\mathscr{B}$ such that $B\subseteq V$, so $S\cap B\subseteq U\cap V = U$. Therefore $\mathscr{B}_{S}$ is a countable neighborhood basis of $x$ in $S$. Since $x$ is an arbitrary point of $S$, we conclude that $S$ is first countable.
		\item This follows directly from part (b).
	\end{enumerate}
\end{proof}

\subsection*{Topological Embeddings}\addcontentsline{toc}{subsection}{Topological Embeddings}

\begin{exercise}{3.13}
	Let $X$ be a topological space and let $S$ be a subspace of $X$. Show that the inclusion map $S\xhookrightarrow{} X$ is a topological embedding.
\end{exercise}

\begin{proof}
	Denote the inclusion map by $\iota_{S}$. If $x, y\in S$ and $x\ne y$ then $\iota_{S}(x)\ne \iota_{S}(y)$ because $\iota_{S}(x) = x, \iota_{S}(y) = y$, so $\iota_{S}$ is injective. $\iota_{S}$ is continuous, according to the characteristic property of subspace topology.

	The image set of $\iota_{S}$ is $S$. Let $f: S\to S$ defined by $f(x) = \iota_{S}(x)$, then $f$ is bijective. For every open subset $U$ of $S$, $f^{-1}(U) = U$, which is open, so $f$ is continuous. $f^{-1} = f$ so $f^{-1}$ is also continuous. Hence $\iota_{S}$ is a homeomorphism from $S$ onto its image.

	Thus the inclusion map $S\xhookrightarrow{} X$ is a topological embedding.
\end{proof}

\begin{exercise}{3.17}
	Give an example of a topological embedding that is neither an open map nor a closed map.
\end{exercise}

\begin{proof}
	Consider $\varphi: \mathbb{Q}\to \mathbb{R}$ where $\varphi(x) = x$. $\varphi$ is a topological embedding because $\mathbb{Q}$ is a subspace of $\mathbb{R}$.

	Let $A$ be a nonempty open subset of $\mathbb{Q}$, then $\varphi(A) = A$. But $A$ is not open in $\mathbb{R}$, because every neighborhood of every point of $A$ contains an irrational number. Therefore $\varphi$ is not open.

	Let $B$ be a nonempty closed subset of $\mathbb{Q}$, then $\varphi(B) = B$. $\mathbb{R}\smallsetminus B$ is not open in $\mathbb{R}$ because every neighborhood of every point of $B$ contains a rational number, so $B$ is not closed. Therefore $\varphi$ is not closed.

	Thus $\varphi$ is a topological embedding that is neither an open map nor a closed map.
\end{proof}

\begin{exercise}{3.19}
	Prove Proposition 3.18.

	A surjective topological embedding is a homeomorphism.
\end{exercise}

\begin{proof}
	Let $\varphi: X\to Y$ be a surjective topological embedding. Because $\varphi$ is surjective, $\varphi(X) = Y$. Since $\varphi$ is a topological embedding and $\varphi(X) = Y$, $\varphi: X\to Y$ is bijective. From the definition of topological embedding, it follows that $\varphi$ is a homeomorphism.
\end{proof}

\section*{Product Spaces}\addcontentsline{toc}{section}{Product Spaces}

\begin{exercise}{3.25}
	Suppose $X_{1}, \ldots, X_{n}$ are arbitrary topological spaces. On their Cartesian product $X_{1}\times \cdots \times X_{n}$, we define the \textbf{product topology} to be the topology generated by the following basis:
	\[
		\mathscr{B} = \{ U_{1} \times \cdots \times U_{n}: \text{$U_{i}$ is an open subset of $X_{i}$, $i = 1,\ldots,n $}  \}.
	\]

	Prove that $\mathscr{B}$ is a basis for a topology.
\end{exercise}

\begin{proof}
	Every point of $X_{1}\times\cdots\times X_{n}$ is in some element of $\mathscr{B}$.

	Let $U, V\in \mathscr{B}$, then $U = U_{1}\times\cdots\times U_{n}$ and $V = V_{1}\times\cdots\times V_{n}$ for some open subsets $U_{i}, V_{i}$ of $X_{i}$, where $i = 1,\ldots,n$. Then
	\[
		U\cap V = (U_{1}\cap V_{1})\times\cdots\times (U_{n}\times V_{n})
	\]

	which means $U\cap V$ is also open in $X_{1}\times\cdots\times X_{n}$.

	Thus $\mathscr{B}$ is a basis for a topology.
\end{proof}

\begin{exercise}{3.26}
	Show that the product topology on $\mathbb{R}^{n} = \mathbb{R}\times\cdots\times\mathbb{R}$ is the same as the metric topology induced by the Euclidean distance function.
\end{exercise}

\begin{proof}
	\begin{itemize}
		\item Prove that the metric topology is coarser than the product topology on $\mathbb{R}^{n}$.

		      Let $B^{(d)}_{r}(x)$ be an open ball in $\mathbb{R}^{n}$ with the metric topology induced by the Euclidean distance function. For every $y\in B^{(d)}_{r}(x)$, consider the following open cube of $\mathbb{R}^{n}$ with the product topology:
		      \[
			      \openinterval{x_{1} - \frac{d(x, y)}{\sqrt{n}}, x_{1} + \frac{d(x, y)}{\sqrt{n}}}\times\cdots\times\openinterval{x_{n} - \frac{d(x, y)}{\sqrt{n}}, x_{n} + \frac{d(x, y)}{\sqrt{n}}}.
		      \]

		      For every point $z$ of this set
		      \[
			      d(x, z) = \sqrt{\sum^{n}_{i=1}{(x_{i} - z_{i})}^{2}} < \sqrt{\sum^{n}_{i=1}{\left(\frac{d(x, y)}{\sqrt{n}}\right)}^{2}} = d(x, y) < r.
		      \]

		      Therefore the given open cube is a subset of $B^{(d)}_{r}(x)$. Hence every open ball is an union of some open cubes, which implies every open set in $\mathbb{R}^{n}$ as a metric topology induced by the Euclidean distance function is also open in $\mathbb{R}^{n}$ with the product topology.
		\item Prove that the metric topology is finer than the product topology on $\mathbb{R}^{n}$.

		      Let
		      \[
			      R = \openinterval{a_{1},b_{1}} \times\cdots\times \openinterval{a_{n},b_{n}}
		      \]
		      be a product open set and $x\in R$. Define $r_{i} = \min\{ x_{i} - a_{i}, b_{i} - x_{i} \}$ and $r = \min\{ r_{i} : i = 1,\ldots, n \}$. For every $y\in B^{(d)}_{r}(x)$, we have
		      \begin{align*}
			      y_{i} - a_{i} & = (y_{i} - x_{i}) + (x_{i} - a_{i})                          \\
			                    & \geq -\left\vert{y_{i} - x_{i}}\right\vert + (x_{i} - a_{i}) \\
			                    & > -r + (x_{i} - a_{i})                                       \\
			                    & \geq (x_{i} - a_{i}) - r_{i}                                 \\
			                    & \geq 0
		      \end{align*}

		      and
		      \begin{align*}
			      b_{i} - y_{i} & = (b_{i} - x_{i}) + (x_{i} - y_{i})                         \\
			                    & \geq (b_{i} - x_{i}) - \left\vert{x_{i} - y_{i}}\right\vert \\
			                    & > (b_{i} - x_{i}) - r                                       \\
			                    & \geq (b_{i} - x_{i}) - r_{i}                                \\
			                    & \geq 0
		      \end{align*}

		      so $a_{i} < y_{i} < b_{i}$ for $i = 1,\ldots, n$, which implies $B^{(d)}_{r}(x)\subseteq R$.

		      Hence every product open set is open in $\mathbb{R}^{n}$ and with the metric topology induced by the Euclidean distance function. Moreover, every open set in $\mathbb{R}^{n}$ with the product topology is open in the metric topology.
	\end{itemize}

	Thus the product topology on $\mathbb{R}^{n}$ is the same as the metric topology induced by the Euclidean distance function.
\end{proof}

\begin{exercise}{3.29}
	If $X_{1}, \ldots, X_{n}$ are topological spaces, each canonical projection $\pi_{i}: X_{1}\times\cdots\times X_{n}\to X_{i}$ is continuous.

	Prove the preceding corollary using only the characteristic property of the product topology.
\end{exercise}

\begin{proof}
	The identity map $\operatorname{Id}_{X_{1}\times\cdots\times X_{n}}$ is continuous. By the characteristic property of the product topology, $\pi_{i}\circ \operatorname{Id}_{X_{1}\times\cdots\times X_{n}}$ is continuous for every $i = 1,\ldots, n$. Moreover, $\pi_{i}\circ \operatorname{Id}_{X_{1}\times\cdots\times X_{n}} = \pi_{i}$ so $\pi_{i}$ is continuous for every $i = 1,\ldots,n$.
\end{proof}

\begin{exercise}{3.32}
	Prove Proposition 3.31 (Other Properties of the Product Topology).

	Let $X_{1}, \ldots, X_{n}$ be topological spaces.
	\begin{enumerate}[label={(\alph*)}]
		\item The product topology is ``associative'' in the sense that the three topologies on
		      the set $X_{1} \times X_{2} \times X_{3}$, obtained by thinking of it as $X_{1} \times X_{2} \times X_{3}, (X_{1} \times X_{2}) \times X_{3}$, or $X_{1} \times (X_{2} \times X_{3})$ are all equal.
		\item For any $i\in\{1,\ldots,n\}$ and any points $x_{j}\in X_{j}$, $j\ne i$, the map $f: X_{i}\to X_{1}\times \cdots \times X_{n}$ given by
		      \[
			      f(x) = (x_{1}, \ldots, x_{i-1}, x, x_{i+1}, \ldots, x_{n})
		      \]

		      is a topological embedding of $X_{i}$ into the product space.
		\item Each canonical projection $\pi_{i}: X_{1}\times\cdots\times X_{n}\to X_{i}$ is an open map.
		\item If for each $i$, $\mathscr{B}_{i}$ is a basis for the topology of $X_{i}$, then the set
		      \[
			      \{ B_{1}\times\cdots\times B_{n} : B_{i}\in\mathscr{B}_{i} \}
		      \]

		      is a basis for the product topology on $X_{1}\times\cdots\times X_{n}$.
		\item If $S_{i}$ is a subspace of $X_{i}$ for $i = 1,\ldots,n$, then the product topology and the subspace topology on $S_{1}\times \cdots \times S_{n}\subseteq X_{1}\times \cdots\times X_{n}$ are equal.
		\item If each $X_{i}$ is Hausdorff, so is $X_{1}\times\cdots\times X_{n}$.
		\item If each $X_{i}$ is first countable, so is $X_{1}\times\cdots\times X_{n}$.
		\item If each $X_{i}$ is second countable, so is $X_{1}\times\cdots\times X_{n}$.
	\end{enumerate}
\end{exercise}

\begin{proof}
	\begin{enumerate}[label={(\alph*)}]
		\item Let $\operatorname{Id}_{(1,2),3}: (X_{1}\times X_{2})\times X_{3}\to X_{1}\times X_{2}\times X_{3}$ be the identity map. The following diagrams (see~\ref{fig:3.32-a-1}) commute
		      \begin{figure}[htp]
			      \renewcommand{\thefigure}{3.32-a-1}
			      \centering
			      \begin{tikzpicture}[every edge/.style = {draw, -latex, thick}]
				      \matrix (m) [matrix of math nodes, row sep=3em, column sep=3em]
				      {
					      (X_{1}\times X_{2})\times X_{3} & X_{1}\times X_{2}\times X_{3} \\
					      X_{1}\times X_{2}               & X_{1}                         \\
				      };
				      \path[->] (m-1-1) edge node[above] {$\operatorname{Id}_{(1,2),3}$} (m-1-2);
				      \path[->] (m-1-1) edge node[left] {$\pi^{(X_{1}\times X_{2})\times X_{3}}_{X_{1}\times X_{2}}$} (m-2-1);
				      \path[->] (m-2-1) edge node[below] {$\pi^{X_{1}\times X_{2}}_{X_{1}}$} (m-2-2);
				      \path[->] (m-1-2) edge node[right] {$\pi^{X_{1}\times X_{2}\times X_{3}}_{X_{1}}$} (m-2-2);
			      \end{tikzpicture}
			      \begin{tikzpicture}[every edge/.style = {draw, -latex, thick}]
				      \matrix (m) [matrix of math nodes, row sep=3em, column sep=3em]
				      {
					      (X_{1}\times X_{2})\times X_{3} & X_{1}\times X_{2}\times X_{3} \\
					      X_{1}\times X_{2}               & X_{2}                         \\
				      };
				      \path[->] (m-1-1) edge node[above] {$\operatorname{Id}_{(1,2),3}$} (m-1-2);
				      \path[->] (m-1-1) edge node[left] {$\pi^{(X_{1}\times X_{2})\times X_{3}}_{X_{1}\times X_{2}}$} (m-2-1);
				      \path[->] (m-2-1) edge node[below] {$\pi^{X_{1}\times X_{2}}_{X_{2}}$} (m-2-2);
				      \path[->] (m-1-2) edge node[right] {$\pi^{X_{1}\times X_{2}\times X_{3}}_{X_{2}}$} (m-2-2);
			      \end{tikzpicture}
			      \begin{tikzpicture}[every edge/.style = {draw, -latex, thick}]
				      \matrix (m) [matrix of math nodes, row sep=3em, column sep=3em]
				      {
					      (X_{1}\times X_{2})\times X_{3} & X_{1}\times X_{2}\times X_{3} \\
					                                      & X_{3}                         \\
				      };
				      \path[->] (m-1-1) edge node[above] {$\operatorname{Id}_{(1,2),3}$} (m-1-2);
				      \path[->] (m-1-1) edge node[below left] {$\pi^{(X_{1}\times X_{2})\times X_{3}}_{X_{3}}$} (m-2-2);
				      \path[->] (m-1-2) edge node[right] {$\pi^{X_{1}\times X_{2}\times X_{3}}_{X_{3}}$} (m-2-2);
			      \end{tikzpicture}
			      \caption{}\label{fig:3.32-a-1}
		      \end{figure}

		      Due to the characteristic property of the product topology, and the composition of continuous maps is continuous, it follows that $\operatorname{Id}_{(1,2),3}$ is continuous.

		      Define $f: X_{1}\times X_{2}\times X_{3}\to X_{1}\times X_{2}$ by $f(x_{1}, x_{2}, x_{3}) = (x_{1}, x_{2})$. The two following diagrams (see~\ref{fig:3.32-a-2}) commute
		      \begin{figure}[htp]
			      \renewcommand{\thefigure}{3.32-a-2}
			      \centering
			      \begin{tikzpicture}[every edge/.style = {draw, -latex, thick}]
				      \matrix (m) [matrix of math nodes, row sep=3em, column sep=3em] {
					      X_{1}\times X_{2}\times X_{3} & X_{1}\times X_{2} \\
					                                    & X_{1}             \\
				      };
				      \path[->] (m-1-1) edge node[above] {$f$} (m-1-2);
				      \path[->] (m-1-2) edge node[right] {$\pi^{X_{1}\times X_{2}}_{X_{1}}$} (m-2-2);
				      \path[->] (m-1-1) edge node[below left] {$\pi^{X_{1}\times X_{2}\times X_{3}}_{X_{1}}$} (m-2-2);
			      \end{tikzpicture}
			      \begin{tikzpicture}[every edge/.style = {draw, -latex, thick}]
				      \matrix (m) [matrix of math nodes, row sep=3em, column sep=3em] {
					      X_{1}\times X_{2}\times X_{3} & X_{1}\times X_{2} \\
					                                    & X_{2}             \\
				      };
				      \path[->] (m-1-1) edge node[above] {$f$} (m-1-2);
				      \path[->] (m-1-2) edge node[right] {$\pi^{X_{1}\times X_{2}}_{X_{2}}$} (m-2-2);
				      \path[->] (m-1-1) edge node[below left] {$\pi^{X_{1}\times X_{2}\times X_{3}}_{X_{1}}$} (m-2-2);
			      \end{tikzpicture}
			      \caption{}\label{fig:3.32-a-2}
		      \end{figure}

		      By the characteristic property of the product topology, it follows that $f$ is continuous. Moreover, the two following diagrams (see~\ref{fig:3.32-a-3}) commute
		      \begin{figure}[htp]
			      \renewcommand{\thefigure}{3.32-a-3}
			      \centering
			      \begin{tikzpicture}[every edge/.style = {draw, -latex, thick}]
				      \matrix (m) [matrix of math nodes, row sep=3em, column sep=3em] {
					      X_{1}\times X_{2}\times X_{3} & (X_{1}\times X_{2})\times X_{3} \\
					                                    & X_{1} \times X_{2}              \\
				      };
				      \path[->] (m-1-1) edge node[above] {$\operatorname{Id}_{(1,2),3}^{-1}$} (m-1-2);
				      \path[->] (m-1-2) edge node[right] {$\pi^{(X_{1}\times X_{2})\times X_{3}}_{X_{1}\times X_{2}}$} (m-2-2);
				      \path[->] (m-1-1) edge node[below left] {$f$} (m-2-2);
			      \end{tikzpicture}
			      \begin{tikzpicture}[every edge/.style = {draw, -latex, thick}]
				      \matrix (m) [matrix of math nodes, row sep=3em, column sep=3em] {
					      X_{1}\times X_{2}\times X_{3} & (X_{1}\times X_{2})\times X_{3} \\
					                                    & X_{3}                           \\
				      };
				      \path[->] (m-1-1) edge node[above] {$\operatorname{Id}_{(1,2),3}^{-1}$} (m-1-2);
				      \path[->] (m-1-2) edge node[right] {$\pi^{(X_{1}\times X_{2})\times X_{3}}_{X_{3}}$} (m-2-2);
				      \path[->] (m-1-1) edge node[below left] {$\pi^{X_{1}\times X_{2}\times X_{3}}_{X_{3}}$} (m-2-2);
			      \end{tikzpicture}
			      \caption{}\label{fig:3.32-a-3}
		      \end{figure}

		      By the characteristic property of the product topology, it follows that $\operatorname{Id}_{(1,2),3}^{-1}$ is continuous. $\operatorname{Id}_{(1,2),3}$ is bijective and bicontinuous, so it is a homeomorphism. Hence $X_{1}\times X_{2}\times X_{3}$ and $(X_{1}\times X_{2})\times X_{3}$ are homeomorphic.

		      Let $\operatorname{Id}_{1,(2,3)}: X_{1}\times (X_{2}\times X_{3})\to X_{1}\times X_{2}\times X_{3}$ be the identity map. The following diagrams (see~\ref{fig:3.32-a-4}) commute
		      \begin{figure}[htp]
			      \renewcommand{\thefigure}{3.32-a-4}
			      \centering
			      \begin{tikzpicture}[every edge/.style = {draw, -latex, thick}]
				      \matrix (m) [matrix of math nodes, row sep=3em, column sep=3em] {
					      X_{1}\times (X_{2}\times X_{3}) & X_{1}\times X_{2} \times X_{3} \\
					      X_{1}                                                            \\
				      };
				      \path[->] (m-1-1) edge node[above] {$\operatorname{Id}_{1,(2,3)}$} (m-1-2);
				      \path[->] (m-1-1) edge node[left] {$\pi^{X_{1}\times (X_{2}\times X_{3})}_{X_{1}}$} (m-2-1);
				      \path[->] (m-1-2) edge node[below right] {$\pi^{X_{1}\times X_{2}\times X_{3}}_{X_{1}}$} (m-2-1);
			      \end{tikzpicture}
			      \begin{tikzpicture}[every edge/.style = {draw, -latex, thick}]
				      \matrix (m) [matrix of math nodes, row sep=3em, column sep=3em] {
					      X_{1}\times (X_{2}\times X_{3}) & X_{1}\times X_{2} \times X_{3} \\
					      X_{2}\times X_{3}               & X_{2}                          \\
				      };
				      \path[->] (m-1-1) edge node[above] {$\operatorname{Id}_{1,(2,3)}$} (m-1-2);
				      \path[->] (m-1-1) edge node[left] {$\pi^{X_{1}\times (X_{2}\times X_{3})}_{X_{2}\times X_{3}}$} (m-2-1);
				      \path[->] (m-2-1) edge node[above] {$\pi^{X_{2}\times X_{3}}_{X_{2}}$} (m-2-2);
				      \path[->] (m-1-2) edge node[right] {$\pi^{X_{1}\times X_{2}\times X_{3}}_{X_{2}}$} (m-2-2);
			      \end{tikzpicture}
			      \begin{tikzpicture}[every edge/.style = {draw, -latex, thick}]
				      \matrix (m) [matrix of math nodes, row sep=3em, column sep=3em] {
					      X_{1}\times (X_{2}\times X_{3}) & X_{1}\times X_{2} \times X_{3} \\
					      X_{2}\times X_{3}               & X_{3}                          \\
				      };
				      \path[->] (m-1-1) edge node[above] {$\operatorname{Id}_{1,(2,3)}$} (m-1-2);
				      \path[->] (m-1-1) edge node[left] {$\pi^{X_{1}\times (X_{2}\times X_{3})}_{X_{2}\times X_{3}}$} (m-2-1);
				      \path[->] (m-2-1) edge node[above] {$\pi^{X_{2}\times X_{3}}_{X_{3}}$} (m-2-2);
				      \path[->] (m-1-2) edge node[right] {$\pi^{X_{1}\times X_{2}\times X_{3}}_{X_{3}}$} (m-2-2);
			      \end{tikzpicture}
			      \caption{}\label{fig:3.32-a-4}
		      \end{figure}

		      By the characteristic property of the product topology, and the composition of continuous maps is continuous, it follows that $\operatorname{Id}_{1,(2,3)}$ is continuous.

		      Define the map $g: X_{1}\times X_{2}\times X_{3}\to X_{2}\times X_{3}$ by $g(x_{1}, x_{2}, x_{3}) = (x_{2}, x_{3})$. The following diagrams (see~\ref{fig:3.32-a-5}) commute
		      \begin{figure}[htp]
			      \renewcommand{\thefigure}{3.32-a-5}
			      \centering
			      \begin{tikzpicture}[every edge/.style = {draw, -latex, thick}]
				      \matrix (m) [matrix of math nodes, row sep=3em, column sep=3em] {
					      X_{1}\times X_{2} \times X_{3} & X_{2}\times X_{3} \\
					      X_{2}                                              \\
				      };
				      \path[->] (m-1-1) edge node[above] {$g$} (m-1-2);
				      \path[->] (m-1-1) edge node[left] {$\pi^{X_{1}\times X_{2}\times X_{3}}_{X_{2}}$} (m-2-1);
				      \path[->] (m-1-2) edge node[below right] {$\pi^{X_{2}\times X_{3}}_{X_{2}}$} (m-2-1);
			      \end{tikzpicture}
			      \begin{tikzpicture}[every edge/.style = {draw, -latex, thick}]
				      \matrix (m) [matrix of math nodes, row sep=3em, column sep=3em] {
					      X_{1}\times X_{2} \times X_{3} & X_{2}\times X_{3} \\
					      X_{3}                                              \\
				      };
				      \path[->] (m-1-1) edge node[above] {$g$} (m-1-2);
				      \path[->] (m-1-1) edge node[left] {$\pi^{X_{1}\times X_{2}\times X_{3}}_{X_{3}}$} (m-2-1);
				      \path[->] (m-1-2) edge node[below right] {$\pi^{X_{2}\times X_{3}}_{X_{3}}$} (m-2-1);
			      \end{tikzpicture}
			      \caption{}\label{fig:3.32-a-5}
		      \end{figure}

		      By the characteristic property of the product topology, it follows that $g$ is continuous. The following diagrams (see~\ref{fig:3.32-a-6}) commute
		      \begin{figure}[htp]
			      \renewcommand{\thefigure}{3.32-a-6}
			      \centering
			      \begin{tikzpicture}[every edge/.style = {draw, -latex, thick}]
				      \matrix (m) [matrix of math nodes, row sep=3em, column sep=3em] {
					      X_{1}\times X_{2} \times X_{3} & X_{1}\times (X_{2}\times X_{3}) \\
					                                     & X_{2}\times X_{3}               \\
				      };
				      \path[->] (m-1-1) edge node[above] {$\operatorname{Id}_{1,(2,3)}^{-1}$} (m-1-2);
				      \path[->] (m-1-2) edge node[right] {$\pi^{X_{1}\times X_{2}\times X_{3}}_{X_{2}\times X_{3}}$} (m-2-2);
				      \path[->] (m-1-1) edge node[below left] {$g$} (m-2-2);
			      \end{tikzpicture}
			      \begin{tikzpicture}[every edge/.style = {draw, -latex, thick}]
				      \matrix (m) [matrix of math nodes, row sep=3em, column sep=3em] {
					      X_{1}\times X_{2} \times X_{3} & X_{1}\times (X_{2}\times X_{3}) \\
					                                     & X_{1}                           \\
				      };
				      \path[->] (m-1-1) edge node[above] {$\operatorname{Id}_{1,(2,3)}^{-1}$} (m-1-2);
				      \path[->] (m-1-2) edge node[right] {$\pi^{X_{1}\times X_{2}\times X_{3}}_{X_{1}}$} (m-2-2);
				      \path[->] (m-1-1) edge node[below left] {$\pi^{X_{1}\times X_{2}\times X_{3}}_{X_{1}}$} (m-2-2);
			      \end{tikzpicture}
			      \caption{}\label{fig:3.32-a-6}
		      \end{figure}

		      By the characteristic property of the product topology, it follows that $\operatorname{Id}_{1,(2,3)}^{-1}$ is continuous. Hence $\operatorname{Id}_{1,(2,3)}$ is bijective and bicontinuous, therefore it is a homeomorphism. So $X_{1}\times (X_{2}\times X_{3})$ and $X_{1}\times X_{2}\times X_{3}$ are homeomorphic.

		      Thus $X_{1}\times X_{2}\times X_{3}, (X_{1}\times X_{2})\times X_{3}, X_{1}\times (X_{2}\times X_{3})$ are homeomorphic.
		\item The image set of $f$ is $\{ x_{1} \} \times \cdots \times \{ x_{i-1} \} \times X_{i} \times \{ x_{i+1} \} \times \cdots \times \{ x_{n} \}$. So as a map from $X_{i}$ to its image set, $f$ is bijective.

		      The following diagrams commute
		      \begin{figure}[htp]
			      \centering
			      \begin{tikzpicture}[every edge/.style = {draw, -latex, thick}]
				      \matrix (m) [matrix of math nodes, row sep=3em, column sep=3em] {
					      X_{i} & \{ x_{1} \} \times \cdots \times \{ x_{i-1} \} \times X_{i} \times \{ x_{i+1} \} \times \cdots \times \{ x_{n} \} \\
					            & {\{ x_{k} \}}_{\text{where $k\ne i$}}                                                 \\
				      };
				      \path[->] (m-1-1) edge (m-1-2);
				      \path[->] (m-1-1) edge node[below left] {constant map} (m-2-2);
				      \path[->] (m-1-2) edge node[right] {constant map} (m-2-2);
			      \end{tikzpicture}
			      \begin{tikzpicture}[every edge/.style = {draw, -latex, thick}]
				      \matrix (m) [matrix of math nodes, row sep=3em, column sep=3em] {
					      X_{i} & \{ x_{1} \} \times \cdots \times \{ x_{i-1} \} \times X_{i} \times \{ x_{i+1} \} \times \cdots \times \{ x_{n} \} \\
					            & X_{i}                                                                                                             \\
				      };
				      \path[->] (m-1-1) edge (m-1-2);
				      \path[->] (m-1-1) edge node[below left] {$\operatorname{Id}_{X_{i}}$} (m-2-2);
				      \path[->] (m-1-2) edge node[right] {canonical projection} (m-2-2);
			      \end{tikzpicture}
		      \end{figure}

		      It follows from the characteristic property of the product topology that $f$ is continuous.

		      On $\{ x_{1} \} \times \cdots \times \{ x_{i-1} \} \times X_{i} \times \{ x_{i+1} \} \times \cdots \times \{ x_{n} \}$, we use the subspace topology.

		      If $U\subseteq X_{i}$ is open, then
		      \begin{align*}
			      f(U) & = \{ x_{1} \} \times \cdots \times \{ x_{i-1} \} \times U \times \{ x_{i+1} \} \times \cdots \times \{ x_{n} \}                    \\
			           & = \left( \{ x_{1} \} \times \cdots \times \{ x_{i-1} \} \times X_{i} \times \{ x_{i+1} \} \times \cdots \times \{ x_{n} \} \right) \\
			           & \cap (X_{1}\times\cdots \times X_{i-1}\times U\times X_{i+1}\times \cdots \times X_{n}).
		      \end{align*}

		      So $f(U)$ is open in $\{ x_{1} \} \times \cdots \times \{ x_{i-1} \} \times X_{i} \times \{ x_{i+1} \} \times \cdots \times \{ x_{n} \}$.

		      Hence $f: X_{i}\to \{ x_{1} \} \times \cdots \times \{ x_{i-1} \} \times X_{i} \times \{ x_{i+1} \} \times \cdots \times \{ x_{n} \}$ is bijective, continuous, and open, so it is a homeomorphism. Thus $f$ is a topological embedding.
		\item Let $U$ be an open map in $X_{1}\times\cdots\times X_{n}$, then $U$ is an union of product open sets in $X_{1}\times\cdots\times X_{n}$. Moreover, the image of an (arbitrary) union is the union of images, in other words, $f(\bigcup_{i\in I}A_{i}) = \bigcup_{i\in I}f(A_{i})$. Then it suffices to prove that the image of a product open set is open. Let $U_{1}\times\cdots \times U_{n}$ be a product open set in $X_{1}\times \cdots\times X_{n}$, then $\pi_{i}(U_{1}\times \cdots\times U_{n}) = U_{i}$, which is open in $X_{i}$. Thus $\pi_{i}$ is an open map, for every $i = 1,\ldots,n$.
		\item Let $\mathscr{B} = \{ B_{1}\times\cdots\times B_{n} : B_{i}\in\mathscr{B}_{i} \}$, then every element of $\mathscr{B}$ is open in $X_{1}\times \cdots \times X_{n}$.

		      Every open set in $X_{1}\times\cdots\times X_{n}$ is an union of product open sets. Let $U$ is a nonempty open set in $X_{1}\times\cdots\times X_{n}$. If $(x_{1}, \ldots, x_{n})\in U$ then there exist $U_{1}\times \cdots\times U_{n}$ such that $U_{i}$ is open in $X_{i}$ for $i = 1,\ldots,n$, and $(x_{1}, \ldots, x_{n})\in U_{1}\times\cdots\times U_{n}$ (because $U$ is an union of product open sets). $\mathscr{B}_{i}$ is a basis for the topology on $X_{i}$, so there is $B_{i}\in\mathscr{B}_{i}$ such that $x_{i}\in B_{i}\subseteq U_{i}$ for $i = 1,\ldots,n$. Hence $(x_{1}, \ldots, x_{n})\in B_{1}\times\cdots\times B_{n} \subseteq U_{1}\times\cdots\times U_{n} \subseteq U$.

		      Thus $\mathscr{B}$ is a basis for the product topology on $X_{1}\times\cdots\times X_{n}$.
		\item Let ${(S_{1}\times \cdots \times S_{n})}_{s}$ be $S_{1}\times \cdots \times S_{n}$ with the subspace topology and ${(S_{1}\times \cdots \times S_{n})}_{p}$ be $S_{1}\times\cdots\times S_{n}$ with the product topology.

		      For every $i=1,\ldots,n$, define $f_{i}: {(S_{1}\times\cdots\times S_{n})}_{s}\to S_{i}$ by $f(x_{1}, \ldots, x_{n}) = x_{i}$. Let $\operatorname{Id}_{ps}: {(S_{1}\times\cdots S_{n})}_{p} \to {(S_{1}\times\cdots\times S_{n})}_{s}$ be the identity map. The following diagram (see~\ref{fig:3.32-e-1}) commutes.
		      \begin{figure}[htp]
			      \renewcommand{\thefigure}{3.32-e-1}
			      \centering
			      \begin{tikzpicture}[every edge/.style = {draw, -latex, thick}]
				      \matrix (m) [matrix of math nodes, row sep=3em, column sep=3em] {
					      {(S_{1}\times\cdots\times S_{n})}_{p} & {(S_{1}\times\cdots\times S_{n})}_{s} & X_{1}\times\cdots\times X_{n} \\
					                                                                                                & S_{i}                                                                                     & X_{i}                         \\
				      };
				      \path[->] (m-1-1) edge node[below] {$\operatorname{Id}_{ps}$} (m-1-2);
				      \path[right hook->] (m-1-2) edge node[below] {$\iota_{Ss}$} (m-1-3);
				      \path[->] (m-1-3) edge node[right] {$\pi^{X_{1}\times\cdots\times X_{n}}_{X_{i}}$} (m-2-3);
				      \path[->] (m-1-2) edge node[right] {$f_{i}$} (m-2-2);
				      \path[right hook->] (m-2-2) edge node[below] {$\iota_{S_{i}}$} (m-2-3);
				      \path[->] (m-1-1) edge node[below left] {$\pi^{{(S_{1}\times\cdots\times S_{n})}_{p}}_{S_{i}}$} (m-2-2);
				      \path[->] (m-1-2) edge [bend right=45] node[below] {$\operatorname{Id}_{ps}^{-1}$} (m-1-1);
				      \path[->] (m-1-1) edge [bend left=60] node[below] {$\iota_{Sp}$} (m-1-3);
			      \end{tikzpicture}
			      \caption{}\label{fig:3.32-e-1}
		      \end{figure}

		      For every $i=1,\ldots,n$, $\iota_{S_{p}}\circ \pi^{X_{1}\times\cdots\times X_{n}}_{X_{i}} = \iota_{S_{i}}\circ \pi^{{(S_{1}\times\cdots\times S_{n})}_{p}}_{S_{i}}$, and $\pi^{{(S_{1}\times\cdots\times S_{n})}_{p}}_{S_{i}}, \iota_{S_{i}}, \pi^{X_{1}\times\cdots\times X_{n}}_{X_{i}}$ are continuous. By the characteristic property of the product topology, it follows that $\iota_{S_{p}}$ is continuous. On the other hand, $\iota_{Ss}\circ \operatorname{Id}_{ps} = \iota_{Sp}$ is continuous, so by the characteristic property of the subspace topology, it follows that $\operatorname{Id}_{ps}$ is continuous.

		      For every $i=1,\ldots,n$, $\pi^{X_{1}\times\cdots\times X_{n}}_{X_{i}}\circ \iota_{Ss} = \iota_{S_{i}}\circ f_{i}$, and $\pi^{X_{1}\times\cdots\times X_{n}}_{X_{i}}, \iota_{Ss}, \iota_{S_{i}}$ are continuous. By the characteristic property of the subspace topology, it follows that $f_{i}$ is continuous for every $i=1,\ldots,n$. Therefore, for every $i=1,\ldots,n$, $f_{i} = \pi^{{(S_{1}\times\cdots\times S_{n})}_{p}}_{S_{i}}\circ \operatorname{Id}_{ps}^{-1}$ is continuous, so by the characteristic property of the product topology, it follows that $\operatorname{Id}^{-1}_{ps}$ is continuous.

		      Hence $\operatorname{Id}_{ps}$ is bijective and bicontinuous, which means it is a homeomorphism. Thus ${(S_{1}\times\cdots\times S_{n})}_{p}$ and ${(S_{1}\times\cdots\times S_{n})}_{s}$ are homeomorphic, so the product topology and the subspace topology on $S_{1}\times\cdots\times S_{n}\subseteq X_{1}\times\cdots\times X_{n}$ are equal.
		\item Let $(x_{1}, \ldots, x_{n})$ and $(y_{1}, \ldots, y_{n})$ be two distinct points of $X_{1}\times \cdots\times X_{n}$. Because they are distinct, there is at least an index $k$ such that $x_{k}\ne y_{k}$. Since $X_{k}$ is Hausdorff, $x_{k}$ and $y_{k}$ are separated by two neighborhoods $U_{k}$ and $V_{k}$. Moreover, $X_{1}\times\cdots \times U_{k}\times\cdots\times X_{n}$ is a neighborhood of $(x_{1}, \ldots, x_{n})$, $X_{1}\times\cdots \times V_{k}\times\cdots\times X_{n}$ is a neighborhood of $(y_{1}, \ldots, y_{n})$ and
		      \[
			      (X_{1}\times\cdots \times U_{k}\times\cdots\times X_{n}) \cap (X_{1}\times\cdots \times V_{k}\times\cdots\times X_{n}) = X_{1}\times\cdots \times \varnothing\times\cdots\times X_{n} = \varnothing
		      \]

		      so $(x_{1}, \ldots, x_{n})$ and $(y_{1}, \ldots, y_{n})$ are separated by some of their neighborhoods. Because the two distinct points are arbitrary, we conclude that $X_{1}\times\cdots\times X_{n}$ is Hausdorff.
		\item Let $(x_{1}, \ldots, x_{n})$ be a point of $X_{1}\times\cdots\times X_{n}$ and $U$ be a neighborhood of $(x_{1}, \ldots, x_{n})$. Because $U$ is open in $X_{1}\times\cdots\times X_{n}$, there are open sets $U_{i}\subseteq X_{i}$ such that $(x_{1}, \ldots, x_{n})\in U_{1}\times\cdots\times U_{n}\subseteq U$. Let $\mathscr{B}_{x_{i}}$ be a countable neighborhood basis of $x_{i}$ in $X_{i}$ (the existence of such a basis is guaranteed by first countablility of $X_{i}$). Because $X_{i}$ is first countable and $U_{i}$ is a neighborhood of $x_{i}$, there is $B_{i}\in \mathscr{B}_{x_{i}}$ such that $x_{i}\in B_{i}\subseteq U_{i}$. Therefore $(x_{1}, \ldots, x_{n})\in B_{1}\times\cdots\times B_{n}\subseteq U_{1}\times\cdots\times U_{n}\subseteq U$. So the set $\{ B_{1}\times\cdots\times B_{n}: B_{i}\in\mathscr{B}_{x_{i}} \} = \mathscr{B}_{x_{1}}\times\cdots\times\mathscr{B}_{x_{n}}$ is a neighborhood basis of $(x_{1}, \ldots, x_{n})$ in $X_{1}\times\cdots\times X_{n}$. On the other hand, the Cartesian product of finitely many countable sets is countable, so $\mathscr{B}_{x_{1}}\times\cdots\times\mathscr{B}_{x_{n}}$ is countable. Thus $X_{1}\times\cdots\times X_{n}$ is first countable.
		\item For every $i=1,\ldots,n$, let $\mathscr{B}_{i}$ be a countable basis of $X_{i}$ (it exists because $X_{i}$ is second countable). It follows from part (d) that $\mathscr{B}_{1}\times\cdots\times \mathscr{B}_{n}$ is a basis for the product topology on $X_{1}\times\cdots\times X_{n}$. Because the Cartesian product of finitely many countable sets is countable, we conclude that $\mathscr{B}_{1}\times\cdots\times \mathscr{B}_{n}$ is countable. Hence $X_{1}\times\cdots\times X_{n}$ is second countable.
	\end{enumerate}
\end{proof}

\begin{exercise}{3.34}
	Suppose $f_{1}, f_{2}: X\to \mathbb{R}$ are continuous functions. Their \textbf{pointwise sum} $f_{1} + f_{2}: X\to \mathbb{R}$ and \textbf{pointwise product} $f_{1}f_{2}: X\to \mathbb{R}$ are real-valued functions defined by
	\[
		(f_{1} + f_{2})(x) = f_{1}(x) + f_{2}(x)\qquad (f_{1}f_{2})(x) = f_{1}(x)f_{2}(x).
	\]

	Pointwise sums and products of complex-valued functions are defined similarly. Use the characteristic property of the product topology to show that pointwise sums and products of real-valued or complex-valued continuous functions are continuous.
\end{exercise}

\begin{proof}
	Denote by $\mathbb{F}$ the real field $\mathbb{R}$ or the complex field $\mathbb{C}$.

	\begin{figure}[htp]
		\renewcommand{\thefigure}{3.34-1}
		\centering
		\begin{tikzpicture}[every edge/.style={draw, -latex, thick}]
			\matrix (m) [matrix of math nodes, row sep=3em, column sep=3em] {
				X & \mathbb{F}\times \mathbb{F} & \mathbb{F} \\
				  & \mathbb{F}                  & \mathbb{F} \\
			};
			\path[->] (m-1-1) edge node[above] {$f$} (m-1-2);
			\path[->] (m-1-2) edge node[right] {$\pi_{1}$} (m-2-2);
			\path[->] (m-1-2) edge node[above] {$+$} (m-1-3);
			\path[->] (m-1-2) edge node[above right] {$\pi_{2}$} (m-2-3);
			\path[->] (m-1-1) edge node[below left] {$f_{1}$} (m-2-2);
			\path[->] (m-1-1) edge [bend right=90] node[below left] {$f_{2}$} (m-2-3);
		\end{tikzpicture}
		\begin{tikzpicture}[every edge/.style={draw, -latex, thick}]
			\matrix (m) [matrix of math nodes, row sep=3em, column sep=3em] {
				X & \mathbb{F}\times \mathbb{F} & \mathbb{F} \\
				  & \mathbb{F}                  & \mathbb{F} \\
			};
			\path[->] (m-1-1) edge node[above] {$f$} (m-1-2);
			\path[->] (m-1-2) edge node[right] {$\pi_{1}$} (m-2-2);
			\path[->] (m-1-2) edge node[above] {$\cdot$} (m-1-3);
			\path[->] (m-1-2) edge node[above right] {$\pi_{2}$} (m-2-3);
			\path[->] (m-1-1) edge node[below left] {$f_{1}$} (m-2-2);
			\path[->] (m-1-1) edge [bend right=90] node[below left] {$f_{2}$} (m-2-3);
		\end{tikzpicture}
		\caption{}\label{fig:3.34-1}
	\end{figure}

	The product map $f_{1}\times f_{2}: X\times X\to \mathbb{F}^{2}$ defined by $f_{1}\times f_{2}(x, y) = (f_{1}(x), f_{2}(y))$ is continuous. Define $f: X\to \mathbb{F}\times\mathbb{F}$ by $f(x) = (f_{1}(x), f_{2}(x))$, then the following diagrams (see~\ref{fig:3.34-1}) commute.

	Because $f_{1}, f_{2}$ are continuous, it follows from the characteristic property of the product topology that $f$ is continuous. The maps $+: \mathbb{F}\times\mathbb{F}\to \mathbb{F}$ and $\cdot: \mathbb{F}\times\mathbb{F}\to \mathbb{F}$ are continuous. Therefore $f_{1} + f_{2} = +\circ f$ and $f_{1}f_{2} = \cdot\circ f$ are continuous. Thus the pointwise sum and pointwise product of continuous real-valued (similarly, complex-valued) functions are continuous.
\end{proof}

\subsection*{Infinite Product}\addcontentsline{toc}{subsection}{Infinite Product}

\begin{exercise}{3.38}
	Prove Theorem 3.37 (Characteristic Property of Infinite Product Spaces).

	Let ${(X_{\alpha})}_{\alpha\in A}$ be an indexed family of topological spaces. For any topological space $Y$, a map $f: Y\to \prod_{\alpha\in A}X_{\alpha}$ is continuous if and only if each of its component functions $f_{\alpha} = \pi_{\alpha}\circ f$ is continuous. The product topology is the unique topology on $\prod_{\alpha\in A}X_{\alpha}$ that satisfies this property.
\end{exercise}

\begin{proof}
	$(\Longrightarrow)$ Suppose $f$ is continuous.

	Let $U_{\alpha}$ be an open set in $X_{\alpha}$. We have ${(\pi_{\alpha}\circ f)}^{-1}(U_{\alpha}) = f^{-1}(\pi_{\alpha}^{-1}(U_{\alpha}))$. On the other hand, $\pi_{\alpha}^{-1}(U_{\alpha}) = \prod_{i\in A}Y_{i}$ where $Y_{i} = X_{i}$ if $i\ne\alpha$ and $Y_{i} = U_{i}$ if $i = \alpha$. By the definition of infinite product topology, $\prod_{i\in A}Y_{i}$ is open. Because $f$ is continuous, $f^{-1}(\pi_{\alpha}^{-1}(U_{\alpha})) = f^{-1}\left(\prod_{i\in A}Y_{i}\right)$ is open. Therefore $f_{\alpha} = \pi_{\alpha}\circ f$ is continuous for every $\alpha\in A$.

	$(\Longleftarrow)$ Suppose $f_{\alpha} = \pi_{\alpha}\circ f$ is continuous for every $\alpha\in A$.

	Let $U$ be an open set of $\prod_{\alpha\in A}X_{\alpha}$. By the definition of infinite product topology, $U$ is an union of product open sets. We need to show that $f^{-1}(U)$ is open in $Y$. However, since the preimage of an arbitrary union is the union of the preimage, so it suffices to prove for the case $U$ is a product set. Let $U = \prod_{\alpha\in A}U_{\alpha}$ where $U_{\alpha} = X_{\alpha}$ for all but finitely many $\alpha\in A$. Let $A'$ be the set of elements $\alpha\in A$ such that $U_{\alpha}\ne X_{\alpha}$, then $A'$ is finite.
	\begin{align*}
		f^{-1}\left(\prod_{\alpha\in A}U_{\alpha}\right) & = f^{-1}\left(\bigcap_{\alpha\in A'}\prod_{i\in A}W_{\alpha, i}\right) \\
		                                                 & = \bigcap_{\alpha\in A'}f^{-1}\left(\prod_{i\in A}W_{\alpha,i}\right)  \\
		                                                 & = \bigcap_{\alpha\in A'}f^{-1}(\pi_{\alpha}^{-1}(U_{\alpha}))          \\
		                                                 & = \bigcap_{\alpha\in A'}{(\pi_{\alpha}\circ f)}^{-1}(U_{\alpha})
	\end{align*}

	where $W_{\alpha, i} = U_{\alpha}$ if $\alpha = i$, $W_{\alpha, i} = X_{\alpha}$ if $\alpha\ne i$. Because $A'$ is finite and ${(\pi_{\alpha}\circ f)}^{-1}(U_{\alpha})$ is open for every $\alpha\in A'$, it follows that $\bigcap_{\alpha\in A'}{(\pi_{\alpha}\circ f)}^{-1}(U_{\alpha})$ is open. Hence $f^{-1}(U)$ is open, so $f$ is continuous.

	\bigskip
	Let $X$ be $\prod_{\alpha\in A}X_{\alpha}$ with the product topology. Let $X'$ be $\prod_{\alpha\in A}X_{\alpha}$ with a topology satisfying the characteristic property. Let $\operatorname{Id}_{X,X'}: X\to X'$ be the identity map. By the characteristic property of infinite product topology, it follows that $\operatorname{Id}_{X,X'}$ and $\operatorname{Id}_{X,X'}^{-1}$ are continuous. Moreover, $\operatorname{Id}_{X,X'}$ is bijective, so $\operatorname{Id}_{X,X'}$ is a homeomorphism. Thus $X$ and $X'$ are equal, which means the product topology on $\prod_{\alpha\in A}X_{\alpha}$ is the only topology satisfying the characteristic property.
\end{proof}

\section*{Disjoint Union Spaces}\addcontentsline{toc}{section}{Disjoint Union Spaces}

\begin{exercise}{3.40}
	Show that the disjoint union topology is indeed a topology.
\end{exercise}

\begin{proof}
	Let ${(X_{\alpha})}_{\alpha\in A}$ be an indexed family of nonempty sets.

	$\varnothing, \coprod_{\alpha\in A}X_{\alpha}$ are open by definition.

	Let ${(U_{i})}_{i\in I}$ be an indexed family of subsets of the disjoint union $\coprod_{\alpha\in A}X_{\alpha}$ where $U_{i}\cap X_{\alpha}$ is open in $X_{\alpha}$ for every $i\in I, \alpha\in A$. For every $\alpha\in A$, the set
	\[
		X_{\alpha}\cap \left(\bigcup_{i\in I}U_{i}\right) = \bigcup_{i\in I}(X_{\alpha}\cap U_{i})
	\]

	is open in $X_{\alpha}$. So $\bigcup_{i\in I}U_{i}$ is open.

	Let $U_{1}, \ldots, U_{n}$ be open sets in $\coprod_{\alpha\in A}X_{\alpha}$. If $n = 0$, $\bigcap^{n}_{j=1}U_{j} = \coprod_{\alpha\in A}X_{\alpha}$. If $n > 0$
	\[
		X_{\alpha}\cap \left(\bigcap^{n}_{j=1}U_{j}\right) = \bigcap^{n}_{j=1}(X_{\alpha}\cap U_{j})
	\]

	is open in $X_{\alpha}$ for every $\alpha\in A$. Thus the disjoint union topology is indeed a topology.
\end{proof}

\begin{exercise}{3.43}\label{exercise:3.43}
	Prove Proposition 3.42 (Other Properties of Disjoint Union Spaces).

	Let ${(X_{\alpha})}_{\alpha\in A}$ be an indexed family of topological spaces.
	\begin{enumerate}[label={(\alph*)}]
		\item A subset of $\coprod_{\alpha\in A}X_{\alpha}$ is closed if and only if its intersection with each $X_{\alpha}$ is closed.
		\item Each canonical injection $\iota_{\alpha}: X_{\alpha}\to \coprod_{\alpha\in A}X_{\alpha}$ is a topological embedding and an open and closed map.
		\item If each $X_{\alpha}$ is Hausdorff, then so is $\coprod_{\alpha\in A}X_{\alpha}$.
		\item If each $X_{\alpha}$ is first countable, then so is $\coprod_{\alpha\in A}X_{\alpha}$.
		\item If each $X_{\alpha}$ is second countable and the index set $A$ is countable, then $\coprod_{\alpha\in A}X_{\alpha}$ is second countable.
	\end{enumerate}
\end{exercise}

\begin{proof}
	\begin{enumerate}[label={(\alph*)}]
		\item Let $G$ be a subset of $\coprod_{\alpha\in A}X_{\alpha}$, then its complement $\left(\coprod_{\alpha\in A}X_{\alpha}\right)\smallsetminus G$ is $\coprod_{\alpha\in A}X_{\alpha}$. For every $\alpha\in A$, $X_{\alpha}\cap \left(\left(\coprod_{\alpha\in A}X_{\alpha}\right)\smallsetminus G\right) = X_{\alpha}\smallsetminus (X_{\alpha}\cap G)$.

		      $G$ is closed if and only if its complement is open. The complement of $G$ is open if and only if $X_{\alpha}\smallsetminus (X_{\alpha}\cap G)$ is open in $X_{\alpha}$ for each $\alpha\in A$. Equivalently, the complement of $G$ is open if and only if $X_{\alpha}\cap G$ is closed in $X_{\alpha}$ for each $\alpha\in A$.

		      Thus $G\subseteq \coprod_{\alpha\in A}X_{\alpha}$ is closed if and only if its intersection with $X_{\alpha}$ is closed in $X_{\alpha}$ for each $\alpha\in A$.
		\item The image set of $\iota_{\alpha}$ is $\iota_{\alpha}(X_{\alpha}) = X_{\alpha}^{*} = X_{\alpha}$. So as a map from $X_{\alpha}$ to its image set, $\iota_{\alpha}$ is bijective. Let $V$ be an open subset of $\coprod_{\alpha\in A}X_{\alpha}$, then $\iota_{\alpha}^{-1}(V) = V\cap X_{\alpha}$, which is open in $X_{\alpha}$ (due to the definition of disjoint union topology), so $\iota_{\alpha}$ is continuous. Moreover, for every $U\subseteq X_{\alpha}$, $\iota_{\alpha}(U)\cap X_{\alpha}^{*} = \iota_{\alpha}(U)$ (which is open/closed in $X_{\alpha}$ if $U$ is open/closed in $X_{\alpha}$), and $\iota_{\alpha}(U)\cap X^{*}_{i} = \varnothing$ if $i\ne \alpha$ (which is open and closed in $X_{i}$). For each $\alpha\in A$, $X_{\alpha}$ is both open and closed in $\coprod_{\alpha\in A}X_{\alpha}$. So if $U$ is open/closed in $X_{\alpha}$, $\iota_{\alpha}(U)$ is open/closed in $\coprod_{\alpha\in A}X_{\alpha}$. Thus $\iota_{\alpha}: X_{\alpha}\to \coprod_{\alpha\in A}X_{\alpha}$ is a topological embedding, an open and closed map.
		\item Let $x, y$ be distinct points of $\coprod_{\alpha\in A}X_{\alpha}$. If $x, y$ are in the same $X_{\alpha}$, then $x, y$ are separated by some of their neighborhoods in $X_{\alpha}$ because $X_{\alpha}$ is Hausdorff. If $x\in X_{i}$ and $y\in X_{j}$ and $i\ne j$ then $x$ and $y$ are separated by $X_{i}$ and $X_{j}$. Therefore $\coprod_{\alpha\in A}X_{\alpha}$ is Hausdorff.
		\item Let $x$ be a point of $\coprod_{\alpha\in A}X_{\alpha}$, then there is $\alpha\in A$ such that $x\in X_{\alpha}$. Let $\mathscr{B}_{x}$ be a countable neighborhood basis of $x$ in $X_{\alpha}$ (such a neighborhood basis exists because $X_{\alpha}$ is first countable) and $U$ is a neighborhood of $x$ in $\coprod_{\alpha\in A}X_{\alpha}$, then $X_{\alpha}\cap U$ is a neighborhood of $x$ in $X_{\alpha}$. Since $\mathscr{B}_{x}$ is a countable neighborhood basis of $x$ in $X_{\alpha}$, then $U$ contains an element of $\mathscr{B}_{x}$. Hence $\mathscr{B}_{x}$ is also a countable neighborhood basis of $x$ in $\coprod_{\alpha\in A}X_{\alpha}$, so $\coprod_{\alpha\in A}X_{\alpha}$ is first countable.
		\item For each $\alpha\in A$, let $\mathscr{B}_{\alpha}$ be a countable basis for $X_{\alpha}$ and $\mathscr{B} = \bigcup_{\alpha\in A}\mathscr{B}_{\alpha}$. Let $U$ be an open subset of $\coprod_{\alpha\in A}X_{\alpha}$. We have $U = \bigcup_{\alpha\in A}(U\cap X_{\alpha})$, because $U\cap X_{\alpha}$ is open in $X_{\alpha}$, it is an union of elements of $\mathscr{B}_{\alpha}$, for each $\alpha\in A$. Therefore $U$ is an union of elements of $\mathscr{B}$, so $\mathscr{B}$ is a basis for the disjoint union topology on $\coprod_{\alpha\in A}X_{\alpha}$. $\mathscr{B}$ is a countable union of countable sets, so it is countable. Thus $\coprod_{\alpha\in A}X_{\alpha}$ is second countable.
	\end{enumerate}
\end{proof}

\begin{exercise}{3.44}
	Suppose ${(X_{\alpha})}_{\alpha\in A}$ is an indexed family of nonempty $n$-manifolds. Show that the disjoint union $\coprod_{\alpha\in A}X_{\alpha}$ is an $n$-manifold if and only if $A$ is countable.
\end{exercise}

\begin{proof}
	$(\Longrightarrow)$ Suppose $A$ is countable.

	According to Exercise~\ref{exercise:3.43} (c) and (e), $\coprod_{\alpha\in A}X_{\alpha}$ is Hausdorff and second countable. Let $x$ be a point of $\coprod_{\alpha\in A}X_{\alpha}$ then there is $\alpha\in A$ such that $x\in X_{\alpha}$. Because $X_{\alpha}$ is an $n$-manifolds, there is a neighborhood of $x$ in $X_{\alpha}$ which is homeomorphic to $\mathbb{R}^{n}$. Because the canonical injection $\iota_{\alpha}: X_{\alpha}\to \coprod_{\alpha\in A}X_{\alpha}$ is a topological embedding, an open map, there is a neighborhood of $x$ in $\coprod_{\alpha\in A}X_{\alpha}$ which is homeomorphic to $\mathbb{R}^{n}$. Thus $\coprod_{\alpha\in A}X_{\alpha}$ is an $n$-manifold.

	$(\Longleftarrow)$ Suppose $A$ is not countable.

	Let $\mathscr{B}$ be a basis for the disjoint union topology on $\coprod_{\alpha\in A}X_{\alpha}$. For each $\alpha\in A$, there is $B_{\alpha}\in\mathscr{B}$ such that $B_{\alpha}\subseteq X_{\alpha}$. Because $A$ is not countable, ${\{ B_{\alpha}: \alpha\in A \}}$ is not countable. Moreover, ${\{ B_{\alpha}: \alpha\in A \}}\subseteq \mathscr{B}$, so the basis $\mathscr{B}$ is not countable, which implies that $\coprod_{\alpha\in A}X_{\alpha}$ is not second countable. Hence $\coprod_{\alpha\in A}X_{\alpha}$ is not an $n$-manifold.

	Thus the disjoint union $\coprod_{\alpha\in A}X_{\alpha}$ is an $n$-manifold if and only if $A$ is countable.
\end{proof}

\begin{exercise}{3.45}
	Let $X$ be any space and $Y$ be a discrete space. Show that the Cartesian product $X\times Y$ is equal to the disjoint union $\coprod_{y\in Y}X$, and the product topology is the same as the disjoint union topology.
\end{exercise}

\begin{proof}
	By definition, $X\times Y$ is the set of ordered pairs $(x, y)$ where $x\in X, y\in Y$. Also by definition, $\coprod_{y\in Y}X = \{ (x, y): x\in X \land y\in Y \}$. Hence $X\times Y$ is equal to the disjoint union $\coprod_{y\in Y}X$.

	Let $U$ be an open subset of $\coprod_{y\in Y}X = \bigcup_{y\in Y}(X\times\{y\})$, then $U = \bigcup_{y\in Y}(U_{y}\times \{ y \})$ for some $U_{y}\subseteq X$ is open for each $y\in Y$. Because $Y$ is a discrete space, $\{y\}$ is open for every $y\in Y$. On the other hand, $U_{y}\times\{y\}$ is a product open set for each $y\in Y$, so $U$ is open in $X\times Y$. So the product topology is finer than the disjoint union topology.

	Let $V$ be an open subset of $X\times Y$ then $V$ is an union of product open sets. Let $U_{x}\times W_{y}$ be a product open set in that union. $U_{x}\times W_{y} = \bigcup_{y\in W_{y}}(U_{x}\times\{y\})$ so $U_{x}\times W_{y}$ is open in $\coprod_{y\in Y}X$. Hence $V$ is open in $\coprod_{y\in Y}X$. So the disjoint union topology is finer than the product topology.

	Thus the product topology on $X\times Y$ and the disjoint union topology on $\coprod_{y\in Y}X$ are the same.
\end{proof}

\section*{Quotient Spaces}\addcontentsline{toc}{section}{Quotient Spaces}

\begin{exercise}{3.46}
	Show that the quotient topology is indeed a topology.
\end{exercise}

\begin{proof}
	Let $q: X\to Y$ be a surjective map and $\mathscr{T}$ be a collection of subsets of $Y$ where $V\in\mathscr{T}$ if and only if $q^{-1}(V)$ is open in $X$.

	$q^{-1}(\varnothing) = \varnothing$ and $q^{-1}(Y) = X$ so $\varnothing, Y\in\mathscr{T}$.

	Suppose ${(V_{i})}_{i\in I}$ is an indexed family of elements of $\mathscr{T}$. Because the preimage of an arbitrary union is the union of preimages, it follows that $q^{-1}\left(\bigcup_{i\in I}V_{i}\right) = \bigcup_{i\in I}q^{-1}(V_{i})$. $q^{-1}(V_{i})$ is open in $X$ for every $i\in I$ so $q^{-1}\left(\bigcup_{i\in I}V_{i}\right)$ is open in $X$. Hence $\bigcup_{i\in I}V_{i}\in \mathscr{T}$.

	Suppose $V_{1}, \ldots, V_{n}\in \mathscr{T}$. Because the preimage of an arbitrary intersection is the intersection of preimages, we obtain $q^{-1}\left(\bigcap^{n}_{i=1}V_{i}\right) = \bigcap^{n}_{i=1}q^{-1}(V_{i})$. $q^{-1}(V_{i})$ is open in $X$ for every $i=1,\ldots,n$, so $q^{-1}\left(\bigcap^{n}_{i=1}V_{i}\right)$ is open in $X$, which means $\bigcap^{n}_{i=1}V_{i}\in\mathscr{T}$.

	Thus $\mathscr{T}$ is a topology on $Y$.
\end{proof}

\begin{exercise}{3.55}
	Show that every wedge sum of Hausdorff spaces is Hausdorff.
\end{exercise}

The following proof makes use of saturated sets and their properties, which are introduced in the next section.

\begin{proof}
	Let ${(X_{\alpha})}_{\alpha\in A}$ be an indexed family of Hausdorff spaces. For each $\alpha\in A$, let $x_{\alpha}$ be a point of $X_{\alpha}$. Use these points as base points, we obtain the wedge sum $\bigvee_{\alpha\in A}X_{\alpha} = \left(\coprod_{\alpha\in A}X_{\alpha}\right)/B$ where $B = \set{ x_{\alpha} : \alpha\in A }$. From the definition of wedge sum of topological spaces, every point of $B$ are identified, and the other points remain distinct from each other and from points of $B$.

	Let $q: \coprod_{\alpha\in A}X_{\alpha}\to \bigvee_{\alpha\in A}X_{\alpha}$ be the corresponding quotient map. Let $q(y), q(z)$ be two distinct points of the wedge sum. Exactly one of the three following cases holds

	\textbf{Case 1.} There exists $X_{\alpha}$ such that $y, z \in X_{\alpha}\smallsetminus\set{x_{\alpha}}$.

	From the equivalence relation on $\coprod_{\alpha\in A}X_{\alpha}$, two points $q(y), q(z)$ are distinct points of the wedge sum. Because $X_{\alpha}$ is Hausdorff
	\begin{itemize}
		\item $y$ and $z$ are separated by some disjoint neighborhoods $U_{y}, U_{z}$, respectively
		\item $y$ has a neighborhood $V_{y}$ in $X_{\alpha}$ which doesn't contain $x_{\alpha}$; $z$ has a neighborhood $V_{z}$ in $X_{\alpha}$ which doesn't contain $x_{\alpha}$
	\end{itemize}

	so $U_{y}\cap V_{y}$ and $U_{z}\cap V_{z}$ are disjoint neighborhoods of $y, z$ not containing $x_{\alpha}$. Moreover, $U_{y}\cap V_{y}$ and $U_{z}\cap V_{z}$ are saturated sets with respect to $q$ (because their points are pairwise inequivalent), hence $q(U_{y}\cap V_{y})$ and $q(U_{z}\cap V_{z})$ are disjoint neighborhoods of $q(y), q(z)$.

	\textbf{Case 2.} There are distinct $X_{\alpha}$ and $X_{\beta}$ such that $y\in X_{\alpha}$ and $z\in X_{\beta}$.

	From this assumption, it follows that $y, z\notin B$ (which can be explained using proof by contradiction). Because $X_{\alpha}$ is Hausdorff, $y$ has a neighborhood $V_{y}$ in $X_{\alpha}$ which doesn't contain $x_{\alpha}$. Similarly, $z$ has a neighborhood $V_{z}$ in $X_{\beta}$ which doesn't contain $x_{\beta}$. $V_{y}$ and $V_{z}$ are disjoint saturated neighborhoods of $y$ and $z$, and every pair of points from both neighborhoods is inequivalent, so $q(V_{y})$ and $q(V_{z})$ are disjoint neighborhoods of $q(y), q(z)$.

	\textbf{Case 3.} There is $X_{\gamma}$ such that $y\in X_{\gamma}\smallsetminus\set{x_{\gamma}}$ and $z\in B$.

	Because $X_{\gamma}$ is Hausdorff, $y, z$ are separated by disjoint neighborhoods $U_{y}, U_{z} \subseteq X_{\gamma}$.

	The disjoint union $\coprod_{\alpha\in A} V_{\alpha}$ where
	\begin{equation*}
		V_{\alpha} = \begin{cases}
			U_{z}      & \text{if $\alpha = \gamma$}   \\
			X_{\alpha} & \text{if $\alpha \ne \gamma$}
		\end{cases}
	\end{equation*}

	is a saturated open subset of $\coprod_{\alpha\in A}X_{\alpha}$ with respect to $q$ and it is disjoint from $U_{y}$. On the other hand $q(U_{y})$ is a neighborhood of $y$ because $U_{y}$ is a saturated open set of $X_{\gamma}$. No point of $U_{y}$ is identified with points of $\coprod_{\alpha\in A} V_{\alpha}$ and vice versa, so $q(y)$ and $q(z)$ are separated by disjoint neighborhoods
	\begin{equation*}
		q(U_{y}) \quad\text{and}\quad q\left(\coprod_{\alpha\in A} V_{\alpha}\right)
	\end{equation*}

	respectively.

	Hence, in either cases, $q(y)$ and $q(z)$ are separated by disjoint neighborhoods. Thus every wedge sum of Hausdorff spaces is Hausdorff.
\end{proof}

\subsection*{Recognizing Quotient Maps Between Known Spaces}\addcontentsline{toc}{subsection}{Recognizing Quotient Maps Between Known Spaces}

\begin{exercise}{3.59}\label{exercise:3.59}
	Let $q: X\to Y$ be any map. For a subset $U\subseteq X$, show that the following are equivalent
	\begin{enumerate}[label={(\alph*)}]
		\item $U$ is saturated.
		\item $U = q^{-1}(q(U))$.
		\item $U$ is a union of fibers.
		\item If $x\in U$, then every point $x'\in X$ such that $q(x) = q(x')$ is also in $U$.
	\end{enumerate}
\end{exercise}

\begin{proof}
	Suppose (a) is true, then there is a subset $V\subseteq Y$ such that $U = q^{-1}(V)$. Without loss of generality, we can assume that $V\subseteq q(X)$ (because adding points in $Y\smallsetminus q(X)$ to $V$ doesn't change $q^{-1}(V)$). We always have $U\subseteq q^{-1}(q(U))$, so $q^{-1}(V)\subseteq q^{-1}(q(U))$, which implies $V\subseteq q(U)$. From $U = q^{-1}(V)$, it follows that $q(U) = q(q^{-1}(V))\subseteq V$. Hence $q(U) = V$, which means $U = q^{-1}(q(U))$. So (b) is true.

	Suppose (b) is true. Let $x\in U$ then $q(x)\in q(U)$. So $x\in q^{-1}(\{ q(x) \})\subseteq q^{-1}(q(U)) = U$, which implies every point of $U$ is contained in a fiber of $q$, which is again contained in $U$. Therefore $U$ is a union of fibers of $q$, so (c) is true.

	Suppose (c) is true. If $x\in U$ and $x'\in X$ such that $q(x) = q(x')$, then $x'\in q^{-1}(\{q(x)\})$, which is a fiber of $q$ contained in $U$. Therefore $x'$ is also in $U$, so (d) is true.

	Suppose (d) is true. If $x\in U$, then $x\in q^{-1}(\{ q(x) \})\subseteq U$, which means
	\[
		U = \bigcup_{x\in U}q^{-1}(\{ q(x) \}) = q^{-1}\left(\bigcup_{x\in U}\{ q(x) \}\right) = q^{-1}(q(U))
	\]

	which implies $U$ is saturated. So (a) is true.

	We showed that $(a)\implies (b) \implies (c) \implies (d) \implies (a)$, so (a), (b), (c), (d) are equivalent.
\end{proof}

\begin{exercise}{3.61}\label{exercise:3.61}
	Prove Proposition 3.60.

	A continuous surjective map $q: X\to Y$ is a quotient map if and only if it takes saturated open subsets to open subsets, or saturated closed subsets to closed subsets.
\end{exercise}

\begin{proof}
	$(\Longrightarrow)$ Suppose $q$ is a quotient map.

	Let $U$ be a saturated subset of $X$. By Exercise~\ref{exercise:3.59}, $U = q^{-1}(q(U))$, moreover, preimages are well-behaved with set differences, so
	\[
		X\smallsetminus U = X\smallsetminus q^{-1}(q(U)) = q^{-1}(Y)\smallsetminus q^{-1}(q(U)) = q^{-1}(Y\smallsetminus q(U)).
	\]

	Because $q$ is a quotient map, $U$ is open if and only if $q(U)$ is open. Also because of this, $X\smallsetminus U$ is open if and only if $Y\smallsetminus q(U)$ is open, which means $U$ is closed if and only if $q(U)$ is closed. Hence $q$ takes saturated open subsets to open subsets, saturated closed subsets to closed subsets.

	$(\Longleftarrow)$ Suppose $q$ takes saturated open subsets to open subsets.

	Let $V$ be a subset of $Y$ and $U = q^{-1}(V)$. If $U$ is open then $q(U) = q(q^{-1}(V)) = V$ (because $q$ is surjective) is open, because $q$ takes saturated open subsets to open subsets and $U$ is a saturated open subset of $X$. Conversely, if $V$ is open, then $U = q^{-1}(V)$ is open, because $q$ is continuous. Therefore $q$ is surjective, and $q^{-1}(V)$ is open if and only if $V$ is open, so $q$ is a quotient map.

	$(\Longleftarrow)$ Suppose $q$ takes saturated closed subsets to closed subsets.

	Let $V$ be a subset of $Y$ and $U = q^{-1}(V)$, it follows that $X\smallsetminus U = X\smallsetminus q^{-1}(V) = q^{-1}(Y)\smallsetminus q^{-1}(V) = q^{-1}(Y\smallsetminus V)$. If $U$ is closed, then $q(U) = q(q^{-1}(V)) = V$ is closed, because $q$ takes saturated closed subsets to closed subsets and $U$ is a saturated closed set. If $V$ is closed, then $Y\smallsetminus V$ is open and so is $X\smallsetminus U = q^{-1}(Y\smallsetminus V)$, from which we obtain $U$ is closed. Since $q$ is surjective, $q^{-1}(Y\smallsetminus V)$ is open if and only if $Y\smallsetminus V$ is open, it follows that $q$ is a quotient map.
\end{proof}

\begin{exercise}{3.63}\label{exercise:3.63}
	Prove Proposition 3.62 (Properties of Quotient Maps).

	\begin{enumerate}[label={(\alph*)}]
		\item Any composition of quotient maps is a quotient map.
		\item An injective quotient map is a homeomorphism.
		\item If $q: X\to Y$ is a quotient map, a subset $K\subseteq Y$ is closed if and only if $q^{-1}(K)$ is closed in $X$.
		\item If $q: X\to Y$ is a quotient map and $U\subseteq X$ is a saturated open or closed subset, then the restriction $q\vert_{U}: U\to q(U)$ is a quotient map.
		\item If ${\{ q_{\alpha}: X_{\alpha}\to Y_{\alpha} \}}_{\alpha\in A}$ is an indexed family of quotient maps, then the map $q: \coprod_{\alpha}X_{\alpha} \to \coprod_{\alpha}Y_{\alpha}$ whose restriction to each $X_{\alpha}$ is equal to $q_{\alpha}$ is a quotient map.
	\end{enumerate}
\end{exercise}

\begin{proof}
	\begin{enumerate}[label={(\alph*)}]
		\item Let $q: X\to Y$ and $r: Y\to Z$ be quotient maps, then $q, r$ are surjective and continuous.

		      Because the composition of surjective maps is surjective, and composition of continuous map is continuous, it follows that $r\circ q$ is a continuous surjective map. Let $W$ be a subset of $Z$, we have ${(r\circ q)}^{-1}(W) = q^{-1}(r^{-1}(W))$.

		      If ${(r\circ q)}^{-1}(W)$ is open (it is also saturated), then $U = q^{-1}(r^{-1}(W))$ is open. By Exercise~\ref{exercise:3.61}, together with the surjectivity of $q, r$, we get $q(U) = r^{-1}(W)$ is a saturated open subset of $Y$, and $W = r(q(U))$ is an open subset of $Z$.

		      By Exercise~\ref{exercise:3.61}, we conclude that $r\circ q$ is a quotient map.
		\item Let $q: X\to Y$ be an injective quotient map.

		      Because a quotient map is surjective, it follows that $q$ is bijective. $q$ is continuous, which directly follows from the definition of quotient map. Because $q$ is bijective, $q^{-1}(q(U)) = U$ for every subset $U\subseteq X$, on the other hand $q$ is a quotient map, so $q(U)$ is open if and only if $U$ is open, which means $q$ is an open map. $q$ is bijective, continuous, and open, so $q$ is a homeomorphism.
		\item Because preimages are well-behaved with set differences, $X\smallsetminus q^{-1}(K) = q^{-1}(Y)\smallsetminus q^{-1}(K) = q^{-1}(Y\smallsetminus K)$. Because $q$ is a quotient map, $Y\smallsetminus K$ is open if and only if $q^{-1}(Y\smallsetminus K) = X\smallsetminus q^{-1}(K)$ is open. Therefore $q^{-1}(K)$ is closed if and only if $K$ is closed.
		\item The following proof implicitly uses subspace topologies.

		      Suppose $U\subseteq X$ is a saturated open subset, then $U = q^{-1}(q(U))$. The restriction $q\vert_{U}$ is still surjective and continuous. By Exercise~\ref{exercise:3.61}, $q(U)$ is open in $Y$. Let $W$ be a subset of $q(U)$.

		      If $W$ is open in $q(U)$, then it is also an open in $Y$. ${(q\vert_{U})}^{-1}(W) = q^{-1}(W)$ is open in $X$. Moreover, $q^{-1}(W)\subseteq q^{-1}(q(U)) = U$. Therefore $q^{-1}(W)$ is open in $U$.

		      If ${(q\vert_{U})}^{-1}(W)$ is open (it is also saturated), $(q\vert_{U})({(q\vert_{U})}^{-1}(W)) = (q\vert_{U})(q^{-1}(W)) = q(q^{-1}(W)) = W$ is open in $Y$. $q(U)$ is open in $Y$ and $W\subseteq q(U)\subseteq Y$ is open in $Y$, so $W$ is open in $q(U)$.

		      Suppose $U\subseteq X$ is a saturated closed subset. By Exercise~\ref{exercise:3.61}, $q(U)$ is closed in $Y$. Let $W$ be a subset of $q(U)$.

		      ${(q\vert_{U})}^{-1}(q(U)\smallsetminus W) = U\smallsetminus {(q\vert_{U})}^{-1}(W)$, so analogously $W$ is closed in $q(U)$ if and only if ${(q\vert_{U})}^{-1}(W) = q^{-1}(W)$ is closed in $U$.

		      Hence $q\vert_{U}: U\to q(U)$ is a quotient map.
		\item Let $(y, \alpha)$ be a point of $\coprod_{\alpha\in A}Y_{\alpha}$, then there is $x\in X_{\alpha}$ such that $q_{\alpha}(x) = y$. Therefore $q((x, \alpha)) = (y, \alpha)$, which implies that $q$ is surjective.

		      Let $V$ be a subset of $\coprod_{\alpha\in A}Y_{\alpha}$. We have
		      \[
			      V = \coprod_{\alpha\in A}(V\cap Y_{\alpha})
		      \]

		      and because preimages is well-behaved with arbitrary intersection and arbitrary unions,
		      \[
			      q^{-1}(V) = \coprod_{\alpha\in A}(q^{-1}(V)\cap X_{\alpha}) = \coprod_{\alpha\in A}(q^{-1}(V)\cap q^{-1}(Y_{\alpha})) = \coprod_{\alpha\in A}q^{-1}(V\cap Y_{\alpha}).
		      \]

		      If $V$ is open, then $q_{\alpha}^{-1}(V\cap Y_{\alpha})$ is open for every $\alpha\in A$, so $q^{-1}(V)$ is open in $\coprod_{\alpha\in A}X_{\alpha}$.

		      If $q^{-1}(V)$ is open, then $q^{-1}(V)\cap X_{\alpha} = q^{-1}(V)\cap q^{-1}(Y_{\alpha}) = q^{-1}(V\cap Y_{\alpha}) = q_{\alpha}^{-1}(V\cap Y_{\alpha})$ is open for every $\alpha\in A$. Because $q_{\alpha}$ is a quotient map, and $q_{\alpha}^{-1}(V\cap Y_{\alpha})$ is a saturated open subset of $X_{\alpha}$, it follows that $V\cap Y_{\alpha}$ is open in $Y_{\alpha}$ for every $\alpha\in A$. Therefore $V$ is open in $\coprod_{\alpha\in A}Y_{\alpha}$.

		      So $V$ is open in $\coprod_{\alpha\in A}Y_{\alpha}$ if and only if $q^{-1}(V)$ is open in $\coprod_{\alpha\in A}X_{\alpha}$. $q$ is also surjective, so $q$ is a quotient map.
	\end{enumerate}
\end{proof}

\subsection*{The Characteristic Property and Uniqueness}\addcontentsline{toc}{subsection}{The Characteristic Property and Uniqueness}

\begin{exercise}{3.72}
	Prove Theorem 3.71 (Uniqueness of the Quotient Topology).

	Given a topological space $X$, a set $Y$, and a surjective map $q: X \to Y$, the quotient topology is the only topology on $Y$ for which the characteristic property holds.
\end{exercise}

\begin{proof}
	Let $\mathscr{T}$ be the quotient topology on $Y$ induced by the map $q$, and $\mathscr{T}'$ be a topology on $Y$ for which the characteristic property holds. Let $\operatorname{Id}: Y\to Y'$ be the identity map, where $Y'$ is $Y$ with the topology $\mathscr{T}'$. The following diagram (see~\ref{fig:3.72}) commutes.
	\begin{figure}[htp]
		\renewcommand{\thefigure}{3.72}
		\centering
		\begin{tikzpicture}[every edge/.style={draw, -latex, thick}]
			\matrix (m) [matrix of math nodes, row sep=4em, column sep=4em] {
				X &    \\
				Y & Y' \\
			};

			\path[->] (m-1-1) edge node[left] {$q$} (m-2-1);
			\path[->] (m-1-1) edge node[above right] {$q$} (m-2-2);
			\path[->] (m-2-1) edge node[below] {$\operatorname{Id}$} (m-2-2);
		\end{tikzpicture}
		\caption{}\label{fig:3.72}
	\end{figure}

	By the characteristic property of the quotient topology, $q: X\to Y$ and $q: X\to Y'$ are continuous. On the other hand, $q = \operatorname{Id}\circ q$ and $q = \operatorname{Id}^{-1}\circ q$ are continuous. Also by the characteristic property, $\operatorname{Id}$ and $\operatorname{Id}^{-1}$ are continuous. Hence $Y$ and $Y'$ are homeomorphic, which implies the quotient topology is the only topology on $Y$ for which the characteristic property holds.
\end{proof}

\section*{Adjunction Spaces}\addcontentsline{toc}{section}{Adjunction Spaces}

No exercises.

\begin{prop}{3.77}\label{prop:3.77}
	Let $X \cup_{f} Y$ be an adjunction space, and let $q: X \amalg Y \to X \cup_{f} Y$ be the associated quotient map.
	\begin{enumerate}[label={(\alph*)}]
		\item The restriction of $q$ to $X$ is a topological embedding, whose image set $q(X)$ is a closed subspace of $X \cup_{f} Y$.
		\item The restriction of $q$ to $Y\smallsetminus A$ is a topological embedding, whose image set $q(Y\smallsetminus A)$ is an open subspace of $X \cup_{f} Y$.
		\item As a set, $X \cup_{f} Y$ is the disjoint union of $q(X)$ and $q(Y\smallsetminus A)$.
	\end{enumerate}
\end{prop}

\begin{note}
	In other documents, an adjunction space is defined for any subset $A$ of $Y$, not just closed subsets. However, in this Proposition, the hypothesis ``$A\subseteq Y$ is closed'' is important, as shown in the following proof.
\end{note}

\begin{proof}
	\begin{enumerate}[label={(\alph*)}]
		\item First, we show that $q\vert_{X}$ is a closed map.

		      Let $B$ be a closed subset of $X$. We will prove that $q^{-1}(q(B))$ is closed in $X \amalg Y$ by showing that the intersection of this set with $X$ (and $Y$) is closed in $X$ (and $Y$). From the definition of $q$, $q^{-1}(q(B)) \cap X = B$, which is closed in $X$. $q^{-1}(q(B)) \cap Y = f^{-1}(B)$, which is closed in $A$ because $f: A\to X$ is continuous. Because $A$ is closed in $Y$, it follows that $f^{-1}(B)$ is closed in $Y$. Hence $q^{-1}(q(B)) \cap Y$ is closed in $Y$. Since the intersections of $q^{-1}(q(B)) \cap X$ and $q^{-1}(q(B)) \cap Y$ are closed in $X$ and $Y$, correspondingly, we conclude that $q^{-1}(q(B))$ is closed in $X \amalg Y$. Therefore $q(B)$ is closed (according to the definition of a quotient map), which means $q\vert_{X}$ is a closed map. Hence $q(X)$ is a closed subspace of $X \cup_{f} Y$.

		      Because $q\vert_{X}$ is injective, continuous and closed, it is also a topological embedding.
		\item $Y\smallsetminus A$ is a saturated open subset of $X \amalg Y$, so $q\vert_{Y\smallsetminus A}: Y\smallsetminus A \to q(Y\smallsetminus A)$ is a quotient map (Exercise~\ref{exercise:3.63}). $q\vert_{Y\smallsetminus A}$ is also a bijection so it is a homeomorphism. Hence the restriction of $q$ to $Y\smallsetminus A$ is a topological embedding.

		      $Y\smallsetminus A$ is a saturated open subset under $q$ so its image $q(Y\smallsetminus A)$ is open in $X \cup_{f} Y$.
		\item $X = (X\smallsetminus f(A)) \cup f(A)$, $Y = (Y\smallsetminus A) \cup A$, $q(f(A)) = q(A)$. Therefore
		      \begin{align*}
			      X \cup_{f} Y & = q(X\amalg Y)                                                             & \text{(as $q$ is surjective)} \\
			                   & = q(X\smallsetminus f(A)) \cup q(f(A)) \cup q(Y\smallsetminus A) \cup q(A)                                 \\
			                   & = q(X\smallsetminus f(A)) \cup q(f(A)) \cup q(Y\smallsetminus A)           & \text{(as $q(f(A)) = q(A)$)}  \\
			                   & = q(X) \cup q(Y\smallsetminus A)
		      \end{align*}

		      On the other hand, the sets $q(X)$ and $q(Y\smallsetminus A)$ are disjoint (as the elements of $Y\smallsetminus A$ are not identified by $q$ to any element), we conclude that $X \cup_{f} Y$ is the union of disjoint sets $q(X)$ and $q(Y\smallsetminus A)$.
	\end{enumerate}
\end{proof}

\begin{prop}{3.81}
	Every $n$-manifold with boundary is homeomorphic to a closed subset of an $n$-manifold without boundary.
\end{prop}

\begin{proof}
	Let $M$ be an $n$-manifold with boundary. Denote by $h: \partial M \to \partial M$ the identity map and $q: M\amalg M \to M \cup_{h} M$ the associated quotient map. By Theorem 3.79, $D(M) = M \cup_{h} M$ is an $n$-manifold without boundary. By the proof of Proposition~\ref{prop:3.77}, $q\vert_{M}$ is continuous, closed, injective, hence $M$ is homeomorphic to $q(M)$, which is a closed subset of $D(M)$ (which is an $n$-manifold without boundary).
\end{proof}

\section*{Topological Groups and Group Actions}\addcontentsline{toc}{section}{Topological Groups and Group Actions}

\begin{exercise}{3.83}
	Verify that each of the above examples is a topological group.
\end{exercise}

\begin{proof}
	\begin{enumerate}[label={(\alph*)}]
		\item the real line $\mathbb{R}$ with its additive group structure and Euclidean topology

		      $m(x, y) = x + y$. Let $\openinterval{a, b}$ be a nonempty open interval of $\mathbb{R}$. We have
		      \[
			      m^{-1}(\openinterval{a,b}) = \bigcup_{a < c < b}\{ (x, y): x + y = c \}.
		      \]

		      Let $(x_{0}, y_{0})$ be a point of $m^{-1}(\openinterval{a,b})$ and $x_{0} + y_{0} = c$. Let $\delta = \min\left\{ \frac{c-a}{2}, \frac{b-c}{2} \right\}$. For every $(x, y)\in \openinterval{x_{0} - \delta, x_{0} + \delta}\times\openinterval{y_{0} - \delta, y_{0} + \delta}$,
		      \begin{align*}
			      x + y & < (x_{0} + y_{0}) + 2\delta = c + 2\delta \leq b \\
			      x + y & > (x_{0} + y_{0}) - 2\delta = c - 2\delta \geq a
		      \end{align*}

		      so the neighborhood $\openinterval{x_{0} - \delta, x_{0} + \delta}\times\openinterval{y_{0} - \delta, y_{0} + \delta}$ of $(x_{0}, y_{0})$ is contained in $m^{-1}(\openinterval{a, b})$. Hence $m^{-1}(\openinterval{a, b})$ is open in $\mathbb{R}\times\mathbb{R}$.

		      On the other hand, $i^{-1}(\openinterval{a, b}) = \openinterval{-b, -a}$, which is open in $\mathbb{R}$.

		      Let $U$ be an open subset of $\mathbb{R}$ then $U$ is an union of open intervals, so $m^{-1}(U)$ is open in $\mathbb{R}\times\mathbb{R}$, and $i^{-1}(U)$ is open in $\mathbb{R}$. Therefore $m, i$ are continuous, and it follows that the real line with the additive group structure and Euclidean topology is a topological group.
		\item the set $\mathbb{R}^{*} = \mathbb{R}\smallsetminus\{0\}$ of nonzero real numbers under multiplication, with the Euclidean topology

		      First, we prove that $\cdot: \mathbb{R}\times\mathbb{R}\to\mathbb{R}$ is continuous. Let $(x_{0}, y_{0})\in\mathbb{R}\times\mathbb{R}$. For every $\varepsilon > 0$, let $\delta = -\frac{\abs{x_{0}}+\abs{y_{0}}}{2} + \sqrt{\varepsilon + {\left(\frac{\abs{x_{0}}+\abs{y_{0}}}{2}\right)}^{2}}$ then for every $(x, y)$ such that $\sqrt{{(x - x_{0})}^{2} + {(y - y_{0})}^{2}} < \delta$,
		      \begin{align*}
			      \abs{xy - x_{0}y_{0}} & = \abs{(x - x_{0})(y - y_{0}) + x_{0}(y - y_{0}) + y_{0}(x - x_{0})}                        \\
			                            & \leq \abs{(x - x_{0})(y - y_{0})} + \abs{x_{0}}\abs{y - y_{0}} + \abs{y_{0}}\abs{x - x_{0}} \\
			                            & \leq \frac{{(x - x_{0})}^{2} + {(y - y_{0})}^{2}}{2} + \delta(\abs{x_{0}} + \abs{y_{0}})    \\
			                            & < \delta^{2} + \delta(\abs{x_{0}} + \abs{y_{0}}) = \varepsilon
		      \end{align*}

		      Hence $\cdot: \mathbb{R}\times\mathbb{R}\to \mathbb{R}$ is continuous. Therefore $m: \mathbb{R}^{*}\times\mathbb{R}^{*}\to \mathbb{R}^{*}$ is also continuous.

		      A basis for the topology on $\mathbb{R}^{*}$ is the collection of open intervals of which endpoints are both positive, or negative.
		      \[
			      i^{-1}(\openinterval{a, b}) = \begin{cases}
				      \openinterval{\frac{1}{b}, \frac{1}{a}}   & \text{if $a, b > 0$} \\
				      \openinterval{\frac{-1}{a}, \frac{-1}{b}} & \text{if $a, b < 0$}
			      \end{cases}
		      \]

		      Hence $i: \mathbb{R}^{*}\to \mathbb{R}^{*}$ is continuous. Thus $\mathbb{R}^{*}$ under multiplication with the Euclidean topology is a topological group.
		\item the set $\mathbb{C}^{*} = \mathbb{C}\smallsetminus\{0\}$ of nonzero complex numbers under complex multiplication, with the Euclidean topology

		      We will prove that $\cdot: \mathbb{C}\times\mathbb{C}\to\mathbb{C}$ is continuous.

		      Let $(z_{0}, w_{0})\in \mathbb{C}\times\mathbb{C}$. For every $\varepsilon > 0$, let $\delta = -\frac{\abs{z_{0}} + \abs{w_{0}}}{2} + \sqrt{\varepsilon + {\left(\frac{\abs{z_{0}} + \abs{w_{0}}}{2}\right)}^{2}}$, then for every $(z, w)\in B_{\delta}((z_{0}, w_{0}))$
		      \begin{align*}
			      \abs{zw - z_{0}w_{0}} & = \abs{(z - z_{0})(w - w_{0}) + z_{0}(w - w_{0}) + w_{0}(z - z_{0})}                                                                   \\
			                            & \leq \abs{(z - z_{0})(w - w_{0})} + \abs{z_{0}}\abs{w - w_{0}} + \abs{w_{0}}\abs{z - z_{0}}                                            \\
			                            & \leq \frac{\abs{z - z_{0}}^{2} + \abs{w - w_{0}}^{2}}{2} + (\abs{z_{0}} + \abs{w_{0}})\sqrt{\abs{z - z_{0}}^{2} + \abs{w - w_{0}}^{2}} \\
			                            & < \frac{\delta^{2}}{2} + \delta(\abs{z_{0}} + \abs{w_{0}})                                                                             \\
			                            & < \delta^{2} + \delta(\abs{z_{0}} + \abs{w_{0}}) = \varepsilon
		      \end{align*}

		      so $\cdot: \mathbb{C}\times\mathbb{C}\to\mathbb{C}$ is continuous. Hence $m: \mathbb{C}^{*}\times\mathbb{C}^{*}\to \mathbb{C}^{*}$ is continuous.

		      Let $z_{0}\in\mathbb{C}^{*}$ and $B_{\varepsilon}(1/z_{0})\subseteq \mathbb{C}^{*}$. Let $\delta = \frac{\varepsilon\abs{z_{0}}^{2}}{1 + \varepsilon\abs{z_{0}}}$, then for every $z\in B_{\delta}(z_{0})$, $\abs{z}\geq \abs{z_{0}} - \abs{z - z_{0}} > \abs{z_{0}} - \delta > 0$, and
		      \begin{align*}
			      \abs{i(z) - i(z_{0})} & = \abs{\frac{1}{z} - \frac{1}{z_{0}}}                            \\
			                            & = \frac{\abs{z - z_{0}}}{\abs{z}\abs{z_{0}}}                     \\
			                            & \leq \frac{\delta}{(\abs{z_{0}} - \abs{z - z_{0}})\abs{z_{0}}}   \\
			                            & < \frac{\delta}{(\abs{z_{0}} - \delta)\abs{z_{0}}} = \varepsilon
		      \end{align*}

		      so $i: \mathbb{C}^{*}\to \mathbb{C}^{*}$ is continuous. Thus $\mathbb{C}^{*}$ under multiplication with the Euclidean topology is a topological group.
		\item the general linear group $\operatorname{GL}(n, \mathbb{R})$, which is the set of $n\times n$ invertible real matrices under matrix multiplication, with the subspace topology obtained from $\mathbb{R}^{n^{2}}$ (where we identify an $n\times n$ matrix with a point in $\mathbb{R}^{n^{2}}$ by using the matrix entries as coordinates)

		      $\cdot: \mathbb{R}^{n^{2}}\times \mathbb{R}^{n^{2}}\to \mathbb{R}^{n^{2}}$ is a continuous map (because each entries of the product matrix is a polynomial of entries of the component matrices), so the restriction $m: \operatorname{GL}(n, \mathbb{R})\times\operatorname{GL}(n, \mathbb{R})\to \operatorname{GL}(n, \mathbb{R})$ is continuous.

		      $\det: \mathbb{R}^{n^{2}}\to \mathbb{R}$ defined by $\det A = \sum_{\sigma\in S_{n}}\operatorname{sign}(\sigma)A_{\sigma(1),1}\cdots A_{\sigma(n),n}$ is continuous, because $\det$ is the pointwise sum of continuous real-valued functions, namely, $A\mapsto \operatorname{sign}(\sigma)A_{\sigma(1),1}\cdots A_{\sigma(n),n}$, which are pointwise products of continuous real-valued functions. $\mathbb{R}\smallsetminus\{0\}$ is open, so $\operatorname{GL}(n, \mathbb{R}) = \det^{-1}(\mathbb{R}\smallsetminus\{0\})$ is open in $\mathbb{R}^{n^{2}}$. Consequently, the adjugate map $\operatorname{adj}: \mathbb{R}^{n^{2}}\to \mathbb{R}^{n^{2}}$ is also continuous (because entries of an adjugate matrix is defined by determinants).

		      Let $A\in \operatorname{GL}(n, \mathbb{R})$, then $i(A) = A^{-1} = \frac{1}{\det A}\operatorname{adj}A$. $A\mapsto \det A\mapsto 1/\det A$ is continuous, because $A$ is invertible and $\det$ is continuous, $\operatorname{adj}: \mathbb{R}^{n^{2}}\to \mathbb{R}^{n^{2}}$ is also continuous, so $i: \operatorname{GL}(n, \mathbb{R})\to \operatorname{GL}(n, \mathbb{R})$ is continuous.

		      Thus the real general linear group is a topological group.
		\item the complex general linear group $\operatorname{GL}(n, \mathbb{C})$, the set of $n\times n$ invertible complex matrices under matrix multiplication

		      The proof is similar to part (d).
		\item any group whatsoever, with the discrete topology (any such group is called a discrete group)

		      Let $G$ be a group with the discrete topology, then the product topology on $G\times G$ is also discrete, so $m: G\times G\to G$ is continuous. $G$ has the discrete topology, so $i: G\to G$ is also continuous. Hence every group with the discrete topology is a topological group.
	\end{enumerate}
\end{proof}

\begin{exercise}{3.85}
	Prove Proposition 3.84.

	Any subgroup of a topological group is a topological group with the subspace topology. Any finite product of topological groups is a topological group with the direct product group structure and the product topology.
\end{exercise}

\begin{proof}
	Let $H$ be a subgroup of the topological group $G$.

	$m: G\times G\to G$ is continuous, so $m\vert_{H\times H}: H\times H\to H$ is also continuous (the subspace topology and the product topology on $H\times H\subseteq G\times G$ are the same). $i: G\to G$ is continuous. Because $H$ is closure under multiplication, $i(H) = H$, and $i\vert_{H}: H\to H$ is also continuous. So $H$ with the subspace topology is a topological subgroup of $G$.

	Let $G_{1}, G_{2}, \ldots, G_{n}$ be topological groups. The following diagram (see~\ref{fig:3.85}) commutes.
	\begin{figure}[htp]
		\renewcommand{\thefigure}{3.85}
		\centering
		\begin{tikzpicture}[every edge/.style={draw,-latex,thick}]
			\matrix (m) [matrix of math nodes,row sep=3em,column sep=3em] {
				(G_{1}\times\cdots\times G_{n})\times (G_{1}\times\cdots\times G_{n}) & G_{1} \times\cdots\times G_{n} \\
				G_{i}\times G_{i}                                                     & G_{i}                          \\
			};
			\path[->] (m-1-1) edge[above] node {$m$} (m-1-2);
			\path[->] (m-1-1) edge[left] node {$f_{i}$} (m-2-1);
			\path[->] (m-1-2) edge[left] node {$\pi_{i}$} (m-2-2);
			\path[->] (m-2-1) edge[below] node {$m_{i}$} (m-2-2);
		\end{tikzpicture}
		\caption{}\label{fig:3.85}
	\end{figure}

	Define $f_{i}: (G_{1}\times\cdots\times G_{n})\times (G_{1}\times\cdots\times G_{n})\to G_{i}\times G_{i}$ for $i = 1, \ldots, n$ by
	\[
		f_{i}((a_{1}, \ldots, a_{n}), (b_{1}, \ldots, b_{n})) = (a_{i}, b_{i})
	\]

	$f_{i}$ is continuous, because the preimage of a product open set $B\times B'\subseteq G_{i}\times G_{i}$ is
	\[
		(G_{1}\times\cdots\times G_{i-1}\times B\times G_{i+1}\times\cdots\times G_{n})\times (G_{1}\times\cdots\times G_{i-1}\times B'\times G_{i+1}\times\cdots\times G_{n}).
	\]

	$m_{i}: G_{i}\times G_{i}\to G_{i}$ is the operation on $G_{i}$, which is continuous, for $i = 1, \ldots, n$. $\pi_{i}\circ m = m_{i}\circ f_{i}$ is continuous for $i = 1, \ldots, n$, so $m$ is continuous, by the characteristic property of the product topology. Hence $m$ is continuous.

	$i: G_{1}\times\cdots\times G_{n}\to G_{1}\times\cdots\times G_{n}$. Let $B_{1}\times\cdots\times B_{n}$ be a product open set in $G_{1}\times\cdots\times G_{n}$, then $i^{-1}(B_{1}\times\cdots\times B_{n}) = i_{1}^{-1}(B_{1})\times\cdots\times i_{n}^{-1}(B_{n})$, which is open because $i_{i}^{-1}(B_{i})\subseteq G_{i}$ is open for $i = 1,\ldots,n$, since $G_{i}$ is a topological group. Hence $i$ is continuous.

	Therefore $G_{1}\times\cdots\times G_{n}$ is a topological group. Thus the finite product of topological groups with the direct product and the product topology is a topological group.
\end{proof}

\subsection*{Group Actions}\addcontentsline{toc}{subsection}{Group Actions}

No exercises.

\section*{Problems}\addcontentsline{toc}{section}{Problems}

\begin{problem}{3-1}
Suppose $M$ is an $n$-dimensional manifold with boundary. Show that $\partial M$ is an $(n-1)$-manifold (without boundary) when endowed with the subspace topology. You may use without proof the fact that $\operatorname{Int}M$ and $\partial M$ are disjoint.
\end{problem}

\begin{proof}
	This proof assumes that $\operatorname{Int}M$ and $\partial M$ are disjoint.

	Let $p$ be a point of $\partial M$ then $p$ is in the domain of a boundary chart $(U, \varphi)$ that takes $p$ to $\partial\mathbb{H}^{n}$. Due to the definition of boundary point of a manifold with boundary, $\varphi(p) \in \partial\mathbb{H}^{n}$. Let's use the subspace topology on $\partial\mathbb{H}^{n}$. $U$ is a neighborhood of $p$ in $M$, and $\varphi: U \to \varphi(U)$ is a homeomorphism, so $\varphi(U)$ is open in $\mathbb{H}^{n}$, which implies $\varphi(U) \cap \partial\mathbb{H}^{n}$ is open in $\partial\mathbb{H}^{n}$.

	Let $V = \varphi^{-1}(\varphi(U) \cap \partial\mathbb{H}^{n})$ then $V$ is homeomorphic to $\varphi(U) \cap \partial\mathbb{H}^{n}$ and $V$ is also a neighborhood of $p$. On the other hand, $\partial\mathbb{H}^{n}$ is homeomorphic to $\mathbb{R}^{n-1}$, so $V$ is a neighborhood of $p$ which is homeomorphic to an open subset of $\mathbb{R}^{n-1}$. Therefore $\partial M$ is an $(n-1)$-manifold without boundary.
\end{proof}

\begin{problem}{3-2}
Suppose $X$ is a topological space and $A\subseteq B\subseteq X$. Show that $A$ is dense in $X$ if and only if $A$ is dense in $B$ and $B$ is dense in $X$.
\end{problem}

\begin{proof}
	$(\Longrightarrow)$ Suppose $A$ is dense in $X$.

	By Exercise~\ref{exercise:3.7}, the closure of $A$ in $B$ is the intersection of $B$ and the closure of $A$ in $X$, therefore $A$ is dense in $B$. Every nonempty open subset of $X$ intersects $A$ so every nonempty subset of $X$ also intersects $B$ (because $A\subseteq B$), so $B$ is dense in $X$.

	$(\Longleftarrow)$ Suppose $A$ is dense in $B$ and $B$ is dense in $X$.

	Let $U$ be a nonempty open subset of $X$. Because $B$ is dense in $X$, $U$ intersects $B$. Let $V = U\cap B$ then $V$ is open in $B$ with the subspace topology. Because $A$ is dense in $B$, $V$ intersects $A$. Therefore $U$ intersects $A$, which means every nonempty open subset of $X$ intersects $A$. Thus $A$ is dense in $X$.
\end{proof}

\begin{problem}{3-3}
Show by giving a counterexample that the conclusion of the gluing lemma (Lemma 3.23) need not hold if $\{ A_{i} \}$ is an infinite closed cover.
\end{problem}

\begin{proof}
	Let $X = \{ 0 \} \cup \{ 1/n : n\in\mathbb{N} \}$ and the topology on $X$ is the subspace topology inherited from $\mathbb{R}$. Let $A_{0} = \{ 0 \}$ and $A_{n} = \{ 1/n \}$ for $n\in\mathbb{N}$ then they are closed subsets of $X$, because they are closed in $\mathbb{R}$ and
	\[
		\left\{ \frac{1}{n} \right\} = \left\{\frac{1}{n}\right\} \cap X,\qquad \{ 0 \} = \{ 0 \} \cap X.
	\]

	$\{ 0 \}$ is not open in $X$, because there is no open subset $U\subseteq\mathbb{R}$ such that $\{ 0 \} = U\cap X$ (since every neighborhood $U$ of $0$ in $\mathbb{R}$ always contains infinitely many $1/n$).

	Let $f: X\to \mathbb{R}$ be the function such that $f(0) = 0$, $f(1/n) = 1$ for $n\in\mathbb{N}$, then $f$ is not continuous because the preimage under $f$ of any neighborhood of $0$ is $\{ 0 \}$, which is not open in $X$. However $f\vert_{A_{n}}: A_{n}\to \mathbb{R}$ is continuous for every $n\in \{ 0 \}\cup\mathbb{N}$.

	So the gluing lemma doesn't necessarily hold if the closed cover is infinite.
\end{proof}

\begin{problem}{3-4}
Show that every closed ball in $\mathbb{R}^{n}$ is an $n$-dimensional manifold with boundary, as is the complement of every open ball. Assuming the theorem on the invariance of the boundary, show that the manifold boundary of each is equal to its topological boundary as a subset of $\mathbb{R}^{n}$, namely a sphere. [Hint: for the unit ball in $\mathbb{R}^{n}$, consider the map $\pi\circ\sigma^{-1}: \mathbb{R}^{n}\to\mathbb{R}^{n}$, where $\sigma$ is the stereographic projection and $\pi$ is a projection from $\mathbb{R}^{n+1}$ to $\mathbb{R}^{n}$ that omits some coordinate other than the last.]
\end{problem}

\begin{proof}
	We give a proof in the following steps.
	\begin{enumerate}[label={\textbf{Step \arabic*.}},itemindent=1cm]
		\item (Every inversion is a homeomorphism) For every point $a\in\mathbb{R}^{n}$, and $r\in\mathbb{R}$, define $I^{r}_{a}: \mathbb{R}^{n}\smallsetminus\{ a \} \to \mathbb{R}^{n}\smallsetminus\{ a \}$ by
		      \[
			      I^{r}_{a}(x) = \left(a_{1} + \frac{r(x_{1} - a_{1})}{\abs{x - a}^{2}}, \ldots, a_{n} + \frac{r(x_{n} - a_{n})}{\abs{x-a}^{2}}\right).
		      \]

		      Because every inversion maps a sphere not passing through the inversion center to a sphere not passing through the inversion center, maps an open ball to an open ball, and every inversion is involutive (its inverse is the inversion itself), it follows that every inversion is a homeomorphism.
		\item $\mathbb{H}^{n}$ is homeomorphic to $U_{n} = \{ x\in\mathbb{R}^{n} : a_{1}x_{1} + \cdots + a_{n}x_{n} \geq 0 \}$ and $L_{n} = \{ x\in\mathbb{R}^{n} : a_{1}x_{1} + \cdots + a_{n}x_{n} \leq 0 \}$ where $a_{1}, \ldots, a_{n}\in\mathbb{R}$ and $a_{1}, \ldots, a_{n}$ are not simutaneously zero.

		      Without loss of generality, assume that $\alpha_{n} = (a_{1}, \ldots, a_{n})\in\mathbb{R}^{n}$ is a unit vector of $\mathbb{R}^{n}$. Let $\alpha_{1}, \ldots, \alpha_{n}$ be an orthonormal basis of $\mathbb{R}^{n}$. Consider the linear map $\varphi: \mathbb{R}^{n}\to \mathbb{R}^{n}$ where $\varphi(e_{i}) = \alpha_{i}$ for every $i = 1,\ldots, n$ and $e_{1}, \ldots, e_{n}$ is the standard basis of $\mathbb{R}^{n}$. $\varphi$ is a linear isomorphism and unitary operator.

		      Because $\varphi$ preserves norms and inner products, it follows that $\innerprod{\varphi(x), \varphi(e_{n})} = \innerprod{x, e_{n}}$ for every $x\in\mathbb{R}^{n}$. Therefore $\innerprod{x, e_{n}} = 0$ if and only if $\innerprod{\varphi(x), \varphi(e_{n})} = 0$. If $x, y$ satisfies $\innerprod{x, e_{n}} \cdot \innerprod{y, e_{n}} > 0$ (which means $x, y$ are on the same side of the hyperplane $x_{n} = 0$) then
		      \[
			      \innerprod{\varphi(x), \varphi(e_{n})}\cdot\innerprod{\varphi(y), \varphi(e_{n})} = \innerprod{x, e_{n}}\cdot\innerprod{y, e_{n}} > 0
		      \]

		      which implies that $\varphi(x), \varphi(y)$ are on the same side of the hyperplane $a_{1}x_{1} + \cdots + a_{n}x_{n} = 0$. Hence $\varphi\vert_{\mathbb{H}^{n}}$ is either onto $U_{n}$ or $L_{n}$.

		      Linear operators on topological vector spaces are continuous, and the inverse of a linear isomorphism is linear, so a isomorphism linear operator is also a homeomorphism (bijective and bicontinuous). Hence $\varphi$ is a homeomorphism.

		      $U_{n}$ and $L_{n}$ are homeomorphic because of the homeomorphism $f: x\mapsto -x$, and $\varphi$ is a homeomorphism, $\varphi(\mathbb{H}^{n})$ is either $U_{n}$ or $L_{n}$, we conclude that $\mathbb{H}^{n}, U_{n}, L_{n}$ are pairwise homeomorphic.
		\item $\overline{\mathbb{B}}^{n}$ is an $n$-manifold with boundary.

		      $\overline{\mathbb{B}}^{n}$ is Hausdorff and second-countable, because it is a subspace of $\mathbb{R}^{n}$.

		      For every $a\in \mathbb{B}^{n}\subseteq\overline{\mathbb{B}}^{n}$, $\mathbb{B}^{n}$ is a neighborhood of $x$ in $\overline{\mathbb{B}}^{n}$, and $\mathbb{B}^{n}$ is an open subset of $\mathbb{R}^{n}$.

		      Suppose $a\in \overline{\mathbb{B}}^{n}\smallsetminus\mathbb{B}^{n} = \mathbb{S}^{n-1}$. The point $-a$ is the antipodal of $a$. Let $U_{n} = \{ x\in\mathbb{R}^{n}: a_{1}x_{1} + \cdots + a_{n}x_{n} \geq 0 \}$ and $B_{1}(a)$ be an open ball in $\mathbb{R}^{n}$ so by the definition of the subspace topology, $B_{1}(a)\cap \overline{\mathbb{B}}^{n}$ is open in $\overline{\mathbb{B}}^{n}$. Let $P$ be the hyperplane with equation $a_{1}x_{1} + \cdots + a_{n}x_{n} = 0$. For every $y\in B_{1}(a)\cap \overline{\mathbb{B}}^{n}$, we have
		      \[
			      \innerprod{y, a} = \frac{1}{2}(\abs{y}^{2} + \abs{a}^{2} - \abs{y - a}^{2}) = \frac{1}{2}(\abs{y}^{2} + 1 - \abs{y - a}^{2}) > \frac{1}{2}\abs{y}^{2}\geq 0.
		      \]

		      so two points $y, a$ are on the same side of the hyperplane $P$, from which it follows that $B_{1}(a)$ and $a$ are on the same side of the hyperplane $P$. Consider the inversion $I^{2}_{-a}$ maps $a$ to $0$. The inversion $I^{2}_{-a}$ maps $B_{1}(a)$ to another open ball, and maps $a$ to $0\in \partial U_{n}$. If $x\in B_{1}(a) \cap\overline{\mathbb{B}}^{n}$ then
		      \[
			      I^{2}_{-a}(x) = \left( -a_{1} + \frac{2(x_{1} + a_{1})}{\abs{x + a}^{2}}, \ldots, -a_{n} + \frac{2(x_{n} + a_{n})}{\abs{x + a}^{2}} \right) = (y_{1}, \ldots, y_{n}).
		      \]

		      We have
		      \begingroup
		      \allowdisplaybreaks{}
		      \begin{align*}
			      a_{1}y_{1} + \cdots + a_{n}y_{n} & = \sum^{n}_{i=1}\left(-a_{i}^{2} + \frac{2a_{i}(x_{i} + a_{i})}{\abs{x + a}^{2}}\right)                                                \\
			                                       & = -1 + \sum^{n}_{i=1}\frac{2a_{i}(x_{i} + a_{i})}{\abs{x + a}^{2}}                                                                     \\
			                                       & = -1 + \sum^{n}_{i=1}\frac{{(x_{i} + a_{i})}^{2} + a_{i}^{2} - x_{i}^{2}}{\abs{x + a}^{2}}                                             \\
			                                       & = -1 + \frac{1}{\abs{x + a}^{2}}\left( \sum^{n}_{i=1}{(x_{i} + a_{i})}^{2} + \sum^{n}_{i=1}a_{i}^{2} - \sum^{n}_{i=1}x_{i}^{2} \right) \\
			                                       & = -1 + \frac{\abs{x+a}^{2}}{\abs{x+a}^{2}} + \frac{1}{\abs{x+a}^{2}}(1 - \abs{x})                                                      \\
			                                       & = \frac{1}{\abs{x+a}^{2}}(1 - \abs{x}) \geq 0.
		      \end{align*}
		      \endgroup

		      So $I^{2}_{-a}(x)$ and $a$ are on the same side of the hyperplane $P$. The image of $B_{1}(a)\cap \overline{\mathbb{B}}^{n}$ under $I^{2}_{-a}$ is an open subset of $U_{n}$, and $I^{2}_{-a}(a) = 0$, which lies on the boundary of $U_{n}$. Moreover, $I^{2}_{-a}$ is a homeomorphism, so we conclude that every point $a\in\mathbb{S}^{-n-1}$ has a neighborhood in $\overline{\mathbb{B}}^{n}$ that is homeomorphic to an open subset of $U_{n}$. Together with step 1, we conclude that every open subset of $U_{n}$ is homeomorphic to an open subset of $\mathbb{H}^{n}$.

		      Thus $\overline{\mathbb{B}}^{n}$ is an $n$-manifold with boundary.
		\item The manifold boundary of $\overline{\mathbb{B}}^{n}$ is $\mathbb{S}^{n-1}$.

		      According to Step 3, every point of $\mathbb{S}^{n-1}$ is a boundary point. Every point of $\mathbb{B}^{n}$ is an interior point. Because the interior and the boundary of a manifold with boundary are disjoint, it follows that the manifold boundary of $\overline{\mathbb{B}}^{n}$ is $\mathbb{S}^{n-1}$.
		\item $\mathbb{R}^{n}\smallsetminus \mathbb{B}^{n}$ is an $n$-manifold with boundary and the manifold boundary of $\mathbb{R}^{n}\smallsetminus \mathbb{B}^{n}$ is $\mathbb{S}^{n-1}$.

		      As a subspace of $\mathbb{R}^{n}$, $\mathbb{R}^{n}\smallsetminus \mathbb{B}^{n}$ is Hausdorff and second countable.

		      $\mathbb{R}^{n}\smallsetminus\mathbb{B}^{n}$ is the disjoint union of $\mathbb{R}^{n}\smallsetminus\overline{\mathbb{B}}^{n}$ and $\mathbb{S}^{n-1}$. $\mathbb{R}^{n}\smallsetminus\overline{\mathbb{B}}^{n}$ is an $n$-manifold without boundary. Similar to Step 3, 4, we can show that $\mathbb{R}^{n}\smallsetminus \mathbb{B}^{n}$ is an $n$-manifold with boundary and $\mathbb{S}^{n-1}$ is its manifold boundary.
	\end{enumerate}

	Finally, because every closed ball in $\mathbb{R}^{n}$ are homeomorphic, and $\overline{\mathbb{B}}^{n}$ is an $n$-manifold with boundary, we conclude that every closed ball and the complement of every open ball in $\mathbb{R}^{n}$ are $n$-manifolds with boundary.
\end{proof}

\begin{problem}{3-5}\label{problem:3-5}
Show that a finite product of open maps is open; give a counterexample to show that a finite product of closed maps need not be closed.
\end{problem}

\begin{proof}
	Let $f_{i}: X_{i}\to Y_{i}$ be open maps for every $i = 1,\ldots,n$. Let $U$ be an open subset of $X_{1}\times\cdots\times X_{n}$. Due to the definition of the product topology, $U$ is an union of product open sets. If $B_{i}\subseteq X_{i}$ is open for every $i = 1,\ldots,n$ then $f_{i}(B_{i})\subseteq Y_{i}$ is open for every $i = 1,\ldots,n$. Therefore $f_{1}\times\cdots\times f_{n}(B_{1}\times\cdots\times B_{n})$ is open, so $f_{1}\times\cdots\times f_{n}(U)$ is open (every map is well-behaved with arbitrary unions). Hence a finite product of open maps is open.

	Let $X_{1} = \mathbb{R}, Y_{1} = \{ 0 \}, X_{2} = Y_{2} = \mathbb{R}$ where $Y_{1}$ has discrete topology. Define $f_{1}: X_{1}\to Y_{1}$ by $f(x) = 0$, $f_{2} = \operatorname{Id}_{\mathbb{R}}$, then $f_{1}, f_{2}$ are closed maps. Let $A = \{ (x, 1/x): x\in\mathbb{R}\smallsetminus\{0\} \}\subseteq X_{1}\times X_{2}$ then $A$ is closed in $X_{1}\times X_{2}$. The image $f_{1}\times f_{2}(A) = \{ 0 \}\times (\mathbb{R}\smallsetminus\{0\})$ is not closed because the closure of $\{ 0 \}\times (\mathbb{R}\smallsetminus\{0\})$ contains $(0, 0)$ but $\{ 0 \}\times (\mathbb{R}\smallsetminus\{0\})$ does not contain $(0, 0)$. Hence $f_{1}\times f_{2}$ is not closed. Thus a finite product of closed maps need not be closed.
\end{proof}

\begin{problem}{3-6}
Let $X$ be a topological space. The \textbf{diagonal} of $X\times X$ is the subset $\Delta = \{ (x, x): x\in X \}\subseteq X\times X$. Show that $X$ is Hausdorff if and only if $\Delta$ is closed in $X\times X$.
\end{problem}

\begin{proof}
	Let $(x_{1}, x_{2})\in (X\times X)\smallsetminus\Delta$ then $x_{1}\ne x_{2}$.

	$(\Longrightarrow)$ Suppose $X$ is Hausdorff.

	Because $X$ is Hausdorff, $x_{1}, x_{2}$ are separated by some neighborhoods $U_{1}, U_{2}$. $U_{1}$ and $U_{2}$ are disjoint so $U_{1}\times U_{2}\subseteq X\times X$ is open. So there is a neighborhood of $(x_{1}, x_{2})$ contained in $(X\times X)\smallsetminus\Delta$. Therefore $(X\times X)\smallsetminus\Delta$ is open, which means $\Delta$ is closed.

	$(\Longleftarrow)$ Suppose $\Delta$ is closed in $X\times X$.

	$(X\times X)\smallsetminus\Delta$ is open in $X\times X$, so there is a neighborhood $U$ of $(x_{1}, x_{2})$. The collection of product open sets is a basis for the product topology on $X\times X$, so there exist open subsets $B_{1}\subseteq X, B_{2}\subseteq X$ such that
	\[
		(x_{1}, x_{2})\in B_{1}\times B_{2}\subseteq U_{1}\times U_{2}\subseteq (X\times X)\smallsetminus\Delta.
	\]

	Therefore $x_{1}, x_{2}$ are separated by neighborhoods $B_{1}, B_{2}$. Because $x_{1}, x_{2}$ is an arbitrary pair of distinct points of $X$, it follows that $X$ is Hausdorff.
\end{proof}

\begin{problem}{3-7}
Show that the space $X$ of Problem 2-22 is homeomorphic to $\mathbb{R}_{d}\times\mathbb{R}$, where $\mathbb{R}_{d}$ is the set $\mathbb{R}$ with the discrete topology.
\end{problem}

\begin{proof}
	Let $\varphi: \mathbb{R}_{d}\times\mathbb{R}\to X$ be the identity map.

	The collection $\mathscr{B}$ of $I_{a,b,c}$ is a basis for the topology on $\mathbb{R}_{d}\times\mathbb{R}$ because $I_{a,b,c}$ is open in $\mathbb{R}_{d}\times\mathbb{R}$ and for every $x\in U$ where $U$ is open in $\mathbb{R}_{d}\times\mathbb{R}$, there is some $I_{a,b,c}$ such that $x\in I_{a,b,c} \subseteq U$. From Problem~\ref{problem:2-22}, it follows that $\mathscr{B}$ is a basis for the topology on $X$. Hence $X$ and $\mathbb{R}_{d}\times\mathbb{R}$ have the same topology. Therefore $\varphi$ is a continuous and open map. Together with $\varphi$ being a bijective, we conclude that $\varphi$ is a homeomorphism. Hence $X$ is homeomorphic to $\mathbb{R}_{d}\times\mathbb{R}$.
\end{proof}

\begin{problem}{3-8}
Let $X$ denote the Cartesian product of countably infinitely many copies of $\mathbb{R}$ (which is just the set of all infinite sequences of real numbers), endowed with the box topology. Define a map $f: \mathbb{R}\to X$ by $f(x) = (x, x, x, \ldots)$. Show that $f$ is not continuous, even though each of its component functions is.
\end{problem}

\begin{proof}
	For every $(x, x, x, \ldots) \in X$, the following set
	\begin{align*}
		\prod_{n\in\mathbb{N}}\openinterval{x - \frac{1}{n}, x + \frac{1}{n}} = \openinterval{x - 1, x + 1} \times \openinterval{x - \frac{1}{2}, x + \frac{1}{2}} \times \openinterval{x - \frac{1}{3}, x + \frac{1}{3}} \times \cdots
	\end{align*}

	is open in $X$ according to the definition of the box topology, and is a neighborhood of $(x, x, x, \ldots)$. However, the preimage of this set is the singleton set $\{ x \}$, which is not open in $\mathbb{R}$. Hence $f$ is not continuous.
\end{proof}

\begin{problem}{3-9}
Let $X$ be as in the preceding problem. Let $X^{+}\subseteq X$ be the subset consisting of sequences of strictly positive real numbers, and let $z$ denote the zero sequence, that is, the one whose terms are $z_{i} = 0$ for all $i$. Show that $z$ is in the closure of $X^{+}$, but there is no sequence of elements of $X^{+}$ converging to $z$. Then use the sequence lemma to conclude that $X$ is not first countable, and thus not metrizable.
\end{problem}

\begin{proof}
	Let $U$ be a neighborhood of $z$ in $X$. According to the definition of box topology, there are open intervals $\openinterval{a_{1}, b_{1}}, \openinterval{a_{2}, b_{2}}, \openinterval{a_{3}, b_{3}}, \ldots$ such that
	\begin{align*}
		z \in \prod_{n\in\mathbb{Z}_{> 0}}\openinterval{a_{n}, b_{n}} \subseteq U.
	\end{align*}

	Each set $\openinterval{a_{n}, b_{n}}$ contains a positive real number, so $\prod_{n\in\mathbb{Z}_{> 0}}\openinterval{a_{n}, b_{n}}$ intersects $X^{+}$. Hence $z$ is in the closure of $X^{+}$ in $X$.

	Assume for the sake of contradiction that there is a sequence of elements of $X^{+}$ converging to $z$. Let ${(x_{i})}_{i\in\mathbb{N}}$ be a such a sequence, where $x_{i} = (x_{i,1}, x_{i,2}, x_{i,3}, \ldots)$. The following set
	\begin{align*}
		U = \prod_{n\in\mathbb{N}}\openinterval{-x_{n,n}, x_{n,n}} = \openinterval{-x_{1,1}, x_{1,1}}\times \openinterval{-x_{2,2}, x_{2,2}} \times \openinterval{-x_{3,3}, x_{3,3}}\times \cdots
	\end{align*}

	is a neighborhood of $z$ in $X$. Because $x_{i}$ converges to $z$, there is $N\in\mathbb{N}$ such that $x_{n}\in U$ for every $n\geq N$. Hence $x_{N} \in \openinterval{-x_{N}, x_{N}}$, which is a contradiction because $x_{N} \notin \openinterval{-x_{N}, x_{N}}$. Hence there is no sequence of elements of $X^{+}$ converging to $z$.

	According to the sequence lemma, $X$ is not first countable (if $X$ is first countable, then there is a sequence of elements of $X^{+}$ converging to $z$, which is a point in the closure of $X^{+}$).
\end{proof}

\begin{problem}{3-10}
Prove Theorem 3.41 (the characteristic property of disjoint union spaces).

Suppose ${(X_{\alpha})}_{\alpha\in A}$ is an indexed family of topological spaces and $Y$ is any topological space. A map $f: \coprod_{\alpha\in A}X_{\alpha}\to Y$ is continuous if and only if its restriction to each $X_{\alpha}$ is continuous. The disjoint union topology is the unique topology on $\coprod_{\alpha\in A}X_{\alpha}$ with this property.
\end{problem}

\begin{proof}
	We have $f\vert_{X_{\alpha}} = f\circ \iota_{\alpha}$

	See Figure~\ref{fig:3-10}. Suppose $f$ is continuous. Because $\iota_{\alpha}: X_{\alpha} \to X_{\alpha}^{*}$ is continuous for every $\alpha\in A$, it follows that $f\vert_{X_{\alpha}}$ is continuous.

	Conversely, suppose that $f\vert_{X_{\alpha}}$ is continuous for every $\alpha\in A$. Let $V$ be an open subset of $Y$ then $f^{-1}(V)$ is open because $f^{-1}(V) \cap X_{\alpha} = {(f\vert_{X_{\alpha}})}^{-1}(V)$ is open for every $\alpha\in A$. Therefore $f$ is continuous.
	\begin{figure}[htp]
		\renewcommand{\thefigure}{3-10}
		\centering
		\begin{tikzpicture}[every edge/.style = {draw, -latex, thick}]
			\matrix (m) [matrix of math nodes, row sep=3em, column sep=3em]
			{
			\coprod_{\alpha\in A}X_{\alpha} & Y \\
			X_{\alpha} \\
			};
			\path[->] (m-1-1) edge node[above] {$f$} (m-1-2);
			\path[->] (m-2-1) edge node[left] {$\iota_{\alpha}$} (m-1-1);
			\path[->] (m-2-1) edge node[below right] {$f\vert_{X_{\alpha}} = f\circ\iota_{\alpha}$} (m-1-2);
		\end{tikzpicture}
		\caption{}\label{fig:3-10}
	\end{figure}

	Assume that $\mathscr{T}$ is a topology on $\coprod_{\alpha\in A}X_{\alpha}$ that satisfies this property. Denote $\left(\coprod_{\alpha\in A}X_{\alpha}, \mathscr{T}\right)$ by $Y$. Consider the identity map $\operatorname{Id}: \coprod_{\alpha\in A}X_{\alpha}\to Y$. From the property, it follows that the restriction of $\operatorname{Id}$ to $X_{\alpha}$ is continuous for every $\alpha\in A$. Also from the property, it follows that $\operatorname{Id}^{-1}$ is continuous. Because $\operatorname{Id}$ is bicontinuous and bijective, we conclude that $\coprod_{\alpha\in A}X_{\alpha}$ and $Y$ are identical, which means the disjoint union topology is the unique topology on $\coprod_{\alpha\in A}X_{\alpha}$ with the given property.
\end{proof}

\begin{problem}{3-11}
Proposition 3.62(d) (see Exercise~\ref{exercise:3.63}) showed that the restriction of a quotient map to a saturated open subset is a quotient map onto its image. Show that the ``saturated'' hypothesis is necessary, by giving an example of a quotient map $f: X \to Y$ and an open subset $U\subseteq X$ such that $f\vert_{U}: U\to Y$ is surjective but not a quotient map.
\end{problem}

\begin{proof}
	The map $f: \closedinterval{0, 1} \to \mathbb{S}^{1}$ defined by $x\mapsto e^{2\pi\iota x}$ is a quotient map, according to the characteristic property of product topology and the continuity of the exponential function.

	The set $U = \halfopenright{0, 1}$ is an open subset of $\closedinterval{0, 1}$ and $f\vert_{U}: U\to \mathbb{S}^{1}$ is bijective (and so surjective). We will show that $f\vert_{U}$ is not an open map. For every $0 < a < 1$, $\halfopenright{0, a}$ is an open subset of $U$ and $\halfopenright{0, 1}$.

	$f\vert_{U}\left(\halfopenright{0, a}\right) = \left\{ e^{2\pi\iota s} \mid s\in\halfopenright{0, a} \right\}$. Let $A$ be the point with coordinate $e^{2\pi\iota a}$. The stereographic projection from $\mathbb{S}^{1}\smallsetminus\{e^{2\pi\iota a}\}$ onto the straight line through the origin that is perpendicular to $OA$ is a homeomorphism. Therefore the topological space $\mathbb{S}^{1}\smallsetminus \{ e^{2\pi\iota a} \}$ is homeomorphic to $\mathbb{R}$ with the Euclidean topology and $\left\{ e^{2\pi\iota s} \mid s\in\halfopenright{0, a} \right\}$ is homeomorphic to a half-open interval, which is not an open set in $\mathbb{R}$. Hence $f\vert_{U}(\halfopenright{0, a})$ is not open in $\mathbb{S}^{1}$ and $f\vert_{U}$ is not an open map.

	$f\vert_{U}$ is bijective, continuous and not an open map, so it is not a quotient map.
\end{proof}

\begin{problem}{3-12}
Suppose $X$ is a topological space and ${(X_{\alpha})}_{\alpha\in A}$ is an indexed family of topological spaces.
\begin{enumerate}[label={(\alph*)}]
	\item For any subset $S\subseteq X$, show that the subspace topology on $S$ is the coarsest topology for which $\iota_{S}: S\hookrightarrow{}X$ is continuous.
	\item Show that the product topology is the coarsest topology on $\prod_{\alpha\in A}X_{\alpha}$ for which every canonical projection $\pi_{\alpha}: \prod_{\alpha\in A}X_{\alpha} \to X_{\alpha}$ is continuous.
	\item Show that the disjoint union topology is the finest topology on $\coprod_{\alpha\in A}X_{\alpha}$ for which every canonical injection $\iota_{\alpha}: X_{\alpha} \to \coprod_{\alpha\in A}X_{\alpha}$ is continuous.
	\item Show that if $q: X\to Y$ is any surjective map, the quotient topology on $Y$ is the finest topology for which $q$ is continuous.
\end{enumerate}
\end{problem}

\begin{proof}
	\begin{enumerate}[label={(\alph*)}]
		\item Let $\mathscr{T}$ be a topology on $S$ for which $\iota_{S}: S\hookrightarrow{} X$ is continuous. For every open subset $U\subseteq X$, $\iota_{S}^{-1}(U) = U\cap S$. Because $\iota_{S}$ is continuous, $U\cap S$ is open in $S$. Therefore the subspace topology on $S$ is contained in $\mathscr{T}$. Hence the subspace topology on $S$ is the coarsest topology for which $\iota_{S}: S\hookrightarrow{} X$ is continuous.
		\item Let $\mathscr{T}$ be a topology on $\prod_{\alpha\in A}X_{\alpha}$ for which every canonical projection $\pi_{\alpha}: \prod_{\alpha\in A}X_{\alpha} \to X_{\alpha}$ is continuous.

		      Consider the product $\prod_{\alpha\in A}U_{\alpha}$ where $U_{\alpha} = X_{\alpha}$ for all but finitely many $\alpha\in A$ and $U_{\alpha}\subseteq X_{\alpha}$ is open. Denote by $C$ the subset of $A$ consisting of $\alpha$ which satisfy $U_{\alpha}\ne X_{\alpha}$. For every $\alpha\in A$, $\pi_{\alpha}^{-1}(U_{\alpha})$ is open because $\pi_{\alpha}$ is continuous. We have $\bigcap_{\alpha\in C}\pi_{\alpha}^{-1}(U_{\alpha}) = \prod_{\alpha\in A}U_{\alpha}$ so $\prod_{\alpha\in A}U_{\alpha}$ is an open set of $\mathscr{T}$. Therefore $\mathscr{T}$ contains every elements of the basis for the product topology on $\prod_{\alpha\in A}X_{\alpha}$, which implies that $\mathscr{T}$ contains the product topology on $\prod_{\alpha\in A}X_{\alpha}$.

		      Thus the product topology is the coarsest topology on $\prod_{\alpha\in A}X_{\alpha}$ for which every canonical projection $\pi_{\alpha}$ is continuous.
		\item Let $\mathscr{T}$ be a topology on $\coprod_{\alpha\in A}X_{\alpha}$ for which every canonical injection $\iota_{\alpha}: X_{\alpha}\to \coprod_{\alpha\in A}X_{\alpha}$ is continuous. Let $U$ be an open subset of $\coprod_{\alpha\in A}X_{\alpha}$ then for every $\alpha\in A$, $\iota_{\alpha}^{-1}(U) = U\cap X_{\alpha}$ is open in $X_{\alpha}$ because $\iota_{\alpha}$ is continuous. Therefore $U$ is an open set in the disjoint union topology on $\coprod_{\alpha\in A}X_{\alpha}$. Because $U$ is an arbitrary open set in $\mathscr{T}$, it follows that $\mathscr{T}$ is contained in the disjoint union topology on $\coprod_{\alpha\in A}X_{\alpha}$.

		      Hence the disjoint union topology is the finest topology on $\coprod_{\alpha\in A}X_{\alpha}$ for which every canonical $\iota_{\alpha}: X_{\alpha}\to \coprod_{\alpha\in A}X_{\alpha}$ is continuous.
		\item Let $\mathscr{T}$ be a topology on $Y$ for which $q$ is continuous.

		      Let $U \in \mathscr{T}$. Because $q$ is continuous, $q^{-1}(U)$ is open in $X$. Since $q$ is surjective, we have $q(q^{-1}(U)) = U$. From the definition of quotient topology, it follows that $U$ is in the quotient topology on $Y$ induced by $q$. Therefore the quotient topology on $Y$ induced by $q$ is the finest topology on $Y$ such that $q$ is continuous.
	\end{enumerate}
\end{proof}

\begin{problem}{3-13}
Suppose $X$ and $Y$ are topological spaces and $f: X \to Y$ is a continuous map. Prove the following:
\begin{enumerate}[label={(\alph*)}]
	\item If $f$ admits a continuous left inverse, it is a topological embedding.
	\item If $f$ admits a continuous right inverse, it is a quotient map.
	\item Give examples of a topological embedding with no continuous left inverse, and a quotient map with no continuous right inverse.
\end{enumerate}
\end{problem}

\begin{proof}
	\begin{enumerate}[label={(\alph*)}]
		\item Suppose $f$ admits a continuous left inverse $g: Y \to X$, then $g\circ f = \operatorname{Id}_{X}$. $g\circ f = \operatorname{Id}_{X}$ implies that $g$ is surjective and $f$ is injective. So $f$ is a continuous injection, $g$ is a continuous surjection.

		      Let $U$ be an open subset of $X$. $g(f(U)) = (g\circ f)(U) = \operatorname{Id}_{X}(U) = U$, it follows that $f(U) \subseteq g^{-1}(U)$. Moreover, $f(U) \subseteq {(g\vert_{f(X)})}^{-1}(U)$. We will show that $f(U) = {(g\vert_{f(X)})}^{-1}(U)$.

		      Let $y\in {(g\vert_{f(X)})}^{-1}(U)$ then $(g\vert_{f(X)})(y) \in U$. Because $y\in f(X)$ and $f$ is injective, there is a unique $x\in X$ such that $f(x) = y$. Therefore $(g\vert_{f(X)})(y) = (g\vert_{X})(f(x)) = x \in U$, so $y = f(x) \in f(U)$. Hence ${(g\vert_{f(X)})}^{-1}(U) \subseteq f(U)$, so $f(U) = {(g\vert_{f(X)})}^{-1}(U)$.

		      $g$ is continuous so the restriction of $g$ to $f(X)$ is also continuous. Therefore ${(g\vert_{f(X)})}^{-1}(U)$ is open in $f(X)$, and $f(U)$ is open in $f(X)$.

		      Because $f$ is continuous, injective, and it maps open subset of $X$ to open subset of $f(X)$, we conclude that $f$ is a topological embedding.
		\item Suppose $f$ admits a continuous right inverse $g: Y\to X$, then $f\circ g = \operatorname{Id}_{Y}$. $f\circ g = \operatorname{Id}_{Y}$ implies that $f$ is surjective and $g$ is injective. Let $U$ be a subset of $Y$. We always have $f(f^{-1}(U)) = U$ because $f$ is surjective.

		      If $U$ is open in $Y$, then $f^{-1}(U)$ is open in $X$, since $f$ is continuous.

		      If $f^{-1}(U)$ is open in $X$ then $g^{-1}(f^{-1}(U))$ is open in $Y$ because $g$ is continuous. On the other hand
		      \begin{align*}
			      g^{-1}(f^{-1}(U)) = {(f \circ g)}^{-1}(U) = U
		      \end{align*}

		      so $U$ is open in $Y$. Hence $f$ is a quotient map.
		\item Consider $f: \halfopenright{0, 1} \to \closedinterval{0, 1}$ defined by $f(x) = x$. $f$ is a topological embedding. Because $f$ is injective, $f$ has a left inverse $g: \closedinterval{0, 1} \to \halfopenright{0, 1}$. Then $0\leq g(1) < 1$. Let $b$ be a real number such that $g(1) < b < 1$ then $\openinterval{0, b}$ is open in $\halfopenright{0, 1}$ and $\closedinterval{0, 1}$. Therefore $g^{-1}\left(\openinterval{0, b}\right) = \openinterval{0, b} \cup \{ 1 \}$, which is not open in $\closedinterval{0, 1}$. Hence $f$ doesn't have a continuous left inverse.

		      Consider the quotient map $q: \closedinterval{0, 1}\to \mathbb{S}^{1}$ defined by $q(s) = e^{2\pi\iota s}$. It is a quotient map because $\mathbb{S}^{1}$ is homeomorphic to $\closedinterval{0, 1}/\{ 0, 1 \}$. Let $f: \mathbb{S}^{1}\to \closedinterval{0, 1}$ be a right inverse of $q$. Let $0 < \varepsilon < 1$ then $\halfopenleft{\varepsilon, 1}$ is open in $\closedinterval{0, 1}$ and $f^{-1}(\halfopenleft{\varepsilon, 1}) = \{ e^{2\pi\iota s}: \varepsilon < s \leq 1 \}$, which is not open in $\mathbb{S}^{1}$. Hence $q$ doesn't have a continuous right inverse.
	\end{enumerate}
\end{proof}

\begin{problem}{3-14}\label{problem:3-14}
Show that real projective space $\mathbb{RP}^{n}$ is an $n$-manifold.
\end{problem}

\begin{proof}
	Let $\sim$ be the equivalence relation on $\mathbb{R}^{n+1}\smallsetminus\{0\}$ where $a\sim b$ if and only if there is a nonzero real number $\lambda$ such that $a = \lambda b$. Consider the quotient map $q: \mathbb{R}^{n+1}\smallsetminus\{0\} \to \mathbb{RP}^{n}$ defined by $q(x) = \operatorname{span}(x)$.

	If $A$ is open in $\mathbb{R}^{n+1}\smallsetminus\set{0}$ then $\bigcup_{\lambda\in\mathbb{R}\smallsetminus\set{0}}\lambda A$ is a saturated open set with respect to $q$, where $\lambda A = \set{ \lambda a : a\in A }$. Therefore $q^{-1}(q(A))$ is open, which means $q(A)$ is open. Hence $q$ is an open quotient map.

	\textbf{Locally Euclidean property.}

	For each $i\in \{ 1, \ldots, n+1 \}$, let $V_{i}$ be the subset of $\mathbb{R}^{n+1}$ where $x_{i} \ne 0$. $q(V_{1}), \ldots, q(V_{n+1})$ cover $\mathbb{RP}^{n}$. The complement of $V_{i}$ in $\mathbb{R}^{n+1}$, which is the hyperspace $x_{i} = 0$, is closed, so $V_{i}$ is open in $\mathbb{R}^{n+1}$, hence open in $\mathbb{R}^{n+1}\smallsetminus\{0\}$ (using subspace topology). Moreover, $V_{i}$ is saturated with respect to $q$ (Exercise~\ref{exercise:3.59}). Because a quotient map maps open saturated open sets to open sets, $q(V_{i})$ is open in $\mathbb{RP}^{n}$ for every $i\in \{ 1, \ldots, n+1 \}$.

	Let $U_{i} = q(V_{i})$ and define $\varphi_{i}: U_{i} \to \mathbb{R}^{n}$ by
	\begin{align*}
		\varphi_{i}({[x]}_{\sim}) = \left(\frac{x_{1}}{x_{i}}, \ldots, \frac{x_{i-1}}{x_{i}}, \frac{x_{i+1}}{x_{i}}, \ldots, \frac{x_{n+1}}{x_{i}}\right).
	\end{align*}

	This map is well-defined because of $a, b \in [x]\in U_{i}$ then $a = (a_{i}/b_{i})b$.

	$\varphi_{i}$ is injective and surjective (because the image of $\varphi_{i}$ is the entire $\mathbb{R}^{n}$). Now we prove that $\varphi_{i}$ is continuous and open. Let $q_{i}: V_{i} \to \mathbb{R}^{n}$ defined by $q_{i}(x) = \left(\frac{x_{1}}{x_{i}}, \ldots, \frac{x_{i-1}}{x_{i}}, \frac{x_{i+1}}{x_{i}}, \ldots, \frac{x_{n+1}}{x_{i}}\right)$, then $q_{i} = \varphi_{i}\circ (q\vert_{V_{i}})$.

	$q$ is a quotient map, so $q\vert_{V_{i}}: V_{i} \to q(V_{i}) = U_{i}$ is also a quotient map. $q_{i}$ is continuous so from the characteristic property of quotient map, it follows that $\varphi_{i}$ is continuous.

	Let $U$ be an open subset of $U_{i} \subseteq \mathbb{RP}^{n}$ then $U = (q\vert_{V_{i}})({(q\vert_{V_{i}})}^{-1}(U))$ because $q\vert_{V_{i}}$ is surjective. It follows that $\varphi_{i}(U) = q_{i}({(q\vert_{V_{i}})}^{-1}(U))$. Because $q\vert_{V_{i}}$ is continuous, $V = {(q\vert_{V_{i}})}^{-1}(U)$ is an open subset of $V_{i}$, so it is also an open subset of $\mathbb{R}^{n+1}\smallsetminus\{ 0 \}$.

	Let $y$ be a point of $q_{i}({(q\vert_{V_{i}})}^{-1}(U)) = q_{i}(V) \subseteq \mathbb{R}^{n}$. There is $x\in V$ such that $q_{i}(x) = y$. Because $V$ is open, there is an open cube $\openinterval{a_{1}, b_{1}}\times\cdots\times\openinterval{a_{n+1}, b_{n+1}} \subseteq \mathbb{R}^{n+1}\smallsetminus\{0\}$ such that
	\begin{align*}
		x \in \openinterval{a_{1}, b_{1}}\times\cdots\times\openinterval{a_{n+1}, b_{n+1}} \subseteq V
	\end{align*}

	where each interval $\openinterval{a_{j}, b_{j}}$ doesn't contain $0$ (these product open sets comprise a basis for the subspace topology on $\mathbb{R}^{n+1}\smallsetminus\{0\}$). If $a_{i}, b_{i} > 0$ then
	\begin{multline*}
		y = q_{i}(x) \in q_{i}(\openinterval{a_{1}, b_{1}}\times\cdots\times\openinterval{a_{n+1}, b_{n+1}}) \\
		= \bigcup_{a_{i} < c_{i} < b_{i}}\left(\openinterval{\frac{a_{1}}{c_{i}}, \frac{b_{1}}{c_{i}}} \times \cdots \times \openinterval{\frac{a_{i-1}}{c_{i}}, \frac{b_{i-1}}{c_{i}}} \times \openinterval{\frac{a_{i+1}}{c_{i}}, \frac{b_{i+1}}{c_{i}}} \times \cdots \times \openinterval{\frac{a_{n+1}}{c_{i}}, \frac{b_{n+1}}{c_{i}}}\right)
	\end{multline*}

	which is a neighborhood of $y$ contained in $q_{i}(V)$. Hence $q_{i}(V)$ is open in $\mathbb{R}^{n}$, so $\varphi_{i}(U)$ is open for every open subset $U\subseteq U_{i}$, which means $\varphi_{i}$ is open. If $a_{i}, b_{i} < 0$, the reasoning remains the same, except for the ends of open intervals are swapped.

	Therefore $\varphi_{i}$ is a homeomorphism from $U_{i}$ onto $\mathbb{R}^{n}$, so the $n$-dimensional real projective space $\mathbb{RP}^{n}$ is locally Euclidean of dimension $n$.

	\textbf{Hausdorffness} (As for me, this is the hardest part.)

	We will use Proposition 3.57.

	Let $X = \mathbb{R}^{n+1}\smallsetminus\set{0}$ and $Y = \mathbb{RP}^{n}$, then $q: X\to Y$ is an open quotient map. According to Proposition 3.57, it suffices to prove that $\mathscr{R} = \set{ (x_{1}, x_{2}) : q(x_{1}) = q(x_{2}) }$ is closed in $X\times X$.

	$q(x_{1}) = q(x_{2})$ if and only if $x_{1}, x_{2}$ are linearly dependent. So $(x_{1}, x_{2}) \in (X\times X)\smallsetminus\mathscr{R}$ if and only if $x_{1}, x_{2}$ are linearly independent.

	Extend the list $x_{1}, x_{2}$ to a basis $x_{1}, x_{2}, \ldots, x_{n+1}$ for $\mathbb{R}^{n+1}$, the restriction $\det(x_{1}, \ldots, x_{n+1}) \ne 0$. Because $\det$ is continuous, $\det: X\times X\times\set{x_{3}}\times\cdots\times\set{x_{n+1}} \to \mathbb{R}$ is continuous, so there is an open subset $W$ of $X\times X$ such that $\det(x_{1}', x_{2}', x_{3}, \ldots, x_{n+1}) \ne 0$ for every $(x_{1}', x_{2}') \in W$. Hence every pair of vectors in $W$ is linearly independent, which implies that $W \subseteq (X\times X)\smallsetminus\mathscr{R}$, so $(X\times X)\smallsetminus\mathscr{R}$ is open. Therefore $\mathscr{R}$ is closed in $X\times X$.

	From Proposition 3.57, it follows that $\mathbb{RP}^{n}$ is Hausdorff.

	\textbf{Second countability.}

	$U_{i}$ is homeomorphic to $\mathbb{R}^{n}$ and $\mathbb{R}^{n}$ is second countable then so is $U_{i}$ so $U_{1}, \ldots, U_{n+1}$ are all second countable. $U_{1}, \ldots, U_{n+1}$ cover $\mathbb{RP}^{n}$. From Problem~\ref{problem:2-19} it follows that $\mathbb{RP}^{n}$ is second countable.

	Thus $\mathbb{RP}^{n}$ is an $n$-manifold.
\end{proof}

\begin{problem}{3-15}\label{problem:3-15}
Let ${\mathbb{CP}}^{n}$ denote the set of all $1$-dimensional complex subspaces of $\mathbb{C}^{n+1}$, called \textbf{$n$-dimensional complex projective space}. Topologize ${\mathbb{CP}}^{n}$ as the quotient ${(\mathbb{C}^{n+1}\smallsetminus \{0\})}/\mathbb{C}^{*}$, where $\mathbb{C}^{*}$ is the group of nonzero complex numbers acting by scalar multiplication. Show that $\mathbb{CP}^{n}$ is an $2n$-manifold.
\end{problem}

\begin{proof}
	Let $\sim$ be the equivalence relation on $\mathbb{C}^{n+1}\smallsetminus\{0\}$ where $a\sim b$ if and only if there is a nonzero complex number $\lambda$ such that $a = \lambda b$. Consider the quotient map $q: \mathbb{C}^{n+1}\smallsetminus\{0\} \to \mathbb{CP}^{n}$ defined by $q(x) = \operatorname{span}(x)$.

	If $A$ is open in $\mathbb{C}^{n+1}\smallsetminus\set{0}$ then $\bigcup_{\lambda\in\mathbb{C}^{*}}\lambda A$ is a saturated open set with respect to $q$, where $\lambda A = \set{ \lambda a : a\in A }$. Therefore $q^{-1}(q(A))$ is open, which means $q(A)$ is open. Hence $q$ is an open quotient map.

	\textbf{Locally Euclidean property}

	For every $i\in \{ 1, \ldots, n+1 \}$, let $V_{i} = \set{ z \in \mathbb{C}^{n+1}\smallsetminus\set{0} \mid z_{i} \ne 0 }$ then $V_{i}$ is a saturated set with respect to $q$. The complement of $V_{i}$ in $\mathbb{C}^{n+1}$ is closed in $\mathbb{C}^{n+1}$, so $V_{i}$ is open in $\mathbb{C}^{n+1}$ and hence open in $\mathbb{C}^{n+1}\smallsetminus\set{0}$ (with subspace topology). Because $V_{i}$ is an open saturated set of $\mathbb{C}^{n+1}\smallsetminus\set{0}$, the set $q(V_{i})$ is open in $\mathbb{CP}^{n}$.

	Define $U_{i} = q(V_{i})$ and $\varphi_{i}: U_{i} \to \mathbb{C}^{n}$ by
	\begin{align*}
		\varphi_{i}({[z]}_{\sim}) = \left(\frac{z_{1}}{z_{i}}, \ldots, \frac{z_{i-1}}{z_{i}}, \frac{z_{i+1}}{z_{i}}, \ldots, \frac{z_{n+1}}{z_{i}}\right).
	\end{align*}

	$\varphi_{i}$ is bijective. $q$ is a quotient map, so $q\vert_{V_{i}}: V_{i} \to q(V_{i}) = U_{i}$ is also a quotient map. Define $q_{i}: V_{i}\to \mathbb{C}^{n}$ by $q_{i}(z) = \left( \frac{z_{1}}{z_{i}}, \ldots, \frac{z_{i-1}}{z_{i}}, \frac{z_{i+1}}{z_{i}}, \ldots, \frac{z_{n+1}}{z_{i}} \right)$, then $q_{i} = \varphi_{i} \circ (q\vert_{V_{i}})$.

	$q_{i}$ is continuous because of the characteristic property of product topology, and hence $\varphi_{i}$ is continuous because of the characteristic property of quotient topology.

	Let $U$ be an open subset of $U_{i} \subseteq \mathbb{CP}^{n}$ then $U = (q\vert_{V_{i}})({(q\vert_{V_{i}})}^{-1}(U))$ because $q\vert_{V_{i}}$ is surjective (it is a quotient map). Due to the definition of $q_{i}$, $\varphi_{i}(U) = q_{i}({(q\vert_{V_{i}})}^{-1}(U))$. Because $q\vert_{V_{i}}$ is continuous and $U$ is open in $U_{i}$ (the codomain of $q\vert_{V_{i}}$), the preimage $V = {(q\vert_{V_{i}})}^{-1}(U)$ is open in $V_{i}$.

	Suppose $w$ is a point of $q_{i}(V)$ then there is $z\in V$ such that $q_{i}(z) = w$. Since $V$ is open in $V_{i}$ (hence also open in $\mathbb{C}^{n+1}\smallsetminus\set{0}$), there exist open disks $D_{1}, \ldots, D_{n+1}$ in $\mathbb{C}$ such that
	\begin{align*}
		z \in (D_{1}\smallsetminus\set{0}) \times \cdots \times (D_{n+1}\smallsetminus\set{0}) \subseteq V \subseteq \mathbb{C}^{n+1}\smallsetminus\set{0}
	\end{align*}

	Then
	\begin{multline*}
		w = q_{i}(z) \in q_{i}\left((D_{1}\smallsetminus\set{0}) \times \cdots \times (D_{n+1}\smallsetminus\set{0})\right) \\
		\bigcup_{c_{i} \in D_{i}}\left( (D_{1}' \smallsetminus \set{0}) \times \cdots \times (D_{i-1}' \smallsetminus \set{0}) \times (D_{i+1}' \smallsetminus \set{0}) \times \cdots \times (D_{n+1}' \smallsetminus \set{0}) \right)
	\end{multline*}

	where each $D_{j}'\smallsetminus\set{0}$ is the image of $D_{j}\smallsetminus\set{0}$ under the map $m_{j}: \mathbb{C}^{*}\to \mathbb{C}^{*}$ defined by $m_{j}(z) = c_{i}^{-1}z$. The sets $D'_{j}$ are also open disks in $\mathbb{C}$ so $D_{j}'\smallsetminus\set{0}$ are open in $\mathbb{C}$. Hence the set
	\begin{align*}
		\bigcup_{c_{i} \in D_{i}}\left( (D_{1}' \smallsetminus \set{0}) \times \cdots \times (D_{i-1}' \smallsetminus \set{0}) \times (D_{i+1}' \smallsetminus \set{0}) \times \cdots \times (D_{n+1}' \smallsetminus \set{0}) \right)
	\end{align*}

	is a neighborhood of $w$ contained in $q_{i}(V)$. Therefore $q_{i}(V)$ is open, so $\varphi_{i}(U)$ is open for every open subset $U$ of $U_{i}$, which implies that $\varphi_{i}$ is an open map.

	Hence $\varphi_{i}$ is a homeomorphism from $U_{i}$ to $\mathbb{C}^{n}$. $U_{1}, \ldots, U_{n+1}$ cover $\mathbb{C}^{n+1}\smallsetminus\set{0}$ and they are open in $\mathbb{C}^{n+1}\smallsetminus\set{0}$ so every point of $\mathbb{CP}^{n}$ has a neighborhood homeomorphic to $\mathbb{C}^{n}$. On the other hand $\mathbb{C}^{n}$ is homeomorphic to $\mathbb{R}^{2n}$, so $\mathbb{CP}^{n}$ is locally Euclidean of dimension $2n$.

	\textbf{Hausdorffness}

	This part is similar to that of Problem~\ref{problem:3-14}.

	\textbf{Second countability}

	This part is similar to that of Problem~\ref{problem:3-14}.

	Thus $\mathbb{CP}^{n}$ is a $2n$-manifold.
\end{proof}

\begin{problem}{3-16}
Let $X$ be the subset $(\mathbb{R}\times\set{0}) \cup (\mathbb{R}\times\set{1}) \subseteq \mathbb{R}^{2}$. Define an equivalence relation on $X$ by declaring $(x, 0) \sim (x, 1)$ if $x\ne 0$. Show that the quotient space $X/_{\sim}$ is locally Euclidean and second countable, but not Hausdorff. (This space is called the \textbf{line with two origins}.)
\end{problem}

\begin{proof}
	Let $q$ be the quotient map $X\to X/_{\sim}$ that induces the quotient space $X/_{\sim}$.

	For every $(x, 0) \in X/_{\sim}$, $\openinterval{x - 1, x + 1} \times \set{0}$ is open, because its preimage is open, which is the following set
	\begin{equation*}
		(\openinterval{x - 1, x + 1} \times \set{0}) \cup (\openinterval{x - 1, x + 1} \times \set{1})
	\end{equation*}

	so $\openinterval{x - 1, x + 1} \times \set{0}$ is a neighborhood of $(x, 0)$. The map $\pi: \mathbb{R} \times \set{0} \to \mathbb{R}$ defined by $\pi(a, 0) = a$ is a homeomorphism, so $(x, 0)$ has a neighborhood which is homeomorphic to an open subset of $\mathbb{R}$. Analogously, $(x, 1)$ has a neighborhood which is homeomorphic to an open subset of $\mathbb{R}$. Therefore $X/_{\sim}$ is locally Euclidean of dimension 1.

	$\mathbb{R}$ is second countable, let $\mathscr{B}$ be a countable basis for $\mathbb{R}$. The collection
	\begin{equation*}
		\mathscr{B'} = \set{ B \times\set{0} : B\in\mathscr{B} } \cup \set{ B \times\set{1} : B\in\mathscr{B} }
	\end{equation*}

	is a countable basis for $X/_{\sim}$, so $X/_{\sim}$ is second countable.

	$(0, 0)$ and $(0, 1)$ are distinct points of $X/_{\sim}$. Let $U, V$ be neighborhoods of $(0, 0)$ and $(0, 1)$ in $X/_{\sim}$ then $\pi_{1}(U)$ and $\pi_{1}(V)$ contain an open interval $\openinterval{a, b}$ of $\mathbb{R}$ containing $0$, where $\pi_{1}: \mathbb{R}\times\mathbb{R} \to \mathbb{R}$ is a canonical projection. Let $c\in \openinterval{a, b}$ such that $c\ne 0$ then $(c, 0) = (c, 1) \in U, V$, which implies that $U$ and $V$ are not disjoint. Hence $X/_{\sim}$ is not Hausdorff.
\end{proof}

\begin{problem}{3-17}
This problem shows that the conclusion of Proposition 3.57 need not be true if the quotient map is not assumed to be open. Let $X$ be the following subset of $\mathbb{R}^{2}$
\begin{equation*}
	X = \left(\openinterval{0, 1} \times \openinterval{0, 1}\right) \cup \set{(0, 0)} \cup \set{(1, 0)}.
\end{equation*}

For any $\varepsilon \in \openinterval{0, 1}$, let $C_{\varepsilon}$ and $D_{\varepsilon}$ be the sets
\begin{align*}
	C_{\varepsilon} & = \set{(0, 0)} \cup \left(\openinterval{0, \frac{1}{2}}\times \openinterval{0, \varepsilon}\right), \\
	D_{\varepsilon} & = \set{(1, 0)} \cup \left(\openinterval{\frac{1}{2}, 1}\times \openinterval{0, \varepsilon}\right).
\end{align*}

Define a basis $\mathscr{B}$ for a topology on $X$ consisting of all open rectangles of the form $\openinterval{a_{1}, b_{1}} \times \openinterval{a_{2}, b_{2}}$ with $0\leq a_{1} < b_{1}\leq 1$ and $0\leq a_{2} < b_{2} \leq 1$, together with all subsets of the form $C_{\varepsilon}$ or $D_{\varepsilon}$.
\begin{enumerate}[label={(\alph*)}]
	\item Show that $\mathscr{B}$ is a basis for a topology on $X$.
	\item Show that this topology is Hausdorff.
	\item Show that the subset $A = \set{(0, 0)} \cup \left( \halfopenleft{0, \frac{1}{2}} \times \openinterval{0,1} \right)$ is closed in $X$.
	\item Let $\sim$ be the relation on $X$ generated by $a \sim a'$ for all $a, a'\in A$. Show that $\sim$ is closed in $X\times X$.
	\item Show that the quotient space $X/A$ obtained by collapsing $A$ to a point is not Hausdorff.
\end{enumerate}
\end{problem}

\begin{proof}
	\begin{enumerate}[label={(\alph*)}]
		\item Observe that all open rectangles and the family of sets $C_{\varepsilon}, D_{\varepsilon}$ (for any $\varepsilon \in \openinterval{0,1}$) are subsets of $X$.

		      The union of all open rectangles is equal to $\openinterval{0,1}\times\openinterval{0,1}$. On the other hand
		      \begin{align*}
			      (0, 0) \in \bigcup_{\varepsilon\in\openinterval{0,1}}C_{\varepsilon} \qquad
			      (1, 0) \in \bigcup_{\varepsilon\in\openinterval{0,1}}D_{\varepsilon}
		      \end{align*}

		      so
		      \begin{align*}
			      X = \left(\bigcup_{\substack{ 0\leq a_{1} < b_{1}\leq 1 \\ 0\leq a_{2} < b_{2}\leq 1 }} \openinterval{a_{1}, b_{1}} \times \openinterval{a_{2}, b_{2}}\right) \cup \left(\bigcup_{\varepsilon\in\openinterval{0,1}}C_{\varepsilon}\right) \cup \left(\bigcup_{\varepsilon\in\openinterval{0,1}}D_{\varepsilon}\right).
		      \end{align*}

		      The intersection of any two open rectangles, if not empty, is also an open rectangle. The intersection of any two sets of type $C_{\varepsilon}$, if not empty, is a set of this type. The intersection of any two sets of type $D_{\varepsilon}$, if not empty, is a set of this type.

		      The intersection of an open rectangle and a set of type $C_{\varepsilon}$ or of type $D_{\varepsilon}$, if not empty, is an open rectangle. The intersection of a set of type $C_{\varepsilon}$ and a set of type $D_{\varepsilon}$ is empty.

		      Therefore $\mathscr{B}$ is a basis for a topology on $X$.
		\item Let $(x_{1}, y_{1})$ and $(x_{2}, y_{2})$ be distinct points of $X$. We consider the following cases (these are all possible cases).

		      Case 1. $(x_{1}, y_{1}) = (0, 0)$ and $(x_{2}, y_{2}) = (1, 0)$. $C_{\varepsilon}$ and $D_{\varepsilon}$ are disjoint neighborhoods of these points.

		      Case 2. $(x_{1}, y_{1}) = (0, 0)$ and $(x_{2}, y_{2})\ne (1, 0)$. Then $C_{y_{2}/2}$ and $\openinterval{0, 1}\times\openinterval{y_{2}/2, (y_{2}+1)/2}$ are disjoint neighborhoods of these points.

		      Case 3. $(x_{1}, y_{1}) = (1, 0)$ and $(x_{2}, y_{2}) \ne (0, 0)$. Then $D_{y_{2}/2}$ and $\openinterval{0, 1}\times\openinterval{y_{2}/2, (y_{2}+1)/2}$ are disjoint neighborhoods of these points.

		      Case 4. $0 < x_{1}, y_{1}, x_{2}, y_{2} < 1$.

		      $\openinterval{0, 1}\times \openinterval{0, 1}$ is a subset of $X$ and the subspace topology inherited from $X$ is the same to that inherited from the Euclidean topology on $\mathbb{R}\times\mathbb{R}$ (roughly because they admit open rectangles as bases). $\mathbb{R}\times\mathbb{R}$ is Hausdorff, then so is $\openinterval{0, 1}\times \openinterval{0, 1}$. Hence $(x_{1}, y_{1})$ and $(x_{2}, y_{2})$ are separated open neighborhoods.

		      Thus the topology in part (a) is Hausdorff.
		\item The complement of $A$ is
		      \begin{align*}
			      \set{(1, 0)} \cup \left( \openinterval{\frac{1}{2}, 1} \times \openinterval{0, 1} \right) & = \set{(1, 0)} \cup \bigcup_{\varepsilon\in\openinterval{0,1}} \left(\openinterval{\frac{1}{2}, 1} \times \openinterval{0, \varepsilon}\right)              \\
			                                                                                                & = \bigcup_{\varepsilon\in\openinterval{0,1}} \left(\set{(1, 0)} \cup \left(\openinterval{\frac{1}{2}, 1} \times \openinterval{0, \varepsilon}\right)\right) \\
			                                                                                                & = \bigcup_{\varepsilon\in\openinterval{0,1}} D_{\varepsilon}
		      \end{align*}

		      which is open. So $A$ is closed.
		\item Consider two inequivalent points $(x_{1}, y_{1})$ and $(x_{2}, y_{2})$ of $X$, in other words, $\left((x_{1}, y_{1}), (x_{2}, y_{2})\right) \in (X\times X)\smallsetminus\sim$. Because $X$ with the topology in part (a) is Hausdorff, there exist disjoint neighborhoods $U_{1}$ and $U_{2}$ of these points. There are precisely the two following cases.

		      Case 1. $(x_{1}, y_{1}) \in A$, $(x_{2}, y_{2}) \notin A$.

		      $X\smallsetminus A$ is open, according to part (c). $U_{2}\cap (X\smallsetminus A)$ is then a neighborhood of $(x_{2}, y_{2})$. From these, it follows that $U_{1} \times (U_{2}\cap (X\smallsetminus A))$ is disjoint from $\sim$. Moreover, $U_{1} \times (U_{2}\cap (X\smallsetminus A))$ is a neighborhood contained in $(X\times X)\smallsetminus\sim$ of $\left((x_{1}, y_{1}), (x_{2}, y_{2})\right)$.

		      Case 2. Two points doesn't lie in $A$.

		      Then $U_{1}\cap (X\smallsetminus A)$ and $U_{2} \cap (X\smallsetminus A)$ are also disjoint neighborhoods of $(x_{1}, y_{1})$ and $(x_{2}, y_{2})$. Since they are contained in $X\smallsetminus A$, it follows that $(U_{1}\cap (X\smallsetminus A)) \times (U_{2}\cap (X\smallsetminus A))$ is a neighborhood of $\left((x_{1}, y_{1}), (x_{2}, y_{2})\right)$ contained in $(X\times X)\smallsetminus\sim$.

		      From the two cases, it follows that $(X\times X)\smallsetminus\sim$ is open, hence $\sim$ is closed in $X\times X$.
		\item Let $q: X \to X/A$ be the quotient map that induces the quotient topology.

		      We will show that two points (equivalence classes) $q((0, 0))$ and $q((1, 0))$ cannot be separated by disjoint neighborhoods.

		      Any neighborhood $U$ of $q((1, 0))$ must contain a set $D_{\varepsilon_{0}}$ for some $\varepsilon_{0} \in \openinterval{0,1}$.

		      Let $V$ be a neighborhood of $q((0, 0))$ in $X/A$ then the preimage $q^{-1}(V)$ is a neighborhood containing $A$, since $q^{-1}((0, 0)) = A$. For every $\delta \in \openinterval{0,1}$, $\left(\frac{1}{2}, \delta\right) \in A \subseteq q^{-1}(V)$, so $\left(\frac{1}{2}, \delta\right)$ has a neighborhood $N_{\delta}$ contained in $q^{-1}(V)$.

		      According to part (b), the basis for the topology on $X$ consists of open rectangles (whose vertices are within $\openinterval{0,1}\times\openinterval{0,1}$) and sets of type $C_{\varepsilon}, D_{\varepsilon}$. Because (for every $\delta \in \openinterval{0,1}$) the point $\left(\frac{1}{2}, \delta\right)$ doesn't lie in any set of type $C_{\varepsilon}$ or $D_{\varepsilon}$, there is an open rectangle $R_{\delta}$ such that $\left(\frac{1}{2}, \delta\right) \in R_{\delta} \subseteq N_{\delta} \subseteq q^{-1}(V)$. Consider a $\delta \in \openinterval{0, 1}$ such that $\delta < \varepsilon_{0}$, we have an open rectangle $R_{\delta}$ containing points with $x$-ordinate greater than $\frac{1}{2}$ and $y$-ordinate less than $\varepsilon_{0}$. This implies that $V$ is not disjoint from $D_{\varepsilon_{0}}$, hence $U$ and $V$ are not disjoint.

		      Hence any neighborhoods of $q((0, 0))$ and $q((1, 0))$ are not disjoint. Thus $X/A$ is not Hausdorff.
	\end{enumerate}
\end{proof}

\begin{problem}{3-18}
Let $A\subseteq \mathbb{R}$ be the set of integers, and let $X$ be the quotient space $\mathbb{R}/A$ obtained by collapsing $A$ to a point as in Example 3.52. (We are not using the notation $\mathbb{R}/\mathbb{Z}$ for this space because that has a different meaning, described in Example 3.92.)
\begin{enumerate}[label={(\alph*)}]
	\item Show that $X$ is homeomorphic to a wedge sum of countably infinitely many circles. [Hint: express both spaces as quotients of a disjoint union
			      of intervals.]
	\item Show that the equivalence class $A$ does not have a countable neighborhood in $X$, so $X$ is not first or second countable.
\end{enumerate}
\end{problem}

\begin{proof}
	\begin{enumerate}[label={(\alph*)}]
		\item Consider the countable infinite set of ``circles'' $\set{\mathbb{S}^{1}\times\set{n} : n\in \mathbb{Z}}$. A wedge sum of these pointed circles is
		      \begin{align*}
			      \bigvee_{n\in \mathbb{Z}}\mathbb{S}^{1} & = \left(\coprod_{n\in \mathbb{Z}} \mathbb{S}^{1}\times \set{n}\right) / \set{(1, n) : n\in \mathbb{Z}}
		      \end{align*}

		      We will show that the map $q: \mathbb{R} \to \bigvee_{n\in \mathbb{Z}}\mathbb{S}^{1}$ defined by setting
		      \begin{align*}
			      q(x) = \begin{cases}
				             (e^{2\pi\iota x}, n)                      & \text{if $x \in \openinterval{n, n+1}$ for some integer $n$} \\
				             (1, 0) = (1, \pm 1) = (1, \pm 2) = \cdots & \text{if $x \in \mathbb{Z}$}
			             \end{cases}
		      \end{align*}

		      is a quotient map.

		      Let $U$ be a subset of $\bigvee_{n\in \mathbb{Z}}\mathbb{S}^{1}$. Firstly, $q$ is surjective, so $q(q^{-1}(U)) = U$. Remind that $r_{n}: \closedinterval{n, n+1}\to \mathbb{S}^{1}$ defined by $r_{n}(x) = e^{2\pi\iota x}$ is a quotient map.
		      \begin{equation*}
			      q^{-1}(U) = \bigcup_{n\in\mathbb{Z}}r_{n}^{-1}(U \cap (\mathbb{S}^{1}\times\set{n})).
		      \end{equation*}

		      If $U$ is open then $q^{-1}(U)$ is open $U \cap (\mathbb{S}^{1}\times\set{n})$ because $r_{n}$ is a quotient map for every $n\in\mathbb{Z}$.

		      Conversely, if $q^{-1}(U)$ is open then it is an union of open intervals. Let $(z, n)$ be a point in $U$. If $z\ne 1$ then there is $s_{n} \in \openinterval{n, n+1}$ such that $q(s_{n}) = (z, n)$, and there is $\varepsilon_{n} > 0$ such that
		      \begin{equation*}
			      \openinterval{s_{n} - \varepsilon_{n}, s_{n} + \varepsilon_{n}} \subseteq q^{-1}(U).
		      \end{equation*}

		      Otherwise, $z = 1$ then for every $n\in\mathbb{Z}$, both $n$ and $n+1$ are in $r_{n}^{-1}(U \cap (\mathbb{S}^{1}\times\set{n}))$ so there is $\varepsilon_{n} > 0$ such that
		      \begin{equation*}
			      \halfopenright{n, n + \varepsilon_{n}} \cup \halfopenleft{n+1 - \varepsilon_{n}, n+1} \subseteq r_{n}^{-1}(U \cap (\mathbb{S}^{1}\times\set{n})) \subseteq q^{-1}(U).
		      \end{equation*}

		      In both cases, $(z, n)$ has a neighborhood contained in $\bigvee_{n\in\mathbb{Z}}\mathbb{S}^{1}$, so $U$ is open.

		      Hence $q$ is a quotient map by definition. On the other hand, the quotient map $p: \mathbb{R} \to \mathbb{R}/A$ and $q$ have the same identification, so $\mathbb{R}/A$ and $\bigvee_{n\in \mathbb{Z}}\mathbb{S}^{1}$ are homeomorphic, according to the uniqueness of quotient spaces.

		      Thus $\mathbb{R}/A$ is homeomorphic to a wedge sum of countably infinitely many circles.
		\item Let ${(U_{n})}_{n\in\mathbb{Z}}$ be a family of countably many neighborhoods of the equivalence class $A$. Denote the quotient map $\mathbb{R}\to \mathbb{R}/A$ by $q$.

		      For every $n\in\mathbb{Z}$
		      \begin{itemize}
			      \item the preimage $q^{-1}(U_{n}) \subseteq \mathbb{R}$ is an open set containing $A$,
			      \item there is $\varepsilon_{n} \in \openinterval{0, \frac{1}{2}}$ such that $n \in \openinterval{n - \varepsilon_{n}, n + \varepsilon_{n}} \subseteq q^{-1}(U_{n})$,
			      \item there is $\delta_{n} \in \openinterval{0, \frac{1}{2}}$ such that $\delta_{n} < \varepsilon_{n}$.
		      \end{itemize}

		      ($\openinterval{0, \frac{1}{2}}$ is used to make sure the open intervals $\openinterval{n - \varepsilon_{n}, n + \varepsilon_{n}}$ are pairwise disjoint, $\openinterval{n - \delta_{n}, n + \delta_{n}}$ are pairwise disjoint.)

		      Define a saturated open set $V = \bigcup_{n\in\mathbb{Z}} \openinterval{n - \delta_{n}, n + \delta_{n}} \subseteq \mathbb{R}$. $q(V)$ then is open in $\mathbb{R}/A$. However, $V$ doesn't contain any open interval $\openinterval{n - \varepsilon_{n}, n + \varepsilon_{n}}$ so together with the equivalence relation defined on $\mathbb{R}$, we deduce that $q(V)$ doesn't contain any $U_{n}$.

		      Hence the equivalence class $A$ doesn't have a countable neighborhood basis, so $\mathbb{R}/A$ is not first or second countable.
	\end{enumerate}
\end{proof}

\begin{problem}{3-19}\label{problem:3-19}
Let $G$ be a topological group and let $H\subseteq G$ be a subgroup. Show that its closure $\overline{H}$ is also a subgroup.
\end{problem}

\begin{proof}
	In this proof, we denote by $m$ the map $m: G\times G\to G$ given by $m(g, g') = gg'$ and by $i$ the map $i: G\to G$ given by $i(g) = g^{-1}$. We will show that $m(\overline{H}\times\overline{H}) = \overline{H}$ and $i(\overline{H}) = \overline{H}$.

	$H$ is a subgroup of $G$ so $H$ contains the identity element $e_{G}$ of $G$. $H \subseteq \overline{H}$ so $e_{G} \in \overline{H}$. For every $x \in \overline{H}$, $x = x\cdot e_{G}$, so $\overline{H} \subseteq m(\overline{H}\times\overline{H})$.

	Let $x \in m(\overline{H}\times \overline{H})$ and $U$ be a neighborhood of $x$. There is $(g_{1}, g_{2}) \in \overline{H}\times\overline{H}$ such that $m(g_{1}, g_{2}) = x$, so $(g_{1}, g_{2}) \in m^{-1}(U)$. Since $m$ is continuous, $m^{-1}(U)$ is open and therefore a neighborhood of $(g_{1}, g_{2})$. The collection of product open sets is a basis for the product topology on $G\times G$ so there is a product open set $B_{1}\times B_{2}$ such that $(g_{1}, g_{2}) \in B_{1}\times B_{2} \subseteq m^{-1}(U)$. Because $g_{1}, g_{2} \in \overline{H}$, from Exercise~\ref{exercise:2.9} (d), it follows that $B_{1}$ contains an element of $H$ and $B_{2}$ also contains an element of $H$. So $m^{-1}(U)$ contains an element of $H\times H$, which implies $U$ contains an element of $H$. Hence every neighborhood of $x \in m(\overline{H}\times \overline{H})$ contains an element of $H$. From Exercise~\ref{exercise:2.9} (d), we conclude that $x \in \overline{H}$, so $m(\overline{H}\times \overline{H}) \subseteq \overline{H}$.

	Hence $m(\overline{H}\times\overline{H}) = \overline{H}$, which means $\overline{H}$ is closed under group multiplication.

	\bigskip

	$i$ is continuous. It is also bijective because the inverse of $i$ is itself, so the inverse of $i$ is continuous, therefore $i$ is a homeomorphism, hence an open and closed map. $i(H) = H$ because $H$ is a subgroup of $G$. $i(\overline{H})$ is a closed set containing $H$. Because $\overline{H}$ is the smallest closed set containing $H$, we get $\overline{H} \subseteq i(\overline{H})$.

	Let $x \in i(\overline{H})$ and $U$ be a neighborhood of $x$ in $G$. There is $h\in \overline{H}$ such that $i(h) = x$. Because $i$ is continuous, $i^{-1}(U)$ is open, hence a neighborhood of $h$. Since $h\in \overline{H}$, every neighborhood of $h$ contains an element of $H$. Therefore $U = i(i^{-1}(U))$ contains an element of $i(H) = H$. Hence every neighborhood of $x$ contains an element of $H$, from which we deduce that $x \in \overline{H}$. Hence $i(\overline{H}) \subseteq \overline{H}$.

	So $i(\overline{H}) = \overline{H}$, which implies that $\overline{H}$ is closed under the inverse map $i$.

	Thus the closure $\overline{H}$ of the subgroup $H$ of group $G$ is also a subgroup of $G$.
\end{proof}

\begin{problem}{3-20}
Suppose $G$ is a group that is also a topological space. Show that $G$ is a topological group if and only if the map $G\times G\to G$ given by $(x, y) \mapsto xy^{-1}$ is continuous.
\end{problem}

\begin{proof}
	In this proof, we denote by $m$ the map $m: G\times G\to G$ given by $m(g, g') = gg'$ and by $i$ the map $i: G\to G$ given by $i(g) = g^{-1}$.

	Denote by $f$ the map $G\times G\to G$ given by $(x, y) \mapsto xy^{-1}$. The map $f$ is decomposed into
	\begin{center}
		\begin{tikzcd}
			G\times G \arrow{r}[below]{j} \arrow[bend left=30]{rr}{f} & G\times G \arrow{r}[below]{m} & G
		\end{tikzcd}
	\end{center}

	where $j(x, y) = (x, y^{-1})$, in other words, $j = \operatorname{Id}_{G} \times i$.

	$(\Longrightarrow)$ Suppose $G$ is a topological group.

	$m$ and $i$ are continuous maps. Because $f = m\circ j$ and $j, m$ are continuous, it follows that $f$ is continuous.

	$(\Longleftarrow)$ Suppose $f$ is continuous.

	$i$ is decomposed into
	\begin{center}
		\begin{tikzcd}
			G \arrow{r}[below]{k} \arrow[bend left=30]{rr}{i} & G\times G \arrow{r}[below]{f} & G
		\end{tikzcd}
	\end{center}

	where $k(g) = (e_{G}, g)$. Since $k$ and $f$ are continuous, $i$ is also continuous. On the other hand, $m = f \circ j$, which is a composition of two continuous maps, so $m$ is continuous. Hence $G$ is a topological group by definition.
\end{proof}

\begin{problem}{3-21}
Let $G$ be a topological group and $\Gamma \subseteq G$ be a subgroup.
\begin{enumerate}[label={(\alph*)}]
	\item For each $g\in G$, show that there is a homeomorphism $\theta_{g}: G/\Gamma \to G/\Gamma$ such that the following diagram commutes
	      \begin{figure}[htp]
		      \centering
		      \begin{tikzcd}
			      G \arrow{r}{L_{g}} \arrow{d} & G \arrow{d} \\
			      G/\Gamma \arrow{r}[below]{\theta_{g}} & G/\Gamma
		      \end{tikzcd}
	      \end{figure}
	\item Show that every coset space is topologically homogeneous.
\end{enumerate}
\end{problem}

\begin{proof}
	\begin{enumerate}[label={(\alph*)}]
		\item Denote by $q$ the quotient map $G\to G/\Gamma$.

		      $G/\Gamma$ consists of equivalence classes where $x$ is equivalent to $y$ if and only if $xy^{-1} \in \Gamma$. Denote the equivalence relation by $\sim$, and define $\theta_{g}: G/\Gamma \to G/\Gamma$ by $\theta_{g}([g']) = [gg']$ (it is well-defined!)

		      Because $\theta_{g}\circ \theta_{g^{-1}} = \theta_{g^{-1}}\circ \theta_{g} = \operatorname{Id}_{G/\Gamma}$, it follows that $\theta_{g}$ is bijective. Let $U$ be an open subset of $G/\Gamma$, then
		      \begin{align*}
			      \theta_{g}^{-1}(U) & = \set{ [x] : [x] \in G/\Gamma, \theta_{g}([x]) \in U } \\
			                         & = \set{ [x] : [x] \in G/\Gamma, [gx] \in U }
		      \end{align*}

		      On the other hand
		      \begin{align*}
			      q^{-1}(\theta_{g}^{-1}(U)) & = \bigcup_{[gx] \in U} [gx] \\
			      q^{-1}(U)                  & = \bigcup_{[x] \in U} [x]
		      \end{align*}

		      it follows that the left translation $L_{g}$ maps $\bigcup_{[x] \in U} [x]$ onto $\bigcup_{[gx] \in U} [gx]$ bijectively. Because $L_{g}$ is a homeomorphism and $q^{-1}(U)$ is open, we deduce that $q^{-1}(\theta_{g}^{-1}(U))$ is open, so $\theta_{g}^{-1}(U)$ is open. Hence $\theta_{g}$ is continuous. Analogously, $\theta_{g}^{-1} = \theta_{g^{-1}}$ is continuous, so $\theta_{g}$ is a homeomorphism.

		      Last but not least, for every $x\in G$
		      \begin{equation*}
			      (q\circ L_{g})(x) = q(gx) = [gx] = \theta_{g}([x]) = \theta_{g}(q(x)) = (\theta_{g}\circ q)(x)
		      \end{equation*}

		      so $q\circ L_{g} = \theta_{g}\circ q$, which implies that the diagram commutes.
		\item Consider the coset space $G/\Gamma$. Let $[x], [y] \in G/\Gamma$. There exists $g\in G$ such that $L_{g}(x) = y$. According to part (a), $\theta_{g}([x]) = [y]$ and $\theta_{g}$ is a homeomorphism on $G/\Gamma$, so $G/\Gamma$ is topologically homogeneous, by definition. Thus every coset space is topologically homogeneous.
	\end{enumerate}
\end{proof}

\begin{problem}{3-22}\label{problem:3-22}
Let $G$ be a group acting by homeomorphisms on a topological space $X$, and let $\mathcal{O} \subseteq X\times X$ be the subset defined by
\begin{equation*}
	\mathcal{O} = \set{ (x_{1}, x_{2}) : x_{1} = g\boldsymbol{\cdot} x_{2} \text{ for some $g\in G$} }.
\end{equation*}

It is called the \textbf{orbit relation} because $(x_{1}, x_{2}) \in \mathcal{O}$ if and only if $x_{1}$ and $x_{2}$ are in the same orbit.
\begin{enumerate}[label={(\alph*)}]
	\item Show that the quotient map $X \to X/G$ is an open map.
	\item Conclude that $X/G$ is Hausdorff if and only if $\mathcal{O}$ is closed in $X\times X$.
\end{enumerate}
\end{problem}

\begin{proof}
	\begin{enumerate}[label={(\alph*)}]
		\item Denote by $q$ the quotient map $X \to X/G$. Let $U$ be an open subset of $X$ then $q^{-1}(q(U))$ is the saturation of $U$ in $X$ with respect to $q$. On the other hand, from the definition of orbit relation, the saturation of $U$ is
		      \begin{align*}
			      q^{-1}(q(U)) = \bigcup_{x\in U} \set{ g\boldsymbol{\cdot}x : g \in G } = \bigcup_{g\in G} \set{ g\boldsymbol{\cdot} x : x\in U }
		      \end{align*}

		      Because $G$ acts by homeomorphisms on $X$, then the set $\set{ g\boldsymbol{\cdot} x : x\in U }$ is open for every $g\in G$ (because $x\mapsto g\boldsymbol{\cdot} x$ is a homeomorphism according to the definition of action by homeomorphisms). $q^{-1}(q(U))$ is therefore an union of open sets of $X$, so it is open. Moreover, since $q$ is a quotient map, $q(U)$ is open. Hence $q$ is an open map.
		\item $\mathcal{O}$ is an equivalence relation and $X/G$ is the corresponding quotient. From Proposition 3.57, and the quotient map $X\to X/G$ is open, it follows that $X/G$ is Hausdorff if and only if $\mathcal{O}$ is closed in $X\times X$.
	\end{enumerate}
\end{proof}

\begin{problem}{3-23}
Suppose $\Gamma$ is a normal subgroup of the topological group $G$. Show that the quotient group $G/\Gamma$ is a topological group with the quotient topology. [Hint: it might be helpful to use Problems 3-5 and 3-22.]
\end{problem}

\begin{proof}
	The group multiplication of $G$ is given by $m_{G}: G\times G\to G$ and that of $G/\Gamma$ is given by $m_{G/\Gamma}: (G/\Gamma)\times (G/\Gamma) \to G/\Gamma$. $i_{G}: G\to G$ is the inverse operation on $G$ and $i_{G/\Gamma}: G/\Gamma \to G/\Gamma$ is the inverse operation on $G/\Gamma$.

	Because $q$ is continuous, the product map $q\times q: G\times G \to (G/\Gamma)\times (G/\Gamma)$ is also continuous. According to Problem~\ref{problem:3-22}, the quotient map $q: G \to G/\Gamma$ is open. By Problem~\ref{problem:3-5}, $q\times q$ is open. $q\times q$ is continuous, open, and surjective, so it is a quotient map.

	The following diagrams commute (see Figure~\ref{fig:3-22}).
	\begin{figure}[htp]
		\centering
		\renewcommand{\thefigure}{3-22}
		\begin{tikzpicture}[every edge/.style = {draw, -latex, thick}]
			\matrix (m) [matrix of math nodes, row sep=3em, column sep=3em]
			{
				G\times G                   & G        \\
				(G/\Gamma)\times (G/\Gamma) & G/\Gamma \\
			};

			\path[->] (m-1-1) edge node[above] {$m_{G}$} (m-1-2);
			\path[->] (m-2-1) edge node[below] {$m_{G/\Gamma}$} (m-2-2);
			\path[->] (m-1-1) edge node[left] {$q\times q$} (m-2-1);
			\path[->] (m-1-2) edge node[right] {$q$} (m-2-2);
		\end{tikzpicture}
		\qquad
		\begin{tikzpicture}[every edge/.style = {draw, -latex, thick}]
			\matrix (m) [matrix of math nodes, row sep=3em, column sep=3em]
			{
				G        & G        \\
				G/\Gamma & G/\Gamma \\
			};

			\path[->] (m-1-1) edge node[above] {$i_{G}$} (m-1-2);
			\path[->] (m-2-1) edge node[below] {$i_{G/\Gamma}$} (m-2-2);
			\path[->] (m-1-1) edge node[left] {$q$} (m-2-1);
			\path[->] (m-1-2) edge node[right] {$q$} (m-2-2);
		\end{tikzpicture}
		\caption{}\label{fig:3-22}
	\end{figure}

	Let $U$ be an open set of $G/\Gamma$, then $q^{-1}(U)$ is open in $G$. The following set is open in $G\times G$
	\begin{align*}
		{(q \times q)}^{-1}(m_{G/\Gamma}^{-1}(U)) & = {(m_{G/\Gamma} \circ (q\times q))}^{-1}(U) \\
		                                          & = {(q\circ m_{G})}^{-1}(U)                   \\
		                                          & = m_{G}^{-1}(q^{-1}(U))
	\end{align*}

	because $q$ and $m_{G}$ are continuous. Since $q\times q$ is a quotient map, it follows that $m_{G/\Gamma}^{-1}(U)$ is open in $(G/\Gamma)\times (G/\Gamma)$. Hence $m_{G/\Gamma}$ is continuous. On the other hand
	\begin{align*}
		q^{-1}(i_{G/\Gamma}^{-1}(U)) = {(i_{G/\Gamma}\circ q)}^{-1}(U) = {(q\circ i_{G})}^{-1}(U)
	\end{align*}

	is open because $q$ and $i_{G}$ are continuous, so $q^{-1}(i_{G/\Gamma}^{-1}(U))$ is open. By the definition of quotient map, we conclude that $i_{G/\Gamma}$ is continuous.

	Thus the quotient group $G/\Gamma$ with the quotient topology induced by $G\to G/\Gamma$ is a topological group.
\end{proof}

\begin{problem}{3-24}
Consider the action of $\operatorname{O}(n)$ on $\mathbb{R}^{n}$ by matrix multiplication as in Example 3.88(b). Prove that the quotient space is homeomorphic to $\halfopenright{0, \infty}$. [Hint: consider the function $f: \mathbb{R}^{n} \to \halfopenright{0, \infty}$ given by $f(x) = \abs{x}$.]
\end{problem}

\begin{proof}
	Firstly, we prove that $f$ is a quotient map. A basis for the topology on $\halfopenright{0, \infty}$ (inherited from $\mathbb{R}$ with the Euclidean topology) is the collection of intervals $\halfopenright{0, c}$ and $\openinterval{a, b}$ where $a, b, c$ are positive real numbers and $a < b$. Let $U$ be a subset of $\halfopenright{0, \infty}$.

	Suppose $U$ is open. To prove that $f^{-1}(U)$ is open, it suffices to solve two cases $U = \halfopenright{0, c}$ and $U = \openinterval{a, b}$ because these types of sets constitute a basis for the topology on $\halfopenright{0, \infty}$ and the preimage of an union is the union of preimages. For every positive real numbers $a, b, c$ such that $a < b$
	\begin{align*}
		f^{-1}(\halfopenright{0, c}) & = B^{n}_{c}(0)                                                                                           \\
		f^{-1}(\openinterval{a, b})  & = \mathbb{R}^{n} \smallsetminus \left( \set{x : \abs{x} \leq a } \cup \set{ x : \abs{x} \geq b } \right)
	\end{align*}

	are open subsets of $\mathbb{R}^{n}$, so $f^{-1}(U)$ is open.

	Conversely, suppose $f^{-1}(U)$ is open. If $c\in U$ and $c = 0$ then $0 \in f^{-1}(U)$ so there is an open ball $B^{n}_{r}(0) \subseteq f^{-1}(U)$ and it follows that $\halfopenright{0, r} \subseteq U$. If $c\in U$ and $c \ne 0$, let $x_{0} \in f^{-1}(\set{c})$ then there is $r > 0$ such that $x_{0} \in B^{n}_{r}(x_{0}) \subseteq f^{-1}(U)$. We can even assume that $r < c$ because the open ball with smaller radius is contained in the open ball with larger radius given that the two open balls have the same center. The points $x_{0} \pm x_{0}\frac{r}{c}$ are in $B^{n}_{r}(x_{0})$, and $f\left(x_{0} \pm x_{0}\frac{r}{c}\right) = c \pm r$, so $c \in \openinterval{c - r, c + r} \subseteq U$. Hence every point of $U$ has a neighborhood contained in $U$, which means $U$ is open in $\halfopenright{0, \infty}$.

	Therefore $U$ is open if and only if $f^{-1}(U)$ is open. Moreover, $f$ is surjective, so $f$ is a quotient map.

	Consider the quotient map $q: \mathbb{R}^{n} \to \mathbb{R}^{n}/\operatorname{O}(n)$. $q(x) = q(y)$ if and only if there is $A \in \operatorname{O}(n)$ such that $y = Ax$. Suppose there is $A \in \operatorname{O}(n)$ such that $y = Ax$. Because $A$ is an orthogonal matrix, $AA^{\bot} = A^{\bot}A = I$ and
	\begin{equation*}
		\abs{y}^{2} = y^{\bot}y = {(Ax)}^{\bot}(Ax) = x^{\bot}(A^{\bot}A)x = x^{\bot}x = \abs{x}^{2}
	\end{equation*}

	so $\abs{y} = \abs{x}$. Conversely, suppose that $\abs{y} = \abs{x}$. Let $e_{1} = x/\abs{x}$ and $f_{1} = y/\abs{y}$, $e_{1}, \ldots, e_{n}$ and $f_{1}, \ldots, f_{n}$ be orthonormal bases for $\mathbb{R}^{n}$. Let $A$ be the matrix of an unitary operator on $\mathbb{R}^{n}$ that maps $e_{i}$ to $f_{i}$ for every $i \in \set{1, \ldots, n}$, then $y = Ax$. Hence there is $A \in \operatorname{O}(n)$ such that $y = Ax$ if and only if $\abs{x} = \abs{y}$, in other words, $q(x) = q(y)$ if and only if $f(x) = f(y)$.

	From the uniqueness of quotient spaces, we conclude that the orbit space $\mathbb{R}^{n}/\operatorname{O}(n)$ and $\halfopenright{0, \infty}$ are homeomorphic.
\end{proof}
