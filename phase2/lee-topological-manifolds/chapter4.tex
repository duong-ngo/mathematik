% chktex-file 8
\chapter{Connectedness and Compactness}

\section*{Connectedness}\addcontentsline{toc}{section}{Connectedness}

\subsection*{Definitions and Basic Properties}\addcontentsline{toc}{subsection}{Definitions and Basic Properties}

\begin{exercise}{4.3}
	Suppose $X$ is a connected topological space, and $\sim$ is an equivalence relation on $X$ such that every equivalence class is open. Show that there is exactly one equivalence class, namely $X$ itself.
\end{exercise}

\begin{proof}
	Suppose $X$ has exactly $n$ equivalence classes, all of which are open. Because all equivalence classes constitute a partition of $X$, then $X$ is the union of these open, disjoint equivalence classes. On the other hand, $X$ is connected, so $n = 1$. Hence $X$ has exactly one equivalence class.
\end{proof}

\begin{exercise}{4.4}
	Prove that a topological space $X$ is disconnected if and only if there exists a nonconstant continuous function from $X$ to the discrete space $\set{0,1}$.
\end{exercise}

\begin{proof}
	Suppose there is a nonconstant continuous function $f: X\to \set{0,1}$, then $f^{-1}(0)$ and $f^{-1}(1)$ are nonempty, open, disjoint, and their union is $X$. Therefore $X$ is disconnected.

	Conversely, suppose $X$ is disconnected, then there are nonempty, open, disjoint subsets $U, V\subseteq X$ such that $X = U\cup V$. Define $f: X\to \set{0,1}$ as follows: $f(x) = 0$ if $x\in U$ and $f(x) = 1$ if $x\notin U$, then $f$ is nonconstant and $f$ is continuous (because the preimages under $f$ of $\varnothing, \set{0}, \set{1}, \set{0, 1}$ are open in $X$).
\end{proof}

\begin{exercise}{4.5}
	Prove that a topological space is disconnected if and only if it is homeomorphic to a disjoint union of two or more nonempty spaces.
\end{exercise}

\begin{proof}
	Let $X$ be a topological space.

	Suppose $X$ is disconnected, then there are subsets $U, V\subseteq X$ that disconnect $X$. The map $\varphi: X \to U\sqcup V$ given by
	\begin{equation*}
		\varphi(x) = \begin{cases}
			(x, 0) & \text{if $x \in U$} \\
			(x, 1) & \text{if $x \in V$}
		\end{cases}
	\end{equation*}

	is a homeomorphism, so $X$ is homeomorphic to the disjoint union $U\sqcup V$. Hence $X$ is homeomorphic to a disjoint union of at least two nonempty spaces.

	Conversely, suppose $X$ is homeomorphic to a disjoint union of $n\geq 2$ nonempty spaces $U_{1}, \ldots, U_{n}$. Let $f: X\to \coprod^{n}_{i=1}U_{i}$ be a homeomorphism, then $f^{-1}(U_{1}), \ldots, f^{-1}(U_{n})$ constitutes a partition on $X$, and $f^{-1}(U_{1}), \ldots, f^{-1}(U_{n})$ are nonempty open subsets of $X$. Because $n\geq 2$, it follows that $X$ is disconnected.
\end{proof}

\begin{exercise}{4.10}
	Suppose $M$ is a connected manifold with nonempty boundary. Show that its double $D(M)$ is connected.
\end{exercise}

\begin{proof}
	Let $h: \partial M\to \partial M$ be the identity map from the boundary of $M$ onto itself. $\sim$ is the equivalence relation on $D(M)$ defined by $x\sim h(x)$ for $x\in\partial M$, and other points are equivalent to itself. By the definition of the double of a manifold with boundary
	\begin{equation*}
		D(M) = M\cup_{h}M = (M\sqcup M)/_{\sim}.
	\end{equation*}

	Assume that $D(M)$ is disconnected by open subsets $U, V$. Let $q: M \sqcup M \to (M \sqcup M)/_{\sim}$ be the quotient map, then $q^{-1}(U)$ and $q^{-1}(V)$ disconnect $M\sqcup M \approx (M\times\set{0}) \cup (M\times\set{1})$. Because $M\times\set{0}, M\times\set{1} \subseteq q^{-1}(U) \cup q^{-1}(V)$ and $M\times\set{0}, M\times\set{1}$ are connected, it follows from Proposition 4.9 that either $M\times\set{0} \subseteq q^{-1}(U)$ or $M\times\set{0} \subseteq q^{-1}(V)$, either $M\times\set{1} \subseteq q^{-1}(U)$ or $M\times\set{1} \subseteq q^{-1}(V)$. Because $q^{-1}(U)$ and $q^{-1}(V)$ are nonempty, it follows that either $M\times\set{0} \subseteq q^{-1}(U)$ and $M\times\set{1} \subseteq q^{-1}(V)$ or $M\times\set{0} \subseteq q^{-1}(V)$ and $M\times\set{1} \subseteq q^{-1}(U)$. Without loss of generality, suppose that the former is the case. Since the boundary $\partial M$ is nonempty, there is $a \in \partial M$. $a \sim h(a)$, $(a, 0) \in M\times\set{0} = q^{-1}(U)$ and $(a, 1) \in M\times\set{1} = q^{-1}(V)$ so $q(a) = q(h(a)) \in U, V$. Hence $U$ and $V$ are not disjoint, which contradicts our assumption.

	Thus $D(M)$ is connected.
\end{proof}

\subsection*{Path Connectedness}\addcontentsline{toc}{subsection}{Path Connectedness}

\begin{exercise}{4.14}\label{exercise:4.14}
	Prove Proposition 4.13 (Properties of Path-Connected Spaces).
	\begin{enumerate}[label={(\alph*)}]
		\item Every continuous image of a path-connected space is path-connected.
		\item Let $X$ be a space, and let ${\{ B_{\alpha} \}}_{\alpha\in A}$ be a collection of path-connected subspaces of $X$ with a point in common. Then $\bigcup_{\alpha\in A}B_{\alpha}$ is path-connected.
		\item Every product of finitely many path-connected spaces is path-connected.
		\item Every quotient space of a path-connected space is path-connected.
	\end{enumerate}
\end{exercise}

\begin{proof}
	\begin{enumerate}[label={(\alph*)}]
		\item Let $f: X\to Y$ be a continuous map and $X$ is a path-connected space. Let $p, q$ be two points of $f(X)$, let $a \in f^{-1}(p)$ and $b \in f^{-1}(q)$. Because $X$ is path-connected, there is a continuous map $g: \closedinterval{0, 1} \to X$ such that $g(0) = a$ and $g(1) = b$, so the composition $f\circ g: \closedinterval{0, 1} \to f(X)$ is continuous and $(f\circ g)(0) = p$ and $(f\circ g)(1) = q$. Hence for every two points $p, q$ of $f(X)$, there is a continuous map from $\closedinterval{0, 1}$ to $f(X)$ such that the images of $0, 1$ are $p, q$, respectively, which implies $f(X)$ is path-connected.
		\item Let $p, q$ be two points of $\bigcup_{\alpha\in A}B_{\alpha}$.

		      $p \in B_{\alpha_{p}}$ and $q \in B_{\alpha_{q}}$ for some $\alpha_{p}, \alpha_{q} \in A$. Let $x \in \bigcap_{\alpha\in A}B_{\alpha}$ (these sets have a point in common). Because $B_{\alpha_{p}}, B_{\alpha_{q}}$ are path-connected, there are continuous maps $f_{p}: \closedinterval{0, 1} \to B_{\alpha_{p}}$ such that $f_{p}(0) = p, f_{p}(1) = x$, and $f_{q}: \closedinterval{0, 1} \to B_{\alpha_{q}}$ such that $f_{q}(0) = x, f_{q}(1) = q$.

		      The maps $g: \closedinterval{0, \frac{1}{2}} \to \closedinterval{0, 1}$ given by $g(t) = 2t$ and $h: \closedinterval{\frac{1}{2}, 1} \to \closedinterval{0, 1}$ given by $h(t) = 2t - 1$ are continuous. The compositions $f_{p}\circ g$ and $f_{q}\circ h$ are therefore continuous, and they agree on $\closedinterval{0, \frac{1}{2}} \cap \closedinterval{\frac{1}{2}, 1}$, since $(f_{p}\circ g)(1/2) = x = (f_{q}\circ h)(1/2)$. $\closedinterval{0, \frac{1}{2}}, \closedinterval{\frac{1}{2}, 1}$ constitute a finite closed cover of $\closedinterval{0, 1}$, so by the gluing lemma, there is a unique continuous map $f: \closedinterval{0, 1} \to B_{\alpha_{p}} \cup B_{\alpha_{q}}$ such that $f\vert_{\closedinterval{0, \frac{1}{2}}} = f_{p}\circ g$ and $f\vert_{\closedinterval{\frac{1}{2}, 1}} = f_{q}\circ h$. Moreover, $f(0) = p, f(1) = q$. Hence there is a path in $\bigcup_{\alpha\in A}B_{\alpha}$ from $p$ to $q$.

		      Thus $\bigcup_{\alpha\in A}B_{\alpha}$ is path-connected.
		\item It suffices to prove that the product of two path-connected spaces is path-connected.

		      Let $X, Y$ be path-connected spaces and $(x_{1}, y_{1}), (x_{2}, y_{2})$ are two points of $X\times Y$. The maps $i_{y_{0}}: X\to X\times Y$ given by $i_{y_{0}}(x) = (x, y_{0})$ and $i_{x_{0}}: Y\to X\times Y$ given by $i_{x_{0}}(y) = (x_{0}, y)$ are continuous. From part (a), it follows that $X\times\set{y_{0}}$ and $\set{x_{0}}\times Y$ are path-connected for every $y_{0} \in Y, x_{0}\in X$. Hence there is a path $f_{1}$ in $X\times Y$ from $(x_{1}, y_{1})$ to $(x_{2}, y_{1})$ and a path $f_{2}$ in $X\times Y$ from $(x_{2}, y_{1})$ to $(x_{2}, y_{2})$.

		      The maps $g: \closedinterval{0, \frac{1}{2}} \to \closedinterval{0, 1}$ given by $g(t) = 2t$ and $h: \closedinterval{\frac{1}{2}, 1} \to \closedinterval{0, 1}$ given by $h(t) = 2t - 1$ are continuous. The compositions $f_{1}\circ g$ and $f_{2}\circ h$ are therefore continuous, and they agree on $\closedinterval{0, \frac{1}{2}} \cap \closedinterval{\frac{1}{2}, 1}$, since $(f_{1}\circ g)(1/2) = (x_{2}, y_{1}) = (f_{2}\circ h)(1/2)$. From the gluing lemma, it follows that there is a unique continuous map $f: \closedinterval{0, 1} \to X\times Y$ such that $f\vert_{\closedinterval{0, \frac{1}{2}}} = f_{1}\circ g$ and $f\vert_{\closedinterval{\frac{1}{2}, 1}} = f_{2}\circ h$. Moreover, $f(0) = (x_{1}, y_{1})$ and $f(1) = (x_{2}, y_{2})$.

		      Therefore $X\times Y$ is path-connected. From mathematical induction, it follows that the finite product of path-connected spaces is path-connected.
		\item Since every quotient map is continuous and surjective, from part (a), it follows that every quotient space of a path-connected space is path-connected.
	\end{enumerate}
\end{proof}

\subsection*{Components and Path Components}\addcontentsline{toc}{subsection}{Components and Path Components}

\begin{exercise}{4.22}
	Prove Proposition 4.21 (Properties of Path Components).

	Let $X$ be any space.
	\begin{enumerate}[label={(\alph*)}]
		\item The path components of $X$ form a partition of $X$.
		\item Each path component is contained in a single component, and each component is a disjoint union of path components.
		\item Any nonempty path-connected subset of $X$ is contained in a single path component.
	\end{enumerate}
\end{exercise}

\begin{proof}
	\begin{enumerate}[label={(\alph*)}]
		\item Let $U, V$ be non-disjoint path components of $X$. From Exercise~\ref{exercise:4.14} (b), $U\cup V$ is path-connected. Due to the maximality of path components, $U = V = U\cup V$, from which we deduce that non-disjoint path components are identical. Hence distinct path components are disjoint.

		      Let $x$ be a point of $X$. The singleton $\set{x}$ is a path component containing $x$. Let ${(B_{\alpha})}_{\alpha\in A}$ be the family of all path-connected sets containing $x$, then $\bigcup_{\alpha\in A}B_{\alpha}$ is path-connected (according to Exercise~\ref{exercise:4.14} (b)). Moreover $\bigcup_{\alpha\in A}B_{\alpha}$ is a maximal path-connected set so it is a path component containing $x$. Therefore every element of $X$ is in a path component.

		      Thus the path components of $X$ form a partition of $X$.
		\item Let $P$ be a path component of $X$. Because the components of $X$ form a partition of $X$, $P$ has a common point with some component $C$ of $X$. Since a path-connected set is also connected, $P$ is a connected set. Because the union of connected sets with a point in common is connected, $P\cup C$ is connected. From the maximality of $C$, we deduce that $P\cup C = C$, which means $P\subseteq C$. Moreover, distinct components are disjoint, so $P$ is contained in the component $C$ only. Hence every path component is contained in a single component.

		      Let $C$ be a component of $X$. Every point $p$ of $C$ is in some path component of $X$, so the path component containing $p$ is contained in $C$. On the other hand, distinct path components are disjoint. Hence $C$ is a disjoint union of path components.
		\item Let $A$ be a nonempty path-connected subset of $X$.

		      Let $x$ be an element of $A$. According to part (a), $x$ is a point of a path component $P$. According to Exercise~\ref{exercise:4.14} (b), $A\cup P$ is path-connected. Because $P$ is a maximal path-connected set, it follows that $A\cup P = P$, which means $A\subseteq P$.

		      If $A$ is contained in two path components, then the two path components are not disjoint (because $A$ is nonempty) and it follows that the two path components are identical (according to part (a)).

		      Therefore any nonempty path-connected subset of $X$ is contained in a single path component of $X$.
	\end{enumerate}
\end{proof}

\begin{exercise}{4.24}
	Prove Proposition 4.23.

	Every manifold (with or without boundary) is locally connected and locally path-connected.
\end{exercise}

\begin{proof}
	Let $M$ be an $n$-manifold (with or without boundary).

	\textbf{Case 1. $M$ is an $n$-manifold without boundary.}

	According to Problem~\ref{problem:2-23}, every manifold has a basis of coordinate balls. On the other hand, every coordinate ball is homeomorphic to an open ball of $\mathbb{R}^{n}$ and every open ball of $\mathbb{R}^{n}$ is path-connected because every open ball of $\mathbb{R}^{n}$ is homeomorphic to $\mathbb{R}^{n}$ (which is path-connected). Therefore every coordinate ball is path-connected (and hence connected). Hence $M$ is locally path-connected and locally connected.

	\textbf{Case 2. $M$ is an $n$-manifold with boundary.}

	Firstly, we construct a basis for $M$. Let $U$ be a nonempty open subset of $M$ and $x\in U$, then $x$ is in the domain of an interior chart or that of a boundary chart.

	If $x$ is in the domain of an interior chart $(V, \varphi_{x})$, then $\varphi_{x}(V)$ is an open subset of $\mathbb{R}^{n}$. $U\cap V$ is homeomorphic to $\varphi_{x}(U\cap V)$ and $\varphi_{x}(U\cap V)$ is an open subset of $\mathbb{R}^{n}$. Because $\varphi_{x}(U\cap V)$ is open and $\varphi_{x}(x)$ is a point of this set, there is an open ball $B_{r}(\varphi_{x}(x)) \subseteq \varphi_{x}(U\cap V)$. Therefore $\varphi_{x}^{-1}(B_{r}(\varphi_{x}(x)))$ and $B_{r}(\varphi_{x}(x))$ are homeomorphic. So $x$ is in the domain of the following interior chart
	\begin{equation*}
		(\varphi_{x}^{-1}(B_{r}(\varphi_{x}(x))), \varphi_{x}\vert_{\varphi_{x}^{-1}(B_{r}(\varphi_{x}(x)))})
	\end{equation*}

	where the domain is contained in $U$ and is homeomorphic to an open ball in $\mathbb{R}^{n}$.

	If $x$ is in the domain of a boundary chart $(V, \varphi_{x})$, then $\varphi_{x}(x) \in \partial\mathbb{H}^{n}$ and $\varphi_{x}(V)$ is an open subset of $\mathbb{H}^{n}$. $U\cap V$ is homeomorphic to $\varphi_{x}(U\cap V)$ and $\varphi_{x}(U\cap V)$ is an open subset of $\mathbb{H}^{n}$. Because $\varphi_{x}(U\cap V)$ is open and $\varphi_{x}(x)$ is a point of this set, there is an open ball $B_{r}(\varphi_{x}(x))$ in $\mathbb{R}^{n}$ such that $B_{r}(\varphi_{x}(x)) \cap \mathbb{H}^{n} \subseteq \varphi_{x}(U\cap V)$ (here we make use of the subspace topology on $\mathbb{H}^{n}$ and the basis for $\mathbb{R}^{n}$ containing open balls). Therefore $\varphi_{x}^{-1}(B_{r}(\varphi_{x}(x)) \cap \mathbb{H}^{n})$ and $B_{r}(\varphi_{x}(x)) \cap \mathbb{H}^{n}$ are homeomorphic, and $x$ is in the domain of the following boundary chart
	\begin{equation*}
		(\varphi_{x}^{-1}(B_{r}(\varphi_{x}(x)) \cap \mathbb{H}^{n}), \varphi_{x}\vert_{\varphi_{x}^{-1}(B_{r}(\varphi_{x}(x)) \cap \mathbb{H}^{n})})
	\end{equation*}

	where the domain is contained in $U$ and is homeomorphic to a half of an open ball in $\mathbb{H}^{n}$ (it is halved by taking intersection of $\mathbb{H}^{n}$ and an open ball in $\mathbb{R}^{n}$ with center on $\partial \mathbb{H}^{n}$).

	Hence the collection of open sets of $M$ which are domains of some interior chart (and homeomorphic to some open ball in $\mathbb{R}^{n}$) or some boundary chart  (and homeomorphic to half of some open ball in $\mathbb{H}^{n}$) constitutes a basis for the $n$-manifold with boundary $M$.

	On the other hand, a half of an open ball is path-connected (because it is a convex set), so $M$ has a basis of path-connected (hence connected) open sets. Therefore $M$ is locally path-connected and locally connected.

	From the two cases, we conclude that every manifold (without or with boundary) is locally connected and locally path-connected.
\end{proof}

\section*{Compactness}\addcontentsline{toc}{section}{Compactness}

\subsection*{Definitions and Basic Properties}\addcontentsline{toc}{subsection}{Definitions and Basic Properties}

\begin{exercise}{4.28}
	Prove Lemma 4.27 (Compactness Criterion for Subspaces).

	If $X$ is any topological space, a subset $A\subseteq X$ is compact (in the subspace topology) if and only if every cover of $A$ by open subsets of $X$ has a finite subcover.
\end{exercise}

\begin{exercise}{4.29}
	In any topological space $X$, show that every union of finitely many compact subsets of $X$ is compact.
\end{exercise}

\begin{exercise}{4.37}
	Suppose $M$ is a compact manifold with boundary. Show that the double of $M$ is compact.
\end{exercise}

\begin{exercise}{4.38}
	Let $X$ be a compact space, and suppose $\{ F_{n} \}$ is a countable collection of nonempty closed subsets of $X$ that are \textbf{nested}, which means that $F_{n}\supseteq F_{n+1}$ for each $n$. Show that $\bigcap_{n}F_{n}$ is nonempty.
\end{exercise}

\subsection*{Sequential and Limit Point Compactness}\addcontentsline{toc}{subsection}{Sequential and Limit Point Compactness}

\begin{exercise}{4.49}
	Prove the preceding three theorems.

	Theorem 4.46 (Bolzano-Weierstraß). Every bounded sequence in $\mathbb{R}^{n}$ has a convergent subsequence.

	Theorem 4.47. Endowed with the Euclidean metric, a subset of $\mathbb{R}^{n}$ is a complete metric space if and only if it is closed in $\mathbb{R}^{n}$. In particular, $\mathbb{R}^{n}$ is complete.

	Theorem 4.48. Every compact metric space is complete.
\end{exercise}

\subsection*{The Closed Map Lemma}\addcontentsline{toc}{subsection}{The Closed Map Lemma}

\begin{exercise}{4.58}
	Using the map of Example 4.55, show that there is a coordinate ball in $\mathbb{S}^{n}$ whose closure is equal to all of $\mathbb{S}^{n}$.
\end{exercise}

\begin{exercise}{4.61}
	Complete the proof of Proposition 4.60 by showing that $\mathscr{B}$ is a basis.
\end{exercise}

\begin{exercise}{4.62}
	Prove that every manifold with boundary has a countable basis consisting of regular coordinate balls and half-balls.
\end{exercise}

\section*{Local Compactness}\addcontentsline{toc}{section}{Local Compactness}

\section*{Paracompactness}\addcontentsline{toc}{section}{Paracompactness}

\subsection*{Normal Spaces}\addcontentsline{toc}{subsection}{Normal Spaces}

\subsection*{Partition of Unity}\addcontentsline{toc}{subsection}{Partition of Unity}

\section*{Proper Maps}\addcontentsline{toc}{section}{Proper Maps}

\section*{Problems}

\begin{problem}{4-1}
Show that for $n > 1$, $\mathbb{R}^{n}$ is not homeomorphic to any open subset of $\mathbb{R}$.
\end{problem}

\begin{proof}
	Let $n$ be a positive integer greater than $1$. Assume that $\mathbb{R}^{n}$ is homeomorphic to an open subset $U\subseteq \mathbb{R}$, then $U$ is nonempty and there is a homeomorphism $\varphi: \mathbb{R}^{n} \to U$. Let $p$ be a point of $U$. Since $U\subseteq \mathbb{R}$ is open, $U$ is an union of disjoint open intervals, and $x$ lies in one of those open intervals, denote such open interval by $\openinterval{a, b}$, then $\openinterval{a, b}\smallsetminus\set{p}$ is disconnected. Therefore $U\smallsetminus\set{p}$ is disconnected. Because $\varphi$ is a homeomorphism, $\varphi^{-1}(U\smallsetminus\set{p}) = \mathbb{R}^{n} \smallsetminus \set{\varphi^{-1}(p)}$.

	We will show that $\mathbb{R}^{n}\smallsetminus\set{0}$ is path-connected. Let $x, y$ be two points of $\mathbb{R}^{n}\smallsetminus\set{0}$. Because $n > 1$, there exists a nonzero vector $v\in \mathbb{R}^{n}$ such that $x, y$ are not multiples of $v$. Let $v_{x} = \frac{\abs{x}}{\abs{v}}v$ and $v_{y} = \frac{\abs{y}}{\abs{v}}v$. We will construct
	\begin{itemize}
		\item a path in $\mathbb{R}^{n}\smallsetminus\set{0}$ from $x$ to $v_{x}$

		      Note that $\abs{x} = \abs{v_{x}}$. $x = (x_{1}, \ldots, x_{n})$ and $v_{x} = (v_{x,1}, \ldots, v_{x,n})$.

		      There exist $\varphi_{x,1}, \ldots, \varphi_{x,n-1} \in \mathbb{R}$ such that
		      \begin{align*}
			      x_{1}   & = \abs{x}\cos(\varphi_{x,1})                                                 \\
			      x_{2}   & = \abs{x}\sin(\varphi_{x,1})\cos(\varphi_{x,2})                              \\
			      \cdots  &                                                                              \\
			      x_{n-1} & = \abs{x}\sin(\varphi_{x,1})\cdots\sin(\varphi_{x,n-2})\cos(\varphi_{x,n-1}) \\
			      x_{n}   & = \abs{x}\sin(\varphi_{x,1})\cdots\sin(\varphi_{x,n-2})\sin(\varphi_{x,n-1})
		      \end{align*}

		      Also there exist $\varphi_{v_{x},1}, \ldots, \varphi_{v_{x},n-1} \in \mathbb{R}$ such that
		      \begin{align*}
			      v_{x,1}   & = \abs{x}\cos(\varphi_{v_{x},1})                                                         \\
			      v_{x,2}   & = \abs{x}\sin(\varphi_{v_{x},1})\cos(\varphi_{v_{x},2})                                  \\
			      \cdots    &                                                                                          \\
			      v_{x,n-1} & = \abs{x}\sin(\varphi_{v_{x},1})\cdots\sin(\varphi_{v_{x},n-2})\cos(\varphi_{v_{x},n-1}) \\
			      v_{x,n}   & = \abs{x}\sin(\varphi_{v_{x},1})\cdots\sin(\varphi_{v_{x},n-2})\sin(\varphi_{v_{x},n-1})
		      \end{align*}

		      The maps $f_{i}: \closedinterval{0, 1} \to \mathbb{R}$ given by
		      \begin{equation*}
			      f_{i}(t) = \abs{x}\sin((1-t)\varphi_{x,1} + t\varphi_{v_{x},1})\cdots \sin((1-t)\varphi_{x,i-1} + t\varphi_{v_{x},i-1})\cos((1-t)\varphi_{x,i} + t\varphi_{v_{x},i})
		      \end{equation*}

		      if $i < n$ and
		      \begin{equation*}
			      f_{n}(t) = \abs{x}\sin((1-t)\varphi_{x,1} + t\varphi_{v_{x},1})\cdots \sin((1-t)\varphi_{x,n-1} + t\varphi_{v_{x},n-1})
		      \end{equation*}

		      are continuous. So ${(f_{1}(t))}^{2} + \cdots + {(f_{n}(t))}^{2} = \abs{x}^{2} \ne 0$ for every $t\in \closedinterval{0,1}$. Hence the map $f_{x}: \closedinterval{0, 1} \to \mathbb{R}^{n}\smallsetminus\set{0}$ given by
		      \begin{equation*}
			      f_{x}(t) = (f_{1}(t), \ldots, f_{n}(t))
		      \end{equation*}

		      is continuous, due to the characteristic property of product topology. Hence $f$ is a path in $\mathbb{R}^{n}\smallsetminus\set{0}$ from $x$ to $v_{x}$.
		\item a path in $\mathbb{R}^{n}\smallsetminus\set{0}$ from $v_{x}$ to $v_{y}$

		      The map $f: \closedinterval{0, 1} \to \mathbb{R}^{n}\smallsetminus\set{0}$ given by
		      \begin{equation*}
			      f(t) = (1 - t)v_{x} + tv_{y}
		      \end{equation*}

		      is continuous (this map is well-defined because the line segment connecting $v_{x}$ and $v_{y}$ lies entirely in $\mathbb{R}^{n}\smallsetminus\set{0}$) so there is a path in $\mathbb{R}^{n}$ from $v_{x}$ to $v_{y}$.
		\item a path in $\mathbb{R}^{n}\smallsetminus\set{0}$ from $v_{y}$ to $y$

		      Similar to the first contruction, we can construct a path $f_{y}$ in $\mathbb{R}^{n}\smallsetminus\set{0}$ from $v_{y}$ to $y$.
	\end{itemize}

	From these constructions, we deduce that there are continuous maps $g_{x}: \closedinterval{0, \frac{1}{3}} \to \mathbb{R}^{n}\smallsetminus\set{0}$ such that $g_{x}(0) = x$ and $g_{x}(1/3) = v_{x}$, $g: \closedinterval{\frac{1}{3}, \frac{2}{3}} \to \mathbb{R}^{n}\smallsetminus\set{0}$ such that $g(1/3) = v_{x}$ and $g(2/3) = v_{y}$, $g_{y}: \closedinterval{\frac{2}{3}, 1} \to \mathbb{R}^{n}\smallsetminus\set{0}$ such that $g_{y}(2/3) = v_{y}$ and $g_{y}(1) = y$. By the gluing lemma, there is a unique continuous map $f: \closedinterval{0, 1} \to \mathbb{R}^{n}\setminus \set{0}$ such that $f\vert_{\closedinterval{0, \frac{1}{3}}} = g_{x}$, $f\vert_{\closedinterval{\frac{1}{3}, \frac{2}{3}}} = g$, $f\vert_{\closedinterval{\frac{2}{3}, 1}} = g_{y}$. Hence there is a path in $\mathbb{R}^{n}\smallsetminus\set{0}$ from $x$ to $y$.

	Back to the set $\mathbb{R}^{n}\smallsetminus\set{\varphi^{-1}(p)}$. For every $c, d \in \mathbb{R}^{n}\smallsetminus\set{\varphi^{-1}(p)}$, there is a path in $\mathbb{R}^{n}\smallsetminus\set{0}$ from $c - \varphi^{-1}(p)$ to $d - \varphi^{-1}(p)$, so there is a path in $\mathbb{R}^{n}\smallsetminus\set{\varphi^{-1}(p)}$ from $c$ to $d$. Therefore $\mathbb{R}^{n}\smallsetminus\set{\varphi^{-1}(p)}$ is path-connected. Since $\varphi$ is a homeomorphism
	\begin{equation*}
		U\smallsetminus\set{p} = \varphi(\varphi^{-1}(U\smallsetminus\set{p})) = \varphi(\mathbb{R}^{n}\smallsetminus\set{\varphi^{-1}(p)})
	\end{equation*}

	is also path-connected, which is a contradiction because $U\smallsetminus\set{p}$ is disconnected.

	Thus for $n > 1$, $\mathbb{R}^{n}$ is not homeomorphic to any open subset $U\subseteq \mathbb{R}$.
\end{proof}
