\chapter{Connectedness and Compactness}

\section*{Connectedness}\addcontentsline{toc}{section}{Connectedness}

\subsection*{Definitions and Basic Properties}\addcontentsline{toc}{subsection}{Definitions and Basic Properties}

\begin{exercise}{4.3}
    Suppose $X$ is a connected topological space, and $\sim$ is an equivalence relation on $X$ such that every equivalence class is open. Show that there is exactly one equivalence class, namely $X$ itself.
\end{exercise}

\begin{proof}
    Suppose $X$ has exactly $n$ equivalence classes, all of which are open. Because all equivalence classes constitute a partition of $X$, then $X$ is the union of these open, disjoint equivalence classes. On the other hand, $X$ is connected, so $n = 1$.
\end{proof}

\begin{exercise}{4.4}
    Prove that a topological space $X$ is disconnected if and only if there exists a nonconstant continuous function from $X$ to the discrete space $\{ 0, 1 \}$.
\end{exercise}

\begin{proof}
    $(\Rightarrow)$ Suppose there is a nonconstant continuous function $f: X\to \{ 0, 1 \}$, then $f^{-1}(0)$ and $f^{-1}(1)$ are nonempty, open, disjoint, and their union is $X$. Therefore $X$ is disconnected.

    $(\Leftarrow)$ Suppose $X$ is disconnected, then there are nonempty, open, disjoint subsets $U, V\subseteq X$ such that $X = U\cup V$. Define $f: X\to \{ 0, 1 \}$ as follows: $f(x) = 0$ if $x\in U$ and $f(x) = 1$ if $x\notin U$, then $f$ is nonconstant and $f$ is continuous (the preimages of $\varnothing, \{0\}, \{1\}, \{0,1\}$ are open in $X$).
\end{proof}

\begin{exercise}{4.5}
    Prove that a topological space is disconnected if and only if it is homeomorphic to a disjoint union of two or more nonempty spaces.
\end{exercise}

\begin{proof}
    Let $X$ be a topological space.

    $(\Rightarrow)$ Suppose $X$ is disconnected, then there are nonempty, open, disjoint subsets $U, V\subseteq X$ such that $X = U\cup V$. $U\cup V$ is homeomorphic to $U\times\{0\} \cup V\times\{1\} = U\sqcup V$. Hence $X$ is homeomorphic to a disjoint union of at least two nonempty spaces.

    $(\Leftarrow)$ Suppose $X$ is homeomorphic to a disjoint union of $n\geq 2$ nonempty spaces $U_{1}, \ldots, U_{n}$. Let $f: X\to \coprod_{i\in \{1,\ldots,n\}}U_{i}$ be a homeomorphism, then $f^{-1}(U_{1}), \ldots, f^{-1}(U_{n})$ constitutes a partition on $X$, and $f^{-1}(U_{1}), \ldots, f^{-1}(U_{n})$ are nonempty open subsets of $X$. Because $n\geq 2$, it follows that $X$ is disconnected.
\end{proof}

\begin{exercise}{4.10}
    Suppose $M$ is a connected manifold with nonempty boundary. Show that its double $D(M)$ is connected.
\end{exercise}

\begin{proof}
    Let $h: \partial M\to \partial M$ be the identity map from the boundary of $M$ onto itself. $\sim$ is the equivalence relation on $D(M)$ defined by $x\sim h(x)$ for $x\in\partial h(x)$, other points are equivalent to itself. By th definition of the double of a manifold with boundary, $D(M) = M\cup_{h}M = (M\sqcup M)/\sim$. Because $M$ is connected, and $M\cap M\ne\varnothing$, it follows that $M\sqcup M$ is connected. Moreover, the quotient space of a connected space is connected, so $(M\sqcup M)/\sim$ is connected. Thus the double $D(M)$ is connected.
\end{proof}

\subsection*{Path Connectedness}\addcontentsline{toc}{subsection}{Path Connectedness}

\begin{exercise}{4.14}
    Prove Proposition 4.13 (Properties of Path-Connected Spaces).
    \begin{enumerate}[label={(\alph*)}]
        \item Every continuous image of a path-connected space is path-connected.
        \item Let $X$ be a space, and let ${\{ B_{\alpha} \}}_{\alpha\in A}$ be a collection of path-connected subspaces of $X$ with a point in common. Then $\bigcup_{\alpha\in A}B_{\alpha}$ is path-connected.
        \item Every product of finitely many path-connected spaces is path-connected.
        \item Every quotient space of a path-connected space is path-connected.
    \end{enumerate}
\end{exercise}

\begin{proof}
\end{proof}

\subsection*{Components and Path Components}\addcontentsline{toc}{subsection}{Components and Path Components}

\begin{exercise}{4.22}
    Prove Proposition 4.21 (Properties of Path Components).

    Let $X$ be any space.
    \begin{enumerate}[label={(\alph*)}]
        \item The path components of $X$ form a partition of $X$.
        \item Each path component is contained in a single component, and each component is a disjoint union of path components.
        \item Any nonempty path-connected subset of $X$ is contained in a single path component.
    \end{enumerate}
\end{exercise}

\begin{exercise}{4.24}
    Prove Proposition 4.23.

    Every manifold (with or without boundary) is locally connected and locally path-connected.
\end{exercise}

\section*{Compactness}\addcontentsline{toc}{section}{Compactness}

\subsection*{Definitions and Basic Properties}\addcontentsline{toc}{subsection}{Definitions and Basic Properties}

\begin{exercise}{4.28}
    Prove Lemma 4.27 (Compactness Criterion for Subspaces).

    If $X$ is any topological space, a subset $A\subseteq X$ is compact (in the subspace topology) if and only if every cover of $A$ by open subsets of $X$ has a finite subcover.
\end{exercise}

\begin{exercise}{4.29}
    In any topological space $X$, show that every union of finitely many compact subsets of $X$ is compact.
\end{exercise}

\begin{exercise}{4.37}
    Suppose $M$ is a compact manifold with boundary. Show that the double of $M$ is compact.
\end{exercise}

\begin{exercise}{4.38}
    Let $X$ be a compact space, and suppose $\{ F_{n} \}$ is a countable collection of nonempty closed subsets of $X$ that are \textbf{nested}, which means that $F_{n}\supseteq F_{n+1}$ for each $n$. Show that $\bigcap_{n}F_{n}$ is nonempty.
\end{exercise}

\subsection*{Sequential and Limit Point Compactness}\addcontentsline{toc}{subsection}{Sequential and Limit Point Compactness}

\begin{exercise}{4.49}
    Prove the preceding three theorems.

    Theorem 4.46 (Bolzano-Weierstraß). Every bounded sequence in $\mathbb{R}^{n}$ has a convergent subsequence.

    Theorem 4.47. Endowed with the Euclidean metric, a subset of $\mathbb{R}^{n}$ is a complete metric space if and only if it is closed in $\mathbb{R}^{n}$. In particular, $\mathbb{R}^{n}$ is complete.

    Theorem 4.48. Every compact metric space is complete.
\end{exercise}

\subsection*{The Closed Map Lemma}\addcontentsline{toc}{subsection}{The Closed Map Lemma}

\section*{Local Compactness}\addcontentsline{toc}{section}{Local Compactness}

\section*{Paracompactness}\addcontentsline{toc}{section}{Paracompactness}

\subsection*{Normal Spaces}

\subsection*{Partition of Unity}

\section*{Proper Maps}

\section*{Problems}
