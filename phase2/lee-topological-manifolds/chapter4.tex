% chktex-file 8
\chapter{Connectedness and Compactness}

\section*{Connectedness}\addcontentsline{toc}{section}{Connectedness}

\subsection*{Definitions and Basic Properties}\addcontentsline{toc}{subsection}{Definitions and Basic Properties}

\begin{exercise}{4.3}
	Suppose $X$ is a connected topological space, and $\sim$ is an equivalence relation on $X$ such that every equivalence class is open. Show that there is exactly one equivalence class, namely $X$ itself.
\end{exercise}

\begin{proof}
	Suppose $X$ has exactly $n$ equivalence classes, all of which are open. Because all equivalence classes constitute a partition of $X$, then $X$ is the union of these open, disjoint equivalence classes. On the other hand, $X$ is connected, so $n = 1$. Hence $X$ has exactly one equivalence class.
\end{proof}

\begin{exercise}{4.4}
	Prove that a topological space $X$ is disconnected if and only if there exists a nonconstant continuous function from $X$ to the discrete space $\set{0,1}$.
\end{exercise}

\begin{proof}
	Suppose there is a nonconstant continuous function $f: X\to \set{0,1}$, then $f^{-1}(0)$ and $f^{-1}(1)$ are nonempty, open, disjoint, and their union is $X$. Therefore $X$ is disconnected.

	Conversely, suppose $X$ is disconnected, then there are nonempty, open, disjoint subsets $U, V\subseteq X$ such that $X = U\cup V$. Define $f: X\to \set{0,1}$ as follows: $f(x) = 0$ if $x\in U$ and $f(x) = 1$ if $x\notin U$, then $f$ is nonconstant and $f$ is continuous (because the preimages under $f$ of $\varnothing, \set{0}, \set{1}, \set{0, 1}$ are open in $X$).
\end{proof}

\begin{exercise}{4.5}
	Prove that a topological space is disconnected if and only if it is homeomorphic to a disjoint union of two or more nonempty spaces.
\end{exercise}

\begin{proof}
	Let $X$ be a topological space.

	Suppose $X$ is disconnected, then there are subsets $U, V\subseteq X$ that disconnect $X$. The map $\varphi: X \to U\sqcup V$ given by
	\begin{equation*}
		\varphi(x) = \begin{cases}
			(x, 0) & \text{if $x \in U$} \\
			(x, 1) & \text{if $x \in V$}
		\end{cases}
	\end{equation*}

	is a homeomorphism, so $X$ is homeomorphic to the disjoint union $U\sqcup V$. Hence $X$ is homeomorphic to a disjoint union of at least two nonempty spaces.

	Conversely, suppose $X$ is homeomorphic to a disjoint union of $n\geq 2$ nonempty spaces $U_{1}, \ldots, U_{n}$. Let $f: X\to \coprod^{n}_{i=1}U_{i}$ be a homeomorphism, then $f^{-1}(U_{1}), \ldots, f^{-1}(U_{n})$ constitutes a partition on $X$, and $f^{-1}(U_{1}), \ldots, f^{-1}(U_{n})$ are nonempty open subsets of $X$. Because $n\geq 2$, it follows that $X$ is disconnected.
\end{proof}

\begin{exercise}{4.10}
	Suppose $M$ is a connected manifold with nonempty boundary. Show that its double $D(M)$ is connected.
\end{exercise}

\begin{proof}
	Let $h: \partial M\to \partial M$ be the identity map from the boundary of $M$ onto itself. $\sim$ is the equivalence relation on $D(M)$ defined by $x\sim h(x)$ for $x\in\partial M$, and other points are equivalent to itself. By the definition of the double of a manifold with boundary
	\begin{equation*}
		D(M) = M\cup_{h}M = (M\sqcup M)/_{\sim}.
	\end{equation*}

	Assume that $D(M)$ is disconnected by open subsets $U, V$. Let $q: M \sqcup M \to (M \sqcup M)/_{\sim}$ be the quotient map, then $q^{-1}(U)$ and $q^{-1}(V)$ disconnect $M\sqcup M \approx (M\times\set{0}) \cup (M\times\set{1})$. Because $M\times\set{0}, M\times\set{1} \subseteq q^{-1}(U) \cup q^{-1}(V)$ and $M\times\set{0}, M\times\set{1}$ are connected, it follows from Proposition 4.9 that either $M\times\set{0} \subseteq q^{-1}(U)$ or $M\times\set{0} \subseteq q^{-1}(V)$, either $M\times\set{1} \subseteq q^{-1}(U)$ or $M\times\set{1} \subseteq q^{-1}(V)$. Because $q^{-1}(U)$ and $q^{-1}(V)$ are nonempty, it follows that either $M\times\set{0} \subseteq q^{-1}(U)$ and $M\times\set{1} \subseteq q^{-1}(V)$ or $M\times\set{0} \subseteq q^{-1}(V)$ and $M\times\set{1} \subseteq q^{-1}(U)$. Without loss of generality, suppose that the former is the case. Since the boundary $\partial M$ is nonempty, there is $a \in \partial M$. $a \sim h(a)$, $(a, 0) \in M\times\set{0} = q^{-1}(U)$ and $(a, 1) \in M\times\set{1} = q^{-1}(V)$ so $q(a) = q(h(a)) \in U, V$. Hence $U$ and $V$ are not disjoint, which contradicts our assumption.

	Thus $D(M)$ is connected.
\end{proof}

\subsection*{Path Connectedness}\addcontentsline{toc}{subsection}{Path Connectedness}

\begin{exercise}{4.14}\label{exercise:4.14}
	Prove Proposition 4.13 (Properties of Path-Connected Spaces).
	\begin{enumerate}[label={(\alph*)}]
		\item Every continuous image of a path-connected space is path-connected.
		\item Let $X$ be a space, and let ${\{ B_{\alpha} \}}_{\alpha\in A}$ be a collection of path-connected subspaces of $X$ with a point in common. Then $\bigcup_{\alpha\in A}B_{\alpha}$ is path-connected.
		\item Every product of finitely many path-connected spaces is path-connected.
		\item Every quotient space of a path-connected space is path-connected.
	\end{enumerate}
\end{exercise}

\begin{proof}
	\begin{enumerate}[label={(\alph*)}]
		\item Let $f: X\to Y$ be a continuous map and $X$ is a path-connected space. Let $p, q$ be two points of $f(X)$, let $a \in f^{-1}(p)$ and $b \in f^{-1}(q)$. Because $X$ is path-connected, there is a continuous map $g: \closedinterval{0, 1} \to X$ such that $g(0) = a$ and $g(1) = b$, so the composition $f\circ g: \closedinterval{0, 1} \to f(X)$ is continuous and $(f\circ g)(0) = p$ and $(f\circ g)(1) = q$. Hence for every two points $p, q$ of $f(X)$, there is a continuous map from $\closedinterval{0, 1}$ to $f(X)$ such that the images of $0, 1$ are $p, q$, respectively, which implies $f(X)$ is path-connected.
		\item Let $p, q$ be two points of $\bigcup_{\alpha\in A}B_{\alpha}$.

		      $p \in B_{\alpha_{p}}$ and $q \in B_{\alpha_{q}}$ for some $\alpha_{p}, \alpha_{q} \in A$. Let $x \in \bigcap_{\alpha\in A}B_{\alpha}$ (these sets have a point in common). Because $B_{\alpha_{p}}, B_{\alpha_{q}}$ are path-connected, there are continuous maps $f_{p}: \closedinterval{0, 1} \to B_{\alpha_{p}}$ such that $f_{p}(0) = p, f_{p}(1) = x$, and $f_{q}: \closedinterval{0, 1} \to B_{\alpha_{q}}$ such that $f_{q}(0) = x, f_{q}(1) = q$.

		      The maps $g: \closedinterval{0, \frac{1}{2}} \to \closedinterval{0, 1}$ given by $g(t) = 2t$ and $h: \closedinterval{\frac{1}{2}, 1} \to \closedinterval{0, 1}$ given by $h(t) = 2t - 1$ are continuous. The compositions $f_{p}\circ g$ and $f_{q}\circ h$ are therefore continuous, and they agree on $\closedinterval{0, \frac{1}{2}} \cap \closedinterval{\frac{1}{2}, 1}$, since $(f_{p}\circ g)(1/2) = x = (f_{q}\circ h)(1/2)$. $\closedinterval{0, \frac{1}{2}}, \closedinterval{\frac{1}{2}, 1}$ constitute a finite closed cover of $\closedinterval{0, 1}$, so by the gluing lemma, there is a unique continuous map $f: \closedinterval{0, 1} \to B_{\alpha_{p}} \cup B_{\alpha_{q}}$ such that $f\vert_{\closedinterval{0, \frac{1}{2}}} = f_{p}\circ g$ and $f\vert_{\closedinterval{\frac{1}{2}, 1}} = f_{q}\circ h$. Moreover, $f(0) = p, f(1) = q$. Hence there is a path in $\bigcup_{\alpha\in A}B_{\alpha}$ from $p$ to $q$.

		      Thus $\bigcup_{\alpha\in A}B_{\alpha}$ is path-connected.
		\item It suffices to prove that the product of two path-connected spaces is path-connected.

		      Let $X, Y$ be path-connected spaces and $(x_{1}, y_{1}), (x_{2}, y_{2})$ are two points of $X\times Y$. The maps $i_{y_{0}}: X\to X\times Y$ given by $i_{y_{0}}(x) = (x, y_{0})$ and $i_{x_{0}}: Y\to X\times Y$ given by $i_{x_{0}}(y) = (x_{0}, y)$ are continuous. From part (a), it follows that $X\times\set{y_{0}}$ and $\set{x_{0}}\times Y$ are path-connected for every $y_{0} \in Y, x_{0}\in X$. Hence there is a path $f_{1}$ in $X\times Y$ from $(x_{1}, y_{1})$ to $(x_{2}, y_{1})$ and a path $f_{2}$ in $X\times Y$ from $(x_{2}, y_{1})$ to $(x_{2}, y_{2})$.

		      The maps $g: \closedinterval{0, \frac{1}{2}} \to \closedinterval{0, 1}$ given by $g(t) = 2t$ and $h: \closedinterval{\frac{1}{2}, 1} \to \closedinterval{0, 1}$ given by $h(t) = 2t - 1$ are continuous. The compositions $f_{1}\circ g$ and $f_{2}\circ h$ are therefore continuous, and they agree on $\closedinterval{0, \frac{1}{2}} \cap \closedinterval{\frac{1}{2}, 1}$, since $(f_{1}\circ g)(1/2) = (x_{2}, y_{1}) = (f_{2}\circ h)(1/2)$. From the gluing lemma, it follows that there is a unique continuous map $f: \closedinterval{0, 1} \to X\times Y$ such that $f\vert_{\closedinterval{0, \frac{1}{2}}} = f_{1}\circ g$ and $f\vert_{\closedinterval{\frac{1}{2}, 1}} = f_{2}\circ h$. Moreover, $f(0) = (x_{1}, y_{1})$ and $f(1) = (x_{2}, y_{2})$.

		      Therefore $X\times Y$ is path-connected. From mathematical induction, it follows that the finite product of path-connected spaces is path-connected.
		\item Since every quotient map is continuous and surjective, from part (a), it follows that every quotient space of a path-connected space is path-connected.
	\end{enumerate}
\end{proof}

\subsection*{Components and Path Components}\addcontentsline{toc}{subsection}{Components and Path Components}

\begin{exercise}{4.22}
	Prove Proposition 4.21 (Properties of Path Components).

	Let $X$ be any space.
	\begin{enumerate}[label={(\alph*)}]
		\item The path components of $X$ form a partition of $X$.
		\item Each path component is contained in a single component, and each component is a disjoint union of path components.
		\item Any nonempty path-connected subset of $X$ is contained in a single path component.
	\end{enumerate}
\end{exercise}

\begin{proof}
	\begin{enumerate}[label={(\alph*)}]
		\item Let $U, V$ be non-disjoint path components of $X$. From Exercise~\ref{exercise:4.14} (b), $U\cup V$ is path-connected. Due to the maximality of path components, $U = V = U\cup V$, from which we deduce that non-disjoint path components are identical. Hence distinct path components are disjoint.

		      Let $x$ be a point of $X$. The singleton $\set{x}$ is a path component containing $x$. Let ${(B_{\alpha})}_{\alpha\in A}$ be the family of all path-connected sets containing $x$, then $\bigcup_{\alpha\in A}B_{\alpha}$ is path-connected (according to Exercise~\ref{exercise:4.14} (b)). Moreover $\bigcup_{\alpha\in A}B_{\alpha}$ is a maximal path-connected set so it is a path component containing $x$. Therefore every element of $X$ is in a path component.

		      Thus the path components of $X$ form a partition of $X$.
		\item Let $P$ be a path component of $X$. Because the components of $X$ form a partition of $X$, $P$ has a common point with some component $C$ of $X$. Since a path-connected set is also connected, $P$ is a connected set. Because the union of connected sets with a point in common is connected, $P\cup C$ is connected. From the maximality of $C$, we deduce that $P\cup C = C$, which means $P\subseteq C$. Moreover, distinct components are disjoint, so $P$ is contained in the component $C$ only. Hence every path component is contained in a single component.

		      Let $C$ be a component of $X$. Every point $p$ of $C$ is in some path component of $X$, so the path component containing $p$ is contained in $C$. On the other hand, distinct path components are disjoint. Hence $C$ is a disjoint union of path components.
		\item Let $A$ be a nonempty path-connected subset of $X$.

		      Let $x$ be an element of $A$. According to part (a), $x$ is a point of a path component $P$. According to Exercise~\ref{exercise:4.14} (b), $A\cup P$ is path-connected. Because $P$ is a maximal path-connected set, it follows that $A\cup P = P$, which means $A\subseteq P$.

		      If $A$ is contained in two path components, then the two path components are not disjoint (because $A$ is nonempty) and it follows that the two path components are identical (according to part (a)).

		      Therefore any nonempty path-connected subset of $X$ is contained in a single path component of $X$.
	\end{enumerate}
\end{proof}

\begin{exercise}{4.24}
	Prove Proposition 4.23.

	Every manifold (with or without boundary) is locally connected and locally path-connected.
\end{exercise}

\begin{proof}
	Let $M$ be an $n$-manifold (with or without boundary).

	\textbf{Case 1. $M$ is an $n$-manifold without boundary.}

	According to Problem~\ref{problem:2-23}, every manifold has a basis of coordinate balls. On the other hand, every coordinate ball is homeomorphic to an open ball of $\mathbb{R}^{n}$ and every open ball of $\mathbb{R}^{n}$ is path-connected because every open ball of $\mathbb{R}^{n}$ is homeomorphic to $\mathbb{R}^{n}$ (which is path-connected). Therefore every coordinate ball is path-connected (and hence connected). Hence $M$ is locally path-connected and locally connected.

	\textbf{Case 2. $M$ is an $n$-manifold with boundary.}

	Firstly, we construct a basis for $M$. Let $U$ be a nonempty open subset of $M$ and $x\in U$, then $x$ is in the domain of an interior chart or that of a boundary chart.

	If $x$ is in the domain of an interior chart $(V, \varphi_{x})$, then $\varphi_{x}(V)$ is an open subset of $\mathbb{R}^{n}$. $U\cap V$ is homeomorphic to $\varphi_{x}(U\cap V)$ and $\varphi_{x}(U\cap V)$ is an open subset of $\mathbb{R}^{n}$. Because $\varphi_{x}(U\cap V)$ is open and $\varphi_{x}(x)$ is a point of this set, there is an open ball $B_{r}(\varphi_{x}(x)) \subseteq \varphi_{x}(U\cap V)$. Therefore $\varphi_{x}^{-1}(B_{r}(\varphi_{x}(x)))$ and $B_{r}(\varphi_{x}(x))$ are homeomorphic. So $x$ is in the domain of the following interior chart $(\varphi_{x}^{-1}(B_{r}(\varphi_{x}(x))), \varphi_{x}\vert_{\varphi_{x}^{-1}(B_{r}(\varphi_{x}(x)))})$ where the domain is contained in $U$ and homeomorphic to an open ball in $\mathbb{R}^{n}$.

	If $x$ is in the domain of a boundary chart $(V, \varphi_{x})$, then $\varphi_{x}(x) \in \partial\mathbb{H}^{n}$ and $\varphi_{x}(V)$ is an open subset of $\mathbb{H}^{n}$. $U\cap V$ is homeomorphic to $\varphi_{x}(U\cap V)$ and $\varphi_{x}(U\cap V)$ is an open subset of $\mathbb{H}^{n}$. Because $\varphi_{x}(U\cap V)$ is open and $\varphi_{x}(x)$ is a point of this set, there is an open ball $B_{r}(\varphi_{x}(x))$ in $\mathbb{R}^{n}$ such that $B_{r}(\varphi_{x}(x)) \cap \mathbb{H}^{n} \subseteq \varphi_{x}(U\cap V)$ (here we make use of the subspace topology on $\mathbb{H}^{n}$ and the basis for $\mathbb{R}^{n}$ containing open balls). Therefore $W = B_{r}(\varphi_{x}(x)) \cap \mathbb{H}^{n}$ and $\varphi_{x}^{-1}(W)$ are homeomorphic, and $x$ is in the domain of the following boundary chart $(\varphi_{x}^{-1}(W), \varphi_{x}\vert_{W})$ where the domain is contained in $U$ and is homeomorphic to an open half-ball in $\mathbb{H}^{n}$ (it is halved by taking intersection of $\mathbb{H}^{n}$ and an open ball in $\mathbb{R}^{n}$ with center on $\partial \mathbb{H}^{n}$).

	Hence the collection of open sets of $M$ which are domains of some interior chart (and homeomorphic to some open ball in $\mathbb{R}^{n}$) or some boundary chart  (and homeomorphic to some open half-ball in $\mathbb{H}^{n}$) constitutes a basis for the $n$-manifold with boundary $M$.

	On the other hand, an open ball or an open half-ball is path-connected (because it is a convex set), so $M$ has a basis of path-connected (hence connected) open sets. Therefore $M$ is locally path-connected and locally connected.

	From the two cases, we conclude that every manifold (without or with boundary) is locally path-connected and locally connected.
\end{proof}

\section*{Compactness}\addcontentsline{toc}{section}{Compactness}

\subsection*{Definitions and Basic Properties}\addcontentsline{toc}{subsection}{Definitions and Basic Properties}

\begin{exercise}{4.28}
	Prove Lemma 4.27 (Compactness Criterion for Subspaces).

	If $X$ is any topological space, a subset $A\subseteq X$ is compact (in the subspace topology) if and only if every cover of $A$ by open subsets of $X$ has a finite subcover.
\end{exercise}

\begin{proof}
	Suppose $A\subseteq X$ is compact with the subspace topology. Let ${(U_{i})}_{i\in I}$ be a cover of $A$ by open subsets of $X$, then ${(U_{i}\cap A)}_{i\in I}$ is a cover of $A$ by open subsets of $A$. Because $A$ is compact with the subspace topology, ${(U_{i}\cap A)}_{i\in I}$ contains a finite subcover of $A$ by open subsets of $A$, which implies ${(U_{i})}_{i\in I}$ contains a finite subcover of $A$ by open subsets of $X$.

	Conversely, suppose that every cover of $A$ by open subsets of $X$ has a finite subcover. Let ${(V_{i})}_{i\in I}$ be a cover of $A$ by open subsets of $A$. From the definition of subspace topology, for each $i\in I$, there is an open subset $U_{i}\subseteq X$ such that $V_{i} = U_{i}\cap A$. Therefore ${(U_{i})}_{i\in I}$ is a cover of $A$ by open subsets of $X$, so ${(U_{i})}_{i\in I}$ contains a finite subcover of $A$ by open subsets of $X$, which implies that ${(V_{i})}_{i\in I}$ contains a finite subcover of $A$ by open subsets of $A$. Hence $A\subseteq X$ is compact with the subspace topology.
\end{proof}

\begin{exercise}{4.29}
	In any topological space $X$, show that every union of finitely many compact subsets of $X$ is compact.
\end{exercise}

\begin{proof}
	It suffices to prove that the union of two compact subsets of $X$ is compact. Let $A, B$ be compact subsets of $X$ and $\mathcal{O}$ be an open cover of $A\cup B$ by open subsets of $X$. Since $A$ is compact, there exist finitely many open subsets $A_{1}, \ldots, A_{m}$ from $\mathcal{O}$ that covers $A$ and finitely many open subsets $B_{1}, \ldots, B_{n}$ from $\mathcal{O}$ that covers $B$. Hence $\mathcal{O}$ has a finite subcover, which consists of $A_{1}, \ldots, A_{m}, B_{1}, \ldots, B_{n}$. Therefore $A\cup B$ is compact.

	The union of zero compact subsets of $X$ is compact (because it is the empty set). Assume that the union of $n - 1$ compact subsets of $X$ is compact. Let $A_{1}, \ldots, A_{n}$ be $n$ compact subsets of $X$. By the inductive hypothesis, $\bigcup^{n-1}_{i=1}A_{i}$ is a compact subset of $X$. From the previous paragraph, we deduce that $\bigcup^{n}_{i=1}A_{i}$ is a compact subset of $X$. By the principle of mathematical induction, we conclude that every union of finitely many compact subsets of $X$ is compact.
\end{proof}

\begin{exercise}{4.37}
	Suppose $M$ is a compact manifold with boundary. Show that the double of $M$ is compact.
\end{exercise}

\begin{proof}
	Let $h$ be the identity map $\partial M\to \partial M$. From the definition of the double of a manifold with boundary, $D(M) = M\cup_{h} M = {(M\sqcup M)}/_{\sim}$ (where $(a, 0) \sim (h(a), 1)$ for $a\in \partial H$). Because $M\sqcup M \approx (M\times\set{0}) \cup (M\times\set{1})$ is the union of two compact manifolds (which are homeomorphic to the compact manifold $M$), it follows that $M\sqcup M$ is compact.

	Let $q: M\sqcup M \to (M\sqcup M)/_{\sim} = D(M)$ be the quotient map. Because every quotient of a compact space is compact, it follows that $D(M)$ is compact.
\end{proof}

\begin{exercise}{4.38}
	Let $X$ be a compact space, and suppose $\set{F_{n}}$ is a countable collection of nonempty closed subsets of $X$ that are \textbf{nested}, which means that $F_{n}\supseteq F_{n+1}$ for each $n$. Show that $\bigcap_{n}F_{n}$ is nonempty.
\end{exercise}

\begin{proof}
	Each $F_{n}$ is a closed subset $X$, so $X\smallsetminus F_{n}$ is open. Assume for the sake of contradiction that $\bigcap_{n}F_{n}$ is empty then
	\begin{equation*}
		X = X\smallsetminus \left(\bigcap_{n}F_{n}\right) = \bigcup_{n}(X\smallsetminus F_{n})
	\end{equation*}

	so $\set{X\smallsetminus F_{n}}_{n}$ is an open cover of $X$. Because of the compactness of $X$, $\set{X\smallsetminus F_{n}}_{n}$ contains a finite subcover, say $X\smallsetminus F_{k_{1}}, \ldots, X\smallsetminus F_{k_{m}}$, and we can relabel these sets such that $k_{1} < \cdots < k_{m}$. Since $F_{k_{1}} \supseteq \cdots \supseteq F_{k_{m}}$, we have $X\smallsetminus F_{k_{1}} \subseteq \cdots \subseteq X\smallsetminus F_{k_{m}}$ hence $X = \bigcup^{m}_{i=1}(X\smallsetminus F_{k_{i}}) = X \smallsetminus F_{k_{m}}$, which is a contradiction since $X$ is a proper superset of $X\smallsetminus F_{k_{m}}$, because $F_{k_{m}}$ is a nonempty subset of $X$. Hence $\bigcap_{n}F_{n}$ is nonempty.
\end{proof}

\subsection*{Sequential and Limit Point Compactness}\addcontentsline{toc}{subsection}{Sequential and Limit Point Compactness}

\begin{exercise}{4.49}
	Prove the preceding three theorems.

	Theorem 4.46 (Bolzano-Weierstraß). Every bounded sequence in $\mathbb{R}^{n}$ has a convergent subsequence.

	Theorem 4.47. Endowed with the Euclidean metric, a subset of $\mathbb{R}^{n}$ is a complete metric space if and only if it is closed in $\mathbb{R}^{n}$. In particular, $\mathbb{R}^{n}$ is complete.

	Theorem 4.48. Every compact metric space is complete.
\end{exercise}

\begin{proof}
	\textbf{Proof for Theorem 4.46.} Let ${(x_{i})}_{i\in\mathbb{N}}$ be a bounded sequence in $\mathbb{R}^{n}$, then there exists $a > 0$ such that $x_{i} \in {\closedinterval{-a, a}}^{n}$ for every $i\in\mathbb{N}$. If the sequence ${(x_{i})}_{i\in\mathbb{N}}$ takes on finitely many values then it has a subsequence that is eventually constant. Otherwise, suppose ${(x_{i})}_{i\in\mathbb{N}}$ takes on infinitely many values. ${\closedinterval{-a, a}}^{n}$ is closed and bounded in $\mathbb{R}^{n}$ so it is compact, due to Heine-Borel's theorem. The space ${\closedinterval{-a, a}}^{n}$ with subspace topology inherited from $\mathbb{R}^{n}$ is second countable (implies first countable) and Hausdorff so from Lemma 4.42 (compactness implies limit point compactness) and Lemma 4.43 (In first countable Hausdorff spaces, limit point compactness implies sequential compactness), it follows that ${(x_{i})}_{i\in\mathbb{N}}$ has a convergent subsequence.

	\textbf{Proof for Theorem 4.47.} Let $M$ be subset of $\mathbb{R}^{n}$, then $M$ is a metric space with the Euclidean metric restricted to $M$. If $M$ is complete, let $x\in \mathbb{R}^{n}$ be a limit point of $M$. For every $i\in\mathbb{N}$, $B_{1/2^{i}}(x)$ contains a point $a_{i}$ in $M$, then the sequence ${(a_{i})}_{i\in\mathbb{N}}$ is a Cauchy sequence. Since $M$ is complete, the sequence converges to a point of $M$. Because $a_{i}$ converges to $x$, it follows that $x\in M$, hence $M$ contains all of its limit points, which means $M$ is closed. Conversely, if $M$ is closed, let ${(x_{i})}_{i\in\mathbb{N}}$ be a Cauchy sequence of points in $M$. In a metric space, a Cauchy sequence is a bounded sequence. By Theorem 4.46, ${(x_{i})}_{i\in\mathbb{N}}$ has a convergent subsequence. A Cauchy sequence having a convergent subsequence is convergent, hence ${(x_{i})}_{i\in\mathbb{N}}$ is convergent, so $M$ is complete.

	\textbf{Proof for Theorem 4.48.} Let $M$ be a compact metric space and ${(x_{i})}_{i\in\mathbb{N}}$ be a Cauchy sequence of points in $M$. $M$ is a metric space so it is first countable and Hausdorff. By Lemma 4.42, $M$ is limit point compact. By Lemma 4.43 (In first countable Hausdorff spaces, limit point compactness implies sequential compactness), $M$ is sequentially compact. Hence ${(x_{i})}_{i\in\mathbb{N}}$ has a convergent subsequence. A Cauchy sequence having a convergent subsequence is convergent, hence ${(x_{i})}_{i\in\mathbb{N}}$ is convergent, so $M$ is complete.
\end{proof}

\subsection*{The Closed Map Lemma}\addcontentsline{toc}{subsection}{The Closed Map Lemma}

\begin{exercise}{4.58}
	Using the map of Example 4.55, show that there is a coordinate ball in $\mathbb{S}^{n}$ whose closure is equal to all of $\mathbb{S}^{n}$.
\end{exercise}

\begin{proof}
	The quotient map of Example 4.55 is $q: {\overline{B}}^{n} \to \mathbb{S}^{n}$ given by
	\begin{equation*}
		q(x) = \tuple{2\sqrt{1 - {\abs{x}}^{2}}x, 2{\abs{x}}^{2} - 1}.
	\end{equation*}

	Let $N$ be the point of $\mathbb{R}^{n+1}$ whose coordinates are $\tuple{0, \ldots, 0, 1}$ then $N\in \mathbb{S}^{n}$. The singleton set $\set{N}$ is closed in $\mathbb{S}^{n}$ so its complement $\mathbb{S}^{n}\smallsetminus \set{N}$ is open in $\mathbb{S}^{n}$. On the other hand, $q^{-1}(N) = \partial{\overline{B}}^{n}$, so $q^{-1}(\mathbb{S}^{n}\smallsetminus \set{N}) = {\overline{B}}^{n} \smallsetminus \partial{\overline{B}}^{n} = B^{n}$. Moreover, the restriction of $q$ on $\mathbb{S}^{n}\smallsetminus \set{N}$ is injective, it follows that $\mathbb{S}^{n}\smallsetminus \set{N}$ is homeomorphic to $B^{n}$, hence $\mathbb{S}^{n}\smallsetminus \set{N}$ is a coordinate ball in $\mathbb{S}^{n}$. The closure of $\mathbb{S}^{n}\smallsetminus \set{N}$ is the entire $\mathbb{S}^{n}$.

	Thus $\mathbb{S}^{n}\smallsetminus \set{N}$ is a coordinate ball in $\mathbb{S}^{n}$ whose closure is equal to all of $\mathbb{S}^{n}$.
\end{proof}

\begin{exercise}{4.61}\label{exercise:4.61}
	Complete the proof of Proposition 4.60 by showing that $\mathscr{B}$ is a basis.
\end{exercise}

\begin{proof}
	Proposition 4.60. Every manifold has a countable basis of regular coordinate balls.

	Let $M$ be an $n$-manifold. Every point of $M$ has an Euclidean neighborhood (or coordinate domain), and these Euclidean neighborhoods cover $M$. Since $M$ is second countable, every open cover of $X$ has a countable subcover (This is Theorem 2.50). Let $\set{U_{i}}_{i\in\mathbb{N}}$ be a countable collection of such neighborhoods, and for each $U_{i}$, there is a homeomorphism $\varphi_{i}$ from $U_{i}$ onto an open subset $\hat{U}_{i} \subseteq \mathbb{R}^{n}$. For each $x \in \hat{U}_{i}$ there is a positive number $r(x)$ such that $B_{r(x)}(x) \subseteq \hat{U}_{i}$.

	For every $\hat{U}_{i}$, consider the open balls $B_{r}(x)$ where $B_{r}(x) \subseteq \hat{U}_{i}$, all coordinates of $x\in \hat{U}_{i}$ are rational, $r$ is a positive rational number strictly less than $r(x)$. From Lemma 4.59, $\varphi_{i}^{-1}(B_{r}(x))$ is a regular coordinate ball. Let $\mathscr{B}_{i}$ be the collection of those $\varphi_{i}^{-1}(B_{r}(x))$ then $\mathscr{B}_{i}$ is countable. Therefore $\mathscr{B} = \bigcup_{i\in\mathbb{N}}\mathscr{B}_{i}$ is countable.

	Every element of $\mathscr{B}$ is an open subset of $M$. Let $U$ be a nonempty open subset of $M$, then
	\begin{equation*}
		U = \bigcup_{i\in\mathbb{N}}(U\cap U_{i}).
	\end{equation*}

	Remind that the collection of open balls with rational radii and rational coordinates only is a basis for the Euclidean topology on $\mathbb{R}^{n}$. If $a\in U\cap U_{i}$ then the point $\varphi_{i}(a) \in \varphi_{i}(U_{i}\cap U)$ is contained in some $B_{r}(x)$ where $x$ has rational coordinates only and $0 < r < r(x)$ is rational. So $a\in \varphi_{i}^{-1}(B_{r}(x))$, which is an element of $\mathscr{B}$. Hence every point of $U$ admits an element of $\mathscr{B}$ as its neighborhood and this neighborhood is contained in $U$. Therefore $\mathscr{B}$ is a countable basis for $M$.

	Thus every manifold has a countable basis of regular coordinate balls.
\end{proof}

\begin{exercise}{4.62}
	Prove that every manifold with boundary has a countable basis consisting of regular coordinate balls and half-balls.
\end{exercise}

\begin{proof}
	First, we prove a result similar to Lemma 4.59 as follows: Let $M$ be an $n$-manifold with boundary. $B'\subseteq M$ is the (coordinate half-ball) domain of any boundary chart and $\varphi: B' \to B_{r'}(x)\cap \mathbb{H}^{n}$ where $x \in \partial\mathbb{H}^{n}$ is a homeomorphism, then $\varphi^{-1}(B_{r}(x) \cap \mathbb{H}^{n})$ is a regular coordinate half-ball whenever $0 < r < r'$.

	For every $0 < r < r'$, $B_{r}(x) \cap \mathbb{H}^{n}$ is an open subset of $B_{r'}(x) \cap \mathbb{H}^{n}$, so $B = \varphi^{-1}(B_{r}(x) \cap \mathbb{H}^{n})$ is a coordinate half-ball. Regard $\varphi^{-1}$ as a map from $\bar{B}_{r}(0) \cap \mathbb{H}^{n}$ to $M$, then $\varphi^{-1}$ is a continuous and closed map (because $\bar{B}_{r}(x) \cap \mathbb{H}^{n}$ is compact due to Heine-Borel's theorem, and $M$ is Hausdorff). By Problem~\ref{problem:2-6}, $\varphi^{-1}(\bar{B}_{r}(x) \cap \mathbb{H}^{n}) = \overline{B}$, hence $\varphi(\overline{B}) = \bar{B}_{r}(x) \cap \mathbb{H}^{n}$. Hence $\varphi^{-1}(B_{r}(x) \cap \mathbb{H}^{n})$ is a regular coordinate half-ball whenever $0 < r < r'$.

	An equivalence statement to the result we have just proved is given by replacing the point $x$ in $\partial\mathbb{H}^{n}$ by $0$, and this fits the definition of regular coordinate half-ball in the book. However, for convenience, we will use the definition given at the begining of this proof.

	Back to the proof for the main result. Let $M$ be an $n$-manifold with boundary. Every point of $M$ has an Euclidean neighborhood, so $M$ is covered by those Euclidean neighborhoods. Since $M$ is second countable then the open cover made of those Euclidean neighborhoods has a countable subcover. Let $\set{U_{i}}_{i\in\mathbb{N}}$ be an open cover of $M$ made of Euclidean neighborhoods, then for each $U_{i}$, there is a homeomorphism $\varphi_{i}$ from $U_{i}$ to an open subset $\hat{U}_{i}$ of $\mathbb{R}^{n}$ or $\mathbb{H}^{n}$. For each $x\in \hat{U}_{i}$, there is a positive number $r(x)$ such that
	\begin{equation*}
		\begin{cases}
			\text{if $x$ is in $\partial\mathbb{H}^{n}$ then $B_{r(x)} \cap \mathbb{H}^{n} \subseteq \hat{U}_{i}$,} \\
			\text{if $x$ is not in $\partial\mathbb{H}^{n}$ then $B_{r(x)} \subseteq \hat{U}_{i}$.}
		\end{cases}
	\end{equation*}

	For every $\hat{U}_{i}$, consider the following open sets: $B_{r}(x) \subseteq \hat{U}_{i}$ where $x \notin \partial\mathbb{H}^{n}$ (whose coordinates are all rational) and $r$ is a positive rational number strictly less than $r(x)$; $B_{r}(x) \cap \mathbb{H}^{n} \subseteq \hat{U}_{i}$ where $x \in \partial\mathbb{H}^{n}$ (whose coordinates are all rational) and $r$ is a positive rational number strictly less than $r(x)$. In the former case, $B = \varphi_{i}^{-1}(B_{r}(x))$ is a regular coordinate ball, due to Lemma 4.59; in the latter, $B = \varphi_{i}^{-1}(B_{r}(x) \cap \mathbb{H}^{n})$ is a regular coordinate half-ball, according to the first part of this proof. Denote by $\mathscr{B}_{i}$ the collection of those regular coordinate balls and regular coordinate half-balls, and $\mathscr{B} = \bigcup_{i\in\mathbb{N}}\mathscr{B}_{i}$.

	Since regular coordinate balls and regular coordinate half-balls are open subsets of $M$, every element of $\mathscr{B}$ is an open subset of $M$. Let $U$ be an open subset of $M$ then $U = \bigcup_{i\in\mathbb{N}}(U\cap U_{i})$. Remind that the collection of open ball $B_{r}(x)$ (where all coordinates of $x$ are rational, $r$ is a positive rational number) and open half-ball $B_{r}(x) \cap \mathbb{H}^{n}$ (where all coordinates of $x$ are rational, $x \in \partial\mathbb{H}^{n}$, $r$ is a positive rational number) is a basis for the subspace topology on $\mathbb{H}^{n}$. If $a \in U\cap U_{i}$ then the point $\varphi_{i}(a) \in \varphi_{i}(U\cap U_{i})$ is either in $\partial\mathbb{H}^{n}$ or $\operatorname{Int}\mathbb{H}^{n}$. In either case, there is an element of $\mathscr{B}_{i}$ containing $\varphi_{i}(a)$ and contained in $\varphi_{i}(U\cap U_{i})$. Therefore $\mathscr{B}$ is a countable basis for $M$.

	Thus every manifold with boundary has a countable basis of regular coordinate balls and half-balls.
\end{proof}

\section*{Local Compactness}\addcontentsline{toc}{section}{Local Compactness}

\begin{exercise}{4.67}
	Show that any finite product of locally compact spaces is locally compact.
\end{exercise}

\begin{proof}
	Let $X, Y$ be locally compact spaces and $\tuple{x, y}$ be a point of $X\times Y$. Because $X, Y$ are locally compact, there exist $U_{x}, K_{x} \subseteq X$ and $U_{y}, K_{y} \subseteq Y$ such that $U_{x}, U_{y}$ are open, $K_{x}, K_{y}$ are compact, and $x\in U_{x} \subseteq K_{x}, y\in U_{y} \subseteq K_{y}$. So $\tuple{x, y} \in U_{x}\times U_{y} \subseteq K_{x} \times K_{y}$, where $U_{x}\times U_{y}$ is a product open set and $K_{x}\times K_{y}$ is compact because the product of finitely many compact spaces is compact. Hence $X\times Y$ is locally compact.

	It follows from mathematical induction that any finite product of locally compact spaces is locally compact.
\end{proof}

\begin{exercise}{4.70}
	Prove Proposition 4.69: In a Baire space, every meager subset has dense complement.
\end{exercise}

\begin{quote}
	A subset $F$ of a topological space $X$ is said to be \textbf{nowhere dense} if $\overline{F}$ has a dense complement, and $F$ is said to be \textbf{meager} if it can be expressed as a union of countably many nowhere dense subsets.
\end{quote}

\begin{proof}
	Let $X$ be a Baire space and $A\subseteq X$ is a meager subset. The meager set $A$ can be expressed as a union of countable many nowhere dense subsets $\set{U_{i}}_{i\in\mathbb{N}}$. $\overline{X\smallsetminus \overline{U_{i}}} = X$ for every $i\in\mathbb{N}$. By De Morgan's law and $U_{i}\subseteq \overline{U_{i}}$, we obtain
	\begin{equation*}
		X\smallsetminus A = X\smallsetminus \left(\bigcup_{i\in\mathbb{N}}U_{i}\right) = \bigcap_{i\in\mathbb{N}}(X\smallsetminus U_{i}) \supseteq \bigcap_{i\in\mathbb{N}}(X\smallsetminus \overline{U_{i}})
	\end{equation*}

	Because $X$ is a Baire space and $X\smallsetminus \overline{U_{i}}$ is a dense open subset for every $i\in\mathbb{N}$, the intersection $\bigcap_{i\in\mathbb{N}} (X\smallsetminus \overline{U_{i}})$ is dense, from which we deduce that $X\smallsetminus A$ (which is a superset of the intersection) is dense. Thus every meager subset has a dense complement.
\end{proof}

\section*{Paracompactness}\addcontentsline{toc}{section}{Paracompactness}

\subsection*{Normal Spaces}\addcontentsline{toc}{subsection}{Normal Spaces}

\subsection*{Partition of Unity}\addcontentsline{toc}{subsection}{Partition of Unity}

\section*{Proper Maps}\addcontentsline{toc}{section}{Proper Maps}

\section*{Problems}

\begin{problem}{4-1}\label{problem:4-1}
Show that for $n > 1$, $\mathbb{R}^{n}$ is not homeomorphic to any open subset of $\mathbb{R}$.
\end{problem}

\begin{proof}
	Let $n$ be a positive integer greater than $1$. Assume that $\mathbb{R}^{n}$ is homeomorphic to an open subset $U\subseteq \mathbb{R}$, then $U$ is nonempty and there is a homeomorphism $\varphi: \mathbb{R}^{n} \to U$. Let $p$ be a point of $U$. Since $U\subseteq \mathbb{R}$ is open, $U$ is an union of disjoint open intervals, and $x$ lies in one of those open intervals, denote such open interval by $\openinterval{a, b}$, then $\openinterval{a, b}\smallsetminus\set{p}$ is disconnected. Therefore $U\smallsetminus\set{p}$ is disconnected. Because $\varphi$ is a homeomorphism, $\varphi^{-1}(U\smallsetminus\set{p}) = \mathbb{R}^{n} \smallsetminus \set{\varphi^{-1}(p)}$.

	We will show that $\mathbb{R}^{n}\smallsetminus\set{0}$ is path-connected. Let $x, y$ be two points of $\mathbb{R}^{n}\smallsetminus\set{0}$. Because $n > 1$, there exists a nonzero vector $v\in \mathbb{R}^{n}$ such that $x, y$ are not multiples of $v$. Let $v_{x} = \frac{\abs{x}}{\abs{v}}v$ and $v_{y} = \frac{\abs{y}}{\abs{v}}v$. We will construct
	\begin{itemize}
		\item a path in $\mathbb{R}^{n}\smallsetminus\set{0}$ from $x$ to $v_{x}$

		      Note that $\abs{x} = \abs{v_{x}}$. $x = (x_{1}, \ldots, x_{n})$ and $v_{x} = (v_{x,1}, \ldots, v_{x,n})$.

		      There exist $\varphi_{x,1}, \ldots, \varphi_{x,n-1} \in \mathbb{R}$ such that
		      \begin{align*}
			      x_{1}   & = \abs{x}\cos(\varphi_{x,1})                                                 \\
			      x_{2}   & = \abs{x}\sin(\varphi_{x,1})\cos(\varphi_{x,2})                              \\
			      \cdots  &                                                                              \\
			      x_{n-1} & = \abs{x}\sin(\varphi_{x,1})\cdots\sin(\varphi_{x,n-2})\cos(\varphi_{x,n-1}) \\
			      x_{n}   & = \abs{x}\sin(\varphi_{x,1})\cdots\sin(\varphi_{x,n-2})\sin(\varphi_{x,n-1})
		      \end{align*}

		      Also there exist $\varphi_{v_{x},1}, \ldots, \varphi_{v_{x},n-1} \in \mathbb{R}$ such that
		      \begin{align*}
			      v_{x,1}   & = \abs{x}\cos(\varphi_{v_{x},1})                                                         \\
			      v_{x,2}   & = \abs{x}\sin(\varphi_{v_{x},1})\cos(\varphi_{v_{x},2})                                  \\
			      \cdots    &                                                                                          \\
			      v_{x,n-1} & = \abs{x}\sin(\varphi_{v_{x},1})\cdots\sin(\varphi_{v_{x},n-2})\cos(\varphi_{v_{x},n-1}) \\
			      v_{x,n}   & = \abs{x}\sin(\varphi_{v_{x},1})\cdots\sin(\varphi_{v_{x},n-2})\sin(\varphi_{v_{x},n-1})
		      \end{align*}

		      The maps $f_{i}: \closedinterval{0, 1} \to \mathbb{R}$ given by
		      \begin{equation*}
			      f_{i}(t) = \abs{x}\sin((1-t)\varphi_{x,1} + t\varphi_{v_{x},1})\cdots \sin((1-t)\varphi_{x,i-1} + t\varphi_{v_{x},i-1})\cos((1-t)\varphi_{x,i} + t\varphi_{v_{x},i})
		      \end{equation*}

		      if $i < n$ and
		      \begin{equation*}
			      f_{n}(t) = \abs{x}\sin((1-t)\varphi_{x,1} + t\varphi_{v_{x},1})\cdots \sin((1-t)\varphi_{x,n-1} + t\varphi_{v_{x},n-1})
		      \end{equation*}

		      are continuous. So ${(f_{1}(t))}^{2} + \cdots + {(f_{n}(t))}^{2} = \abs{x}^{2} \ne 0$ for every $t\in \closedinterval{0,1}$. Hence the map $f_{x}: \closedinterval{0, 1} \to \mathbb{R}^{n}\smallsetminus\set{0}$ given by
		      \begin{equation*}
			      f_{x}(t) = (f_{1}(t), \ldots, f_{n}(t))
		      \end{equation*}

		      is continuous, due to the characteristic property of product topology. Hence $f$ is a path in $\mathbb{R}^{n}\smallsetminus\set{0}$ from $x$ to $v_{x}$.
		\item a path in $\mathbb{R}^{n}\smallsetminus\set{0}$ from $v_{x}$ to $v_{y}$

		      The map $f: \closedinterval{0, 1} \to \mathbb{R}^{n}\smallsetminus\set{0}$ given by
		      \begin{equation*}
			      f(t) = (1 - t)v_{x} + tv_{y}
		      \end{equation*}

		      is continuous (this map is well-defined because the line segment connecting $v_{x}$ and $v_{y}$ lies entirely in $\mathbb{R}^{n}\smallsetminus\set{0}$) so there is a path in $\mathbb{R}^{n}$ from $v_{x}$ to $v_{y}$.
		\item a path in $\mathbb{R}^{n}\smallsetminus\set{0}$ from $v_{y}$ to $y$

		      Similar to the first contruction, we can construct a path $f_{y}$ in $\mathbb{R}^{n}\smallsetminus\set{0}$ from $v_{y}$ to $y$.
	\end{itemize}

	From these constructions, we deduce that there are continuous maps $g_{x}: \closedinterval{0, \frac{1}{3}} \to \mathbb{R}^{n}\smallsetminus\set{0}$ such that $g_{x}(0) = x$ and $g_{x}(1/3) = v_{x}$, $g: \closedinterval{\frac{1}{3}, \frac{2}{3}} \to \mathbb{R}^{n}\smallsetminus\set{0}$ such that $g(1/3) = v_{x}$ and $g(2/3) = v_{y}$, $g_{y}: \closedinterval{\frac{2}{3}, 1} \to \mathbb{R}^{n}\smallsetminus\set{0}$ such that $g_{y}(2/3) = v_{y}$ and $g_{y}(1) = y$. By the gluing lemma, there is a unique continuous map $f: \closedinterval{0, 1} \to \mathbb{R}^{n}\setminus \set{0}$ such that $f\vert_{\closedinterval{0, \frac{1}{3}}} = g_{x}$, $f\vert_{\closedinterval{\frac{1}{3}, \frac{2}{3}}} = g$, $f\vert_{\closedinterval{\frac{2}{3}, 1}} = g_{y}$. Hence there is a path in $\mathbb{R}^{n}\smallsetminus\set{0}$ from $x$ to $y$.

	Back to the set $\mathbb{R}^{n}\smallsetminus\set{\varphi^{-1}(p)}$. For every $c, d \in \mathbb{R}^{n}\smallsetminus\set{\varphi^{-1}(p)}$, there is a path in $\mathbb{R}^{n}\smallsetminus\set{0}$ from $c - \varphi^{-1}(p)$ to $d - \varphi^{-1}(p)$, so there is a path in $\mathbb{R}^{n}\smallsetminus\set{\varphi^{-1}(p)}$ from $c$ to $d$. Therefore $\mathbb{R}^{n}\smallsetminus\set{\varphi^{-1}(p)}$ is path-connected. Since $\varphi$ is a homeomorphism
	\begin{equation*}
		U\smallsetminus\set{p} = \varphi(\varphi^{-1}(U\smallsetminus\set{p})) = \varphi(\mathbb{R}^{n}\smallsetminus\set{\varphi^{-1}(p)})
	\end{equation*}

	is also path-connected, which is a contradiction because $U\smallsetminus\set{p}$ is disconnected.

	Thus for $n > 1$, $\mathbb{R}^{n}$ is not homeomorphic to any open subset $U\subseteq \mathbb{R}$.
\end{proof}

\begin{problem}{4-2}\label{problem:4-2}
\textsc{Invariance of Dimension, 1-Dimensional Case:} Prove that a nonempty topological space cannot be both a 1-manifold and an $n$-manifold for some $n > 1$.
\end{problem}

\begin{proof}
	Let $M$ be a nonempty $n$-manifold where $n > 1$. Assume that $M$ is also a 1-manifold. Let $x$ be a point of $M$. Because $M$ is an $n$-manifold and a 1-manifold, $x$ has a neighborhood $U$ which admits a homeomorphism $\varphi: U\to \mathbb{R}^{n}$ and a neighborhood $V$ which admits a homeomorphism $\psi: V\to \mathbb{R}$. $U\cap V$ is nonempty because it contains $x$ and it is open. The restrictions $\varphi\vert_{U\cap V}: U\cap V \to \varphi(U\cap V)$ and $\psi\vert_{U\cap V}: U\cap V \to \psi(U\cap V)$ are also a homemorphisms. Since $\varphi(U\cap V)$ is open (because a homemorphism is an open map), there is an open $n$-ball $B^{n}_{r}(\varphi(x)) \subseteq \varphi(U\cap V)$. Denote by $W$ the preimage $\varphi^{-1}(B^{n}_{r}(\varphi(x)))$ then $W\subseteq U\cap V$. The restrictions $\varphi\vert_{W}: W \to \varphi(W) = B^{n}_{r}(\varphi(x))$ and $\psi\vert_{W}: W \to \psi(W)$ are also homeomophism, so the open $n$-ball $B^{n}_{r}(\varphi(x))$ and $\psi(W)$ are homeomorphic. On the other hand, every open $n$-ball is homeomorphic to $\mathbb{R}^{n}$ and $\mathbb{R}^{n}$ is not homeomorphic to $\psi(W)$ (which is an open subset of $\mathbb{R}$) according to Problem~\ref{problem:4-1}, hence the contradiction. Thus a nonempty topological space cannot be both a 1-manifold and an $n$-manifold for some $n > 1$.
\end{proof}

\begin{problem}{4-3}
\textsc{Invariance of the Boundary, 1-Dimensional Case:} Suppose $M$ is a 1-dimensional manifold with boundary. Show that a point of $M$ cannot be both a boundary point and an interior point.
\end{problem}

\begin{proof}
	Firstly, we prove that $\mathbb{R}$ and $\mathbb{H}$ are not homeomorphic. Assume that there is a homeomorphism $f: \mathbb{H} \to \mathbb{R}$. $\mathbb{H}\smallsetminus\set{0}$ is path-connected, however, $\mathbb{R}\smallsetminus\set{f(0)}$ is not path-connected, which is a contradiction because a homeomorphism preserves path-connectedness. Hence $\mathbb{R}$ and $\mathbb{H}$ are not homeomorphic.

	Assume that $M$ has a point $x$ which is both a boundary point and an interior point. Because $x$ is a boundary point, $x$ is in the domain of a boundary chart $(U, \varphi)$ where $\varphi(x) \in \partial\mathbb{H}$ (which implies $\varphi(x) = 0$). Because $x$ is an interior point, $x$ is in the domain of an interior chart $(V, \psi)$. $U\cap V$ is a neighborhood of $x$ and the restrictions $\varphi\vert_{U\cap V}: U\cap V \to \varphi(U\cap V)$, $\psi\vert_{U\cap V}: U\cap V \to \psi(U\cap V)$. Since $\varphi(U\cap V)$ is open and $\varphi(x) = 0$, there exists $r > 0$ such that $\halfopenright{0, r} \subseteq \varphi(U\cap V)$.

	Denote $W = \varphi^{-1}(\halfopenright{0, r})$ then $W \subseteq U\cap V$. It follows that $\halfopenright{0, r}$ and $W$ are homeomorphic, $W$ and $\psi(W)$ are homeomorphic. On the other hand, $W$ is open (because $\halfopenright{0, a} \subseteq \mathbb{H}$ is open) and $\varphi$ is continuous, so $\psi(W) \subseteq \mathbb{R}$ is open (because $\psi$ is a homeomorphism, hence an open map), so $\halfopenright{0, r} \subseteq \mathbb{H}$ is homeomorphic to an open subset $A\subseteq \mathbb{R}$.

	Since $\halfopenright{0, r}$ is connected, $A$ is also connected. $A$ is a connected and open subset of $\mathbb{R}$ so $A$ is an open interval. $\halfopenright{0, r}$ is homeomorphic to $\mathbb{H}$, an open interval is homeomorphic to $\mathbb{R}$, hence $\mathbb{R}$ and $\mathbb{H}$ are homeomorphic, which is a contradiction.

	Thus a point of a 1-dimensional manifold with boundary cannot be both a boundary point and an interior point.
\end{proof}

\begin{problem}{4-4}
Show that the following topological spaces are not manifolds
\begin{enumerate}[label={(\alph*)}]
	\item the union of the $x$-axis and the $y$-axis in $\mathbb{R}^{2}$
	\item the conical surface $C\subseteq \mathbb{R}^{3}$ defined by
	      \begin{equation*}
		      C = \set{(x,y,z) : z^{2} = x^{2} + y^{2}}
	      \end{equation*}
\end{enumerate}
\end{problem}

\begin{proof}
	\begin{enumerate}[label={(\alph*)}]
		\item Assume that $M = (\mathbb{R}\times\set{0}) \cup (\set{0}\times\mathbb{R}) \subseteq \mathbb{R}^{2}$ is an $n$-manifold for some positive integer $n$.

		      Let $x$ be a nonzero real number. From the definition of manifold, $x$ has a neighborhood $U$ which is homeomorphic to an open subset of $\mathbb{R}^{n}$. On the other hand, a basis for the topology on $M$ is obtained by taking the intersection of $M$ and open balls in $\mathbb{R}^{2}$, so there is an open ball $B_{r}(\tuple{x,0})$ such that $\tuple{x,0} \in B_{r}(\tuple{x,0}) \cap M \subseteq U$. Let $\varepsilon$ be a positive number such that $\varepsilon < \min\set{r, \abs{x}}$ then
		      \begin{equation*}
			      \tuple{x, 0} \in \openinterval{x - \varepsilon, x + \varepsilon} \times \set{0} \subseteq B_{r}(\tuple{x,0}) \cap M \subseteq U.
		      \end{equation*}

		      $\openinterval{x - \varepsilon, x + \varepsilon} \times \set{0}$ is a neighborhood of $(x, 0)$ in $M$ and it is homeomorphic to an open subset of $\mathbb{R}$ and an open subset of $\mathbb{R}^{n}$. From Problem~\ref{problem:4-2}, we deduce that $n = 1$.

		      Because $M$ is a 1-manifold, $\tuple{0,0}$ has a neighborhood $V$ which admits a homeomorphism $\varphi: V \to \mathbb{R}$. There is an open ball $B_{r}(\tuple{0,0})$ in $\mathbb{R}^{2}$ such that
		      \begin{equation*}
			      \tuple{0, 0} \in B_{r}(\tuple{0, 0}) \cap M \subseteq M.
		      \end{equation*}

		      Denote $B_{r}(\tuple{0, 0}) \cap M$ by $W$ then the restriction $\varphi\vert_{W}: W \to \varphi(W)$ is also a homeomorphism. $W$ is path-connected because it is the union of path-connected sets with a point in common (the origin)
		      \begin{equation*}
			      W = (\openinterval{-r, r} \times\set{0}) \cup (\set{0} \times \openinterval{-r, r}).
		      \end{equation*}

		      so $\varphi(W) \subseteq \mathbb{R}$ is path-connected, hence it is an interval. $W\smallsetminus\set{0}$ has four path components, namely
		      \begin{equation*}
			      \openinterval{0, r}\times\set{0};\quad \openinterval{-r, 0}\times\set{0};\quad \set{0}\times\openinterval{0, r};\quad \set{0}\times\openinterval{-r, 0}
		      \end{equation*}

		      but $\varphi(W\smallsetminus\set{0}) \subseteq \mathbb{R}$ has two path components (because it is an open interval minus a point), which is a contradiction.

		      Thus $M$ is not a manifold.
		\item Assume that $C$ is an $n$-manifold for some positive integer $n$.

		      Every point on $C$ other than $(0, 0, 0)$ is of the form $(r\cos\theta, r\sin\theta, r)$ for some $r\ne 0$. Consider a point $(r\cos\theta, r\sin\theta, r)$ where $r > 0$. The set $H = \set{ \tuple{x, y, z} \in \mathbb{R}^{3} : z > 0 }$ is open in $\mathbb{R}^{3}$, so $H \cap C$ is open in $C$ (using the subspace topology). In fact, $H\cap C$ is the set of points on $C$ with positive $z$-ordinate. The map $f: H\cap C \to \mathbb{R}^{2}$ given by $f(x, y, z) = (x, y)$ is a homeomorphism, so $(r\cos\theta, r\sin\theta, r)$ has a neighborhood that is homeomorphic to $\mathbb{R}^{2}$. From Problem~\ref{problem:4-2} we deduce that $n > 1$.

		      $\tuple{0, 0, 0}$ has a neighborhood $U \subseteq C$ which is homeomorphic to $\mathbb{R}^{n}$ with the coordinate map $\varphi$. There exists $r > 0$ such that
		      \begin{equation*}
			      \tuple{0,0,0} \in B_{r}(\tuple{0,0,0}) \cap C \subseteq U \cap C
		      \end{equation*}

		      because the set of open balls is a basis for the Euclidean topology on $\mathbb{R}^{3}$ (and taking the intersection with a subset of $\mathbb{R}^{3}$, we obtain a basis for the subspace topology).

		      Denote $B_{r}(\tuple{0,0,0}) \cap C$ by $V$. $V$ is connected. $V\smallsetminus\set{\tuple{0,0,0}}$ is disconnected since it has two components, namely
		      \begin{equation*}
			      \begin{split}
				      \set{\tuple{(x, y, z)} \in \mathbb{R}^{3} : 0 < z < r} \cap C \\
				      \set{\tuple{(x, y, z)} \in \mathbb{R}^{3} : -r < z < 0} \cap C
			      \end{split}
		      \end{equation*}

		      and $\varphi(V \smallsetminus \set{\tuple{0,0,0}})$ is therefore disconnected. However, because $\varphi(V)$ is connected, $\varphi(V \smallsetminus \set{\tuple{0,0,0}}) = \varphi(V) \smallsetminus\set{\varphi(\tuple{0,0,0})}$. A connected open subset in $\mathbb{R}^{n}$ (where $n > 1$) minus a point is still connected, hence a contradiction.

		      Thus $C$ is not a manifold.
	\end{enumerate}
\end{proof}

\begin{problem}{4-5}
Let $M = \mathbb{S}^{1}\times\mathbb{R}$, and let $A = \mathbb{S}^{1} \times \set{0}$. Show that the space $M/A$ obtained by collapsing $A$ to a point is homeomorphic to the space $C$ of Problem 4-4 (b), and thus is Hausdorff and second countable but not locally Euclidean.
\end{problem}

\begin{proof}
	Consider the map $f: \mathbb{S}^{1}\times\mathbb{R} \to C$ given by
	\begin{equation*}
		f(e^{\iota\theta}, r) = (r\cos\theta, r\sin\theta, r)
	\end{equation*}

	is a quotient map. Meanwhile, the quotient map $q: M\to M/A$ and $f$ have the same identification, hence $M/A$ and $C$ are homeomorphism, according to the uniqueness of quotient space. Thus $C$ is Hausdorff, second countable but not locally Euclidean.
\end{proof}

% \begin{problem}{4-6}
% Like Problem~\ref{problem:2-22}, this problem constructs a space that is locally Euclidean and Hausdorff but not second countable. Unlike that example, however, this one is connected.
% \begin{enumerate}[label={(\alph*)}]
% 	\item Recall that a totally ordered set is said to be well ordered if every nonempty subset has a smallest element. Show that the well-ordering theorem implies that there exists an uncountable well-ordered set $Y$ such that for every $y_{0}\in Y$, there are only countably many $y\in Y$ such that $y < y_{0}$.
% 	\item Now let $\mathscr{R} = Y \times \halfopenright{0,1}$, with the \textbf{dictionary order}: this means that $(y_{1}, s_{1}) < (y_{2}, s_{2})$ if either $y_{1} < y_{2}$, or $y_{1} = y_{2}$ and $s_{1} < s_{2}$. With the order topology, $\mathscr{R}$ is called the \textbf{long ray}. The \textbf{long line} $\mathscr{L}$ is the wedge sum $\mathscr{R}\vee \mathscr{R}$ obtained by identifying both copies of $(y_{0}, 0)$ with each other, where $y_{0}$ is the smallest element in $Y$. Show that $\mathscr{L}$ is locally Euclidean, Hausdorff, and first countable, but not second countable.
% 	\item Show that $\mathscr{L}$ is path-connected.
% \end{enumerate}
% \end{problem}

% \begin{problem}{4-7}
% Let $q: X\to Y$ be an open quotient map. Show that if $X$ is locally connected, locally path-connected, or locally compact, then $Y$ has the same property.
% \end{problem}

% \begin{problem}{4-8}
% Show that a locally connected topological space is homeomorphic to the disjoint union of its components.
% \end{problem}
