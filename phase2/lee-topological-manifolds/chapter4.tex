% chktex-file 8
% chktex-file 44
\chapter{Connectedness and Compactness}

\section*{Connectedness}\addcontentsline{toc}{section}{Connectedness}

\subsection*{Definitions and Basic Properties}\addcontentsline{toc}{subsection}{Definitions and Basic Properties}

\begin{exercise}{4.3}
	Suppose $X$ is a connected topological space, and $\sim$ is an equivalence relation on $X$ such that every equivalence class is open. Show that there is exactly one equivalence class, namely $X$ itself.
\end{exercise}

\begin{proof}
	Suppose $X$ has exactly $n$ equivalence classes, all of which are open. Because all equivalence classes constitute a partition of $X$, then $X$ is the union of these open, disjoint equivalence classes. On the other hand, $X$ is connected, so $n = 1$. Hence $X$ has exactly one equivalence class.
\end{proof}

\begin{exercise}{4.4}
	Prove that a topological space $X$ is disconnected if and only if there exists a nonconstant continuous function from $X$ to the discrete space $\set{0,1}$.
\end{exercise}

\begin{proof}
	Suppose there is a nonconstant continuous function $f: X\to \set{0,1}$, then $f^{-1}(0)$ and $f^{-1}(1)$ are nonempty, open, disjoint, and their union is $X$. Therefore $X$ is disconnected.

	Conversely, suppose $X$ is disconnected, then there are nonempty, open, disjoint subsets $U, V\subseteq X$ such that $X = U\cup V$. Define $f: X\to \set{0,1}$ as follows: $f(x) = 0$ if $x\in U$ and $f(x) = 1$ if $x\notin U$, then $f$ is nonconstant and $f$ is continuous (because the preimages under $f$ of $\varnothing, \set{0}, \set{1}, \set{0, 1}$ are open in $X$).
\end{proof}

\begin{exercise}{4.5}
	Prove that a topological space is disconnected if and only if it is homeomorphic to a disjoint union of two or more nonempty spaces.
\end{exercise}

\begin{proof}
	Let $X$ be a topological space.

	Suppose $X$ is disconnected, then there are subsets $U, V\subseteq X$ that disconnect $X$. The map $\varphi: X \to U\sqcup V$ given by
	\begin{equation*}
		\varphi(x) = \begin{cases}
			(x, 0) & \text{if $x \in U$} \\
			(x, 1) & \text{if $x \in V$}
		\end{cases}
	\end{equation*}

	is a homeomorphism, so $X$ is homeomorphic to the disjoint union $U\sqcup V$. Hence $X$ is homeomorphic to a disjoint union of at least two nonempty spaces.

	Conversely, suppose $X$ is homeomorphic to a disjoint union of $n\geq 2$ nonempty spaces $U_{1}, \ldots, U_{n}$. Let $f: X\to \coprod^{n}_{i=1}U_{i}$ be a homeomorphism, then $f^{-1}(U_{1}), \ldots, f^{-1}(U_{n})$ constitutes a partition on $X$, and $f^{-1}(U_{1}), \ldots, f^{-1}(U_{n})$ are nonempty open subsets of $X$. Because $n\geq 2$, it follows that $X$ is disconnected.
\end{proof}

\begin{exercise}{4.10}
	Suppose $M$ is a connected manifold with nonempty boundary. Show that its double $D(M)$ is connected.
\end{exercise}

\begin{proof}
	Let $h: \partial M\to \partial M$ be the identity map from the boundary of $M$ onto itself. $\sim$ is the equivalence relation on $D(M)$ defined by $x\sim h(x)$ for $x\in\partial M$, and other points are equivalent to itself. By the definition of the double of a manifold with boundary
	\begin{equation*}
		D(M) = M\cup_{h}M = (M\sqcup M)/_{\sim}.
	\end{equation*}

	Assume that $D(M)$ is disconnected by open subsets $U, V$. Let $q: M \sqcup M \to (M \sqcup M)/_{\sim}$ be the quotient map, then $q^{-1}(U)$ and $q^{-1}(V)$ disconnect $M\sqcup M \approx (M\times\set{0}) \cup (M\times\set{1})$. Because $M\times\set{0}, M\times\set{1} \subseteq q^{-1}(U) \cup q^{-1}(V)$ and $M\times\set{0}, M\times\set{1}$ are connected, it follows from Proposition 4.9 that either $M\times\set{0} \subseteq q^{-1}(U)$ or $M\times\set{0} \subseteq q^{-1}(V)$, either $M\times\set{1} \subseteq q^{-1}(U)$ or $M\times\set{1} \subseteq q^{-1}(V)$. Because $q^{-1}(U)$ and $q^{-1}(V)$ are nonempty, it follows that either $M\times\set{0} \subseteq q^{-1}(U)$ and $M\times\set{1} \subseteq q^{-1}(V)$ or $M\times\set{0} \subseteq q^{-1}(V)$ and $M\times\set{1} \subseteq q^{-1}(U)$. Without loss of generality, suppose that the former is the case. Since the boundary $\partial M$ is nonempty, there is $a \in \partial M$. $a \sim h(a)$, $(a, 0) \in M\times\set{0} = q^{-1}(U)$ and $(a, 1) \in M\times\set{1} = q^{-1}(V)$ so $q(a) = q(h(a)) \in U, V$. Hence $U$ and $V$ are not disjoint, which contradicts our assumption.

	Thus $D(M)$ is connected.
\end{proof}

\subsection*{Path Connectedness}\addcontentsline{toc}{subsection}{Path Connectedness}

\begin{exercise}{4.14}\label{exercise:4.14}
	Prove Proposition 4.13 (Properties of Path-Connected Spaces).
	\begin{enumerate}[label={(\alph*)}]
		\item Every continuous image of a path-connected space is path-connected.
		\item Let $X$ be a space, and let ${\{ B_{\alpha} \}}_{\alpha\in A}$ be a collection of path-connected subspaces of $X$ with a point in common. Then $\bigcup_{\alpha\in A}B_{\alpha}$ is path-connected.
		\item Every product of finitely many path-connected spaces is path-connected.
		\item Every quotient space of a path-connected space is path-connected.
	\end{enumerate}
\end{exercise}

\begin{proof}
	\begin{enumerate}[label={(\alph*)}]
		\item Let $f: X\to Y$ be a continuous map and $X$ is a path-connected space. Let $p, q$ be two points of $f(X)$, let $a \in f^{-1}(p)$ and $b \in f^{-1}(q)$. Because $X$ is path-connected, there is a continuous map $g: \closedinterval{0, 1} \to X$ such that $g(0) = a$ and $g(1) = b$, so the composition $f\circ g: \closedinterval{0, 1} \to f(X)$ is continuous and $(f\circ g)(0) = p$ and $(f\circ g)(1) = q$. Hence for every two points $p, q$ of $f(X)$, there is a continuous map from $\closedinterval{0, 1}$ to $f(X)$ such that the images of $0, 1$ are $p, q$, respectively, which implies $f(X)$ is path-connected.
		\item Let $p, q$ be two points of $\bigcup_{\alpha\in A}B_{\alpha}$.

		      $p \in B_{\alpha_{p}}$ and $q \in B_{\alpha_{q}}$ for some $\alpha_{p}, \alpha_{q} \in A$. Let $x \in \bigcap_{\alpha\in A}B_{\alpha}$ (these sets have a point in common). Because $B_{\alpha_{p}}, B_{\alpha_{q}}$ are path-connected, there are continuous maps $f_{p}: \closedinterval{0, 1} \to B_{\alpha_{p}}$ such that $f_{p}(0) = p, f_{p}(1) = x$, and $f_{q}: \closedinterval{0, 1} \to B_{\alpha_{q}}$ such that $f_{q}(0) = x, f_{q}(1) = q$.

		      The maps $g: \closedinterval{0, \frac{1}{2}} \to \closedinterval{0, 1}$ given by $g(t) = 2t$ and $h: \closedinterval{\frac{1}{2}, 1} \to \closedinterval{0, 1}$ given by $h(t) = 2t - 1$ are continuous. The compositions $f_{p}\circ g$ and $f_{q}\circ h$ are therefore continuous, and they agree on $\closedinterval{0, \frac{1}{2}} \cap \closedinterval{\frac{1}{2}, 1}$, since $(f_{p}\circ g)(1/2) = x = (f_{q}\circ h)(1/2)$. $\closedinterval{0, \frac{1}{2}}, \closedinterval{\frac{1}{2}, 1}$ constitute a finite closed cover of $\closedinterval{0, 1}$, so by the gluing lemma, there is a unique continuous map $f: \closedinterval{0, 1} \to B_{\alpha_{p}} \cup B_{\alpha_{q}}$ such that $f\vert_{\closedinterval{0, \frac{1}{2}}} = f_{p}\circ g$ and $f\vert_{\closedinterval{\frac{1}{2}, 1}} = f_{q}\circ h$. Moreover, $f(0) = p, f(1) = q$. Hence there is a path in $\bigcup_{\alpha\in A}B_{\alpha}$ from $p$ to $q$.

		      Thus $\bigcup_{\alpha\in A}B_{\alpha}$ is path-connected.
		\item It suffices to prove that the product of two path-connected spaces is path-connected.

		      Let $X, Y$ be path-connected spaces and $(x_{1}, y_{1}), (x_{2}, y_{2})$ are two points of $X\times Y$. The maps $i_{y_{0}}: X\to X\times Y$ given by $i_{y_{0}}(x) = (x, y_{0})$ and $i_{x_{0}}: Y\to X\times Y$ given by $i_{x_{0}}(y) = (x_{0}, y)$ are continuous. From part (a), it follows that $X\times\set{y_{0}}$ and $\set{x_{0}}\times Y$ are path-connected for every $y_{0} \in Y, x_{0}\in X$. Hence there is a path $f_{1}$ in $X\times Y$ from $(x_{1}, y_{1})$ to $(x_{2}, y_{1})$ and a path $f_{2}$ in $X\times Y$ from $(x_{2}, y_{1})$ to $(x_{2}, y_{2})$.

		      The maps $g: \closedinterval{0, \frac{1}{2}} \to \closedinterval{0, 1}$ given by $g(t) = 2t$ and $h: \closedinterval{\frac{1}{2}, 1} \to \closedinterval{0, 1}$ given by $h(t) = 2t - 1$ are continuous. The compositions $f_{1}\circ g$ and $f_{2}\circ h$ are therefore continuous, and they agree on $\closedinterval{0, \frac{1}{2}} \cap \closedinterval{\frac{1}{2}, 1}$, since $(f_{1}\circ g)(1/2) = (x_{2}, y_{1}) = (f_{2}\circ h)(1/2)$. From the gluing lemma, it follows that there is a unique continuous map $f: \closedinterval{0, 1} \to X\times Y$ such that $f\vert_{\closedinterval{0, \frac{1}{2}}} = f_{1}\circ g$ and $f\vert_{\closedinterval{\frac{1}{2}, 1}} = f_{2}\circ h$. Moreover, $f(0) = (x_{1}, y_{1})$ and $f(1) = (x_{2}, y_{2})$.

		      Therefore $X\times Y$ is path-connected. From mathematical induction, it follows that the finite product of path-connected spaces is path-connected.
		\item Since every quotient map is continuous and surjective, from part (a), it follows that every quotient space of a path-connected space is path-connected.
	\end{enumerate}
\end{proof}

\subsection*{Components and Path Components}\addcontentsline{toc}{subsection}{Components and Path Components}

\begin{exercise}{4.22}
	Prove Proposition 4.21 (Properties of Path Components).

	Let $X$ be any space.
	\begin{enumerate}[label={(\alph*)}]
		\item The path components of $X$ form a partition of $X$.
		\item Each path component is contained in a single component, and each component is a disjoint union of path components.
		\item Any nonempty path-connected subset of $X$ is contained in a single path component.
	\end{enumerate}
\end{exercise}

\begin{proof}
	\begin{enumerate}[label={(\alph*)}]
		\item Let $U, V$ be non-disjoint path components of $X$. From Exercise~\ref{exercise:4.14} (b), $U\cup V$ is path-connected. Due to the maximality of path components, $U = V = U\cup V$, from which we deduce that non-disjoint path components are identical. Hence distinct path components are disjoint.

		      Let $x$ be a point of $X$. The singleton $\set{x}$ is a path component containing $x$. Let ${(B_{\alpha})}_{\alpha\in A}$ be the family of all path-connected sets containing $x$, then $\bigcup_{\alpha\in A}B_{\alpha}$ is path-connected (according to Exercise~\ref{exercise:4.14} (b)). Moreover $\bigcup_{\alpha\in A}B_{\alpha}$ is a maximal path-connected set so it is a path component containing $x$. Therefore every element of $X$ is in a path component.

		      Thus the path components of $X$ form a partition of $X$.

			      [Another approach to part (a) is to prove path-connectedness of two points is an equivalence relation on the given topological space.]
		\item Let $P$ be a path component of $X$. Because the components of $X$ form a partition of $X$, $P$ has a common point with some component $C$ of $X$. Since a path-connected set is also connected, $P$ is a connected set. Because the union of connected sets with a point in common is connected, $P\cup C$ is connected. From the maximality of $C$, we deduce that $P\cup C = C$, which means $P\subseteq C$. Moreover, distinct components are disjoint, so $P$ is contained in the component $C$ only. Hence every path component is contained in a single component.

		      Let $C$ be a component of $X$. Every point $p$ of $C$ is in some path component of $X$, so the path component containing $p$ is contained in $C$. On the other hand, distinct path components are disjoint. Hence $C$ is a disjoint union of path components.
		\item Let $A$ be a nonempty path-connected subset of $X$.

		      Let $x$ be an element of $A$. According to part (a), $x$ is a point of a path component $P$. According to Exercise~\ref{exercise:4.14} (b), $A\cup P$ is path-connected. Because $P$ is a maximal path-connected set, it follows that $A\cup P = P$, which means $A\subseteq P$.

		      If $A$ is contained in two path components, then the two path components are not disjoint (because $A$ is nonempty) and it follows that the two path components are identical (according to part (a)).

		      Therefore any nonempty path-connected subset of $X$ is contained in a single path component of $X$.
	\end{enumerate}
\end{proof}

\begin{exercise}{4.24}
	Prove Proposition 4.23.

	Every manifold (with or without boundary) is locally connected and locally path-connected.
\end{exercise}

\begin{proof}
	Let $M$ be an $n$-manifold (with or without boundary).

	\textbf{Case 1. $M$ is an $n$-manifold without boundary.}

	According to Problem~\ref{problem:2-23}, every manifold has a basis of coordinate balls. On the other hand, every coordinate ball is homeomorphic to an open ball of $\mathbb{R}^{n}$ and every open ball of $\mathbb{R}^{n}$ is path-connected because every open ball of $\mathbb{R}^{n}$ is homeomorphic to $\mathbb{R}^{n}$ (which is path-connected). Therefore every coordinate ball is path-connected (and hence connected). Hence $M$ is locally path-connected and locally connected.

	\textbf{Case 2. $M$ is an $n$-manifold with boundary.}

	Firstly, we construct a basis for $M$. Let $U$ be a nonempty open subset of $M$ and $x\in U$, then $x$ is in the domain of an interior chart or that of a boundary chart.

	If $x$ is in the domain of an interior chart $(V, \varphi_{x})$, then $\varphi_{x}(V)$ is an open subset of $\mathbb{R}^{n}$. $U\cap V$ is homeomorphic to $\varphi_{x}(U\cap V)$ and $\varphi_{x}(U\cap V)$ is an open subset of $\mathbb{R}^{n}$. Because $\varphi_{x}(U\cap V)$ is open and $\varphi_{x}(x)$ is a point of this set, there is an open ball $B_{r}(\varphi_{x}(x)) \subseteq \varphi_{x}(U\cap V)$. Therefore $\varphi_{x}^{-1}(B_{r}(\varphi_{x}(x)))$ and $B_{r}(\varphi_{x}(x))$ are homeomorphic. So $x$ is in the domain of the following interior chart $(\varphi_{x}^{-1}(B_{r}(\varphi_{x}(x))), \varphi_{x}\vert_{\varphi_{x}^{-1}(B_{r}(\varphi_{x}(x)))})$ where the domain is contained in $U$ and homeomorphic to an open ball in $\mathbb{R}^{n}$.

	If $x$ is in the domain of a boundary chart $(V, \varphi_{x})$, then $\varphi_{x}(x) \in \partial\mathbb{H}^{n}$ and $\varphi_{x}(V)$ is an open subset of $\mathbb{H}^{n}$. $U\cap V$ is homeomorphic to $\varphi_{x}(U\cap V)$ and $\varphi_{x}(U\cap V)$ is an open subset of $\mathbb{H}^{n}$. Because $\varphi_{x}(U\cap V)$ is open and $\varphi_{x}(x)$ is a point of this set, there is an open ball $B_{r}(\varphi_{x}(x))$ in $\mathbb{R}^{n}$ such that $B_{r}(\varphi_{x}(x)) \cap \mathbb{H}^{n} \subseteq \varphi_{x}(U\cap V)$ (here we make use of the subspace topology on $\mathbb{H}^{n}$ and the basis for $\mathbb{R}^{n}$ containing open balls). Therefore $W = B_{r}(\varphi_{x}(x)) \cap \mathbb{H}^{n}$ and $\varphi_{x}^{-1}(W)$ are homeomorphic, and $x$ is in the domain of the following boundary chart $(\varphi_{x}^{-1}(W), \varphi_{x}\vert_{W})$ where the domain is contained in $U$ and is homeomorphic to an open half-ball in $\mathbb{H}^{n}$ (it is halved by taking intersection of $\mathbb{H}^{n}$ and an open ball in $\mathbb{R}^{n}$ with center on $\partial \mathbb{H}^{n}$).

	Hence the collection of open sets of $M$ which are domains of some interior chart (and homeomorphic to some open ball in $\mathbb{R}^{n}$) or some boundary chart  (and homeomorphic to some open half-ball in $\mathbb{H}^{n}$) constitutes a basis for the $n$-manifold with boundary $M$.

	On the other hand, an open ball or an open half-ball is path-connected (because it is a convex set), so $M$ has a basis of path-connected (hence connected) open sets. Therefore $M$ is locally path-connected and locally connected.

	From the two cases, we conclude that every manifold (without or with boundary) is locally path-connected and locally connected.
\end{proof}

\section*{Compactness}\addcontentsline{toc}{section}{Compactness}

\subsection*{Definitions and Basic Properties}\addcontentsline{toc}{subsection}{Definitions and Basic Properties}

\begin{exercise}{4.28}
	Prove Lemma 4.27 (Compactness Criterion for Subspaces).

	If $X$ is any topological space, a subset $A\subseteq X$ is compact (in the subspace topology) if and only if every cover of $A$ by open subsets of $X$ has a finite subcover.
\end{exercise}

\begin{proof}
	Suppose $A\subseteq X$ is compact with the subspace topology. Let ${(U_{i})}_{i\in I}$ be a cover of $A$ by open subsets of $X$, then ${(U_{i}\cap A)}_{i\in I}$ is a cover of $A$ by open subsets of $A$. Because $A$ is compact with the subspace topology, ${(U_{i}\cap A)}_{i\in I}$ contains a finite subcover of $A$ by open subsets of $A$, which implies ${(U_{i})}_{i\in I}$ contains a finite subcover of $A$ by open subsets of $X$.

	Conversely, suppose that every cover of $A$ by open subsets of $X$ has a finite subcover. Let ${(V_{i})}_{i\in I}$ be a cover of $A$ by open subsets of $A$. From the definition of subspace topology, for each $i\in I$, there is an open subset $U_{i}\subseteq X$ such that $V_{i} = U_{i}\cap A$. Therefore ${(U_{i})}_{i\in I}$ is a cover of $A$ by open subsets of $X$, so ${(U_{i})}_{i\in I}$ contains a finite subcover of $A$ by open subsets of $X$, which implies that ${(V_{i})}_{i\in I}$ contains a finite subcover of $A$ by open subsets of $A$. Hence $A\subseteq X$ is compact with the subspace topology.
\end{proof}

\begin{exercise}{4.29}
	In any topological space $X$, show that every union of finitely many compact subsets of $X$ is compact.
\end{exercise}

\begin{proof}
	It suffices to prove that the union of two compact subsets of $X$ is compact. Let $A, B$ be compact subsets of $X$ and $\mathcal{O}$ be an open cover of $A\cup B$ by open subsets of $X$. Since $A$ is compact, there exist finitely many open subsets $A_{1}, \ldots, A_{m}$ from $\mathcal{O}$ that covers $A$ and finitely many open subsets $B_{1}, \ldots, B_{n}$ from $\mathcal{O}$ that covers $B$. Hence $\mathcal{O}$ has a finite subcover, which consists of $A_{1}, \ldots, A_{m}, B_{1}, \ldots, B_{n}$. Therefore $A\cup B$ is compact.

	The union of zero compact subsets of $X$ is compact (because it is the empty set). Assume that the union of $n - 1$ compact subsets of $X$ is compact. Let $A_{1}, \ldots, A_{n}$ be $n$ compact subsets of $X$. By the inductive hypothesis, $\bigcup^{n-1}_{i=1}A_{i}$ is a compact subset of $X$. From the previous paragraph, we deduce that $\bigcup^{n}_{i=1}A_{i}$ is a compact subset of $X$. By the principle of mathematical induction, we conclude that every union of finitely many compact subsets of $X$ is compact.
\end{proof}

\begin{exercise}{4.37}
	Suppose $M$ is a compact manifold with boundary. Show that the double of $M$ is compact.
\end{exercise}

\begin{proof}
	Let $h$ be the identity map $\partial M\to \partial M$. From the definition of the double of a manifold with boundary, $D(M) = M\cup_{h} M = {(M\sqcup M)}/_{\sim}$ (where $(a, 0) \sim (h(a), 1)$ for $a\in \partial H$). Because $M\sqcup M \approx (M\times\set{0}) \cup (M\times\set{1})$ is the union of two compact manifolds (which are homeomorphic to the compact manifold $M$), it follows that $M\sqcup M$ is compact.

	Let $q: M\sqcup M \to (M\sqcup M)/_{\sim} = D(M)$ be the quotient map. Because every quotient of a compact space is compact, it follows that $D(M)$ is compact.
\end{proof}

\begin{exercise}{4.38}
	Let $X$ be a compact space, and suppose $\set{F_{n}}$ is a countable collection of nonempty closed subsets of $X$ that are \textbf{nested}, which means that $F_{n}\supseteq F_{n+1}$ for each $n$. Show that $\bigcap_{n}F_{n}$ is nonempty.
\end{exercise}

\begin{proof}
	Each $F_{n}$ is a closed subset $X$, so $X\smallsetminus F_{n}$ is open. Assume for the sake of contradiction that $\bigcap_{n}F_{n}$ is empty then
	\begin{equation*}
		X = X\smallsetminus \left(\bigcap_{n}F_{n}\right) = \bigcup_{n}(X\smallsetminus F_{n})
	\end{equation*}

	so $\set{X\smallsetminus F_{n}}_{n}$ is an open cover of $X$. Because of the compactness of $X$, $\set{X\smallsetminus F_{n}}_{n}$ contains a finite subcover, say $X\smallsetminus F_{k_{1}}, \ldots, X\smallsetminus F_{k_{m}}$, and we can relabel these sets such that $k_{1} < \cdots < k_{m}$. Since $F_{k_{1}} \supseteq \cdots \supseteq F_{k_{m}}$, we have $X\smallsetminus F_{k_{1}} \subseteq \cdots \subseteq X\smallsetminus F_{k_{m}}$ hence $X = \bigcup^{m}_{i=1}(X\smallsetminus F_{k_{i}}) = X \smallsetminus F_{k_{m}}$, which is a contradiction since $X$ is a proper superset of $X\smallsetminus F_{k_{m}}$, because $F_{k_{m}}$ is a nonempty subset of $X$. Hence $\bigcap_{n}F_{n}$ is nonempty.
\end{proof}

\subsection*{Sequential and Limit Point Compactness}\addcontentsline{toc}{subsection}{Sequential and Limit Point Compactness}

\begin{exercise}{4.49}
	Prove the preceding three theorems.

	Theorem 4.46 (Bolzano-Weierstraß). Every bounded sequence in $\mathbb{R}^{n}$ has a convergent subsequence.

	Theorem 4.47. Endowed with the Euclidean metric, a subset of $\mathbb{R}^{n}$ is a complete metric space if and only if it is closed in $\mathbb{R}^{n}$. In particular, $\mathbb{R}^{n}$ is complete.

	Theorem 4.48. Every compact metric space is complete.
\end{exercise}

\begin{proof}
	\textbf{Proof for Theorem 4.46.} Let ${(x_{i})}_{i\in\mathbb{N}}$ be a bounded sequence in $\mathbb{R}^{n}$, then there exists $a > 0$ such that $x_{i} \in {\closedinterval{-a, a}}^{n}$ for every $i\in\mathbb{N}$. If the sequence ${(x_{i})}_{i\in\mathbb{N}}$ takes on finitely many values then it has a subsequence that is eventually constant. Otherwise, suppose ${(x_{i})}_{i\in\mathbb{N}}$ takes on infinitely many values. ${\closedinterval{-a, a}}^{n}$ is closed and bounded in $\mathbb{R}^{n}$ so it is compact, due to Heine-Borel's theorem. The space ${\closedinterval{-a, a}}^{n}$ with subspace topology inherited from $\mathbb{R}^{n}$ is second countable (implies first countable) and Hausdorff so from Lemma 4.42 (compactness implies limit point compactness) and Lemma 4.43 (In first countable Hausdorff spaces, limit point compactness implies sequential compactness), it follows that ${(x_{i})}_{i\in\mathbb{N}}$ has a convergent subsequence.

	\textbf{Proof for Theorem 4.47.} Let $M$ be subset of $\mathbb{R}^{n}$, then $M$ is a metric space with the Euclidean metric restricted to $M$. If $M$ is complete, let $x\in \mathbb{R}^{n}$ be a limit point of $M$. For every $i\in\mathbb{N}$, $B_{1/2^{i}}(x)$ contains a point $a_{i}$ in $M$, then the sequence ${(a_{i})}_{i\in\mathbb{N}}$ is a Cauchy sequence. Since $M$ is complete, the sequence converges to a point of $M$. Because $a_{i}$ converges to $x$, it follows that $x\in M$, hence $M$ contains all of its limit points, which means $M$ is closed. Conversely, if $M$ is closed, let ${(x_{i})}_{i\in\mathbb{N}}$ be a Cauchy sequence of points in $M$. In a metric space, a Cauchy sequence is a bounded sequence. By Theorem 4.46, ${(x_{i})}_{i\in\mathbb{N}}$ has a convergent subsequence. A Cauchy sequence having a convergent subsequence is convergent, hence ${(x_{i})}_{i\in\mathbb{N}}$ is convergent, so $M$ is complete.

	\textbf{Proof for Theorem 4.48.} Let $M$ be a compact metric space and ${(x_{i})}_{i\in\mathbb{N}}$ be a Cauchy sequence of points in $M$. $M$ is a metric space so it is first countable and Hausdorff. By Lemma 4.42, $M$ is limit point compact. By Lemma 4.43 (In first countable Hausdorff spaces, limit point compactness implies sequential compactness), $M$ is sequentially compact. Hence ${(x_{i})}_{i\in\mathbb{N}}$ has a convergent subsequence. A Cauchy sequence having a convergent subsequence is convergent, hence ${(x_{i})}_{i\in\mathbb{N}}$ is convergent, so $M$ is complete.
\end{proof}

\subsection*{The Closed Map Lemma}\addcontentsline{toc}{subsection}{The Closed Map Lemma}

\begin{exercise}{4.58}
	Using the map of Example 4.55, show that there is a coordinate ball in $\mathbb{S}^{n}$ whose closure is equal to all of $\mathbb{S}^{n}$.
\end{exercise}

\begin{proof}
	The quotient map of Example 4.55 is $q: {\overline{B}}^{n} \to \mathbb{S}^{n}$ given by
	\begin{equation*}
		q(x) = \tuple{2\sqrt{1 - {\abs{x}}^{2}}x, 2{\abs{x}}^{2} - 1}.
	\end{equation*}

	Let $N$ be the point of $\mathbb{R}^{n+1}$ whose coordinates are $\tuple{0, \ldots, 0, 1}$ then $N\in \mathbb{S}^{n}$. The singleton set $\set{N}$ is closed in $\mathbb{S}^{n}$ so its complement $\mathbb{S}^{n}\smallsetminus \set{N}$ is open in $\mathbb{S}^{n}$. On the other hand, $q^{-1}(N) = \partial{\overline{B}}^{n}$, so $q^{-1}(\mathbb{S}^{n}\smallsetminus \set{N}) = {\overline{B}}^{n} \smallsetminus \partial{\overline{B}}^{n} = B^{n}$. Moreover, the restriction of $q$ on $\mathbb{S}^{n}\smallsetminus \set{N}$ is injective, it follows that $\mathbb{S}^{n}\smallsetminus \set{N}$ is homeomorphic to $B^{n}$, hence $\mathbb{S}^{n}\smallsetminus \set{N}$ is a coordinate ball in $\mathbb{S}^{n}$. The closure of $\mathbb{S}^{n}\smallsetminus \set{N}$ is the entire $\mathbb{S}^{n}$.

	Thus $\mathbb{S}^{n}\smallsetminus \set{N}$ is a coordinate ball in $\mathbb{S}^{n}$ whose closure is equal to all of $\mathbb{S}^{n}$.
\end{proof}

\begin{exercise}{4.61}\label{exercise:4.61}
	Complete the proof of Proposition 4.60 by showing that $\mathscr{B}$ is a basis.
\end{exercise}

\begin{quotation}
	Proposition 4.60. Every manifold has a countable basis of regular coordinate balls.

	The author gave an incorrect proof/idea to this proposition. He suggested a correction on his website.
\end{quotation}

\begin{proof}
	Let $M$ be an $n$-manifold.

	Every point of $M$ has an Euclidean neighborhood (or coordinate domain), and these Euclidean neighborhoods cover $M$. Since $M$ is second countable, every open cover of $X$ has a countable subcover (This is Theorem 2.50). Let $\set{U_{i}}_{i\in\mathbb{N}}$ be such a countable subcover. Each $U_{i}$ is a coordinate domain, so there is a homeomorphism $\varphi_{i}$ from $U_{i}$ onto an open subset $\widehat{U}_{i} \subseteq \mathbb{R}^{n}$.

	Let $\mathscr{B}$ be the collection of all open subsets of $M$ of the form $\varphi_{i}^{-1}(B_{r}(x))$, where $x\in \widehat{U}_{i}$ is a point with rational coordinates only and $r$ is any positive rational number such that there exists a real number $r'$ where $B_{r}(x) \subseteq B_{r'}(x) \subseteq \widehat{U}_{i}$ (such open balls exist because of the density of $\mathbb{Q}^{n}$ in $\mathbb{R}^{n}$ and $\widehat{U}_{i} \subseteq \mathbb{R}^{n}$ is open.)

	By Lemma 4.59 and the way we define the elements of $\mathscr{B}$, every element of $\mathscr{B}$ is a regular coordinate ball. On the other hand, with these choices of coordinates and radii, $\mathscr{B}$ is countable. Now we need to prove $\mathscr{B}$ is a basis for $M$.

	Let $U$ be an open subset of $M$, then $U = \bigcup_{i\in\mathbb{N}}(U\cap U_{i})$. $U\cap U_{i} \subseteq M$ and $\varphi_{i}(U\cap U_{i}) \subseteq \mathbb{R}^{n}$ are homeomorphic (by the homeomorphism $\varphi_{i}$). $\mathbb{R}^{n}$ admits the collection of open balls with rational coordinates and rational radii as a basis, so for every $x\in \varphi_{i}(U\cap U_{i})$, there exists $B_{r'}(x)$ where $y\in \mathbb{Q}^{n}$ and $r' \in \mathbb{Q}$ such that $B_{r'}(y) \subseteq \varphi_{i}(U\cap U_{i})$. $\operatorname{dist}(y, x) < r'$, there is a positive rational number $r$ such that $\operatorname{dist}(y, x) < r < r'$ then $x \in B_{r}(y) \subseteq B_{r'}(y) \subseteq \varphi_{i}(U\cap U_{i})$, this implies $\varphi_{i}^{-1}(B_{r}(y))$ is an element of $\mathscr{B}$. Hence $\mathscr{B}$ is a basis for $M$.

	Thus every manifold has a countable basis of regular coordinate balls.
\end{proof}

\begin{exercise}{4.62}
	Prove that every manifold with boundary has a countable basis consisting of regular coordinate balls and half-balls.
\end{exercise}

\begin{proof}
	This proof uses the invariance of boundary.

	In this proof we denote by $B_{r}(x)$ an entire open ball in $\mathbb{R}^{n}$.

	First, we prove a result similar to Lemma 4.59 as follows

	\textbf{Similar to Lemma 4.59.} Let $M$ be an $n$-manifold with boundary. $B'\subseteq M$ is a (coordinate half-ball) domain of any boundary chart and $\varphi: B' \to B_{r'}(x)\cap \mathbb{H}^{n}$ is a homeomorphism where $x \in \partial\mathbb{H}^{n}$, then $\varphi^{-1}(B_{r}(x) \cap \mathbb{H}^{n})$ is a regular coordinate half-ball whenever $0 < r < r'$.

	\textit{Proof for the above result.} For every $0 < r < r'$, $B_{r}(x) \cap \mathbb{H}^{n}$ is an open subset of $B_{r'}(x) \cap \mathbb{H}^{n}$, so $B = \varphi^{-1}(B_{r}(x) \cap \mathbb{H}^{n})$ is a coordinate half-ball. Regard $\varphi^{-1}$ as a map from $\bar{B}_{r}(0) \cap \mathbb{H}^{n}$ to $M$, then $\varphi^{-1}$ is a continuous and closed map (because $\bar{B}_{r}(x) \cap \mathbb{H}^{n}$ is compact due to Heine-Borel's theorem, and $M$ is Hausdorff). By Problem~\ref{problem:2-6}, $\varphi^{-1}(\bar{B}_{r}(x) \cap \mathbb{H}^{n}) = \overline{B}$, hence $\varphi(\overline{B}) = \bar{B}_{r}(x) \cap \mathbb{H}^{n}$. Hence $\varphi^{-1}(B_{r}(x) \cap \mathbb{H}^{n})$ is a regular coordinate half-ball whenever $0 < r < r'$. $\hfill\square$

	\textbf{A basis for $\mathbb{H}^{n}$.} The collection of the following open sets
	\begin{itemize}[itemsep=0pt]
		\item $B_{r}(x)$ where $x$ is a point of $\operatorname{Int}\mathbb{H}^{n}$ with rational coordinates, $r$ is rational, and $B_{r}(x) \subseteq \operatorname{Int}\mathbb{H}^{n}$
		\item $B_{r}(x) \cap \mathbb{H}^{n}$ where $x$ is a point of $\partial\mathbb{H}^{n}$ with rational coordinates, $r$ is rational
	\end{itemize}

	is a countable basis for the subspace topology on $\mathbb{H}^{n} \subseteq \mathbb{R}^{n}$.

		[An equivalence statement to the result we have just proved is given by replacing the point $x$ in $\partial\mathbb{H}^{n}$ by $0$, and this fits the definition of regular coordinate half-ball in the book. However, for convenience, we will use the definition given at the beginning of this proof.]

	Back to the proof for the main result.

	Let $M$ be an $n$-manifold with boundary. Every point of $M$ has an Euclidean neighborhood, so $M$ is covered by those Euclidean neighborhoods. Since $M$ is second countable then the open cover made of those Euclidean neighborhoods has a countable subcover. Let $\set{U_{i}}_{i\in\mathbb{N}}$ be an open cover of $M$ made of Euclidean neighborhoods, then for each $U_{i}$, there is a homeomorphism $\varphi_{i}$ from $U_{i}$ to an open subset $\widehat{U}_{i}$ of $\mathbb{R}^{n}$ or $\mathbb{H}^{n}$.

	For every $\widehat{U}_{i}$, consider the following open sets
	\begin{itemize}
		\item (Type 1) $B_{r}(x) \subseteq \widehat{U}_{i}$ where $x$ is a point of $\operatorname{Int}\mathbb{H}^{n}$ with rational coordinates, and $r$ is a positive rational number such that there are $r' > 0$ where $B_{r}(x) \subseteq B_{r'}(x) \subseteq \operatorname{Int}\mathbb{H}^{n}$.
		\item (Type 2) $B_{r}(x) \subseteq \widehat{U}_{i}$ where $x$ is a point of $\partial\mathbb{H}^{n}$ with rational coordinates, and $r$ is a positive rational number such that there are $r' > 0$ where $B_{r}(x) \cap \mathbb{H}^{n} \subseteq B_{r'}(x) \cap \mathbb{H}^{n} \subseteq \mathbb{H}^{n}$.
	\end{itemize}

	The open sets of type 1 are regular coordinate balls, ones of type 2 are regular coordinate half-balls. The collection of all these regular coordinate balls and half-balls (where $i$ varies in $\mathbb{N}$) is countable.

	Let $U$ be an open subset of $M$, then $U = \bigcup_{i\in\mathbb{N}}(U\cap U_{i})$. $U\cap U_{i}$ is homeomorphic to $\varphi_{i}(U\cap U_{i})$. If $x \in \varphi_{i}(U\cap U_{i})$ then $x$ is either an interior point or a boundary point.

	If $x$ is an interior point, then there exists $B_{r'}(y)$ where $y\in\mathbb{Q}^{n}$ and $r'\in\mathbb{Q}$ such that $x \in B_{r'}(y) \subseteq \varphi_{i}(U\cap U_{i})$. Since $\operatorname{dist}(y, x) < r'$, there is a rational number $r$ such that $\operatorname{dist}(y, x) < r < r'$. Therefore $x \in B_{r}(y) \subseteq B_{r'}(y) \subseteq \varphi_{i}(U\cap U_{i})$, which means $x$ is contained in an open set of type 1 in $\mathscr{B}$.

	If $x$ is a boundary point, then there exists $B_{r'}(y)$ where $y\in\mathbb{Q}^{n}$ and $r'\in\mathbb{Q}$ such that $x \in B_{r'}(y) \cap \mathbb{H}^{n} \subseteq \varphi_{i}(U\cap U_{i})$. Since $\operatorname{dist}(y, x) < r'$, there is a rational number $r$ such that $\operatorname{dist}(y, x) < r < r'$. Therefore $x \in B_{r}(y) \cap \mathbb{H}^{n} \subseteq B_{r'}(y) \cap \mathbb{H}^{n} \subseteq \varphi_{i}(U\cap U_{i})$, which means $x$ is contained in an open set of type 2 in $\mathscr{B}$.

	Hence $\mathscr{B}$ is a basis for $M$. Thus every manifold with boundary has a countable basis of regular coordinate balls and half-balls.
\end{proof}

\section*{Local Compactness}\addcontentsline{toc}{section}{Local Compactness}

\begin{lemma}{4.65}
	Let $X$ be a locally compact Hausdorff space. If $x\in X$ and $U$ is any neighborhood of $x$, there exists a precompact neighborhood $V$ of $x$ such that $\overline{V} \subseteq U$.
\end{lemma}

\begin{proof}
	Since $X$ is a locally compact Hausdorff space, $x$ has a precompact neighborhood $W$. $\overline{W}$ is compact and $\overline{W}\smallsetminus U = \overline{W} \cap (X\smallsetminus U)$ is a closed subset of $\overline{W}$, so $\overline{W}\smallsetminus U$ is compact.

	In a Hausdorff space, disjoint compact subsets have disjoint neighborhoods, so there exist disjoint open subsets $Y, Y'\subseteq X$ such that $\set{x} \subseteq Y$ and $\overline{W}\smallsetminus U \subseteq Y'$. Define $V = W\cap Y$ then $V \subseteq W$, which implies $\overline{V} \subseteq \overline{W}$. Therefore $\overline{V}$ is compact (because it is a closed subset of the compact set $\overline{W}$). We will show that $\overline{V}\subseteq U$.

	$Y, Y'$ are disjoint and $Y, Y'\subseteq X$ so $Y\cup Y' \subseteq X$. Hence $V = W\cap Y \subseteq Y \subseteq X\smallsetminus Y'$, it follows that $\overline{V} \subseteq \overline{Y} \subseteq \overline{X\smallsetminus Y'} = X\smallsetminus Y'$. Hence $\overline{V} \subseteq \overline{W} \cap (X\smallsetminus Y') = \overline{W} \smallsetminus Y' \subseteq U$, so $V$ is a precompact neighborhood of $x$ such that $\overline{V} \subseteq U$.
\end{proof}

\begin{exercise}{4.67}
	Show that any finite product of locally compact spaces is locally compact.
\end{exercise}

\begin{proof}
	Let $X, Y$ be locally compact spaces and $\tuple{x, y}$ be a point of $X\times Y$. Because $X, Y$ are locally compact, there exist $U_{x}, K_{x} \subseteq X$ and $U_{y}, K_{y} \subseteq Y$ such that $U_{x}, U_{y}$ are open, $K_{x}, K_{y}$ are compact, and $x\in U_{x} \subseteq K_{x}, y\in U_{y} \subseteq K_{y}$. So $\tuple{x, y} \in U_{x}\times U_{y} \subseteq K_{x} \times K_{y}$, where $U_{x}\times U_{y}$ is a product open set and $K_{x}\times K_{y}$ is compact because the product of finitely many compact spaces is compact. Hence $X\times Y$ is locally compact.

	It follows from mathematical induction that any finite product of locally compact spaces is locally compact.
\end{proof}

\begin{exercise}{4.70}
	Prove Proposition 4.69: In a Baire space, every meager subset has dense complement.
\end{exercise}

\begin{quote}
	A subset $F$ of a topological space $X$ is said to be \textbf{nowhere dense} if $\overline{F}$ has a dense complement, and $F$ is said to be \textbf{meager} if it can be expressed as a union of countably many nowhere dense subsets.
\end{quote}

\begin{proof}
	Let $X$ be a Baire space and $A\subseteq X$ is a meager subset. The meager set $A$ can be expressed as a union of countable many nowhere dense subsets $\set{U_{i}}_{i\in\mathbb{N}}$. Because $U_{i}$ is nowhere dense, $X\smallsetminus\overline{U_{i}}$ is dense for every $i\in\mathbb{N}$. By De Morgan's law and $U_{i}\subseteq \overline{U_{i}}$ for every $i\in \mathbb{N}$, we obtain
	\begin{equation*}
		X\smallsetminus A = X\smallsetminus \left(\bigcup_{i\in\mathbb{N}}U_{i}\right) = \bigcap_{i\in\mathbb{N}}(X\smallsetminus U_{i}) \supseteq \bigcap_{i\in\mathbb{N}}(X\smallsetminus \overline{U_{i}})
	\end{equation*}

	Because $X$ is a Baire space and $X\smallsetminus \overline{U_{i}}$ is a dense open subset for every $i\in\mathbb{N}$, the intersection $\bigcap_{i\in\mathbb{N}} (X\smallsetminus \overline{U_{i}})$ is dense, from which we deduce that $X\smallsetminus A$ (which is a superset of the intersection) is dense. Thus every meager subset has a dense complement.
\end{proof}

\begin{example}{4.71}
	The solution set of any polynomial equation in two variables is nowhere dense in $\mathbb{R}^{2}$.

	There are points in the plane that satisfy no rational polynomial equation.
\end{example}

\begin{proof}
	Let $f \in \mathbb{R}[x, y]$ be a nonzero polynomial in two variables. $\set{0}$ is closed in $\mathbb{R}$ and the solution set of $p$ is $f^{-1}(0)$, which is closed because $f$ is continuous. Assume that the interior (which is an open set) of $f^{-1}(0)$ is nonempty, then there is an open disk of $\mathbb{R}^{2}$ on which $f$ vanishes. This implies that $f$ is the zero polynomial, which is a contradiction. Hence the interior of $f^{-1}(0)$ is empty, so it is nowhere dense.

	There are countably many polynomials with rational coefficients, and the union of their solution sets (they are nowhere dense) is a meager subset of $\mathbb{R}^{2}$, so the complement of this meager subset is dense in $\mathbb{R}^{2}$, which implies the existence of points satisfying no rational polynomial equation.
\end{proof}

\section*{Paracompactness}\addcontentsline{toc}{section}{Paracompactness}

\begin{exercise}{4.73}
	Suppose $\mathscr{A}$ is an \textit{open} cover of $X$ such that each element of $\mathscr{A}$ intersects only finitely many others. Show that $\mathscr{A}$ is locally finite. Give a counterexample to show that this need not be true when the elements of $\mathscr{A}$ are not open.
\end{exercise}

\begin{proof}
	Because $\mathscr{A}$ is an open cover of $X$, for every element of $x$, there exists $A\in\mathscr{A}$ such that $x\in A$. Moreover, $A$ is a neighborhood of $x$ and $A$ intersects at most finitely many other elements of $\mathscr{A}$. Therefore $\mathscr{A}$ is locally finite.

	Here is a counterexample: Consider the $(n+1)$-space $\mathbb{R}^{n+1}$ where $n > 1$. Define $\mathscr{A}$ to be the set of 1-dimensional subspaces of $\mathbb{R}^{n+1}$ then each element of $\mathscr{A}$ is not open and $\mathscr{A}$ covers $\mathbb{R}^{n+1}$. However, every neighborhood of the origin intersects every element of $\mathscr{A}$ and $\mathscr{A}$ has infinite elements.
\end{proof}

\subsection*{Normal Spaces}\addcontentsline{toc}{subsection}{Normal Spaces}

\begin{exercise}{4.78}
	Show that every compact Hausdorff space is normal.
\end{exercise}

\begin{proof}
	Suppose $X$ is a compact Hausdorff space.

	Let $A, B$ be two disjoint closed subsets of $X$. Closed subsets of a compact space is compact, so $A, B$ are compact subsets of $X$. In a Hausdorff space, two disjoint compact sets are separated by open sets. Therefore $A, B$ are separated by some open subsets $U, V\subseteq X$. By definition, $X$ is normal.
\end{proof}

\begin{exercise}{4.79}
	Show that every closed subspace of a normal space is normal.
\end{exercise}

\begin{proof}
	Suppose $X$ is a normal space and $S$ is a closed subspace of $X$.

	$X$ is Hausdorff so its subspaces are Hausdorff. Therefore $S$ is Hausdorff. Let $A, B$ be disjoint closed subsets of $S$. Because $A, B$ are closed in $S$ (with the subspace topology) and $S$ is closed in $X$, it follows that $A, B$ are closed in $X$. $X$ is normal, so there are disjoint open sets $U, V\subseteq X$ such that $A\subseteq U$ and $B\subseteq V$. Furthermore, $S\cap U$ and $S\cap V$ are open subsets of $S$, $A$ and $B$ are separated by open subsets $S\cap U$ and $S\cap V$ of $S$. Hence $S$ is normal.

	Thus every closed subspace of a normal space is normal.
\end{proof}

\begin{lemma}{4.80}
	Let $X$ be a Hausdorff space. Then $X$ is normal if and only if it satisfies the following condition: whenever $A$ is a closed subset of $X$ and $U$ is a neighborhood of $A$, there exists a neighborhood $V$ of $A$ such that $\overline{V} \subseteq U$.
\end{lemma}

\begin{proof}
	$(\Longrightarrow)$ $X$ is normal.

	Let $A$ be a closed subset of $X$, $U$ be a neighborhood of $A$, and $B = X\smallsetminus U$. $A, B$ are disjoint closed subsets of $X$ so there are disjoint open subsets $V, W$ such that $A \subseteq V, B \subseteq W$. On the other hand $V \subseteq X\smallsetminus W$. $\overline{V}$ is the smallest closed set containing $V$, $X\smallsetminus W$ is a closed set containing $V$, so $\overline{V} \subseteq X\smallsetminus W \subseteq X\smallsetminus B = U$. Hence there is a neighborhood $V$ of $A$ such that $\overline{V} \subseteq U$.

	$(\Longleftarrow)$ Whenever $A$ is a closed subset of $X$ and $U$ is a neighborhood of $A$, there exists a neighborhood $V$ of $A$ such that $\overline{V} \subseteq U$.

	Let $A, B$ be disjoint closed subsets of $X$. $X\smallsetminus B$ is a neighborhood of $A$, so there is a neighborhood $V\supseteq A$ such that $\overline{V} \subseteq X\smallsetminus B$, then $X\smallsetminus \overline{V} \supseteq B$. Hence $A, B$ have disjoint neighborhoods $V$ and $X\smallsetminus \overline{V}$. Therefore $X$ is normal.
\end{proof}

\begin{theorem}{4.81}
	Every paracompact Hausdorff space is normal.
\end{theorem}

\begin{proof}
	Suppose $X$ is a paracompact Hausdorff space.

	\textbf{Prove that $X$ is regular.}

	Let $A$ be a closed subset of $X$ and $q\in X\smallsetminus A$, then $\set{q}$ is closed in $X$ (because every finite subset of a Hausdorff space is closed).

	For each $p\in A$, there are disjoint open subsets $U_{p}\ni p$ and $V_{p}\ni q$, because of the Hausdorffness of $X$. The collection $\set{U_{p}}_{p\in A} \cup \set{X\smallsetminus A}$ is an open cover of $X$. By paracompactness, this collection has a locally finite open refinement $\mathcal{W}$. Each open sets in $\mathcal{W}$ is contained either in $U_{p}$ for some $p\in A$, or in $X\smallsetminus A$. Let $\mathcal{U}$ be the collection of sets in $\mathcal{W}$ that are contained in some $U_{p}$, then $\mathcal{U}$ is also locally finite and it is an open cover of $A$.

	Let $\mathbb{U} = \bigcup_{U\in\mathcal{U}} U_{p}$ and $\mathbb{V} = X\smallsetminus \overline{\mathbb{U}}$. By Lemma 4.75, $\overline{\mathbb{U}} = \bigcup_{U\in \mathcal{U}} \overline{U}$, because $\mathcal{U}$ is locally finite. For every $U \in \mathcal{U}$, there is $V_{p}\ni q$ that is disjoint from $U$, so $q\notin \overline{U}$. Therefore $q\notin \overline{\mathbb{U}}$, which means $q\in \mathbb{V}$. The open sets $\mathbb{U}$ and $\mathbb{V}$ separate $A$ and $\set{q}$. Hence $X$ is regular.

	\textbf{Prove that $X$ is normal.}

	Let $A, B$ be disjoint closed subsets of $X$.

	For each $p\in A$, there are disjoint open subsets $U_{p}\ni p$ and $V_{p}\supseteq B$, because of the regularity of $X$. The collection $\set{U_{p}}_{p\in A} \cup \set{X\smallsetminus A}$ is an open cover of $X$. Because $X$ is paracompact, this collection has locally finite open refinement $\mathcal{W}$. Each open sets in $\mathcal{W}$ is contained either in $U_{p}$ for some $p\in A$, or in $X\smallsetminus A$. Let $\mathcal{U}$ be the collection of sets in $\mathcal{W}$ that are contained in some $U_{p}$, then $\mathcal{U}$ is also locally finite and it is an open cover of $A$.

	Let $\mathbb{U} = \bigcup_{U\in\mathcal{U}} U$ and $\mathbb{V} = X\smallsetminus \overline{\mathbb{U}}$. By Lemma 4.75, $\overline{\mathbb{U}} = \bigcup_{U\in \mathcal{U}} \overline{U}$, because $\mathcal{U}$ is locally finite. For every $U \in \mathcal{U}$, there is $V_{p}\supseteq B$ that is disjoint from $U$, so $B \subseteq X\smallsetminus \overline{U}$ (the exterior of $U$), it follows that
	\begin{equation*}
		B \subseteq \bigcap_{U\in\mathcal{U}} X\smallsetminus \overline{U} = X\smallsetminus \left(\bigcup_{U\in\mathcal{U}}\overline{U}\right) = X\smallsetminus \overline{\mathbb{U}} = \mathbb{V}.
	\end{equation*}

	Hence $A, B$ are separated by open sets $\mathbb{U}$ and $\mathbb{V}$. Therefore $X$ is normal.
\end{proof}

\subsection*{Partition of Unity}\addcontentsline{toc}{subsection}{Partition of Unity}

\begin{exercise}{4.87}
	Show that every compact manifold with boundary is homeomorphic to a subset of some Euclidean space. [Hint: use the double.]
\end{exercise}

\section*{Proper Maps}\addcontentsline{toc}{section}{Proper Maps}

\section*{Problems}

\begin{problem}{4-1}\label{problem:4-1}
Show that for $n > 1$, $\mathbb{R}^{n}$ is not homeomorphic to any open subset of $\mathbb{R}$.
\end{problem}

\begin{proof}
	Let $n$ be a positive integer greater than $1$. Assume that $\mathbb{R}^{n}$ is homeomorphic to an open subset $U\subseteq \mathbb{R}$, then $U$ is nonempty and there is a homeomorphism $\varphi: \mathbb{R}^{n} \to U$. Let $p$ be a point of $U$. Since $U\subseteq \mathbb{R}$ is open, $U$ is an union of disjoint open intervals, and $x$ lies in one of those open intervals, denote such open interval by $\openinterval{a, b}$, then $\openinterval{a, b}\smallsetminus\set{p}$ is disconnected. Therefore $U\smallsetminus\set{p}$ is disconnected. Because $\varphi$ is a homeomorphism, $\varphi^{-1}(U\smallsetminus\set{p}) = \mathbb{R}^{n} \smallsetminus \set{\varphi^{-1}(p)}$.

	We will show that $\mathbb{R}^{n}\smallsetminus\set{0}$ is path-connected. Let $x, y$ be two points of $\mathbb{R}^{n}\smallsetminus\set{0}$. Because $n > 1$, there exists a nonzero vector $v\in \mathbb{R}^{n}$ such that $x, y$ are not multiples of $v$. Let $v_{x} = \frac{\abs{x}}{\abs{v}}v$ and $v_{y} = \frac{\abs{y}}{\abs{v}}v$. We will construct
	\begin{itemize}
		\item a path in $\mathbb{R}^{n}\smallsetminus\set{0}$ from $x$ to $v_{x}$

		      Note that $\abs{x} = \abs{v_{x}}$. $x = (x_{1}, \ldots, x_{n})$ and $v_{x} = (v_{x,1}, \ldots, v_{x,n})$.

		      There exist $\varphi_{x,1}, \ldots, \varphi_{x,n-1} \in \mathbb{R}$ such that
		      \begin{align*}
			      x_{1}   & = \abs{x}\cos(\varphi_{x,1})                                                 \\
			      x_{2}   & = \abs{x}\sin(\varphi_{x,1})\cos(\varphi_{x,2})                              \\
			      \cdots  &                                                                              \\
			      x_{n-1} & = \abs{x}\sin(\varphi_{x,1})\cdots\sin(\varphi_{x,n-2})\cos(\varphi_{x,n-1}) \\
			      x_{n}   & = \abs{x}\sin(\varphi_{x,1})\cdots\sin(\varphi_{x,n-2})\sin(\varphi_{x,n-1})
		      \end{align*}

		      Also there exist $\varphi_{v_{x},1}, \ldots, \varphi_{v_{x},n-1} \in \mathbb{R}$ such that
		      \begin{align*}
			      v_{x,1}   & = \abs{x}\cos(\varphi_{v_{x},1})                                                         \\
			      v_{x,2}   & = \abs{x}\sin(\varphi_{v_{x},1})\cos(\varphi_{v_{x},2})                                  \\
			      \cdots    &                                                                                          \\
			      v_{x,n-1} & = \abs{x}\sin(\varphi_{v_{x},1})\cdots\sin(\varphi_{v_{x},n-2})\cos(\varphi_{v_{x},n-1}) \\
			      v_{x,n}   & = \abs{x}\sin(\varphi_{v_{x},1})\cdots\sin(\varphi_{v_{x},n-2})\sin(\varphi_{v_{x},n-1})
		      \end{align*}

		      The maps $f_{i}: \closedinterval{0, 1} \to \mathbb{R}$ given by
		      \begin{equation*}
			      f_{i}(t) = \abs{x}\sin((1-t)\varphi_{x,1} + t\varphi_{v_{x},1})\cdots \sin((1-t)\varphi_{x,i-1} + t\varphi_{v_{x},i-1})\cos((1-t)\varphi_{x,i} + t\varphi_{v_{x},i})
		      \end{equation*}

		      if $i < n$ and
		      \begin{equation*}
			      f_{n}(t) = \abs{x}\sin((1-t)\varphi_{x,1} + t\varphi_{v_{x},1})\cdots \sin((1-t)\varphi_{x,n-1} + t\varphi_{v_{x},n-1})
		      \end{equation*}

		      are continuous. So ${(f_{1}(t))}^{2} + \cdots + {(f_{n}(t))}^{2} = \abs{x}^{2} \ne 0$ for every $t\in \closedinterval{0,1}$. Hence the map $f_{x}: \closedinterval{0, 1} \to \mathbb{R}^{n}\smallsetminus\set{0}$ given by
		      \begin{equation*}
			      f_{x}(t) = (f_{1}(t), \ldots, f_{n}(t))
		      \end{equation*}

		      is continuous, due to the characteristic property of product topology. Hence $f$ is a path in $\mathbb{R}^{n}\smallsetminus\set{0}$ from $x$ to $v_{x}$.
		\item a path in $\mathbb{R}^{n}\smallsetminus\set{0}$ from $v_{x}$ to $v_{y}$

		      The map $f: \closedinterval{0, 1} \to \mathbb{R}^{n}\smallsetminus\set{0}$ given by
		      \begin{equation*}
			      f(t) = (1 - t)v_{x} + tv_{y}
		      \end{equation*}

		      is continuous (this map is well-defined because the line segment connecting $v_{x}$ and $v_{y}$ lies entirely in $\mathbb{R}^{n}\smallsetminus\set{0}$) so there is a path in $\mathbb{R}^{n}$ from $v_{x}$ to $v_{y}$.
		\item a path in $\mathbb{R}^{n}\smallsetminus\set{0}$ from $v_{y}$ to $y$

		      Similar to the first contruction, we can construct a path $f_{y}$ in $\mathbb{R}^{n}\smallsetminus\set{0}$ from $v_{y}$ to $y$.
	\end{itemize}

	From these constructions, we deduce that there are continuous maps $g_{x}: \closedinterval{0, \frac{1}{3}} \to \mathbb{R}^{n}\smallsetminus\set{0}$ such that $g_{x}(0) = x$ and $g_{x}(1/3) = v_{x}$, $g: \closedinterval{\frac{1}{3}, \frac{2}{3}} \to \mathbb{R}^{n}\smallsetminus\set{0}$ such that $g(1/3) = v_{x}$ and $g(2/3) = v_{y}$, $g_{y}: \closedinterval{\frac{2}{3}, 1} \to \mathbb{R}^{n}\smallsetminus\set{0}$ such that $g_{y}(2/3) = v_{y}$ and $g_{y}(1) = y$. By the gluing lemma, there is a unique continuous map $f: \closedinterval{0, 1} \to \mathbb{R}^{n}\setminus \set{0}$ such that $f\vert_{\closedinterval{0, \frac{1}{3}}} = g_{x}$, $f\vert_{\closedinterval{\frac{1}{3}, \frac{2}{3}}} = g$, $f\vert_{\closedinterval{\frac{2}{3}, 1}} = g_{y}$. Hence there is a path in $\mathbb{R}^{n}\smallsetminus\set{0}$ from $x$ to $y$.

	Back to the set $\mathbb{R}^{n}\smallsetminus\set{\varphi^{-1}(p)}$. For every $c, d \in \mathbb{R}^{n}\smallsetminus\set{\varphi^{-1}(p)}$, there is a path in $\mathbb{R}^{n}\smallsetminus\set{0}$ from $c - \varphi^{-1}(p)$ to $d - \varphi^{-1}(p)$, so there is a path in $\mathbb{R}^{n}\smallsetminus\set{\varphi^{-1}(p)}$ from $c$ to $d$. Therefore $\mathbb{R}^{n}\smallsetminus\set{\varphi^{-1}(p)}$ is path-connected. Since $\varphi$ is a homeomorphism
	\begin{equation*}
		U\smallsetminus\set{p} = \varphi(\varphi^{-1}(U\smallsetminus\set{p})) = \varphi(\mathbb{R}^{n}\smallsetminus\set{\varphi^{-1}(p)})
	\end{equation*}

	is also path-connected, which is a contradiction because $U\smallsetminus\set{p}$ is disconnected.

	Thus for $n > 1$, $\mathbb{R}^{n}$ is not homeomorphic to any open subset $U\subseteq \mathbb{R}$.
\end{proof}

\begin{problem}{4-2}\label{problem:4-2}
\textsc{Invariance of Dimension, 1-Dimensional Case:} Prove that a nonempty topological space cannot be both a 1-manifold and an $n$-manifold for some $n > 1$.
\end{problem}

\begin{proof}
	Let $M$ be a nonempty $n$-manifold where $n > 1$. Assume that $M$ is also a 1-manifold. Let $x$ be a point of $M$. Because $M$ is an $n$-manifold and a 1-manifold, $x$ has a neighborhood $U$ which admits a homeomorphism $\varphi: U\to \mathbb{R}^{n}$ and a neighborhood $V$ which admits a homeomorphism $\psi: V\to \mathbb{R}$. $U\cap V$ is nonempty because it contains $x$ and it is open. The restrictions $\varphi\vert_{U\cap V}: U\cap V \to \varphi(U\cap V)$ and $\psi\vert_{U\cap V}: U\cap V \to \psi(U\cap V)$ are also a homemorphisms. Since $\varphi(U\cap V)$ is open (because a homemorphism is an open map), there is an open $n$-ball $B^{n}_{r}(\varphi(x)) \subseteq \varphi(U\cap V)$. Denote by $W$ the preimage $\varphi^{-1}(B^{n}_{r}(\varphi(x)))$ then $W\subseteq U\cap V$. The restrictions $\varphi\vert_{W}: W \to \varphi(W) = B^{n}_{r}(\varphi(x))$ and $\psi\vert_{W}: W \to \psi(W)$ are also homeomophism, so the open $n$-ball $B^{n}_{r}(\varphi(x))$ and $\psi(W)$ are homeomorphic. On the other hand, every open $n$-ball is homeomorphic to $\mathbb{R}^{n}$ and $\mathbb{R}^{n}$ is not homeomorphic to $\psi(W)$ (which is an open subset of $\mathbb{R}$) according to Problem~\ref{problem:4-1}, hence the contradiction. Thus a nonempty topological space cannot be both a 1-manifold and an $n$-manifold for some $n > 1$.
\end{proof}

\begin{problem}{4-3}
\textsc{Invariance of the Boundary, 1-Dimensional Case:} Suppose $M$ is a 1-dimensional manifold with boundary. Show that a point of $M$ cannot be both a boundary point and an interior point.
\end{problem}

\begin{proof}
	Firstly, we prove that $\mathbb{R}$ and $\mathbb{H}$ are not homeomorphic. Assume that there is a homeomorphism $f: \mathbb{H} \to \mathbb{R}$. $\mathbb{H}\smallsetminus\set{0}$ is path-connected, however, $\mathbb{R}\smallsetminus\set{f(0)}$ is not path-connected, which is a contradiction because a homeomorphism preserves path-connectedness. Hence $\mathbb{R}$ and $\mathbb{H}$ are not homeomorphic.

	Assume that $M$ has a point $x$ which is both a boundary point and an interior point. Because $x$ is a boundary point, $x$ is in the domain of a boundary chart $(U, \varphi)$ where $\varphi(x) \in \partial\mathbb{H}$ (which implies $\varphi(x) = 0$). Because $x$ is an interior point, $x$ is in the domain of an interior chart $(V, \psi)$. $U\cap V$ is a neighborhood of $x$ and the restrictions $\varphi\vert_{U\cap V}: U\cap V \to \varphi(U\cap V)$, $\psi\vert_{U\cap V}: U\cap V \to \psi(U\cap V)$. Since $\varphi(U\cap V)$ is open and $\varphi(x) = 0$, there exists $r > 0$ such that $\halfopenright{0, r} \subseteq \varphi(U\cap V)$.

	Denote $W = \varphi^{-1}(\halfopenright{0, r})$ then $W \subseteq U\cap V$. It follows that $\halfopenright{0, r}$ and $W$ are homeomorphic, $W$ and $\psi(W)$ are homeomorphic. On the other hand, $W$ is open (because $\halfopenright{0, a} \subseteq \mathbb{H}$ is open) and $\varphi$ is continuous, so $\psi(W) \subseteq \mathbb{R}$ is open (because $\psi$ is a homeomorphism, hence an open map), so $\halfopenright{0, r} \subseteq \mathbb{H}$ is homeomorphic to an open subset $A\subseteq \mathbb{R}$.

	Since $\halfopenright{0, r}$ is connected, $A$ is also connected. $A$ is a connected and open subset of $\mathbb{R}$ so $A$ is an open interval. $\halfopenright{0, r}$ is homeomorphic to $\mathbb{H}$, an open interval is homeomorphic to $\mathbb{R}$, hence $\mathbb{R}$ and $\mathbb{H}$ are homeomorphic, which is a contradiction.

	Thus a point of a 1-dimensional manifold with boundary cannot be both a boundary point and an interior point.
\end{proof}

\begin{problem}{4-4}
Show that the following topological spaces are not manifolds
\begin{enumerate}[label={(\alph*)}]
	\item the union of the $x$-axis and the $y$-axis in $\mathbb{R}^{2}$
	\item the conical surface $C\subseteq \mathbb{R}^{3}$ defined by
	      \begin{equation*}
		      C = \set{(x,y,z) : z^{2} = x^{2} + y^{2}}
	      \end{equation*}
\end{enumerate}
\end{problem}

\begin{proof}
	\begin{enumerate}[label={(\alph*)}]
		\item Assume that $M = (\mathbb{R}\times\set{0}) \cup (\set{0}\times\mathbb{R}) \subseteq \mathbb{R}^{2}$ is an $n$-manifold for some positive integer $n$.

		      Let $x$ be a nonzero real number. From the definition of manifold, $x$ has a neighborhood $U$ which is homeomorphic to an open subset of $\mathbb{R}^{n}$. On the other hand, a basis for the topology on $M$ is obtained by taking the intersection of $M$ and open balls in $\mathbb{R}^{2}$, so there is an open ball $B_{r}(\tuple{x,0})$ such that $\tuple{x,0} \in B_{r}(\tuple{x,0}) \cap M \subseteq U$. Let $\varepsilon$ be a positive number such that $\varepsilon < \min\set{r, \abs{x}}$ then
		      \begin{equation*}
			      \tuple{x, 0} \in \openinterval{x - \varepsilon, x + \varepsilon} \times \set{0} \subseteq B_{r}(\tuple{x,0}) \cap M \subseteq U.
		      \end{equation*}

		      $\openinterval{x - \varepsilon, x + \varepsilon} \times \set{0}$ is a neighborhood of $(x, 0)$ in $M$ and it is homeomorphic to an open subset of $\mathbb{R}$ and an open subset of $\mathbb{R}^{n}$. From Problem~\ref{problem:4-2}, we deduce that $n = 1$.

		      Because $M$ is a 1-manifold, $\tuple{0,0}$ has a neighborhood $V$ which admits a homeomorphism $\varphi: V \to \mathbb{R}$. There is an open ball $B_{r}(\tuple{0,0})$ in $\mathbb{R}^{2}$ such that
		      \begin{equation*}
			      \tuple{0, 0} \in B_{r}(\tuple{0, 0}) \cap M \subseteq M.
		      \end{equation*}

		      Denote $B_{r}(\tuple{0, 0}) \cap M$ by $W$ then the restriction $\varphi\vert_{W}: W \to \varphi(W)$ is also a homeomorphism. $W$ is path-connected because it is the union of path-connected sets with a point in common (the origin)
		      \begin{equation*}
			      W = (\openinterval{-r, r} \times\set{0}) \cup (\set{0} \times \openinterval{-r, r}).
		      \end{equation*}

		      so $\varphi(W) \subseteq \mathbb{R}$ is path-connected, hence it is an interval. $W\smallsetminus\set{0}$ has four path components, namely
		      \begin{equation*}
			      \openinterval{0, r}\times\set{0};\quad \openinterval{-r, 0}\times\set{0};\quad \set{0}\times\openinterval{0, r};\quad \set{0}\times\openinterval{-r, 0}
		      \end{equation*}

		      but $\varphi(W\smallsetminus\set{0}) \subseteq \mathbb{R}$ has two path components (because it is an open interval minus a point), which is a contradiction.

		      Thus $M$ is not a manifold.
		\item Assume that $C$ is an $n$-manifold for some positive integer $n$.

		      Every point on $C$ other than $(0, 0, 0)$ is of the form $(r\cos\theta, r\sin\theta, r)$ for some $r\ne 0$. Consider a point $(r\cos\theta, r\sin\theta, r)$ where $r > 0$. The set $H = \set{ \tuple{x, y, z} \in \mathbb{R}^{3} : z > 0 }$ is open in $\mathbb{R}^{3}$, so $H \cap C$ is open in $C$ (using the subspace topology). In fact, $H\cap C$ is the set of points on $C$ with positive $z$-ordinate. The map $f: H\cap C \to \mathbb{R}^{2}$ given by $f(x, y, z) = (x, y)$ is a homeomorphism, so $(r\cos\theta, r\sin\theta, r)$ has a neighborhood that is homeomorphic to $\mathbb{R}^{2}$. From Problem~\ref{problem:4-2} we deduce that $n > 1$.

		      $\tuple{0, 0, 0}$ has a neighborhood $U \subseteq C$ which is homeomorphic to $\mathbb{R}^{n}$ with the coordinate map $\varphi$. There exists $r > 0$ such that
		      \begin{equation*}
			      \tuple{0,0,0} \in B_{r}(\tuple{0,0,0}) \cap C \subseteq U \cap C
		      \end{equation*}

		      because the set of open balls is a basis for the Euclidean topology on $\mathbb{R}^{3}$ (and taking the intersection with a subset of $\mathbb{R}^{3}$, we obtain a basis for the subspace topology).

		      Denote $B_{r}(\tuple{0,0,0}) \cap C$ by $V$. $V$ is connected. $V\smallsetminus\set{\tuple{0,0,0}}$ is disconnected since it has two components, namely
		      \begin{equation*}
			      \begin{split}
				      \set{\tuple{(x, y, z)} \in \mathbb{R}^{3} : 0 < z < r} \cap C \\
				      \set{\tuple{(x, y, z)} \in \mathbb{R}^{3} : -r < z < 0} \cap C
			      \end{split}
		      \end{equation*}

		      and $\varphi(V \smallsetminus \set{\tuple{0,0,0}})$ is therefore disconnected. However, because $\varphi(V)$ is connected, $\varphi(V \smallsetminus \set{\tuple{0,0,0}}) = \varphi(V) \smallsetminus\set{\varphi(\tuple{0,0,0})}$. A connected open subset in $\mathbb{R}^{n}$ (where $n > 1$) minus a point is still connected, hence a contradiction.

		      Thus $C$ is not a manifold.
	\end{enumerate}
\end{proof}

\begin{problem}{4-5}
Let $M = \mathbb{S}^{1}\times\mathbb{R}$, and let $A = \mathbb{S}^{1} \times \set{0}$. Show that the space $M/A$ obtained by collapsing $A$ to a point is homeomorphic to the space $C$ of Problem 4-4 (b), and thus is Hausdorff and second countable but not locally Euclidean.
\end{problem}

\begin{proof}[Unrigorous Proof]
	The map $f: \mathbb{S}^{1}\times\mathbb{R} \to C$ given by
	\begin{equation*}
		f(e^{\iota\theta}, r) = (r\cos\theta, r\sin\theta, r)
	\end{equation*}

	is a quotient map (I don't have a rigorous proof for this yet). Moreover, the quotient map $q: M\to M/A$ and $f$ have the same identification, hence $M/A$ and $C$ are homeomorphism, according to the uniqueness of quotient space. Thus $C$ is Hausdorff, second countable but not locally Euclidean.
\end{proof}

\begin{problem}{4-6}
Like Problem~\ref{problem:2-22}, this problem constructs a space that is locally Euclidean and Hausdorff but not second countable. Unlike that example, however, this one is connected.
\begin{enumerate}[label={(\alph*)}]
	\item Recall that a totally ordered set is said to be well ordered if every nonempty subset has a smallest element. Show that the well-ordering theorem implies that there exists an uncountable well-ordered set $Y$ such that for every $y_{0}\in Y$, there are only countably many $y\in Y$ such that $y < y_{0}$.
	\item Now let $\mathscr{R} = Y \times \halfopenright{0,1}$, with the \textbf{dictionary order}: this means that $(y_{1}, s_{1}) < (y_{2}, s_{2})$ if either $y_{1} < y_{2}$, or $y_{1} = y_{2}$ and $s_{1} < s_{2}$. With the order topology, $\mathscr{R}$ is called the \textbf{long ray}. The \textbf{long line} $\mathscr{L}$ is the wedge sum $\mathscr{R}\vee \mathscr{R}$ obtained by identifying both copies of $(y_{0}, 0)$ with each other, where $y_{0}$ is the smallest element in $Y$. Show that $\mathscr{L}$ is locally Euclidean, Hausdorff, and first countable, but not second countable.
	\item Show that $\mathscr{L}$ is path-connected.
\end{enumerate}
\end{problem}

\begin{proof}
	Unsolved.
	\begin{enumerate}[label={(\alph*)}]
		\item This assumes the well-ordering theorem.

		      Let $X$ be an uncountable well-ordered set. If for every element of $X$, there is only countably many elements strictly less than it, then we are done. If this is not the case, the subset $X_{0}\subseteq X$ of elements $x$ such that there are uncountably many elements strictly less than $x$, is nonempty. Due to the well-ordering theorem, $X_{0}$ has a smallest element $x_{0}$. Let $Y$ be the subset of $X$ containing element strictly less than $x_{0}$ then $Y$ is uncountable and has the desired property (otherwise, it contradicts the minimality of $x_{0}$).
		\item
		\item
	\end{enumerate}
\end{proof}

\begin{note}[Characterizations of Local Connectedness]\label{note:characterizations-of-local-connectedness}
	Let $X$ be a topological space. The following statements are equivalent
	\begin{enumerate}[label={(\alph*)}]
		\item $X$ is locally connected.
		\item Every open subset of $X$ is locally connected.
		\item All components of every open subset of $X$ are open in $X$.
		\item For every $x\in X$, every neighborhood of $x$ contains a connected open set containing $x$.
		\item Every point of $X$ has a connected neighborhood basis (local basis).
	\end{enumerate}
\end{note}

\begin{proof}
	$(a)\implies (b)$ $X$ is locally connected then $X$ has a basis $\mathscr{B}$ of connected open sets. Let $U$ be an open subset of $X$ and $\mathscr{B}_{U} = \set{ B\in \mathscr{B} : B\subseteq U }$. If $x\in U$ and $V$ is a neighborhood of $x$ in $U$, then $V$ is also open in $X$, so there is $B\in\mathscr{B}$ such that $x\in B\subseteq V\subseteq U$, which implies $B\in\mathscr{B}_{U}$. Hence $\mathscr{B}_{U}$ is a basis for the subspace topology on $U$, which consists of connected open subsets of $U$. Therefore every open subset of $X$ is locally connected.

	$(b)\implies (a)$ $X$ is an open subset of $X$ so $X$ is locally connected.

	$(a)\implies (c)$ Let $U$ be an open subset of $X$, then $U$ is locally connected ($(a) \implies (b)$). Because $U$ is locally connected, every component of $U$ is open in $U$. Therefore every component of $U$ is open in $X$.

	$(c)\implies (a)$ Let $x$ be a point of $X$. For every neighborhood $U$ of $x$, all components of $U$ are open in $X$, let $C_{U,x}$ be the component of $U$ containing $x$. Therefore the collection of $C_{U,x}$ is a connected neighborhood basis of $X$, so $X$ is locally connected.

	$(a)\implies (d)$ Let $x$ be a point of $X$ and $U$ be a neighborhood of $x$. Since $X$ has a basis $\mathscr{B}$ consisting of connected open sets, there exists $B\in\mathscr{B}$ such that $x \in B\subseteq U$. Hence every neighborhood of $x$ contains a connected open set containing $x$.

	$(d)\implies (c)$ Let $U$ be an open subset of $X$ and $C$ be a component of $U$. Let $x$ be a point of $C$ then there is a connected open set $V$ such that $x\in V\subseteq U$. Since $x\in V, x\in C$ and $V, C$ are connected subsets of $U$, $V\cup C$ is a connected subset of $U$. From the maximality of $C$, it follows that $V\subseteq C$, which means there is a neighborhood of $x$ contained in $C$. This is true for every $x\in C$ so $C$ is open in $U$, hence open in $X$ (because $U\subseteq X$ is open). Hence all components of every open subset of $X$ are open in $X$.

	$(a)\implies (e)$ Let $\mathscr{B}$ be a basis of $X$ consisting connected open sets. Let $x$ be a point of $X$ and $\mathscr{B}_{x} = \set{ B\in\mathscr{B}: x\in B }$. For every neighborhood $U$ of $x$, there exists $B\in\mathscr{B}$ such that $x\in B\subseteq U$, which implies $B\in \mathscr{B}_{x}$. Hence $\mathscr{B}_{x}$ is a connected neighborhood basis of $x$.

	$(e)\implies (a)$ Since every point of $X$ has a connected neighborhood basis, the union of these neighborhood bases gives a basis for the topology on $X$, which contains connected open sets.
\end{proof}

\begin{note}[Characterizations of Local Path-Connectedness]\label{note:characterizations-of-local-path-connectedness}
	Let $X$ be a topological space. The following statements are equivalent
	\begin{enumerate}[label={(\alph*)}]
		\item $X$ is locally path-connected.
		\item Every open subset of $X$ is locally path-connected.
		\item All path components of every open subset of $X$ are open in $X$.
		\item For every $x\in X$, every neighborhood of $x$ contains a path-connected open set containing $x$.
		\item Every point of $X$ has a path-connected neighborhood basis (local basis).
	\end{enumerate}
\end{note}

\begin{proof}
	$(a)\implies (b)$ $X$ is locally path-connected then $X$ has a basis $\mathscr{B}$ of path-connected open sets. Let $U$ be an open subset of $X$ and $\mathscr{B}_{U} = \set{ B\in \mathscr{B} : B\subseteq U }$. If $x\in U$ and $V$ is a neighborhood of $x$ in $U$, then $V$ is also open in $X$, so there is $B\in\mathscr{B}$ such that $x\in B\subseteq V\subseteq U$, which implies $B\in\mathscr{B}_{U}$. Hence $\mathscr{B}_{U}$ is a basis for the subspace topology on $U$, which consists of path-connected open subsets of $U$. Therefore every open subset of $X$ is locally path-connected.

	$(b)\implies (a)$ $X$ is an open subset of $X$ so $X$ is locally connected.

	$(a)\implies (c)$ $X$ is locally path-connected so there is a basis $\mathscr{B}$ for the topology on $X$ consisting of path-connected open sets. Let $U$ be an open subset of $X$ and $C$ be a path component of $U$. For every $x\in C$, there exists $B\in\mathscr{B}$ such that $x\in B \subseteq U$. Since $C$ and $B$ have the point $x$ in common and they are path-connected, $C\cup B$ is a path-connected subset of $U$. Because of the maximality of $C$, $C\cup B = C$, so $B\subseteq C$. Hence every point of $C$ has a path-connected neighborhood contained in $C$, so $C$ is open in $X$. Therefore all path components of every open subset of $X$ are open in $X$.

	$(c)\implies (a)$ Let $x$ be a point of $X$. For every neighborhood $U$ of $x$, all path components of $U$ are open in $X$, let $P_{U,x}$ be the path component of $U$ containing $x$. The collection of $P_{U,x}$ ($x$ varies on $X$, every neighborhood $U$ of $x$) is then a basis for $X$, which consists of path-connected open sets of $X$. Hence $X$ is locally path-connected.

	$(a)\implies (d)$ Let $x$ be a point of $X$ and $U$ be a neighborhood of $x$. Since $X$ has a basis $\mathscr{B}$ consisting of path-connected open sets, there exists $B\in\mathscr{B}$ such that $x \in B\subseteq U$. Hence every neighborhood of $x$ contains a path-connected open set containing $x$.

	$(d)\implies (c)$ Let $U$ be an open subset of $X$ and $P$ be a path component of $U$. Let $x$ be a point of $P$, then there is a path-connected open set $V$ such that $x\in V\subseteq U$. Since $x\in V, x\in P$ and $V, P$ are connected subsets of $U$, $V\cup P$ is a connected subset of $U$. From the maximality of $P$, it follows that $V\subseteq P$, which means there is a neighborhood of $x$ contained in $P$. This is true for every $x\in P$ so $P$ is open in $U$, hence open in $X$ (because $U\subseteq X$ is open). Hence all path components of every open subset of $X$ are open in $X$.

	$(a)\implies (e)$ Let $\mathscr{B}$ be a basis of $X$ consisting connected open sets. Let $x$ be a point of $X$ and $\mathscr{B}_{x} = \set{ B\in\mathscr{B}: x\in B }$. For every neighborhood $U$ of $x$, there exists $B\in\mathscr{B}$ such that $x\in B\subseteq U$, which implies $B\in \mathscr{B}_{x}$. Hence $\mathscr{B}_{x}$ is a path-connected neighborhood basis of $x$.

	$(e)\implies (a)$ Since every point of $X$ has a path-connected neighborhood basis, the union of these neighborhood bases gives a basis for the topology on $X$, which contains path-connected open sets.
\end{proof}

\begin{note}[Product of Local Connected and Local Path-Connected Spaces]
	\begin{enumerate}[label={(\alph*)}]
		\item The product of finitely many local connected spaces is locally connected.
		\item The product of finitely many local path-connected spaces is locally path-connected.
	\end{enumerate}
\end{note}

\begin{proof}
	\begin{enumerate}[label={(\alph*)}]
		\item Suppose $X, Y$ are locally connected spaces.

		      Let $\mathscr{B}_{X}$ be a basis for the topology on $X$ consisting of connected open sets of $X$, $\mathscr{B}_{Y}$ be a basis for the topology on $Y$ consisting of connected open sets of $Y$.

		      For every $U\in \mathscr{B}_{X}, V\in \mathscr{B}_{Y}$, $U\times V$ is connected (because the product of two connected spaces is connected). Hence $\mathscr{B} = \set{ U\times V : U\in \mathscr{B}_{X}, V\in \mathscr{B}_{Y} }$ is a basis for the product topology on $X\times Y$ consisting of connected product open sets. Therefore $X\times Y$ is locally connected.

		      By mathematical induction, we conclude that the product of finitely many connected spaces is locally connected.
		\item The proof for local path-connectedness is similar.
	\end{enumerate}
\end{proof}

\begin{problem}{4-7}\label{problem:4-7}
Let $q: X\to Y$ be a quotient map. Show that
\begin{itemize}
	\item if $X$ is locally connected then $Y$ is locally connected,
	\item if $X$ is locally path-connected then $Y$ is locally path-connected,
	\item if $q$ is open and $X$ is locally compact, then $Y$ is locally compact.
\end{itemize}
\end{problem}

\begin{proof}
	A continuous map maps connected sets to connected sets, path-connected sets to path-connected sets, and compact sets to compact sets.

	\textbf{$X$ is locally connected.}

	Let $V$ be an open subset of $Y$, $y$ be a point of $V$, and $C_{y}$ be the component of $V$ containing $y$. We will show that $C_{y}$ is open in $Y$.

	Let $x$ be a point of $q^{-1}(C_{y})$. $q(x) \in C_{y} \subseteq V$ so $x \in q^{-1}(V)$, which is open because $q$ is continuous. Because $X$ is locally connected, $x$ has a connected neighborhood $U_{x}$ such that $x\in U_{x} \subseteq q^{-1}(V)$. Since continuous maps map connected sets to connected sets, $q(U_{x})$ is connected. Connected sets $q(U_{x})$ and $C_{y}$ have the point $q(x)$ is common so $q(U_{x}) \cup C_{y}$ is connected. On the other hand, $q(U_{x})$ and $C_{y}$ are subsets of $V$, and $C_{y}$ is a maximal connected set of $V$, so $q(U_{x}) \cup C_{y} = C_{y}$. Therefore $q(U_{x}) \subseteq C_{y}$, so $U_{x} \subseteq q^{-1}(q(U_{x})) \subseteq q^{-1}(C_{y})$, which means $U_{x}$ is a neighborhood of $x$ contained in $q^{-1}(C_{y})$.

	Because of the arbitrariness of $x$, we deduce that $q^{-1}(C_{y})$ is open. According to the definition of quotient map, $C_{y}$ is open, hence all components of every open subset of $Y$ is open in $Y$. From Note~\ref{note:characterizations-of-local-connectedness} (c), we conclude that $Y$ is locally connected.

	\textbf{$X$ is locally path-connected.}

	Let $V$ be an open subset of $Y$, $y$ be a point of $V$, and $P_{y}$ be the path component of $V$ containing $y$. We will show that $P_{y}$ is open in $Y$ and use the characterization in Note~\ref{note:characterizations-of-local-path-connectedness} (c).

	Let $x$ be a point of $q^{-1}(P_{y})$. $q(x) \in P_{y} \subseteq V$ so $x \in q^{-1}(V)$. The preimage $q^{-1}(V)$ is open because $q$ is continuous. Since $X$ is locally path-connected and $q^{-1}(V)$ is a neighborhood of $x$, there is a path-connected open set $U_{x}$ such that $x \in U_{x} \subseteq q^{-1}(V)$. Continuous maps map path-connected sets to path-connected sets so $q(U_{x})$ is path-connected. $q(U_{x})$ and $P_{y}$ are path-connected and have the point $q(x)$ in common so $q(U_{x}) \cup P_{y}$ is path-connected. Because of the maximality of $P_{y}$ (it is a path component), $q(U_{x}) \subseteq P_{y}$. Therefore $x \in U_{x} \subseteq q^{-1}(q(U_{x})) \subseteq q^{-1}(P_{y})$, which means $U_{x}$ is a neighborhood of $x$ contained in $q^{-1}(P_{y})$.

	Together with the arbitrariness of $x$, we deduce that $q^{-1}(P_{y})$ is open, so $P_{y}$ is open, since $q$ is a quotient map. Hence all path components of every open subset of $Y$ is open in $Y$, this implies $Y$ is locally path-connected.

	\textbf{$q$ is open and $X$ is locally compact.}

	Let $y$ be an element of $Y$ then there exists $x\in X$ such that $q(x) = y$. Since $X$ is locally compact, there is an open subset $U$ and a compact subset $K$ of $X$ such that $x \in U \subseteq K$. Because $q$ is continuous, $q(K)$ is a compact subset of $Y$. Because $q$ is an open map, $q(U)$ is an open subset of $Y$. Therefore $y = q(x) \in q(U) \subseteq q(K)$. Together with the arbitrariness of $y$, we conclude that $Y$ is locally compact.
\end{proof}

\begin{problem}{4-8}\label{problem:4-8}
Show that a locally connected topological space is homeomorphic to the disjoint union of its components.
\end{problem}

\begin{proof}
	Let $X$ be a locally connected topological space and $\mathscr{C}$ be the collection of its components. Denote by $Y$ the space $\coprod_{C\in\mathscr{C}}C$ with the disjoint union topology.

	Consider the map $f: X\to Y$ where $f(x)$ is defined to be $\iota_{C}(x)$ where $C$ is the component containing $x$. $f$ is well-defined because the components of a topological space give a partition of the given space. $f$ is bijective by definition.

	Let $U$ be an open subset of $X$, then $U = \bigcup_{C\in\mathscr{C}}(U\cap C)$ and $\iota_{C}(U\cap C)$ is open in $C$. Therefore $f(U)$ is open in $Y$. Hence $f$ is an open map.

	Let $V$ be an open subset of $Y$, then $V\cap C \subseteq C$ is open for every $C\in\mathscr{C}$. The preimage of $V$ is
	\begin{equation*}
		f^{-1}(V) = f^{-1}\left( \coprod_{C\in\mathscr{C}} (V\cap C) \right) = \bigcup_{C\in\mathscr{C}}\iota_{C}^{-1}(V\cap C).
	\end{equation*}

	$\iota_{C}^{-1}(V\cap C)$ is open in $C$ for every $C\in\mathscr{C}$. On the other hand, \textbf{$X$ is locally connected so every component of $X$ is open}, so $C$ is an open subset of $X$ for every $C\in\mathscr{C}$. From Exercise~\ref{exercise:3.6} (Proposition 3.5), it follows that $\iota_{C}^{-1}(V\cap C)$ is open in $X$. Hence $f^{-1}(V)$ is open in $X$. Therefore $f$ is continuous.

	Thus $f$ is a homeomorphism, which implies that $X$ is homeomorphic to the disjoint union of its components.
\end{proof}

\begin{problem}{4-9}
Show that every $n$-manifold is homeomorphic to a disjoint union of countably many connected $n$-manifolds, and every $n$-manifold with boundary is homeomorphic to a disjoint union of countably many connected $n$-manifolds with (possibly empty) boundaries.
\end{problem}

\begin{proof}
	Every manifold (with or without boundary) is locally path-connected (hence locally connected).

	Suppose $M$ is an $n$-manifold (with or without boundary). From Problem~\ref{problem:4-8}, $M$ is locally connected, so $M$ is homeomorphic to the disjoint union of its components. Because $M$ is locally connected, every component of $M$ is open. Assume for the sake of contradiction that $M$ has uncountably many components. $M$ is second countable so it has a countable basis $\mathscr{B}$. Each component $C$ of $M$ is open and nonempty so it contains an element $B_{C}$ of $\mathscr{B}$. Since the components of $M$ are disjoint and uncountably many, it follows that $\mathscr{B}$ is uncountably infinite, which is a contradiction. Hence $M$ has countably many components.

	Let $C$ be a component of $M$ and $x$ be an element of $C$.

	\textbf{$M$ is an $n$-manifold without boundary.}

	Because $M$ is an $n$-manifold, $x$ is contained in some coordinate ball $B$. Moreover, every coordinate ball of an $n$-manifold is connected (because it is homeomorphic to an open ball of $\mathbb{R}^{n}$) so $B$ is contained in a single component of $M$. Since $x\in B, x\in C$, then $B \subseteq C$, so $C$ is locally Euclidean of dimension $n$. On the other hand, $C\subseteq M$ is second countable and Hausdorff (with the subspace topology) so $C$ is a connected $n$-manifold. Therefore $M$ is homeomorphic to a disjoint union of countably many connected $n$-manifolds.

	\textbf{$M$ is an $n$-manifold with boundary.}

	Because $M$ is an $n$-manifold with boundary, $x$ is contained in some regular coordinate ball or half-ball $B$. Every regular coordinate ball (or half-ball) of an $n$-manifold is connected, so $B$ is connected and is contained in a single component. Since $x\in B, x\in C$, then $B\subseteq C$, so $C$ is locally Euclidean of dimension $n$. On the other hand, $C\subseteq M$ is second countable and Hausdorff (with the subspace topology) so $C$ is a connected $n$-manifold. Therefore $M$ is homeomorphic to a disjoint union of countably many connected $n$-manifolds with (possibly empty) boundaries.
\end{proof}

\begin{problem}{4-10}\label{problem:4-10}
Let $S$ be the square $I\times I$ with the order topology generated by the dictionary order.
\begin{enumerate}[label={(\alph*)}]
	\item Show that $S$ has the least upper bound property.
	\item Show that $S$ is connected.
	\item Show that $S$ is locally connected, but not locally path-connected.
\end{enumerate}
\end{problem}

\begin{proof}
	Denote by $\pi_{1}, \pi_{2}$ the canonical maps from $I\times I$ to $I$ given by $\pi_{1}(x, y) = x$ and $\pi_{2}(x, y) = y$. $I$ has the least upper bound property.
	\begin{enumerate}[label={(\alph*)}]
		\item Let $A$ be a nonempty subset of $S$. $\pi_{1}(A)$ is a nonempty subset of $I$ so it has a least upper bound $x_{0}$.

		      If $x_{0} \notin \pi_{1}(A)$ then $\tuple{x_{0}, 0}$ is an upper bound of $A$. If $\tuple{x, y} < \tuple{x_{0}, 0}$ then $x < x_{0}$ so there is $x' \in \pi_{1}(A)$ such that $x < x' < x_{0}$, which means any element of $S$ less than $\tuple{x_{0}, 0}$ is not an upper bound of $A$. So $\tuple{x_{0}, 0}$ is the least upper bound of $A$.

		      If $x_{0} \in \pi_{1}(A)$, define $B = \set{ \tuple{x, y} \in A : x = x_{0} }$ then $B$ is nonempty. $\pi_{2}(B)$ is a nonempty subset of $I$ so it has a least upper bound $y_{0}$. If $\tuple{x, y} < \tuple{x_{0}, y_{0}}$ then either $x = x_{0}$ and $y < y_{0}$ or $x < x_{0}$. In case $x = x_{0}$ and $y < y_{0}$, there is $y' \in \pi_{2}(B)$ such that $y < y'$, so $\tuple{x, y} < \tuple{x_{0}, y'} \leq \tuple{x_{0}, y_{0}}$. In case $x < x_{0}$, for every $y' \in \pi_{2}(B)$, $\tuple{x, y} < \tuple{x_{0}, y'} \leq \tuple{x_{0}, y_{0}}$. In either case, $\tuple{x, y}$ is not an upper bound. So $\tuple{x_{0}, y_{0}}$ is the least upper bound of $A$.

		      Hence $S$ has the least upper bound property.
		\item We will prove a stronger result, which is a generalization for Proposition 4.11: A subset $I$ of a topological space $X$ where
		      \begin{itemize}
			      \item $X$ is a linear continuum
			      \item $X$ is endowed with the order topology
			      \item $X$ has the least upper bound property
		      \end{itemize}

		      is connected if and only if $I$ is a singleton or an interval.

		      The proof for this result is similar to that of Proposition 4.11. In fact, the three properties of $X$ listed above are precisely the properties of $\mathbb{R}$ which are used in the proof of Proposition 4.11 in the book. However, we write a proof in details as follows.

		      A basis for the order topology on $X$ is the collection of open intervals of $X$.

		      If $I$ is a singleton then it is connected, so assume that $I$ has at least two elements.

		      Suppose $I$ is an interval. Assume for the sake of contradiction that $I$ is not connected then there are open subsets $U, V\subseteq X$ such that $U\cap I$ and $V\cap I$ disconnect $I$. Choose $a \in U\cap I$ and $b\in V\cap I$ and assume (interchanging $U, V$ if necessary) that $a < b$. Since $I$ is an interval, the closed interval $\closedinterval{a, b}$ is contained in $I$. Because $U, V$ are open, there exist $a'\in U\cap I$, $b'\in V\cap I$, such that $a < a', b' < b$ and $\halfopenright{a, a'} \subseteq U\cap I$ and $\halfopenleft{b', b}\subseteq V\cap I$. From this, we deduce that $a' \leq b'$ (since otherwise, $\halfopenright{a, a'}$ and $\halfopenleft{b', b}$ are not disjoint, so $U$ and $V$ are not disjoint.)

		      Let $c = \sup (U\cap \closedinterval{a, b})$. Since $\halfopenright{a, a'} \subseteq U\cap I \subseteq U$ and $\halfopenright{a, b'} \subseteq \closedinterval{a, b}$, it follows that $\halfopenright{a, a'} \subseteq U\cap \closedinterval{a, b}$, so $a' = \sup\halfopenright{a, a'} \leq \sup (U\cap \closedinterval{a, b}) = c$. Assume that $c > b'$ then $b'$ is not an upper bound of $U\cap\closedinterval{a, b}$, which means there exists $d \in U\cap\closedinterval{a, b}$ such that $d > b'$, which implies $U\cap I$ and $V\cap I$ are not disjoint (because they both contain $d$), which is a contradiction, so $c\leq b'$. Hence $c \in \closedinterval{a', b'} \subseteq \openinterval{a, b} \subseteq \closedinterval{a, b} \subseteq I$.

		      If $c\in U$ then $c\in U\cap\closedinterval{a, b}$, there is a neighborhood $c\in \openinterval{c_{1}, c_{2}} \subseteq U$, so $c\in \openinterval{c_{1}, c_{2}} \cap \closedinterval{a, b} \subseteq U\cap\closedinterval{a, b}$, this contradicts $c$ being the least upper bound of $U\cap\closedinterval{a, b}$.

		      Otherwise, $c\in V$, then there is a neighborhood $c\in \openinterval{c_{1}, c_{2}} \subseteq V$, so $c\in \openinterval{c_{1}, c_{2}} \cap I \subseteq V\cap I$, which means $\openinterval{c_{1}, c_{2}}$ is disjoint from $U\cap I$ (because $U\cap I$ and $V\cap I$ are disjoint). Again, this contradicts the definition of $c$ because $\closedinterval{c_{1}, c_{2}}$ (a neighborhood of the supremum of $U\cap \closedinterval{a, b}$) must intersect $U\cap \closedinterval{a, b}$, hence intersects $U\cap I$.

		      Hence the assumption is false, so $I$ is connected.

		      Conversely, suppose that $I$ is not an interval, so there exist $a < c < b$ such that $a, b\in I$ but $c\notin I$. The sets $\openinterval{-\infty, c}\cap I$ and $\openinterval{c, \infty}\cap I$ disconnects $I$ so $I$ is not connected.

		      Back to the problem. $S$ is the interval $\closedinterval{\tuple{0,0}, \tuple{1,1}}$ so it is connected.
		\item Because the collection of open intervals of $S$ is a basis for the order topology on $S$ and every open interval of $S$ is connected, it follows that $S$ is locally connected.

		      Assume that $S$ is path-connected, so there is a continuous map $f: \closedinterval{0, 1} \to S$ such that $f(0) = \tuple{0,0}$ and $f(1) = \tuple{1,1}$. $f(\closedinterval{0,1})$ is connected because the image of a connected set under a continuous map. $f(\closedinterval{0,1})$ is connected so it is an interval, according to the proof in part (b) and since it contains $\tuple{0, 0}$ and $\tuple{1, 1}$, we conclude that $f(\closedinterval{0, 1})$ contains every point of $S$.

		      For each $x\in I$, $U_{x} = f^{-1}(\set{x} \times \openinterval{0,1})$ is a nonempty open subset of $\closedinterval{0,1}$, because $f$ is continuous and $\set{x} \times \openinterval{0, 1} = \openinterval{\tuple{x, 0}, \tuple{x, 1}}$ is an open interval (which is an open subset) in $S$. For each $x\in I$, choose a rational number $q_{x} \in U_{x}$. Because the sets $U_{x}$ are pairwise disjoint, the map $x\mapsto q_{x}$ is an injection from $I$ into $\mathbb{Q}\cap I$. This means the cardinality of $I$ is less than or equal to the cardinality of $\mathbb{Q}\cap I$ but this is a contradiction because $I$ is uncountable and $\mathbb{Q}\cap I$ is countable. Therefore the assumption is false and $S$ is not path-connected.

		      Assume that $S$ is locally path-connected. In a locally path-connected space, connectedness and path-connectedness are equivalent, so $S$ is path-connected, which is a contradiction. Hence $S$ is not locally path-connected.
	\end{enumerate}
\end{proof}

\begin{problem}{4-11}\label{problem:4-11}
Let $X$ be a topological space, and let $CX$ be the cone on $X$ (see Example 3.53).
\begin{enumerate}[label={(\alph*)}]
	\item Show that $CX$ is path-connected.
	\item Show that $CX$ is locally connected if and only if $X$ is, and locally path-connected if and only if $X$ is.
\end{enumerate}
\end{problem}

\begin{proof}
	$CX = (X \times I)/(X \times\set{0})$. Denote by $q$ the quotient map $X\times I \to (X \times I)/(X \times\set{0})$.

	\begin{enumerate}[label={(\alph*)}]
		\item $q(\tuple{y, 0})$ and $q(\tuple{y, a})$ are path-connected for every $a\in I$, because the map $f: I \to CX$ given by $f(t) = q(\tuple{y, t\cdot a})$ is continuous since $f = q\circ g$ where $g: I \to X\times I$, $g(t) = \tuple{y, t\cdot a}$. Hence $f$ is continuous and $f(0) = q(\tuple{y, 0})$ and $f(1) = q(\tuple{y, a})$, so there is a path from $q(\tuple{y, 0})$ to $q(\tuple{y, a})$.

		      Let $q(\tuple{x, a})$ and $q(\tuple{y, b})$ be two arbitrary points of $CX$. According to the previous paragraph, there is a path from $q(\tuple{x, a})$ to $q(\tuple{x, 0}) = q(\tuple{y, 0})$ and a path from $q(\tuple{y, 0})$ to $q(\tuple{y, b})$, so there is a path from $q(\tuple{x, a})$ and $q(\tuple{y, b})$.

		      Hence $CX$ is path-connected.
		\item Suppose $X$ is locally {(path-)}connected then $X\times I$ is locally {(path-)}connected, since $I$ is locally {(path-)}connected and the product of two local {(path-)}connected spaces is locally {(path-)}connected. From Problem~\ref{problem:4-7}, we conclude that the quotient space $CX = (X\times I)/(X\times\set{0})$ is locally {(path-)}connected.

		      Conversely, suppose $CX$ is locally {(path-)}connected. Let $x\in X$ and $U$ be a neighborhood of $x$. The product set $U\times\halfopenleft{0, 1}$ is open in $X\times I$ and saturated (any two points of $U\times\halfopenleft{0, 1}$ are not identified by $q$, since $U\times\halfopenleft{0, 1}$ doesn't intersect $X\times\set{0}$), so $q(U\times \halfopenleft{0,1})$ is open in $CX$. Because $CX$ is locally {(path-)}connected, there is a {(path-)}connected open set $V_{x}$ such that $q(\tuple{x, 1}) \in V_{x} \subseteq q(U\times \halfopenleft{0,1})$. Therefore $\tuple{x, 1} \in q^{-1}(V_{x}) \subseteq q^{-1}(q(U\times \halfopenleft{0, 1})) = U\times\halfopenleft{0,1}$. The set $q^{-1}(V_{x})$ is saturated (by definition) and open in $X\times I$ (since $q$ is continuous).

		      Since the restriction of a quotient map to a saturated open set is still a quotient map, it follows that $q\vert_{q^{-1}(V_{x})}: q^{-1}(V_{x}) \to V_{x}$ is a quotient map. On the other hand, any two points of $q^{-1}(V_{x})$ are not identified by $q$ (since its superset $U\times\halfopenleft{0,1}$ has this property) so $q\vert_{q^{-1}(V_{x})}$ is injective. Every injective quotient map is a homeomorphism, so $q\vert_{q^{-1}(V_{x})}$ is a homeomorphism. Because of this homeomorphism $V_{x}$ being {(path-)}connected implies $q^{-1}(V_{x})$ being {(path-)}connected. The canonical projection $\pi_{1}: X\times I \to X$ is open and continuous, so $\pi_{1}(q^{-1}(V_{x}))$ is a {(path-)}connected open subset of $X$. Hence $x \in \pi_{1}(q^{-1}(V_{x})) \subseteq \pi_{1}(U\times\halfopenleft{0,1}) = U$, from which we conclude that $X$ is locally {(path-)}connected.
	\end{enumerate}
\end{proof}

\begin{problem}{4-12}
Suppose $X$ is a topological space and $S\subseteq X$ is a subset that is both open and closed in $X$. Show that $S$ is a union of components of $X$.
\end{problem}

\begin{proof}
	If $S$ is empty then the statement is true. Suppose that $S$ is nonempty, then $S$ intersects some component $C$ of $X$.

	Assume that $C\smallsetminus S$ is nonempty. We have $C\smallsetminus S = C\cap (X\smallsetminus S)$ and $C = (C\cap S) \cup (C\cap (X\smallsetminus S))$ where $C\cap S$ and $C\smallsetminus S$ are nonempty. On the other hand, $C\cap S$ is open in $C$ (due to the definition of subspace topology and $S\subseteq X$ is open) and $C\cap (X\smallsetminus S)$ is open in $C$ (due to the  definition of subspace topology and $X\smallsetminus S \subseteq X$ is open). So $C\cap S$ and $C\cap (X\smallsetminus S)$ disconnect $C$, which is a contradiction since $C$ is a component.

	Hence $C\smallsetminus S$ is empty and it follows that $C\subseteq S$. Therefore any component of $X$ that intersects $S$ is contained in $S$, thus $S$ is a union of components of $X$.
\end{proof}

\begin{problem}{4-13}\label{problem:4-13}
Let $T$ be the topologist's sine curve (Example 4.17).
\begin{enumerate}[label={(\alph*)}]
	\item Show that $T$ is connected but not path-connected or locally connected.
	\item Determine the components and that path components of $T$.
\end{enumerate}
\end{problem}

\begin{proof}
	We write down the definition of the topologist's sine curve.
	\begin{align*}
		T_{0} & = \set{\tuple{x,y} : x = 0 \text{ and } y \in \closedinterval{-1,1}};           \\
		T_{+} & = \set{\tuple{x,y} : x \in \halfopenleft{0, 2/\pi} \text{ and } y = \sin(1/x)}; \\
		T     & = T_{0} \cup T_{+}.
	\end{align*}
	\begin{enumerate}[label={(\alph*)}]
		\item Consider the function $f: \halfopenleft{0, 2/\pi} \to \mathbb{R}$ given by $f(x) = \sin(1/x)$. $f$ is the composition of the two continuous functions $x\mapsto 1/x$ (where $x\in \halfopenleft{0, 2/\pi}$) and $x\mapsto \sin(x)$ so $f$ is continuous. Moreover, $\halfopenleft{0, 2/\pi}$ is connected, so $f(\halfopenleft{0, 2/\pi})$ is also connected. The set $T_{+}$ is the graph of $f$, in other words, $T_{+} = \set{ (x, f(x)) : x\in\halfopenleft{0,2/\pi} }$. Due to the limit property of continuous functions, every point of $T_{+}$ is a limit point of $T_{+}$. However, the points of $T_{+}$ are not all limit points of $T_{+}$.

		      \textbf{$T$ is connected.}

		      Assume that $\tuple{x_{0}, y_{0}}$ is a limit point of $T_{+}$ which is not in $T_{+}$. We will show that $x_{0} = 0$ and $y_{0} \in \closedinterval{-1,1}$.

		      If $y_{0} > 1$ then $\openinterval{x_{0} - 1, x_{0} + 1} \times \openinterval{(1 + y_{0})/2, 1 + y_{0}}$ is a neighborhood of $\tuple{x_{0}, y_{0}}$, which doesn't intersect $T_{+}$ (since the $y$-coordinate of every point in this neighborhood is greater than 1).

		      If $y_{0} < -1$ then $\openinterval{x_{0} - 1, x_{0} + 1} \times \openinterval{-1 + y_{0}, (-1 + y_{0})/2}$ is a neighborhood of $\tuple{x_{0}, y_{0}}$, which doesn't intersect $T_{+}$ (since the $y$-coordinate of every point in this neighborhood is smaller than -1).

		      Hence $y_{0} \in \closedinterval{-1,1}$.

		      If $x_{0} < 0$ then $\openinterval{x_{0} - 1, x_{0}/2} \times \openinterval{y_{0} - 1, y_{0} + 1}$ is a neighborhood of $\tuple{x_{0}, y_{0}}$, which doesn't intersect $T_{0}$ (since the $x$-coordinate of every point in this neighborhood is smaller than 0).

		      If $x_{0} > 2/\pi$ then $\openinterval{(x_{0} + 2/\pi)/2, x_{0} + 1} \times \openinterval{y_{0} - 1, y_{0} + 1}$ is a neighborhood of $\tuple{x_{0}, y_{0}}$, which doesn't intersect $T_{0}$ (since the $x$-coordinate of every point in this neighborhood is greater than $2/\pi$).

		      If $x_{0} \in \openinterval{0, 2/\pi}$ then $y_{0} \ne \sin(1/x_{0})$. Consider the set $\closedinterval{x_{0}/2, 2/\pi}$ and the sequence of points ${(\tuple{x_{i}, f(x_{i})})}_{i\in\mathbb{N}}$ that converges to $\tuple{x_{0}, y_{0}}$, then $x_{i} \to x_{0}$. Because $f$ is continuous on $\closedinterval{x_{0}/2, 2/\pi}$, $f(x_{i}) \to f(x_{0})$, so $f(x_{0}) = y_{0}$, which is a contradiction.

		      Hence $x_{0} = 0$.

		      By using the sequence lemma (Problem~\ref{problem:2-14}), we will show that $\tuple{x_{0}, y_{0}}$ where $x_{0} = 0$ and $y_{0} \in \closedinterval{-1,1}$ is a limit point of $T_{+}$. Because $y_{0} \in \closedinterval{-1,1}$, there exists $\alpha \in \closedinterval{\pi/2, 3\pi/2}$ such that $\sin(\alpha) = y_{0}$. Consider the sequence ${(\tuple{x_{i}, f(x_{i})})}_{i\in\mathbb{N}}$ where $x_{i} = 1/(\alpha + 2i\pi)$. This sequence converges to $\tuple{0, \sin(\alpha)} = \tuple{x_{0}, y_{0}}$.

		      Therefore the closure of $T_{+}$ in $\mathbb{R}^{2}$ is $T$. Since $T_{+}$ is connected, its closure is also connected, which means $T$ is connected.

		      \textbf{$T$ is not path-connected.}

		      Assume that there is a continuous $\gamma: \closedinterval{0, 1} \to T$ such that $\gamma(0) = \tuple{0, y_{0}}$ and $\gamma(1)$ is a point of $T_{+}$ where $y_{0}\in\closedinterval{-1,1}$. Because $\gamma$ is continuous and $T_{0}$ is closed in $T$, $\gamma^{-1}(T_{0})$ is closed in $\closedinterval{0,1}$. Let $t_{0} = \sup \gamma^{-1}(T_{0})$ then $0\leq t_{0} < 1$. If $t_{0} = 0$ then $\gamma$ maps 0 to every point of $T_{0}$, which violates the definition of function, so $0 < t_{0} < 1$.

		      Let $\delta > 0$ such that $\openinterval{t_{0}, t_{0} + \delta} \subseteq \closedinterval{0, 1}$. $(\pi_{2} \circ \gamma)(t_{0}) = b \in \closedinterval{-1,1}$. For every $t > t_{0}$, $\gamma(t) \in T_{+}$, which means $(\pi_{1}\circ\gamma)(t_{0} + \delta) > 0$.

		      There exist $\theta, \phi \in \closedinterval{-\pi/2, \pi/2}$ such that $\sin(\theta) = b$ and $\abs{\sin(\phi) - \sin(\theta)} \geq 1$. There exists a positive integer $n$ such that $0 = (\pi_{1}\circ \gamma)(t_{0}) < \frac{2}{4n\pi + 2\phi} < (\pi_{1}\circ\gamma)(t_{0} + \delta)$. By the intermediate value theorem, there exists $t \in \openinterval{t_{0}, t_{0} + \delta}$ such that $(\pi_{1}\circ \gamma)(t) = \frac{2}{4n\pi + 2\phi}$, then $(\pi_{2}\circ \gamma)(t) = \sin \left(\frac{4n\pi + 2\phi}{2}\right) = \sin(\phi)$, so $\abs{ (\pi_{2}\circ \gamma)(t) - (\pi_{2}\circ \gamma)(t_{0}) } = \abs{ \sin(\phi) - \sin(\theta) } \geq 1$.

		      Hence $\pi_{2}\circ \gamma$ is not continuous at $t_{0}$, which is a contradiction because $\pi_{2}, \gamma$ are continuous.

		      So there is no path from a point of $T_{0}$ to a point of $T_{+}$. Therefore $T$ is not path-connected.

		      \textbf{$T$ is not locally connected.}

		      Assume that $T$ is locally connected. In a locally connected space, the components of every open subset are open in that space.

		      The square $S_{r} = \openinterval{-r, r}\times\openinterval{-r, r}$ where $0 < r < 1$ is open in $\mathbb{R}^{2}$ and it contains $\tuple{0, 0}$. So $S_{r}\cap T$ is a neighborhood of $\tuple{0,0}$ in $T$. Let $C$ be the component of $S_{r}\cap T$ which contains $\tuple{0,0}$, then $C$ is open in $T$. On the other hand, every nonempty subset of $T_{0}$ is not open, so $C$ is not a subset of $T_{0}$ and $C$ intersects $T_{+}$. Let $\tuple{c, f(c)}$ be a point of $C\cap T_{+}$. If there is a point $\tuple{d, f(d)}$ in $T_{+}$ which is not in $C$ and $0 < d < c$ then $C$ is disconnected by the following two open sets
		      \begin{align*}
			       & C\cap (T_{0} \cup \set{ \tuple{x, f(x)} : x < d }), \\
			       & C\cap \set{\tuple{x, f(x) : x > d }}.
		      \end{align*}

		      The set $T_{0} \cup \set{ \tuple{x, f(x)} : x < d }$ is open in $T$ because it is equal to $T \cap (\openinterval{-\infty, d} \times \mathbb{R})$.

		      The set $\set{\tuple{x, f(x) : x > d }}$ is open in $T$ because it is equal to $T\cap (\openinterval{d, \infty} \times \mathbb{R})$.

		      Hence every point $\tuple{d, f(d)} \in T_{+}$ where $d < c$ is in $C$. On the other hand, there exists a positive integer $n$ such that $t = \frac{2}{4n\pi + \pi} < c$. Then $\tuple{t, \sin(1/t)} = \tuple{t, 1} \in C$. Meanwhile, $C = S_{r}\cap T$, $0 < r < 1$, so $\tuple{t, 1} \notin C$, which is a contradiction.

		      Thus $T$ is not locally connected.
		\item Because $T$ is connected, $T$ has exactly one component, which is $T$.

		      $T_{+}$ and $T_{0}$ are path-connected and disjoint. From the proof in part (a), there is no path connecting a point of $T_{0}$ to a point of $T_{1}$. Therefore $T_{+}$ and $T_{0}$ are maximal, which means $T_{+}$ and $T_{0}$ are the path components of $T$.
	\end{enumerate}
\end{proof}

\begin{problem}{4-14}
This chapter introduced four connectedness properties: connectedness, path connectedness, local connectedness, and local path connectedness. Use the following examples to show that any subset of these four properties can be true while the others are false, except those combinations that are disallowed by Theorem 4.15 and Proposition 4.26(a,e).
\begin{enumerate}[label={(\alph*)}]
	\item The set $\mathbb{Q}^{2}$ of rational points in the plane.
	\item The topologist's sine curve $T$ (Example 4.17).
	\item The union of $T$ with the $x$-axis.
	\item The space $S$ of Problem~\ref{problem:4-10}.
	\item The cone on $S$ (see Problem~\ref{problem:4-11}).
	\item The disjoint union of two copies of $S$.
	\item Any disconnected manifold.
	\item Any nonempty connected manifold.
\end{enumerate}
\end{problem}

\begin{table}[htp]
	\centering
	\begin{tabular*}{\textwidth}{c|c|c|c|c}
		& Connected & Path Connected & Local Connected & Local Path-Connected \\
		\toprule
		\bottomrule
		$\mathbb{Q}^{2}$ & no        & no             & no              & no \\
		\hline
		$T$              & yes       & no             & no              & no \\
		\hline
		$T\cup X$        & yes       & yes            & no              & no \\
		\hline
		$S$              & yes       & no             & yes             & no \\
		\hline
		$CS$             & yes       & yes            & yes             & no \\
		\hline
		$S\sqcup S$      & no        & no             & yes             & no \\
		\hline
		\makecell{disconnected \\ manifolds}              & no        & no             & yes             & yes \\
		\hline
		\makecell{nonempty \\ connected \\ manifolds}              & yes       & yes            & yes             & yes
	\end{tabular*}
	\caption{Combinations of connected properties with examples. The other combinations violate Theorem 4.15 or Proposition 4.26(a,e).}
\end{table}

\begin{proof}
	\begin{enumerate}[label={(\alph*)}]
		\item Define $A = \set{ \tuple{x, y}\in \mathbb{Q}^{2} : x > \sqrt{2} }$ and $B = \set{ \tuple{x, y}\in \mathbb{Q}^{2} : x < \sqrt{2} }$. These are open subsets of $\mathbb{Q}^{2}$ since $A = \mathbb{Q}^{2} \cap (\openinterval{\sqrt{2}, \infty} \times \mathbb{R})$ and $B = \mathbb{Q}^{2} \cap (\openinterval{-\infty, \sqrt{2}} \times \mathbb{R})$. On the other hand, $A$ and $B$ are disjoint. Therefore $A, B$ disconnect $\mathbb{Q}^{2}$, which means $\mathbb{Q}^{2}$ is not connected. Consequently, $\mathbb{Q}^{2}$ is not path-connected.

		      Any subset of $\mathbb{Q}^{2}$ containing at least two elements is disconnected (see Example 4.19 (c)). Hence the nonempty connected subsets of $\mathbb{Q}^{2}$ are singletons. On the other hand, the singletons of $\mathbb{Q}^{2}$ are not open. Therefore $\mathbb{Q}^{2}$ is not locally connected or locally path-connected.
		\item $T$ is connected, but not path-connected.

		      $T$ is not locally connected, so it is not locally path-connected.
		\item Denote the $x$-axis by $X$ and the union of $T$ and $X$ by $T'$.

		      $T_{0} \cup X$ is path-connected because $T_{0}, X$ are path-connected and have a common point.

		      $T_{+}\cup X$ is path-connected because of the same reason.

		      Hence $T' = T\cup X = (T_{0} \cup X)\cup (T_{+}\cup X)$ is path-connected, and also connected.

		      Assume that $T'$ is locally connected. The proof is similar to that of Problem~\ref{problem:4-13}, but we use the point $\tuple{0,1}$ instead of $\tuple{0,0}$.

		      The open rectangle $U = \openinterval{-r, r} \times \openinterval{0, 1+\varepsilon}$ where $r, \varepsilon > 0$ is open in $\mathbb{R}^{2}$ and contains $\tuple{0,1}$, so $U \cap T'$ is a neighborhood of $\tuple{0,1}$ in $T'$. Because $T'$ is locally connected, there is a connected neighborhood $V$ such that $\tuple{0,1}\subseteq V \subseteq U\cap T'$. Since $U$ doesn't intersect the $x$-axis, then so does $V$. Any nonempty subset of $T_{0}$ is not open in $T'$, so $V$ is not a subset of $T_{0}$, therefore $V$ intersects $T_{+}$. Let $\tuple{c, \sin(1/c)}$ be a point of $V\cap T_{+}$. If there is $0 < d < c$ such that $\tuple{d, \sin(1/d)} \notin V\cap T_{+}$, then $V$ is disconnected by the following sets
		      \begin{itemize}
			      \item $V \cap (T_{0} \cup \set{ \tuple{x,0} : x < d } \cup \set{ \tuple{x, \sin(1/x)} : x < d })$.

			            This is open in $V$ (with the subspace topology) because
			            \begin{equation*}
				            T_{0} \cup \set{ \tuple{x,0} : x < d } \cup \set{ \tuple{x, \sin(1/x)} : x < d } = T' \cap (\openinterval{-\infty, d} \times\mathbb{R})
			            \end{equation*}

			            which is open in $T'$.
			      \item $V \cap (\set{\tuple{x,0} : x > d} \cup \set{ \tuple{x, \sin(1/x)} : x > d })$.

			            This is open in $V$ (with the subspace topology) because
			            \begin{equation*}
				            \set{\tuple{x,0} : x > d} \cup \set{ \tuple{x, \sin(1/x)} : x > d } = T' \cap (\openinterval{d, \infty} \times\mathbb{R})
			            \end{equation*}

			            which is open in $T'$.
		      \end{itemize}

		      Hence $V$ contains every point $\tuple{d, \sin(1/d)}$ where $0 < d < c$. There exists a positive integer $n$ such that $\frac{1}{2n\pi} < c$, then $\tuple{\frac{1}{2n\pi}, \sin(2n\pi)} = \tuple{\frac{1}{2n\pi}, 0} \in V$. However, this is a contradiction because due to the definition of $U$ (a superset of $V$), every point of $U$ (and so is $V$) has $x$-ordinate greater than 0. Therefore $T'$ is not locally connected.

		      Thus $T\cup X$ is not locally connected, and consequently, it is not locally path-connected.
		\item $S$ is connected but not path-connected.

		      $S$ is locally connected but not locally path-connected.
		\item $CS$ is path-connected, so it is connected.

		      $S$ is locally connected then so is $CS$. $S$ is not locally path-connected then neither is $CS$.
		\item Let $S_{1}, S_{2}$ be disjoint copies of $S$.

		      $S\sqcup S \approx S_{1}\cup S_{2}$. $S_{1}, S_{2}$ are disjoint open subsets of $S_{1}\cup S_{2}$, so $S\sqcup S$ is not connected. Therefore $S\sqcup S$ is not path-connected.

		      $S$ is locally connected so $S_{1}, S_{2}$ are locally connected. Let $\mathscr{B}_{1}, \mathscr{B}_{2}$ be the basis for the topologies on $S_{1}, S_{2}$, correspondingly, where every element of them are connected. Denote $\mathscr{B} = \mathscr{B}_{1} \cup \mathscr{B}_{2}$ and let $U$ be an open subset of $S_{1}\cup S_{2}$. $U = (U\cap S_{1}) \cup (U\cap S_{2})$. Since $U\cap S_{1}$ and $U\cap S_{2}$ are open, they can be expressed as unions of elements of $\mathscr{B}_{1}, \mathscr{B}_{2}$. Hence $\mathscr{B}$ is a connected neighborhood for $S_{1}\cup S_{2}$, which means $S\sqcup S$ is locally connected.

		      Assume $S_{1}\cup S_{2}$ is locally path-connected, then every open subset of $S_{1}\cup S_{2}$ is locally path-connected. Therefore $S_{1}, S_{2}$ are locally path-connected, which is a contradiction. So $S\sqcup S$ is not locally path-connected.
		\item Let $M$ be a disconnected manifold.

		      $M$ is locally path-connected and locally connected.

		      In a local path-connected spaces, connectedness and path-connected are equivalent, so $M$ is not path-connected.
		\item Let $M$ be a nonempty connected manifold.

		      $M$ is locally path-connected and locally connected.

		      In a local path-connected spaces, connectedness and path-connected are equivalent, so $M$ is path-connected.
	\end{enumerate}
\end{proof}

\begin{problem}{4-15}
Suppose $G$ is a topological group.
\begin{enumerate}[label={(\alph*)}]
	\item Show that every open subgroup of $G$ is also closed.
	\item For any neighborhood $U$ of 1, show that the subgroup $\anglebracket{U}$ generated by $U$ is open and closed in $G$.
	\item For any connected subset $U\subseteq G$ containing 1, show that $\anglebracket{U}$ is connected.
	\item Show that if $G$ is connected, then every neighborhood of 1 generated $G$.
\end{enumerate}
\end{problem}

\begin{proof}
	\begin{enumerate}[label={(\alph*)}]
		\item Let $H$ be an open subgroup of $G$. The left translation $L_{g}: G\to G$ given by $g'\mapsto gg'$ is a homeomorphism. Therefore $gH$ is an open subset of $G$ for every $g\in G$ (\textbf{every coset of an open subgroup is open}). On the other hand, $G\smallsetminus H = \bigcup_{g\in G\smallsetminus H} gH$, so the complement of $H$ is open. Therefore $H$ is closed.
		\item $1\in U$ so $U$ is nonempty. For every $g\in \anglebracket{U}$, the set $gU = \set{ gg' : g'\in U }$ is open, since the left translation $L_{g}$ is a homeomorphism. On the other hand, $\anglebracket{U} = \bigcup_{g\in \anglebracket{U}}gU$, which is the union of open sets, so $\anglebracket{U}$ is open. According to part (a), $\anglebracket{U}$ is closed. Hence $\anglebracket{U}$ is open and closed in $G$.
		\item We prove the following statements. We use the theorem: \textbf{the union of connected sets with a common point is connected.}

		      \textbf{Statement 1.} $\bigcup^{k}_{i=0} g^{i}U$ is connected for every $g\in U$, every $k\in \mathbb{Z}_{\geq 0}$.

		      This is true for $k = 0$. Assume that it is true for $k = n - 1$. $g^{n}$ is a common point of $g^{n}U$ and $g^{n-1}U$, so $\bigcup^{n-1}_{i=0}g^{i}U$ and $g^{n}U$ have a common point. The left translation is a homeomorphism so $g^{n}U$ is connected. Due to the inductive hypothesis, $\bigcup^{n-1}_{i=0}g^{i}U$ is connected. Therefore $\bigcup^{n}_{i=0}g^{i}U$ is connected. By the principle of mathematical induction, statement 1 is true.

		      \textbf{Statement 2.} $\bigcup^{k}_{i=0}g^{-i}U$ is connected for every $g\in U$, every $k\in \mathbb{Z}_{\geq 0}$.

		      This is true for $k = 0$. Assume that it is true for $k = n - 1$. $g^{1-n}$ is a common point of $g^{-n}U$ and $g^{1-n}U$ (because $g^{1-n} = g^{1-n}1 = g^{-n}g$), so $\bigcup^{n-1}_{i=0}g^{-i}U$ and $g^{-n}U$ have a common point. The left translation is a homeomorphism so $g^{-n}U$ is connected. Due to the inductive hypothesis, $\bigcup^{n-1}_{i=0}g^{-i}U$ is connected. Therefore $\bigcup^{n}_{i=0}g^{-i}U$ is connected. By the principle of mathematical induction, statement 2 is true.

		      \textbf{Statement 3.} $\bigcup_{k\in\mathbb{Z}}g^{k}U$ is connected for every $g\in U$.
		      \begin{equation*}
			      \bigcup_{k\in\mathbb{Z}}g^{k}U = \bigcup_{k\in\mathbb{Z}_{\geq 0}}\left(\bigcup^{k}_{i=0}g^{i}U\right) \cup \bigcup_{k\in\mathbb{Z}_{\geq 0}}\left(\bigcup^{k}_{i=0}g^{-i}U\right).
		      \end{equation*}

		      From statments 1, 2, and the above equality, we deduce that $\bigcup_{k\in\mathbb{Z}}g^{k}U$ is the union of connected sets with common point (they all contain the nonempty set $U$). Therefore $\bigcup_{k\in\mathbb{Z}}g^{k}U$ is connected.

		      \textbf{Statement 4.} $m$ is a positive integer. For any $m$ elements (not necessarily distinct) $g_{1}, \ldots, g_{m}$ of $U$, the set
		      \begin{equation*}
			      \bigcup_{k_{m}\in\mathbb{Z}}\cdots \bigcup_{k_{1}\in\mathbb{Z}} g_{1}^{k_{1}}\cdots g_{m}^{k_{m}}U
		      \end{equation*}

		      is connected.

		      When $m = 1$, the statement is true because it is statement 3. Assume the statement is true for $m = n - 1$. Consider $n$ elements $g_{1}, \ldots, g_{n}$ of $U$. Denote
		      \begin{equation*}
			      V = \bigcup_{k_{n-1}\in\mathbb{Z}}\cdots \bigcup_{k_{1}\in\mathbb{Z}} g_{1}^{k_{1}}\cdots g_{n-1}^{k_{n-1}}U
		      \end{equation*}

		      then $V$ is connected by the inductive hypothesis. Therefore
		      \begin{equation*}
			      \bigcup_{k_{n}\in\mathbb{Z}}\cdots \bigcup_{k_{1}\in\mathbb{Z}} g_{1}^{k_{1}}\cdots g_{n}^{k_{n}}U = \bigcup_{k_{n}\in\mathbb{Z}}g_{n}^{k_{n}}V
		      \end{equation*}

		      is connected, according to statement 3. By the principle of mathematical induction, statement 3 is true.

		      \textbf{Main result.} $\anglebracket{U}$ is connected.

		      $\anglebracket{U}$ is the set of all finite products of integral powers of elements of $U$, so $\anglebracket{U}$ is the union of the sets of the form
		      \begin{equation*}
			      \bigcup_{k_{m}\in\mathbb{Z}}\cdots \bigcup_{k_{1}\in\mathbb{Z}} g_{1}^{k_{1}}\cdots g_{m}^{k_{m}}U.
		      \end{equation*}

		      where $g_{1}, \ldots, g_{m}$ are any $m$ elements of $U$, not necessarily distinct. On the other hand, these sets have common points (since they all contain the nonempty set $U$) and are connected, so the union of them are connected, which means $\anglebracket{U}$ is connected.
		\item Let $U$ be a neighborhood of 1. According to part (b), $\anglebracket{U}$ is an open and closed subgroup of $G$. $G$ is connected, $\anglebracket{U}$ is a nonempty subset of $G$ which is both open and closed, to $\anglebracket{U} = G$. Thus $U$ generates $G$.
	\end{enumerate}
\end{proof}

\begin{problem}{4-16}
A topological space is said to be \textbf{$\boldsymbol{\sigma}$-compact} if it can be expressed as a union of countably many compact subspaces. Show that a locally Euclidean Hausdorff space is a topological manifold if and only if it is $\sigma$-compact.
\end{problem}

\begin{proof}
	Let $M$ be a locally Euclidean (of dimension $n$) Hausdorff space.

	Suppose $M$ is a topological manifold, then $M$ is has a countable basis $\mathscr{B}$ of regular coordinate balls. Every regular coordinate ball is a precompact open set, so $\overline{B}$ is compact for every $B\in\mathscr{B}$. $M = \bigcup_{B\in\mathscr{B}}B \subseteq \bigcup_{B\in\mathscr{B}}\overline{B}$ and $\bigcup_{B\in\mathscr{B}}\overline{B} \subseteq M$, so $M = \bigcup_{B\in\mathscr{B}}\overline{B}$, which means $M$ is $\sigma$-compact.

	Conversely, suppose $M$ is $\sigma$-compact. Since $M$ is $\sigma$-compact, there is a countable collection $\mathscr{K}$ of compact subspaces of $M$ that covers $M$.

	Let $K \in \mathscr{K}$. For every $p\in K$, there is a neighborhood $B_{p}$ of $p$ which is homeomorphic to an open ball of $\mathbb{R}^{n}$. The collection of $B_{p}$ covers $K$. Since $K$ is compact, there exists $p_{1}, \ldots, p_{m}\in K$ such that $B_{p_{1}}, \ldots, B_{p_{m_{K}}}$ cover $K$. The collection of $B_{p_{1}}, \ldots, B_{p_{m_{K}}}$ of all $K \in \mathscr{K}$ is then a countable open cover of $M$. Denote this countable open cover by $\mathscr{B}$. Since every element of the open cover $\mathscr{B}$ is homeomorphic to an open ball of $\mathbb{R}^{n}$ (which is second countable), every element of $\mathscr{B}$ as a subspace of $M$ is second countable. According to Problem~\ref{problem:2-19}, $M$ is second countable. Therefore $M$ is an $n$-manifold by definition (local Euclidean of dimension $n$, Hausdorff, and second countable).

	Thus $M$ is a topological manifold if and only if it is $\sigma$-compact.
\end{proof}

\begin{problem}{4-17}
Suppose $M$ is a manifold of dimension $n\geq 1$, and $B\subseteq M$ is a regular coordinate ball. Show that $M\smallsetminus B$ is an $n$-manifold with boundary, whose boundary is homeomorphic to $\mathbb{S}^{n-1}$. (You may use the theorem on invariance of the boundary.)
\end{problem}

\begin{proof}
\end{proof}

\begin{problem}{4-21}
Let $V$ be a finite-dimensional real vector space. A \textbf{norm} on $V$ is a real-valued function on $V$, written $x\mapsto \abs{x}$, satisfying the following properties:
\begin{itemize}
	\item \textsc{Positivity}: $\abs{x} \geq 0$, and $\abs{x} = 0$ if and only if $x = 0$.
	\item \textsc{Homogeneity}: $\abs{cx} = \abs{c}\abs{x}$ for all $c\in \mathbb{R}$ and $x\in V$.
	\item \textsc{Triangle inequality}: $\abs{x + y} \leq \abs{x} + \abs{y}$.
\end{itemize}

A norm determines a metric by $d(x, y) = \abs{x - y}$. Show that all norms determine the same topology on $V$. [Hint: first consider the case $V = \mathbb{R}^{n}$, and consider the restriction of the norm to the unit sphere.]
\end{problem}

\begin{proof}
	Consider $V = \mathbb{R}^{n}$.

	\textbf{Firstly, we prove that every norm is a pointwise continuous function.}

	For every $x\in V$, $\varepsilon > 0$, let $\delta(\varepsilon) = \varepsilon$, then whenever $y\in V$ satisfies $\abs{x - y} < \delta(\varepsilon)$, we have $\abs{\abs{x} - \abs{y}} \leq \abs{x - y} < \delta(\varepsilon) = \varepsilon$. Hence $x\mapsto \abs{x}$ is a pointwise continuous function.

	\textbf{Secondly, we prove that every norm is equivalent to the Euclidean norm.}

	Denote the Euclidean norm by $\abs{\cdot}_{2}$ and an arbitrary norm by $\abs{\cdot}$. Since $x\mapsto \abs{x}$ is continuous, then image of the unit sphere $\mathbb{S}^{n-1}$ (which is compact) under the norm is compact. By the extremum value theorem, the norms of vector on the unit sphere can attain maximum and minimum. Moreover, due to the positivity of norm, $\abs{\frac{x}{\abs{x}_{2}}}$ is nonzero for every nonzero $x$ in $V$. Hence there exist positive number $c_{1}, c_{2}$ such that $c_{1} \leq \abs{\frac{x}{\abs{x}_{2}}} \leq c_{2}$ for all $x \in \mathbb{R}^{n}\smallsetminus \set{0}$. Equivalently, $c_{1}\abs{x}_{2} \leq \abs{x} \leq c_{2}\abs{x}_{2}$ for all $x \in \mathbb{R}^{n}$ (The case $x = 0$ is obvious.) This means every norm on $\mathbb{R}^{n}$ are equivalent.

	\textbf{Lastly, we prove that all norms on a finite-dimensional real vector space are equivalent.}

	Let $V$ be an $n$-dimensional real vector space then $V$ is isomorphic to $\mathbb{R}^{n}$ (because all finite-dimensional vector spaces of the same dimension and on the same field are isomorphic). Let $f: V \to \mathbb{R}^{n}$ be an isomorphism. $f$ is linear and $f^{-1}$ is linear. Using $f$, we construct a one-to-one correspondence between norms on $V$ and norms on $\mathbb{R}^{n}$, this is a routine procedure.

	Denote by $\abs{\cdot}_{V}$ a norm on $V$, then we define a real-value function $\abs{\cdot}: \mathbb{R}^{n} \to \mathbb{R}$ as follows: $\abs{x} = \abs{f^{-1}(x)}_{V}$. We will check whether this is a norm.
	\begin{itemize}
		\item $\abs{x} = \abs{f^{-1}(x)}_{V} \geq 0$ for all $x\in \mathbb{R}^{n}$. $\abs{x} = 0$ if and only if $\abs{f^{-1}(x)}_{V} = 0$, $\abs{f^{-1}(x)}_{V} = 0$ if and only if $f^{-1}(x) = 0$, $f^{-1}(x) = 0$ if and only if $x = 0$.
		\item For every $c\in \mathbb{R}, x\in\mathbb{R}^{n}$, $\abs{cx} = \abs{f^{-1}(cx)}_{V} = \abs{cf^{-1}(x)}_{V} = \abs{c}\abs{f^{-1}(x)}_{V} = \abs{c}\abs{x}$.
		\item For every $x, y\in \mathbb{R}^{n}$
		      \begin{equation*}
			      \abs{x + y} = \abs{f^{-1}(x + y)}_{V} = \abs{f^{-1}(x) + f^{-1}(y)}_{V} \leq \abs{f^{-1}(x)}_{V} + \abs{f^{-1}(y)}_{V} = \abs{x} + \abs{y}
		      \end{equation*}
	\end{itemize}

	so $\abs{\cdot}$ is indeed a norm on $\mathbb{R}^{n}$.

	Conversely, if $\abs{\cdot}$ is a norm on $\mathbb{R}^{n}$, $\abs{\cdot}_{V}: V\to \mathbb{R}$ given by $\abs{v}_{V} = \abs{f(v)}$ is a norm. Proof for this is similar.
	\begin{itemize}
		\item $\abs{v}_{V} = \abs{f(v)} \geq 0$ for all $v\in V$. $\abs{v}_{V} = 0$ if and only if $\abs{f(v)} = 0$, $\abs{f(v)} = 0$ if and only if $f(v) = 0$, $f(v) = 0$ if and only if $v = 0$.
		\item For every $c\in \mathbb{R}, v\in V$, $\abs{cv}_{V} = \abs{f(cv)} = \abs{cf(v)} = \abs{c}\abs{f(v)} = \abs{c}\abs{v}_{V}$.
		\item For every $v, w\in V$
		      \begin{equation*}
			      \abs{v + w}_{V} = \abs{f(v + w)} = \abs{f(v) + f(w)} \leq \abs{f(v)} + \abs{f(w)} = \abs{v}_{V} + \abs{w}_{V}.
		      \end{equation*}
	\end{itemize}

	Hence there is a one-to-one correspondence between norms on $V$ and norms on $\mathbb{R}^{n}$. Moreover, every norm on $\mathbb{R}^{n}$ is equivalent to the Euclidean norm, so all norms on $V$ are equivalent.

	Because all norms on $V$ are equivalent, they determine the same topology (see Exercise~\ref{exercise:2.4}).
\end{proof}
