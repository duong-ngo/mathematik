% chktex-file 8
\chapter{Preliminaries: Set theory and categories}

\section{Naive set theory}

\begin{exercise}{1.1}
	Locate a discussion of Russell's paradox, and understand it.
\end{exercise}

\begin{proof}
	I did it.
\end{proof}

\begin{exercise}{1.2}
	Prove that if $\sim$ is an equivalence relation on a set $S$, then the corresponding family $\mathscr{P}_{\sim}$ defined in $\S{1.5}$ is indeed a partion of $S$: that is, its elements are nonempty, disjoint, and their union is $S$.
\end{exercise}

\begin{proof}
	Every element of $\mathscr{P}_{\sim}$ is an equivalence class ${[x]}_{\sim}$ where $x\in S$, hence a nonempty subset of $S$.

	If ${[x]}_{\sim}$ and ${[y]}_{\sim}$ are not disjoint, then there exists $z\in S$ such that $z \in {[x]}_{\sim}$ and $z \in {[y]}_{\sim}$. $s\in {[x]}_{\sim}$ iff $s \sim x$, $s\in {[y]}_{\sim}$ iff $s \sim y$. Together with $z \sim x$ and $z \sim y$, we obtain that $s\in {[x]}_{\sim}$ iff $s \sim z$, $s\in {[y]}_{\sim}$ iff $s \sim z$. Hence $s\in {[x]}_{\sim}$ iff $s\in {[y]}_{\sim}$, which means ${[x]}_{\sim} = {[y]}_{\sim}$. Hence distinct elements of $\mathscr{P}_{\sim}$ are disjoint sets.

	For every $x\in S$, $x \in {[x]}_{\sim}$, so $S \subseteq \bigcup_{x\in S}{[x]}_{\sim} = \bigcup_{P\in \mathscr{P}_{\sim}}P$. Let $x\in \bigcup_{P\in \mathscr{P}_{\sim}}P$ then there is $P\in \mathscr{P}_{\sim}$ such that $x\in P$. Moreover, $P\subseteq S$ so $x\in S$, it follows that $\bigcup_{P\in \mathscr{P}_{\sim}}P \subseteq S$. Hence $S = \bigcup_{P\in \mathscr{P}_{\sim}}P$.

	Thus $\mathscr{P}_{\sim}$ is a partition of $S$.
\end{proof}

\begin{exercise}{1.3}
	Given a partition $\mathscr{P}$ on a set $S$, show how to define a relation $\sim$ on $S$ such that $\mathscr{P}$ is the corresponding partition.
\end{exercise}

\begin{proof}
	Definition: Two elements $x, y$ of $S$ are called equivalent iff there exists $P\in \mathscr{P}$ such that $x\in P$ and $y\in P$.
\end{proof}

\begin{exercise}{1.4}
	How many different equivalence relations may be defined on the set $\set{1, 2, 3}$?
\end{exercise}

\begin{proof}
	Because an equivalence relation on a given set is uniquely determined by one of the set's partition, the number of equivalence relations may be defined on a set is equal to the number of possible partitions on the set. $\set{1, 2, 3}$ has 5 possible partition, namely $\set{\set{1}, \set{2}, \set{3}}$, $\set{\set{1}, \set{2, 3}}$, $\set{\set{2}, \set{1, 3}}$, $\set{\set{3}, \set{1, 2}}$, $\set{\set{1, 2, 3}}$. Hence there are totally 5 different equivalence relations can be defined on the set $\set{1, 2, 3}$.
\end{proof}

\begin{exercise}{1.5}
	Give an example of a relation that is reflexive and symmetric but not transitive. What happens if you attempt to use this relation to define a partition on the set?
\end{exercise}

\begin{proof}
	On $\mathbb{Z}$, we define the relation $R$ as follows: $xRy$ iff $x - y$ is divisible by $2$ or $3$. By this definition, $R$ is reflexive and symmetric. However, $1R3$ and $3R6$ but it is not the case that $1R6$ so $R$ is not transitive. If we define a partition on the set, there might be elements of the partition class which are not disjoint.
\end{proof}

\begin{exercise}{1.6}\label{exercise:1-1.6}
	Define a relation $\sim$ on the set $\mathbb{R}$ of real numbers by setting $a\sim b \Longleftarrow b - a\in\mathbb{Z}$. Prove that this is an equivalence relation, and find a `compelling' description for $\mathbb{R}/\sim$. Do the same for the relation $\approx$ on the plane $\mathbb{R}\times\mathbb{R}$ defined by declaring $\tuple{a_{1}, a_{2}} \approx \tuple{b_{1}, b_{2}} \Longleftrightarrow b_{1} - a_{1} \in \mathbb{Z}$ and $b_{2} - a_{2}\in\mathbb{Z}$.
\end{exercise}

\begin{proof}
	For every $a\in \mathbb{R}$, $a - a = 0 \in \mathbb{Z}$ so $a\sim a$.

	For every $a, b\in\mathbb{R}$, if $a\sim b$ then $a - b\in \mathbb{Z}$ and $b - a\in \mathbb{R}$, which implies that $b\sim a$.

	For every $a, b, c\in\mathbb{R}$, if $a\sim b$ and $b\sim c$ then $a - c = (a - b) + (b - c) \in \mathbb{Z}$, so $a\sim c$.

	Hence $\sim$ is an equivalence relation on $\mathbb{R}$. We can describe the quotient of $\mathbb{R}$ with respect to $\sim$ like this: $\mathbb{R}/_{\sim} = \set{{[x]}_{\sim} \mid 0\leq x < 1}$.

	\bigskip
	For every $\tuple{a_{1}, a_{2}} \in \mathbb{R}\times\mathbb{R}$, $a_{1} - a_{1} = 0\in \mathbb{Z}$ and $a_{2} - a_{2} = 0\in \mathbb{Z}$, so $\tuple{a_{1}, a_{2}} \approx \tuple{a_{1}, a_{2}}$.

	For every $\tuple{a_{1}, a_{2}}, \tuple{b_{1}, b_{2}} \in \mathbb{R}\times\mathbb{R}$, if $\tuple{a_{1}, a_{2}} \approx \tuple{b_{1}, b_{2}}$ then $b_{1} - a_{1}\in \mathbb{Z}$ and $b_{2} - a_{2}\in \mathbb{Z}$. So $a_{1} - b_{1}\in \mathbb{Z}$ and $a_{2} - b_{2}\in \mathbb{Z}$, it follows that $\tuple{b_{1}, b_{2}} \approx \tuple{a_{1}, a_{2}}$.

	For every $\tuple{a_{1}, a_{2}}, \tuple{b_{1}, b_{2}}, \tuple{c_{1}, c_{2}} \in \mathbb{R}\times\mathbb{R}$, if $\tuple{a_{1}, a_{2}} \approx \tuple{b_{1}, b_{2}}$ and $\tuple{b_{1}, b_{2}} \approx \tuple{c_{1}, c_{2}}$ then $a_{1} - b_{1} \in \mathbb{Z}$, $a_{2} - b_{2}\in \mathbb{Z}$, $b_{1} - c_{1}\in\mathbb{Z}$, $b_{2} - c_{2}\in \mathbb{Z}$. Therefore $a_{1} - c_{1} = (a_{1} - b_{1}) + (b_{1} - c_{1}) \in \mathbb{Z}$ and $a_{2} - c_{2} = (a_{2} - b_{2}) + (b_{2} - c_{2}) \in \mathbb{Z}$, which means $\tuple{a_{1}, a_{2}} \approx \tuple{c_{1}, c_{2}}$.

	Hence $\approx$ is an equivalence relation on $\mathbb{R}\times\mathbb{R}$. We can describe the quotient of $\mathbb{R}\times\mathbb{R}$ with respect to $\approx$ as follows: $(\mathbb{R}\times\mathbb{R})/_{\approx} = \set{ {[(x, y)]}_{\approx} \mid 0 \leq x < 1, 0 \leq y < 1 }$.
\end{proof}

\section{Functions between sets}

\begin{exercise}{2.1}
	How many different bijections are there between a set $S$ with $n$ elements and itself?
\end{exercise}

\begin{proof}
	This proof implicitly uses the following result: the composition of two bijections is a bijection.

	$S$ has $n$ elements, it means there is a bijection from $S$ onto $\set{1, \ldots, n}$. Each bijection from $S$ onto itself correspond one-to-one to substitution on $\set{1, \ldots, n}$. Because $\set{1, \ldots, n}$ has $\factorial{n}$ substitutions, it follows that there are $\factorial{n}$ bijections between $S$ and itself.
\end{proof}

\begin{exercise}{2.2}
	Prove statement (2) in Proposition 2.1. You may assume that given a family of disjoint nonempty subsets of a set, there is a way to choose one element in each
	member of the family.
\end{exercise}

\begin{proof}
	Statement (2) in Proposition 2.1: $f: A\to B$ where $A\ne\emptyset$ has a right-inverse if and only if it is surjective.

	$(\Longrightarrow)$ $f$ has a right-inverse. So there is $g: B\to A$ such that $f\circ g = \operatorname{id}_{B}$. Let $b$ be an arbitrary element of $B$, then $f(g(b)) = b$. Hence $f$ is surjective.

	$(\Longleftarrow)$ [This part assumes the axiom of choice] $f$ is surjective. We define $g: B\to A$ as follows: For every $b\in B$, the fiber $f^{-1}(b)$ of $b$ is not empty (because $f$ is surjective), we choose an element from $f^{-1}(b)$ and define it to be $g(b)$. From this definition, it follows that $f\circ g = \operatorname{id}_{B}$, which means $f$ has a right-inverse.

	Hence a map (which is not the empty map) has a right-inverse if and only if it is surjective.
\end{proof}

\begin{exercise}{2.3}
	Prove that the inverse of a bijection is a bijection and that the composition of two bijections is a bijection.
\end{exercise}

\begin{proof}
	Let $f: A\to B$ be a bijection. Because $f$ is both an injection and a surjection, it has a left-inverse $g_{1}$ and a right-inverse $g_{2}$. So
	\begin{align*}
		g_{1} = g_{1}\circ \operatorname{id}_{B} = g_{1}\circ (f\circ g_{2}) = (g_{1}\circ f)\circ g_{2} = \operatorname{id}_{A}\circ g_{2} = g_{2}.
	\end{align*}

	Hence $g_{1} = g_{2}$, this implies the existence and uniqueness of the inverse of the bijection $f$. Because $f^{-1}\circ f = \operatorname{id}_{A}$ and $f\circ f^{-1} = \operatorname{id}_{B}$, it follows that $f^{-1}$ is a bijection, because it has a left-inverse and a right-inverse.

	\bigskip
	Let $f: A\to B$ and $g: B\to C$ be two bijections. We have
	\begin{align*}
		(f^{-1}\circ g^{-1})\circ (g\circ f)  & = f^{-1}\circ ((g^{-1}\circ g)\circ f) = f^{-1}\circ (\operatorname{id}_{B}\circ f) = f^{-1}\circ f = \operatorname{id}_{A}, \\
		(g\circ f) \circ (f^{-1}\circ g^{-1}) & = (g\circ (f\circ f^{-1}))\circ g^{-1} = (g\circ\operatorname{id}_{B})\circ g^{-1} = g\circ g^{-1} = \operatorname{id}_{C}.
	\end{align*}

	So $g\circ f$ has a left-inverse and a right-inverse, which implies $g\circ f$ is a bijection.
\end{proof}

\begin{exercise}{2.4}
	Prove that `isomorphism' is an equivalence relation (on any set of sets).
\end{exercise}

\begin{proof}
	Let $\mathscr{S}$ be a set of sets.

	For every $A\in\mathscr{S}$, $\operatorname{id}_{A}: A\to A$ is a bijection, so $A$ is isomorphic to itself.

	For every $A, B\in\mathscr{S}$, if $A$ is isomorphic to $B$, there is a bijection $f: A\to B$. The inverse of a bijection is a bijection, so $f^{-1}: B\to A$ is a bijection, which implies $B$ is isomorphic to $A$.

	For every $A, B, C\in\mathscr{S}$, if $A$ is isomorphic to $B$ and $B$ is isomorphic to $C$, then there exist bijections $f: A\to B$ and $g: B\to C$. The composition of two bijections is a bijection, so $g\circ f: A\to C$ is a bijection, which means $A$ is isomorphic to $C$.

	Hence isomorphism is an equivalence relation.
\end{proof}

\begin{exercise}{2.5}
	Formulate a notion of \textit{epimorphism}, in the style of the notion of \textit{monomorphism} seen in $\S{2.6}$, and prove a result analogous to Proposition 2.3, for epimorphisms and surjections.
\end{exercise}

\begin{proof}
	A function $f: A\to B$ is a epimorphism if
	\begin{equation*}
		\begin{gathered}
			\text{for all sets $Z$ and all functions $\alpha', \alpha'': B\to Z$} \\
			\alpha'\circ f = \alpha''\circ f \implies \alpha' = \alpha''.
		\end{gathered}
	\end{equation*}

	A result analogous to Proposition 2.3 would be: A function is surjective if and only if it is an epimorphism.

	$(\Longrightarrow)$ $f: A\to B$ is surjective, then $f$ has a right-inverse $g: B\to A$. For all sets $Z$ and all functions $\alpha', \alpha'': B\to Z$, if $\alpha'\circ f = \alpha''\circ f$ then
	\begin{align*}
		\alpha' & = \alpha'\circ \operatorname{id}_{B} = \alpha'\circ (f\circ g)                \\
		        & = (\alpha'\circ f)\circ g = (\alpha''\circ f)\circ g                          \\
		        & = \alpha'' \circ (f\circ g) = \alpha''\circ \operatorname{id}_{B} = \alpha''.
	\end{align*}

	So $f$ is an epimorphism.

	$(\Longleftarrow)$ $f$ is an epimorphism.

	Let $\alpha': B\to \set{0, 1}$ defined by $\alpha'(b) = 1$ for every $b\in B$. Let $\alpha'': B\to \set{0, 1}$ defined by $\alpha''(b) = 1$ if and only if the fiber of $f$ over $b$ is nonempty. From these definitions, $\alpha'\circ f = \alpha''\circ f$, and it follows that $\alpha' = \alpha''$. Therefore $\alpha''(b) = 1$ for every $b\in B$, so the fiber of $f$ over any element of $b$ is nonempty, which means $f$ is surjective.

	Hence a function is surjective if and only if it is an epimorphism.
\end{proof}

\begin{exercise}{2.6}
	With notation as in Example 2.4, explain how any function $f: A\to B$ determines a section of $\pi_{A}$.
\end{exercise}

\begin{proof}
	Let $\Gamma_{f}$ be the graph of $f$, then $\Gamma_{f}\subseteq A\times B$. For every $a\in A$, $(a, f(a)) \in \Gamma_{f}$ and $\pi_{A}$ maps $(a, f(a))$ to $a$. Hence every function $f: A\to B$ determines a section of $\pi_{A}$.
\end{proof}

\begin{exercise}{2.7}
	Let $f: A\to B$ be any function. Prove that the graph $\Gamma_{f}$ of $f$ is isomorphic to $A$.
\end{exercise}

\begin{proof}
	Let's define $\varphi: A\to \Gamma_{f}$ by $\varphi(a) = (a, f(a))$. Let $\psi$ be the restriction of $\pi_{A}$ to $\Gamma_{f}$ then $\psi\circ\varphi = \operatorname{id}_{A}$ and $\varphi\circ\psi = \operatorname{id}_{\Gamma_{f}}$. Therefore $\varphi$ is a bijection, which implies that the graph $\Gamma_{f}$ of $f$ is isomorphic to $A$.
\end{proof}

\begin{exercise}{2.8}
	Describe as explicitly as you can all terms in the canonical decomposition of the function $\mathbb{R}\to \mathbb{C}$ defined by $r\mapsto e^{2\pi ir}$. (This exercise matches one assigned previously. Which one?)
\end{exercise}

\begin{proof}
	On $\mathbb{R}$, we define the relation $\sim$ as follows: $a\sim b$ if and only if $e^{2\pi ia} = e^{2\pi ib}$. Because $e^{2\pi ia} = e^{2\pi ib}$ if and only if $b - a\in\mathbb{Z}$, it follows that $a\sim b$ if and only if $b - a\in\mathbb{Z}$. This is identical to the relation defined in Exercise~\ref{exercise:1-1.6}.

	The image of $f$ is the unit circle $\mathbb{S}^{1}$. So the canonical decomposition of $f$ is
	\begin{figure}[htp]
		\centering
		\renewcommand{\thefigure}{1-2.8}
		\centering
		\begin{tikzpicture}[>=angle 90, every edge/.style={draw, thick}]
			\matrix (m) [matrix of math nodes, row sep=3em, column sep=3em] {
				\mathbb{R} & {\mathbb{R}/\mathbb{Z}} & {\mathbb{S}^{1}} & \mathbb{C} \\
			};
			\path[->>] (m-1-1) edge (m-1-2);
			\path[right hook->] (m-1-3) edge (m-1-4);
			\path[->] (m-1-2) edge node[above] {$\small\sim$} node[below] {$\overline{f}$} (m-1-3);
			\path[->] (m-1-1) edge[bend left=30] node[above] {$f$} (m-1-4);
		\end{tikzpicture}
		\caption{}
	\end{figure}
\end{proof}

\begin{exercise}{2.9}
	Show that if $A'\cong A''$ and $B'\cong B''$, and further $A'\cap B' = \varnothing$ and $A''\cap B'' = \varnothing$, then $A'\cup B' \cong A''\cup B''$. Conclude that the operation $A\coprod B$ (as described in $\S{1.4}$) is well-defined \textit{up to isomorphism}.
\end{exercise}

\begin{proof}
	Let $f$ be a bijection from $A'$ onto $A''$ and $g$ be a bijection from $B'$ onto $B''$. Let's define $h: A'\cup B' \to A''\cup B''$ and $k: A''\cup B''\to A'\cup B'$ as follows:
	\begin{align*}
		h(x) = \begin{cases}
			       f(x) & \text{if $x\in A'$} \\
			       g(x) & \text{if $x\in B'$}
		       \end{cases}
		\qquad
		k(y) = \begin{cases}
			       f^{-1}(y) & \text{if $y\in A''$} \\
			       g^{-1}(y) & \text{if $y\in B''$}
		       \end{cases}.
	\end{align*}

	From this definition, it follows that $k\circ h = \operatorname{id}_{A'\cup B'}$ and $h\circ k = \operatorname{id}_{A''\cup B''}$. Therefore $h, k$ are bijections, which implies $A'\cup B'$ and $A''\cup B''$ are isomorphic.

	Since $A\coprod B$ is defined as the union of two disjoint copies of $A$ and $B$, the proved result implies that the definition of $A\coprod B$ is well-defined up to isomorphism.
\end{proof}

\begin{exercise}{2.10}
	Show that if $A$ and $B$ are finite sets, then $\card{B^{A}} = {\card{B}}^{\card{A}}$.
\end{exercise}

\begin{proof}
	A function is identified with its graph. Let's count how many graphs can be constructed: for every $a\in A$, any element of $B$ can be defined as the image of $a$, which means there are $\card{B}$ choices. Therefore, there are ${\card{B}}^{\card{A}}$ possible graphs. Hence $\card{B^{A}} = {\card{B}}^{\card{A}}$.
\end{proof}

\begin{exercise}{2.11}
	In view of Exercise 2.10, it is not unreasonable to use $2^{A}$ to denote the set of functions from an arbitrary set $A$ to a set with 2 elements (say $\set{0, 1}$). Prove that there is a bijection between $2^{A}$ and the \textit{power set} of $A$.
\end{exercise}

\begin{proof}
	Denote the power set of $A$ by $\mathscr{P}(A)$. Let's define $\varphi: 2^{A}\to \mathscr{P}(A)$ and $\psi: \mathscr{P}(A)\to 2^{A}$ as follows:
	\begin{align*}
		\varphi(f) & = \set{x \mid x\in A \text{ and } f(x) = 1},                      \\
		\psi(S)    & = f \qquad\text{such that $f(x) = 1 \Longleftrightarrow x\in S$},
	\end{align*}

	where $f$ is a function from $A$ to $\set{0, 1}$ and $S$ is a subset of $A$. From this definition, $\psi\circ\varphi = \operatorname{id}_{2^{A}}$ and $\varphi\circ\psi = \operatorname{id}_{\mathscr{P}(A)}$. Therefore $\varphi$ is a bijection, so there is a bijection between $2^{A}$ and the power set of $A$.
\end{proof}

\section{Categories}


\section{Morphisms}


\section{Universal properties}

