\chapter{Metric spaces}

\section{Concept of a metric space}

\subsection*{Exercises}

\begin{enumerate}[label={(\alph*)}]
	\item For \( j = 1, 2, 3 \) give an example of a function \( \rho_{j} \) which associates to each pair from a three-element set \( X \) a real number in such a way that axiom \( (M_{j}) \) is not satisfied while the other two axioms for a metric are.
	\item Show that the axiom system (M1), (M2), (M3) is equivalent to the axiom system consisting of (M1) and (M3'), where
	      \begin{equation}
		      \rho(z, x) \le \rho(x, y) + \rho(y, z) \quad \text{for every} \quad x, y, z \in X.
		      \tag{M3'}
	      \end{equation}
	\item Let \( S^{\omega} = \left\{ x = \left\{ x^{1}, x^{2}, \ldots \right\} \in \mathbb{R}^{\omega}: \sum_{i=1}^{\infty} {(x^{i})}^{2} = 1 \right\} \). Examine whether the function \( \hat{\rho} \) defined by the conditions: \( 0 \le \hat{\rho}(x, y) \le \pi \), \( \cos \hat{\rho}(x, y) = \sum_{i=1}^{\infty} x^{i}y^{i} \) for \( x = \left\{ x^{1}, x^{2}, \ldots \right\}, y = \left\{ y^{1}, y^{2}, \ldots \right\} \in S^{\omega} \) is a metric.
\end{enumerate}

\begin{proof}
	\begin{enumerate}[label={(\alph*)}]
		\item Let \( X = \left\{ a, b, c \right\} \).

		      Define \( \rho_{1} \) by setting
		      \[
			      \rho_{1}(a, a) = \rho_{1}(b, b) = \rho_{1}(c, c) = \rho_{1}(a, b) = \rho_{1}(b, a) = \rho_{1}(b, c) = \rho_{1}(c, b) = \rho_{1}(c, a) = \rho_{1}(a, c) = 0.
		      \]

		      The function \( \rho_{1} \) satisfies (M2) and (M3) but (M1).

		      Define \( \rho_{2} \) by setting \( \rho_{2}(a, a) = \rho_{2}(b, b) = \rho_{2}(c, c) = 0 \) and
		      \[
			      \rho_{2}(a, b) = 1, \rho_{2}(b, a) = 1/2, \rho_{2}(b, c) = 1, \rho_{2}(c, b) = 1, \rho_{2}(a, c) = 1, \rho_{2}(c, a) = 1.
		      \]

		      The function \( \rho_{2} \) satisfies (M1) and (M3) but (M2).

		      Define \( \rho_{3} \) by setting \( \rho_{3}(a, a) = \rho_{3}(b, b) = \rho_{3}(c, c) = 0 \) and
		      \[
			      \rho_{3}(a, b) = \rho_{3}(b, a) = 1, \rho_{3}(b, c) = \rho_{3}(c, b) = 2, \rho_{3}(c, a) = \rho_{3}(a, c) = \dfrac{1}{2}.
		      \]

		      The function \( \rho_{3} \) satisfies (M1) and (M2) but (M3).
		\item Assume that \( \rho \) satisfies (M1), (M2), (M3). For every \( x, y, z \in X \), according to (M2) and (M3)
		      \[
			      \rho(z, x) = \rho(x, z) \le \rho(x, y) + \rho(y, z)
		      \]

		      so (M3') is true.

		      Conversely, assume that \( \rho \) satisfies (M1) and (M3'). For every \( x, y \in X \)
		      \[
			      \rho(x, y) \le \rho(y, x) + \rho(x, x) = \rho(y, x),\qquad \rho(y, x) \le \rho(x, y) + \rho(y, y) = \rho(x, y)
		      \]

		      which implies \( \rho(x, y) = \rho(y, x) \), so (M2) is satisfied. For every \( x, y, z \in X \)
		      \[
			      \rho(x, z) = \rho(z, x) \le \rho(x, y) + \rho(y, z)
		      \]

		      so (M3) is satisfied.

		      Thus the axiom system (M1), (M2), (M3) is equivalent to the axiom system (M1), (M3').
		\item For each positive integer \( m \), one has
		      \[
			      0 \le \sum_{i=1}^{m} \abs{x^{i}y^{i}} \le \sqrt{\sum_{i=1}^{m} {(x^{i})}^{2}} \cdot \sqrt{\sum_{i=1}^{m} {(y^{i})}^{2}} \le 1
		      \]

		      Hence the sequence \( \sum_{i=1}^{n} \abs{x^{i}y^{i}} \) is bounded. According to the monotone convergence theorem, \( \sum_{i=1}^{\infty} \abs{x^{i}y^{i}} \) is convergent. Therefore, the series \( \sum_{i=1}^{\infty} x^{i}y^{i} \) is absolutely convergent, which means \( \hat{\rho} \) is well-defined.

		      Evidently, \( \hat{\rho} \) satisfies (M2).

		      If \( x, y \in S^{\omega} \) then
		      \begingroup
		      \allowdisplaybreaks%
		      \begin{align*}
			      \norm{x}^{2}\norm{y}^{2} - {(x\cdot y)}^{2} & = {(\norm{x}\norm{y})}^{2} - 2{(x\cdot y)}^{2} + {(x\cdot y)}^{2}                                                                \\
			                                                  & = \norm{x} y\cdot \norm{x} y - 2 \norm{x}y \cdot \dfrac{x\cdot y}{\norm{x}} x + {\left( \dfrac{x\cdot y}{\norm{x}}x \right)}^{2} \\
			                                                  & = {\left( \norm{x}y - \dfrac{x\cdot y}{\norm{x}}x \right)}^{2}                                                                   \\
			                                                  & = {\left( y - (x\cdot y)x \right)}^{2}                                                                                           \\
			                                                  & \ge 0.
		      \end{align*}
		      \endgroup

		      The equality holds if and only if \( x, y \) are linearly independent.

		      Assume that \( \hat{\rho(x, y)} = 0 \) then \( \cos\hat{\rho}(x, y) = 1 \) and \( x\cdot y = \norm{x}\norm{y} \), this implies \( x = y \). Hence \( \hat{\rho} \) satisfies (M1).

		      Let \( x, y, z \in S^{\omega} \) and \( 0 \le a, b, c \le \pi \) such that \( a = \hat{\rho}(y, z), b = \hat{\rho}(z, x), c = \hat{\rho}(x, y) \). Let \( s = \dfrac{1}{2}(a + b + c) \). On the one hand
		      \begingroup
		      \allowdisplaybreaks%
		      \begin{align*}
			      4\sin(s - a)\sin(s - b)\sin(s - c)\sin(s) & = (2\sin(s - a)\sin(s - b))(2\sin(s - c)\sin(s))                                   \\
			                                                & = (\cos(a - b) - \cos(c))(\cos(c) - \cos(a + b))                                   \\
			                                                & = -{(\cos(c))}^{2} - \cos(a - b)\cos(a + b) + 2\cos(a)\cos(b)\cos(c)               \\
			                                                & = 1 + 2\cos(a)\cos(b)\cos(c) - {(\cos(a))}^{2} - {(\cos(b))}^{2} - {(\cos(c))}^{2} \\
			                                                & = (1 - {(\cos(b))}^{2})(1 - {(\cos(c))}^{2}) - {(\cos(a) - \cos(b)\cos(c))}^{2}.
		      \end{align*}
		      \endgroup

		      On the other hand
		      \begingroup
		      \allowdisplaybreaks%
		      \begin{align*}
			       & \phantom{=} (1 - {(\cos(b))}^{2})(1 - {(\cos(c))}^{2}) - {(\cos(a) - \cos(b)\cos(c))}^{2}           \\
			       & = {(y - (x\cdot y)x)}^{2}{(z - (x\cdot z)x)}^{2} - {((y - (x\cdot y)x)\cdot (z - (z\cdot x)x))}^{2}
		      \end{align*}
		      \endgroup

		      which is nonnegative according to the Cauchy-Schwarz's inequality.

		      \( s = \dfrac{1}{2}(a, b, c) \) and \( 0 \le a, b, c \le \pi \) so \( 0 \le s \le \dfrac{3\pi}{2} \) and \( \dfrac{-\pi}{2} \le s - a, s - b, s - c \le \dfrac{\pi}{2} \).

		      As \( (s - a) + (s - b) = c \ge 0, (s - b) + (s - c) = a \ge 0, (s - c) + (s - a) = b \ge 0 \), then at most one \( s - a, s - b, s - c \) is negative. Without loss of generality, suppose that \( s - a < 0 \). Because \( (s - b) + (s - c) = a \le \pi \), it follows that \( 0 \le s = (s - a) + (s - b) + (s - c) < \pi \) and \( \sin(s) \ge 0 \). However, \( \sin(s) = 0 \) is not the case because \( \sin(s) = 0 \) implies that \( s = 0 \) (as \( 0 \le s < \pi \)) and \( a = b = c = 0 \). Therefore \( \sin(s) > 0 \) and \( \sin(s - a)\sin(s - b)\sin(s - c) \ge 0 \). This is a contradiction because \( s - a < 0, s - b > 0, s - c > 0 \) implies \( \sin(s - a)\sin(s - b)\sin(s - c) < 0 \).

		      Thus \( \hat{\rho} \) is a metric on \( S^{\omega} \).
	\end{enumerate}
\end{proof}

\section{Operation on metric spaces}

\subsection*{Exercises}

\begin{enumerate}[label={(\alph*)}]
	\item Suppose that \( (X_{i}, \rho_{i}) \) is a metric space for \( i = 1, 2, \ldots, m \) and let \( X = \prod_{i=1}^{m} X_{i} \). Show that the function \( \rho \) which associates with every pair of points \( x = (x_{1}, x_{2}, \ldots, x_{m}) \) and \( y = (y_{1}, y_{2}, \ldots, y_{m}) \) of the set \( X \) the number \( \rho(x, y) = \sum_{i=1}^{m} \rho_{i}(x_{i}, y_{i}) \) is a metric on \( X \).
	\item Suppose that \( (X_{i}, \rho_{i}) \) is a metric space for \( i = 1, 2, \ldots, m \) and let \( X = \prod_{i=1}^{m} X_{i} \). Show that the function \( \rho \) which associates with every pair of points \( x = (x_{1}, x_{2}, \ldots, x_{m}) \) and \( y = (y_{1}, y_{2}, \ldots, y_{m}) \) of the set \( X \) the number \( \rho(x, y) = \max\left\{ \rho_{i}(x_{i}, y_{i}) : i = 1, 2, \ldots, m \right\} \) is a metric on \( X \).
	\item Suppose \( X = \bigcup_{j=1}^{n} X_{j} \) with \( X_{j} \cap X_{k} = \left\{ _{0} \right\} \) for \( j \ne k \) and suppose \( \rho_{j} \) is a metric on \( X_{j} \) for \( j = 1, 2, \ldots, n \). Show that the function \( \rho \) defined by the formula:
	      \[
		      \rho(x, y) = \begin{cases}
			      \rho_{j}(x, y),                          & \text{if } x, y \in X_{j},                    \\
			      \rho_{j}(x, x_{0}) + \rho_{k}(x_{0}, y), & \text{if } x \in X_{j}, y \in X_{k}, j \ne k,
		      \end{cases}
	      \]

	      is a metric on \( X \).
\end{enumerate}

\begin{proof}
	\begin{enumerate}[label={(\alph*)}]
		\item \( \rho(x, y) = 0 \) iff \( \rho_{i}(x_{i}, y_{i}) = 0 \) for every \( i = 1, 2, \ldots, m \), and \( \rho_{i}(x_{i}, y_{i}) = 0 \) for every \( i = 1, 2, \ldots, m \) iff \( x_{i} = y_{i} \) for every \( i = 1, 2, \ldots, m \). So \( \rho(x, y) = 0 \) iff \( x = y \).

		      \( \rho(x, y) = \sum_{i=1}^{m} \rho_{i}(x_{i}, y_{i}) = \sum_{i=1}^{m} \rho_{i}(y_{i}, x_{i}) = \rho(y, x) \).

		      For every \( x, y, z \in X \)
		      \[
			      \rho(x, z) = \sum_{i=1}^{m} \rho_{i}(x_{i}, z_{i}) \le \sum_{i=1}^{m} (\rho_{i}(x_{i}, y_{i}) + \rho_{i}(y_{i}, z_{i})) = \sum_{i=1}^{m} \rho_{i}(x_{i}, y_{i}) + \sum_{i=1}^{m} \rho_{i}(y_{i}, z_{i}) = \rho(x, y) + \rho(y, z).
		      \]

		      Hence \( \rho \) is a metric on \( X \).
		\item \( \rho(x, y) = 0 \) iff \( \max\left\{ \rho_{i}(x_{i}, y_{i}): i = 1, 2, \ldots, m \right\} = 0 \), and \( \max\left\{ \rho_{i}(x_{i}, y_{i}): i = 1, 2, \ldots, m \right\} = 0 \) iff \( \rho_{i}(x_{i}, y_{i}) = 0 \) for every \( i = 1, 2, \ldots, m \). Hence \( \rho(x, y) = 0 \) iff \( x = y \).

		      \( \rho(x, y) = \max\left\{ \rho_{i}(x_{i}, y_{i}): i = 1, 2, \ldots, m \right\} = \max\left\{ \rho_{i}(y_{i}, x_{i}): i = 1, 2, \ldots, m \right\} = \rho(y, x) \).

		      For every \( x, y, z \in X \)
		      \[
			      \rho_{i}(x_{i}, z_{i}) \le \rho_{i}(x_{i}, y_{i}) + \rho_{i}(y_{i}, z_{i}) \le \rho(x, y) + \rho(y, z)
		      \]

		      so \( \rho(x, z) \le \rho(x, y) + \rho(y, z) \).

		      Hence \( \rho \) is a metric on \( X \).
		\item Suppose that \( \rho(x, y) = 0 \). If \( x \in X_{j}, y \in X_{k} \) and \( j \ne k \) then \( \rho_{j}(x, x_{0}) + \rho_{k}(x_{0}, y) = \rho(x, y) = 0 \), which implies that \( \rho_{j}(x, x_{0}) + \rho_{k}(x_{0}, y) = 0 \), so \( x = y = x_{0} \). Otherwise, \( x \in X_{j}, y \in X_{j} \) then \( 0 = \rho(x, y) = \rho_{j}(x, y) \) so \( x = y \).

		      If \( x \in X_{j}, y \in X_{k} \) and \( j \ne k \) then \( \rho(x, y) = \rho_{j}(x, x_{0}) + \rho_{k}(x_{0}, y) = \rho_{k}(y, x_{0}) + \rho_{j}(x_{0}, x) = \rho(y, x) \). Otherwise,  \( x \in X_{j}, y \in X_{j} \) then \( \rho(x, y) = \rho_{j}(x, y) = \rho_{j}(y, x) = \rho(y, x) \).

		      Let \( x, y, z \in X \) then the following cases are exhaustive.
		      \begin{itemize}
			      \item \( x \in X_{j}, y \in X_{k}, z \in X_{\ell} \) where \( j, k, \ell \) are pairwise distinct.
			            \[
				            \rho(x, z) = \rho_{j}(x, x_{0}) + \rho_{\ell}(x_{0}, z) \le \rho_{j}(x, x_{0}) + \rho_{k}(x_{0}, y) + \rho_{k}(y, x_{0}) + \rho_{\ell}(x_{0}, z) = \rho(x, y) + \rho(y, z).
			            \]
			      \item \( x \in X_{j}, y \in X_{k}, z \in X_{j} \) where \( j \ne k \)
			            \[
				            \rho(x, z) = \rho_{j}(x, z) \le \rho_{j}(x, x_{0}) + \rho_{j}(x_{0}, z) \le \rho(x, y) + \rho(y, z).
			            \]
			      \item \( x \in X_{j}, y \in X_{k}, z \in X_{k} \) where \( j \ne k \)
			            \begingroup
			            \allowdisplaybreaks%
			            \begin{align*}
				            \rho(x, z) & = \rho_{j}(x, x_{0}) + \rho_{k}(x_{0}, z)                                             \\
				                       & \le \rho_{j}(x, x_{0}) + \rho_{k}(x_{0}, y) + \rho_{k}(y, x_{0}) + \rho_{k}(x_{0}, z) \\
				                       & = \rho(x, y) + \rho_{k}(y, x_{0}) + \rho_{k}(x_{0}, z)                                \\
				                       & \le \rho(x, y) + \rho_{k}(y, z)                                                       \\
				                       & = \rho(x, y) + \rho(y, z).
			            \end{align*}
			            \endgroup
			      \item \( x \in X_{j}, y \in X_{j}, z \in X_{k} \) where \( j \ne k \)
			            \[
				            \rho(x, y) = \rho_{j}(x, y) \le \rho_{j}(x, y) + \rho_{j}(y, x_{0}) \le \rho_{j}(x, y) + \rho_{j}(y, x_{0}) + \rho_{k}(x_{0}, z) = \rho(x, y) + \rho(y, z).
			            \]
			      \item \( x, y, z \in X_{j} \)
			            \[
				            \rho(x, z) = \rho_{j}(x, z) \le \rho_{j}(x, y) + \rho_{j}(y, z) = \rho(x, y) + \rho(y, z).
			            \]
		      \end{itemize}

		      Hence \( \rho \) is a metric on \( X \).
	\end{enumerate}
\end{proof}

\section{Maps on metric spaces}

\subsection*{Exercises}

\begin{enumerate}[label={(\alph*)}]
	\item Show that for \( j = 0, 1, \ldots \) the maps \( p_{j}: \mathbb{R}^{\omega} \to \mathbb{R} \) defined by the formula \( p_{j}(x^{1}, x^{2}, \ldots) = x^{j} \) are non-expansive.
	\item Prove that inversion and stereographic projection are not uniform homeomorphisms.
	\item Give an example of a bijective map \( f: \mathbb{R} \to \mathbb{R} \) which is uniformly continuous but is not a uniform homeomorphism.
	\item Show that every continuous bijective map of the real line \( \mathbb{R} \) onto itself is a homeomorphism.
\end{enumerate}

\begin{proof}
	\begin{enumerate}[label={(\alph*)}]
		\item For every \( j \) and \( x, y \in \mathbb{R}^{\omega} \)
		      \[
			      \rho(x, y) = \sqrt{\sum_{i=1}^{\infty} {(x^{i} - y^{i})}^{2}} \ge \left\vert x^{j} - y^{j} \right\vert = \rho(p_{j}(x), p_{j}(y))
		      \]

		      so \( p_{j} \) is non-expansive.
		\item Consider the inversion \( i: \mathbb{R}^{m}\smallsetminus\left\{0\right\} \to \mathbb{R}^{m}\smallsetminus\left\{0\right\} \) defined by \( i(x) = \dfrac{x}{\norm{x}^{2}} \). Choose \( \varepsilon = 1 \). For every \( \delta > 0 \), let \( x = \left( a, 0, \ldots, 0 \right) \) and \( x^{\prime} = \left( -a, 0, \ldots, 0 \right) \) in which \( a = \min\left\{ \delta; 2 \right\} \) then
		      \[
			      \norm{ i(x) - i(x^{\prime}) } = \norm{ (1/a, 0, \ldots, 0) - (-1/a, 0, \ldots, 0) } = \dfrac{2}{a} \ge 1 = \varepsilon
		      \]

		      Hence \( i \) is not uniformly continuous, hence not an uniform homeomorphism. In general, every inversion is not an uniform homeomorphism.

		      Every stereographic projection is an inversion hence it is not an uniform homeomorphism.
		\item The function \( f: \mathbb{R} \to \mathbb{R} \) given by \( f(x) = \sqrt[3]{x} \) is bijective.

		      For every \( a, b \in \mathbb{R} \)
		      \begingroup
		      \allowdisplaybreaks%
		      \begin{align*}
			      \abs{a - b}^{3} \le 4\abs{a^{3} - b^{3}} & \iff \abs{a - b}{(a - b)}^{2} \le 4\abs{a - b}{(a^{2} + ab + b^{2})} \\
			                                               & \iff 0 \le \abs{a - b}\cdot (3a^{2} + 6ab + 3b^{2})                  \\
			                                               & \iff 0 \le 3\abs{a - b}{(a + b)}^{2}
		      \end{align*}
		      \endgroup

		      So for every \( \varepsilon > 0 \)
		      \[
			      0 < \abs{x - y} < \dfrac{\varepsilon^{3}}{4} \implies 0 < \abs{\sqrt[3]{x} - \sqrt[3]{y}} \le \sqrt[3]{4\abs{x - y}} < \varepsilon
		      \]

		      which means \( f \) is uniformly continuous.

		      The inverse of \( f \) is \( f^{-1} \) in which \( f^{-1}(x) = x^{3} \). Choose \( \varepsilon = 1 \). For every \( \delta > 0 \), let \( x = \dfrac{1}{\delta^{2}} \) and \( x^{\prime} = \dfrac{1}{\delta^{2}} + \dfrac{\delta}{3} \)
		      \[
			      \abs{x^{3} - {(x^{\prime})}^{3}} = \dfrac{\delta^{3}}{8} + \dfrac{1}{\delta} + 1 > 1 = \varepsilon.
		      \]

		      This implies that \( f^{-1} \) is not uniformly continuous.

		      Thus \( f \) is bijective, uniformly continuous but not an uniform homeomorphism.
		\item Let \( f: \mathbb{R} \to \mathbb{R} \) be a continuous bijective map. Assume that \( f \) is not strictly monotonic then there exist \( a < b < c \) such that \( f(b) \) is not between \( f(a) \) and \( f(c) \).

		      \textbf{Case 1.} \( f(b) < f(a) \) and \( f(b) < f(c) \).

		      If \( f(a) < f(c) \) then \( f(b) < f(a) < f(c) \). According to the intermediate value theorem, there exists \( x \in \openinterval{b, c} \) such that \( f(x) = f(a) \). But \( a \notin \openinterval{b, c} \), which is a contradiction.

		      If \( f(c) < f(a) \) then \( f(b) < f(c) < f(a) \). According to the intermediate value theorem, there exists \( x \in \openinterval{a, b} \) such that \( f(x) = f(c) \). But \( c \notin \openinterval{a, b} \), which is a contradiction.

		      \textbf{Case 2.} \( f(b) > f(a) \) and \( f(b) > f(c) \).

		      If \( f(a) < f(c) \) then \( f(a) < f(c) < f(b) \). According to the intermediate value theorem, there exists \( x \in \openinterval{a, b} \) such that \( f(x) = f(c) \). But \( c \notin \openinterval{a, b} \), which is a contradiction.

		      If \( f(c) < f(a) \) then \( f(c) < f(a) < f(b) \). According to the intermediate value theorem, there exists \( x \in \openinterval{b, c} \) such that \( f(x) = f(a) \). But \( a \notin \openinterval{b, c} \), which is a contradiction.

		      Hence \( f \) is strictly monotonic. Due to this result and the intermediate value, one can show that \( f \) maps open intervals to open intervals.

		      Consider \( x_{0} \in \mathbb{R} \). For every \( \varepsilon > 0 \), \( f(\openinterval{f^{-1}(x_{0}) - \varepsilon, f^{-1}(x_{0}) + \varepsilon}) \) is an open interval. Since \( x_{0} \in f(\openinterval{f^{-1}(x_{0}) - \varepsilon, f^{-1}(x_{0}) + \varepsilon}) \) then there exists \( \delta \) such that
		      \[
			      \openinterval{x_{0} - \delta, x_{0} + \delta} \subset f(\openinterval{f^{-1}(x_{0}) - \varepsilon, f^{-1}(x_{0}) + \varepsilon}).
		      \]

		      Therefore, for every \( \varepsilon > 0 \), there exists \( \delta > 0 \) such that
		      \[
			      0 < \left\vert x - x_{0} \right\vert < \delta \implies \left\vert f^{-1}(x) - f^{-1}(x_{0}) \right\vert < \varepsilon.
		      \]

		      Hence \( f, f^{-1} \) are continuous and bijective, so \( f \) is a homeomorphism. \qedhere
	\end{enumerate}
\end{proof}

\section{Metric concepts}

\subsection*{Exercises}

\begin{enumerate}[label={(\alph*)}]
	\item Give an example of a metric space \( X \), a point \( c \in X \) and a positive real number \( r \) with the property that \( \operatorname{diam} \overline{B}(c; r) < 2r \).
	\item Prove that if every proper subset of a metric space \( X \) is bounded, then the space \( X \) itself is bounded.
\end{enumerate}

\begin{proof}
	\begin{enumerate}[label={(\alph*)}]
		\item Let \( X \) be a discrete metric space with more than one point and \( c \) a point in \( X \).
		      \[
			      \operatorname{diam}\overline{B}(c; 1) = 1 < 2.
		      \]
		\item If the underlying set of \( X \) has less than \( 3 \) elements then \( X \) is bounded. Now assume that \( X \) has more than two points. Let \( a, b \) be two distinct points in \( X \).

		      \( X - \left\{ a \right\} \) and \( X - \left\{ b \right\} \) are bounded so
		      \[
			      \begin{split}
				      \sup\left\{ \rho(x, y) : x, y \ne a \right\} = \delta_{a} \in \mathbb{R} \\
				      \sup\left\{ \rho(x, y) : x, y \ne b \right\} = \delta_{b} \in \mathbb{R}
			      \end{split}
		      \]

		      Let \( \delta = \delta_{a} + \delta_{b} \) and \( x, y \in X \).
		      \begin{itemize}
			      \item If \( x \ne a, y \ne a \) then \( \rho(x, y) \le \delta_{a} < \delta \).
			      \item If \( x = a, y = a  \) then \( \rho(x, y) = 0 < \delta \).
			      \item If \( x = a, y \ne a \) then \( \rho(x, y) \le \rho(x, z) + \rho(z, y) \le \delta_{b} + \delta_{a} = \delta \) in which \( z \) is a point in \( X - \left\{ a, b \right\} \).
			      \item If \( x \ne a, y = a \) then \( \rho(x, y) \le \rho(x, z) + \rho(z, y) \le \delta_{a} + \delta_{b} = \delta \) in which \( z \) is a point in \( X - \left\{ a, b \right\} \).
		      \end{itemize}

		      Hence \( d(x, y) \le \delta \) for every \( x, y \in X \), which means \( X \) is bounded.
	\end{enumerate}
\end{proof}

\section{Convergence and limits}

\subsection*{Exercises}

\begin{enumerate}[label={(\alph*)}]
	\item Prove that if in a metric space \( X \) the only convergent sequences are those that are almost constant, then \( X \) is homeomorphic to a space with discrete metric.
	\item Prove that the map \( f: X \to Y \) is uniformly continuous if and only if for any two sequences \( x_{n}, x^{\prime}_{n} \), where \( n = 1, 2, \ldots \) the equation \( \lim\limits_{n\to\infty} \rho(x_{n}, x^{\prime}_{n}) = 0 \) implies the equation \( \lim\limits_{n\to\infty} \rho(f(x_{n}), f(x^{\prime}_{n})) = 0 \).
	\item Suppose that \( x_{n} = (x_{n}^{1}, x_{n}^{2}, \ldots) \in I^{\omega} \) for \( n = 1, 2, \ldots \) Show that \( \lim\limits_{n\to\infty} x_{n} = x_{0} \) in \( I^{\omega} \) if and only if \( \lim\limits_{n\to\infty} x^{i}_{n} = x^{i}_{0} \) in \( \mathbb{R} \) for \( i = 1, 2, \ldots \) Give an example to show that the analogous proposition is false for the Hilbert space \( \mathbb{R}^{\omega} \).
\end{enumerate}

\begin{proof}
	\begin{enumerate}[label={(\alph*)}]
		\item By using contradiction, we deduce that for each \( x \in X \), there exists \( r > 0 \) such that \( B(x; r) = \left\{ x \right\} \).

		      Let \( Y \) be a discrete metric space in which the underlying sets of \( X \) and \( Y \) are equinumerous, then there is a bijection \( f: X \to Y \).

		      For every \( x_{0} \in X, \varepsilon > 0 \), there exists \( r > 0 \) such that \( B(x_{0}; r) = \left\{ x_{0} \right\} \) and
		      \[
			      x \in B(x_{0}; r) \implies f(B(x_{0}; r)) = \left\{ f(x_{0}) \right\} \subset B(f(x_{0}); \varepsilon)
		      \]

		      so \( f \) is continuous. For every \( y_{0} \in Y, \varepsilon > 0 \)
		      \[
			      y \in B(y_{0}; 1) \implies f^{-1}(B(y_{0}; 1)) = \left\{ f^{-1}(y_{0}) \right\} \subset B(f^{-1}(y_{0}); \varepsilon)
		      \]

		      so \( f^{-1} \) is continuous.

		      Hence \( f \) is a homeomorphism and \( X, Y \) are homeomorphic.
		\item Suppose that \( f \) is uniformly continuous and \( x_{n}, x^{\prime}_{n} \), where \( n = 1, 2, \ldots \) are two sequences such that \( \lim\limits_{n\to\infty} \rho(x_{n}, x^{\prime}_{n}) = 0 \).

		      For every \( \varepsilon > 0 \), there exists \( \delta > 0 \) such that for every \( x, y \in X \)
		      \[
			      \rho(x, y) < \delta \implies \rho(f(x), f(y)) < \varepsilon
		      \]

		      There exists a positive integer \( N \) such that \( \rho(x_{n}, x^{\prime}_{n}) < \delta \) for every \( n > N \) as \( \lim\limits_{n\to\infty} \rho(x_{n}, x^{\prime}_{n}) = 0 \). Hence for every \( n > N \), one has \( \rho(f(x_{n}), f(x^{\prime}_{n})) < \varepsilon \). Thus \( \lim\limits_{n\to\infty} \rho(f(x_{n}), f(x^{\prime}_{n})) = 0 \).

		      Conversely, suppose that for any two sequences \( x_{n}, x^{\prime}_{n} \), where \( n = 1, 2, \ldots \) the equation \( \lim\limits_{n\to\infty} \rho(x_{n}, x^{\prime}_{n}) = 0 \) implies the equation \( \lim\limits_{n\to\infty} \rho(f(x_{n}), f(x^{\prime}_{n})) = 0 \).

		      Assume that \( f \) is not uniformly continuous then there exists \( \varepsilon_{0} > 0 \) such that for every \( \delta > 0 \), there exist \( x, y \) such that \( \rho(x, y) < \delta \) and \( \rho(f(x), f(y)) \ge \varepsilon_{0} \).

		      For every positive integer \( n \), choose \( \delta = \dfrac{1}{n} \), there exist \( x_{n}, x^{\prime}_{n} \) such that \( \rho(x_{n}, x^{\prime}_{n}) < \dfrac{1}{n} \) and \( \rho(f(x_{n}), f(x^{\prime}_{n})) \ge \varepsilon_{0} \). Hence \( \lim\limits_{n\to\infty} \rho(x_{n}, x^{\prime}_{n}) = 0 \) and the numeric sequence \( \rho(f(x_{n}), f(x^{\prime}_{n})) \) is either not convergent or doesn't converge to \( 0 \), which is a contradiction. Thus \( f \) is uniformly continuous.
		\item We show that: For each positive integer \( j \), the projection \( p_{j}: I^{\omega} \to \closedinterval{0, 1/j} \) is uniformly continuous. This is indeed true. For every \( \varepsilon > 0 \), let \( \delta = \varepsilon \), then for every \( x, y \in I^{\omega} \) such that \( \norm{x - y} < \varepsilon \), one has \( \abs{p_{j}(x) - p_{j}(y)} \le \norm{x - y} < \varepsilon \).

		      Suppose that \( \lim\limits_{n\to\infty} x_{n} = x_{0} \) then \( \lim\limits_{n\to\infty} \norm{x_{n} - x_{0}} = 0 \). According to (b) and the uniform continuity of \( p_{i} \), we conclude that \( \lim\limits_{n\to\infty} \norm{p_{i}(x_{n}) - p_{i}(x_{0})} = 0 \), which means \( \lim\limits_{n\to\infty} x^{i}_{n} = x^{i}_{0} \).

		      Conversely, suppose that \( \lim\limits_{n\to\infty} x^{i}_{n} = x^{i}_{0} \) for every \( i \) then  \( \lim\limits_{n\to\infty} \abs{x^{i}_{n} - x^{i}_{0}} = 0 \) for every \( i \).

		      Let \( \varepsilon > 0 \). The series \( \sum_{i=1}^{\infty} \abs{x^{i}_{n} - x^{i}_{0}}^{2} \) converges to \( \norm{x_{n} - x_{0}}^{2} \). There exists a positive integer \( N \) such that \( \sum^{\infty}_{i=N+1} \abs{x^{i}_{n} - x^{i}_{0}}^{2} < \dfrac{\varepsilon}{2} \). For each \( 1 \le i \le N \), there exists a positive integer \( M_{i} \) such that \( \abs{x^{i}_{n} - x^{i}_{0}}^{2} < \dfrac{\varepsilon}{2^{i+1}} \) whenever \( n > N_{i} \). Let \( M = \max\left\{ M_{1}, \ldots, M_{N} \right\} \) then for every \( n > M \)
		      \[
			      \sum_{i=1}^{\infty} \abs{x^{i}_{n} - x^{i}_{0}}^{2} < \sum^{N}_{i=1} \dfrac{\varepsilon}{2^{i+1}} + \dfrac{\varepsilon}{2} < \dfrac{\varepsilon}{2} + \dfrac{\varepsilon}{2} = \varepsilon.
		      \]

		      Hence \( \lim\limits_{n\to\infty} \norm{x_{n} - x_{0}} = 0 \), which means \( \lim\limits_{n\to\infty} x_{n} = x_{0} \).

		      To show that the analogous proposition is false for the Hilbert space \( \mathbb{R}^{\omega} \), consider then sequence \( x_{n} \) in which \( x^{i}_{n} = \delta^{i}_{n} \) (Kronecker delta). However, \( \lim\limits_{n\to\infty} x^{i}_{n} = 0 \in \mathbb{R} \) for each positive integer \( i \) but \( x_{n} \) doesn't converge to \( 0 \in \mathbb{R}^{\omega} \). \qedhere
	\end{enumerate}
\end{proof}

\section{Open and closed sets}

\section{Connected spaces}

\section{Compact spaces}

\section{Complete spaces}

\section{Metric and topological concepts in Euclidean spaces}

\section*{Problems}
