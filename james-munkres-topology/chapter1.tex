\chapter{Set Theory and Logic}

\section{Fundamental Concepts}

% chapter1:section1:exercise1
\begin{exercise}\label{chapter1:section1:exercise1}
    Prove the distributive laws for $\cup$ and $\cap$ and DeMorgan's laws.
\end{exercise}

\begin{proof}
    \begin{align*}
        x \in A\cap (B\cup C) & \Longleftrightarrow (x\in A) \land (x \in B\cup C)                \\
                              & \Longleftrightarrow (x\in A) \land (x\in B \lor x\in C)           \\
                              & \Longleftrightarrow (x\in A\land x\in B)\lor (x\in A\land x\in C) \\
                              & \Longleftrightarrow (x\in A\cap B) \lor (x\in A\cap C)            \\
                              & \Longleftrightarrow x\in (A\cap B)\cup (A\cap C).
    \end{align*}

    So $A\cap (B\cup C) = (A\cap B)\cup (A\cap C)$.
    \begin{align*}
        x \in A\cup (B\cap C) & \Longleftrightarrow (x\in A) \lor (x\in B\cap C)                  \\
                              & \Longleftrightarrow (x\in A) \lor (x\in B \land x\in C)           \\
                              & \Longleftrightarrow (x\in A\lor x\in B) \land (x\in A\lor x\in C) \\
                              & \Longleftrightarrow (x\in A\cup B)\land (x\in A\cup C)            \\
                              & \Longleftrightarrow x\in (A\cup B)\cap (A\cup C).
    \end{align*}

    So $A\cup (B\cap C) = (A\cup B)\cap (A\cup C)$.
    \begin{align*}
        x\in A - (B\cup C) & \Longleftrightarrow (x\in A)\land (x\notin B\cup C)                               \\
                           & \Longleftrightarrow (x\in A)\land (x\notin B \land x\notin C)                     \\
                           & \Longleftrightarrow ((x\in A)\land (x\notin B)) \land ((x\in A)\land (x\notin C)) \\
                           & \Longleftrightarrow (x\in A - B)\land (x\in A - C)                                \\
                           & \Longleftrightarrow x\in (A - B)\cap (A - C),                                     \\
        x\in A - (B\cap C) & \Longleftrightarrow (x\in A)\land (x\notin B\cap C)                               \\
                           & \Longleftrightarrow (x\in A)\land ((x\notin B)\lor (x\notin C))                   \\
                           & \Longleftrightarrow ((x\in A)\land (x\notin B))\lor ((x\in A)\land (x\notin C))   \\
                           & \Longleftrightarrow (x\in A - B)\lor (x\in A - C)                                 \\
                           & \Longleftrightarrow x\in (A - B)\cup (A - C).
    \end{align*}

    So $A - (B\cup C) = (A - B)\cap (A - C)$ and $A - (B\cap C) = (A - B)\cup (A - C)$.
\end{proof}

% chapter1:section1:exercise2
\begin{exercise}\label{chapter1:section1:exercise2}
    Determine which of the following statements are true for all sets $A$, $B$, $C$, and $D$. If a double implication fails, determine whether one or the other of the possible implications holds. If an equality fails, determine whether the statement becomes true if the ``equals'' symbol is replaced by one or the other of the inclusion symbols $\subset$ or $\supset$.
    \begin{enumerate}[label={(\alph*)}]
        \item $A\subset B \land A\subset C \Leftrightarrow A\subset (B\cup C)$.
        \item $A\subset B \lor A\subset C \Leftrightarrow A\subset (B\cup C)$.
        \item $A\subset B \land A\subset C \Leftrightarrow A\subset (B\cap C)$.
        \item $A\subset B \lor A\subset C \Leftrightarrow A\subset (B\cap C)$.
        \item $A - (A - B) = B$.
        \item $A - (B - A) = A - B$.
        \item $A\cap (B - C) = (A\cap B) - (A\cap C)$.
        \item $A\cup (B - C) = (A\cup B) - (A\cup C)$.
        \item $(A\cap B) \cup (A - B) = A$.
        \item $A\subset C\land B\subset D \Rightarrow (A\times B)\subset (C\times D)$
        \item The converse of (j).
        \item The converse of (j) assuming that $A$ and $B$ are nonempty.
        \item $(A\times B)\cup (C\times D) = (A\cup C)\times (B\cup D)$.
        \item $(A\times B)\cap (C\times D) = (A\cap C)\times (B\cap D)$.
        \item $A\times (B - C) = (A\times B) - (A\times C)$.
        \item $(A - B)\times (C - D) = (A\times C - B\times C) - (A\times D)$.
        \item $(A\times B) - (C\times D) = (A - C)\times (B - D)$.
    \end{enumerate}
\end{exercise}

\begin{proof}
    \begin{enumerate}[label={(\alph*)}]
        \item The statement is false.

              $A\subset B \land A\subset C \Rightarrow A\subset (B\cup C)$ is true.

              However, $A\subset B \land A\subset C \Leftarrow A\subset (B\cup C)$ is false.
        \item The statement is false.

              $A\subset B \lor A\subset C \Rightarrow A\subset (B\cup C)$ is true.

              However $A\subset B \lor A\subset C \Leftarrow A\subset (B\cup C)$ is false.
        \item The statement is true.
        \item The statement is false.

              $A\subset B\lor A\subset C \Rightarrow A\subset (B\cap C)$ is false

              $A\subset B\lor A\subset C \Leftarrow A\subset (B\cap C)$ is true.
        \item The statement is false. It should be $A - (A - B) = A\cap B\subset B$.
        \item The statement is false. It should be $A - (B - A) = A \supset A - B$.
        \item The statement is true.
        \item The statement is false. It should be $A\cup (B - C)\supset (A\cup B) - (A\cup C)$.
        \item The statement is true.
        \item The statement is true.
        \item The statement is false.
        \item The statement is true.
        \item The statement is false. It should be $(A\times B)\cup (C\times D) \subset (A\cup C)\times (B\cup D)$.
        \item The statement is true.
        \item The statement is true.
        \item The statement is false. It should be $(A - B)\times (C - D) \supset (A\times C - B\times C) - (A\times D)$.
        \item The statement is false. It should be $(A\times B) - (C\times D) \supset (A - C)\times (B - D)$.
    \end{enumerate}
\end{proof}

% chapter1:section1:exercise3
\begin{exercise}\label{chapter1:section1:exercise3}
    \begin{enumerate}[label={(\alph*)}]
        \item Write the contrapositive and converse of the following statement: ``If $x < 0$, then $x^{2} - x > 0$,{}'' and determine which (if any) of the three statements are true.
        \item Do the same for the statement ``If $x > 0$, then $x^{2} - x > 0$.{}''
    \end{enumerate}
\end{exercise}

\begin{proof}
    \begin{enumerate}[label={(\alph*)}]
        \item The contrapositive statement is ``If $x^{2} - x\leq 0$, then $x\geq 0${.}''

              The converse statement is ``If $x^{2} - x > 0$, then $x < 0${.}''

              The original statement and its contrapositive are true, the converse is false.
        \item The contrapositive statement is ``If $x^{2} - x\leq 0$, then $x\leq 0${.}''

              The converse statement is ``If $x^{2} - x > 0$, then $x > 0${.}''

              None of these statements is true.
    \end{enumerate}
\end{proof}

% chapter1:section1:exercise4
\begin{exercise}\label{chapter1:section1:exercise4}
    Let $A$ and $B$ e sets of real numbers. Write the negation of each of the following statements:
    \begin{enumerate}[label={(\alph*)}]
        \item For every $a\in A$, it is true that $a^{2}\in B$.
        \item For at least one $a\in A$, it is true that $a^{2}\in B$.
        \item For every $a\in A$, it is true that $a^{2}\notin B$.
        \item For at least one $a\notin A$, it is true that $a^{2}\in B$.
    \end{enumerate}
\end{exercise}

\begin{proof}
    \begin{enumerate}[label={(\alph*)}]
        \item There exists $a\in A$ such that $a^{2}\notin B$.
        \item For every $a\in A$, it is true that $a^{2}\notin B$.
        \item There exists $a\in A$ such that $a^{2}\in B$.
        \item For every $a\notin A$, it is true that $a^{2}\notin B$.
    \end{enumerate}
\end{proof}

% chapter1:section1:exercise5
\begin{exercise}\label{chapter1:section1:exercise5}
    Let $\mathscr{A}$ be a nonempty collection of sets. Deternune the truth of each of the following statements and of their converses:
    \begin{enumerate}[label={(\alph*)}]
        \item $x\in \bigcup_{A\in\mathscr{A}}A \Rightarrow x\in A$ for at least one $A\in\mathscr{A}$.
        \item $x\in \bigcup_{A\in\mathscr{A}}A \Rightarrow x\in A$ for every $A\in\mathscr{A}$.
        \item $x\in \bigcap_{A\in\mathscr{A}}A \Rightarrow x\in A$ for at least one $A\in\mathscr{A}$.
        \item $x\in \bigcap_{A\in\mathscr{A}}A \Rightarrow x\in A$ for every $A\in\mathscr{A}$.
    \end{enumerate}
\end{exercise}

\begin{proof}
    \begin{enumerate}[label={(\alph*)}]
        \item The statement is true.
        \item The statement is false.
        \item The statement is true.
        \item The statement is true.
    \end{enumerate}
\end{proof}

% chapter1:section1:exercise6
\begin{exercise}\label{chapter1:section1:exercise6}
    Write the contrapositive of each of the statements of Exercise~\ref{chapter1:section1:exercise5}.
\end{exercise}

\begin{proof}
    \begin{enumerate}[label={(\alph*)}]
        \item $x\notin A$ for every $A\in\mathscr{A}$ $\Rightarrow$ $x\notin \bigcup_{A\in\mathscr{A}} A$.
        \item $x\notin A$ for at least one $A\in\mathscr{A}$ $\Rightarrow$ $x\notin \bigcup_{A\in\mathscr{A}} A$.
        \item $x\notin A$ for every $A\in\mathscr{A}$ $\Rightarrow$ $x\notin \bigcap_{A\in\mathscr{A}} A$.
        \item $x\notin A$ for at least one $A\in\mathscr{A}$ $\Rightarrow$ $x\notin \bigcap_{A\in\mathscr{A}} A$.
    \end{enumerate}
\end{proof}

% chapter1:section1:exercise7
\begin{exercise}\label{chapter1:section1:exercise7}
    Given sets $A$, $B$, and $C$, express each of the following sets in terms of $A$, $B$, and $C$, using the symbols $\cup$, $\cap$, and $-$.
    \begin{align*}
        D & = \{ x \mid x \in A \land (x \in B \lor x \in C) \},        \\
        E & = \{ x \mid (x \in A \land x \in B) \lor x \in C \},        \\
        F & = \{ x \mid x \in A \land (x \in B \Rightarrow x \in C) \}.
    \end{align*}
\end{exercise}

\begin{proof}
    \begin{align*}
        D & = A \cap (B\cup C), \\
        E & = (A\cap B)\cup C,  \\
        F & = A - (B - C).
    \end{align*}
\end{proof}

% chapter1:section1:exercise8
\begin{exercise}\label{chapter1:section1:exercise8}
    If a set $A$ has two elements, show that $\mathscr{P}(A)$ has four elements. How many elements does $\mathscr{P}(A)$ have if $A$ has one element? Three elements? No elements? Why is $\mathscr{P}(A)$ called the power set of $A$?
\end{exercise}

\begin{proof}
    If $A$ has one element, then $\mathscr{P}(A)$ has two elements. If $A$ has three elements, then $\mathscr{P}(A)$ has eight elements. If $A$ has no elements, then $\mathscr{P}(A)$ has one element. $\mathscr{P}(A)$ is called the power set of $A$ because of $A$ has $n$ elements, then $\mathscr{P}(A)$ has $2^{n}$ elements.
\end{proof}

% chapter1:section1:exercise9
\begin{exercise}\label{chapter1:section1:exercise9}
    Formulate and prove De Morgan's laws for arbitrary unions and intersections.
\end{exercise}

\begin{proof}
    De Morgan's laws for arbitrary unions and intersections:
    \begin{align*}
        X - \bigcup_{A\in\mathscr{A}}A & = \bigcap_{A\in\mathscr{A}}(X - A), \\
        X - \bigcap_{A\in\mathscr{A}}A & = \bigcup_{A\in\mathscr{A}}(X - A).
    \end{align*}

    Here is my proof.
    \begin{align*}
        x\in X - \bigcup_{A\in\mathscr{A}}A & \Longleftrightarrow (x\in X) \land \left(x\notin \bigcup_{A\in\mathscr{A}}A\right) \\
                                            & \Longleftrightarrow (x\in X) \land (\forall A\in\mathscr{A}\, x\notin A)           \\
                                            & \Longleftrightarrow \forall A\in\mathscr{A}\, x\in X - A                           \\
                                            & \Longleftrightarrow x\in \bigcap_{A\in\mathscr{A}}(X - A).
    \end{align*}

    Hence $X - \bigcup_{A\in\mathscr{A}}A = \bigcap_{A\in\mathscr{A}}(X - A)$.
    \begin{align*}
        x\in X - \bigcap_{A\in\mathscr{A}}A & \Longleftrightarrow (x\in X) \land \left(x\notin \bigcap_{A\in\mathscr{A}}A\right) \\
                                            & \Longleftrightarrow (x\in X) \land (\exists A\in\mathscr{A}\, x\notin A)           \\
                                            & \Longleftrightarrow \exists A\in\mathscr{A}\, x\in X - A                           \\
                                            & \Longleftrightarrow x\in \bigcup_{A\in\mathscr{A}}(X - A).
    \end{align*}

    Hence $X - \bigcap_{A\in\mathscr{A}}A = \bigcup_{A\in\mathscr{A}}(X - A)$.
\end{proof}

% chapter1:section1:exercise10
\begin{exercise}\label{chapter1:section1:exercise10}
    Let $\mathbb{R}$ denote the set of real numbers. For each of the following subsets of $\mathbb{R}\times\mathbb{R}$, determine whether it is equal to the Cartesian product of two subsets of $\mathbb{R}$.
    \begin{enumerate}[label={(\alph*)}]
        \item $\{ (x, y) \mid x\in\mathbb{Z} \}$.
        \item $\{ (x, y) \mid 0 < y \leq 1 \}$.
        \item $\{ (x, y) \mid y > x \}$.
        \item $\{ (x, y) \mid x\notin \mathbb{Z} \land y\in\mathbb{Z} \}$.
        \item $\{ (x, y) \mid x^{2} + y^{2} < 1 \}$.
    \end{enumerate}
\end{exercise}

\begin{proof}
    \begin{enumerate}[label={(\alph*)}]
        \item This set is $\mathbb{Z}\times\mathbb{R}$.
        \item This set is $\mathbb{R}\times \halfopenleft{0, 1}$.
        \item Assume there exist $A, B\subset\mathbb{R}$ such that $\{ (x, y) \mid y > x \} = A\times B$. Due to the definition of this set, $A = \mathbb{R}$. $B$ is not empty because the given set is not empty. Let $a\in A$, then $(a, a)\in \{ (x, y) \mid y > x \}$, which is a contradiction to the definition of the given set.

              Hence this set is not equal to the cartesian product of two subsets of $\mathbb{R}$.
        \item This set is $(\mathbb{R} - \mathbb{Z})\times \mathbb{Z}$.
        \item Assume there exist $A, B\subset\mathbb{R}$ such that $\{ (x, y) \mid x^{2} + y^{2} < 1 \} = A\times B$. Due to the definition of this set, $A\subset \openinterval{0, 1}$ and $B\subset \openinterval{0, 1}$. $\left(\frac{1}{\sqrt{2}}, \frac{1}{\sqrt{3}}\right)$ and $\left(\frac{1}{\sqrt{3}}, \frac{1}{\sqrt{2}}\right)$ are elements of the given set, so $\frac{1}{\sqrt{2}}$ is a common element of $A$ and $B$. $\left(\frac{1}{\sqrt{2}}, \frac{1}{\sqrt{2}}\right)\in A\times B$ in accordance with the definition of cartesian product, but this contradicts the definition of the given set.

              Hence this set is not equal to the cartesian product of two subsets of $\mathbb{R}$.
    \end{enumerate}
\end{proof}

\section{Functions}

% chapter1:section2:exercise1
\begin{exercise}\label{chapter1:section2:exercise1}
    Let $f: A\to B$. Let $A_{0}\subset A$ and $B_{0}\subset B$.
    \begin{enumerate}[label={(\alph*)}]
        \item Show that $A_{0}\subset f^{-1}(f(A_{0}))$ and that equality holds if $f$ is injective.
        \item Show that $f(f^{-1}(B_{0}))\subset B_{0}$ and that equality holds if $f$ is surjective.
    \end{enumerate}
\end{exercise}

\begin{proof}
    \begin{enumerate}[label={(\alph*)}]
        \item If $x\in A_{0}$ then $f(x)\in f(A_{0})$. By the definition of preimage, $x\in f^{-1}(f(A_{0}))$. Because we are using an arbitrary $x\in A_{0}$, we conclude that $A_{0}\subset f^{-1}(f(A_{0}))$.

              Suppose $f$ is injective and $x\in f^{-1}(f(A_{0}))$, so $f(x)\in f(A_{0})$. By the definition of the image set, there exists $x'\in A_{0}$ such that $f(x) = f(x')$. However, $f$ is injective, so $x = x'$. Therefore $x\in A_{0}$. Hence $A_{0}\subset f^{-1}(f(A_{0}))$ and $f^{-1}(f(A_{0}))\subset A_{0}$. Thus $A_{0} = f^{-1}(f(A_{0}))$ if $f$ is injective.
        \item If $y\in f(f^{-1}(B_{0}))$ then $f^{-1}(y) \subset f^{-1}(B_{0})$. By the definition of preimage, it follows that $y\in B_{0}$.

              Suppose $f$ is surjective and $y\in B_{0}$. Because $f$ is surjective, there exists $x\in A$ such that $f(x) = y$. $f^{-1}(y)$ is the preimage of $y$, then $f^{-1}(y)\subset f^{-1}(B_{0})$. So $y\in f(f^{-1}(y))\subset B_{0}$. Hence $y\in f(f^{-1}(B_{0}))$. Thus $f(f^{-1}(B_{0})) = B_{0}$ if $f$ is surjective.
    \end{enumerate}
\end{proof}

% chapter1:section2:exercise2
\begin{exercise}\label{chapter1:section2:exercise2}
    Let $f: A\to B$ and let $A_{i}\subset A$ and $B_{i}\subset B$ for $i = 0$ and $i = 1$. Show that $f^{-1}$ preserves inclusions, unions, intersections, and differences of sets and $f$ preserves inclusions and unions only:
    \begin{enumerate}[label={(\alph*)}]
        \item $B_{0}\subset B_{1}\Rightarrow f^{-1}(B_{0})\subset f^{-1}(B_{1})$.
        \item $f^{-1}(B_{0}\cup B_{1}) = f^{-1}(B_{0}) \cup f^{-1}(B_{1})$.
        \item $f^{-1}(B_{0}\cap B_{1}) = f^{-1}(B_{0}) \cap f^{-1}(B_{1})$.
        \item $f^{-1}(B_{0} - B_{1}) = f^{-1}(B_{0}) - f^{-1}(B_{1})$.
        \item $A_{0}\subset A_{1} \Rightarrow f(A_{0}) \subset f(A_{1})$.
        \item $f(A_{0}\cup A_{1}) = f(A_{0})\cup f(A_{1})$.
        \item $f(A_{0}\cap A_{1}) \subset f(A_{0}) \cap f(A_{1})$; show that equality holds if $f$ is injective.
        \item $f(A_{0} - A_{1}) \supset f(A_{0}) - f(A_{1})$; show that equality holds if $f$ is injective.
    \end{enumerate}
\end{exercise}

\begin{proof}
    \begin{enumerate}[label={(\alph*)}]
        \item Suppose $B_{0}\subset B_{1}$ and $x\in f^{-1}(B_{0})$, then $f(x)\in B_{0}\subset B_{1}$. Therefore $x\in f^{-1}(B_{1})$. Hence $x\in f^{-1}(B_{0})\Rightarrow x\in f^{-1}(B_{1})$, which implies $f^{-1}(B_{0})\subset f^{-1}(B_{1})$.
        \item \begin{align*}
                  x\in f^{-1}(B_{0}\cup B_{1}) & \Longleftrightarrow f(x) \in B_{0}\cup B_{1}                   \\
                                               & \Longleftrightarrow f(x)\in B_{0} \lor f(x)\in B_{1}           \\
                                               & \Longleftrightarrow x\in f^{-1}(B_{0}) \lor x\in f^{-1}(B_{1}) \\
                                               & \Longleftrightarrow x\in f^{-1}(B_{0})\cup f^{-1}(B_{1}).
              \end{align*}

              Hence $f^{-1}(B_{0}\cup B_{1}) = f^{-1}(B_{0}) \cup f^{-1}(B_{1})$.
        \item \begin{align*}
                  x\in f^{-1}(B_{0}\cap B_{1}) & \Longleftrightarrow f(x) \in B_{0}\cap B_{1}                    \\
                                               & \Longleftrightarrow f(x)\in B_{0} \land f(x)\in B_{1}           \\
                                               & \Longleftrightarrow x\in f^{-1}(B_{0}) \land x\in f^{-1}(B_{1}) \\
                                               & \Longleftrightarrow x\in f^{-1}(B_{0})\cap f^{-1}(B_{1}).
              \end{align*}

              Hence $f^{-1}(B_{0}\cap B_{1}) = f^{-1}(B_{0}) \cap f^{-1}(B_{1})$.
        \item \begin{align*}
                  x\in f^{-1}(B_{0} - B_{1}) & \Longleftrightarrow f(x) \in B_{0} - B_{1}                           \\
                                             & \Longleftrightarrow f(x) \in B_{0} \land f(x)\notin B_{1}            \\
                                             & \Longleftrightarrow x \in f^{-1}(B_{0}) \land x \notin f^{-1}(B_{1}) \\
                                             & \Longleftrightarrow x\in f^{-1}(B_{0}) - f^{-1}(B_{1}).
              \end{align*}

              Hence $f^{-1}(B_{0} - B_{1}) = f^{-1}(B_{0}) - f^{-1}(B_{1})$.
        \item Suppose $A_{0}\subset A_{1}$ and $y\in f(A_{0})$. By the definition of image set, there is $x\in A_{0}$ such that $f(x) = y$. Because $x\in A_{0}\subset A_{1}$ so $y = f(x)\in f(A_{1})$. Hence $y\in f(A_{0})\Rightarrow y\in f(A_{1})$. Thus $A_{0}\subset A_{1}$ implies $f(A_{0})\subset f(A_{1})$.
        \item \begin{align*}
                  y\in f(A_{0}\cup A_{1}) & \Longleftrightarrow \exists x\in A_{0}\cup A_{1}: f(x) = y \\
                                          & \Longleftrightarrow y\in f(A_{0}) \lor y\in f(A_{1})       \\
                                          & \Longleftrightarrow y\in f(A_{0}) \lor f(A_{1}).
              \end{align*}

              Hence $f(A_{0}\cup A_{1}) = f(A_{0})\cup f(A_{1})$.
        \item Suppose $y\in f(A_{0}\cap A_{1})$, then there exists $x\in A_{0}\cap A_{1}$ such that $f(x) = y$. Because $x\in A_{0}$ and $x\in A_{1}$, it follows that $f(x)\in f(A_{0})$ and $f(x)\in f(A_{1})$, which implies $y = f(x)\in f(A_{1})\cap f(A_{2})$. Hence $f(A_{1}\cap A_{2})\subset f(A_{1})\cap f(A_{2})$.

              Suppose $f$ is injective. Let $y\in f(A_{1})\cap f(A_{2})$ then $y\in f(A_{1})$ and $y\in f(A_{2})$. According to the definition of image sets, there exist $x_{1}\in A_{1}$ and $x_{2}\in A_{2}$ such that $f(x_{1}) = y$ and $f(x_{2}) = y$. Because $f$ is injective, $x_{1} = x_{2}\in A_{1}\cap A_{2}$, so $y\in f(A_{1}\cap A_{2})$. Thus $f(A_{1}\cap A_{2}) = f(A_{1})\cap f(A_{2})$ if $f$ is injective.
        \item Suppose $y\in f(A_{0}) - f(A_{1})$, then $y\in f(A_{0})$ and $y\notin f(A_{1})$. There exists $x\in A_{0}$ such that $f(x) = y$, so $x\in A_{0}$. By the definition of image sets, $x\notin A_{1}$. Therefore $x\in A_{0} - A_{1}$, which implies $y = f(x)\in f(A_{0} - A_{1})$. Hence $f(A_{0} - A_{1})\supset f(A_{0}) - f(A_{1})$.

              Suppose $f$ is injective. Let $y\in f(A_{1} - A_{2})$, then there exists $x\in A_{1} - A_{2}$ such that $f(x) = y$. $x\in A_{1} - A_{2}$ implies $x\in A_{1}$ and $x\notin A_{2}$, so $y = f(x)\in f(A_{1})$. Assume $f(x)\in f(A_{2})$ then there is $x'\in A_{2}$ such that $f(x) = f(x')$. $x\ne x'$ because $x\notin A_{2}$ and $x'\in A_{2}$, this contradicts the injectivity of $f$. So $f(x)\notin f(A_{2})$. Hence $f(A_{1} - A_{2}) = f(A_{1}) - f(A_{2})$ if $f$ is injective.
    \end{enumerate}
\end{proof}

% chapter1:section2:exercise3
\begin{exercise}\label{chapter1:section2:exercise3}
    Show that (b), (c), (f), and (g) of Exercise~\ref{chapter1:section2:exercise2} hold for arbitrary unions and intersections.
\end{exercise}

\begin{proof}
    \begin{align*}
        x \in f^{-1}\left(\bigcup_{B\in\mathscr{B}}B\right) & \Longleftrightarrow f(x) \in \bigcup_{B\in\mathscr{B}}B      \\
                                                            & \Longleftrightarrow \exists B\in\mathscr{B}: f(x)\in B       \\
                                                            & \Longleftrightarrow x\in \bigcup_{B\in\mathscr{B}}f^{-1}(B).
    \end{align*}

    Hence $f^{-1}\left(\bigcup_{B\in\mathscr{B}}B\right) = \bigcup_{B\in\mathscr{B}}f^{-1}(B)$.
    \begin{align*}
        x \in f^{-1}\left(\bigcap_{B\in\mathscr{B}}B\right) & \Longleftrightarrow f(x) \in \bigcap_{B\in\mathscr{B}}B      \\
                                                            & \Longleftrightarrow \forall B\in\mathscr{B}: f(x)\in B       \\
                                                            & \Longleftrightarrow x\in \bigcap_{B\in\mathscr{B}}f^{-1}(B).
    \end{align*}

    Hence $f^{-1}\left(\bigcap_{B\in\mathscr{B}}B\right) = \bigcap_{B\in\mathscr{B}}f^{-1}(B)$.
    \begin{align*}
        y\in f\left(\bigcup_{A\in\mathscr{A}}A\right) & \Longleftrightarrow \exists x\in \bigcup_{A\in\mathscr{A}}A : f(x) = y \\
                                                      & \Longleftrightarrow \exists A\in\mathscr{A}: f(x) = y                  \\
                                                      & \Longleftrightarrow \exists A\in\mathscr{A}: y\in f(A)                 \\
                                                      & \Longleftrightarrow \bigcup_{A\in\mathscr{A}}f(A).
    \end{align*}

    Hence $f\left(\bigcup_{A\in\mathscr{A}}A\right) = \bigcup_{A\in\mathscr{A}}f(A)$.

    \begin{align*}
        y\in f\left(\bigcap_{A\in\mathscr{A}}A\right) & \implies \exists x\in \bigcap_{A\in\mathscr{A}}A : f(x) = y \\
                                                      & \implies \forall A\in\mathscr{A}: y\in f(A)                 \\
                                                      & \implies y\in \bigcap_{A\in\mathscr{A}}f(A).
    \end{align*}

    Hence $f\left(\bigcap_{A\in\mathscr{A}}A\right) \subset \bigcap_{A\in\mathscr{A}}f(A)$.

    Suppose $f$ is injective, let $y\in \bigcap_{A\in\mathscr{A}}f(A)$, then $y\in f(A)$ for every $A\in\mathscr{A}$. For each $A\in\mathscr{A}$, there is $x_{A}\in A$ such that $f(x_{A}) = y$. Because $f$ is injective, all $x_{A}$ are equal, so $x_{A}\in \bigcap_{A\in\mathscr{A}}A$, which implies $y\in f\left(\bigcap_{A\in\mathscr{A}}A\right)$. Thus $f\left(\bigcap_{A\in\mathscr{A}}A\right) = \bigcap_{A\in\mathscr{A}}f(A)$ if $f$ is injective.
\end{proof}

% chapter1:section2:exercise4
\begin{exercise}\label{chapter1:section2:exercise4}
    Let $f: A\to B$ and $g: B\to C$.
    \begin{enumerate}[label={(\alph*)}]
        \item If $C_{0}\subset C$, show that ${(g\circ f)}^{-1}(C_{0}) = f^{-1}(g^{-1}(C_{0}))$.
        \item If $f$ and $g$ are injective, show that $g\circ f$ is injective.
        \item If $g\circ f$ is injective, what can you say about injectivity of $f$ and $g$?
        \item If $f$ and $g$ are surjective, show that $g\circ f$ is surjective.
        \item If $g\circ f$ is surjective, what can you say about surjectivity of $f$ and $g$?
        \item Summarize your answer to {(b)}-{(e)} in the form of a theorem.
    \end{enumerate}
\end{exercise}

\begin{proof}
    \begin{enumerate}[label={(\alph*)}]
        \item \begin{align*}
                  x\in {(g\circ f)}^{-1}(C_{0}) & \Longleftrightarrow g(f(x))\in C_{0}            \\
                                                & \Longleftrightarrow f(x)\in g^{-1}(C_{0})       \\
                                                & \Longleftrightarrow x\in f^{-1}(g^{-1}(C_{0})).
              \end{align*}

              Hence ${(g\circ f)}^{-1}(C_{0}) = f^{-1}(g^{-1}(C_{0}))$.
        \item Suppose $f$ and $g$ are injective and $(g\circ f)(x_{1}) = (g\circ f)(x_{2})$. Because $g$ is injective, $f(x_{1}) = f(x_{2})$. Because $f$ is injective, $x_{1} = x_{2}$. Hence $g\circ f$ is injective if $f$ and $g$ are injective.
        \item If $g\circ f$ is injective, then $f$ is injective, but we cannot conclude that $g$ is injective.
        \item Suppose $f$ and $g$ are surjective. Let $c\in C_{0}$. Because $g$ is surjective, there is $b\in B$ such that $g(b) = c$. Because $f$ is surjective, there is $a\in A$ such that $f(a) = b$. Hence $(g\circ f)(a) = g(f(a)) = g(b) = c$. So $g\circ f$ is surjective if $f$ and $g$ are surjective.
        \item If $g\circ f$ is surjective, then $g$ is surjective, but we cannot conclude that $f$ is surjective.
        \item Theorem:

              If $f$ and $g$ are injective then $g\circ f$ is injective. If $g\circ f$ is injective then $f$ is injective.

              If $f$ and $g$ are surjective then $g\circ f$ is surjective. If $g\circ f$ is surjective then $g$ is surjective.
    \end{enumerate}
\end{proof}

% chapter1:section2:exercise5
\begin{exercise}\label{chapter1:section2:exercise5}
    In general, let us denote the identity function for a set $C$ by $i_{C}$. That is, define $i_{C}: C \to C$ to be the function given by the rule $i_{C}(x) = x$ for all $x \in C$. Given $f: A \to B$, we say that a function $g: B \to A$ is a left inverse for $f$ if $g\circ f = i_{A}$; and we say that $h: B \to A$ is a right inverse for $f$ if $f\circ h = i_{B}$.
    \begin{enumerate}[label={(\alph*)}]
        \item Show that if $f$ has a left inverse, $f$ is injective; and if $f$ has a right inverse, $f$ is surjective.
        \item Give an example of a function that has a left inverse but no right inverse.
        \item Give an example of a function that has a right inverse but no left inverse.
        \item Can a function have more than one left inverse? More than one right inverse?
        \item Show that if $f$ has both a left inverse $g$ and a right inverse $h$, then $f$ is bijective and $g = h = f^{-1}$.
    \end{enumerate}
\end{exercise}

\begin{proof}
    \begin{enumerate}[label={(\alph*)}]
        \item If $f$ has a left inverse, then there is $g: B\to A$ such that $g\circ f = i_{A}$. By Exercise~\ref{chapter1:section2:exercise4}, $f$ is injective.

              If $f$ has a right inverse, then there is $h: B\to A$ such that $f\circ h = i_{B}$. By Exercise~\ref{chapter1:section2:exercise4}, $f$ is surjective.
        \item Let $A = \mathbb{Z}$ and $B = \mathbb{R}$. $f: A\to B$ is defined by $f(x) = x$ for every $x\in A$. Let $g: B\to A$ defined by $g(x) = \floor{x}$, then $g\circ f = i_{A}$. Assume $f$ has a right inverse $h$, then
              \[
                  \sqrt{2} = i_{B}(\sqrt{2}) = f(h(\sqrt{2})) = h(\sqrt{2}).
              \]

              However, $h: B\to A$ so $h(\sqrt{2})$ must be an integer, which is a contradiction. Therefore $f$ has a left inverse but has no right inverse.
        \item Let $A = \mathbb{R}$ and $B = \mathbb{Z}$. $f: A\to B$ is defined by $f(x) = \floor{x}$ for every $x\in A$. Let $h: B\to A$ defined by $h(x) = x$, then $f\circ h = i_{B}$. Assume $f$ has a left inverse $g$, then
              \[
                  \sqrt{2} = i_{A}(\sqrt{2}) = g(f(\sqrt{2})) = g(1) = g(f(1)) = 1
              \]

              which is a contradiction. Therefore $f$ has a right inverse but has no left inverse.
        \item Yes, a function can have more than one left inverse. For example:
              \begin{align*}
                  f:     & \quad \mathbb{Z} \to \mathbb{R}                                    \\
                         & \quad x \mapsto x                                                  \\
                  g_{1}: & \quad\mathbb{R} \to \mathbb{Z}                                     \\
                         & \quad x\mapsto \floor{x}                                           \\
                  g_{2}: & \quad\mathbb{R} \to \mathbb{Z}                                     \\
                         & \quad x \mapsto \begin{cases}
                                               x        & \text{if $x\in\mathbb{Z}$} \\
                                               \ceil{x} & \text{otherwise}
                                           \end{cases}
              \end{align*}

              then $g_{1}\circ f = g_{2}\circ f = i_{\mathbb{Z}}$ but $g_{1}\ne g_{2}$.

              Also, a function can have more than one right inverse. For example:
              \begin{align*}
                  f:     & \quad \mathbb{R} \to \mathbb{Z} \\
                         & \quad x \mapsto \floor{x}       \\
                  h_{1}: & \quad\mathbb{Z} \to \mathbb{R}  \\
                         & \quad x\mapsto x                \\
                  h_{2}: & \quad\mathbb{Z} \to \mathbb{R}  \\
                         & \quad x \mapsto x + \frac{1}{2}
              \end{align*}

              then $f\circ h_{1} = f\circ h_{2} = i_{B}$ but $h_{1}\ne h_{2}$.
        \item Suppose $g$ is a left inverse of $f$ and $h$ is a right inverse of $f$, then $g\circ f = i_{A}$ and $f\circ h = i_{B}$. By Exercise~\ref{chapter1:section2:exercise4}, $f$ is injective and surjective, hence bijective.
              \begin{align*}
                  h & = i_{A}\circ h = (g\circ f)\circ h = g\circ (f\circ h) = g\circ i_{B} = g,                     \\
                  h & = i_{A}\circ h = (f^{-1}\circ f)\circ h = f^{-1}\circ (f\circ h) = f^{-1}\circ i_{B} = f^{-1}.
              \end{align*}

              Hence $g = h = f^{-1}$.
    \end{enumerate}
\end{proof}

% chapter1:section2:exercise6
\begin{exercise}\label{chapter1:section2:exercise6}
    Let $f: \mathbb{R}\to\mathbb{R}$ be the function $f(x) = x^{3} - x$. By restricting the domain and range of $f$ appropriately, obtain from $f$ a bijective function $g$. Draw the graphs of $g$ and $g^{-1}$.
\end{exercise}

\begin{proof}
    Let $A = \closedinterval{1, 2}$. If $f(x_{1}) = f(x_{2})$ for some $x_{1}, x_{2}\in A$, then
    \[
        0 = x_{1}^{3} - x_{1} - (x_{2}^{3} - x_{2}) = (x_{1} - x_{2})(x_{1}^{2} + x_{1}x_{2} + x_{2}^{2} - 1)
    \]

    which implies $x_{1} = x_{2}$ because $x_{1}^{2} + x_{1}x_{2} + x_{2}^{2} - 1 > 0$ for all $x_{1}, x_{2}\in A$. $g: \closedinterval{1, 1}\to \closedinterval{0, 7}$ defined by $g(x) = f\vert_{\closedinterval{0,1}}(x)$ is bijective.
    \begin{figure}[htp]
        \centering
        \begin{tikzpicture}
            \begin{axis}[
                    unit vector ratio*=1 1 1,
                    axis lines=center,
                    xmin=-0.5,
                    xmax = 7.5,
                    ymin=-0.5,
                    ymax = 7.5,
                    ylabel=$y$,
                    xlabel=$x$,
                ]
                \addplot [domain=1:2,smooth, blue] {x*x*x - x}
                node [pos=0.9, above left] {$g$};
                \addplot [domain=1:2,smooth, red] (x*x*x - x, x)
                node [pos=0.9, below right] {$g^{-1}$};
            \end{axis}
        \end{tikzpicture}
        \caption*{Graphs of $g$ and $g^{-1}$.}
    \end{figure}
\end{proof}

\section{Relations}

\subsection*{Equivalence Relations}

% chapter1:section3:exercise1
\begin{exercise}\label{chapter1:section3:exercise1}
    Define two points $(x_{0}, y_{0})$ and $(x_{1}, y_{1})$ of the plane to be equivalent if $y_{0} - x_{0}^{2} = y_{1} - x_{1}^{2}$. Check that this is an equivalence relation and describe the equivalence classes.
\end{exercise}

\begin{proof}
    For every point $(x_{0}, y_{0})$, we have $y_{0} - x_{0}^{2} = y_{0} - x_{0}^{2}$ so the relation is reflexive.

    $y_{0} - x_{0}^{2} = y_{1} - x_{1}^{2}$ if and only if $y_{1} - x_{1}^{2} = y_{0} - x_{0}^{2}$, so the relation is symmetric.

    If $y_{0} - x_{0}^{2} = y_{1} - x_{1}^{2}$ and $y_{1} - x_{1}^{2} = y_{2} - x_{2}^{2}$ then $y_{0} - x_{0}^{2} = y_{2} - x_{2}^{2}$. So the relation is transitive.

    Hence this is an equivalence relation. The equivalence classes of this equivalence relation are the following sets:
    \[
        \{ (x, y) \mid y - x^{2} = t \}
    \]

    where $t$ is a real number.
\end{proof}

% chapter1:section3:exercise2
\begin{exercise}\label{chapter1:section3:exercise2}
    Let $C$ be a relation on a set $A$. If $A_{0} \subset A$, define the restriction of $C$ to $A_{0}$ to be the relation $C \cap (A_{0} \times A_{0})$. Show that the restriction of an equivalence relation is an equivalence relation.
\end{exercise}

\begin{proof}
    Denote the restriction of $C$ to $A_{0}$ by $C_{0}$.

    Let $a\in A_{0}$, then $a\in A$ so $aCa$. Because $a\in A_{0}$, we have $aC_{0}a$. So $C_{0}$ is reflexive.

    Let $a, b\in A_{0}$ such that $aC_{0}b$, then $aCb$ so $bCa$. Because $a, b\in A_{0}$, we have $bC_{0}a$. So $C_{0}$ is symmetric.

    If $aC_{0}b$ and $bC_{0}c$ then $aCb$ and $bCc$. Therefore $aCc$. Because $a, b, c\in A_{0}$, we have $aC_{0}c$. So $C_{0}$ is transitive.

    Hence $C_{0}$ is an equivalence relation. Thus the restriction of an equivalence relation is an equivalence relation.
\end{proof}

% chapter1:section3:exercise3
\begin{exercise}\label{chapter1:section3:exercise3}
    Here is a ``proof'' that every relation $C$ that is both symmetric and transitive is also reflexive: ``Since $C$ is symmetric, $aCb$ implies $bCa$. Since $C$ is transitive, $aCb$ and $bCa$ together imply $aCa$, as desired{.}'' Find the flaw in this argument.
\end{exercise}

\begin{proof}
    The flaw is: From a given an element $a$, there does not necessarily exist an element $b$ such that $aCb$.
\end{proof}

% chapter1:section3:exercise4
\begin{exercise}\label{chapter1:section3:exercise4}
    Let $f: A\to B$ be a surjective function. Let us define a relation on $A$ by setting $a_{0}\sim a_{1}$ if
    \[
        f(a_{0}) = f(a_{1})
    \]

    \begin{enumerate}[label={(\alph*)}]
        \item Show that this is an equivalence relation.
        \item Let $A^{*}$ be the set of equivalence classes. Show there is a bijective correspondence of $A^{*}$ with $B$.
    \end{enumerate}
\end{exercise}

\begin{proof}
    \begin{enumerate}[label={(\alph*)}]
        \item For every $a\in A$, $f(a) = f(a)$, so $\sim$ is reflexive.

              If $f(a_{0}) = f(a_{1})$ then $f(a_{1}) = f(a_{0})$, so $\sim$ is symmetric.

              If $f(a_{0}) = f(a_{1})$ and $f(a_{1}) = f(a_{2})$ then $f(a_{0}) = f(a_{2})$, so $\sim$ is transitive.

              Hence this is an equivalence relation.
        \item For every element $a$ of $A$, we denote by $E_{a}$ the equivalence class containing $a$. We define the function $\tau: A^{*}\to B$ as follows:
              \[
                  \tau(E_{a}) = f(a).
              \]

              This function is well-defined because for every $x\in E_{a}$, $\tau(E_{x}) = f(x) = f(a)$, by the definition of $\sim$.

              If $\tau(E_{a_{1}}) = \tau(E_{a_{2}})$ then $f(a_{1}) = f(a_{2})$, which implies $a_{1}\sim a_{2}$, which implies $E_{a_{1}} = E_{a_{2}}$. Therefore $\tau$ is injective.

              Let $b$ be an element of $B$. Because $f$ is surjective, there exists $a\in A$ such that $f(a) = b$. So $\tau(E_{a}) = f(a) = b$, which implies $\tau$ is surjective.

              Hence $\tau$ is a bijective correspondence of $A^{*}$ with $B$.
    \end{enumerate}
\end{proof}


% chapter1:section3:exercise5
\begin{exercise}\label{chapter1:section3:exercise5}
    Let $S$ and $S'$ be the following subsets of the plane.
    \begin{align*}
        S  & = \{ (x, y) \mid y = x + 1 \land 0 < x < 2 \}, \\
        S' & = \{ (x, y) \mid y - x \in\mathbb{Z} \}
    \end{align*}

    \begin{enumerate}[label={(\alph*)}]
        \item Show that $S'$ is an equivalence relation on the real line and $S'\supset S$. Describe the equivalence classes of $S'$.
        \item Show that given any collection of equivalence relations on a set $A$, their intersection is an equivalence relation on $A$.
        \item Describe the equivalence relation $T$ on the real line that is the intersection of all equivalence relations on the real line that contain $S$. Describe the equivalence classes of $T$.
    \end{enumerate}
\end{exercise}

\begin{proof}
    \begin{enumerate}[label={(\alph*)}]
        \item For every $x\in\mathbb{Z}$, $x - x = 0\in\mathbb{Z}$ so $S'$ is reflexive.

              If $y - x\in\mathbb{Z}$ then $x - y = -(y - x)\in\mathbb{Z}$, so $S'$ is symmetric.

              If $y - x\in\mathbb{Z}$ and $z - y\in\mathbb{Z}$ then $z - x = (z - y) + (y - x)\in\mathbb{Z}$, so $S'$ is transitive.

              Hence $S'$ is an equivalence relation.

              If $(x, y)\in S$, then $y - x = 1\in\mathbb{Z}$, so $(x, y)\in S'$. Therefore $S\subset S'$.

              The equivalence classes of $S'$ are the sets $\{ x + n \mid n\in\mathbb{Z} \}$ where $x\in \halfopenright{0, 1}$. In other words, an equivalence class of $S'$ is the set of real numbers $x$ such that $x - \floor{x}$ are equal.
        \item Let $\mathscr{C}$ be a collection of equivalence relations on the set $A$. Let
              \[
                  \sim = \bigcap_{C\in\mathscr{C}}C.
              \]

              For every $a\in A$, for every $C\in\mathscr{C}$, $aCa$, so $a\sim a$. Hence $\sim$ is reflexive.

              If $a_{0}\sim a_{1}$, then $a_{0}Ca_{1}$ for every $C\in\mathscr{C}$, so $a_{1}Ca_{0}$ for every $C\in\mathscr{C}$. Therefore $a_{1}\sim a_{0}$, which implies $\sim$ is symmetric.

              If $a_{0}\sim a_{1}$ and $a_{1}\sim a_{2}$ then $a_{0}Ca_{1}$ and $a_{1}Ca_{2}$ for every $C\in\mathscr{C}$. Therefore $a_{0}Ca_{2}$ for every $C\in\mathscr{C}$, hence $a_{0}\sim a_{2}$. So $\sim$ is transitive.

              So $\sim$ is an equivalence relation. Thus the intersection of a collection of equivalence relations is an equivalence relation.
        \item $T$ is the smallest equivalence relation on the real line containing $S$. Because $S$ is not an equivalence relation and $S'$ is an equivalence relation containing $S$, then $S\subset T \subset S'$. So if $(x, y)\in T$, then $y - x\in\mathbb{Z}$.

              $T$ is symmetric and reflexive, so
              \[
                  T\supset \{ (x, y) \mid y = x + 1 \land 0 < x < 2 \} \cup \{ (x, y) \mid y = x - 1 \land 1 < x < 3 \} \cup \{ (x, x) \mid x\in\mathbb{R} \}.
              \]

              Because $T$ is transitive, it follows that
              \[
                  (x, x+1), (x+1, x+2), (x, x+2)\in T\quad \forall x\in\openinterval{0, 1}
              \]

              The following partition of $\mathbb{R}$
              \[
                  {\{ x, x + 1, x + 2 \}}_{x\in\openinterval{0, 1}} \cup \{ 1, 2 \} \cup {\{ x \}}_{x\leq 0\lor x\geq 3}
              \]

              corresponds to the smallest equivalence relation containing $S$.
    \end{enumerate}
\end{proof}

\subsection*{Order Relations}

% chapter1:section3:exercise6
\begin{exercise}\label{chapter1:section3:exercise6}
    Define a relation on the plane by setting
    \[
        (x_{0}, y_{0}) < (x_{1}, y_{1})
    \]

    if either $y_{0} - x_{0}^{2} < y_{1} - x_{1}^{2}$, or $y_{0} - x_{0}^{2} = y_{1} - x_{1}^{2}$ and $x_{0} < x_{1}$. Show that this is an order relation on the plane, and describe it geometrically.
\end{exercise}

\begin{proof}
    For every point $(x_{0}, y_{0})$ on the plane, $y_{0} - x_{0}^{2} = y_{0} - x_{0}^{2}$ and $x_{0} = x_{0}$. Therefore $(x_{0}, x_{0}) < (x_{0}, y_{0})$ is false for every point $(x_{0}, y_{0})$.

    Let $(x_{0}, y_{0}), (x_{1}, y_{1})$ be two distinct points on the plane. Then either $y_{0} - x_{0}^{2} < y_{1} - x_{1}^{2}$, or $y_{1} - x_{1}^{2} < y_{0} - x_{0}^{2}$, or $y_{0} - x_{0}^{2} = y_{1} - x_{1}^{2}$. If $y_{0} - x_{0}^{2} = y_{1} - x_{1}^{2}$ then either $x_{0} < x_{1}$ or $x_{1} < x_{0}$. Hence either $(x_{0}, y_{0}) < (x_{1}, y_{1})$ or $(x_{1}, y_{1}) < (x_{0}, y_{0})$.

    If $(x_{0}, y_{0}) < (x_{1}, y_{1})$ and $(x_{1}, y_{1}) < (x_{2}, y_{2})$ then exactly one of the following occurs:
    \begin{itemize}
        \item $y_{0} - x_{0}^{2} < y_{1} - x_{1}^{2}$ and $y_{1} - x_{1}^{2} < y_{2} - x_{2}^{2}$ then $y_{0} - x_{0}^{2} < y_{2} - x_{2}^{2}$.
        \item $y_{0} - x_{0}^{2} < y_{1} - x_{1}^{2}$, $y_{1} - x_{1}^{2} = y_{2} - x_{2}^{2}$, and $x_{1} < x_{2}$ then $y_{0} - x_{0}^{2} < y_{2} - x_{2}^{2}$.
        \item $y_{0} - x_{0}^{2} = y_{1} - x_{1}^{2}$, $x_{0} < x_{1}$, and $y_{1} - x_{1}^{2} < y_{2} - x_{2}^{2}$ then $y_{0} - x_{0}^{2} < y_{2} - x_{2}^{2}$.
        \item $y_{0} - x_{0}^{2} = y_{1} - x_{1}^{2}$, $x_{0} < x_{1}$, $y_{1} - x_{1}^{2} = y_{2} - x_{2}^{2}$, and $x_{1} < x_{2}$ then $y_{0} - x_{0}^{2} = y_{2} - x_{2}^{2}$ and $x_{0} < x_{2}$.
    \end{itemize}

    In all cases, we obtain $(x_{0}, y_{0}) < (x_{2}, y_{2})$, therefore the relation is transitive.

    Hence this is an order relation on the plane.

    I describe the relation like this: $(x_{0}, y_{0}) < (x_{1}, y_{1})$ if the parabola $y - x^{2} = a$ containing $(x_{0}, y_{0})$ is below the parabola of the same sort containing $(x_{1}, y_{1})$, or they lie on a same parabola of this sort and $x_{0} < x_{1}$.
\end{proof}

% chapter1:section3:exercise7
\begin{exercise}\label{chapter1:section3:exercise7}
    Show that the restriction of an order relation is an order relation.
\end{exercise}

\begin{proof}
    Let $<$ be an order relation on $A$ and $A_{0}$ is a subset of $A$.

    If $x\in A_{0}$ then $x < x$ is false, so $x <_{A_{0}} x$ is also false.

    If $x\ne y$ and $x, y\in A_{0}$ then $x < y$ or $y < x$, so $x <_{A_{0}} y$ and $y <_{A_{0}} x$.

    If $x, y, z\in A_{0}$ such that $x <_{A_{0}} y$ and $y <_{A_{0}} z$ then $x < y$ and $y < z$. Because $<$ is transitive, $x < z$. Because $x, z\in A_{0}$, $x <_{A_{0}} z$.

    Hence the restriction of an order relation is an order relation.
\end{proof}

% chapter1:section3:exercise8
\begin{exercise}\label{chapter1:section3:exercise8}
    Check that the relation defined in Example 7 is an order relation.

    Consider the relation on the real line consisting of all pairs $(x, y)$ of real
    numbers such that $x < y$. It is an order relation, called the ``usual order relation{,}'' on the real line. A less familiar order relation on the real line is the following: Define $xCy$ if $x^{2} < y^{2}$, or if $x^{2} = y^{2}$ and $x < y$.
\end{exercise}

\begin{proof}
    For every real number $x$, $xCx$ is false because $x^{2} = x^{2}$ and $x = x$.

    For every pair of distinct real numbers $x, y$, either
    \begin{itemize}[itemsep=0pt]
        \item $x^{2} < y^{2}$
        \item or $y^{2} < x^{2}$
        \item or $x^{2} = y^{2}$ and $x < y$
        \item or $x^{2} = y^{2}$ and $x > y$
    \end{itemize}

    so either $xCy$ or $yCx$.

    If $xCy$ and $yCz$ then either
    \begin{itemize}[itemsep=0pt]
        \item $x^{2} < y^{2}$ and $y^{2} < z^{2}$ so $x^{2} < z^{2}$.
        \item $x^{2} < y^{2}$ and $y^{2} = z^{2}$ and $y < z$ so $x^{2} < z^{2}$.
        \item $x^{2} = y^{2}$ and $x < y$ and $y^{2} < z^{2}$ so $x^{2} < z^{2}$.
        \item $x^{2} = y^{2}$ and $x < y$ and $y^{2} = z^{2}$ and $y < z$ then $x^{2} = z^{2}$ and $x < z$.
    \end{itemize}

    So $xCz$.

    Hence $C$ is an order relation.
\end{proof}

% chapter1:section3:exercise9
\begin{exercise}\label{chapter1:section3:exercise9}
    Check that the dictionary order is an order relation.
\end{exercise}

\begin{proof}
    For every $a\times b\in A\times b$, $a\times b < a\times b$ is false because $a = a$ and $b = b$.

    Let $a_{0}\times b_{0}$, $a_{1}\times b_{1}$ be distinct elements of $A\times B$, then either
    \begin{itemize}
        \item $a_{0} <_{A} a_{1}$
        \item $a_{1} <_{A} a_{0}$
        \item $a_{0} = a_{1}$ and $b_{0} <_{B} b_{1}$
        \item $a_{0} = a_{1}$ and $b_{1} <_{B} b_{0}$
    \end{itemize}

    so either $a_{0}\times b_{0} < a_{1}\times b_{1}$ or $a_{1}\times b_{1} < a_{0}\times b_{0}$.

    If $a_{0}\times b_{0} < a_{1}\times b_{1}$ and $a_{1}\times b_{1} < a_{2}\times b_{2}$ then either
    \begin{itemize}[itemsep=0pt]
        \item $a_{0} <_{A} a_{1}$ and $a_{1} <_{A} a_{2}$ so $a_{0} <_{A} a_{2}$.
        \item $a_{0} <_{A} a_{1}$ and $a_{1} = a_{2}$ and $b_{1} <_{B} b_{2}$ so $a_{0} <_{A} a_{2}$.
        \item $a_{0} = a_{1}$ and $b_{0} <_{B} b_{1}$ and $a_{1} <_{A} a_{2}$ so $a_{0} <_{A} a_{2}$.
        \item $a_{0} = a_{1}$ and $b_{0} <_{B} b_{1}$ and $a_{1} = a_{2}$ and $b_{1} <_{B} b_{2}$ so $a_{0} = a_{2}$ and $b_{0} <_{B} b_{2}$.
    \end{itemize}

    So $a_{0}\times b_{0} < a_{1}\times b_{1}$.

    Hence the dictionary order relation is indeed an order relation.
\end{proof}

% chapter1:section3:exercise10
\begin{exercise}\label{chapter1:section3:exercise10}
    \begin{enumerate}[label={(\alph*)}]
        \item Show that the map $f: \openinterval{-1, 1}\to\mathbb{R}$ of Example 9 is order preserving.
        \item Show that the equation $g(y) = 2y/\left( 1 + {(1 + 4y^{2})}^{1/2} \right)$ defines a function $g: \mathbb{R}\to \openinterval{-1, 1}$ that is both a left and a right inverse for $f$.
    \end{enumerate}
\end{exercise}

\begin{proof}
    \begin{enumerate}[label={(\alph*)}]
        \item If $x, y\in \openinterval{-1, 1}$ and $x < y$ then
              \[
                  \frac{x}{1-x^{2}} - \frac{y}{1-y^{2}} = \frac{(x - y)(1 + xy)}{(1 - x^{2})(1 - y^{2})} < 0
              \]

              so $f$ is order preserving.
        \item \begin{align*}
                  g(f(x)) & = \frac{\frac{2x}{1-x^{2}}}{1 + {\left(1 + \frac{4x^{2}}{{(1-x^{2})}^{2}}\right)}^{1/2}} = \frac{\frac{2x}{1-x^{2}}}{1 + {\left(\frac{{(1+x^{2})}^{2}}{{(1-x^{2})}^{2}}\right)}^{1/2}} = \frac{\frac{2x}{1-x^{2}}}{1 + \frac{1+x^{2}}{1-x^{2}}} = x,                                      \\
                  f(g(y)) & = \frac{\frac{2y}{1 + {\left(1 + 4y^{2}\right)}^{1/2}}}{1 - \frac{4y^{2}}{{\left(1 + {\left(1 + 4y^{2}\right)}^{1/2}\right)}^{2}}} = \frac{2y(1 + {(1 + 4y^{2})}^{1/2})}{{(1 + {(1 + 4y^{2})}^{1/2})}^{2} - 4y^{2}} = \frac{2y(1 + {(1 + 4y^{2})}^{1/2})}{2 + 2{(1 + 4y^{2})}^{1/2}} = y.
              \end{align*}

              Hence $g$ is a left and right inverse for $f$.
    \end{enumerate}
\end{proof}

% chapter1:section3:exercise11
\begin{exercise}\label{chapter1:section3:exercise11}
    Show that an element in an ordered set has at most one immediate successor and at most one immediate predecessor. Show that a subset of an ordered set has at most one smallest element and at most one largest element.
\end{exercise}

\begin{proof}
    Let $S$ be an order set with the order relation $<$ and $a\in S$.

    Assume $b$ and $c$ are distinct immediate successors of $a$, then $a < b$ and $a < c$. Because $<$ is a total order relation, then $b < c$ or $c < b$. This contradicts the definition of immediate successor. Hence every element of an ordered set has at most one immediate successor.

    Assume $b$ and $c$ are distinct immediate predecessors of $a$, then $b < a$ and $c < a$. Because $<$ is a total order relation, then $b < c$ or $c < b$. This contradicts the definition of immediate predecessor. Hence every element of an ordered set has at most one immediate predecessor.

    Let $S'$ be a subset of $S$.

    Assume $S'$ has two distinct smallest elements $x, y$, then either $x < y$ or $y < x$, which contradicts the definition of smallest element. Hence every subset of an ordered set has at most one smallest element.

    Assume $S'$ has two distinct largest elements $x, y$, then either $x < y$ or $y < x$, which contradicts the definition of largest element. Hence every subset of an ordered set has at most one largest element.
\end{proof}

% chapter1:section3:exercise12
\begin{exercise}\label{chapter1:section3:exercise12}
    Let $\mathbb{Z}_{+}$ denote the set of positive integers. Consider the following order relations on $\mathbb{Z}_{+}\times\mathbb{Z}_{+}$:
    \begin{enumerate}[label={(\roman*)}]
        \item The dictionary order.
        \item $(x_{0}, y_{0}) < (y_{0}, y_{1})$ if either $x_{0} - y_{0} < x_{1} - y_{1}$, or $x_{0} - y_{0} = x_{1} - y_{1}$ and $y_{0} < y_{1}$.
        \item $(x_{0}, y_{0}) < (y_{0}, y_{1})$ if either $x_{0} + y_{0} < x_{1} + y_{1}$, or $x_{0} + y_{0} = x_{1} + y_{1}$ and $y_{0} < y_{1}$.
    \end{enumerate}

    In these order relations, which elements have immediate predecessors? Does the set have a smallest element? Show that all three order types are different.
\end{exercise}

\begin{proof}
    \begin{enumerate}[label={(\roman*)}]
        \item Every element $(a, b)\in \mathbb{Z}_{+}\times\mathbb{Z}_{+}$ where $b > 1$ has a immediate predecessor, which is $(a, b-1)$. The elements $(a, 1)$ have no immediate predecessor. The set has a smallest element, which is $(1, 1)$.
        \item If $(a, b)\in\mathbb{Z}_{+}\times\mathbb{Z}_{+}$ and $a, b > 1$ then $(a-1, b-1) < (a, b)$. If $(x, y) < (a, b)$ then $x - y < a - b$ or $x - y = a - b$ and $y < b$. If $x - y < a - b$ then $(x, y) < (a-1, b-1)$. If $x - y = a - b$ and $y < b$, then $y\leq b-1$ and $(x, y)\leq (a-1, b-1)$. Therefore $(a-1, b-1)$ is the predecessor of $(a, b)$.

              For $(a, 1)$ where $a > 1$, we have $(a-1, 1) < (a, b)$. If $(x, y) < (a, 1)$ then $x - y < a - 1$ or $x - y = a - 1$ and $y < 1$. The latter case is impossible because $y\geq 1$. Therefore $x - y < a - 1$, so $x - y\leq a - 2$. If $x - y < a - 2$ then $(x, y) < (a - 1, 1)$. If $x - y = a - 2$ then $y\geq 1$, moreover
              \[
                  (a-1, 1) < (a-1, 2) < \cdots < (a, 1)
              \]

              so $(a, 1)$ with $a > 1$ have no predecessor because there will be always something between.

              For $(1, b)$, we have $(1, b-1) < (1, b)$. If $(x, y) < (1, b)$ then either $x - y < 1 - b$ or $x - y = 1 - b$ and $y < b$.

              If $x - y < 1 - b$ then $1 - y\leq x - y < 1 - b$, which implies $y > b$. Moreover, $(x, y) < (x+1, y+1) < (1, b)$ so $(x, y)$ is not a predecessor of $(1, b)$ if $x - y < 1 - b$.

              If $x - y = 1 - b$ and $y < b$ then $y\leq b - 1$. $x - y = 1 - b$ so $x = y + (1 - b)\leq b - 1 + (1 - b) = 0$, which is impossible.

              So $(1, b)$ have no predecessor.

              There is no smallest element in this case, because for every $(x, y)\in\mathbb{Z}_{+}\times\mathbb{Z}_{+}$, we can always find a smaller element, for example $(x, y+1) < (x, y)$.
        \item For $(a, b)\in\mathbb{Z}_{+}\times\mathbb{Z}_{+}$ such that $a, b > 1$ and $(x, y) < (a, b)$ then either $x + y < a + b$ or $x + y = a + b$ and $y < b$.

              If $x + y < a + b$ then $x + y \leq a + b - 1$.
              \begin{itemize}
                  \item if $x + y < a + b - 1$ then $(x, y) < (a, b - 1) < (a, b)$.\item if $x + y = a + b - 1$ and $x > 1$ then $(x, y) < (x-1, y+1) < (a, b)$
                  \item if $x + y = a + b - 1$ and $y > 1$ then $(x, y) < (x+1, y-1) < (a, b)$
                  \item if $x + y = a + b - 1$ and $x = y = 1$ then $(x, y) = (1, 1) < (a, b)$. $(1, 1)$ is the predecessor of $(a, b)$ if and only if $(a, b) = (2, 1)$. However, we are working on the case $a, b > 1$.
              \end{itemize}

              If $x + y = a + b$ and $y < b$ then $x\geq a + 1$ and $y\leq b - 1$. Moreover, $(x, y)\leq (a+1, b-1) < (a, b)$. There is no element between $(a+1,b-1)$ and $(a, b)$. So $(a+1, b-1)$ is the predecessor of $(a, b)$.

              For $(a, 1)$ where $a > 1$, $(a-1, 1)$ is the predecessor of $(a, 1)$.

              For $(1, b)$ where $b > 1$, $(2, b-1)$ is the predecessor of $(1, b)$.

              $(1, 1)$ has no predecessor, and $(1, 1)$ is also the smallest element.
    \end{enumerate}

    With all these differences (which elements has predecessors, to which relation the set has a smallest element), we conclude that the given three order types are different.
\end{proof}

% chapter1:section3:exercise13
\begin{exercise}\label{chapter1:section3:exercise13}
    Prove the following: Theorem. If an ordered set $A$ has the least upper bound property, then it has the greatest lower bound property.
\end{exercise}

\begin{proof}
    Let $A_{0}$ be a subset of $A$, which is bounded below. Because $A_{0}$ is bounded below, then there exists $b\in A$ such that for every $a\in A_{0}$, $b\leq a$. Let $B$ be the set of all lower bounds of $A_{0}$ then $B$ is bounded above by any element of $A_{0}$. Because of this definition, there is no element $x$ of $A$ such that $b < x < a$ for every $a\in A_{0}, b\in B$.

    Since $A$ has the least upper bound property, $B$ has a least upper bound.
    \begin{itemize}
        \item Suppose the least upper bound of $B$ is an element of $B$ then it is also the largest lower bound of $A_{0}$.
        \item Suppose the least upper bound of $B$ is not an element of $B$. Either $\sup B > a$ for some $a\in A$ or $\sup B\leq a$ for all $a\in A$. The former case contradicts the definition of least upper bound. Hence $\sup B\leq a$ for all $a\in A$. By the last sentence of the previous paragraph, $\sup B$ is an element of $A_{0}$. $\sup B$ is also the smallest element of $A_{0}$. Therefore $\sup B$ is the largest lower bound of $A_{0}$.
    \end{itemize}

    Thus $A$ has the greatest lower bound property.
\end{proof}

% chapter1:section3:exercise14
\begin{exercise}\label{chapter1:section3:exercise14}
    If $C$ is a relation on a set $A$, define a new relation $D$ on $A$ by letting $(b, a)\in D$ if $(a, b)\in C$.
    \begin{enumerate}[label={(\alph*)}]
        \item Show that $C$ is symmetric if and only if $C = D$.
        \item Show that if $C$ is an order relation, $D$ is also an order relation.
        \item Prove the converse of the theorem in Exercise~\ref{chapter1:section3:exercise13}.
    \end{enumerate}
\end{exercise}

\begin{proof}
    \begin{enumerate}[label={(\alph*)}]
        \item Suppose $C$ is symmetric and $(a, b)\in C$. Then $(b, a)\in C$ because $C$ is symmetric. By the definition of $D$, $(a, b)\in D$, so $C\subset D$. Let $(a, b)\in D$ then $(b, a)\in C$. Because $C$ is symmetric, $(a, b)\in C$. Therefore $D\subset C$. Hence $C = D$.

              Suppose $C = D$ then if $(a, b)\in C$ then $(a, b)\in D$, which implies $(b, a)\in C$. Therefore $C$ is symmetric.
        \item Suppose $C$ is an order relation.

              For every $a\in A$, $(a, a)\notin C$ so $(a, a)\notin D$.

              For every $a_{0}, a_{1}\in A$ and $a_{0}\ne a_{1}$, either $a_{0}Ca_{1}$ or $a_{1}Ca_{0}$ so either $a_{0}Da_{1}$ or $a_{1}Da_{0}$.

              If $a_{0}Da_{1}$ and $a_{1}Da_{2}$ then $a_{2}Ca_{1}$ and $a_{1}Ca_{0}$, which implies $a_{2}Ca_{0}$ and hence $a_{0}Da_{2}$.

              Hence $D$ is also an order relation.
        \item Suppose $A$ has the largest lower bound property with relation $C$, then $A$ has the least upper bound property with relation $D$. By Exercise~\ref{chapter1:section3:exercise13}, $D$ has the largest lower bound property. Therefore $C$ has the least upper bound property.
    \end{enumerate}
\end{proof}

% chapter1:section3:exercise15
\begin{exercise}\label{chapter1:section3:exercise15}
    Assume that the real line has the least upper bound property.
    \begin{enumerate}[label={(\alph*)}]
        \item Do the sets $\closedinterval{0, 1}$ and $\halfopenright{0, 1}$ have the least upper bound property?
        \item Does $\closedinterval{0,1}\times\closedinterval{0,1}$ in the dictionary order have the least upper bound property? What about $\closedinterval{0,1}\times\halfopenright{0,1}$? What about $\halfopenright{0,1}\times\closedinterval{0,1}$?
    \end{enumerate}
\end{exercise}


\begin{proof}
    \begin{enumerate}[label={(\alph*)}]
        \item Let $S$ be a nonempty subset of $\closedinterval{0, 1}$ then $1$ is an upper bound of $S$. $S\subset\mathbb{R}$ so $S$ has a least upper bound. Moreover, $0\leq \sup S\leq 1$ so $\sup S\in\closedinterval{0, 1}$. Hence $\closedinterval{0, 1}$ has the least upper bound property.

              $\halfopenright{0, 1}$ does not have the least upper bound property.
        \item Yes, $\closedinterval{0, 1}\times\closedinterval{0, 1}$ in the dictionary order has the least upper bound property.

              However, $\closedinterval{0, 1}\times\halfopenright{0, 1}$ and $\halfopenright{0, 1}\times\closedinterval{0, 1}$ don't have the least upper bound property.
    \end{enumerate}
\end{proof}

\section{The Integers and the Real Numbers}

% chapter1:section4:exercise1
\begin{exercise}\label{chapter1:section4:exercise1}
    Prove the following ``laws of algebra'' for $\mathbb{R}$, using only axioms {(1)}-{(5)}
    \begin{enumerate}[label={(\alph*)}]
        \item If $x + y = x$, then $y = 0$.
        \item $0\cdot x = 0$.
        \item $-0 = 0$.
        \item $-(-x) = x$.
        \item $x(-y) = -(xy) = (-x)y$.
        \item $(-1)x = -x$.
        \item $x(y - z) = xy - xz$.
        \item $-(x + y) = -x - y; -(x - y) = -x + y$.
        \item If $x\ne 0$ and $x\cdot y = x$, then $y = 1$.
        \item $x/x = 1$ if $x\ne 0$.
        \item $x/1 = x$.
        \item $x\ne 0$ and $y\ne 0\Rightarrow xy\ne 0$.
        \item $(1/y)(1/z) = 1/(yz)$ if $y, z\ne 0$.
        \item $(x/y)(w/z) = (xw)/(yz)$ if $y, z\ne 0$.
        \item $(x/y) + (w/z) = (xz + wy)/(yz)$ if $y, z\ne 0$.
        \item $x\ne 0\Rightarrow 1/x\ne 0$.
        \item $1/(w/z) = z/w$ if $w, z\ne 0$.
        \item $(x/y)/(w/z) = (xz)/(yw)$ if $y, w, z\ne 0$.
        \item $(ax)/y = a(x/y)$ if $y\ne 0$.
        \item $(-x)/y = x/(-y) = -(x/y)$ if $y\ne 0$.
    \end{enumerate}
\end{exercise}

\begin{proof}
    \begin{enumerate}[label={(\alph*)}]
        \item If $x + y = x$, then
              \[
                  y = 0 + y = ((-x) + x) + y = (-x) + (x + y) = (-x) + x = 0.
              \]
        \item $x\cdot x = (x + 0)\cdot x = x\cdot x + 0\cdot x$. By (a), we conclude that $0\cdot x = 0$.
        \item If $0 + x = x + 0 = 0$, then $x = 0$, by part (a). Therefore $-0 = 0$.
        \item $(-x) + x = x + (-x) = 0$ so $-(-x) = x$.
        \item $x(-y) + xy = xy + x(-y) = x\cdot (y\cdot (-y)) = x\cdot 0 = 0\cdot x = 0$. Therefore $x(-y) = -xy$. Similarly, $(-x)y = -xy$.
        \item $x + (-1)x = (-1)x + x = ((-1) + 1)x = 0\cdot x = 0$. Therefore $(-1)x = -x$.
        \item $x(y - z) = x(y + (-z)) = xy + x(-z) = xy - xz$, according to (e).
        \item \begin{align*}
                  (x + y) + (-x - y) & = (-x - y) + (x + y)      \\
                                     & = ((-x) + (-y)) + (x + y) \\
                                     & = ((-y) + (-x)) + (x + y) \\
                                     & = (-y) + ((-x) + (x + y)) \\
                                     & = (-y) + (((-x) + x) + y) \\
                                     & = (-y) + y                \\
                                     & = 0
              \end{align*}

              so $-x - y = -(x + y)$. Moreover
              \[
                  -(x - y) = -(x + (-y)) = -x - (-y) = -x + y.
              \]
        \item \[
                  y = 1\cdot y = (x^{-1}x)y = x^{-1}(xy) = x^{-1}x = 1.
              \]
        \item $x/x = x\cdot (1/x) = 1$.
        \item $x\cdot 1 = x$ so $x/1 = (x\cdot 1)\cdot (1/1) = (x\cdot 1)\cdot 1 = x$.
        \item If $x\ne 0$ and $y\ne 0$, then $(1/y)(1/x)xy = xy(1/y)(1/x) = 1$, which means $1/(xy) = (1/y)(1/x)$. Therefore $xy$ has a multiplicative inverse, hence $xy\ne 0$.
        \item $zy(1/y)(1/z) = z(1/z) = 1$, $(1/y)(1/z)zy = (1/y)y = 1$, hence $1/(yz) = 1/(zy) = (1/y)(1/z)$.
        \item $(x/y)(w/z) = x(1/y)w(1/z) = xw(1/y)(1/z) = (xw)/(yz)$.
        \item \begin{align*}
                  (x/y) + (w/z) & = (xz)/(yz) + (yw)/(yz)                   \\
                                & = (xz)\cdot (1/(yz)) + (yw)\cdot (1/(yz)) \\
                                & = (xz + yw)\cdot (1/(yz))                 \\
                                & = (xz + yw)/(yz)
              \end{align*}
        \item Assume $1/x = 0$ then $1 = x(1/x) = 0$, which is a contradiction. Hence $x\ne 0\Rightarrow 1/x\ne 0$.
        \item $(w/z)(z/w) = w(1/z)z(1/w) = w(1/w) = 1$. Hence $1/(w/z) = z/w$ if $w, z\ne 0$.
        \item $(x/y)/(w/z) = (x/y)(1/(w/z)) = (x/y)(z/w) = (xz)/(yw)$.
        \item $(ax)/y = (ax)(1/y) = a(x(1/y)) = a(x/y)$.
        \item \begin{align*}
                  x/(-y) & = ((-1)(-x))/((-1)y) = ((-1)/(-1))((-x)/y) = (-x)/y \\
                  (-x)/y & = ((-1)x)/y = (-1)(x/y) = -(x/y)
              \end{align*}

              Hence $(-x)/y = x/(-y) = -(x/y)$
    \end{enumerate}
\end{proof}

% chapter1:section4:exercise2
\begin{exercise}\label{chapter1:section4:exercise2}
    Prove the following ``laws of inequalities'' for $\mathbb{R}$, using axioms {(1)}-{(6)} along with the results of Exercise~\ref{chapter1:section4:exercise1}
    \begin{enumerate}[label={(\alph*)}]
        \item $x > y \land w > z \Rightarrow x + w > y + z$.
        \item $x > 0 \land y > 0 \Rightarrow x + y > 0 \land x\cdot y > 0$.
        \item $x > 0 \Leftrightarrow -x < 0$.
        \item $x > y \Leftrightarrow -x < -y$.
        \item $x > y\land z < 0\Rightarrow xz < yz$.
        \item $x\ne 0 \Rightarrow x^{2} > 0$, where $x^{2} = x\cdot x$.
        \item $-1 < 0 < 1$.
        \item $xy > 0 \Leftrightarrow $ $x$ and $y$ are both positive or both negative.
        \item $x > 0 \Rightarrow 1/x > 0$.
        \item $x > y > 0 \Rightarrow 1/x < 1/y$.
        \item $x < y \Rightarrow x < (x + y)/2 < y$.
    \end{enumerate}
\end{exercise}

\begin{proof}
    \begin{enumerate}[label={(\alph*)}]
        \item Suppose $x > y$ and $w > z$
              \begin{align*}
                  x + w & > y + w & \text{(because $x > y$)} \\
                        & > y + z & \text{(because $w > z$)}
              \end{align*}

              so $x + w > y + z$.
        \item Suppose $x > 0$ and $y > 0$. By (a), $x + y > 0$. By axiom 6, $x\cdot y > 0\cdot y = 0$.
        \item Suppose $x > 0$ and assume $-x \geq 0$. If $-x = 0$ then $x = 0$, which is a contradiction. If $-x > 0$ then by (a), $0 = x + (-x) > 0$, which is also a contradiction. Hence $-x < 0$.

              Suppose $-x < 0$ and assume $x\leq 0$. If $x = 0$ then $-x = 0$, which is a contradiction. If $x < 0$, then $0 = x + (-x) < 0 + (-x) < 0 + 0 = 0$, which is also a contradiction. Hence $x > 0$.

              Thus $x > 0 \Leftrightarrow -x < 0$.
        \item \begin{align*}
                  x > y & \Longleftrightarrow x + (-y) > 0                      \\
                        & \Longleftrightarrow -(x + (-y)) < 0 & \text{(by (c))} \\
                        & \Longleftrightarrow -x - (-y) < 0                     \\
                        & \Longleftrightarrow -x < -y.
              \end{align*}
        \item Suppose $x > y$ and $z < 0$, then $-z > 0$. By axiom 6, $x(-z) > y(-z)$. Moreover, $x(-z) = -(xz)$ and $y(-z) = -(yz)$. By part (d), we conclude that $xz < yz$.
        \item Suppose $x\ne 0$, then either $x > 0$ or $x < 0$.

              If $x > 0$ then $x^{2} = x\cdot x > 0$. If $x < 0$, $-x > 0$ and $x^{2} = x\cdot x = -(x\cdot (-x)) = (-x)\cdot (-x) > 0$.
        \item The multiplicative inverse of $1$ is $1$. By (f), we conclude that $0 < 1\cdot 1 = 1$. By (c), we conclude that $-1 < 0$.
        \item Suppose $xy > 0$ then $x\ne 0, y\ne 0$. If $x > 0$, then $y > 0$ (because otherwise, $xy < 0$). If $x < 0$ then $y < 0$ (because otherwise, $xy < 0$). Hence $x$ and $y$ are both positive or negative.

              Suppose $x$ and $y$ are both positive or negative. If $x > 0$ and $y > 0$ then $xy > 0\cdot y = 0$. If $x < 0$ and $y < 0$ then $xy = -(x\cdot (-y)) = -(-((-x)(-y))) = (-x)(-y) > 0$. Hence $xy > 0$.
        \item Suppose $x > 0$ then $1/x\ne 0$. Moreover, $x(1/x) = 1 > 0$ so $1/x > 0$.
        \item Suppose $x > y > 0$ then $1/x > 0$ and $1/y > 0$. It follows that $1 = x(1/x) > y(1/x) > 0$. Therefore $1/y = 1\cdot(1/y) > (1/y)(y(1/x)) = 1/x$. Hence $1/x < 1/y$.
        \item Suppose $x < y$, then $x + x < x + y < y + y$, which implies $2x < x + y < 2y$. Because $2 > 0$, it follows that $x < (x + y)/2 < y$.
    \end{enumerate}
\end{proof}

% chapter1:section4:exercise3
\begin{exercise}\label{chapter1:section4:exercise3}
    \begin{enumerate}[label={(\alph*)}]
        \item Show that if $\mathscr{A}$ is a nonempty collection of inductive sets, then the intersection of the elements of $\mathscr{A}$ is an inductive set.
        \item Prove the basis properties (1) and (2) of $\mathbb{Z}_{+}$.
    \end{enumerate}
\end{exercise}

\begin{proof}
    \begin{enumerate}[label={(\alph*)}]
        \item Let $B = \bigcap_{A\in\mathscr{A}}A$.

              For every $A\in\mathscr{A}$, $A$ is an inductive set, so $1\in B$. Moreover, if $x\in B$ then $x\in A$ for every $A\in\mathscr{A}$, which implies $x + 1\in A$ for every $A\in\mathscr{A}$, so $x + 1\in B$. Hence $B$ is an inductive set.
        \item $\mathbb{Z}_{+}$ is defined to be the intersection of all inductive subsets of $\mathbb{R}$, so $\mathbb{Z}_{+}$ is inductive, in accordance with (a).

              Suppose $A$ is an inductive subset of real numbers, then $\mathbb{Z}_{+} \subset A$, according to the definition of $\mathbb{Z}_{+}$. Because $A$ is a set of positive integers, so $A\subset \mathbb{Z}_{+}$. Hence $A = \mathbb{Z}_{+}$.
    \end{enumerate}
\end{proof}

% chapter1:section4:exercise4
\begin{exercise}\label{chapter1:section4:exercise4}
    \begin{enumerate}[label={(\alph*)}]
        \item Prove by induction that given $n\in\mathbb{Z}_{+}$, every nonempty subset of $\{ 1,\ldots,n \}$ has a largest element.
        \item Explain why you cannot conclude from (a) that every nonempty subset of $\mathbb{Z}_{+}$ has a largest element.
    \end{enumerate}
\end{exercise}

\begin{proof}
    \begin{enumerate}[label={(\alph*)}]
        \item The only nonempty subset of $\{ 1 \}$ is $\{ 1 \}$ and it has a largest element.

              Assume every nonempty subset of $\{ 1,\ldots,n \}$ has a largest element. Let $C$ be a nonempty subset of $\{ 1, \ldots, n+1 \}$. If $C$ contains $n + 1$ then it is the largest element of $C$. If $C$ does not contain $n + 1$ then $C\subset \{ 1,\ldots, n \}$, which implies $C$ has a largest element, due to the induction hypothesis.

              Hence for every $n\in\mathbb{Z}_{+}$, every nonempty subset of $\{ 1, \ldots,n \}$ has a largest element.
        \item We cannot conclude from (a) that every nonempty subset of $\mathbb{Z}_{+}$ has a largest element because there is no positive integer $n$ such that $\mathbb{Z}_{+} = S_{n+1} = \{ 1,\ldots,n \}$.
    \end{enumerate}
\end{proof}

% chapter1:section4:exercise5
\begin{exercise}\label{chapter1:section4:exercise5}
    Prove the following properties of $\mathbb{Z}$ and $\mathbb{Z}_{+}$
    \begin{enumerate}[label={(\alph*)}]
        \item $a, b\in\mathbb{Z}_{+}\Rightarrow a + b\in\mathbb{Z}_{+}$.
        \item $a, b\in\mathbb{Z}_{+} \Rightarrow a\cdot b\in\mathbb{Z}_{+}$.
        \item Show that $a\in\mathbb{Z}_{+} \Rightarrow a - 1\in \mathbb{Z}_{+}\cup\{ 0 \}$.
        \item $c, d\in\mathbb{Z}\Rightarrow c + d\in\mathbb{Z}$ and $c - d\in\mathbb{Z}$.
        \item $c, d\in\mathbb{Z}\Rightarrow c\cdot d\in \mathbb{Z}$.
    \end{enumerate}
\end{exercise}

\begin{proof}
    \begin{enumerate}[label={(\alph*)}]
        \item Because $\mathbb{Z}_{+}$ is inductive, so if $a\in\mathbb{Z}_{+}$, then $a + 1\in\mathbb{Z}_{+}$.

              Assume $a + n\in\mathbb{Z}_{+}$ for some $n\in\mathbb{Z}_{+}$, then because $\mathbb{Z}_{+}$ is inductive, it follows that $(a + n) + 1 = a + (n + 1)\in\mathbb{Z}_{+}$.

              Hence by the principle of mathematical induction, $a, b\in\mathbb{Z}_{+}\Rightarrow a + b\in\mathbb{Z}_{+}$.
        \item $a\cdot 1 = a\in\mathbb{Z}_{+}$.

              Assume $a\cdot n\in\mathbb{Z}_{+}$ for some $n\in\mathbb{Z}_{+}$. By the induction hypothesis and part (a)
              \[
                  a\cdot (n + 1) = a\cdot n + a \in\mathbb{Z}_{+}
              \]

              so $a\cdot (n + 1)\in\mathbb{Z}_{+}$.

              Hence by the principle of mathematical induction, $a, b\in\mathbb{Z}_{+}\Rightarrow a\cdot b\in\mathbb{Z}_{+}$.
        \item Let $X = \{ x \mid x\in\mathbb{R} \land x - 1\in\mathbb{Z}_{+}\cup\{ 0 \} \}$ then $1\in X$. Assume $n\in X$ for some $n\in\mathbb{Z}_{+}$. $n\in\mathbb{Z}_{+}$ so $(n + 1) - 1\in\mathbb{Z}_{+}$, hence $n + 1\in X$. Hence by the principle of mathematical induction, $\mathbb{Z}_{+}\subset X$.

              Thus $a\in\mathbb{Z}_{+} \Rightarrow a - 1\in\mathbb{Z}_{+}\cup\{0\}$.
        \item Suppose $c\in\mathbb{Z}$. $c + 0 = c\in\mathbb{Z}$.

              Then either $c\in\mathbb{Z}_{+}$, $c = 0$, or $-c\in\mathbb{Z}_{+}$.
              \begin{itemize}[itemsep=0pt]
                  \item If $c\in\mathbb{Z}_{+}$ then $c + 1\in\mathbb{Z}_{+}$, $c - 1\in\mathbb{Z}_{+}\cup\{0\}$.
                  \item If $c = 0$ then $c + 1 = 1$ and $c - 1 = -1$.
                  \item If $-c \in\mathbb{Z}_{+}$ then $-(c+1)\in\mathbb{Z}_{+}\cup\{0\}$ and $-(c - 1)\in\mathbb{Z}_{+}$.
              \end{itemize}

              In either case, we conclude that $c + 1\in\mathbb{Z}$ and $c - 1\in\mathbb{Z}$.

              Assume $c + n$ and $c - n\in\mathbb{Z}$ for some positive integer $n$, then $(c + n) + 1$ and $(c - n) - 1$ are in $\mathbb{Z}$.

              Hence by the principle of mathematical induction, $c, d\in\mathbb{Z}\Rightarrow c + d\in\mathbb{Z}\land c - d\in\mathbb{Z}$.
        \item If $d = 0$ then $c\cdot d = 0\in\mathbb{Z}$. If $d = 1$ then $c\cdot d = c\in\mathbb{Z}$.

              Assume $c\cdot n\in\mathbb{Z}$ for some positive integer $n$, then $c\cdot (n + 1) = c\cdot n + c\in\mathbb{Z}$. Hence $c\cdot d\in\mathbb{Z}$ if $c\in\mathbb{Z}$ and $d\in\mathbb{Z}_{+}\cup\{ 0 \}$.

              If $-d\in\mathbb{Z}_{+}$, then $c\cdot d = -(c\cdot d)\in\mathbb{Z}$. Thus $c\cdot d\in\mathbb{Z}$ if $c\in\mathbb{Z}$ and $d\in\mathbb{Z}$.
    \end{enumerate}
\end{proof}

% chapter1:section4:exercise6
\begin{exercise}\label{chapter1:section4:exercise6}
    Let $a\in\mathbb{R}$. Define inductively
    \begin{align*}
        a^{1}   & = a,           \\
        a^{n+1} & = a^{n}\cdot a
    \end{align*}

    for $n\in\mathbb{Z}_{+}$. Show that for $n, m\in\mathbb{Z}_{+}$ and $a, b\in\mathbb{R}$,
    \begin{align*}
        a^{n}a^{m}    & = a^{n+m},    \\
        {(a^{n})}^{m} & = a^{nm},     \\
        a^{m}b^{m}    & = {(ab)}^{m}.
    \end{align*}

    These are called the laws of exponents.
\end{exercise}

\begin{proof}
    $a^{n}a^{1} = a^{n+1}$, ${(a^{n})}^{1} = a^{n} = a^{n\cdot 1}$, $a^{1}b^{1} = ab = {(ab)}^{1}$.

    Assume $a^{n}a^{m} = a^{n+m}$ for some positive integer $m$. Then
    \[
        a^{n}a^{m+1} = a^{n}(a^{m}a^{1}) = (a^{n}a^{m})a = a^{n+m}\cdot a = a^{n+(m+1)}
    \]

    therefore $a^{n}a^{m} = a^{n+m}$ for $n, m\in\mathbb{Z}_{+}$.

    Assume ${(a^{n})}^{m} = a^{nm}$ for some positive integer $m$. Then
    \[
        {(a^{n})}^{m+1} = {(a^{n})}^{m}\cdot a^{n} = a^{nm}\cdot a^{n} = a^{nm + n} = a^{n(m+1)}
    \]

    therefore ${(a^{n})}^{m} = a^{nm}$ for $n, m\in\mathbb{Z}_{+}$.

    Assume $a^{m}b^{m} = {(ab)}^{m}$ for some positive integer $m$. Then
    \[
        a^{m+1}b^{m+1} = (aa^{m})(b^{m}b) = a(a^{m}b^{m})b = a{(ab)}^{m}b = {(ab)}^{m}(ab) = {(ab)}^{m+1}
    \]

    therefore $a^{m}b^{m} = {(ab)}^{m}$ for $m\in\mathbb{Z}_{+}$.
\end{proof}

% chapter1:section4:exercise7
\begin{exercise}\label{chapter1:section4:exercise7}
    Let $a\in\mathbb{R}$ and $a\ne 0$. Define $a^{0} = 1$, and for $n\in\mathbb{Z}_{+}$, $a^{-n} = 1/a^{n}$. Show that the laws of exponents hold for $a, b\ne 0$ and $n, m\in\mathbb{Z}$.
\end{exercise}

\begin{proof}
    $1^{0} = 1^{1} = 1$. Assume $1^{n} = 1$ for some positive integer $n$, then $1^{n+1} = 1^{n}\cdot 1 = 1^{n} = 1$. Hence $1^{n} = 1$ for $n\in\mathbb{Z}_{+}$. If $-n\in\mathbb{Z}_{+}$, then $1^{n} = 1/1^{-n} = 1/1 = 1$. Therefore $1^{n} = 1$ for $n\in\mathbb{Z}$.

    $a^{0} = 1/a^{0}$. If $n\in\mathbb{Z}_{+}$ then $a^{-n} = 1/a^{n}$, so $a^{n} = 1/a^{-n}$. Hence for $n\in\mathbb{Z}$, $a^{-n} = 1/a^{n}$.

    \bigskip
    \textbf{Prove that $a^{m}b^{m} = {(ab)}^{m}$ for $a, b\ne 0$ and $m\in\mathbb{Z}$.}

    By Exercise~\ref{chapter1:section4:exercise6}, ${(ab)}^{m} = a^{m}b^{m}$ for $m\in\mathbb{Z}_{+}$. If $m = 0$ then ${(ab)}^{m} = 1 = 1\cdot 1 = a^{m}b^{m}$. If $-m\in\mathbb{Z}^{+}$ then
    \[
        {(ab)}^{m} = 1/{(ab)}^{-m} = 1/((a^{-m})(b^{-m})) = (1/a^{-m})(1/b^{-m}) = a^{m}b^{m}.
    \]

    Therefore ${(ab)}^{m} = a^{m}b^{m}$ for $a, b\ne 0$ and $m\in\mathbb{Z}$.

    \bigskip
    \textbf{Prove that $a^{n}a^{m} = a^{n+m}$ for $a\ne 0$ and $n, m\in\mathbb{Z}$.}

    If $n\in\mathbb{Z}_{+}$, then $a^{n}a^{1} = a^{n+1}$. If $n = 0$ then $a^{n}a^{1} = 1\cdot a^{1} = a^{n+1}$. If $-n\in\mathbb{Z}_{+}$, let $p = -n$, then $a^{p-1}a^{1} = a^{p}$ and
    \begin{align*}
        a^{n}a^{1} & = a^{-p}a^{1} = (1/a^{p})a^{1}          \\
                   & = (1/a^{p})(1/a^{-1}) = 1/(a^{p}a^{-1}) \\
                   & = 1/a^{p-1} = a^{1-p}                   \\
                   & = a^{n+1}.
    \end{align*}

    Hence $a^{n+1} = a^{n}\cdot a$ for $n\in\mathbb{Z}$.

    Assume $a^{n}a^{m} = a^{n+m}$ for some positive integer $m$, then
    \[
        a^{n}a^{m+1} = a^{n}(a^{m}a) = (a^{n}a^{m})a = a^{n+m}a = a^{n+(m+1)}
    \]

    so $a^{n}a^{m} = a^{n+m}$ for $n\in\mathbb{Z}, m\in\mathbb{Z}_{+}$.

    If $m = 0$, then $a^{n}a^{m} = a^{n}\cdot 1 = a^{n} = a^{n+m}$. If $-m\in\mathbb{Z}_{+}$ then $a^{-n}a^{-m} = a^{-(m+n)}$, so
    \[
        a^{n+m} = 1/(a^{-n}a^{-m}) = (1/a^{-n})(1/a^{-m}) = a^{n}a^{m}.
    \]

    Hence $a^{n+m} = a^{n}a^{m}$ for $n,m\in\mathbb{Z}$ and $a\ne 0$.

    \bigskip
    \textbf{Prove that ${(a^{n})}^{m} = a^{nm}$ for $a\ne 0$ and $n, m\in\mathbb{Z}$.}

    If $n\in\mathbb{Z}_{+}$ and $m\in\mathbb{Z}_{+}$ then ${(a^{n})}^{m} = a^{nm}$.

    If $n\in\mathbb{Z}_{+}$ and $m = 0$ then ${(a^{n})}^{m} = 1 = a^{nm}$.

    If $n\in\mathbb{Z}_{+}$ and $-m\in\mathbb{Z}_{+}$ then ${(a^{n})}^{m} = 1/{(a^{n})}^{-m} = 1/{(a^{n(-m)})} = 1/a^{-nm} = a^{nm}$.

    If $n = 0$ then ${(a^{n})}^{m} = 1^{m} = 1 = a^{nm}$.

    If $-n\in\mathbb{Z}_{+}$ then ${(a^{n})}^{m} = {(1/a^{-n})}^{m} = 1/{(a^{-n})}^{m} = 1/{a^{(-n)m}} = 1/{a^{-(nm)}} = a^{nm}$.
\end{proof}

% chapter1:section4:exercise8
\begin{exercise}\label{chapter1:section4:exercise8}
    \begin{enumerate}[label={(\alph*)}]
        \item Show that $\mathbb{R}$ has the greatest lower bound property.
        \item Show that $\inf\{ 1/n \mid n\in\mathbb{Z}_{+} \} = 0$.
        \item Show that given $a$ with $0 < a < 1$, $\inf\{ a^{n} \mid n\in\mathbb{Z}_{+} \} = 0$.
    \end{enumerate}
\end{exercise}

\begin{proof}
    \begin{enumerate}[label={(\alph*)}]
        \item The result follows Exercise~\ref{chapter1:section3:exercise13}.
        \item $1/n > 0$ for $n\in\mathbb{Z}_{+}$ so $0$ is a lower bound of the given set. Let $\varepsilon > 0$. Because $\mathbb{Z}_{+}$ is not bounded above in $\mathbb{R}$, there exists $m\in\mathbb{Z}_{+}$ such that $m > \frac{1}{\varepsilon}$. Hence $\varepsilon > \frac{1}{m}$, so every positive real number is not a lower bound of the given set. Hence $\inf\{ 1/n \mid n\in\mathbb{Z}_{+} \} = 0$.
        \item $a^{n} > 0$ for $0 < a < 1$ and $n\in\mathbb{Z}_{+}$ so $0$ is a lower bound of the given set. By the Bernoulli's inequality
              \[
                  {\left(1/a\right)}^{n} = {\left(1 + (1/a - 1)\right)}^{n} \geq 1 + n(1/a - 1).
              \]

              Therefore $a^{n} \leq \frac{1}{1 + n(1/a - 1)}$. Let $\varepsilon > 0$. Because $\mathbb{Z}_{+}$ is not bounded above in $\mathbb{R}$, there is $n\in\mathbb{Z}_{+}$ such that $n > \frac{1/\varepsilon - 1}{1/a - 1}$ and thereby
              \[
                  a^{n} \leq \frac{1}{1 + n(1/a - 1)} < \varepsilon.
              \]

              Hence every positive real number is not a lower bound of the given set. Thus $\inf\{ a^{n} \mid n\in\mathbb{Z}_{+} \} = 0$ if $0 < a < 1$.
    \end{enumerate}
\end{proof}

% chapter1:section4:exercise9
\begin{exercise}\label{chapter1:section4:exercise9}
    \begin{enumerate}[label={(\alph*)}]
        \item Show that every nonempty subset of $\mathbb{Z}$ that is bounded above has a greatest element.
        \item If $x\notin\mathbb{Z}$, show there is exactly one $n\in\mathbb{Z}$ such that $n < x < n + 1$.
        \item If $x - y > 1$, show there is at least one $n\in \mathbb{Z}$ such that $y < n < x$.
        \item If $y < x$, show there is a rational number $z$ such that $y < z < x$.
    \end{enumerate}
\end{exercise}

\begin{proof}
    \begin{enumerate}[label={(\alph*)}]
        \item Let $A$ be a nonempty subset of $\mathbb{Z}$ that is bounded above. Let $b$ be an upper bound of $A$ such that $b\in\mathbb{Z}$ and define
              \[
                  A' = \{ b + 1 - a \mid a\in A \}
              \]

              then $A'$ is a subset of $\mathbb{Z}_{+}$ hence $A'$ has a smallest element $b + 1 - x$ where $x\in A$. Therefore $b + 1 - a\geq b + 1 - x$ for every $a\in A$. Hence $a\leq x$ for every $a\in A$. Thus $A$ has a greatest element.
        \item Let $A$ be the set of integers $a$ such that $a\leq x$. By part (a), we obtain that $A$ has a greatest element $n$. By the definition of $A$, $n + 1\notin A$, so $x < n + 1$. Moreover, $x\notin\mathbb{Z}$ so $n < x$. Hence $n < x < n + 1$.

              Assume $m$ is an integer such that $m < x < m + 1$, then $m$ is the largest integer not exceeding $x$. Therefore $m = n$. Thus there is exactly one $n\in\mathbb{Z}$ such that $n < x < n + 1$.
        \item By (b), there is an integer $m$ such that $m\leq y < m + 1$. Therefore $x = y + (x - y) > m + 1$. Hence $y < m + 1 < x$. Thus there is at least one $n\in\mathbb{Z}$ such that $y < n < x$.
        \item $x - y > 0$, so there exists a positive integer $n$ such that $x - y > \frac{1}{n}$. Therefore $n(x - y) > 1$. By part (c), there is an integer $m$ such that $ny < m < nx$, hence $y < \frac{m}{n} < x$. Thus there is a rational number $z$ such that $y < z < x$.
    \end{enumerate}
\end{proof}

% chapter1:section4:exercise10
\begin{exercise}\label{chapter1:section4:exercise10}
    Show that every positive number $a$ has exactly one positive square root, as follows:
    \begin{enumerate}[label={(\alph*)}]
        \item Show that if $x > 0$ and $0\leq h < 1$, then
              \begin{align*}
                  {(x + h)}^{2} & \leq x^{2} + h(2x + 1), \\
                  {(x - h)}^{2} & \geq x^{2} - h(2x).
              \end{align*}
        \item Let $x > 0$. Show that if $x^{2} < a$, then ${(x + h)}^{2} < a$ for some $h > 0$; and if $x^{2} > a$, then ${(x - h)}^{2} > a$ for some $h > 0$.
        \item Given $a > 0$, let $B$ be the set of all real numbers $x$ such that $x^{2} < a$. Show that $B$ is bounded above and contains at least one positive number. Let $b = \sup B$; show that $b^{2} = a$.
        \item Show that if $b$ and $c$ are positive and $b^{2} = c^{2}$, then $b = c$.
    \end{enumerate}
\end{exercise}

\begin{proof}
    \begin{enumerate}[label={(\alph*)}]
        \item Because $0\leq h < 1$, then $0\leq h^{2} \leq h$.
              \begin{align*}
                  {(x + h)}^{2} & = x^{2} + 2xh + h^{2} \leq x^{2} + 2xh + h = x^{2} + h(2x + 1), \\
                  {(x - h)}^{2} & = x^{2} - 2xh + h^{2} \geq x^{2} - h(2x).
              \end{align*}
        \item Suppose $x^{2} < a$ and $x > 0$. Let $h = \frac{a - x^{2}}{2x + 1 + (a - x^{2})}$ then $h < 1$ and $h < \frac{a - x^{2}}{2x + 1}$. Therefore ${(x + h)}^{2}\leq x^{2} + h(2x + 1) < x^{2} + (a - x^{2}) < a$.

              Suppose $x^{2} > a$ and $x > 0$. Let $h = \frac{x^{2} - a}{2x + (x^{2} - a)}$, then $0\leq h < 1$ and $h < \frac{x^{2} - a}{2x}$. Therefore ${(x - h)}^{2}\geq x^{2} - h(2x) > x^{2} - (x^{2} - a) = a$.
        \item Let $x\in B$. If $x\leq 0$ then $x < a + 1$. If $x > 0$, then $0 < x^{2} < a + a + a^{2} + 1 = {(a + 1)}^{2}$. Therefore $(x - a - 1)(x + a + 1) < 0$, which implies $x < a + 1$. Hence $B$ is bounded above.
              \[
                  {\left(\frac{a}{a + 1}\right)}^{2} = a\cdot \frac{a}{a+1}\cdot\frac{1}{a + 1} < a\cdot 1\cdot 1 = a.
              \]

              So $\frac{a}{a + 1}\in B$, which means $B$ contains at least one positive number.

              $b = \sup B$. If $b^{2} < a$ then part (b) implies there is $h\in\openinterval{0, 1}$ such that ${(b + h)}^{2} < a$, which is a contradiction because ${(b + h)}^{2} < a$ means $b$ is not $\sup B$. If $b^{2} > a$ then part (b) implies there is $h\in\openinterval{0, 1}$ such that ${(b - h)}^{2} > a$, which also contradicts the definition of least upper bound because $(b - h)$ is not $\sup B$. Hence the assumptions are false and we conclude $b^{2} = a$.
        \item If $b$ and $c$ are positive numbers such that $b^{2} = c^{2}$, then $(b - c)(b + c) = 0$. Because $b + c > 0$ so $b - c = 0$, which means $b = c$.
    \end{enumerate}
\end{proof}

% chapter1:section4:exercise11
\begin{exercise}\label{chapter1:section4:exercise11}
    Given $m\in\mathbb{Z}$, we say that $m$ is even if $m/2\in\mathbb{Z}$, and $m$ is odd otherwise.
    \begin{enumerate}[label={(\alph*)}]
        \item Show that if $m$ is odd, $m = 2n + 1$ for some $n\in\mathbb{Z}$.
        \item Show that if $p$ and $q$ are odd, so are $p\cdot q$ and $p^{n}$, for any $n\in\mathbb{Z}_{+}$.
        \item Show that if $a > 0$ is rational, then $a = m/n$ for some $m, n\in\mathbb{Z}_{+}$ where not both $m$ and $n$ are even.
        \item Theorem. $\sqrt{2}$ is irrational.
    \end{enumerate}
\end{exercise}

\begin{proof}
    \begin{enumerate}[label={(\alph*)}]
        \item If $m$ is odd, then $m/2\notin\mathbb{Z}$. By Exercise~\ref{chapter1:section4:exercise9}, there exists an integer $n$ such that $n < m/2 < n+1$. Therefore $2n < m < 2n + 2$. The only integer strictly between $2n$ and $2n + 2$ is $2n + 1$. Hence $m = 2n + 1$ for some $n\in\mathbb{Z}$.
        \item If $p$ and $q$ are odd then there exist integers $n, m$ such that $p = 2n + 1$ and $q = 2m + 1$.
              \[
                  p\cdot q = (2n + 1)(2m + 1) = 4nm + 2n + 2m + 1 = 2(2nm + n + m) + 1
              \]

              therefore $(p\cdot q)/2$ is not an integer. Hence $p\cdot q$ is odd.

              $p^{1} = p$ is odd. Assume $p^{n}$ is odd for some $n\in\mathbb{Z}_{+}$, then $p^{n+1} = p^{n}\cdot p$ is odd. Thus $p^{n}$ is odd for any $n\in\mathbb{Z}_{+}$, due to the principle of mathematical induction.
        \item If $a > 0$ and $a$ is rational, there exist integers $p, q$ such that $p, q > 0$ and $a = p/q$. Let $A = \{ x \mid x\in\mathbb{Z}_{+}\land x\cdot a\in\mathbb{Z}_{+} \}$ then $A$ is not empty because $q\in A$. By the well-ordering property of $\mathbb{Z}_{+}$, $A$ has a smallest element. Let $n$ be the smallest element of $A$ and $m = a\cdot n$. If $m$ and $n$ are even, then $m/2 = a\cdot (n/2)$, which contradicts $n$ being the smallest element of $A$. Thus there exist positive integers $m, n$ such that $a = m/n$ and $m, n$ are not both even.
        \item Assume $\sqrt{2}$ is rational, then there exist positive integers $m, n$ such that $\sqrt{2} = m/n$ and $m, n$ are not both even. Therefore $m^{2} = 2n^{2}$, so $m$ is even and there exists $p\in\mathbb{Z}_{+}$ such that $m = 2p$. It follows that $n^{2} = 2p^{2}$, and $n$ is even, which is a contradiction because $m, n$ are not both even. Thus $\sqrt{2}$ is irrational.
    \end{enumerate}
\end{proof}

\section{Cartesian Products}

% chapter1:section5:exercise1
\begin{exercise}\label{chapter1:section5:exercise1}
    Show there is a bijective correspondence of $A\times B$ and $B\times A$.
\end{exercise}

\begin{proof}
    Define $f: A\times B\to B\times A$ as follows
    \[
        f(a, b) = (b, a).
    \]

    We have
    \begin{align*}
        f(a_{0}, b_{0}) = f(a_{1}, b_{1}) & \Longleftrightarrow (b_{0}, a_{0}) = (b_{1}, a_{1})   \\
                                          & \Longleftrightarrow b_{0} = b_{1} \land a_{0} = a_{1} \\
                                          & \Longleftrightarrow (a_{0}, b_{0}) = (a_{1}, b_{1}).
    \end{align*}

    So $f$ is injective.

    Moreover, for every $(b, a)\in B\times A$, $(a, b)\in A\times B$ and $f(a, b) = (b, a)$. Therefore $f$ is surjective.

    Thus there is a bijective correspondence of $A\times B$ and $B\times A$.
\end{proof}

% chapter1:section5:exercise2
\begin{exercise}\label{chapter1:section5:exercise2}
    \begin{enumerate}[label={(\alph*)}]
        \item Show that if $n > 1$ there is a bijective correspondence of
              \[
                  A_{1}\times\cdots\times A_{n} \quad\text{with}\quad (A_{1}\times\cdots\times A_{n-1})\times A_{n}.
              \]
        \item Given the indexed family $\{ A_{1}, A_{2}, \ldots \}$, let $B_{i} = A_{2i-1}\times A_{2i}$ for each positive integer $i$. Show that there is bijective correspondence of $A_{1}\times A_{2}\times \cdots$ with $B_{1}\times B_{2}\times\cdots$.
    \end{enumerate}
\end{exercise}

\begin{proof}
    \begin{enumerate}[label={(\alph*)}]
        \item There is a bijective correspondence of $A_{1}$ and $A_{1}$, which is $i_{A_{1}}$.

              Define $f: A_{1}\times \cdots \times A_{n} \to (A_{1}\times \cdots \times A_{n-1}) \times A_{n}$ as follows
              \[
                  f(a_{1}, \ldots, a_{n}) = ((a_{1}, \ldots, a_{n-1}), a_{n}).
              \]

              We have
              \begin{align*}
                  f(a_{1}, \ldots, a_{n}) = f(b_{1}, \ldots, b_{n}) & \Longleftrightarrow ((a_{1}, \ldots, a_{n-1}), a_{n}) = ((b_{1}, \ldots, b_{n-1}), b_{n})   \\
                                                                    & \Longleftrightarrow (a_{1}, \ldots, a_{n-1}) = (b_{1}, \ldots, b_{n-1}) \land a_{n} = b_{n} \\
                                                                    & \Longleftrightarrow \forall k\in S_{n+1}\, a_{k} = b_{k}                                    \\
                                                                    & \Longleftrightarrow (a_{1}, \ldots, a_{n}) = (b_{1}, \ldots, b_{n})
              \end{align*}

              so $f$ is injective.

              Moreover, for every $((a_{1}, \ldots, a_{n-1}), a_{n})\in (A_{1}\times \cdots \times A_{n-1}) \times A_{n}$,
              \[
                  f(a_{1}, \ldots, a_{n}) = ((a_{1}, \ldots, a_{n-1}), a_{n}).
              \]

              So $f$ is surjective.

              Thus there is a bijective correspondence of $A_{1}\times\cdots\times A_{n}$ with $(A_{1}\times\cdots\times A_{n-1})\times A_{n}$.
        \item Define $f: A_{1}\times A_{2}\times \cdots \to B_{1} \times B_{2} \times\cdots$ as follows
              \[
                  f(\mathbf{a}) = \mathbf{b}
              \]

              where $\mathbf{b}_{i} = (\mathbf{a}_{2i-1}, \mathbf{a}_{2i})$ for every positive integer $i$.
              \begin{align*}
                  f(\mathbf{a}) = f(\mathbf{a}') & \Longleftrightarrow \forall i\in\mathbb{Z}_{+}\, (\mathbf{a}_{2i-1}, \mathbf{a}_{2i}) = (\mathbf{a}'_{2i-1}, \mathbf{a}'_{2i}) \\
                                                 & \Longleftrightarrow \forall i\in\mathbb{Z}_{+}\, \mathbf{a}_{i} = \mathbf{a}'_{i}                                              \\
                                                 & \Longleftrightarrow \mathbf{a} = \mathbf{a}'
              \end{align*}

              so $f$ is injective.

              Let $\mathbf{b}\in B_{1}\times B_{2}\times \cdots$. Define $\mathbf{c}\in A_{1}\times A_{2}\times \cdots$ as follows
              \begin{align*}
                  \mathbf{c}_{2i-1} & = \mathbf{b}_{i}(1) \\
                  \mathbf{c}_{2i}   & = \mathbf{b}_{i}(2)
              \end{align*}

              for every positive integer $i$. Then $f(\mathbf{c}) = \mathbf{b}$, which means $f$ is surjective.

              Thus there is a bijective correspondence of $A_{1}\times A_{2}\times \cdots$ with $B_{1}\times B_{2}\times \cdots$
    \end{enumerate}
\end{proof}

% chapter1:section5:exercise3
\begin{exercise}\label{chapter1:section5:exercise3}
    Let $A = A_{1}\times A_{2}\times \cdots$ and $B = B_{1}\times B_{2}\times\cdots$.
    \begin{enumerate}[label={(\alph*)}]
        \item Show that if $B_{i}\subset A_{i}$ for all $i$, then $B\subset A$.
        \item Show the converse of (a) holds if $B$ is nonempty.
        \item Show that if $A$ is nonempty, each $A_{i}$ is nonempty. Does the converse hold?
        \item What is the relation between the set $A\cup B$ and the cartesian product of the sets $A_{i}\cup B_{i}$? What is the relation between the set $A\cap B$ and the cartesian product of the sets $A_{i}\cap B_{i}$?
    \end{enumerate}
\end{exercise}

\begin{proof}
    \begin{enumerate}[label={(\alph*)}]
        \item Let $\mathbf{x}\in B$. For every positive integer $i$, $\mathbf{x}_{i}\in B_{i}\subset A_{i}$. So $\mathbf{x}\in A$, hence $B\subset A$.
        \item Suppose $B$ is nonempty and $B\subset A$. Then $B_{i}$ is nonempty for every positive integer $i$.

              For every positive integer $i$, let $b_{i}\in B_{i}$. Let $\mathbf{x}\in B$ such that $\mathbf{x}_{i} = b_{i}$. Because $B\subset A$, $\mathbf{x}\in A$, so $b_{i} = \mathbf{x}_{i}\in A_{i}$. Therefore $B_{i}\subset A_{i}$ for all positive integers $i$.
        \item Suppose $A$ is nonempty. Assume there is a positive integer $i$ such that $A_{i}$ is empty, then $A$ is empty, which is a contradiction. Hence each $A_{i}$ is nonempty.

              I believe that the converse hold.
        \item Suppose $\mathbf{x}\in A\cup B$ then $\mathbf{x}\in A$ or $\mathbf{x}\in B$. If $\mathbf{x}\in A$ then $\mathbf{x}_{i}\in A_{i}$ for every positive integer $i$.  If $\mathbf{x}\in B$ then $\mathbf{x}_{i}\in B_{i}$ for every positive integer $i$. Therefore $\mathbf{x}\in\prod_{i\in\mathbb{Z}_{+}}A_{i}\cup B_{i}$, hence
              \[
                  A\cup B \subset \prod_{i\in\mathbb{Z}_{+}}A_{i}\cup B_{i},
              \]

              Moreover the inclusion in reversed order is false.

              Suppose $\mathbf{x}\in A\cap B$.
              \begin{align*}
                  \mathbf{x}\in A\cap B & \Longleftrightarrow \mathbf{x}\in A\land \mathbf{x}\in B                          \\
                                        & \Longleftrightarrow \forall i\in\mathbb{Z}_{+}\,\mathbf{x}_{i}\in A_{i}\cap B_{i} \\
                                        & \Longleftrightarrow x\in\prod_{i\in\mathbb{Z}_{i}}A_{i}\cap B_{i}.
              \end{align*}

              Hence
              \[
                  A\cap B = \prod_{i\in\mathbb{Z}_{+}}A_{i}\cap B_{i}.
              \]
    \end{enumerate}
\end{proof}

% chapter1:section5:exercise4
\begin{exercise}\label{chapter1:section5:exercise4}
    Let $m, n\in\mathbb{Z}_{+}$. Let $X\ne \varnothing$.
    \begin{enumerate}[label={(\alph*)}]
        \item If $m\leq n$, find an injective map $f: X^{m}\to X^{n}$.
        \item Find a bijective map $g: X^{m}\times X^{n}\to X^{m+n}$.
        \item Find an injective map $h: X^{n}\to X^{\omega}$.
        \item Find a bijective map $k: X^{n}\times X^{\omega}\to X^{\omega}$.
        \item Find a bijective map $\ell: X^{\omega}\times X^{\omega}\to X^{\omega}$.
        \item If $A\subset B$, find an injective map $m: {(A^{\omega})}^{n}\to B^{\omega}$.
    \end{enumerate}
\end{exercise}

\begin{proof}
    \begin{enumerate}[label={(\alph*)}]
        \item I define $f: X^{m}\to X^{n}$ as follows
              \begin{align*}
                  f(\mathbf{a})  & = \mathbf{b}                                 \\
                  \mathbf{b}_{i} & = \begin{cases}
                                         \mathbf{a}_{i} & \text{if $1\leq i\leq m$} \\
                                         0              & \text{if $m < i\leq n$}
                                     \end{cases}
              \end{align*}

              Then $f(\mathbf{x}) = f(\mathbf{y})$ if and only if $\mathbf{x}_{i} = \mathbf{y}_{i}$ for every positive integer $i$ not exceeding $m$, which precisely means $\mathbf{x} = \mathbf{y}$. Hence $f$ is injective.
        \item I define $g: X^{m}\times X^{n}\to X^{m+n}$ as follows
              \begin{align*}
                  g(\mathbf{a}, \mathbf{b}) & = \mathbf{c}                                   \\
                  \mathbf{c}_{i}            & = \begin{cases}
                                                    \mathbf{a}_{i}   & \text{if $1\leq i\leq m$} \\
                                                    \mathbf{b}_{i-m} & \text{if $m < i\leq m+n$}
                                                \end{cases}
              \end{align*}

              By this definition $g(\mathbf{a}, \mathbf{b}) = g(\mathbf{x}, \mathbf{y})$ if and only if $\mathbf{a}_{i} = \mathbf{x}_{i}$ for every $i\in S_{m+1}$ and $\mathbf{b}_{i} = \mathbf{y}_{i}$ for every $i\in S_{n+1}$, which means $(\mathbf{a}, \mathbf{b}) = (\mathbf{x}, \mathbf{y})$. Therefore $f$ is injective.

              Let $\mathbf{c}\in X^{m+n}$, let's define $\mathbf{a}\in X^{m}$ and $\mathbf{b}\in X^{n}$ by $\mathbf{a}_{i} = c_{i}$ for every $i\in S_{m+1}$ and $\mathbf{b}_{i} = c_{i-m}$ for every positive integer $i$ such that $m+1\leq i\leq m+n$. Then by the definition of $g$, $g(\mathbf{a}, \mathbf{b}) = \mathbf{c}$, which means $g$ is surjective.

              Thus $g$ is bijective.
        \item I define $h: X^{m}\to X^{\omega}$ as follows
              \begin{align*}
                  h(\mathbf{a})  & = \mathbf{b}                                 \\
                  \mathbf{b}_{i} & = \begin{cases}
                                         \mathbf{a}_{i} & \text{if $1\leq i\leq m$} \\
                                         0              & \text{if $i > m$}
                                     \end{cases}
              \end{align*}

              Then $h(\mathbf{x}) = h(\mathbf{y})$ if and only if $\mathbf{x}_{i} = \mathbf{y}_{i}$ for every positive integer $i$ not exceeding $m$, which precisely means $\mathbf{x} = \mathbf{y}$. Hence $h$ is injective.
        \item I define $k: X^{n}\times X^{\omega}\to X^{\omega}$ as follows
              \begin{align*}
                  k(\mathbf{a}, \mathbf{b}) & = \mathbf{c}                                   \\
                  \mathbf{c}_{i}            & = \begin{cases}
                                                    \mathbf{a}_{i}   & \text{if $1\leq i\leq n$} \\
                                                    \mathbf{b}_{i-n} & \text{if $i > n$}
                                                \end{cases}
              \end{align*}

              Then $k(\mathbf{a}, \mathbf{b}) = k(\mathbf{x}, \mathbf{y})$ if and only if $\mathbf{a}_{i} = \mathbf{x}_{i}$ for every $i\in S_{n+1}$ and $\mathbf{b}_{i} = \mathbf{y}_{i}$ for every $i\in \mathbb{Z}_{+}$, which means $(\mathbf{a}, \mathbf{b}) = (\mathbf{x}, \mathbf{y})$. Therefore $k$ is injective.

              Let $\mathbf{c}\in X^{\omega}$, let's define $\mathbf{a}\in X^{n}$ and $\mathbf{b}\in X^{\omega}$ as follows
              \begin{align*}
                  \mathbf{a}_{i} & = \mathbf{c}_{i} \quad \text{for $i\in S_{n+1}$}         \\
                  \mathbf{b}_{i} & = \mathbf{c}_{i+n} \quad \text{for $i\in\mathbb{Z}_{+}$}
              \end{align*}

              then $k(\mathbf{a}, \mathbf{b}) = \mathbf{c}$, hence $k$ is surjective.

              Thus $k$ is bijective.
        \item I define $\ell: X^{\omega}\times X^{\omega}\to X^{\omega}$ as follows
              \begin{align*}
                  \ell(\mathbf{a}, \mathbf{b}) & = \mathbf{c}                                                 \\
                  \mathbf{c}_{2i-1}            & = \mathbf{a}_{i}\qquad \text{for every $i\in\mathbb{Z}_{+}$} \\
                  \mathbf{c}_{2i}              & = \mathbf{b}_{i}\qquad \text{for every $i\in\mathbb{Z}_{+}$}
              \end{align*}

              Then $\ell(\mathbf{a}, \mathbf{b}) = \ell(\mathbf{x}, \mathbf{y})$ if and only if $\mathbf{a}_{i} = \mathbf{x}_{i}$ for every positive integer $i$ and $\mathbf{b}_{i} = \mathbf{y}_{i}$ for every positive integer $i$, which means $\mathbf{a} = \mathbf{x}$ and $\mathbf{b} = \mathbf{y}$. Therefore $\ell$ is injective.

              Let $\mathbf{c}\in X^{\omega}$, let's define $\mathbf{a}\in X^{\omega}$ and $\mathbf{b}\in X^{\omega}$ as follows: $\mathbf{a}_{i} = \mathbf{c}_{2i-1}$ and $\mathbf{b}_{i} = \mathbf{c}_{2i}$ for every positve integer $i$. Therefore $\ell(\mathbf{a}, \mathbf{b}) = \mathbf{c}$, and hence $\ell$ is surjective.

              Thus $\ell$ is bijective.
        \item In this solution, I will use the following result in arithmetics: For every positive integer $p$, there exist unique positive integers $j, r$ such that $p = n\cdot (j - 1) + r$ where $1\leq r\leq n$.

              I define $m: {(A^{\omega})}^{n}\to B^{\omega}$ as follows
              \begin{align*}
                  f(\mathbf{A})                 & = \mathbf{b}                                                                                          \\
                  \mathbf{b}_{n\cdot (i-1) + k} & = \mathbf{A}_{k, i}\qquad\text{for $i\in\mathbb{Z}_{+}$ and $k\in\mathbb{Z}_{+}$ and $1\leq k\leq n$}
              \end{align*}

              then $m$ is bijective due to the definition of $m$ and the arithmetics result above.
    \end{enumerate}
\end{proof}

% chapter1:section5:exercise5
\begin{exercise}\label{chapter1:section5:exercise5}
    Which of the following subsets of $\mathbb{R}^{\omega}$ can be expressed as the cartesian product of subsets of $\mathbb{R}$?
    \begin{enumerate}[label={(\alph*)}]
        \item $\{ \mathbf{x} \mid \text{$x_{i}$ is an integer for all $i$} \}$.
        \item $\{ \mathbf{x} \mid \text{$x_{i}\geq i$ for all $i$} \}$.
        \item $\{ \mathbf{x} \mid \text{$x_{i}$ is an integer for all $i\geq 100$} \}$.
        \item $\{ \mathbf{x} \mid x_{2} = x_{3} \}$.
    \end{enumerate}
\end{exercise}

\begin{proof}
    \begin{enumerate}[label={(\alph*)}]
        \item This set is $\mathbb{Z}^{\omega}$.
        \item This set is
              \[
                  \prod_{n\in\mathbb{Z}_{+}}\{ n, n+1, \ldots \} = \prod_{n\in\mathbb{Z}_{+}}(\mathbb{Z}_{+} - S_{n}).
              \]
        \item This set is
              \[
                  \prod_{n\in\mathbb{Z}_{+}}A_{n}
              \]

              where $A_{n} = \mathbb{R}$ if $n < 100$ and $A_{n} = \mathbb{Z}$ if $n\geq 100$.
        \item Assume this set can be expressed as the cartesian product of subsets of $\mathbb{R}$
              \[
                  \prod_{n\in\mathbb{Z}_{+}}A_{n}.
              \]

              Due to the definition, $A_{n} = \mathbb{R}$ for positive integers $n$ such that $n\ne 2, n\ne 3$. Let $\mathbf{a}, \mathbf{b}$ be elements of this set such that $\mathbf{a}_{2} = \mathbf{a}_{3} = 0$ and $\mathbf{b}_{2} = \mathbf{b}_{3} = 1$. Let $\mathbf{c}\in \mathbb{R}^{\omega}$ such that $\mathbf{c}_{2} = 0$ and $\mathbf{c}_{3} = 1$. Because $\mathbf{c}_{i}\in A_{i}$ for every positive integer, then $\mathbf{c}$ is an element of the given set, which is a contradiction because $\mathbf{c}_{2}\ne \mathbf{c}_{3}$. Hence the given set cannot be expressed as the cartesian product of subsets of $\mathbb{R}$.
    \end{enumerate}
\end{proof}

\section{Finite Sets}

% chapter1:section6:exercise1
\begin{exercise}\label{chapter1:section6:exercise1}
    \begin{enumerate}[label={(\alph*)}]
        \item Make a list of all the injective maps
              \[
                  f: \{ 1, 2, 3 \}\to \{ 1, 2, 3, 4 \}.
              \]

              Show that none is bijective.
        \item How many injective maps
              \[
                  f: \{ 1, \ldots, 8 \}\to \{ 1, \ldots, 10 \}
              \]

              are there?
    \end{enumerate}
\end{exercise}

\begin{proof}
    \begin{enumerate}[label={(\alph*)}]
        \item All injective maps $f: \{ 1, 2, 3 \}\to \{ 1, 2, 3, 4 \}$ are
              \begin{align*}
                  \{ (1, 1), (2, 2), (3, 3) \}, \\
                  \{ (1, 1), (2, 2), (3, 4) \}, \\
                  \{ (1, 1), (2, 3), (3, 4) \}, \\
                  \{ (1, 1), (2, 3), (3, 2) \}, \\
                  \{ (1, 1), (2, 4), (3, 2) \}, \\
                  \{ (1, 1), (2, 4), (3, 3) \}, \\
                  \{ (1, 2), (2, 1), (3, 3) \}, \\
                  \{ (1, 2), (2, 1), (3, 4) \}, \\
                  \{ (1, 2), (2, 3), (3, 4) \}, \\
                  \{ (1, 2), (2, 3), (3, 1) \}, \\
                  \{ (1, 2), (2, 4), (3, 2) \}, \\
                  \{ (1, 2), (2, 4), (3, 3) \}, \\
                  \{ (1, 3), (2, 1), (3, 2) \}, \\
                  \{ (1, 3), (2, 1), (3, 4) \}, \\
                  \{ (1, 3), (2, 2), (3, 1) \}, \\
                  \{ (1, 3), (2, 2), (3, 4) \}, \\
                  \{ (1, 3), (2, 4), (3, 1) \}, \\
                  \{ (1, 3), (2, 4), (3, 2) \}, \\
                  \{ (1, 4), (2, 1), (3, 2) \}, \\
                  \{ (1, 4), (2, 1), (3, 3) \}, \\
                  \{ (1, 4), (2, 2), (3, 1) \}, \\
                  \{ (1, 4), (2, 2), (3, 3) \}, \\
                  \{ (1, 4), (2, 3), (3, 1) \}, \\
                  \{ (1, 4), (2, 3), (3, 2) \}, \\
              \end{align*}

              and none of these are bijective because there always is an element with empty preimage.
        \item There are $8!\cdot \binom{10}{8}$ of such injective maps.
    \end{enumerate}
\end{proof}

% chapter1:section6:exercise2
\begin{exercise}\label{chapter1:section6:exercise2}
    Show that if $B$ is not finite and $B\subset A$, then $A$ is not finite.
\end{exercise}

\begin{proof}
    Assume $A$ is finite, then every subset of $A$ is finite, which implies $B$ is finite, which is a contradiction. Hence $A$ is not finite.
\end{proof}

% chapter1:section6:exercise3
\begin{exercise}\label{chapter1:section6:exercise3}
    Let $X$ be the two-element set $\{ 0, 1 \}$. Find a bijective correspondence between $X^{\omega}$ and a proper subset of itself.
\end{exercise}

\begin{proof}
    Let's define
    \[
        A = \{ \mathbf{x} \mid \mathbf{x}\in X^{\omega} \land \mathbf{x}_{1} = 0 \}
    \]

    then $A$ is a proper subset of $X^{\omega}$, because there is $\mathbf{x}\in X^{\omega}$ such that $\mathbf{x}_{1} = 1$.

    Let $f: X^{\omega}\to A$ defined by
    \begin{align*}
        f(\mathbf{x})  & = \mathbf{a}                           \\
        \mathbf{a}_{i} & = \begin{cases}
                               0                & \text{if $i = 1$} \\
                               \mathbf{x}_{i-1} & \text{if $i > 1$}
                           \end{cases}
    \end{align*}

    Then $f(\mathbf{x}) = f(\mathbf{y})$ if and only if $\mathbf{x}_{i} = \mathbf{y}_{i}$ for every positive integer $i$. Therefore $f$ is injective.

    If $\mathbf{a}\in A$, let $\mathbf{x}\in X^{\omega}$ such that $\mathbf{x}_{i} = \mathbf{a}_{i+1}$ for every positive integer $i$, then $f(\mathbf{x}) = \mathbf{a}$. Therefore $f$ is surjective.

    Hence $f$ is bijective. Thus there is a bijective correspondence between $X^{\omega}$ and a proper subset of itself.
\end{proof}

% chapter1:section6:exercise4
\begin{exercise}\label{chapter1:section6:exercise4}
    Let $A$ be a nonempty finite simply ordered set.
    \begin{enumerate}[label={(\alph*)}]
        \item Show that $A$ has a largest element.
        \item Show that $A$ has the order type of a section of the positive integers.
    \end{enumerate}
\end{exercise}

\begin{proof}
    \begin{enumerate}[label={(\alph*)}]
        \item If $A$ has cardinality $1$ then $A$ has a single element, which is also the largest element.

              Assume the statement is true for simply ordered set of cardinality $n$ (where $n$ is a positive integer). Let $A$ be nonempty finite simply ordered set of cardinality $(n + 1)$. Let $a_{1}$ be an element of $A$ then $A - \{ a_{1} \}$ has cardinality $n$, which means $A - \{ a_{1} \}$ has a largest element $a$. Let $x$ be the larger element among $a_{1}$ and $a$ then $x$ is the largest element of $A$.

              Thus by the principle of mathematical induction, every nonempty finite simply orderded set has a largest element.
        \item If the cardinality of $A$ is $1$ then $A$ has the order type of $S_{2}$.

              Assume the statement is true for nonempty finite set of cardinality $n$ where $n$ is some positive integer. Suppose $A$ has cardinality $(n + 1)$. By $(a)$, $A$ has a largest element $a_{1}$. $A - \{ a_{1} \}$ has the order type of $S_{n+1}$, due to the induction hypothesis. Therefore $A$ has the order type of $S_{n+2}$.

              Thus by the principle of mathematical induction, $A$ has the order type of a section of the positive integers.
    \end{enumerate}
\end{proof}

% chapter1:section6:exercise5
\begin{exercise}\label{chapter1:section6:exercise5}
    If $A\times B$ is finite, does it follow that $A$ and $B$ are finite?
\end{exercise}

\begin{proof}
    No. If $A$ is the empty set and $B$ is not finite, then $A\times B$ is the empty set, which is finite.
\end{proof}

% chapter1:section6:exercise6
\begin{exercise}\label{chapter1:section6:exercise6}
    \begin{enumerate}[label={(\alph*)}]
        \item Let $A = \{ 1,\ldots,n \}$. Show there is a bijection of $\mathscr{P}(A)$ with the cartesian product $X^{n}$, where $X$ is the two-element set $X = \{ 0, 1 \}$.
        \item Show that if $A$ is finite, then $\mathscr{P}(A)$ is finite.
    \end{enumerate}
\end{exercise}

\begin{proof}
    \begin{enumerate}[label={(\alph*)}]
        \item Let's define $f: \mathscr{P}(A)\to X^{n}$ as follows
              \begin{align*}
                  f(B)           & = \mathbf{x}                \\
                  \mathbf{x}_{i} & = \begin{cases}
                                         1 & \text{if $i\in B$}    \\
                                         0 & \text{if $i\notin B$}
                                     \end{cases}
              \end{align*}

              where $B\subset A$.

              Suppose $f(B) = f(C) = \mathbf{x}$.
              \[
                  i\in B \Longleftrightarrow \mathbf{x}_{i} = 1 \Longleftrightarrow i\in C.
              \]

              Hence $B = C$. So $f$ is injective.

              Let $\mathbf{x}\in X^{n}$, let's define $B\subset A$ as follows
              \[
                  B = \{ i \mid i\in A \land x_{i} = 1 \}
              \]

              then by the definition of $f$, $f(B) = \mathbf{x}$, which implies $f$ is surjective.

              Thus $f$ is a bijection of $\mathscr{P}(A)$ with the cartesian product $X^{n}$.
        \item The cardinality of $X^{0}$ is $1$.

              Assume the cardinality of $X^{n}$ is $2^{n}$. Then there exists a bijection $f: X^{n}\to S_{2^{n} + 1}$. Let's define $g: X^{n+1}\to S_{2^{n+1}+1}$ as follows
              \begin{align*}
                  g(\mathbf{x}) = \begin{cases}
                                      f(\mathbf{y})         & \text{if $\mathbf{x}_{1} = 0$, where $\mathbf{y}_{i} = \mathbf{x}_{i-1}$ for $i > 1$} \\
                                      2^{n} + f(\mathbf{z}) & \text{if $\mathbf{x}_{1} = 1$, where $\mathbf{z}_{i} = \mathbf{x}_{i-1}$ for $i > 1$}
                                  \end{cases}
              \end{align*}

              then $g$ is bijective.

              Thus by the principle of mathematical induction, the cardinality of $X^{n}$ is $2^{n}$ for nonnegative integer $n$.

              Suppose $A$ is finite, then the cardinality of $A$ is $n$ for some nonnegative integer $n$. The cardinality of $X^{n}$ is $2^{n}$. Therefore the cardinality of $\mathscr{P}(A)$ is $2^{n}$. Thus if $A$ is finite then $\mathscr{P}(A)$ is finite.
    \end{enumerate}
\end{proof}

% chapter1:section6:exercise7
\begin{exercise}\label{chapter1:section6:exercise7}
    If $A$ and $B$ are finite, show that the set of all functions $f: A\to B$ is finite.
\end{exercise}

\begin{proof}
    If $A$ is empty, then there is only one function $f: A\to B$, which is the empty function.

    If $B$ is empty and $A$ is not empty then there is no function from $A$ to $B$, because this violates the definition of functions (a function from $A$ to $B$ assigns every element of $A$ to a unique element of $B$).

    Suppose $A$ and $B$ are both nonempty finite sets.

    If $B$ has cardinality $1$ then there is only one function $f: A\to B$. So we'll work on the case $B$ has cardinality larger than $1$.

    If $A$ has cardinality $1$ then the set of all functions $f: A\to B$ has the same cardinality as $B$. Denote the cardinality of $B$ by $b$. Let $k$ be a bijection from $S_{b+1}$ onto $B$.

    Assume if $A$ has cardinality $n$ then the set $B^{A}$ of all functions $f: A\to B$ has cardinality $b^{n}$. Suppose $A$ has cardinality $(n + 1)$. Let $a_{0}$ be an element of $A$ and $g$ be a bijection from $B^{A - \{ a_{0} \}}$ onto $S^{b^{n} + 1}$.

    Let's define $h: B^{A}\to S_{b^{n+1} + 1}$ as follows
    \[
        h(f) = (i - 1)\cdot b^{n} + g(f\vert_{A - \{ a_{0} \}}) \qquad\text{if $f(a_{0}) = k(i)$}
    \]

    then $h$ is a bijection. Therefore the cardinality of $B^{A}$ is $b^{n+1}$.

    Thus if $A$ and $B$ are finite
    \[
        \operatorname{card}B^{A} = \begin{cases}
            0                                               & \text{if $A$ is not empty and $B$ is empty} \\
            1                                               & \text{if $A$ is empty}                      \\
            {(\operatorname{card}B)}^{\operatorname{card}A} & \text{if $A, B$ are finite and not empty}
        \end{cases}
    \]
\end{proof}

\section{Countable and Uncountable Sets}

% chapter1:section7:exercise1
\begin{exercise}\label{chapter1:section7:exercise1}
    Show that $\mathbb{Q}$ is countably infinite.
\end{exercise}

\begin{proof}
    Let $f: \mathbb{Q}\to \mathbb{Z}\times\mathbb{Z}_{+}$ such that
    \[
        f(m/n) = (m, n)
    \]

    where $m/n$ is already in lowest terms. $f$ is injective.

    Let $g: \mathbb{Z}\times\mathbb{Z}_{+}\to \mathbb{Z}_{+}\times\mathbb{Z}_{+}$ such that
    \[
        g(x, y) = \begin{cases}
            (2x, y)     & \text{if $x > 0$}   \\
            (1 - 2x, y) & \text{if $x\leq 0$}
        \end{cases}
    \]

    then $g$ is injective. Because $\mathbb{Z}_{+}\times\mathbb{Z}_{+}$ is countably infinite and $\mathbb{Z}\times\mathbb{Z}_{+}$ is infinite, it follows that $\mathbb{Z}\times\mathbb{Z}_{+}$ is countably infinite. Therefore $\mathbb{Q}$ is countably infinite.
\end{proof}

% chapter1:section7:exercise2
\begin{exercise}\label{chapter1:section7:exercise2}
    Show that the maps $f$ and $g$ of Examples 1 and 2 are bijections.
\end{exercise}

\begin{proof}[Proof of Example 1]
    $f: \mathbb{Z}\to \mathbb{Z}_{+}$ defined by
    \[
        f(n) = \begin{cases}
            2n      & \text{if $n > 0$,}  \\
            -2n + 1 & \text{if $n\leq 0$}
        \end{cases}
    \]

    Suppose $f(m) = f(n)$. If $m > 0$ and $n\leq 0$ or $m\leq 0$ and $n > 0$ then $f(m)\ne f(n)$, so $m, n > 0$ or $m, n\leq 0$. If $m, n > 0$ then $2m = 2n$, which implies $m = n$. If $m, n\leq 0$ then $-2m + 1 = -2n + 1$, which implies $m = n$. Therefore $f$ is injective.

    Let $n$ be a positive integer. If $n$ is even, then $f(n/2) = n$. If $n$ is odd, then $f((1 - n)/2) = n$. Therefore $f$ is surjective.

    Thus $f$ is a bijection.
\end{proof}

\begin{proof}[Proof of Example 2]
    $f: \mathbb{Z}_{+}\times\mathbb{Z}_{+} \to A$, $A$ is a subset of $\mathbb{Z}_{+}\times\mathbb{Z}_{+}$ consisting pairs $(x, y)$ for which $y\leq x$, and
    \[
        f(x, y) = (x + y - 1, y).
    \]

    Suppose $f(x_{0}, y_{0}) = f(x_{1}, y_{1})$ then $(x_{0} + y_{0} - 1, y_{0}) = (x_{1} + y_{1} - 1, y_{1})$. It follows that $y_{0} = y_{1}$ and $x_{0} + y_{0} - 1 = x_{1} + y_{1} - 1$, therefore $x_{0} = x_{1}$. Hence $f$ is injective.

    Let $(x, y)\in A$, then $f(x - y + 1, y) = (x - y + 1 + y - 1, y) = (x, y)$. Therefore $f$ is surjective.

    Hence $f$ is bijective.

    \bigskip
    $g: A\to\mathbb{Z}_{+}$ and
    \[
        g(x, y) = \frac{1}{2}(x - 1)x + y.
    \]

    Suppose $g(x_{0}, y_{0}) = g(x_{1}, y_{1})$ then $\frac{1}{2}(x_{0} - 1)x_{0} + y_{0} = \frac{1}{2}(x_{1} - 1)x_{1} + y_{1}$. Equivalently
    \[
        \frac{1}{2}(x_{0} - x_{1})(x_{0} + x_{1} - 1) + y_{0} = y_{1}.
    \]

    If $x_{0} > x_{1}$, then
    \[
        y_{1} = \frac{1}{2}(x_{0} - x_{1})(x_{0} + x_{1} - 1) + y_{0} \geq \frac{1}{2}\cdot 1\cdot (2x_{1} + 1) > x_{1}
    \]

    which is a contradiction because $x_{1}\geq y_{1}$.

    If $x_{0} < x_{1}$ then
    \[
        y_{0} = \frac{1}{2}(x_{1} - x_{0})(x_{0} + x_{1} - 1) + y_{1} \geq \frac{1}{2}\cdot 2\cdot (2x_{0} + 1) > x_{0}
    \]

    which is a contradiction because $x_{0}\geq y_{0}$.

    Hence $x_{0} = x_{1}$ and $y_{0} = y_{1}$. Therefore $g$ is injective.

    Let $n$ be a positive integer. Let $x$ be the largest integer such that $\frac{1}{2}(x - 1)x < n$, then $\frac{1}{2}x(x + 1)\geq n$. Let $y = n - \frac{1}{2}(x - 1)x$ then $x\geq y$ because $x\geq n - \frac{1}{2}(x - 1)x$ is equivalent to $\frac{1}{2}x(x + 1)\geq n$. Hence $g(x, y) = n$, which means $g$ is surjective.

    Thus $g$ is bijective.
\end{proof}

% chapter1:section7:exercise3
\begin{exercise}\label{chapter1:section7:exercise3}
    Let $X$ be the two-element set $\{ 0, 1 \}$. Show there is a bijective correspondence between the set $\mathscr{P}(\mathbb{Z}_{+})$ and the cartesian product $X^{\omega}$.
\end{exercise}

\begin{proof}
    I define $f: \mathscr{P}(\mathbb{Z}_{+})\to X^{\omega}$ as follows
    \[
        f(A) = \mathbf{x}
    \]

    where $A\subset\mathbb{Z}_{+}$
    \[
        \mathbf{x}_{n} = \begin{cases}
            1 & \text{if $n\in A$}    \\
            0 & \text{if $n\notin A$}
        \end{cases}
    \]

    Suppose $f(A) = f(B) = \mathbf{x}$. $n\in A$ if and only if $\mathbf{x}_{n} = 1$. $n\in B$ if and only if $\mathbf{x}_{n} = 1$. Therefore $A = B$, which means $f$ is injective.

    Let $\mathbf{x}\in X^{\omega}$ and define
    \[
        A = \{ n \mid n\in\mathbb{Z}_{+} \land \mathbf{x}_{n} = 1 \}
    \]

    then $f(A) = \mathbf{x}$. Hence $f$ is surjective.

    Thus $f$ is a bijective correspondence between $\mathscr{P}(\mathbb{Z}_{+})$ and $X^{\omega}$.
\end{proof}

% chapter1:section7:exercise4
\begin{exercise}\label{chapter1:section7:exercise4}
    \begin{enumerate}[label={(\alph*)}]
        \item A real number $x$ is said to be \textbf{algebraic} (over the rationals) if it satisfies some polynomial equation of positive degree
              \[
                  x^{n} + a_{n-1}x^{n-1} + \cdots + a_{1}x + a_{0} = 0
              \]

              with rational coefficients $a_{i}$. Assuming that each polynomial equation has only finitely many roots, show that the set of algebraic numbers is countable.
        \item A real number is said to be transcendental if it is not algebraic. Assuming the reals are uncountable, show that the transcendental numbers are uncountable.
    \end{enumerate}
\end{exercise}

\begin{proof}
    \begin{enumerate}[label={(\alph*)}]
        \item Let $A_{n}$ be the set of real-number roots of all rational-coefficient polynomial of degree $n$ where $n$ is a positive integer. Because the set of rational-coefficient polynomial of degree $n$ is countable, then so is $A_{n}$ (countable union of finite sets).

              $A_{n}$ is countable for $n\in\mathbb{Z}_{+}$ so the following set, which is the set of algebraic numbers
              \[
                  \bigcup_{n\in\mathbb{Z}_{+}}A_{n}
              \]

              is countable, because a countable union of countable sets is countable.
        \item $\mathbb{R}$ is the union of the set of algebraic numbers and the set of transcendental numbers. Assume the set of transcendental numbers is countable, then $\mathbb{R}$ is countable, which is a contradiction. Hence the transcendental numbers are uncountable.
    \end{enumerate}
\end{proof}

% chapter1:section7:exercise5
\begin{exercise}\label{chapter1:section7:exercise5}
    Determine, for each of the following sets, whether or not it is countable. Justify your answers.
    \begin{enumerate}[label={(\alph*)}]
        \item The set $A$ of all functions $f: \{ 0, 1 \}\to \mathbb{Z}_{+}$.
        \item The set $B_{n}$ of all functions $f: \{ 1, \ldots, n \}\to \mathbb{Z}_{+}$.
        \item The set $C = \bigcup_{n\in\mathbb{Z}_{+}}B_{n}$.
        \item The set $D$ of all functions $f: \mathbb{Z}_{+}\to \mathbb{Z}_{+}$.
        \item The set $E$ of all functions $f: \mathbb{Z}_{+}\to \{ 0, 1 \}$.
        \item The set $F$ of all functions $f: \mathbb{Z}_{+}\to \{ 0, 1 \}$ that are ``eventually zero{.}''
        \item The set $G$ of all functions $f: \mathbb{Z}_{+}\to\mathbb{Z}_{+}$ that are eventually $1$.
        \item The set $H$ of all functions $f: \mathbb{Z}_{+}\to\mathbb{Z}_{+}$ that are eventually constant.
        \item The set $I$ of all two-element subsets of $\mathbb{Z}_{+}$.
        \item The set $J$ of all finite subsets of $\mathbb{Z}_{+}$.
    \end{enumerate}
\end{exercise}

\begin{proof}
    \begin{enumerate}[label={(\alph*)}]
        \item $A$ is countable.

              Let $g: A\to \mathbb{Z}_{+}\times\mathbb{Z}_{+}$ defined by $g(f) = (f(0), f(1))$.

              Suppose $g(f_{1}) = g(f_{2})$ then $f_{1}(0) = f_{2}(0)$ and $f_{1}(1) = f_{2}(1)$, which means $f_{1} = f_{2}$. Hence $g$ is injective.

              Let $(a, b)\in \mathbb{Z}_{+}\times\mathbb{Z}_{+}$ and define $f: \{ 0, 1 \}\to \mathbb{Z}_{+}$ by $f(0) = a$, $f(1) = b$. Hence $g(f) = (f(0), f(1)) = (a, b)$, so $g$ is surjective.

              Thus $g$ is bijective. Since $\mathbb{Z}_{+}\times\mathbb{Z}_{+}$ is countable, then $A$ is countable.
        \item $B_{n}$ is countable.

              Let $g: B_{n}\to {(\mathbb{Z}_{+})}^{n}$ defined by $g(f) = \mathbf{x}$ where $\mathbf{x}_{i} = f(i)$ for $i\in S_{n+1} = \{ 1, \ldots, n \}$.

              Suppose $g(f_{1}) = g(f_{2})$ then $f_{1}(i) = f_{2}(i)$ for $i\in S_{n+1}$, which means $f_{1} = f_{2}$. Hence $g$ is injective.

              Let $\mathbf{x}\in {(\mathbb{Z}_{+})}^{n}$ and define $f\in B_{n}$ as follows: $f(i) = \mathbf{x}_{i}$ for $i\in S_{n+1}$. Hence $g(f) = \mathbf{x}$, which means $g$ is surjective.

              Thus $g$ is bijective. Since ${(\mathbb{Z}_{+})}^{n}$ is countable, then $B_{n}$ is countable.
        \item $C$ is countable, because $C$ is a countable union of countable sets.
        \item $D$ is uncountable.

              Let $g: \mathbb{Z}_{+}\to D$ be a function. Let $g(n) = f_{n}$ for every positive integer $n$.
              \begin{align*}
                  g(1) & = f_{1} = (f_{1}(1), f_{1}(2), f_{1}(3), \ldots) \\
                  g(2) & = f_{2} = (f_{2}(1), f_{2}(2), f_{2}(3), \ldots) \\
                  g(3) & = f_{3} = (f_{3}(1), f_{3}(2), f_{3}(3), \ldots) \\
                       & \cdots
              \end{align*}

              Let $f: \mathbb{Z}_{+}\to \mathbb{Z}_{+}$ such that
              \[
                  f(i) = f_{i}(i) + 1\quad\forall i\in\mathbb{Z}_{+}
              \]

              then $f\notin\{ f_{1}, f_{2}, \ldots \} = g(\mathbb{Z}_{+})$ because $f\ne f_{i}$ for $i\in\mathbb{Z}_{+}$, which implies $g$ is not surjective. Hence $D$ is uncountable.
        \item $E$ is uncountable.

              Let $g: \mathbb{Z}_{+}\to E$ be a function. Let $g(n) = f_{n}$ for every positive integer $n$.

              Let's define $f: \mathbb{Z}_{+}\to \{ 0, 1 \}$ as follows
              \[
                  f(i) = \begin{cases}
                      1 & \text{if $f_{i}(i) = 0$} \\
                      0 & \text{if $f_{i}(i) = 1$}
                  \end{cases}\quad\forall i\in\mathbb{Z}_{+}
              \]

              then $f\notin \{ f_{1}, f_{2}, \ldots \} = g(\mathbb{Z}_{+})$ because $f\ne f_{i}$ for $i\in\mathbb{Z}_{+}$, which means $g$ is not surjective. Hence $E$ is uncountable.
        \item $F$ is countable.

              For every $n\in\mathbb{Z}_{+}$, let $F_{n}$ be the set of all functions $f: \mathbb{Z}_{+}\to \{ 0, 1 \}$ such that $f(i) = 0$ for $i\geq n$, then $F_{n}$ is countable because the cardinality of $F_{n}$ is $2^{n-1}$ (we can establich a bijection from $F_{n}$ to the set of functions $h: S_{n}\to\{0, 1\}$)

              By the definition of $F$, it follows that
              \[
                  F = \bigcup_{n\in\mathbb{Z}_{+}} F_{n}
              \]

              so $F$ is countable.
        \item $G$ is countable.

              For every $n\in\mathbb{Z}_{+}$, let $G_{n}$ be the set of all functions $f: \mathbb{Z}_{+}\to \mathbb{Z}_{+}$ such that $f(i) = 1$ for every $i\geq n$. $G_{1}$ has cardinality $1$, $G_{n}$ has cardinality ${(\mathbb{Z}_{+})}^{n}$ if $n > 1$, which is countable.

              By the definition of $G$, we obtain
              \[
                  G = \bigcup_{n\in\mathbb{Z}_{+}}G_{n}
              \]

              so $G$ is countable.
        \item $H$ is countable.

              Let $H_{n}$ be the set of all functions $f: \mathbb{Z}_{+}\to\mathbb{Z}_{+}$ that are eventually $n$. Similar to part (g), $H_{n}$ is countable.

              By the definition of $H$, we have
              \[
                  H = \bigcup_{n\in\mathbb{Z}_{+}} H_{n}
              \]

              so $H$ is countable.
        \item $I$ is countable.

              Let $f: I\to \mathbb{Z}_{+}\times\mathbb{Z}_{+}$ be the function defined by
              \[
                  f(A) = (\min A, \max A).
              \]

              Suppose $f(A) = f(B)$ then $\min A = \min B$ and $\max A = \max B$. Because $A, B$ are two-element sets, we conclude that $A = B$. Hence $f$ is injective. Moreover, $\mathbb{Z}_{+}\times\mathbb{Z}_{+}$ is countable, so $I$ is countable.
        \item $J$ is countable.

              Let $J_{n}$ be the set of subsets of $\mathbb{Z}_{+}$ with cardinality $(n - 1)$.

              $J_{1}$ has cardinality $1$, so it is countable. If $n > 1$, let's define $f: J_{n}\to {(\mathbb{Z}_{+})}^{n-1}$ by $f(A) = \mathbf{x}$ where
              \begin{align*}
                  \mathbf{x}_{1} & = \min A                                                                                \\
                  \mathbf{x}_{i} & = \min (A - \{ \mathbf{x}_{1}, \ldots, \mathbf{x}_{i-1} \}) & \text{if $1 < i\leq n-1$}
              \end{align*}

              From this definition, we deduce that $f$ is injective. Because ${(\mathbb{Z}_{+})}^{n-1}$ is countable, there is an injection $g: {(\mathbb{Z}_{+})}^{n-1}\to \mathbb{Z}_{+}$. Therefore, $g\circ f: J_{n}\to \mathbb{Z}_{+}$ is an injection, so $J_{n}$ is countable for $n > 1$.

              Moreover
              \[
                  J = \bigcup_{n\in\mathbb{Z}_{+}}J_{n}
              \]

              so $J$ is countable.
    \end{enumerate}
\end{proof}

% chapter1:section7:exercise6
\begin{exercise}\label{chapter1:section7:exercise6}
    We say that two sets $A$ and $B$ have the same cardinality if there is a bijection of $A$ with $B$.
    \begin{enumerate}[label={(\alph*)}]
        \item Show that if $B\subset A$ and if there is an injection $f: A\to B$ then $A$ and $B$ have the same cardinality.

              Hint: Define $A_{1} = A$, $B_{1} = B$, and for $n > 1$, $A_{n} = f(A_{n-1})$ and $B_{n} = f(B_{n-1})$. Note that $A_{1}\supset B_{1}\supset A_{2}\supset B_{2}\supset A_{3}\supset\cdots$. Define a bijection $h: A\to B$ by the rule \[h(x) = \begin{cases}f(x) & \text{if $x\in A_{n} - B_{n}$ for some $n$} \\ x & \text{otherwise}\end{cases}\]
        \item If there are injections $f: A\to C$ and $g: C\to A$, then $A$ and $C$ have the same cardinality.
    \end{enumerate}
\end{exercise}

\begin{proof}
    \begin{enumerate}[label={(\alph*)}]
        \item Let $A_{1} = A$, $B_{1} = B$, and for $n > 1$, $A_{n} = f(A_{n-1})$ and $B_{n} = f(B_{n-1})$.

              $A_{1}\supset B_{1}$. Assume $A_{n}\supset B_{n}$ for some positive integer $n$, then $f(A_{n})\supset f(B_{n})$, so $A_{n+1}\supset B_{n+1}$. Hence $A_{n}\supset B_{n}$ for $n\in\mathbb{Z}_{+}$.

              Because $f: A\to B$ so $B\supset f(A)$, which means $B_{1}\supset A_{2}$. Assume $B_{n}\supset A_{n+1}$ for some integer $n$, then $f(B_{n})\supset f(A_{n+1})$, therefore $B_{n+1}\supset A_{n+2}$. Hence $B_{n}\supset A_{n+1}$ for $n\in\mathbb{Z}_{+}$.

              Define $h: A\to B$ by the rule
              \[
                  h(x) = \begin{cases}f(x) & \text{if $x\in A_{n} - B_{n}$ for some $n$} \\ x & \text{otherwise}\end{cases}
              \]

              Suppose $h(x) = h(y)$. Consider the following cases
              \begin{itemize}
                  \item $h(x) = f(x), h(y) = f(y)$.

                        Then $f(x) = f(y)$, which implies $x = y$ because $f$ is injective.
                  \item $h(x) = f(x), h(y) = y$.

                        This is impossible because $x\in A_{n} - B_{n}$ for some $n$ and $y\notin A_{i} - B_{i}$ for every $i\in\mathbb{Z}_{+}$.
                  \item $h(x) = x, h(y) = f(y)$.

                        This is impossible because $x\notin A_{i} - B_{i}$ for every $i\in\mathbb{Z}_{+}$ and $y\in A_{n} - B_{n}$ for some $n$.
                  \item $h(x) = x, h(y) = y$.

                        Then $x = y$.
              \end{itemize}

              So $h$ is injective.

              Let $y\in B$ and
              \[
                  X = A - \bigcup_{n\in\mathbb{Z}_{+}}(A_{n} - B_{n}).
              \]

              If $y\in X$ then $h(y) = y$.

              If $y\notin X$ then there is a positive integer $n$ such that $y\in A_{n} - B_{n}$. According to Exercise~\ref{chapter1:section2:exercise2}
              \[
                  f(A_{i} - B_{i}) = f(A_{i}) - f(B_{i}) = A_{i+1} - B_{i+1}.
              \]

              Because $y\in A_{n} - B_{n}$ and $y\in B = B_{1}$ then $n > 1$. So there exists $x\in A_{n-1} - B_{n-1}$ such that $h(x) = f(x) = y$.

              So $h$ is surjective.

              Thus $h$ is bijective, hence $A$ and $B$ have the same cardinality.
        \item Let $B = f(A)$ then the function $h: A\to B$ defined by $h(x) = f(x)$ is bijective because $f$ is injective. Because $g: C\to A$ is injective then $h\circ g: C\to B$ is injective. By part (a), $B$ and $C$ have the same cardinality. $A$ and $B$ have the same cardinality. Hence $A$ and $C$ have the same cardinality.
    \end{enumerate}
\end{proof}

% chapter1:section7:exercise7
\begin{exercise}\label{chapter1:section7:exercise7}
    Show that the sets $D$ and $E$ of Exercise~\ref{chapter1:section7:exercise5} have the same cardinality.
\end{exercise}

\begin{proof}
    Let's define the function $g: D\to E$ by $g(\mathbf{x}) = \mathbf{y}$ where $\mathbf{y}$ is the sequence such that the first $\mathbf{x}_{1}$ terms are $0$, the next is $1$, the next $\mathbf{x}_{2}$ terms are $0$, the next is $1$, \ldots then $g$ is an injective.

    Let's define the function $h: E\to D$ by $g(\mathbf{y}) = \mathbf{x}$ where $\mathbf{x}_{i} = \mathbf{y}_{i} + 1$ for $i\in\mathbb{Z}_{+}$, then $h$ is injective.

    By Exercise~\ref{chapter1:section7:exercise6}, we conclude that $D$ and $E$ have the same cardinality.
\end{proof}

% chapter1:section7:exercise8
\begin{exercise}\label{chapter1:section7:exercise8}
    Let $X$ denote the two-element set $\{ 0, 1 \}$; let $\mathscr{B}$ be the set of countable subsets of $X^{\omega}$. Show that $X^{\omega}$ and $\mathscr{B}$ have the same cardinality.
\end{exercise}

\begin{proof}
    Unsolved.
\end{proof}

% chapter1:section7:exercise9
\begin{exercise}\label{chapter1:section7:exercise9}
    \begin{enumerate}[label={(\alph*)}]
        \item The formula
              \begin{align*}
                  h(1) & = 1                                                      \\
                  h(2) & = 2                                                      \\
                  h(n) & = {(h(n+1))}^{2} - {(h(n-1))}^{2} & \text{for $n\geq 2$}
              \end{align*}

              is not one to which the principle of recursion definition applies. Show that nevertheless there does exist a function $h: \mathbb{Z}_{+}\to\mathbb{R}$ satisfying this formula.
        \item Show that the formula of part (a) does not determine $h$ uniquely.
        \item Show that there is no function $h: \mathbb{Z}_{+}\to \mathbb{R}$ satisfying the formula
              \begin{align*}
                  h(1) & = 1                                                       \\
                  h(2) & = 2                                                       \\
                  h(n) & = {(h(n+1))}^{2} + {(h(n-1))}^{2} & \text{for $n\geq 2$.}
              \end{align*}
    \end{enumerate}
\end{exercise}

\begin{proof}
    \begin{enumerate}[label={(\alph*)}]
        \item Define $g: \mathbb{Z}_{+}\to \mathbb{R}$ by the formula
              \begin{align*}
                  g(1) & = 1                                                   \\
                  g(2) & = 2                                                   \\
                  g(n) & = \sqrt{g(n-1) + {(g(n-2))}^{2}} & \text{for $n > 2$}
              \end{align*}

              then $g$ satisfies the given formula.
        \item Define $f: \mathbb{Z}_{+}\to \mathbb{R}$ by the formula
              \begin{align*}
                  f(1) & = 1                                                    \\
                  f(2) & = 2                                                    \\
                  f(n) & = -\sqrt{g(n-1) + {(g(n-2))}^{2}} & \text{for $n > 2$}
              \end{align*}

              then $f$ satisfies the given formula and $f\ne g$.
        \item Assume there is such a function $h$.

              \begin{align*}
                  {(h(3))}^{2} & = h(2) - {(h(1))}^{2} = 1            \\
                  {(h(4))}^{2} & = h(3) - {(h(2))}^{2} = h(3) - 4 < 0
              \end{align*}

              which is a contradiction. Hence there is no such function $h$.
    \end{enumerate}
\end{proof}

\section{The Principle of Recursive Definition}

% chapter1:section8:exercise1
\begin{exercise}\label{chapter1:section8:exercise1}
    Let $(b_{1}, b_{2}, \ldots)$ be an infinite sequence of real numbers. The sum $\sum^{n}_{k=1}b_{k}$ is defined by induction as follows:
    \begin{align*}
        \sum^{1}_{k=1}b_{k} & = b_{1}                                                            \\
        \sum^{n}_{k=1}b_{k} & = \left(\sum^{n-1}_{k=1}b_{k}\right) + b_{n} & \text{for $n > 1$.}
    \end{align*}

    Let $A$ be the set of real numbers; choose $\rho$ so that Theorem 8.4 applies to define this sum rigorously. We sometimes denote the sum $\sum^{n}_{k=1}b_{k}$ by the symbol $b_{1} + b_{2} + \cdots + b_{n}$.
\end{exercise}

\begin{proof}
    Let $a_{0} = b_{1}$ and $\rho(f) = f(n) + b_{n+1}$ where $f: S_{n+1}\to \mathbb{R}$. By Theorem 8.4
    \begin{align*}
        \sum^{1}_{k=1}b_{k} & := h(1) = b_{1}                               \\
        \sum^{n}_{k=1}b_{k} & := h(n) = h(n-1) + b_{n} & \text{for $n > 1$}
    \end{align*}
\end{proof}

% chapter1:section8:exercise2
\begin{exercise}\label{chapter1:section8:exercise2}
    Let $(b_{1}, b_{2}, \ldots)$ be an infinite sequence of real numbers. We define the product $\prod^{n}_{k=1}b_{k}$ by the equations
    \begin{align*}
        \prod^{1}_{k=1}b_{k} & = b_{1}                                                                 \\
        \prod^{n}_{k=1}b_{k} & = \left(\prod^{n-1}_{k=1}b_{k}\right) \cdot b_{n} & \text{for $n > 1$.}
    \end{align*}

    Use Theorem 8.4 to define this product rigorously. We somtimes denote the product $\prod^{n}_{k=1}b_{k}$ by the symbol $b_{1}b_{2}\cdots b_{n}$.
\end{exercise}

\begin{proof}
    Let $a_{0} = b_{1}$ and $\rho(f) = f(n)\cdot b_{n+1}$ where $f: S_{n+1}\to \mathbb{R}$. By Theorem 8.4
    \begin{align*}
        \prod^{1}_{k=1}b_{k} & := h(1) = b_{1}                                                                        \\
        \prod^{n}_{k=1}b_{k} & := h(n) = \rho(h\vert_{\{ 1, \ldots, n-1 \}}) = h(n-1)\cdot b_{n} & \text{for $n > 1$}
    \end{align*}
\end{proof}

% chapter1:section8:exercise3
\begin{exercise}\label{chapter1:section8:exercise3}
    Obtain the definitions of $a^{n}$ and $n{!}$ for $n\in\mathbb{Z}_{+}$ as special cases of Exercise~\ref{chapter1:section8:exercise2}.
\end{exercise}

\begin{proof}
    Apply Exercise~\ref{chapter1:section8:exercise2}, we obtain

    $(a, a, \ldots)$ be an infinite sequence of $a$. $a^{n} := \prod^{n}_{k=1}a$.

    $(1, 2, \ldots)$ is the sequence of positive integers. $n! := \prod^{n}_{k=1}k$.
\end{proof}

% chapter1:section8:exercise4
\begin{exercise}\label{chapter1:section8:exercise4}
    The Fibonacci numbers of number theory are defined recursively by the formula
    \begin{align*}
        \lambda_{1} & = \lambda_{2} = 1,                                    \\
        \lambda_{n} & = \lambda_{n-1} + \lambda_{n-2} & \text{for $n > 2$}.
    \end{align*}

    Define the rigorously by the use of Theorem 8.4.
\end{exercise}

\begin{proof}
    $\rho(f) = \begin{cases}f(1) & \text{if $n = 1$} \\ f(n - 1) + f(n) & \text{if $n > 1$} \end{cases}$ where $f: \{ 1,\ldots, n \}\to \mathbb{R}$.
    \begin{align*}
        h(1) & := 1                                                                         \\
        h(2) & = \rho(h\vert_{\{1\}}) = 1                                                   \\
        h(n) & := \rho(h\vert_{\{ 1,\ldots, n-1 \}}) = h(n-1) + h(n-2) & \text{for $n > 2$}
    \end{align*}

    and $\lambda_{n} = h(n)$.
\end{proof}

% chapter1:section8:exercise5
\begin{exercise}\label{chapter1:section8:exercise5}
    Show that there is a unique function $h: \mathbb{Z}_{+}\to\mathbb{R}_{+}$ satisfying the formula
    \begin{align*}
        h(1) & = 3                                           \\
        h(i) & = {(h(i - 1) + 1)}^{1/2} & \text{for $i > 1$}
    \end{align*}

    Explain why this example does not violate the principle of recursive definition.
\end{exercise}

\begin{proof}
    $\rho(f) = \sqrt{f(n) + 1}$ where $f: \{ 1, \ldots, n \}\to \mathbb{R}_{+}$. $\rho$ is well-defined.
    \begin{align*}
        h(1) & := 3                                                     \\
        h(i) & := \rho(h\vert_{\{1,\ldots,i-1\}}) = \sqrt{h(i - 1) + 1}
    \end{align*}

    So by Theorem 8.4, there is a unique $h: \mathbb{Z}_{+}\to \mathbb{R}_{+}$ that satisfies the given recursive formula.
\end{proof}

% chapter1:section8:exercise6
\begin{exercise}\label{chapter1:section8:exercise6}
    \begin{enumerate}[label={(\alph*)}]
        \item Show that there is no function $h: \mathbb{Z}_{+}\to\mathbb{R}_{+}$ satisfying the formula
              \begin{align*}
                  h(1) & = 3                                           \\
                  h(i) & = {(h(i - 1) - 1)}^{1/2} & \text{for $i > 1$}
              \end{align*}

              Explain why this example does not violate the principle of recursive definition.
        \item Consider the recursion formula
              \begin{align*}
                  h(1) & = 3                                                 \\
                  h(i) & = \begin{cases}
                               {(h(i-1) - 1)}^{1/2} & \text{if $h(i - 1) > 1$}   \\
                               5                    & \text{if $h(i - 1)\leq 1$}
                           \end{cases} & \text{for $i > 1$.}
              \end{align*}

              Show that there exists a unique function $h: \mathbb{Z}_{+}\to \mathbb{R}_{+}$ satisfying this formula.
    \end{enumerate}
\end{exercise}

\begin{proof}
    \begin{enumerate}[label={(\alph*)}]
        \item Assume there is such a function $h$. $h(1) = 3$, $h(2) = \sqrt{2}$, $h(3) = \sqrt{\sqrt{2} - 1}$, $h(4)$ is undefined. Hence there is no such function $h$.

              There is no well-defined function $\rho$ so the principle of recursive definition is not violated.
        \item Let
              \[
                  \rho(f) = \begin{cases}
                      \sqrt{f(n - 1) - 1} & \text{if $f(n - 1) > 1$}   \\
                      5                   & \text{if $f(n - 1)\leq 1$}
                  \end{cases}
              \]

              for $f: \{ 1,\ldots, n \}\to \mathbb{R}_{+}$. Then
              \begin{align*}
                  h(1) & := 3                                                                                    \\
                  h(n) & := \rho(h\vert_{\{ 1,\ldots,n-1 \}}) = \begin{cases}
                                                                    \sqrt{h(n - 1) - 1} & \text{if $f(n - 1) > 1$}   \\
                                                                    5                   & \text{if $f(n - 1)\leq 1$}
                                                                \end{cases} & \text{for $n > 1$}
              \end{align*}

              By Theorem 8.4, there exists a unique function $h: \mathbb{Z}_{+}\to\mathbb{R}_{+}$ satisfying this formula.
    \end{enumerate}
\end{proof}

% chapter1:section8:exercise7
\begin{exercise}\label{chapter1:section8:exercise7}
    Prove Theorem 8.4.

    Let $A$ be a set; let $a_{0}$ be an element of $A$. Suppose $\rho$ is a function that assigns, to each function $f$ mapping a nonempty section of the positive integers into $A$, an element of $A$. Then there exists a unique function
    \[
        h: \mathbb{Z}_{+}\to A
    \]

    such that
    \begin{align*}
        h(1) & = a_{0}                                                        \\
        h(i) & = \rho(h\vert_{\{1,\ldots,i-1\}}) & \text{for $i > 1$} \tag{*}
    \end{align*}
\end{exercise}

\begin{proof}
    I define $h_{1}: \{ 1 \}\to A$ by $h_{1}(1) = a_{0}$.

    Assume there exists a function $h_{n-1}: \{ 1, \ldots, n-1 \}\to A$ satisfying (*) for all $i$ in its domain. Let's define $h_{n}: \{ 1, \ldots, n \}\to A$ as follows
    \[
        h_{n}(i) = \begin{cases}
            h_{n-1}(i)    & \text{if $i < n$} \\
            \rho(h_{n-1}) & \text{if $i = n$}
        \end{cases}
    \]

    so $h_{n}$ satisfies (*). Hence for every $n\in\mathbb{Z}_{+}$, $h_{n}$ satisfies (*) for all $i$ in its domain.

    Suppose $g_{m}$ and $h_{n}$ satisfy (*) on their domains and $m\leq n$. $g_{m}(1) = h_{n}(1) = a_{0}$. Assume $g_{m}\vert_{\{ 1,\ldots, k \}} = h_{n}\vert_{\{1,\ldots, k\}}$ for $k$ such that $1\leq k < m$, then $h_{n}\vert_{\{1,\ldots,k+1\}} = g_{m}\vert_{\{1,\ldots,k\}}$. By the principle of mathematical induction, $g_{m}\vert_{\{ 1,\ldots,m \}} = h_{n}\vert_{\{1,\ldots,m\}}$.

    Let's define $h: \mathbb{Z}_{+}\to A$ as follows: $U$ is the union of the rules of all $h_{1}, h_{2}, \ldots$. Let $i$ be a positive integer, then all pairs in $U$ whose first coordinate is $i$ are of the form $(i, h_{n}(i))$ for $n\geq i$. By the previous paragraph, $h_{n}(i) = h_{m}(i)$ for $n, m\geq i$. So for every $i$, there is only one element of $U$ whose first coordinate is $i$. So $h$ is well-defined.

    $h$ satisfies (*) because $h(i) = f_{n}(i)$ for $i\leq n$ and $f_{n}$ satisfies (*) for $i$ in its domain. The uniqueness of $h$ is proved similarly to an above paragraph.
\end{proof}

% chapter1:section8:exercise8
\begin{exercise}\label{chapter1:section8:exercise8}
    Verify the following version of the principle of recursive definition: Let $A$ be a set. Let $\rho$ be a function assigning, to every function $f$ mapping a section $S_{n}$ of $\mathbb{Z}_{+}$ into $A$, an element $\rho(f)$ of $A$. Then there is a unique function $h: \mathbb{Z}_{+}\to A$ such that $h(n) = \rho(h\vert S_{n})$ for each $n\in\mathbb{Z}_{+}$.
\end{exercise}

\begin{proof}
    This new definition is equivalent to the former. Moreover, this definition cover the case $f$ is the empty function, which is vacuously true.
\end{proof}

\section{Infinite Sets and the Axiom of Choice}

% chapter1:section9:exercise1
\begin{exercise}\label{chapter1:section9:exercise1}
    Define an injective map $f: \mathbb{Z}_{+}\to X^{\omega}$, where $X$ is the two-element set $\{ 0, 1 \}$ without using the choice axiom.
\end{exercise}

\begin{proof}
    I define $f: \mathbb{Z}_{+}\to X^{\omega}$ as follows
    \begin{align*}
        f(n)           & = \mathbf{x}            \\
        \mathbf{x}_{i} & = \begin{cases}
                               1 & \text{if $i = n$} \\
                               0 & \text{otherwise}
                           \end{cases}
    \end{align*}

    By this definition, $f(n) = f(m)$ if and only if $n = m$. Hence $f$ is injective.
\end{proof}

% chapter1:section9:exercise2
\begin{exercise}\label{chapter1:section9:exercise2}
    Find if possible a choice function for each of the following collections, without using the choice axiom
    \begin{enumerate}[label={(\alph*)}]
        \item The collection $\mathscr{A}$ of nonempty subsets of $\mathbb{Z}_{+}$.
        \item The collection $\mathscr{B}$ of nonempty subsets of $\mathbb{Z}$.
        \item The collection $\mathscr{C}$ of nonempty subsets of the rational numbers $\mathbb{Q}$.
        \item The collection $\mathscr{D}$ of nonempty subsets of $X^{\omega}$, where $X = \{ 0, 1 \}$.
    \end{enumerate}
\end{exercise}

\begin{proof}
    \begin{enumerate}[label={(\alph*)}]
        \item From each $A\in\mathscr{A}$, let $x_{A}$ be the smallest element of $A$. The choice function $A\mapsto \min A$ does not depend on the choice axiom.
        \item From each $B\in\mathscr{B}$, if $B$ contains a positive integer, let $x_{B}$ be the smallest positive integer in $B$; if $B$ doesn't contain any positive integer, let $x_{B}$ be the largest integer in $B$. The choice function $B\mapsto x_{B}$ does not depend on the choice axiom.
        \item I can't.
        \item I can't.
    \end{enumerate}
\end{proof}

% chapter1:section9:exercise3
\begin{exercise}\label{chapter1:section9:exercise3}
    Suppose that $A$ is a set and ${\{ f_{n} \}}_{n\in\mathbb{Z}_{+}}$ is a given indexed family of injective functions
    \[
        f_{n}: \{ 1, \ldots, n \} \to A.
    \]

    Show that $A$ is infinite. Can you define an injective function $f: \mathbb{Z}_{+}\to A$ without using the choice axiom?
\end{exercise}

\begin{proof}
    Assume $A$ is finite. $A$ is nonempty because if $A$ is empty, then there is no nonempty function whose target set is $A$. Because $A$ is finite, there exists a unique positive integer $m$ such that there is a bijection $f: A\to \{ 1,\ldots, m \}$. By hypothesis, $f\circ f_{m+1}$ is an injection from $\{ 1, \ldots, m+1 \}$ to $\{ 1,\ldots, m \}$, which is a contradiction because there is no injection from $\{ 1, \ldots, m+1 \}$ to its proper subset $\{ 1,\ldots,m \}$. Hence $A$ is infinite.

    Let's define a function $f: \mathbb{Z}_{+}\to A$ using Theorem 8.4.
    \begin{align*}
        f(1) & = a_{0} = f_{1}(1)                                                                                                                      \\
        f(n) & = \rho(h: S_{n+1}\to A) = f_{n+1}(\min\{ k\in \mathbb{Z}_{+} \land k\leq n+1 \land f_{n+1}(k)\notin h(S_{n+1}) \}) & \text{for $n > 1$}
    \end{align*}

    Hence $f$ is injective because for every $n\in\mathbb{Z}_{+}$, $f(n)\notin f(S_{n})$.
\end{proof}

% chapter1:section9:exercise4
\begin{exercise}\label{chapter1:section9:exercise4}
    There was a theorem in Section 7 whose proof involved an infinite number of arbitrary choices. Which one was it? Rewrite the proof so as to make explicit the use of the choice axiom.
\end{exercise}

\begin{proof}
    It was Theorem 7.5. A countable union of countable sets is countable.
\end{proof}

% chapter1:section9:exercise5
\begin{exercise}\label{chapter1:section9:exercise5}
    \begin{enumerate}[label={(\alph*)}]
        \item Use the choice axiom to show that if $f: A \to B$ is surjective, then $f$ has a right inverse $h: B\to A$.
        \item Show that if $f: A\to B$ is injective and $A$ is not empty, then $f$ has a left inverse. Is the axiom of choice needed?
    \end{enumerate}
\end{exercise}

\begin{proof}
    \begin{enumerate}[label={(\alph*)}]
        \item Because $f: A\to B$ is surjective, then for each singleton subset $\{ b \}$, the preimage $f^{-1}(\{b\})$ is nonempty. From each set in the collection ${\{ f^{-1}(\{b\}) \}}_{b\in B}$, we can choose an element $a$. Let $h$ be such a choice function, then $f\circ h = i_{B}$. So $f$ has a right inverse.
        \item Let $a_{0}$ be an element of $A$. Let's define $g: B\to A$ as follows
              \begin{align*}
                  g(y) & = \begin{cases}
                               x     & \text{if $y\in f(A)$ and $f(x) = y$} \\
                               a_{0} & \text{if $y\notin f(A)$}
                           \end{cases}
              \end{align*}

              this function is well-defined because if $y\in f(A)$ there is a unique $x\in A$ such that $f(x) = y$ since $f$ is injective. Hence $g\circ f = i_{A}$, so $f$ has a left inverse. The function $g$ was defined without the axiom of choice?
    \end{enumerate}
\end{proof}

% chapter1:section9:exercise6
\begin{exercise}\label{chapter1:section9:exercise6}
    Most of the famous paradoxes of naive set theory are associated in some way or other with the concept of the ``set of all sets{.}'' None of the rules we have given for forming sets allows us to consider such a set. And for good reason --- the concept itself is self-contradictory. For suppose that $\mathscr{A}$ denotes the ``set of all sets{.}''
    \begin{enumerate}[label={(\alph*)}]
        \item Show that $\mathscr{P}(\mathscr{A})\subset \mathscr{A}$ derive a contradiction.
        \item (Russell's paradox) Let $\mathscr{B}$ be the subset of $\mathscr{A}$ consisting of all sets that are not elements of themselves
              \[
                  \mathscr{B} = \{ A \mid A\in\mathscr{A} \land A\notin A \}.
              \]

              Is $\mathscr{B}$ and element of itself or not?
    \end{enumerate}
\end{exercise}

\begin{proof}
    \begin{enumerate}[label={(\alph*)}]
        \item $\mathscr{A}$ is a proper subset of its power set $\mathscr{P}(\mathscr{A})$. Therefore $\mathscr{P}(\mathscr{A})\subset \mathscr{A}$ leads to a contradiction.
        \item From the definition of $\mathscr{B}$, we conclude that $\mathscr{B}\in\mathscr{B}\Longleftrightarrow \mathscr{B}\notin\mathscr{B}$.
    \end{enumerate}
\end{proof}

% chapter1:section9:exercise7
\begin{exercise}\label{chapter1:section9:exercise7}
    Let $A$ and $B$ be two nonempty sets. If there is an injection of $B$ into $A$, but no injection of $A$ into $B$, we say that $A$ has \textbf{greater cardinality} than $B$.
    \begin{enumerate}[label={(\alph*)}]
        \item Conclude from Theorem 9.1 that every uncountable set has greater cardinality than $\mathbb{Z}_{+}$.
        \item Show that if $A$ has greater cardinality than $B$, and $B$ has greater cardinality than $C$, then $A$ has greater cardinality than $C$.
        \item Find a sequence $A_{1}, A_{2}, \ldots$ of infinite sets, such that for each $n\in\mathbb{Z}_{+}$, the set $A_{n+1}$ has greater cardinality than $A_{n}$.
        \item Find a set that for every $n$ has cardinality greater than $A_{n}$.
    \end{enumerate}
\end{exercise}

\begin{proof}
    \begin{enumerate}[label={(\alph*)}]
        \item Let $A$ be an uncountable set. By Theorem 9.1, there is an injective function $f: \mathbb{Z}_{+}\to A$. Assume there is an injective function of $A$ into $\mathbb{Z}_{+}$ then $A$ and $\mathbb{Z}_{+}$ have the same cardinality (Cantor-Schroeder-Berstein's theorem), which implies $A$ is countable, which is a contradiction. Thus there is no injection of $A$ into $\mathbb{Z}_{+}$, so $A$ has greater cardinality than $\mathbb{Z}_{+}$.
        \item Let $f$ be an injection of $B$ into $A$ and $g$ be an injection of $C$ into $B$, then $f\circ g$ is an injection of $C$ into $A$.

              Assume there is an injection $h$ of $A$ into $C$, then $h\circ f$ is an injection of $B$ into $C$. By Cantor-Schroeder-Bernstein's theorem, $B$ and $C$ have the same cardinality, which is a contradiction. So the assumption is false and we conclude that there is no injection of $A$ into $C$. Thus $A$ has greater cardinality than $C$.
        \item Let's define $A_{n}$ recursively as follows
              \begin{align*}
                  A_{1} & = \mathbb{Z}_{+}                            \\
                  A_{n} & = \mathscr{P}(A_{n-1}) & \text{for $n > 1$}
              \end{align*}

              then $A_{n+1}$ has cardinality larger than $A_{n}$ for $n\in\mathbb{Z}_{+}$.
        \item Let $A = \bigcup_{i\in\mathbb{Z}_{+}}A_{i}$. For every $n\in\mathbb{Z}_{+}$, $\bigcup_{i\in\mathbb{Z}_{+}}A_{i}$ has greater cardinality than $A_{n}$ because it contains $A_{n+1}$.
    \end{enumerate}
\end{proof}

% chapter1:section9:exercise8
\begin{exercise}\label{chapter1:section9:exercise8}
    Show that $\mathscr{P}(\mathbb{Z}_{+})$ and $\mathbb{R}$ have the same cardinality.
\end{exercise}

\begin{proof}
    Define $f: \mathscr{P}(\mathbb{Z}_{+})\to \mathbb{R}$ as follows
    \begin{align*}
        f(A) & = \begin{cases}
                     0                            & \text{if $A = \varnothing$} \\
                     \sum_{n\in A}\frac{1}{2^{n}} & \text{otherwise}
                 \end{cases}
    \end{align*}

    then $f$ is injective (here I am implicitly using binary expansion)

    Define $g: \mathbb{R}\to \mathscr{P}(\mathbb{Q})$ as follows: $g(x) = \{ q \mid q\in\mathbb{Q} \land q < x \}$ (Dedekind cut) then $g$ is injective. $\mathbb{Q}$ and $\mathbb{Z}_{+}$ have the same cardinality so $\mathscr{P}(\mathbb{Q})$ and $\mathscr{P}(\mathbb{Z}_{+})$ have the same cardinality. Hence there is an injection of $\mathbb{R}$ into $\mathscr{P}(\mathbb{Z}_{+})$.

    By Cantor-Schroeder-Bernstein's theorem, $\mathscr{P}(\mathbb{Z}_{+})$ and $\mathbb{R}$ have the same cardinality.
\end{proof}

\section{Well-Ordered Sets}

% chapter1:section10:exercise1
\begin{exercise}\label{chapter1:section10:exercise1}
    Show that every well-ordered set has the least upper bound property.
\end{exercise}

\begin{proof}
    Let $A$ be a well-ordered set and $B$ a bounded above nonempty subset of $A$. Let $C$ be the set of upper bounds of $B$. Because $B$ is bounded above, $C$ is nonempty. $C$ is a nonempty subset of $A$ and $A$ is well-ordered so $C$ has a smallest element, which is the least upper bound of $B$. Hence $A$ has the least upper bound property.

    Thus every well-ordered set has the least upper bound property.
\end{proof}

% chapter1:section10:exercise2
\begin{exercise}\label{chapter1:section10:exercise2}
    \begin{enumerate}[label={(\alph*)}]
        \item Show that in a well-ordered set, every element except the largest (if one exists) has an immediate successor.
        \item Find a set in which every element has an immediate successor that is not well-ordered.
    \end{enumerate}
\end{exercise}

\begin{proof}
    \begin{enumerate}[label={(\alph*)}]
        \item Let $A$ be a well-ordered set and $x$ be an element of $A$ which is not the largest. Let's define
              \[
                  B_{x} = \{ y \mid y\in A \land x < y \}.
              \]

              Since $x$ is not the largest element of $A$, then $B_{x}$ is a nonempty subset of $A$. Because $A$ is well-ordered, $B_{x}$ has a smallest element $x_{0}$. $x < x_{0}$ and there is no element $y\in A$ such that $x < y < x_{0}$ because $x_{0}$ is the smallest element which is also strictly larger than $x$. Hence $x_{0}$ is an immediate successor of $x$. Thus every element in a well-ordered set, except the largest, has an immediate successor.
        \item $\mathbb{Z}$ is not well-ordered with the usual order, but every integer has an immediate successor.
    \end{enumerate}
\end{proof}

% chapter1:section10:exercise3
\begin{exercise}\label{chapter1:section10:exercise3}
    Both $\{ 1,2 \}\times\mathbb{Z}_{+}$ and $\mathbb{Z}_{+}\times\{1,2\}$ are well-ordered in the dictionary order. Do they have the same order type?
\end{exercise}

\begin{proof}
    No, they don't have the same order type.

    Assume that the two sets have the same order type, then there exists a bijection $f: \{ 1,2 \}\times\mathbb{Z}_{+}\to \mathbb{Z}_{+}\times\{1,2\}$ that preserves order. In $\mathbb{Z}_{+}\times\{1,2\}$, every element, except the smallest element (which is $1\times 1$) has an immediate predecessor. Therefore $f(n\times m)$ where $n\times m\ne 1\times 1$ has an immediate predecessor. However, in $\{ 1,2 \}\times\mathbb{Z}_{+}$, the elements $1\times 1$ and $2\times 1$ don't have an immediate predecessor. So the assumption is false.

    Thus $\{ 1,2 \}\times\mathbb{Z}_{+}$ and $\mathbb{Z}_{+}\times\{1,2\}$ don't have the same order type, even though they are well-ordered in the dictionary order.
\end{proof}

% chapter1:section10:exercise4
\begin{exercise}\label{chapter1:section10:exercise4}
    \begin{enumerate}[label={(\alph*)}]
        \item Let $\mathbb{Z}_{-}$ denote the set of negative integers in the usual order. Show that a simply ordered set $A$ fails to be well-ordered if and only if it contains a subset having the same order type as $\mathbb{Z}_{-}$.
        \item Show that if $A$ is simply ordered and every countable subset of $A$ is well-ordered, then $A$ is well-ordered.
    \end{enumerate}
\end{exercise}

\begin{proof}
    \begin{enumerate}[label={(\alph*)}]
        \item Suppose $A$ contains a subset having the same order type as $\mathbb{Z}_{-}$. Because $\mathbb{Z}_{-}$ does not have a smallest element, then the presumed subset of $A$ does not have a smallest element. Therefore $A$ fails to be well-ordered.

              Suppose $A$ fails to be well-ordered, then there exists a nonempty subset $B$ of $A$ which does not have a smallest element. Moreover, we deduce that $B$ is infinite (because if $B$ is finite, then $B$ has a smallest element).

              Let $a_{1}$ be an element of $B$. Because $B$ is infinite and $B$ has no smallest element, $B$ contains an element $a_{2}$ strictly less than $a_{1}$. Inductively, $B$ contains an element $a_{n+1}$ stricly less than $a_{n}$. Let $C = {\{ a_{n} \}}_{n\in\mathbb{Z}_{+}}$. We define the function $f: C \to \mathbb{Z}_{-}$ by $f(a_{n}) = -n$. $f$ is a bijection, so $C$ and $\mathbb{Z}_{-}$ have the same order type.
        \item Assume that $A$ fails to be well-ordered, then there exists a subset $B$ of $A$ having the same order type as $\mathbb{Z}_{-}$. So $B$ does not have a smallest element. However, $B$ is countable, so $B$ is well-ordered, hence $B$ has a smallest element. Hence the assumption is false, so $A$ is well-ordered.
    \end{enumerate}
\end{proof}

% chapter1:section10:exercise5
\begin{exercise}\label{chapter1:section10:exercise5}
    Show the well-ordering theorem implies the choice axiom.
\end{exercise}

\begin{proof}
    Let ${\{ A_{j} \}}_{j\in J}$ be a collection of disjoint nonempty sets. Every set $A_{j}$ can be well-ordered by some order $<_{j}$, and let $a_{j}$ be the smallest element of $A_{j}$ in that order relation. Let $C = {\{ a_{j} \}}_{j\in J}$ then $C\cap A_{j}$ is a singleton set for every $j\in J$. Thus the choice axiom follows.
\end{proof}

% chapter1:section10:exercise6
\begin{exercise}\label{chapter1:section10:exercise6}
    Let $S_{\Omega}$ be the minimal uncountable well-ordered set.
    \begin{enumerate}[label={(\alph*)}]
        \item Show that $S_{\Omega}$ has no largest element.
        \item Show that for every $\alpha\in S_{\Omega}$, the subset $\{ x \mid \alpha < x \}$ is uncountable.
        \item Let $X_{0}$ be the subset of $S_{\Omega}$ consisting of all elements $x$ such that $x$ has no immediate predecessor. Show that $X_{0}$ is uncountable.
    \end{enumerate}
\end{exercise}

\begin{proof}
    \begin{enumerate}[label={(\alph*)}]
        \item Assume that $S_{\Omega}$ has a largest element $x$, then $S_{\Omega} - \{ x \} = \{ y \mid y\in S_{\Omega} \land y < x \}$ is a section of $S_{\Omega}$. By the definition of the minimal uncountable well-ordered set, $S_{\Omega} - \{ x \}$ is countable. Therefore $S_{\Omega} = \{ x \} \cup (S_{\Omega} - \{ x \})$ is countable, which is a contradiction. Hence $S_{\Omega}$ has no largest element.
        \item The complement of $\{ x \mid \alpha < x \}$ is $\{ x \mid x\leq \alpha \}$. Moreover $\{ x \mid x\leq \alpha \} = \{ \alpha \} \cup \{ x \mid x < \alpha \}$.

              Because every section of $S_{\Omega}$ is countable, so the set $\{ x \mid x\leq \alpha \}$ is an union of two countable sets, which means it is also countable. Because $S_{\Omega}$ is uncountable, the complement of the countable set $\{ x \mid x\leq \alpha \}$ is uncountable. Hence the set $\{ x \mid \alpha < x \}$ is uncountable for every $\alpha\in S_{\Omega}$.
        \item By part (b) and Theorem 10.3, we have: $A$ is a nonempty countable subset of $S_{\Omega}$ if and only if $A$ is bounded above.

              Assume $X_{0}$ is countable, then $X_{0}$ is nonempty because $X_{0}$ contains the smallest element of $S_{\Omega}$. $X_{0}$ is countable and nonempty so $X_{0}$ has a strict upper bound $b$ (Theorem 10.3). Let $A = \{ b, \operatorname{succ}(b), \operatorname{succ}^{2}(b), \ldots \}$ then $A$ is countable. By Exercise~\ref{chapter1:section10:exercise1}, $A$ has a least upper bound, let $s = \sup A$. $s\notin A$ because otherwise, $s < \operatorname{succ}(s)$.

              Assume $s$ has an immediate predecessor $r$, then $r$ is not an upper bound of $A$, so there exists $n\in\mathbb{Z}_{+}$ such that $r < \operatorname{succ}^{n}(b)\leq s$. Moreover, $\operatorname{succ}^{n}(b) < s$ because $s\notin A$. So $r < \operatorname{succ}^{n}(b) < s$, which contradicts the definition of immediate predecessor. So $s$ has no immediate predecessor, which implies $s\in X_{0}$. This is a contradiction because $b < s$ and $b$ is a strict upper bound of $X_{0}$.

              Hence $X_{0}$ is uncountable.
    \end{enumerate}
\end{proof}

% chapter1:section10:exercise7
\begin{exercise}\label{chapter1:section10:exercise7}
    Let $J$ be a well-ordered set. A subset $J_{0}$ of $J$ is said to be \textbf{inductive} if for every $\alpha\in J$,
    \[
        (S_{\alpha}\subset J_{0})\implies \alpha \in J_{0}.
    \]

    Theorem (The principle of transfinite induction). If $J$ is a well-ordered set and $J_{0}$ is an inductive subset of $J$, then $J_{0} = J$.
\end{exercise}

\begin{proof}
    Let $A = J - J_{0}$. Assume $A$ is not empty, then $A$ has a smallest element $\alpha$. So every element less than $\alpha$ is in $J_{0}$, which implies $S_{\alpha}\subset J_{0}$. Hence $\alpha\in J_{0}$, which is a contradiction. Therefore $J_{0} = J$.
\end{proof}

% chapter1:section10:exercise8
\begin{exercise}\label{chapter1:section10:exercise8}
    \begin{enumerate}[label={(\alph*)}]
        \item Let $A_{1}$ and $A_{2}$ be disjoint sets, well-ordered by $<_{1}$ and $<_{2}$, respectively. Define an order relation on $A_{1}\cup A_{2}$ by letting $a < b$ either if $a, b\in A_{1}$ and $a <_{1} b$, or if $a, b\in A_{2}$ and $a <_{2} b$, or if $a\in A_{1}$ and $b\in A_{2}$. Show that this is a well-ordering.
        \item Generalize (a) to an arbitrary family of disjoint well-ordered sets, indexed by a well-ordered set.
    \end{enumerate}
\end{exercise}

\begin{proof}
    \begin{enumerate}[label={(\alph*)}]
        \item Let $S$ be a nonempty subset of $A_{1}\cup A_{2}$.

              If $S\cap A_{1}$ is nonempty, then the smallest element of $S\cap A_{1}$ is the smallest element of $S$. If $S\cap A_{1}$ is empty then $S$ is a subset of $A_{2}$ so $S$ has a smallest element.

              Hence $<$ on $A_{1}\cup A_{2}$ is a well-ordering.
        \item Let $J$ be a well-ordered set and ${\{ A_{j} \}}_{j\in J}$ is a collection of disjoint well-ordered sets with order relation $<_{j}$. Define an order relation $<$ on $\bigcup_{j\in J}A_{j}$ by letting $a < b$ either if $a, b\in A_{j}$ and $a < b$, or if $a\in A_{j}, b\in A_{k}$ and $j < k$.

              Let $S$ be a nonempty subset of $\bigcup_{j\in J}A_{j}$. Let $i$ be the smallest element of $J$ such that $S\cap A_{i}$ is nonempty. $S\cap A_{i}$ has a smallest element because $A_{i}$ is well-ordered. Moreover, the smallest element of $S\cap A_{i}$ is also the smallest element of $S$.

              Thus $<$ on $\bigcup_{j\in J}A_{j}$ is a well-ordering.
    \end{enumerate}
\end{proof}

% chapter1:section10:exercise9
\begin{exercise}\label{chapter1:section10:exercise9}
    Consider the subset $A$ of ${(\mathbb{Z}_{+})}^{\omega}$ consisting of all infinite sequences of positive integers $\mathbf{x} = (x_{1}, x_{2}, \ldots)$ that end in an infinite string of $1$'s. Give $A$ the following order: $\mathbf{x} < \mathbf{y}$ if $x_{n} < y_{n}$ and $x_{i} = y_{i}$ for $i > n$. We call this the ``antidictionary order'' on $A$.
    \begin{enumerate}[label={(\alph*)}]
        \item Show that for every $n$, there is a section of $A$ that has the same order type as ${(\mathbb{Z}_{+})}^{n}$ in the dictionary order.
        \item Show $A$ is well-ordered.
    \end{enumerate}
\end{exercise}

\begin{proof}
    \begin{enumerate}[label={(\alph*)}]
        \item Let $\mathbf{a}\in A$. Because $\mathbf{a}$ ends in an infinite string of $1$'s so there exists a smallest positive integer $i$ such that $\mathbf{a}_{j} = 1$ for $j\geq i$. So there are finitely many positive integers $i$ such that $\mathbf{a}_{i}\ne 1$.

              Let $n$ be a fixed positive integer and $\mathbf{a}\in A$ such that
              \[
                  \mathbf{a}_{i} = \begin{cases}
                      1 & \text{if $i\leq n$}   \\
                      2 & \text{if $i = n + 1$} \\
                      1 & \text{if $i > n + 1$}
                  \end{cases}
              \]

              $\mathbf{x} < \mathbf{a}$ if and only if $\mathbf{x}_{i} = 1$ for $i > n$. Let $A_{n} = S_{\mathbf{a}}$ and define $f: A_{n}\to {(\mathbb{Z}_{+})}^{n}$ by $f(\alpha) = \beta$ where $\beta_{i} = \alpha_{n-i+1}$. Then $f$ is a bijection preserving order.
        \item Define $f: A\to \mathbb{Z}_{+}$ by $f(\mathbf{x}) = n$ where $n$ is the smallest positive integer $i$ such that $\mathbf{x}_{j} = 1$ for $j\geq i$. $f(\mathbf{x}) < f(\mathbf{y})\implies \mathbf{x} < \mathbf{y}$.

              Let $B$ be a nonempty subset of $A$ and $J = f(B)\subset \mathbb{Z}_{+}$. If $B$ contains $(1, 1, 1, \ldots)$ then $B$ has a smallest element. Assume the opposite. Because $J$ is nonempty, it has a smallest element $n$ and $n > 1$, because $B$ doesn't contain $(1, 1, 1, \ldots)$. Let $M$ be the subset of $B$ of elements $\mathbf{x}$ such that $f(\mathbf{x}) = n$, $M$ is a subset of the section $A_{n}$, which is well-ordered because it has the same order type as ${(\mathbb{Z}_{+})}^{n}$. So $M$ has a smallest element, hence $B$ has a smallest element.

              Thus $A$ is well-ordered.
    \end{enumerate}
\end{proof}

% chapter1:section10:exercise10
\begin{exercise}\label{chapter1:section10:exercise10}
    Theorem. Let $J$ and $C$ be well-ordered sets; assume that there is no surjective function mapping a section of $J$ onto $C$. Then there exists a unique function $h: J\to C$ satisfying the equation
    \[
        h(x) = \min (C - h(S_{x}))\tag{*}
    \]

    for each $x\in J$, where $S_{x}$ is the section of $J$ by $x$.

    Proof.
    \begin{enumerate}[label={(\alph*)}]
        \item If $h$ and $k$ map sections of $J$, or all of $J$, into $C$ and satisfy (*) for all $x$ in their respective domains, show that $h(x) = k(x)$ for all $x$ in both domains.
        \item If there exists a function $h: S_{\alpha}\to C$ satisfying (*), show that there exists a function $k: S_{\alpha}\cup \{ \alpha \}\to C$ satisfying (*).
        \item If $K\subset J$ and for all $\alpha\in K$ there exists a function $h_{\alpha}: S_{\alpha}\to C$ satisfying (*), show that there exists a function
              \[
                  k: \bigcup_{\alpha\in K}S_{\alpha}\to C
              \]

              satisfying (*).
        \item Show by transfinite induction that for every $\beta\in J$, there exists a function $h_{\beta}: S_{\beta}\to C$ satisfying (*).
        \item Prove the theorem.
    \end{enumerate}
\end{exercise}

\begin{proof}
    \begin{enumerate}[label={(\alph*)}]
        \item Let $S$ be the set of elements $x$ in the both domains of $h$ and $k$ such that $h(x) = k(x)$. $S$ is inductive, so by the principle of transfinite induction, $h(x) = k(x)$ for all $x$ in both domains of $h$ and $k$.
        \item I define $k: S_{\alpha}\cup\{\alpha\}\to C$ as follows
              \begin{align*}
                  k(x)      & := h(x)                                              & \text{if $x\in S_{\alpha}$} \\
                  k(\alpha) & := \min(C - h(S_{\alpha})) = \min(C - k(S_{\alpha}))
              \end{align*}

              then $k$ satisfies (*).
        \item I define the rule of $k: \bigcup_{\alpha\in K}S_{\alpha}\to C$ by the union of the rules of $h_{\alpha}$ where $\alpha\in K$. Let $x\in \bigcup_{\alpha\in K}S_{\alpha}$ and $A$ be the set of pairs whose first coordinate is $x$, then
              \[
                  A = \{ (x, h_{\alpha}(x)) \mid \alpha > x \}
              \]

              By part (a), $h_{\alpha}(x) = h_{\beta}(x)$ for $\alpha, \beta > x$. So $A$ is an one-element set, so $k$ is a well-defined function. $k$ satisfies (*).
        \item Let $J_{0}$ be the subset of $J$ containing elements satisfying (*).

              If $S_{\alpha}\subset J_{0}$ then $\alpha\in J_{0}$. By the principle of transfinite induction,
        \item Part (d) shows that for every section of $J$ there is a function satisfying (*).

              Part (c) shows that there is a function satisfying (*) on the union of all sections of $J$.

              The union of all sections of $J$ contains all elements of $J$ except the largest element $m$ (if it exists). If $J$ has a largest element, we extend the domain of the function to $S_{m}\cup \{ m \}$ by part (b).

              Hence there exists a function $h: J\to C$ satisfying (*). The uniqueness of $h$ follows from (a).
    \end{enumerate}
\end{proof}

% chapter1:section10:exercise11
\begin{exercise}\label{chapter1:section10:exercise11}
    Let $A$ and $B$ be two sets. Using the well-ordering theorem, prove that either they have the same cardinality, or one has cardinality greater than other.

        [Hint: If there is no surjection $f: A\to B$, apply the preceding exercise.]
\end{exercise}

\begin{proof}
    If there is a surjection of $A$ onto $B$, then its right inverse is an injection of $B$ into $A$, so the $A$ and $B$ either have the same cardinality of $A$ has cardinality greater than $B$ (by Cantor-Schroeder-Berstein's theorem).

    If there is no surjection of $A$ onto $B$ then by Exercise~\ref{chapter1:section10:exercise10} then we make $A, B$ well-ordered and conclude that there is an injection of $A$ into $B$. So the $A$ and $B$ either have the same cardinality of $B$ has cardinality greater than $A$ (by Cantor-Schroeder-Berstein's theorem).
\end{proof}

\section{The Maximum Principle}

% chapter1:section11:exercise1
\begin{exercise}\label{chapter1:section11:exercise1}
    If $a$ and $b$ are real numbers, define $a \prec b$ if $b - a$ is positive and rational. Show this is a strict partial order on $\mathbb{R}$. What are the maximal simply ordered subsets?
\end{exercise}

\begin{proof}
    Suppose $S$ is a maximal simply ordered subset of $\mathbb{R}$ and $a\in S$. Then for every $x\in S$ and $x\ne a$, either $x - a$ is a positive rational number or $a - x$ is a positive rational number. Hence $S = \{ a + q \mid q\in\mathbb{Q} \}$.

    Thus the maximal simply ordered subsets of $\mathbb{R}$ are of the form $\{ a + q \mid q\in\mathbb{Q} \}$ where $a\in\mathbb{R}$.
\end{proof}

% chapter1:section11:exercise2
\begin{exercise}\label{chapter1:section11:exercise2}
    \begin{enumerate}[label={(\alph*)}]
        \item Let $\prec$ be a strict partial order on the set $A$. Define a relation on $A$ by letting $a\preceq b$ if either $a\prec b$ or $a = b$. Show that this relation has the following properties, which are called the \textbf{partial order axioms}
              \begin{enumerate}[label={(\roman*)}]
                  \item $a\preceq b$ for all $a\in A$.
                  \item $a\preceq b$ and $b\preceq a$ implies $a = b$.
                  \item $a\preceq b$ and $b\preceq c$ implies $a\preceq c$.
              \end{enumerate}
        \item Let $P$ be a relation on $A$ that satisfies properties {(i)}-{(iii)}. Define a relation $S$ on $A$ by letting $aSb$ if $aPb$ and $a\ne b$. Show that $S$ is a strict partial order on $A$.
    \end{enumerate}
\end{exercise}

\begin{proof}
    \begin{enumerate}[label={(\alph*)}]
        \item For all $a\in A$, $a = a$ so $a\preceq a$.

              Suppose $a\preceq b$ and $b\preceq a$. Assume $a\ne b$, then $a\prec b$ and $b\prec a$, which is a contradiction because the two statements cannot be simutaneously true. Hence $a = b$.

              Suppose $a\preceq b$ and $b\preceq c$, then either
              \begin{itemize}
                  \item $a = b$ and $b = c$, so $a = c$, which implies $a\preceq c$.
                  \item $a\prec b$ and $b = c$, so $a\prec c$, which implies $a\preceq c$.
                  \item $a = b$ and $b\prec c$, so $a\prec c$, which implies $a\preceq c$.
                  \item $a\prec b$ and $b\prec c$, so $a\prec c$, which implies $a\preceq c$.
              \end{itemize}

              So $\preceq$ satisfies the partial order axioms.
        \item For all $a\in A$, $aSa$ is false.

              Suppose $aSb$ and $bSc$, then $aPb$ and $bPc$, which implies $aPc$. Assume $a = c$, then $aPb$ and $bPa$, which implies $a = b$ and this contradicts $aSb$. Therefore $a\ne c$, so $aSc$. Thus $aSb$ and $bSc$ implies $aSc$.

              So $S$ is a strict partial order on $A$.
    \end{enumerate}
\end{proof}

% chapter1:section11:exercise3
\begin{exercise}\label{chapter1:section11:exercise3}
    Let $A$ be a set with a strict partial order $\prec$; let $x\in A$. Suppose that we wish to find a maximal simply ordered subset $B$ of $A$ that contains $x$. One plausible way of attempting to define $B$ is to let $B$ equal the set of all those elements of $A$ that are comparable with $x$
    \[
        B = \{ y \mid y\in A \text{ and either $x\prec y$ or $y\prec x$} \}.
    \]

    But this will not always work. In which of Examples 1 and 2 will this procedure succeed and in which will it not?
\end{exercise}

\begin{proof}
    This procedure works in Example 2 but doesn't work in Example 1.
\end{proof}

% chapter1:section11:exercise4
\begin{exercise}\label{chapter1:section11:exercise4}
    Given two points $(x_{0}, y_{0})$ and $(x_{1}, y_{1})$ of $\mathbb{R}^{2}$, define
    \[
        (x_{0}, y_{0}) \prec (x_{1}, y_{1})
    \]

    if $x_{0} < x_{1}$ and $y_{0}\leq y_{1}$. Show that the curves $y = x^{3}$ and $y = 2$ are maximal simply ordered subsets of $\mathbb{R}^{2}$, and the curve $y = x^{2}$ is not. Find all maximal simply ordered subsets.
\end{exercise}

\begin{proof}
    Suppose $(x_{0}, y_{0})$ and $(x_{1}, y_{1})$ are distinct points on $y = x^{3}$. Then either $x_{0} < x_{1}$ or $x_{1} < x_{0}$. Because $y_{0} = x_{0}^{3}$ and $y_{1} = x_{1}^{3}$ so $y_{0} < y_{1}$ if $x_{0} < x_{1}$, $y_{1} < y_{0}$ if $x_{1} < x_{0}$. So the curve $y = x^{3}$ is a simply ordered subset of $\mathbb{R}^{2}$. Add $(x, y)$ to this subset (where $y\ne x^{3}$) then $(x, y)$ and $(x, x^{3})$ are not comparable. Hence the curve $y = x^{3}$ is a maximal simply ordered subset of $\mathbb{R}^{2}$.

    Suppose $(x_{0}, y_{0})$ and $(x_{1}, y_{1})$ are distinct points on $y = 2$. Then either $x_{0} < x_{1}$ or $x_{1} < x_{0}$. $y_{0} = y_{1} = 2$ so $y_{0}\leq y_{1}$ and $y_{1}\leq x_{1}$, so either $(x_{0}, y_{0})\prec (x_{1}, y_{1})$ or $(x_{1}, y_{1})\prec (x_{0}, y_{0})$. So the curve $y = 2$ is a simply ordered subset of $\mathbb{R}^{2}$. Add $(x, y)$ to this subset (where $y\ne 2$) then $(x, y)$ and $(x, 2)$ are not comparable. Hence the curve $y = 2$ is a maximal simply ordered subset of $\mathbb{R}^{2}$.

    The two points $(1, 1)$ and $(-2, 4)$ are not comparable and lie on the curve $y = x^{2}$. Hence the curve $y = x^{2}$ is not a maximal simply ordered subset of $\mathbb{R}^{2}$.

    Let $A$ be a maximal simply ordered subset of $\mathbb{R}^{2}$, then $A$ is the rule of a nondecreasing real-valued function $f$ whose domain is $A_{1}$, which is the set of first coordinates of the elements of $A$.

    Assume there exists $a\in A - A_{1}$ such that $\{ x \mid x\in A_{1} \land x < a \}$ and $\{ x \mid x\in A_{1} \land x > a \}$ are nonempty. We have
    \begin{align*}
        \sup\limits_{x\in A_{1} \land x < a} f \leq \inf\limits_{x\in A_{1}\land x > a}f
    \end{align*}

    Let $b$ be a real number in the closed interval whose endpoints are $\sup\limits_{x\in A_{1} \land x < a} f$ and $\inf\limits_{x\in A_{1}\land x > a}f$, then $A\cup \{ (a, b) \}$ is simply ordered, which contradicts the maximality if $A$. Therefore $A_{1}$ must be an interval.

    If $A_{1}$ is a closed interval $\closedinterval{x_{0}, x_{1}}$, then $A\cup {\{ (x, f(x_{0})) \}}_{x < x_{0}}$ is simply ordered.

    If $A_{1}$ is a half-left open interval $\halfopenleft{x_{0}, x_{1}}$, then $A\cup {\{ (x, f(x_{1})) \}}_{x > x_{1}}$ is simply ordered.

    If $A_{1}$ is a half-right open interval $\halfopenright{x_{0}, x_{1}}$, then $A\cup {\{ (x, f(x_{0})) \}}_{x < x_{0}}$ is simply ordered.

    If $A_{1}$ is of the form $\halfopenleft{-\infty, x_{1}}$, then $A\cup {\{ (x, f(x_{1})) \}}_{x > x_{1}}$ is simply ordered.

    If $A_{1}$ is of the form $\halfopenright{x_{0}, \infty}$, then $A\cup {\{ (x, f(x_{0})) \}}_{x < x_{0}}$ is simply ordered.

    Hence $A_{1}$ must be an open interval $\openinterval{x_{0}, x_{1}}$ (where $x_{0} < x_{1}$ and $x_{0}, x_{1}\in\overline{\mathbb{R}}$).

    If $x_{0}\ne -\infty$ and $f(A_{1})$ is bounded below by $\ell$, then $A\cup {\{ (x, \ell) \}}_{x \leq x_{0}}$ is simply ordered.

    If $x_{1}\ne \infty$ and $f(A_{1})$ is bounded above by $s$, then $A\cup {\{ (x, s) \}}_{x\geq x_{1}}$ is simply ordered.

    Thus $A$ is a maximal simply ordered subset of $\mathbb{R}^{2}$ if and only if $f: \openinterval{a,b}\to \mathbb{R}$, $a < b$, $f$ is nondecreasing, and
    \begin{itemize}[topsep=0pt,itemsep=0pt]
        \item $a = -\infty$ or $f$ is not bounded below.
        \item $b = \infty$ or $f$ is not bounded above.
    \end{itemize}
\end{proof}

% chapter1:section11:exercise5
\begin{exercise}\label{chapter1:section11:exercise5}
    Show that Zorn's lemma implies the following:

    Lemma (Kuratowski). Let $\mathscr{A}$ be a collection of sets. Suppose that for every subcollection $\mathscr{B}$ of $\mathscr{A}$ that is simply ordered by proper inclusion, the union of the elements of $\mathscr{B}$ belongs to $\mathscr{A}$. Then $\mathscr{A}$ has an element that is properly contained in no other element of $\mathscr{A}$.
\end{exercise}

\begin{proof}
    $\mathscr{A}$ is partially ordered by proper inclusion. For every  subcollection $\mathscr{B}$ of $\mathscr{A}$ that is simply ordered by proper inclusion, the union of the elements of $\mathscr{B}$ belongs to $\mathscr{A}$. The union of the elements of $\mathscr{B}$ is an upper bound of $\mathscr{B}$. Apply Zorn's lemma to $\mathscr{A}$, we conclude that $\mathscr{A}$ has an element that is not properly contained in any other element of $\mathscr{A}$.
\end{proof}

% chapter1:section11:exercise6
\begin{exercise}\label{chapter1:section11:exercise6}
    A collection $\mathscr{A}$ of subsets of a set $X$ is said to be of finite type provided that a subset $B$ of $X$ belongs to $\mathscr{A}$ if and only if every finite subset of $B$ belongs to $\mathscr{A}$. Show that the Kuratowski lemma implies the following.

    Lemma (Tukey, 1940). Let $\mathscr{A}$ be a collection of sets. If $\mathscr{A}$ is of finite type, then $\mathscr{A}$ has an element that is properly contained in no other element of $\mathscr{A}$.
\end{exercise}

\begin{proof}
    Let $\mathscr{B}$ be a nonempty subcollection of $\mathscr{A}$ which is simply ordered by proper inclusion. Let $B$ be the union of the elements of $\mathscr{B}$ and $C$ is a finite subset of $B$. Because $\mathscr{B}$ is simply ordered by proper inclusion, $C$ belongs to $\mathscr{B}$, so $C$ also belongs to $\mathscr{A}$. Hence by Kuratowski lemma, $\mathscr{A}$ has an element that is properly contained in no other element of $\mathscr{A}$.
\end{proof}

% chapter1:section11:exercise7
\begin{exercise}\label{chapter1:section11:exercise7}
    Show that they Tukey lemma implies the Hausdorff maximum principle. [Hint: If $\prec$ is a strict partial order on $A$, let $\mathscr{A}$ be the collection of all subsets of $A$ that are simply ordered by $\prec$. Show that $\mathscr{A}$ is of finite type.]
\end{exercise}

\begin{proof}
    Let $A$ be a partially ordered set by the strict partial order relation $\prec$ and $\mathscr{A}$ the collection of all subsets of $A$ that are simply ordered by $\prec$. Let $B$ be a subset of $A$ in $\mathscr{A}$ and $C$ is an arbitrary finite subset of $B$. $B$ is simply ordered (because $B$ is in $\mathscr{A}$) and $C$ is simply ordered (because $C$ is a subset of $B$), so $C$ belongs to $\mathscr{A}$. Therefore $\mathscr{A}$ is of finite type. By Tukey lemma, we conclude that $\mathscr{A}$ has a maximal ordered set. Thus Tukey lemma implies the Hausdorff maximum principle.
\end{proof}

% chapter1:section11:exercise8
\begin{exercise}\label{chapter1:section11:exercise8}
    A typical use of Zorn's lemma in algebra is the proof that every vector space has a basis. Let $V$ be a vector space. Prove that
    \begin{enumerate}[label={(\alph*)}]
        \item If $A$ is independent and $v\in V$ does not belong to the span of $A$, show $A\cup \{ v \}$ is independent.
        \item Show the collection of all independent sets in $V$ has a maximal element.
        \item Show that $V$ has a basis.
    \end{enumerate}
\end{exercise}

\begin{proof}
    \begin{enumerate}[label={(\alph*)}]
        \item Assume $A\cup \{ v \}$ is not independent, then $v$ belongs to the span of $A$, which is a contradiction to the hypothesis. Therefore $A\cup\{v \}$ is independent.
        \item The collection of all independent sets in $V$ is partially ordered by proper inclusion. Let $\mathscr{A}$ be a simply ordered subcollection of this collection, then $\bigcup_{A\in\mathscr{A}}A$ is an independent set. By Zorn lemma, the collection of all independent sets in $V$ has a maximal element.
        \item Let $B$ be a maximal element of the collection of all independent sets in $V$, then $B$ is independent. For every $v\in V$, $B\cup\{ v \}$ is not independent, so $v$ belongs to the span of $B$. Hence $B$ is a basis for $V$.
    \end{enumerate}
\end{proof}

\begin{section*}{Supplement Exercises: Well-Ordering}
    % chapter1:sectionX:exercise1
    \begin{exercise}\label{chapter1:sectionX:exercise1}
    \end{exercise}

    \begin{proof}
    \end{proof}

    % chapter1:sectionX:exercise2
    \begin{exercise}\label{chapter1:sectionX:exercise2}
    \end{exercise}

    \begin{proof}
    \end{proof}

    % chapter1:sectionX:exercise3
    \begin{exercise}\label{chapter1:sectionX:exercise3}
    \end{exercise}

    \begin{proof}
    \end{proof}

    % chapter1:sectionX:exercise4
    \begin{exercise}\label{chapter1:sectionX:exercise4}
    \end{exercise}

    \begin{proof}
    \end{proof}

    % chapter1:sectionX:exercise5
    \begin{exercise}\label{chapter1:sectionX:exercise5}
    \end{exercise}

    \begin{proof}
    \end{proof}

    % chapter1:sectionX:exercise6
    \begin{exercise}\label{chapter1:sectionX:exercise6}
    \end{exercise}

    \begin{proof}
    \end{proof}

    % chapter1:sectionX:exercise7
    \begin{exercise}\label{chapter1:sectionX:exercise7}
    \end{exercise}

    \begin{proof}
    \end{proof}

    % chapter1:sectionX:exercise8
    \begin{exercise}\label{chapter1:sectionX:exercise8}
    \end{exercise}

    \begin{proof}
    \end{proof}

\end{section*}
