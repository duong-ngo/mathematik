\chapter{The Tangent Space}

\section{The Tangent Space}

\begin{exercise}{8.3}[The differential of a map]
    Check that \( F_{\ast}(X_{p}) \) is a derivation at \( F(p) \) and that \( F_{\ast}: T_{p}N \to T_{F(p)}M \) is a linear map.
\end{exercise}

\begin{proof}
    From the definition of \( F_{\ast} \), for every \( f, g \in C^{\infty}_{F(p)}M \)
    \[
        (F_{\ast}(X_{p}))(fg) = X_{p}((fg) \circ F) = X_{p}((f \circ F) \cdot (g \circ F)).
    \]

    Since \( X_{p} \in T_{p}M \) is a derivation
    \[
        X_{p}((f \circ F) \cdot (g \circ F)) = X_{p}(f \circ F) (g\circ F)(p) + (f\circ F)(p) X_{p}(g\circ F).
    \]

    From the definition of \( F_{\ast} \)
    \[
        X_{p}(f \circ F) (g\circ F)(p) + (f\circ F)(p) X_{p}(g\circ F) = (F_{\ast}(X_{p}))(f) g(F(p)) + f(F(p)) (F_{\ast}(X_{p}))(g).
    \]

    Hence \( F_{\ast}(X_{p}) \) is a derivation at \( F(p) \).

    Now we show that \( F_{\ast} \) is a linear map.

    For every pair of tangent vectors \( X_{p}, Y_{p} \in T_{p}M \) and for every \( f \in C^{\infty}_{F(p)}(M) \)
    \begingroup
    \allowdisplaybreaks%
    \begin{align*}
        (F_{\ast}(X_{p} + Y_{p}))(f) & = (X_{p} + Y_{p})(f \circ F)                  \\
                                     & = X_{p}(f\circ F) + Y_{p}(f\circ F)           \\
                                     & = (F_{\ast}(X_{p}))(f) + (F_{\ast}(Y_{p}))(f)
    \end{align*}
    \endgroup

    so \( F_{\ast}(X_{p} + Y_{p}) = F_{\ast}(X_{p}) + F_{\ast}(Y_{p}) \).

    For every tangent vector \( X_{p} \in T_{p}M \), every real number \( \lambda \) and for every \( f \in C^{\infty}_{F(p)}(M) \)
    \begingroup
    \allowdisplaybreaks%
    \begin{align*}
        (F_{\ast}(\lambda X_{p}))(f) & = (\lambda X_{p})(f \circ F)         \\
                                     & = \lambda \cdot X_{p}(f \circ F)     \\
                                     & = \lambda \cdot (F_{\ast}(X_{p}))(f)
    \end{align*}
    \endgroup

    so \( F_{\ast}(\lambda X_{p}) = \lambda F_{\ast}(X_{p}) \). Hence \( F_{\ast} \) is a linear map.
\end{proof}

\begin{exercise}{8.14}[Velocity vector versus the calculus derivative]
    Let \( c: \openinterval{a, b} \to \mathbb{R} \) be a curve with target space \( \mathbb{R} \). Verify that \( c'(t) = \dot{c}(t) d/dx\vert_{c(t)} \).
\end{exercise}

\begin{proof}
    \( c'(t) \) is a vector at \( c(t) \in \mathbb{R} \). Since \( d/dx\vert_{c(t)} \) is a basis for the tangent space at \( c(t) \in \mathbb{R} \), there exists \( a\in \mathbb{R} \) such that
    \[
        c'(t) = a\left.\dfrac{d}{dx}\right\vert_{c(t)}.
    \]

    From the definition of \( c'(t) \)
    \[
        a = c'(t)x = c_{\ast}\left(\left.\dfrac{d}{dt}\right\vert_{c(t)}\right)x = \left.\dfrac{d}{dt}\right\vert_{c(t)} x\circ c = \left.\dfrac{d}{dt}\right\vert_{c(t)} c = \dot{c}(t).
    \]

    Therefore \( c'(t) = \dot{c}(t) \left.\dfrac{d}{dt}\right\vert_{c(t)} \).
\end{proof}

\section{Submanifolds}

\section{Categories and Functors}

\section{The Rank of a Smooth Map}

\section{The Tangent Bundle}

\section{Bump Functions and Partitions of Unity}

\section{Vector Fields}
