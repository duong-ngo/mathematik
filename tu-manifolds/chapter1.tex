\chapter{Euclidean Spaces}

\section{Smooth Functions on a Euclidean Space}

\begin{problem}{1.1}[A function that is \(C^{2}\) but not \(C^{3}\)]
Let \(g\colon\mathbb{R}\to\mathbb{R}\) be the function in Example 1.2(iii). Show that the function \(h(x)=\int_{0}^{x}g(t)\,dt\) is \(C^{2}\) but not \(C^{3}\) at \(x=0\).
\end{problem}

\begin{proof}
	\( g: \mathbb{R} \to \mathbb{R} \) is given by \( g(x) = \frac{3}{4}x^{4/3} \) so
	\[
		h(x) = \int^{x}_{0} g(t) dt = \int^{x}_{0} \frac{3}{4}t^{4/3} dt = \frac{3}{4}\cdot \left[\frac{3}{7} t^{7/3}\right]\Bigg{\vert}^{t=x}_{t=0} = \frac{9}{28} x^{7/3}.
	\]

	Because \( g \) is \( C^{1} \) but not \( C^{2} \) at \( x = 0 \), it follows that \( h \) is \( C^{2} \) but not \( C^{3} \) at \( x = 0 \).
\end{proof}

\begin{problem}{1.2}[A \(C^{\infty}\) function very flat at \(0\)]
Let \(f(x)\) be the function on \(\mathbb{R}\) defined in Example 1.3.
\begin{enumerate}[label={(\alph*)}]
	\item Show by induction that for \(x>0\) and \(k\geq 0\), the \(k\)th derivative \(f^{(k)}(x)\) is of the form \(p_{2k}(1/x)e^{-1/x}\) for some polynomial \(p_{2k}(y)\) of degree \(2k\) in \(y\).
	\item Prove that \(f\) is \(C^{\infty}\) on \(\mathbb{R}\) and that \(f^{(k)}(0)=0\) for all \(k\geq 0\).
\end{enumerate}
\end{problem}

\begin{proof}
	\begin{enumerate}[label={(\alph*)}]
		\item For \( x > 0 \), \( f(x) = p_{0}(1/x)\cdot \exp(-1/x) \) in which \( p_{0}(t) = 1 \).

		      Assume that for \( x > 0 \) and some nonnegative integer \( k \), the \(k\)th derivative \( f^{(k)}(x) \) is of the form \( p_{2k}(1/x) \exp(-1/x) \) for some polynomial \( p_{2k} \) of degree \( 2k \). Since \( p_{2k}(1/x) \) and \( \exp(-1/x) \) are smooth for \( x > 0 \) then \( p_{2k}(1/x)\exp(-1/x) \) is smooth and
		      \[
			      \frac{d}{dx} [p_{2k}(1/x)\exp(-1/x)] = \frac{-q_{2k-1}(1/x)}{x^{2}}\exp(-1/x) + \frac{p_{2k}(1/x)}{x^{2}}\exp(-1/x)
		      \]

		      in which \( q_{2k-1} \) is the derivative of \( p_{2k} \). Define \( p_{2k+2}(t) = -t^{2}q_{2k-1}(t) + t^{2}p_{2k}(t) \) then \( p_{2k+2} \) is a polynomial of degree \( 2k + 2 \) and the \( (k+1) \)th derivative of \( f \) is \( p_{2k+2}(1/x)\exp(-1/x) \) for \( x > 0 \).

		      Thus for \( x > 0 \) and \( k \ge 0 \), the \( k \)th derivative of \( f \) is of the form \( p_{2k}(1/x)\exp(-1/x) \) for some polynomial \( p_{2k} \) of degree \( 2k \).
		\item  Moreover, for \( x < 0 \) and \( k \ge 0 \), the \( k \)th derivative of \( f \) is 0 and
		      \[
			      \lim\limits_{x\to 0^{+}} f^{(k)}(x) = \lim\limits_{x\to 0^{+}} \frac{p_{2k}(1/x)}{\exp(1/x)} = 0 = \lim\limits_{x\to 0^{-}} f^{(k)}(x)
		      \]

		      so \( f \) is \( C^{k} \) and \( f^{(k)}(0) = 0 \) for each nonnegative integer \( k \). Therefore \( f \) is \( C^{\infty} \)  at \( x = 0 \), since \( f \) is \( C^{\infty} \) for \( x > 0 \) and \( x < 0 \), we conclude that \( f \) is \( C^{\infty} \) on \( \mathbb{R} \).
	\end{enumerate}
\end{proof}

\begin{problem}{1.3}[A diffeomorphism of an open interval with \(\mathbb{R}\)]
Let \(U\subset\mathbb{R}^{n}\) and \(V\subset\mathbb{R}^{n}\) be open subsets. A \(C^{\infty}\) map \(F\colon U\to V\) is called a \textit{diffeomorphism} if it is bijective and has a \(C^{\infty}\) inverse \(F^{-1}\colon V\to U\).
\begin{enumerate}[label={(\alph*)}]
	\item Show that the function \(f\colon\openinterval{-\pi/2,\pi/2}\to\mathbb{R}, f(x)=\tan x\), is a diffeomorphism.
	\item Let \(a,b\) be real numbers with \(a<b\). Find a linear function \(h\colon\openinterval{a,b} \to \openinterval{-1,1}\), thus proving that any two finite open intervals are diffeomorphic.
	\item The exponential function \(\exp\colon\mathbb{R}\to\openinterval{0,\infty}\) is a diffeomorphism. Use it to show that for any real numbers \(a\) and \(b\), the intervals \(\mathbb{R}, \openinterval{a,\infty}\), and \(\openinterval{-\infty,b}\) are diffeomorphic.
\end{enumerate}

The composite \(f\circ h\colon\halfopenright{a,b}\to\mathbb{R}\) is then a diffeomorphism of an open interval with \(\mathbb{R}\).
\end{problem}

\begin{proof}
	\begin{enumerate}[label={(\alph*)}]
		\item \( f \) is bijective (look up the properties of the tangent function).

		      \( f \) is continuous and \( f(x) = p_{1}(\tan x) \) where \( p_{1}(t) = t \). For every \( x_{0} \in \openinterval{-\pi/2,\pi/2} \)
		      \begin{align*}
			      \lim\limits_{x\to x_{0}}\frac{f(x) - f(x_{0})}{x - x_{0}} = \lim\limits_{x\to x_{0}}\frac{\tan(x) - \tan(x_{0})}{x - x_{0}} = \lim\limits_{x\to x_{0}}\frac{\sin(x - x_{0})}{(x - x_{0})\cos(x)\cos(x_{0})} = \frac{1}{{(\cos(x_{0}))}^{2}}
		      \end{align*}

		      so \( f \) is \( C^{1} \).

		      Assume that for \( k \ge 0 \), \( f^{(k)}(x) \) is of the form \( p_{k+1}(\tan x) \) where \( p_{k+1} \) is a polynomial of degree \( k+1 \). Therefore \( p_{k+1}(\tan x) = p_{k+1}(f(x)) \) is \( C^{1} \) according to Leibniz's rule and because \( p_{k+1}, f \) are \( C^{1} \), it follows that \( f \) is \( C^{k+1} \) and
		      \[
			      f^{(k+1)}(x) = \frac{q_{k+1}(\tan x)}{{(\cos(x))}^{2}} = q_{k+1}(\tan x)(1 + {(\tan x)}^{2})
		      \]

		      where \( q_{k+1} \) is the derivative of \( p_{k+1} \). So \( p_{k+2}(t) = q_{k+1}(t)\cdot (1 + t^{2}) \) is a polynomial of degree \( k + 2 \).

		      Hence \( f \) is \( C^{k} \) for every nonnegative integer \( k \), which means \( f \) is \( C^{\infty} \). By the inverse function theorem, \( f \) has a \( C^{\infty} \) inverse. Thus \( f \) is a diffeomorphism.
		\item Let \( h: \openinterval{-1, 1} \to \openinterval{a, b} \) and \( h(x) = \frac{2x - a - b}{b - a} \) then \( h \) is a \( C^{\infty} \) bijection and its inverse \( h^{-1}(y) = \frac{y(b - a) + a + b}{2} \) is \( C^{\infty} \). Therefore \( h \) is a diffeomorphism, so \( \openinterval{a, b} \) and \( \openinterval{-1, 1} \) are diffeomorphic. Consequently, any two finite open intervals are diffeomorphic.
		\item Let \( g_{+} \) be the map \( g_{+}: \openinterval{0, \infty} \to \openinterval{a, \infty} \) given by \( g_{+}(x) = x + a \) and \( g_{-} \) be the map \( g_{-}: \openinterval{0, \infty} \to \openinterval{-\infty, b} \) given by \( g_{+}(x) = b - x \). The maps \( g_{+} \) and \( g_{-} \) are diffeomorphisms so \( \openinterval{0, \infty}, \openinterval{a, \infty}, \openinterval{-\infty, b} \) are diffeomorphic. Since \( \mathbb{R} \) and \( \openinterval{0, \infty} \) are diffeomorphic, it follows that \( \mathbb{R}, \openinterval{a, \infty}, \openinterval{-\infty, b} \) are diffeomorphic.
	\end{enumerate}
\end{proof}

\begin{problem}{1.4}[A diffeomorphism of an open cube with \(\mathbb{R}^{n}\)]
Show that the map
\[
	f\colon{\openinterval{-\frac{\pi}{2},\frac{\pi}{2}}}^{n}\to\mathbb{R}^{n}, \quad f(x_{1},\ldots,x_{n})=(\tan x_{1},\ldots,\tan x_{n}),
\]

is a diffeomorphism.
\end{problem}

\begin{proof}
	\( f \) is a bijection and a \( C^{\infty} \) because each \( f^{i} \) is \( C^{\infty} \). Moreover, the Jacobian matrix of \( f \) at \( x \) is the diagonal matrix
	\[
		\operatorname{diag}\left(\frac{1}{{(\cos x^{1})}^{2}}, \ldots, \frac{1}{{(\cos x^{n})}^{2}}\right)
	\]

	which is invertible. According to the inverse function theorem, \( f \) has a \( C^{\infty} \) inverse. Therefore \( f \) is a diffeomorphism.
\end{proof}

\begin{problem}{1.5}
\end{problem}

\begin{problem}{1.6}
\end{problem}

\begin{problem}{1.7}
\end{problem}

\begin{problem}{1.8}
\end{problem}

\section{Tangent Vectors in \( \mathbb{R}^{n} \) as Derivations}

\section{The Exterior Algebra of Multicovectors}

\section{Differential Forms on \( \mathbb{R}^{n} \)}
