\chapter{Euclidean Spaces}

\section{Smooth Functions on a Euclidean Space}

\begin{problem}{1.1}[A function that is \(C^{2}\) but not \(C^{3}\)]
Let \(g\colon\mathbb{R}\to\mathbb{R}\) be the function in Example 1.2(iii). Show that the function \(h(x)=\int_{0}^{x}g(t)\,dt\) is \(C^{2}\) but not \(C^{3}\) at \(x=0\).
\end{problem}

\begin{proof}
	\( g: \mathbb{R} \to \mathbb{R} \) is given by \( g(x) = \frac{3}{4}x^{4/3} \) so
	\[
		h(x) = \int^{x}_{0} g(t) dt = \int^{x}_{0} \frac{3}{4}t^{4/3} dt = \frac{3}{4}\cdot \left[\frac{3}{7} t^{7/3}\right]\Bigg{\vert}^{t=x}_{t=0} = \frac{9}{28} x^{7/3}.
	\]

	Because \( g \) is \( C^{1} \) but not \( C^{2} \) at \( x = 0 \), it follows that \( h \) is \( C^{2} \) but not \( C^{3} \) at \( x = 0 \).
\end{proof}

\begin{problem}{1.2}[A \(C^{\infty}\) function very flat at \(0\)]
Let \(f(x)\) be the function on \(\mathbb{R}\) defined in Example 1.3.
\begin{enumerate}[label={(\alph*)}]
	\item Show by induction that for \(x>0\) and \(k\geq 0\), the \(k\)th derivative \(f^{(k)}(x)\) is of the form \(p_{2k}(1/x)e^{-1/x}\) for some polynomial \(p_{2k}(y)\) of degree \(2k\) in \(y\).
	\item Prove that \(f\) is \(C^{\infty}\) on \(\mathbb{R}\) and that \(f^{(k)}(0)=0\) for all \(k\geq 0\).
\end{enumerate}
\end{problem}

\begin{proof}
	\begin{enumerate}[label={(\alph*)}]
		\item For \( x > 0 \), \( f(x) = p_{0}(1/x)\cdot \exp(-1/x) \) in which \( p_{0}(t) = 1 \).

		      Assume that for \( x > 0 \) and some nonnegative integer \( k \), the \(k\)th derivative \( f^{(k)}(x) \) is of the form \( p_{2k}(1/x) \exp(-1/x) \) for some polynomial \( p_{2k} \) of degree \( 2k \). Since \( p_{2k}(1/x) \) and \( \exp(-1/x) \) are smooth for \( x > 0 \) then \( p_{2k}(1/x)\exp(-1/x) \) is smooth and
		      \[
			      \frac{d}{dx} [p_{2k}(1/x)\exp(-1/x)] = \frac{-q_{2k-1}(1/x)}{x^{2}}\exp(-1/x) + \frac{p_{2k}(1/x)}{x^{2}}\exp(-1/x)
		      \]

		      in which \( q_{2k-1} \) is the derivative of \( p_{2k} \). Define \( p_{2k+2}(t) = -t^{2}q_{2k-1}(t) + t^{2}p_{2k}(t) \) then \( p_{2k+2} \) is a polynomial of degree \( 2k + 2 \) and the \( (k+1) \)th derivative of \( f \) is \( p_{2k+2}(1/x)\exp(-1/x) \) for \( x > 0 \).

		      Thus for \( x > 0 \) and \( k \ge 0 \), the \( k \)th derivative of \( f \) is of the form \( p_{2k}(1/x)\exp(-1/x) \) for some polynomial \( p_{2k} \) of degree \( 2k \).
		\item  Moreover, for \( x < 0 \) and \( k \ge 0 \), the \( k \)th derivative of \( f \) is 0 and
		      \[
			      \lim\limits_{x\to 0^{+}} f^{(k)}(x) = \lim\limits_{x\to 0^{+}} \frac{p_{2k}(1/x)}{\exp(1/x)} = 0 = \lim\limits_{x\to 0^{-}} f^{(k)}(x)
		      \]

		      so \( f \) is \( C^{k} \) and \( f^{(k)}(0) = 0 \) for each nonnegative integer \( k \). Therefore \( f \) is \( C^{\infty} \)  at \( x = 0 \), since \( f \) is \( C^{\infty} \) for \( x > 0 \) and \( x < 0 \), we conclude that \( f \) is \( C^{\infty} \) on \( \mathbb{R} \).
	\end{enumerate}
\end{proof}

\begin{problem}{1.3}[A diffeomorphism of an open interval with \(\mathbb{R}\)]
Let \(U\subset\mathbb{R}^{n}\) and \(V\subset\mathbb{R}^{n}\) be open subsets. A \(C^{\infty}\) map \(F\colon U\to V\) is called a \textit{diffeomorphism} if it is bijective and has a \(C^{\infty}\) inverse \(F^{-1}\colon V\to U\).
\begin{enumerate}[label={(\alph*)}]
	\item Show that the function \(f\colon\openinterval{-\pi/2,\pi/2}\to\mathbb{R}, f(x)=\tan x\), is a diffeomorphism.
	\item Let \(a,b\) be real numbers with \(a<b\). Find a linear function \(h\colon\openinterval{a,b} \to \openinterval{-1,1}\), thus proving that any two finite open intervals are diffeomorphic.
	\item The exponential function \(\exp\colon\mathbb{R}\to\openinterval{0,\infty}\) is a diffeomorphism. Use it to show that for any real numbers \(a\) and \(b\), the intervals \(\mathbb{R}, \openinterval{a,\infty}\), and \(\openinterval{-\infty,b}\) are diffeomorphic.
\end{enumerate}

The composite \(f\circ h\colon\halfopenright{a,b}\to\mathbb{R}\) is then a diffeomorphism of an open interval with \(\mathbb{R}\).
\end{problem}

\begin{proof}
	\begin{enumerate}[label={(\alph*)}]
		\item \( f \) is bijective (look up the properties of the tangent function).

		      \( f \) is continuous and \( f(x) = p_{1}(\tan x) \) where \( p_{1}(t) = t \). For every \( x_{0} \in \openinterval{-\pi/2,\pi/2} \)
		      \begin{align*}
			      \lim\limits_{x\to x_{0}}\frac{f(x) - f(x_{0})}{x - x_{0}} = \lim\limits_{x\to x_{0}}\frac{\tan(x) - \tan(x_{0})}{x - x_{0}} = \lim\limits_{x\to x_{0}}\frac{\sin(x - x_{0})}{(x - x_{0})\cos(x)\cos(x_{0})} = \frac{1}{{(\cos(x_{0}))}^{2}}
		      \end{align*}

		      so \( f \) is \( C^{1} \).

		      Assume that for \( k \ge 0 \), \( f^{(k)}(x) \) is of the form \( p_{k+1}(\tan x) \) where \( p_{k+1} \) is a polynomial of degree \( k+1 \). Therefore \( p_{k+1}(\tan x) = p_{k+1}(f(x)) \) is \( C^{1} \) according to Leibniz's rule and because \( p_{k+1}, f \) are \( C^{1} \), it follows that \( f \) is \( C^{k+1} \) and
		      \[
			      f^{(k+1)}(x) = \frac{q_{k+1}(\tan x)}{{(\cos(x))}^{2}} = q_{k+1}(\tan x)(1 + {(\tan x)}^{2})
		      \]

		      where \( q_{k+1} \) is the derivative of \( p_{k+1} \). So \( p_{k+2}(t) = q_{k+1}(t)\cdot (1 + t^{2}) \) is a polynomial of degree \( k + 2 \).

		      Hence \( f \) is \( C^{k} \) for every nonnegative integer \( k \), which means \( f \) is \( C^{\infty} \). By the inverse function theorem, \( f \) has a \( C^{\infty} \) inverse. Thus \( f \) is a diffeomorphism.
		\item Let \( h: \openinterval{-1, 1} \to \openinterval{a, b} \) and \( h(x) = \frac{2x - a - b}{b - a} \) then \( h \) is a \( C^{\infty} \) bijection and its inverse \( h^{-1}(y) = \frac{y(b - a) + a + b}{2} \) is \( C^{\infty} \). Therefore \( h \) is a diffeomorphism, so \( \openinterval{a, b} \) and \( \openinterval{-1, 1} \) are diffeomorphic. Consequently, any two finite open intervals are diffeomorphic.
		\item Let \( g_{+} \) be the map \( g_{+}: \openinterval{0, \infty} \to \openinterval{a, \infty} \) given by \( g_{+}(x) = x + a \) and \( g_{-} \) be the map \( g_{-}: \openinterval{0, \infty} \to \openinterval{-\infty, b} \) given by \( g_{+}(x) = b - x \). The maps \( g_{+} \) and \( g_{-} \) are diffeomorphisms so \( \openinterval{0, \infty}, \openinterval{a, \infty}, \openinterval{-\infty, b} \) are diffeomorphic. Since \( \mathbb{R} \) and \( \openinterval{0, \infty} \) are diffeomorphic, it follows that \( \mathbb{R}, \openinterval{a, \infty}, \openinterval{-\infty, b} \) are diffeomorphic.
	\end{enumerate}
\end{proof}

\begin{problem}{1.4}[A diffeomorphism of an open cube with \(\mathbb{R}^{n}\)]
Show that the map
\[
	f\colon{\openinterval{-\frac{\pi}{2},\frac{\pi}{2}}}^{n}\to\mathbb{R}^{n}, \quad f(x_{1},\ldots,x_{n})=(\tan x_{1},\ldots,\tan x_{n}),
\]

is a diffeomorphism.
\end{problem}

\begin{proof}
	\( f \) is a bijection and a \( C^{\infty} \) because each \( f^{i} \) is \( C^{\infty} \). Moreover, the Jacobian matrix of \( f \) at \( x \) is the diagonal matrix
	\[
		\operatorname{diag}\left(\frac{1}{{(\cos x^{1})}^{2}}, \ldots, \frac{1}{{(\cos x^{n})}^{2}}\right)
	\]

	which is invertible. According to the inverse function theorem, \( f \) has a \( C^{\infty} \) inverse. Therefore \( f \) is a diffeomorphism.
\end{proof}

\begin{problem}{1.5}[A diffeomorphism of an open ball with \( \mathbb{R}^{n} \)]
Let \( \mathbf{0} = (0, 0) \) be the origin and \( B(\mathbf{0}, 1) \) the open unit disk in \( \mathbb{R}^{2} \). To find a diffeomorphism between \( B(\mathbf{0}, 1) \) and \( \mathbb{R}^{2} \), we identify \( \mathbb{R}^{2} \) with the \( xy \)-plane in \( \mathbb{R}^{3} \) and introduce the lower open hemisphere
\[
	S: x^{2} + y^{2} + {(z - 1)}^{2} = 1, \quad z < 1,
\]

in \( \mathbb{R}^{3} \) as an intermediate space. First note that the map
\[
	f: B(\mathbf{0}, 1) \to S, \qquad (a, b) \mapsto (a, b, 1 - \sqrt{1 - a^{2} - b^{2}})
\]

is a bijection.
\begin{enumerate}[label={(\alph*)}]
	\item The \textit{stereographic projection} \( g: S \to \mathbb{R}^{2} \) from \( (0, 0, 1) \) is the map that sends a point \( (a, b, c) \in S \) to the intersection of the line through \( (0, 0, 1) \) and \( (a, b, c) \) with the \( xy \)-plane. Show that it is given by
	      \[
		      (a, b, c) \mapsto (u, v) = \left(\frac{a}{1-c}, \frac{b}{1-c}\right), \quad c = 1 - \sqrt{1 - a^{2} - b^{2}}
	      \]

	      with inverse
	      \[
		      (u, v) \mapsto \left(\frac{u}{\sqrt{1 + u^{2} + v^{2}}}, \frac{v}{\sqrt{1 + u^{2} + v^{2}}}, 1 - \frac{1}{\sqrt{1 + u^{2} + v^{2}}}\right).
	      \]
	\item Composing the two maps \( f \) and \( g \) gives the map
	      \[
		      h = g\circ f: B(\mathbf{0}, 1) \to \mathbb{R}^{2},\qquad h(a, b) = \left(\frac{a}{\sqrt{1 - a^{2} - b^{2}}}, \frac{b}{\sqrt{1 - a^{2} - b^{2}}}\right).
	      \]

	      Find a formula for \( h^{-1}(u, v) = {(f^{-1} \circ g^{-1})}(u, v) \) and conclude that \( h \) is a diffeomorphism of the open disk \( B(\mathbf{0}, 1) \) with \( \mathbb{R}^{2} \).
	\item Generalize part (b) to \( \mathbb{R}^{n} \).
\end{enumerate}
\end{problem}

\begin{proof}
	\begin{enumerate}[label={(\alph*)},leftmargin=*]
		\item The intersection of the line through \( (0, 0, 1) \) and \( a, b, c \) with the \( xy \)-plane is of the form \( (1 - t)(0, 0, 1) + t(a, b, c) = (ta, tb, 1 - t + tc) \). Since \( (ta, tb, 1 - t + tc) \) lies on the \( xy \)-plane, \( 1 - t + tc = 0 \), which means \( t = \frac{1}{1 - c} \). Therefore \( g(a, b, c) = \left(\frac{a}{1 - c}, \frac{b}{1 - c}\right) \).

		      Let \( (u, v, 0) \) be a point on the \( xy \)-plane of \( \mathbb{R}^{3} \). The intersection of \( S \) and the line through \( (0, 0, 1) \) and \( (u, v, 0) \) is of the form \( (1 - t)(0, 0, 1) + t(u, v, 0) = (tu, tv, 1 - t) \) in which \( 0 < t < 1 \). Since \( (tu, tv, 1 - t) \) lies on \( S \), it follows that \( t^{2}u^{2} + t^{2}v^{2} + {(-t)}^{2} = 1 \), which means \( t = \frac{1}{\sqrt{1 + u^{2} + v^{2}}} \). Hence the map \( (u, v) \mapsto \left(\frac{u}{\sqrt{1 + u^{2} + v^{2}}}, \frac{v}{\sqrt{1 + u^{2} + v^{2}}}, 1 - \frac{1}{\sqrt{1 + u^{2} + v^{2}}}\right) \) is the inverse of \( g \).
		\item \( h = g\circ f \) and \( f, g \) are \( C^{\infty} \) bijections then so is \( h \). Moreover
		      \begin{align*}
			      h^{-1}(a, b) & = {(g\circ f)}^{-1}(a, b) = (f^{-1} \circ g^{-1}) (a, b)                                                                              \\
			                   & = f^{-1}\left( \frac{a}{\sqrt{1 + a^{2} + b^{2}}}, \frac{b}{\sqrt{1 + a^{2} + b^{2}}}, 1 - \frac{1}{\sqrt{1 + a^{2} + b^{2}}} \right) \\
			                   & = \left( \frac{a}{\sqrt{1 + a^{2} + b^{2}}}, \frac{b}{\sqrt{1 + a^{2} + b^{2}}} \right)
		      \end{align*}

		      is a \( C^{\infty} \) map. Therefore \( h \) is a diffeomorphism of the open disk \( B(\mathbf{0}, 1) \) with \( \mathbb{R}^{2} \).
		\item In \( \mathbb{R}^{2} \), consider the open ball \( B(\mathbf{0}, 1) \). Let \( S \) be the hemisphere \( x_{1}^{2} + \cdots + x_{n}^{2} + {(x_{n+1} - 1)}^{2} = 1 \) in \( \mathbb{R}^{n+1} \) in which \( x_{n+1} < 1 \). Define the maps \( f \) and \( g \) as follows:
		      \begin{align*}
			      f: B(\mathbf{0}, 1) \to S, & \qquad (x_{1}, \ldots, x_{n}) \mapsto \left(x_{1}, \ldots, x_{n}, 1 - \sqrt{1 - x_{1}^{2} - \cdots - x_{n}^{2}}\right)                                                                    \\
			      g: S \to \mathbb{R}^{n},   & \qquad (x_{1}, \ldots, x_{n}, x_{n+1}) \mapsto \left(\frac{x_{1}}{1 - \sqrt{1 + x_{1}^{2} + \cdots + x_{n}^{2}}}, \ldots , \frac{x_{n}}{1 - \sqrt{x_{1}^{2} + \cdots + x_{n}^{2}}}\right)
		      \end{align*}

		      The maps \( f, g \) are diffeomorphisms hence so is their composition \( g\circ f: B(\mathbf{0}, 1) \to \mathbb{R}^{n} \). Thus the open \(n\)-ball \( B(\mathbf{0}, 1) \) and \( \mathbb{R}^{n} \)
	\end{enumerate}
\end{proof}

\begin{problem}{1.6}[Taylor's theorem with remainder to order 2]
Prove that if \( f : \mathbb{R}^{2} \to \mathbb{R} \) is \( C^{\infty} \), then there exist \( C^{\infty} \) functions \( g_{11}, g_{12}, g_{22} \) on \( \mathbb{R}^{2} \) such that
\begin{multline*}
	f(x,y) = f(0,0) + \frac{\partial f}{\partial x} (0,0)x + \frac{\partial f}{\partial y} (0,0)y \\
	+ x^{2} g_{11}(x,y) + xy g_{12}(x,y) + y^{2} g_{22}(x,y).
\end{multline*}
\end{problem}

\begin{proof}
	From Taylor's theorem with remainer to order 1,
	\[
		f(x, y) = f(0, 0) + xg_{1}(x, y) + yg_{2}(x, y)
	\]

	for some \( C^{\infty} \) functions \( g_{1}, g_{2}: \mathbb{R}^{2} \to \mathbb{R} \) such that \( g_{1}(0, 0) = \frac{\partial f}{\partial x}(0, 0) \) and \( g_{2}(0, 0) = \frac{\partial f}{\partial y}(0, 0) \). Apply Taylor's theorem for \( g_{1}, g_{2} \)
	\[
		\begin{split}
			g_{1}(x, y) = g_{1}(0, 0) + xh_{11}(x, y) + yh_{12}(x, y) \\
			g_{2}(x, y) = g_{2}(0, 0) + xh_{21}(x, y) + yh_{22}(x, y)
		\end{split}
	\]

	for some \( C^{\infty} \) functions \( h_{11}, h_{12}, h_{21}, h_{22} \). Therefore
	\begin{align*}
		f(x, y) & = f(0, 0) + xg_{1}(x, y) + yg_{2}(x, y)                                                                                                                            \\
		        & = f(0, 0) + g_{1}(0, 0)x + x^{2}h_{11}(x, y) + xyh_{12}(x, y) + g_{2}(0, 0)y + xyh_{21}(x, y) + y^{2}h_{22}(x, y)                                                  \\
		        & = f(0, 0) + \frac{\partial f}{\partial x}(0, 0)x + \frac{\partial f}{\partial y}(0, 0)y + x^{2}h_{11}(x, y) + xy(h_{12}(x, y) + h_{21}(x, y)) + y^{2}h_{22}(x, y).
	\end{align*}

	Let \( g_{11}(x, y) = h_{11}(x, y), g_{12}(x, y) = h_{12}(x, y) + h_{21}(x, y), g_{22}(x, y) = h_{22}(y, y) \) then \( g_{11}, g_{12}, g_{22} \) are \( C^{\infty} \) and
	\[
		f(x, y) = f(0, 0) + \frac{\partial f}{\partial x}(0, 0)x + \frac{\partial f}{\partial y}(0, 0)y + x^{2}g_{11}(x, y) + xyg_{12}(x, y) + y^{2}g_{22}(x, y). \qedhere
	\]
\end{proof}

\begin{problem}{1.7}[A function with a removable singularity]
Let \( f : \mathbb{R}^{2} \to \mathbb{R} \) be a \( C^{\infty} \) function with \( f(0,0) = \dfrac{\partial f}{\partial x} (0,0) = \dfrac{\partial f}{\partial y} (0,0) = 0 \). Define

\[
	g(t,u) =
	\begin{cases}
		\dfrac{f(t, tu)}{t}, & \text{for } t \neq 0, \\
		0,                   & \text{for } t = 0.
	\end{cases}
\]

Prove that \( g(t,u) \) is \( C^{\infty} \) for \( (t,u) \in \mathbb{R}^{2} \).\@\textit{(Hint: Apply Problem 1.6.)}
\end{problem}

\begin{proof}
	Apply Problem 1.6 to \( f \), then there exist \( C^{\infty} \) maps \( g_{11}, g_{12}, g_{22}: \mathbb{R}^{2} \to \mathbb{R} \) such that
	\[
		f(x, y) = f(0, 0) + \frac{\partial f}{\partial x}(0, 0)x + \frac{\partial f}{\partial y}(0, 0)y + x^{2}g_{11}(x, y) + xyg_{12}(x, y) + y^{2}g_{22}(x, y).
	\]

	Since \( f(0, 0) = \dfrac{\partial f}{\partial x} (0,0) = \dfrac{\partial f}{\partial y} (0,0) = 0 \), it follows that
	\[
		f(x, y) = x^{2}g_{11}(x, y) + xyg_{12}(x, y) + y^{2}g_{22}(x, y)
	\]

	and
	\[
		\frac{f(t, tu)}{t} = \frac{t^{2}g_{11}(t, tu) + t^{2}ug_{12}(t, tu) + t^{2}u^{2}g_{22}(t, tu)}{t} = tg_{11}(t, tu) + tug_{12}(t, tu) + tu^{2}g_{22}(t, tu).
	\]

	The map \( (t, u) \mapsto tg_{11}(t, tu) + tug_{12}(t, tu) + tu^{2}g_{22}(t, tu) \) is \( C^{\infty} \) vanishes on \( \left\{ 0 \right\} \times \mathbb{R} \) so \( tg_{11}(t, tu) + tug_{12}(t, tu) + tu^{2}g_{22}(t, tu) = g(t, u) \) for \( (t, u) \in \mathbb{R}^{2} \). Hence \( g \) is \( C^{\infty} \).
\end{proof}

\begin{problem}{1.8}[Bijective \( C^{\infty} \) maps]
Define \( f : \mathbb{R} \to \mathbb{R} \) by \( f(x) = x^{3} \). Show that \( f \) is a bijective \( C^{\infty} \) map, but that \( f^{-1} \) is not \( C^{\infty} \). (This example shows that a bijective \( C^{\infty} \) map need not have a \( C^{\infty} \) inverse. In complex analysis, the situation is quite different: a bijective holomorphic map \( f : \mathbb{C} \to \mathbb{C} \) necessarily has a holomorphic inverse.)
\end{problem}

\begin{proof}
	\( f \) is continuous, \( \dfrac{df}{dx} = 3x^{2}, \dfrac{d^{2}f}{dxdx} = 6x, \dfrac{d^{3}f}{dxdxdx} = 6, \dfrac{d^{n}f}{dx\cdots dx} = 0 \) for \( n > 3 \). Therefore \( f \) is \( C^{\infty} \).

	\( f \) is bijective as its inverse is \( f^{-1} = g: \mathbb{R} \to \mathbb{R}, x \mapsto x^{1/3} \). However, \( f^{-1} \) is \( C^{0} \) but not \( C^{1} \) at \( x = 0 \). Therefore \( f^{-1} \) is not \( C^{\infty} \).
\end{proof}

\section{Tangent Vectors in \( \mathbb{R}^{n} \) as Derivations}

\begin{problem}{2.1}[Vector fields]
Let \( X \) be the vector field \( x \partial/\partial x + y\partial/\partial y \) and \( f(x, y, z) \) the function \( x^{2} + y^{2} + z^{2} \) on \( \mathbb{R}^{3} \). Compute \( Xf \).
\end{problem}

\begin{proof}
	\( Xf = x\dfrac{\partial f}{\partial x} + y\dfrac{\partial f}{\partial y} = x\cdot 2x + y\cdot 2y = 2(x^{2} + y^{2}) \).
\end{proof}

\begin{problem}{2.2}[Algebra structure on \( C^{\infty}_{p} \)]
Define carefully addition, multiplication, and scalar multiplication in \( C^{\infty}_{p} \). Prove that addition in \( C^{\infty}_{p} \) is commutative.
\end{problem}

\begin{proof}
	Let \( [(f, U)], [(g, V)] \in C^{\infty}_{p} \).

	For every \( \lambda \in \mathbb{R} \), \( (f, U) \in [(f, U)] \) and \( (g, V) \in [(g, V)] \), we define the maps \( h, k, \lambda f \) by
	\begin{align*}
		h: U \cap V \to \mathbb{R},         & \qquad h(x) = f(x) + g(x),                      \\
		k: U \cap V \to \mathbb{R},         & \qquad k(x) = f(x)g(x),                         \\
		(\lambda\cdot f): U \to \mathbb{R}, & \qquad (\lambda\cdot f)(x) = \lambda \cdot f(x)
	\end{align*}

	We define \( [(f, U)] + [(g, V)] \) to be the germ of \( h \) at \( p \), \(  [(f, U)]\cdot [(g, V)] \) to be the germ of \( k \) at \( p \), and \( \lambda\cdot [(f, U)] \) to be the germ of \( \lambda\cdot f \) at \( p \). It remains to show that this definition doesn't depend on the representatives.

	Consider \( (f_{0}, U_{0}) \in [(f, U)], (g_{0}, V_{0}) \in [(g, V)] \). The maps \( h_{0}, k_{0}, \lambda\cdot f_{0} \) are defined similarly as \( h, k, \lambda\cdot f \). There exists a neighborhood \( \mathbb{U} \) of \( p \) contained in \( U_{0} \cap U \), on which \( f, f_{0} \) agree, and a neighborhood \( \mathbb{V} \) of \( p \) contained in \( V_{0} \cap V \), on which \( g, g_{0} \) agree. Therefore, on the neighborhood \( \mathbb{U} \cap \mathbb{V} \subseteq (U \cap V) \cap (U_{0} \cap V_{0}) \) of \( p \), the maps \( h, h_{0} \) agree and \( k, k_{0} \) agree. Moreover, \( \lambda\cdot f \) and \( \lambda\cdot f_{0} \) agree on \( \mathbb{U} \). Hence \( [(f, U)] + [(g, V)] \), \( [(f, U)] \cdot [(g, V)] \) and \( \lambda\cdot[(f, U)] \) are well-defined.

	Let's verify the properties of an algebra
	\begingroup
	\allowdisplaybreaks%
	\begin{align*}
		\left([(f, U)] \cdot [g, V]\right) \cdot [h, W] & = [(f \cdot g, U \cap V)]\cdot [h, W]                                         \\
		                                                & = [((f \cdot g)\cdot h, (U \cap V) \cap W)]                                   \\
		                                                & = [(f \cdot (g \cdot h), U \cap (V \cap W))]                                  \\
		                                                & = [(f, U)]\cdot \left([(g, V)]\cdot [(h, W)]\right),                          \\
		([(f, U)] + [(g, V)])\cdot [(h, W)]             & = [(f + g, U\cap V)]\cdot [(h, W)]                                            \\
		                                                & = [((f + g)\cdot h, (U\cap V) \cap W)]                                        \\
		                                                & = [(f\cdot h + g\cdot h, (U\cap W) \cap (V\cap W))]                           \\
		                                                & = [(f\cdot h, U\cap W)] + [(g\cdot h, V\cap W)]                               \\
		                                                & = [(f, U)]\cdot [(h, W)] + [(g, V)]\cdot [(h, W)],                            \\
		[(f, U)]\cdot ([(g, V)] + [(h, W)])             & = [(f, U)]\cdot [(g + h, V\cap W)]                                            \\
		                                                & = [(f\cdot (g + h), U\cap (V\cap W))]                                         \\
		                                                & = [(f\cdot g + f\cdot h), (U\cap V) \cap (U\cap W)]                           \\
		                                                & = [(f\cdot g, U\cap V)] + [(f\cdot h), U\cap W]                               \\
		                                                & = [(f, U)]\cdot [(g, V)] + [(f, U)]\cdot [(h, W)],                            \\
		\lambda([(f, U)]\cdot [(g, V)])                 & = \lambda\cdot[((f\cdot g), U\cap V)]                                         \\
		                                                & = [(\lambda(f\cdot g), U\cap V)]                                              \\
		                                                & = [((\lambda\cdot f)\cdot g, U\cap V)] = [(\lambda\cdot f, U)]\cdot [(g, V)]  \\
		                                                & = [(f\cdot (\lambda\cdot g), U\cap V)] = [(f, U)]\cdot [(\lambda\cdot g, V)].
	\end{align*}
	\endgroup

	Hence \( C^{\infty}_{p} \) is an algebra over the real field. The addition is commutative because for any \( [(f, U)], [(g, V)] \in C^{\infty}_{p} \), one has
	\[
		[(f, U)] + [(g, V)] = [(f + g, U\cap V)] = [(g + f, V\cap U)] = [(g, V)] + [(f, U)].\qedhere
	\]
\end{proof}

\begin{problem}{2.3}[Vector space structure on derivations at a point]
Let \( D \) and \( D' \) be derivations at \( p \) in \( \mathbb{R}^{n} \), and \( c\in \mathbb{R} \). Prove that
\begin{enumerate}[label={(\alph*)},leftmargin=*,itemsep=0pt]
	\item the sum \( D + D' \) is a derivation at \( p \).
	\item the scalar multiple \( cD \) is a derivation at \( p \).
\end{enumerate}
\end{problem}

\begin{proof}
	\( D, D' \) are linear maps from \( C^{\infty}_{p} \) to \( \mathbb{R} \) then so are \( D + D' \) and \( cD \).

	For any \( f, g \in C^{\infty}_{p} \), one has
	\begingroup
	\allowdisplaybreaks%
	\begin{align*}
		(D + D')(fg) & = D(fg) + D'(fg)                                  \\
		             & = (D(f)g(p) + D(g)f(p)) + (D'(f)g(p) + D'(g)f(p)) \\
		             & = (D(f) + D'(f))g(p) + (D(g) + D'(g))f(p)         \\
		             & = (D + D')(f)g(p) + (D + D')(g)f(p),              \\
		(cD)(fg)     & = c\cdot D(fg)                                    \\
		             & = c\cdot (D(f)g(p) + D(g)f(p))                    \\
		             & = (c\cdot D)(f)g(p) + (c\cdot D)(g)f(p),
	\end{align*}
	\endgroup

	which means \( D + D' \) and \( cD \) satisfy the Leibniz rule. Hence \( D + D' \) and \( cD \) are derivations at \( p \).
\end{proof}

\begin{problem}{2.4}[Product of derivations]
Let \( A \) be an algebra over a field \( K \). If \( D_{1} \) and \( D_{2} \) are derivations of \( A \), show that \( D_{1} \circ D_{2} \) is not necessarily a derivation (it is if \( D_{1} \) or \( D_{2} = 0 \)), but \( D_{1} \circ D_{2} - D_{2} \circ D_{1} \) is always a derivation of \( A \).
\end{problem}

\begin{proof}
	Consider the algebra \( C^{\infty}_{p}(\mathbb{R}^{2}) \) over the real field, let \( D_{1} = \partial/\partial x \)  and \( D_{2} = \partial/\partial y \).
	\begin{align*}
		\left(\frac{\partial}{\partial x}\circ \frac{\partial}{\partial y}\right)(xy)                                                                                         & = \frac{\partial}{\partial x} x = 1, \\
		p^{1}\left(\frac{\partial}{\partial x}\circ \frac{\partial}{\partial y}\right)(x) + p^{2}\left(\frac{\partial}{\partial x}\circ \frac{\partial}{\partial y}\right)(y) & = 0
	\end{align*}

	which means \( D_{1} \circ D_{2} \) doesn't satisfy Leibniz rule, hence not a derivation. So \( D_{1} \circ D_{2} \) is not necessarily a derivation on \( A \).

	On the other hand, \( D = D_{1}\circ D_{2} - D_{2}\circ D_{1} \) is a linear map and
	\begin{align*}
		D(ab) & = D_{1}(D_{2}(ab)) - D_{2}(D_{1}(ab))                                                               \\
		      & = D_{1}((D_{2}a)b + aD_{2}b) - D_{2}((D_{1}a)b - aD_{1}b)                                           \\
		      & = ((D_{1}\circ D_{2})a)b + (D_{2}a)(D_{1}b) + (D_{1}a)(D_{2}b) + a((D_{1}\circ D_{2})b)             \\
		      & \phantom{=} - ((D_{2}\circ D_{1})a)b - (D_{1}a)(D_{2}b) - (D_{2}a)(D_{1}b) - a((D_{2}\circ D_{1})b) \\
		      & = ((D_{1}\circ D_{2} - D_{2}\circ D_{1})a)b - a((D_{1}\circ D_{2} - D_{2}\circ D_{1})b)             \\
		      & = (Da)b - aDb
	\end{align*}

	for all \( a, b \in A \) so \( D_{1}\circ D_{2} - D_{2}\circ D_{1} \) is a derivation of \( A \).
\end{proof}

\section{The Exterior Algebra of Multicovectors}

\section{Differential Forms on \( \mathbb{R}^{n} \)}
