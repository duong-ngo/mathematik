\documentclass[class=linearalgebra,crop=false]{standalone}

\begin{document}

\chapter{Kiến thức chuẩn bị}

\section*{Bài tập}

\setcounter{exercise}{0}

\begin{exercise}Chứng minh các tính chất kết hợp, giao hoán, phân phối của các phép toán hợp và giao trên tập hợp. Chứng minh công thức đối ngẫu De Morgan cho hiệu của hợp và giao của một họ tùy ý các tập hợp.
\end{exercise}

\begin{proof}Ta sử dụng các tập hợp $A, B, C$. Nhắc lại rằng, trên tập hợp các mệnh đề, phép toán $\text{and}$, $\text{or}$ có tính chất kết hợp, giao hoán và phân phối.

    \par \textit{Tính chất kết hợp}.

    \begin{gather*}
        (A\cup B)\cup C = \{ x\ |\ (x\in A \text{ or } x\in B) \text{ or } x\in C \} \\
        A\cup (B\cup C) = \{ x\ |\ x\in A \text{ or } (x\in B \text{ or } x\in C) \}
    \end{gather*}

    \par Toán tử $\text{or}$ có tính chất kết hợp, do đó $(A\cup B)\cup C = A\cup (B\cup C)$.

    \begin{gather*}
        (A\cap B)\cap C = \{ x\ |\ (x\in A \text{ and } x\in B) \text{ and } x\in C \} \\
        A\cap (B\cap C) = \{ x\ |\ x\in A \text{ and } (x\in B \text{ and } x\in C) \}
    \end{gather*}

    \par Toán tử $\text{and}$ có tính chất kết hợp, do đó $(A\cap B)\cap C = A\cap (B\cap C)$.

    \bigskip

    \par \textit{Tính chất giao hoán}.

    \begin{gather*}
        A\cup B = \{ x\ |\ x\in A \text{ or } x\in B \} \\
        B\cup A = \{ x\ |\ x\in B \text{ or } x\in A \}
    \end{gather*}

    \par Toán tử $\text{or}$ có tính chất giao hoán, do đó $A\cup B = B\cup A$.

    \begin{gather*}
        A\cap B = \{ x\ |\ x\in A \text{ and } x\in B \} \\
        B\cap A = \{ x\ |\ x\in B \text{ and } x\in A \}
    \end{gather*}

    \par Toán tử  $\text{and}$ có tính chất giao hoán, do đó $A\cap B = B\cap A$.

    \bigskip

    \par \textit{Tính chất phân phối}.

    \begin{align*}
        A\cap (B\cup C) &= \{ x\ |\ x\in A \text{ and } (x\in B \text{ or } x\in C) \} \\
                        &= \{ x\ |\ (x\in A \text{ and } x\in B) \text{ or } (x\in A \text{ and } x\in C) \} \\
                        & \text{(do phép $\text{and}, \text{or}$ có tính chất phân phối)} \\
                        &= (A\cap B)\cup (A\cap C)
    \end{align*}

    \begin{align*}
        A\cup (B\cap C) &= \{ x\ |\ x\in A \text{ or } (x\in B \text{ and } x\in C) \} \\
                        &= \{ x\ |\ (x\in A \text{ or } x\in B) \text{ and } (x\in A \text{ or } x\in C) \} \\
                        & \text{(do phép $\text{and}, \text{or}$ có tính chất phân phối)} \\
                        &= (A\cup B)\cap (A\cup C)
    \end{align*}

    \par Trong chứng minh công thức De Morgan, chúng ta sẽ sử dụng họ tập hợp $A_{i}$ với $i\in I$ nào đó.

    \begin{align*}
        X\setminus\bigcup_{i\in I} A_{i} &= \{ x\ |\ \overline{\exists i\in I, x\in A_{i}} \} \\
                                         &= \{ x\ |\ \forall i\in I, x\not\in A_{i} \} \\
                                         &= \{ x\ |\ \forall i\in I, x\in (X\setminus A_{i}) \} \\
                                         &= \bigcap_{i\in I}(X\setminus A_{i})
    \end{align*}

    \begin{align*}
        X\setminus\bigcap_{i\in I} A_{i} &= \{ x\ |\ \overline{\forall i\in I, x\in A_{i}} \} \\
                                         &= \{ x\ |\ \exists i\in I, x\not\in A_{i} \} \\
                                         &= \{ x\ |\ \exists i\in I, x\in (X\setminus A_{i}) \} \\
                                         &= \bigcup_{i\in I}(X\setminus A_{i})
    \end{align*}

\end{proof}

\begin{exercise}Chứng minh rằng
    \begin{enumerate}[itemsep=0pt,label = (\alph*)]
        \item $(A\setminus B) \cup (B\setminus A) = \emptyset \Longleftrightarrow A = B$,
        \item $A = (A\setminus B)\cup (A\cap B)$,
        \item $(A\setminus B) \cup (B\setminus A) = (A\cup B)\setminus (A\cap B)$,
        \item $A\cap (B\setminus C) = (A\cap B) \setminus (A\cap C)$,
        \item $A\cup (B\setminus A) = A\cup B$,
        \item $A\setminus (A\setminus B) = A\cap B$
    \end{enumerate}
\end{exercise}

\begin{proof}
    \begin{enumerate}[label = (\alph*)]
        \item Nếu $A = B$ thì $A\setminus B = B\setminus A = \emptyset$

        \par $\Rightarrow (A\setminus B)\cup (B\setminus A) = \emptyset$

        \par Nếu $(A\setminus B)\cup (B\setminus A) = \emptyset$ thì $A\setminus B = B\setminus A = \emptyset$

        \par $\Rightarrow A\subset B, B\subset A\Rightarrow A = B$.

        \item Đẳng thức đúng nếu $A$ là tập rỗng.

        \par Ngược lại, $A$ không rỗng, chọn $x$ là một phần tử của $A$.

        \par Có hai khả năng, $x\in B$ hoặc $x\not\in B$.

        \begin{align*}
            A &= \{ x\ |\ (x\in A \text{ and } x\in B)\text{ or }(x\in A \text{ and } x\not\in B) \} \\
              &= (A\cap B) \cup (A\setminus B)
        \end{align*}

        \item $a(x)$ là mệnh đề $x \in A$, $b(x)$ là mệnh đề $x\in B$.

        \[
            (A\setminus B)\cup (B\setminus A)=\{ x\ |\ (a(x) \text{ and } \overline{b(x)}) \text{ or } (\overline{a(x)} \text{ and } b(x)) \}
        \]

        \begin{align*}
            (a(x) \wedge \overline{b(x)}) \vee (\overline{a(x)} \wedge b(x)) &= ((a(x)\wedge \overline{b(x)})\vee \overline{a(x)}) \wedge ((a(x)\wedge \overline{b(x)})\vee b(x)) \\
            &= ((a(x)\vee \overline{a(x)}) \wedge (\overline{b(x)}\vee\overline{a(x)}))\wedge \\
            &\quad ((a(x)\vee b(x))\wedge (b(x)\vee \overline{b(x)})) \\
            &= (\overline{a(x)}\vee\overline{b(x)}) \wedge (a(x) \vee b(x)) \\
            &= \overline{a(x)\wedge b(x)} \wedge (a(x)\vee b(x)) \\
            &= (x\in A\cup B) \wedge (x\not\in A\cap B)
        \end{align*}

        \par $\Rightarrow (A\setminus B)\cup (B\setminus A) = (A\cup B)\setminus (A\cap B)$

        \item $a(x)$ là mệnh đề $x \in A$, $b(x)$ là mệnh đề $x\in B$, $c(x)$ là mệnh đề $x\in C$.

        \begin{align*}
            A\cap (B\setminus C) &= \{ x\ |\ a(x) \wedge (b(x) \wedge \overline{c(x)}) \} \\
                                 &= \{ x\ |\ (a(x) \wedge b(x)) \wedge (a(x) \wedge\overline{c(x)}) \} \\
                                 &= (A\cap B) \cap (A\setminus C) \\
                                 &= (A\cap B) \cap (A\setminus (A\cap C)) \\
                                 &= (A\cap B) \cap A \cap (X \setminus (A\cap C)) \\
                                 &= (A\cap B) \cap (X \setminus (A\cap C)) \\
                                 &= (A\cap B) \setminus (A\cap C)
        \end{align*}

        \item

        \begin{align*}
            A\cup (B\setminus A) &= \{ x\ |\ a(x) \vee (b(x) \wedge \overline{a(x)}) \} \\
                                 &= \{ x\ |\ (a(x) \vee b(x)) \wedge (a(x) \vee \overline{a(x)}) \} \\
                                 &= \{ x\ |\ a(x)\vee b(x) \} \\
                                 &= A\cup B
        \end{align*}

        \item

        \begin{align*}
            A\setminus (A\setminus B)&= \{ x\ |\ a(x) \wedge \overline{a(x)\wedge \overline{b(x)}} \} \\
                                     &= \{ x\ |\ a(x) \wedge (\overline{a(x)} \vee b(x)) \} \\
                                     &= \{ x\ |\ (a(x) \wedge \overline{a(x)}) \vee (a(x) \wedge b(x)) \} \\
                                     &= \{ x\ |\ a(x) \wedge b(x) \} \\
                                     &= A\cap B
        \end{align*}
    \end{enumerate}
\end{proof}

\begin{exercise}Chứng minh rằng
    \begin{enumerate}[itemsep=0pt,label = (\alph*)]
        \item $(A\times B)\cap (B\times A) = \emptyset \Longleftrightarrow A\cap B = \emptyset$
        \item $(A\times C)\cap (B\times D) = (A\cap B)\times (C\cap D)$
    \end{enumerate}
\end{exercise}

\begin{proof}
    \begin{enumerate}[label = (\alph*)]
        \item Giả sử tập hợp $(A\times B)\cap (B\times A)\ne\emptyset$. Tức là tồn tại bộ có thứ tự $(a, b)$ thuộc tập hợp này.
        \par Điều này kéo theo $(a, b)$ thuộc $A\times B$ lẫn $B\times A$.
        \par Theo định nghĩa của tích trực tiếp thì $a \in A$, $a\in B$. Dẫn đến việc $A\cap B \ne \emptyset$.
        \par Như vậy:
        \begin{align*}
            & (A\times B)\cap (B\times A)\ne\emptyset \Longleftrightarrow A\cap B\ne\emptyset \\
            \Leftrightarrow\quad & (A\times B)\cap (B\times A) = \emptyset \Longleftrightarrow A\cap B = \emptyset
        \end{align*}

        \item Nếu có một bộ có thứ tự $(x, y)$ sao cho $(x, y)\in (A\times B)\cap (C\times D)$ thì theo định nghĩa của tích trực tiếp, $x\in A$ và $x\in C$, $y\in B$ và $y\in D$.
        \par Từ điều này, ta suy ra $x\in A\cap C$ và $y\in B\cap D$.
        \par Cũng theo định nghĩa của tích trực tiếp, $(x, y)\in (A\cap C)\times(B\cap D)$.
        \par Do đó $(A\times B)\cap (C\times D) \subset (A\cap C)\times(B\cap D)$.

        \par Ngược lại, nếu có một bộ có thứ tự $(x, y)$ sao cho $(x, y)\in (A\cap C)\times(B\cap D)$ thì $x\in A\cap C$ và $y\in B\cap D$.
        \par Tức là $x\in A$, $y\in B$, $x\in C$, $y\in D$.
        \par Theo định nghĩa của tích trực tiếp, $(x, y)\in A\times B$, $(x, y)\in C\times D$. Như vậy là $(x, y)\in (A\times B)\cap (C\times D)$.
        \par Theo đó, $(A\cap C)\times (B\cap D) \subset (A\times B)\cap (C\times D)$.

        \par Hai điều vừa chứng minh cho thấy $(A\times C)\cap (B\times D) = (A\cap B)\times (C\cap D)$.

    \end{enumerate}
\end{proof}

\begin{exercise}Giả sử $f: X\rightarrow Y$ là một ánh xạ và $A, B\subset X$. Chứng minh rằng
    \begin{enumerate}[itemsep=0pt,label = (\alph*)]
        \item $f(A\cup B) = f(A)\cup f(B)$
        \item $f(A\cap B) \subset f(A)\cap f(B)$
        \item $f(A\setminus B) \supset f(A)\setminus f(B)$
    \end{enumerate}
    \par Hãy tìm các ví dụ để chứng tỏ rằng không có dấu bằng ở các mục (b) và (c).
\end{exercise}

\begin{proof}
    \begin{enumerate}[label = (\alph*)]
        \item $y$ là một phần tử của $Y$.
        \par Giả sử $y\in f(A\cup B)$. Khi đó, $\exists x\in A\cup B\subset X$ sao cho $f(x) = y$.
        \par $x\in A\cup B$ thì $x\in A$ hoặc $x\in B$, dẫn đến $f(x)\in f(A)$ hoặc $f(x)\in f(B)$.
        \par Như vậy, $y = f(x) \in f(A)\cup f(B)$.
        \par Do đó $f(A\cup B)\subset f(A)\cup f(B)$.

        \bigskip

        \par Giả sử $y\in f(A)\cup f(B)$ thì $y\in f(A)$ hoặc $y\in f(B)$.
        \par Suy ra tồn tại $x\in A$ hoặc $x\in B$ (tức là $x\in A\cup B$) sao cho $y = f(x) \in f(A)\cup f(B)$
        \par $x\in A\cup B$ nên $y = f(x) \in f(A\cup B)$.
        \par Do đó $f(A)\cup f(B)\subset f(A\cup B)$.

        \bigskip
        \par Từ hai điều trên, suy ra $f(A)\cup f(B) = f(A\cup B)$.

        \item $y$ là một phần tử của $Y$.

        \par Giả sử $y\in f(A\cap B)$. Khi đó $\exists x\in A\cap B$ sao cho $y = f(x) \in f(A\cap B)$.
        \par $x\in A\cap B$ nên $x\in A$ và $x\in B$, dẫn đến $f(x) \in f(A)$ và $f(x)\in f(B)$.
        \par Từ đó, $y = f(x) \in f(A)\cap f(B)$.
        \par Do đó, $f(A\cap B)\subset f(A)\cap f(B)$.

        \bigskip
        \par Ví dụ cho việc dấu bằng không thể xảy ra:
        \par Ánh xạ:
        \begin{align*}
            f:\quad& \mathbb{R}\rightarrow \{0\} \\
                   & x\mapsto 0
        \end{align*}
        \par Chọn $A = (1, 2)$ và $B = (3, 4)$.
        \par Lúc này $f(A\cap B) = f(\emptyset) = \emptyset$, còn $f(A)\cap f(B) = \{ 0 \}$.

        \item $y$ là một phần tử của $Y$.

        \par Giả sử $y\in f(A)\setminus f(B)$. Khi đó, $\exists x\in A$ sao cho $y = f(x)$.
        \par Nhưng $x\not\in B$ -- vì nếu $x\in B$ thì $y = f(x)\in f(B)$, trái với giả thiết.
        \par Như vậy, $x\in A\setminus B$, tức là $y = f(x)\in f(A\setminus B)$.
        \par Do đó, $f(A\setminus B)\supset f(A)\setminus f(B)$.

        \bigskip
        \par Ví dụ cho việc dấu bằng không thể xảy ra:
        \par Ánh xạ:
        \begin{align*}
            f:\quad& \mathbb{R}\rightarrow \{0\} \\
                   & x\mapsto 0
        \end{align*}
        \par Chọn $A = (1, 2)$ và $B = (3, 4)$.
        \par Lúc này $f(A)\setminus f(B) = \{0\}\setminus \{0\} = \emptyset$, còn $f(A\setminus B) = \{ 0 \}$.
    \end{enumerate}
\end{proof}

\begin{exercise}Cho ánh xạ $f: X\rightarrow Y$ và các tập con $A, B\subset Y$. Chứng minh rằng:
    \begin{enumerate}[itemsep=0pt,label = (\alph*)]
        \item $f^{-1}(A\cup B) = f^{-1}(A)\cup f^{-1}(B)$,
        \item $f^{-1}(A\cap B) = f^{-1}(A)\cap f^{-1}(B)$,
        \item $f^{-1}(A\setminus B) = f^{-1}(A)\setminus f^{-1}(B)$.
    \end{enumerate}
\end{exercise}

\begin{proof}
\end{proof}

\end{document}
