\documentclass[class=linearalgebra,crop=false]{standalone}

\renewcommand{\phi}{\varphi}

\begin{document}

\chapter{Kiến thức chuẩn bị}

\section*{Bài tập}

\setcounter{exercise}{0}

\begin{exercise}Chứng minh các tính chất kết hợp, giao hoán, phân phối của các phép toán hợp và giao trên tập hợp. Chứng minh công thức đối ngẫu De Morgan cho hiệu của hợp và giao của một họ tùy ý các tập hợp.
\end{exercise}

\begin{proof}Ta sử dụng các tập hợp $A, B, C$. Nhắc lại rằng, trên tập hợp các mệnh đề, phép toán $\text{and}$, $\text{or}$ có tính chất kết hợp, giao hoán và phân phối.

    \par \textit{Tính chất kết hợp}.

    \begin{gather*}
        (A\cup B)\cup C = \{ x\ |\ (x\in A \text{ or } x\in B) \text{ or } x\in C \} \\
        A\cup (B\cup C) = \{ x\ |\ x\in A \text{ or } (x\in B \text{ or } x\in C) \}
    \end{gather*}

    \par Toán tử $\text{or}$ có tính chất kết hợp, do đó $(A\cup B)\cup C = A\cup (B\cup C)$.

    \begin{gather*}
        (A\cap B)\cap C = \{ x\ |\ (x\in A \text{ and } x\in B) \text{ and } x\in C \} \\
        A\cap (B\cap C) = \{ x\ |\ x\in A \text{ and } (x\in B \text{ and } x\in C) \}
    \end{gather*}

    \par Toán tử $\text{and}$ có tính chất kết hợp, do đó $(A\cap B)\cap C = A\cap (B\cap C)$.

    \bigskip

    \par \textit{Tính chất giao hoán}.

    \begin{gather*}
        A\cup B = \{ x\ |\ x\in A \text{ or } x\in B \} \\
        B\cup A = \{ x\ |\ x\in B \text{ or } x\in A \}
    \end{gather*}

    \par Toán tử $\text{or}$ có tính chất giao hoán, do đó $A\cup B = B\cup A$.

    \begin{gather*}
        A\cap B = \{ x\ |\ x\in A \text{ and } x\in B \} \\
        B\cap A = \{ x\ |\ x\in B \text{ and } x\in A \}
    \end{gather*}

    \par Toán tử  $\text{and}$ có tính chất giao hoán, do đó $A\cap B = B\cap A$.

    \bigskip

    \par \textit{Tính chất phân phối}.

    \begin{align*}
        A\cap (B\cup C) &= \{ x\ |\ x\in A \text{ and } (x\in B \text{ or } x\in C) \} \\
                        &= \{ x\ |\ (x\in A \text{ and } x\in B) \text{ or } (x\in A \text{ and } x\in C) \} \\
                        & \text{(do phép $\text{and}, \text{or}$ có tính chất phân phối)} \\
                        &= (A\cap B)\cup (A\cap C)
    \end{align*}

    \begin{align*}
        A\cup (B\cap C) &= \{ x\ |\ x\in A \text{ or } (x\in B \text{ and } x\in C) \} \\
                        &= \{ x\ |\ (x\in A \text{ or } x\in B) \text{ and } (x\in A \text{ or } x\in C) \} \\
                        & \text{(do phép $\text{and}, \text{or}$ có tính chất phân phối)} \\
                        &= (A\cup B)\cap (A\cup C)
    \end{align*}

    \par Trong chứng minh công thức De Morgan, chúng ta sẽ sử dụng họ tập hợp $A_{i}$ với $i\in I$ nào đó.

    \begin{align*}
        X\setminus\bigcup_{i\in I} A_{i} &= \{ x\ |\ \overline{\exists i\in I, x\in A_{i}} \} \\
                                         &= \{ x\ |\ \forall i\in I, x\not\in A_{i} \} \\
                                         &= \{ x\ |\ \forall i\in I, x\in (X\setminus A_{i}) \} \\
                                         &= \bigcap_{i\in I}(X\setminus A_{i})
    \end{align*}

    \begin{align*}
        X\setminus\bigcap_{i\in I} A_{i} &= \{ x\ |\ \overline{\forall i\in I, x\in A_{i}} \} \\
                                         &= \{ x\ |\ \exists i\in I, x\not\in A_{i} \} \\
                                         &= \{ x\ |\ \exists i\in I, x\in (X\setminus A_{i}) \} \\
                                         &= \bigcup_{i\in I}(X\setminus A_{i})
    \end{align*}

\end{proof}

\begin{exercise}Chứng minh rằng
    \begin{enumerate}[itemsep=0pt,label = (\alph*)]
        \item $(A\setminus B) \cup (B\setminus A) = \emptyset \Longleftrightarrow A = B$,
        \item $A = (A\setminus B)\cup (A\cap B)$,
        \item $(A\setminus B) \cup (B\setminus A) = (A\cup B)\setminus (A\cap B)$,
        \item $A\cap (B\setminus C) = (A\cap B) \setminus (A\cap C)$,
        \item $A\cup (B\setminus A) = A\cup B$,
        \item $A\setminus (A\setminus B) = A\cap B$
    \end{enumerate}
\end{exercise}

\begin{proof}
    \begin{enumerate}[label = (\alph*)]
        \item Nếu $A = B$ thì $A\setminus B = B\setminus A = \emptyset$

        \par $\Rightarrow (A\setminus B)\cup (B\setminus A) = \emptyset$

        \par Nếu $(A\setminus B)\cup (B\setminus A) = \emptyset$ thì $A\setminus B = B\setminus A = \emptyset$

        \par $\Rightarrow A\subset B, B\subset A\Rightarrow A = B$.

        \item Đẳng thức đúng nếu $A$ là tập rỗng.

        \par Ngược lại, $A$ không rỗng, chọn $x$ là một phần tử của $A$.

        \par Có hai khả năng, $x\in B$ hoặc $x\not\in B$.

        \begin{align*}
            A &= \{ x\ |\ (x\in A \text{ and } x\in B)\text{ or }(x\in A \text{ and } x\not\in B) \} \\
              &= (A\cap B) \cup (A\setminus B)
        \end{align*}

        \item $a(x)$ là mệnh đề $x \in A$, $b(x)$ là mệnh đề $x\in B$.

        \[
            (A\setminus B)\cup (B\setminus A)=\{ x\ |\ (a(x) \text{ and } \overline{b(x)}) \text{ or } (\overline{a(x)} \text{ and } b(x)) \}
        \]

        \begin{align*}
            (a(x) \wedge \overline{b(x)}) \vee (\overline{a(x)} \wedge b(x)) &= ((a(x)\wedge \overline{b(x)})\vee \overline{a(x)}) \wedge ((a(x)\wedge \overline{b(x)})\vee b(x)) \\
            &= ((a(x)\vee \overline{a(x)}) \wedge (\overline{b(x)}\vee\overline{a(x)}))\wedge \\
            &\quad ((a(x)\vee b(x))\wedge (b(x)\vee \overline{b(x)})) \\
            &= (\overline{a(x)}\vee\overline{b(x)}) \wedge (a(x) \vee b(x)) \\
            &= \overline{a(x)\wedge b(x)} \wedge (a(x)\vee b(x)) \\
            &= (x\in A\cup B) \wedge (x\not\in A\cap B)
        \end{align*}

        \par $\Rightarrow (A\setminus B)\cup (B\setminus A) = (A\cup B)\setminus (A\cap B)$

        \item $a(x)$ là mệnh đề $x \in A$, $b(x)$ là mệnh đề $x\in B$, $c(x)$ là mệnh đề $x\in C$.

        \begin{align*}
            A\cap (B\setminus C) &= \{ x\ |\ a(x) \wedge (b(x) \wedge \overline{c(x)}) \} \\
                                 &= \{ x\ |\ (a(x) \wedge b(x)) \wedge (a(x) \wedge\overline{c(x)}) \} \\
                                 &= (A\cap B) \cap (A\setminus C) \\
                                 &= (A\cap B) \cap (A\setminus (A\cap C)) \\
                                 &= (A\cap B) \cap A \cap (X \setminus (A\cap C)) \\
                                 &= (A\cap B) \cap (X \setminus (A\cap C)) \\
                                 &= (A\cap B) \setminus (A\cap C)
        \end{align*}

        \item

        \begin{align*}
            A\cup (B\setminus A) &= \{ x\ |\ a(x) \vee (b(x) \wedge \overline{a(x)}) \} \\
                                 &= \{ x\ |\ (a(x) \vee b(x)) \wedge (a(x) \vee \overline{a(x)}) \} \\
                                 &= \{ x\ |\ a(x)\vee b(x) \} \\
                                 &= A\cup B
        \end{align*}

        \item

        \begin{align*}
            A\setminus (A\setminus B)&= \{ x\ |\ a(x) \wedge \overline{a(x)\wedge \overline{b(x)}} \} \\
                                     &= \{ x\ |\ a(x) \wedge (\overline{a(x)} \vee b(x)) \} \\
                                     &= \{ x\ |\ (a(x) \wedge \overline{a(x)}) \vee (a(x) \wedge b(x)) \} \\
                                     &= \{ x\ |\ a(x) \wedge b(x) \} \\
                                     &= A\cap B
        \end{align*}
    \end{enumerate}
\end{proof}

\begin{exercise}Chứng minh rằng
    \begin{enumerate}[itemsep=0pt,label = (\alph*)]
        \item $(A\times B)\cap (B\times A) = \emptyset \Longleftrightarrow A\cap B = \emptyset$
        \item $(A\times C)\cap (B\times D) = (A\cap B)\times (C\cap D)$
    \end{enumerate}
\end{exercise}

\begin{proof}
    \begin{enumerate}[label = (\alph*)]
        \item Giả sử tập hợp $(A\times B)\cap (B\times A)\ne\emptyset$. Tức là tồn tại bộ có thứ tự $(a, b)$ thuộc tập hợp này.
        \par Điều này kéo theo $(a, b)$ thuộc $A\times B$ lẫn $B\times A$.
        \par Theo định nghĩa của tích trực tiếp thì $a \in A$, $a\in B$. Dẫn đến việc $A\cap B \ne \emptyset$.
        \par Như vậy:
        \begin{align*}
            & (A\times B)\cap (B\times A)\ne\emptyset \Longleftrightarrow A\cap B\ne\emptyset \\
            \Leftrightarrow\quad & (A\times B)\cap (B\times A) = \emptyset \Longleftrightarrow A\cap B = \emptyset
        \end{align*}

        \item Nếu có một bộ có thứ tự $(x, y)$ sao cho $(x, y)\in (A\times B)\cap (C\times D)$ thì theo định nghĩa của tích trực tiếp, $x\in A$ và $x\in C$, $y\in B$ và $y\in D$.
        \par Từ điều này, ta suy ra $x\in A\cap C$ và $y\in B\cap D$.
        \par Cũng theo định nghĩa của tích trực tiếp, $(x, y)\in (A\cap C)\times(B\cap D)$.
        \par Do đó $(A\times B)\cap (C\times D) \subset (A\cap C)\times(B\cap D)$.

        \par Ngược lại, nếu có một bộ có thứ tự $(x, y)$ sao cho $(x, y)\in (A\cap C)\times(B\cap D)$ thì $x\in A\cap C$ và $y\in B\cap D$.
        \par Tức là $x\in A$, $y\in B$, $x\in C$, $y\in D$.
        \par Theo định nghĩa của tích trực tiếp, $(x, y)\in A\times B$, $(x, y)\in C\times D$. Như vậy là $(x, y)\in (A\times B)\cap (C\times D)$.
        \par Theo đó, $(A\cap C)\times (B\cap D) \subset (A\times B)\cap (C\times D)$.

        \par Hai điều vừa chứng minh cho thấy $(A\times C)\cap (B\times D) = (A\cap B)\times (C\cap D)$.

    \end{enumerate}
\end{proof}

\begin{exercise}Giả sử $f: X\rightarrow Y$ là một ánh xạ và $A, B\subset X$. Chứng minh rằng
    \begin{enumerate}[itemsep=0pt,label = (\alph*)]
        \item $f(A\cup B) = f(A)\cup f(B)$
        \item $f(A\cap B) \subset f(A)\cap f(B)$
        \item $f(A\setminus B) \supset f(A)\setminus f(B)$
    \end{enumerate}
    \par Hãy tìm các ví dụ để chứng tỏ rằng không có dấu bằng ở các mục (b) và (c).
\end{exercise}

\begin{proof}
    \begin{enumerate}[label = (\alph*)]
        \item $y$ là một phần tử của $Y$.
        \par Giả sử $y\in f(A\cup B)$. Khi đó, $\exists x\in A\cup B\subset X$ sao cho $f(x) = y$.
        \par $x\in A\cup B$ thì $x\in A$ hoặc $x\in B$, dẫn đến $f(x)\in f(A)$ hoặc $f(x)\in f(B)$.
        \par Như vậy, $y = f(x) \in f(A)\cup f(B)$.
        \par Do đó $f(A\cup B)\subset f(A)\cup f(B)$.

        \bigskip

        \par Giả sử $y\in f(A)\cup f(B)$ thì $y\in f(A)$ hoặc $y\in f(B)$.
        \par Suy ra tồn tại $x\in A$ hoặc $x\in B$ (tức là $x\in A\cup B$) sao cho $y = f(x) \in f(A)\cup f(B)$
        \par $x\in A\cup B$ nên $y = f(x) \in f(A\cup B)$.
        \par Do đó $f(A)\cup f(B)\subset f(A\cup B)$.

        \bigskip
        \par Từ hai điều trên, suy ra $f(A)\cup f(B) = f(A\cup B)$.

        \item $y$ là một phần tử của $Y$.

        \par Giả sử $y\in f(A\cap B)$. Khi đó $\exists x\in A\cap B$ sao cho $y = f(x) \in f(A\cap B)$.
        \par $x\in A\cap B$ nên $x\in A$ và $x\in B$, dẫn đến $f(x) \in f(A)$ và $f(x)\in f(B)$.
        \par Từ đó, $y = f(x) \in f(A)\cap f(B)$.
        \par Do đó, $f(A\cap B)\subset f(A)\cap f(B)$.

        \bigskip
        \par Ví dụ cho việc dấu bằng không thể xảy ra:
        \par Ánh xạ:
        \begin{align*}
            f:\quad& \mathbb{R}\rightarrow \{0\} \\
                   & x\mapsto 0
        \end{align*}
        \par Chọn $A = (1, 2)$ và $B = (3, 4)$.
        \par Lúc này $f(A\cap B) = f(\emptyset) = \emptyset$, còn $f(A)\cap f(B) = \{ 0 \}$.

        \item $y$ là một phần tử của $Y$.

        \par Giả sử $y\in f(A)\setminus f(B)$. Khi đó, $\exists x\in A$ sao cho $y = f(x)$.
        \par Nhưng $x\not\in B$ -- vì nếu $x\in B$ thì $y = f(x)\in f(B)$, trái với giả thiết.
        \par Như vậy, $x\in A\setminus B$, tức là $y = f(x)\in f(A\setminus B)$.
        \par Do đó, $f(A\setminus B)\supset f(A)\setminus f(B)$.

        \bigskip
        \par Ví dụ cho việc dấu bằng không thể xảy ra:
        \par Ánh xạ:
        \begin{align*}
            f:\quad& \mathbb{R}\rightarrow \{0\} \\
                   & x\mapsto 0
        \end{align*}
        \par Chọn $A = (1, 2)$ và $B = (3, 4)$.
        \par Lúc này $f(A)\setminus f(B) = \{0\}\setminus \{0\} = \emptyset$, còn $f(A\setminus B) = \{ 0 \}$.
    \end{enumerate}
\end{proof}

\begin{exercise}Cho ánh xạ $f: X\rightarrow Y$ và các tập con $A, B\subset Y$. Chứng minh rằng:
    \begin{enumerate}[itemsep=0pt,label = (\alph*)]
        \item $f^{-1}(A\cup B) = f^{-1}(A)\cup f^{-1}(B)$,
        \item $f^{-1}(A\cap B) = f^{-1}(A)\cap f^{-1}(B)$,
        \item $f^{-1}(A\setminus B) = f^{-1}(A)\setminus f^{-1}(B)$.
    \end{enumerate}
\end{exercise}

\begin{proof}$x\in X$.
    \begin{enumerate}[label = (\alph*)]
        \item
        \begin{align*}
            & x\in f^{-1}(A\cup B) \\
            \Leftrightarrow\quad & f(x) \in A\cup B \\
            \Leftrightarrow\quad & f(x)\in A \vee f(x)\in B \\
            \Leftrightarrow\quad & x\in f^{-1}(A) \vee x\in f^{-1}(B) \\
            \Leftrightarrow\quad & x\in f^{-1}(A)\cup f^{-1}(B)
        \end{align*}

        \par Do đó, $f^{-1}(A\cup B) = f^{-1}(A)\cup f^{-1}(B)$.

        \item
        \begin{align*}
            & x\in f^{-1}(A\cap B) \\
            \Leftrightarrow\quad & f(x) \in A\cap B \\
            \Leftrightarrow\quad & f(x) \in A \wedge f(x) \in B \\
            \Leftrightarrow\quad & x\in f^{-1}(A) \wedge x\in f^{-1}(B) \\
            \Leftrightarrow\quad & x\in f^{-1}(A)\cap f^{-1}(B)
        \end{align*}

        \par Do đó, $f^{-1}(A\cap B) = f^{-1}(A)\cap f^{-1}(B)$.

        \item
        \begin{align*}
            & x\in f^{-1}(A\setminus B) \\
            \Leftrightarrow\quad & f(x) \in A\setminus B \\
            \Leftrightarrow\quad & f(x) \in A \wedge f(x) \not\in B \\
            \Leftrightarrow\quad & x\in f^{-1}(A) \wedge x \not\in f^{-1}(B) \\
            \Leftrightarrow\quad & x\in f^{-1}(A)\setminus f^{-1}(B)
        \end{align*}

        \par Do đó, $f^{-1}(A\setminus B) = f^{-1}(A)\setminus f^{-1}(B)$.
    \end{enumerate}
\end{proof}

\begin{exercise}Chứng minh hai mệnh đề về ánh xạ ở cuối $\S{2}$.
\end{exercise}

\begin{proof}\textbf{Mệnh đề 2.8.} \textit{Hợp thành của hai đơn ánh lại là một đơn ánh. Hợp thành của hai toàn ánh lại là một toàn ánh. Hợp thành của hai song ánh lại là một song ánh.}
    \par Chọn hai ánh xạ $f, g$:
    \begin{gather*}
        f:\ X \rightarrow Y \\
        g:\ Y \rightarrow Z
    \end{gather*}
    \par Ánh xạ hợp:
    \[ g\circ f:\ X \rightarrow Z \]

    \begin{itemize}
        \item \textit{$f$ và $g$ là đơn ánh}

        \par Chọn $x_{1}\ne x_{2}$ và $x_{1}, x_{2} \in X$.
        \par Vì $f$ là đơn ánh nên $f(x_{1}) \ne f(x_{2})$.
        \par Vì $g$ là đơn ánh nên $g(f(x_{1})) \ne g(f(x_{2}))$.
        \par Theo định nghĩa về đơn ánh thì $g\circ f$ là đơn ánh.

        \item \textit{$f$ và $g$ là toàn ánh}

        \par Chọn $z \in Z$.
        \par Vì $g$ là toàn ánh nên $\exists y\in Y$ sao cho $g(y) = z$.
        \par Vì $f$ là toàn ánh nên $\exists x\in X$ sao cho $f(x) = y$.
        \par Như vậy, $\exists x\in X$ sao cho $g(f(x)) = z$.
        \par Theo định nghĩa về toàn ánh thì $g\circ f$ là toàn ánh.

        \item \textit{$f$ và $g$ là song ánh}

        \par $f$ và $g$ vừa là đơn ánh, vừa là toàn ánh.
        \par Theo hai ý trên, $g\circ f$ vừa là đơn ánh, vừa là toàn ánh, do đó cũng là song ánh.
    \end{itemize}

    \par \textbf{Mệnh đề 2.9.} \textit{(i) Giả sử $f: X\rightarrow Y$ và $g: Y\rightarrow Z$ là các ánh xạ. Khi đó, nếu $g\circ f$ là một đơn ánh thì $f$ cũng vậy; nếu $g\circ f$ là một toàn ánh thì $g$ cũng vậy.}
    \par \textit{(ii) Ánh xạ $f: X\rightarrow Y$ là một song ánh nếu và chỉ nếu tồn tại một ánh xạ $g: Y\rightarrow X$ sao cho $g\circ f = id_{X}$, $f\circ g = id_{Y}$}.

    \begin{enumerate}[label = (\roman*)]
        \item
        \begin{itemize}
            \item Chọn $x_{1}\ne x_{2}$ và $x_{1}, x_{2} \in X$.
            \par Do $g\circ f$ là đơn ánh nên $g(f(x_{1})) \ne g(f(x_{2}))$.
            \par Điều này dẫn đến việc $f(x_{1}) \ne f(x_{2})$.
            \par Theo định nghĩa về đơn ánh thì $f$ cũng là đơn ánh.

            \item Chọn $z\in Z$.
            \par Cho $g\circ f$ là toàn ánh nên $\exists x\in X$ sao cho $g(f(x)) = z$.
            \par Chọn $y = f(x)$ thì $g(y) = z$.
            \par Theo định nghĩa về toàn ánh thì $g$ cũng là toàn ánh.
        \end{itemize}

    \item \textit{Chiều thuận.} $f: X\rightarrow Y$ là song ánh thì $x\mapsto f(x)$ là tương ứng 1--1, $f(x)\mapsto x$ cũng là một tương ứng 1--1, ký hiệu ánh xạ này bởi $f^{-1}$.
    \par $f^{-1}(f(x)) = x$, $f(f^{-1}(y)) = y$.
    \par Do đó tồn tại ánh xạ $g = f^{-1}: Y\rightarrow X$ sao cho $g\circ f = id_{X}$ và $f\circ g = id_{Y}$.

    \bigskip
    \par \textit{Chiều đảo.} Tồn tại ánh xạ $g = f^{-1}: Y\rightarrow X$ sao cho $g\circ f = id_{X}$ và $f\circ g = id_{Y}$.
    \par Ánh xạ đồng nhất là song ánh, do đó, $f, g$ đều vừa là đơn ánh, vừa là toàn ánh, kéo theo $f$, $g$ là các song ánh.

    \end{enumerate}
\end{proof}

\begin{exercise}Xét xem ánh xạ $f: \mathbb{R}\rightarrow \mathbb{R}$ xác định bởi công thức $f(x) = x^{2} - 3x + 2$ có phải một đơn ánh hay toàn ánh hay không. Tìm $f(\mathbb{R})$, $f(0)$, $f^{-1}(0)$, $f([0,5])$, $f^{-1}([0,5])$.
\end{exercise}

\begin{proof}Với $x = 1$ hay $x = 2$ thì $f(x) = 0$, do đó $f$ không phải đơn ánh. Không tồn tại $x\in \mathbb{R}$ để $f(x) = -1$ (vì phương trình $x^{2} - 3x + 3 = 0$ vô nghiệm) nên $f$ cũng không phải toàn ánh.
    \par $x^{2} - 3x + 2 = \left(x - \frac{3}{2}\right){}^{2} - \frac{1}{4} \ge \frac{1}{4}$. Như vậy $f(\mathbb{R}) = \{ \frac{-1}{4} \}\cup(\frac{-1}{4},+\infty){}$.
    \par $f(0) = 2$.
    \par $x^{2} - 3x + 2 = (x - 1)(x - 2)$ nên $f^{-1}(0) = \{1, 2\}$.
    \par $f([0, 5]) = \left[\frac{-1}{4}, 12\right]$.
    \par $f^{-1}([0,5]) = \left[\frac{3 - \sqrt{21}}{2}, \frac{3 + \sqrt{21}}{2}\right]$.
\end{proof}

\begin{exercise}Giả sử $A$ là một tập hợp gồm đúng $n$ phần tử. Chứng minh rằng tập hợp $\mathcal{P}(A)$ các tập con của $A$ có đúng $2^{n}$ phần tử.
\end{exercise}

\begin{proof}Đặt $A = \{ a_{1}, a_{2}, \ldots a_{n} \}$.
    \par Một tập con của $A$ có thể được biểu thị bằng một dãy nhị phân độ dài $n$. Trong đó, vị trí thứ $i$ cũng dãy nhị phân bằng $0$ có nghĩa là tập con không chứa $a_{i}$, bằng $1$ có nghĩa là tập con chứa $a_{i}$.
    \par Phép tương ứng giữa các tập con của $A$ và các dãy nhị phân độ dài $n$ như trên là một tương ứng 1--1, tức là song ánh.
    \par Do đó, số phần tử của $\mathcal{P}(A)$ bằng số dãy nhị phân độ dài $n$. Theo quy tắc nhân, số dãy nhị phân độ dài $n$ là $2^{n}$.
    \par Vậy $|\mathcal{P}(A)| = 2^{n}$.
\end{proof}

\begin{exercise}Chứng minh rằng tập hợp $\mathbb{R}^{+}$ các số thực dương có lực lượng continum.
\end{exercise}

\begin{proof}Xét ánh xạ $f$:
    \begin{align*}
        f:\ &\mathbb{R} \rightarrow \mathbb{R}^{+} \\
            &x\mapsto e^{x}
    \end{align*}
    \par $f$ có ánh xạ ngược là $g$:
    \begin{align*}
        g:\ &\mathbb{R}^{+} \rightarrow \mathbb{R} \\
            &x\mapsto \ln x
    \end{align*}
    \par Như vậy, $f$ là song ánh. Do đó hai tập hợp $\mathbb{R}^{+}$, $\mathbb{R}$ có cùng lực lượng, tức là $\mathbb{R}^{+}$ cũng có lực lượng continum.
\end{proof}

\begin{exercise}Cho hai số thực $a, b$ với $a < b$. Chứng minh rằng khoảng số thực $(a, b) = \{ x\in \mathbb{R}|\ a < x < b \}$ có lực lượng continum.
\end{exercise}

\begin{proof}Ánh xạ $f_{1}$:
    \[
        \begin{split}
            f_{1}:\ & (a, b) \rightarrow \left(\frac{a - b}{2}, \frac{b - a}{2}\right) \\
                    & x \mapsto x - \frac{a + b}{2}
        \end{split}
    \]
    \par là song ánh. Do đó, $(a, b)$ và $\left(\frac{a - b}{2}, \frac{b - a}{2}\right)$ có cùng lực lượng.
    \par Ánh xạ $f_{2}$:
    \[
        \begin{split}
            f_{2}:\ & \left(\frac{a - b}{2}, \frac{b - a}{2}\right) \rightarrow \left(-\frac{\pi}{2}, \frac{\pi}{2}\right) \\
                    & x \mapsto \frac{\pi x}{b - a}
        \end{split}
    \]
    \par là song ánh. Do đó, $\left(\frac{a - b}{2}, \frac{b - a}{2}\right)$ và $\left(-\frac{\pi}{2}, \frac{\pi}{2}\right)$ có cùng lực lượng.
    \par Ánh xạ $f_{3}$:
    \[
        \begin{split}
            f_{3}:\ & \left(-\frac{\pi}{2}, \frac{\pi}{2}\right) \rightarrow \mathbb{R} \\
                    & x\mapsto \tan x
        \end{split}
    \]
    \par là song ánh. Do đó, $\left(-\frac{\pi}{2}, \frac{\pi}{2}\right)$ và $\mathbb{R}$ có cùng lực lượng.

    \par Cùng lực lượng là một quan hệ tương đương, do đó $(a, b)$ và $\mathbb{R}$ có cùng lực lượng, tức là $(a, b)$ có lực lượng continum.
\end{proof}

\begin{exercise}Một số thực được gọi là một \textit{số đại số} nếu nó là nghiệm của một phương trình đa thức nào đó với các hệ số nguyên. Chứng minh rằng tập hợp các số đại số là một tập vô hạn đếm được. Từ đó suy ra rằng tập hợp các số thực không phải là số đại số là một tập vô hạn không đếm được.
\end{exercise}

\begin{lemma}Hợp của hai tập vô hạn đếm được cũng là một tập vô hạn đếm được.
\end{lemma}

\begin{proof}[Chứng minh bổ đề] Đặt hai tập vô hạn đếm được là:
    \begin{gather*}
        A = \{ a_{1}, a_{2}, \ldots \} \\
        B = \{ b_{1}, b_{2}, \ldots \}
    \end{gather*}

    \par Ta liệt kê các phần tử của $A\cup B$ như sau:
    \begin{gather*}
        a_{1} \rightarrow 1 \\
        b_{1} \rightarrow 2 \\
        a_{2} \rightarrow 3 \\
        b_{2} \rightarrow 4 \\
        \vdots \\
        a_{n} \rightarrow 2n - 1 \\
        b_{n} \rightarrow 2n \\
        \vdots
    \end{gather*}
    \par Như vậy $A\cup B$ vô hạn đếm được.
\end{proof}

\begin{lemma}Hợp của vô hạn đếm được các tập vô hạn đến được là một tập vô hạn đếm được.
\end{lemma}

\begin{proof}[Chứng minh bổ đề]Ta có các tập hợp $A_{1}, A_{2}, A_{3} \ldots$, trong đó:
    \[ A_{i} = \{ a_{i.1}, a_{i.2}, \ldots \} \]
    \par Các phần tử của $\cup_{i\in \mathbb{N}}A_{i}$ được liệt kê như sau:
    \begin{align*}
        & a_{1.1} \rightarrow 1 \\
        & a_{1.2} \rightarrow 2,\ a_{2.1} \rightarrow 3 \\
        & a_{1.3} \rightarrow 4,\ a_{2.2} \rightarrow 5,\ a_{3.1} \rightarrow 6 \\
        & \qquad\vdots \\
        & a_{1.n} \rightarrow \binom{n}{2} + 1,\ a_{2.(n-1)} \rightarrow \binom{n}{2} + 2, \ldots,\ a_{n.1} \rightarrow \binom{n + 1}{2} \\
        & \qquad\vdots
    \end{align*}
\end{proof}

\begin{lemma}Tập hợp các tập con hữu hạn phần tử của một tập vô hạn đếm được cũng là một tập vô hạn đếm được.
\end{lemma}

\begin{proof}[Chứng minh bổ đề]Đặt $A$ là tập vô hạn đếm được.
    \par $A_{i}$ là tập các tập con có số phần tử bằng $i$ của $A$.
    \par Như vậy $A_{i}$ là một tập vô hạn đến được.
    \par Theo bổ đề 2 thì $\cup_{i\in\mathbb{N}}A_{i}$ là tập vô hạn đếm được.
\end{proof}

\begin{proof}Xét đa thức $f(X) \in \mathbb{Z}[X]$ có bậc $n \in \mathbb{N}$:
    \[ f(X) = a_{0} + a_{1}X + a_{2}X^{2} + \cdots + a_{n-1}X^{n-1} + a_{n}X^{n} \]
    \par Theo định lý cơ bản của đại số, phương trình này có $n$ nghiệm phức (nếu tính cả nghiệm bội). Nếu không tính nghiệm bội, giả sử phương trình có $k$ nghiệm. Ta thiết lập một đơn ánh từ tập các số phức là nghiệm của một đa thức hệ số nguyên vào tập các số hữu tỷ $\mathbb{Q}$ như sau:
    \[ x\longmapsto 2^{h - 1}\cdot 3^{a_{0}}\cdot 5^{a_{1}}\cdots p_{n+2}^{a_{n}} \]
    \par trong đó $x$ là nghiệm thứ $h$ của đa thức $f(X)$ và $p_{n}$ là số nguyên tố thứ $n$.
    \par Do $\mathbb{Q}$ vô hạn đếm được nên tập các số phức là nghiệm của một đa thức hệ số nguyên bậc $n$ là vô hạn đếm được.
    \par Theo các bổ đề trên, tập các số phức là nghiệm của một đa thức hệ số nguyên bậc dương (tạm gọi là số phức đại số) là vô hạn đếm được.
    \par Tập các số đại số là tập con của tập các số phức đại số. Thế nhưng tập các số đại số là vô hạn (vì các số nguyên đều là số đại số) nên tập các số đại số là vô hạn đếm được.
\end{proof}

\begin{exercise}Lập bảng cộng và bảng nhân của vành $\mathbb{Z}/n$ với $n = 12$ và $n = 15$. Dựa vào bảng, tìm các phần tử khả nghịch đối với phép nhân trong vành đó.
\end{exercise}

\begin{proof}Đối với vành $\mathbb{Z}/12$:

    \par Bảng cộng
    \\
    % chktex-file 44
    \begin{tabular}{c|cccccccccccc}
        + & 0 & 1 & 2 & 3 & 4 & 5 & 6 & 7 & 8 & 9 & 10 & 11 \\
        \hline
        0 & 0 & 1 & 2 & 3 & 4 & 5 & 6 & 7 & 8 & 9 & 10 & 11 \\
        1 & 1 & 2 & 3 & 4 & 5 & 6 & 7 & 8 & 9 & 10 & 11 & 0 \\
        2 & 2 & 3 & 4 & 5 & 6 & 7 & 8 & 9 & 10 & 11 & 0 & 1 \\
        3 & 3 & 4 & 5 & 6 & 7 & 8 & 9 & 10 & 11 & 0 & 1 & 2 \\
        4 & 4 & 5 & 6 & 7 & 8 & 9 & 10 & 11 & 0 & 1 & 2 & 3 \\
        5 & 5 & 6 & 7 & 8 & 9 & 10 & 11 & 0 & 1 & 2 & 3 & 4 \\
        6 & 6 & 7 & 8 & 9 & 10 & 11 & 0 & 1 & 2 & 3 & 4 & 5 \\
        7 & 7 & 8 & 9 & 10 & 11 & 0 & 1 & 2 & 3 & 4 & 5 & 6 \\
        8 & 8 & 9 & 10 & 11 & 0 & 1 & 2 & 3 & 4 & 5 & 6 & 7 \\
        9 & 9 & 10 & 11 & 0 & 1 & 2 & 3 & 4 & 5 & 6 & 7 & 8 \\
        10 & 10 & 11 & 0 & 1 & 2 & 3 & 4 & 5 & 6 & 7 & 8 & 9 \\
        11 & 11 & 0 & 1 & 2 & 3 & 4 & 5 & 6 & 7 & 8 & 9 & 10
    \end{tabular}

    \bigskip

    \par Bảng nhân
    \\
    % chktex-file 44
    \begin{tabular}{c|cccccccccccc}
        $\cdot$ & 0 & 1 & 2 & 3 & 4 & 5 & 6 & 7 & 8 & 9 & 10 & 11 \\
        \hline
        0 & 0 & 0 & 0 & 0 & 0 & 0 & 0 & 0 & 0 & 0 & 0 & 0 \\
        1 & 0 & 1 & 2 & 3 & 4 & 5 & 6 & 7 & 8 & 9 & 10 & 11 \\
        2 & 0 & 2 & 4 & 6 & 8 & 10 & 0 & 2 & 4 & 6 & 8 & 10 \\
        3 & 0 & 3 & 6 & 9 & 0 & 3 & 6 & 9 & 0 & 3 & 6 & 9 \\
        4 & 0 & 4 & 8 & 0 & 4 & 8 & 0 & 4 & 8 & 0 & 4 & 8 \\
        5 & 0 & 5 & 10 & 3 & 8 & 1 & 6 & 11 & 4 & 9 & 2 & 7 \\
        6 & 0 & 6 & 0 & 6 & 0 & 6 & 0 & 6 & 0 & 6 & 0 & 6 \\
        7 & 0 & 7 & 2 & 9 & 4 & 11 & 6 & 1 & 8 & 3 & 10 & 5 \\
        8 & 0 & 8 & 4 & 0 & 8 & 4 & 0 & 8 & 4 & 0 & 8 & 4 \\
        9 & 0 & 9 & 6 & 3 & 0 & 9 & 6 & 3 & 0 & 9 & 6 & 3 \\
        10 & 0 & 10 & 8 & 6 & 4 & 2 & 0 & 10 & 8 & 6 & 4 & 2 \\
        11 & 0 & 11 & 10 & 9 & 8 & 7 & 6 & 5 & 4 & 3 & 2 & 1
    \end{tabular}

    \par Các phần tử khả nghịch của $\mathbb{Z}/12$ gồm $1, 5, 7, 11$.

    \bigskip
    \bigskip

    \par Đối với vành $\mathbb{Z}/15$:

    \bigskip

    \par Bảng cộng
    \\
    \begin{tabular}{c|ccccccccccccccc}
        + & 0 & 1 & 2 & 3 & 4 & 5 & 6 & 7 & 8 & 9 & 10 & 11 & 12 & 13 & 14 \\
        \hline
        0 & 0 & 1 & 2 & 3 & 4 & 5 & 6 & 7 & 8 & 9 & 10 & 11 & 12 & 13 & 14  \\
        1 & 1 & 2 & 3 & 4 & 5 & 6 & 7 & 8 & 9 & 10 & 11 & 12 & 13 & 14 & 0  \\
        2 & 2 & 3 & 4 & 5 & 6 & 7 & 8 & 9 & 10 & 11 & 12 & 13 & 14 & 0 & 1  \\
        3 & 3 & 4 & 5 & 6 & 7 & 8 & 9 & 10 & 11 & 12 & 13 & 14 & 0 & 1 & 2  \\
        4 & 4 & 5 & 6 & 7 & 8 & 9 & 10 & 11 & 12 & 13 & 14 & 0 & 1 & 2 & 3 \\
        5 & 5 & 6 & 7 & 8 & 9 & 10 & 11 & 12 & 13 & 14 & 0 & 1 & 2 & 3 & 4 \\
        6 & 6 & 7 & 8 & 9 & 10 & 11 & 12 & 13 & 14 & 0 & 1 & 2 & 3 & 4 & 5 \\
        7 & 7 & 8 & 9 & 10 & 11 & 12 & 13 & 14 & 0 & 1 & 2 & 3 & 4 & 5 & 6 \\
        8 & 8 & 9 & 10 & 11 & 12 & 13 & 14 & 0 & 1 & 2 & 3 & 4 & 5 & 6 & 7 \\
        9 & 9 & 10 & 11 & 12 & 13 & 14 & 0 & 1 & 2 & 3 & 4 & 5 & 6 & 7 & 8 \\
        10 & 10 & 11 & 12 & 13 & 14 & 0 & 1 & 2 & 3 & 4 & 5 & 6 & 7 & 8 & 9 \\
        11 & 11 & 12 & 13 & 14 & 0 & 1 & 2 & 3 & 4 & 5 & 6 & 7 & 8 & 9 & 10 \\
        12 & 12 & 13 & 14 & 0 & 1 & 2 & 3 & 4 & 5 & 6 & 7 & 8 & 9 & 10 & 11 \\
        13 & 13 & 14 & 0 & 1 & 2 & 3 & 4 & 5 & 6 & 7 & 8 & 9 & 10 & 11 & 12 \\
        14 & 14 & 0 & 1 & 2 & 3 & 4 & 5 & 6 & 7 & 8 & 9 & 10 & 11 & 12 & 13
    \end{tabular}
    \bigskip

    \par Bảng nhân
    \\
    \begin{tabular}{c|ccccccccccccccc}
        $\cdot$ & 0 & 1 & 2 & 3 & 4 & 5 & 6 & 7 & 8 & 9 & 10 & 11 & 12 & 13 & 14 \\
        \hline
        0 & 0 & 0 & 0 & 0 & 0 & 0 & 0 & 0 & 0 & 0 & 0 & 0 & 0 & 0 & 0 \\
        1 & 0 & 1 & 2 & 3 & 4 & 5 & 6 & 7 & 8 & 9 & 10 & 11 & 12 & 13 & 14 \\
        2 & 0 & 2 & 4 & 6 & 8 & 10 & 12 & 14 & 1 & 3 & 5 & 7 & 9 & 11 & 13 \\
        3 & 0 & 3 & 6 & 9 & 12 & 0 & 3 & 6 & 9 & 12 & 0 & 3 & 6 & 9 & 12 \\
        4 & 0 & 4 & 8 & 12 & 1 & 5 & 9 & 13 & 2 & 6 & 10 & 14 & 3 & 7 & 11 \\
        5 & 0 & 5 & 10 & 0 & 5 & 10 & 0 & 5 & 10 & 0 & 5 & 10 & 0 & 5 & 10 \\
        6 & 0 & 6 & 12 & 3 & 9 & 0 & 6 & 12 & 3 & 9 & 0 & 6 & 12 & 3 & 9 \\
        7 & 0 & 7 & 14 & 6 & 13 & 5 & 12 & 4 & 11 & 3 & 10 & 2 & 9 & 1 & 8 \\
        8 & 0 & 8 & 1 & 9 & 2 & 10 & 3 & 11 & 4 & 12 & 5 & 13 & 6 & 14 & 7 \\
        9 & 0 & 9 & 3 & 12 & 6 & 0 & 9 & 3 & 12 & 6 & 0 & 9 & 3 & 12 & 6 \\
        10 & 0 & 10 & 5 & 0 & 10 & 5 & 0 & 10 & 5 & 0 & 10 & 5 & 0 & 10 & 5 \\
        11 & 0 & 11 & 7 & 3 & 14 & 10 & 6 & 2 & 13 & 9 & 5 & 1 & 12 & 8 & 4 \\
        12 & 0 & 12 & 9 & 6 & 3 & 0 & 12 & 9 & 6 & 3 & 0 & 12 & 9 & 6 & 3 \\
        13 & 0 & 13 & 11 & 9 & 7 & 5 & 3 & 1 & 14 & 12 & 10 & 8 & 6 & 4 & 2 \\
        14 & 0 & 14 & 13 & 12 & 11 & 10 & 9 & 8 & 7 & 6 & 5 & 4 & 3 & 2 & 1
    \end{tabular}

    \bigskip
    \par Các phần tử khả nghịch của $\mathbb{Z}/15$ bao gồm $1, 2, 4, 7, 8, 11, 13, 14$.
\end{proof}

\begin{exercise}Gọi $(\mathbb{Z}/n){}^{*}$ là tập hợp các phần tử khả nghịch đối với phép nhân trong $\mathbb{Z}/n$. Chứng minh rằng:
    \[ (\mathbb{Z}/n){}^{*} = \{ [x]\ |\ \text{$x$ và $n$ nguyên tố cùng nhau} \}. \]
\end{exercise}

\begin{lemma}Với $a, b, c \in \mathbb{Z}, a, b\ne 0$. Phương trình $ax + by = c$ có nghiệm nguyên khi và chỉ khi $\gcd(a, b)\ |\ c$.
\end{lemma}

\begin{proof}$a$ là một số nguyên khác $0$. Theo bổ đề trên, phương trình $ax + ny = 1$ có nghiệm nguyên khi và chỉ khi $(a, n) = 1$.
    \par Mà việc $ax + ny = 1$ có nghiệm nguyên khi và chỉ khi $a$ khả nghịch.
    \par Do đó ta có được điều phải chứng minh.
\end{proof}

\begin{lemma}Trong một vành $R$, $0\cdot a = a\cdot 0 = 0, \forall a\in R$.
\end{lemma}

\begin{proof}
    \begin{align*}
        0\cdot a + 0 &= 0\cdot a = (0 + a)\cdot a = 0\cdot a + 0\cdot a \\
        \Rightarrow 0 &= 0\cdot a\quad\text{(luật giản ước)}
    \end{align*}
    \begin{align*}
        a\cdot 0 + 0 &= a\cdot 0 = a\cdot (0 + 0) = a\cdot 0 + a\cdot 0 \\
        \Rightarrow 0 &= a\cdot 0\quad\text{(luật giản ước)}
    \end{align*}
    \par Đó là điều cần chứng minh.
\end{proof}

\begin{lemma}Trong một vành $R$ có đơn vị, phần tử đối của $a$ là $(-1)\cdot a = a\cdot(-1) = -a$.
\end{lemma}

\begin{proof}$R$ cũng là một nhóm nên phần tử $a$ bất kỳ của $R$ luôn có phần tử đối, ký hiệu $-a$.
    \par $-1$ là phần tử đối của $1$.
    \begin{align*}
        a + (-1)\cdot a &= 1\cdot a + (-1)\cdot a \\
                        &= (1 + (-1))\cdot a \\
                        &= 0\cdot a = 0 \\
        a + a\cdot (-1) &= a\cdot 1 + a\cdot (-1) \\
                        &= a\cdot (1 + (-1)) \\
                        &= a\cdot 0 = 0
    \end{align*}
    \par Đó là điều cần chứng minh.
\end{proof}

\begin{exercise}Cho $R$ là một vành có đơn vị. Gọi $R^{*}$ là tập hợp các phần tử khả nghịch đối với phép nhân trong $R$. Chứng minh rằng $R^{*}$ là một nhóm với các phép nhân của $R$.
\end{exercise}

\begin{proof}$R$ là một vành, tức là phép nhân trên $R$ có tính kết hợp.
    \par Gọi $x, y$ là hai phần tử khả nghịch.
    \par Khi đó, $x\cdot x^{-1} = x^{-1}\cdot x = 1$, $y\cdot y^{-1} = y^{-1}\cdot y = 1$.
    \[ (xy)\cdot (y^{-1}x^{-1}) = x(y\cdot y^{-1})x^{-1} = x\cdot x^{-1} = 1 \]
    \[ (y^{-1}x^{-1})\cdot (xy) = y^{-1}(x^{-1}\cdot x)y = y^{-1}\cdot y = 1 \]
    \par Do đó, $xy$ khả nghịch.

    \par Điều này dẫn đến phép nhân trên $R^{*}$ là đóng trên tập $R^{*}$.
    \par Phép nhân trên $R^{*}$ có tính chất kết hợp (là phép nhân trên $R$ nhưng bị hạn chế trên $R^{*}$).
    \par Phép nhân trên $R^{*}$ có phần tử trung lập là phần tử đơn vị trong phép nhân của $R$.
    \par Do vậy, $R^{*}$ là một nhóm đối với phép nhân của $R$.
\end{proof}

\begin{exercise}Cho $R$ là một vành có đơn vị $1\ne 0$ và các phần tử $x, y\in R$. Chứng minh rằng
    \begin{enumerate}[label = (\alph*)]
        \item Nếu $xy$ và $yx$ khả nghịch thì $x$ và $y$ khả nghịch.
        \item Nếu $R$ không có ước của không và $xy$ khả nghịch thì $x$ và $y$ khả nghịch.
    \end{enumerate}
\end{exercise}

\begin{lemma}Trên một vành, phần tử $x$ khả nghịch khi và chỉ khi nó khả nghịch bên phải lẫn bên trái.
\end{lemma}

\begin{proof}[Chứng minh bổ đề]\textit{Chiều thuận.} $x$ khả nghịch thì tồn tại $x'$ sao cho $xx' = x'x = 1$. Điều này nghĩa là $x$ khả nghịch cả bên phải lẫn bên trái.
    \par\textit{Chiều đảo.} $x$ khả nghịch bên trái thì tồn tại $a$ sao cho $ax = 1$. $x$ khả nghịch bên phải thì tồn tại $b$ sao cho $xb = 1$.
    \begin{align*}
        axb &= (ax)\cdot b = b \\
        axb &= a\cdot (xb) = a
    \end{align*}
    \par Từ điều trên, ta suy ra $a = b$, tức là $ax = xa = 1$. Như vậy $x$ khả nghịch.
\end{proof}

\begin{lemma}Trên một vành, phần tử $x$ khả nghịch khi và chỉ khi tồn tại $a, b$ sao cho $xa$ khả nghịch bên phải, $bx$ khả nghịch bên trái.
\end{lemma}

\begin{proof}[Chứng minh bổ đề]\textit{Chiều thuận.} $x$ khả nghịch.
    \par Chọn $a = b = 1$. Khi đó $xa$ khả nghịch bên phải, $bx$ khả nghịch bên trái.
    \par \textit{Chiều đảo.} Tồn tại $a, b$ sao cho $xa$ khả nghịch bên phải, $bx$ khả nghịch bên trái.
    \par $xa$ khả nghịch bên phải thì tồn tại $c$ sao cho $xa\cdot c = 1\Rightarrow x \cdot (ac) = 1$.
    \par $bx$ khả nghịch bên trái thì tồn tại $d$ sao cho $d\cdot bx = 1\Rightarrow (db)\cdot x = 1$.
    \par Như vậy, $x$ khả nghịch từ cả hai phía, dẫn đến $x$ khả nghịch.
\end{proof}

\begin{proof}
    \begin{enumerate}[label = (\alph*)]
        \item $xy$ và $yx$ khả nghịch nên tồn tại $a, b$ sao cho:
        \[ axy = xya = 1\qquad byx = yxb = 1 \]
        \[
            \begin{cases}
                x\cdot (ya) = 1 \\
                (by)\cdot x = 1
            \end{cases}
            \qquad
            \begin{cases}
                (ax)\cdot y = 1 \\
                y\cdot (xb) = 1
            \end{cases}
        \]
        \par $x, y$ khả nghịch từ cả hai phía nên theo bổ đề trên, $x, y$ khả nghịch.
        \item $xy$ khả nghịch nên tồn tại $a$ là phần tử nghịch đảo của $xy$, tức là
            \[ axy = xya = 1 \]
            \begin{align*}
                &axy\cdot ax = ax \\
                \Leftrightarrow\quad& ax(yax - 1) = 0
            \end{align*}
            \par $R$ không có ước của không nên $a = 0$ hoặc $x = 0$ hoặc $yax = 1$. Nhưng bất cứ phần tử nào nhân với $0$ cũng bằng $0$ nên $a \ne 0, x\ne 0$. Do đó $yax = 1$.
            \par Điều này kéo theo $axy = yax = 1$, tức là $y$ khả nghịch.
            \begin{align*}
                &yax\cdot ya = ya \\
                \Leftrightarrow\quad& (yax - 1)ya = 0
            \end{align*}
            \par $R$ không có ước của không, $a, x \ne 0$ nên $yax = 1$. Do đó $yax = xya = 1$. Tức là $x$ khả nghịch.
    \end{enumerate}
\end{proof}

\begin{exercise}Cho $R$ là một vành hữu hạn. Chứng minh rằng
    \begin{enumerate}[label = (\alph*)]
        \item Nếu $R$ không có ước của không thì nó có đơn vị và mọi phần tử khác không của $R$ đều khả nghịch.
        \item Nếu $R$ có đơn vị thì mọi phần tử khả nghịch một phía trong $R$ đều khả nghịch.
    \end{enumerate}
\end{exercise}

\begin{proof}
    \begin{enumerate}[label = (\alph*)]
        \item Trên nhóm $R$, ta định nghĩa phép trừ là phép cộng với phần tử đối:
            \begin{align*}
                -:\quad& R\times R \rightarrow R \\
                  & x - y \mapsto x + (-y)
            \end{align*}
            \par Trên vành $R$, ta định nghĩa phép lũy thừa:
            \begin{align*}
                x^{n} &= \underbrace{x\cdot x \cdots x}_{\text{$n$ phần tử $x$}}
            \end{align*}
            \par $R$ hữu hạn, ta đặt $R = \{0\}\cup \{a_{1},a_{2},\ldots, a_{n}\}$.
            \par Giả sử phản chứng, $\forall x\in R, \nexists a\ne 0: ax = x$.
            \par Do $R$ không có ước của không nên $\{ aa_{1}, aa_{2}, \ldots, aa_{n} \}$ đôi một khác nhau và khác không.
            \par Chọn $a_{i_{1}}\ne 0$. Theo giả sử phản chứng, $aa_{i_{1}} \ne a_{i_{1}}$, đặt $aa_{i_{1}} = a_{i_{2}}$.
            \par Theo giả sử phản chứng, $a^{2}a_{i_{1}} = aa_{i_{2}}$ khác $a_{i_{1}}, a_{i_{2}}$, đặt $a^{2}a_{i_{1}} = aa_{i_{2}} = a_{i_{3}}$.
            \par Lặp lại lập luận trên, ta được dãy đẳng thức sau:
            \begin{align*}
                aa_{i_{1}} &= a_{i_{2}} \\
                a^{2}a_{i_{1}} &= aa_{i_{2}} = a_{i_{3}} \\
                &\ddots \\
                a^{n-1}a_{i_{1}} &= \cdots = aa_{i_{n-1}} = a_{i_{n}}
            \end{align*}
            \par Các phần tử $a_{i_{1}}, a_{i_{2}}, \ldots, a_{i_{n}} \ne 0$ và đôi một khác nhau, đây chính là một hoán vị của $a_{1}, a_{2}, \ldots, a_{n}$.
            \par $a^{n}a_{i_{1}} = aa_{i_{n}}$ khác $a_{i_{1}}, a_{i_{2}}, \ldots, a_{i_{n}}$, do đó $aa_{i_{n}} = 0$. Điều này mâu thuẫn với giả thiết vành $R$ không có ước của không nên giả sử phản chứng là sai. Như vậy, $\forall x\in R, \exists a\ne 0: ax = x$. Phần tử $a$ như thế với mỗi $x$ là duy nhất vì $ax - bx = (a - b)x = 0 \Leftrightarrow a = b$ (do $R$ không có ước của không).
            \par Không mất tính tổng quát, giả sử $a$ là phần tử thỏa mãn $aa_{1} = a_{1}$. Xét ánh xạ $\ell_{a}: x\mapsto ax$. Do $R$ không có ước của không nên $\ell_{a}$ là đơn ánh, $R$ hữu hạn nên $\ell_{a}(R)$ là một hoán vị của $R$, do đó $\ell_{a}$ cũng là song ánh. Ta sẽ chứng minh $\ell_{a}$ là ánh xạ đồng nhất.
            \par $\ell_{a}$ có ít nhất 2 phần tử cố định là $0, a_{1}$. Giả sử phản chứng rằng $\ell_{a}$ không cố định phần tử nào khác.
            \par Trong các phần tử của $R\setminus\{0, a_{1}\}$, chọn $a_{i_{1}}$.
            \par $aa_{i_{1}}\ne 0, a_{1}$ vì $\ell_{a}$ là một đơn ánh, do đó tồn tại $a_{i_{2}} \in R\setminus\{0, a_{1}, a_{i_{1}}\}$ sao cho $aa_{i_{1}} = a_{i_{2}}$. Lập luận liên tiếp như vậy, ta được:
            \begin{itemize}
                \item $\exists a_{i_{3}} \in R\setminus \{ 0, a_{1}, a_{i_{1}}, a_{i_{2}} \}$ sao cho $aa_{i_{2}} = a_{i_{3}}$.
                \item $\exists a_{i_{4}} \in R\setminus \{ 0, a_{1}, a_{i_{1}}, a_{i_{2}}, a_{i_{3}} \}$ sao cho $aa_{i_{3}} = a_{i_{4}}$.
                \item $\ldots$
                \item Phần tử cuối cùng chưa được chọn $a_{i_{n - 1}}$ thỏa mãn $aa_{i_{n-2}} = a_{i_{n-1}}$
            \end{itemize}
            \par Do $\ell_{a}$ là đơn ánh nên $aa_{i_{n-1}}\not\in\{0, a_{1}, a_{i_{1}}, \ldots, a_{i_{n-1}}\}$. Điều này là vô lý. Do đó $\ell_{a}$ phải có ít nhất một phần tử cố định nữa. Không giảm tổng quát, giả sử phần tử cố định này là $a_{2}$.
            \par Do $R$ có hữu hạn phần tử, ta có thể lặp lại lập luận trên hữu hạn lần nhằm đi tới khẳng định $\ell_{a}$ cố định mọi phần tử. Điều này có nghĩa là $\forall x\in R, ax = x$.
            \par Tương tự, ta cũng chứng minh được tồn tại $b\ne 0$ sao cho $\forall x\in R, xb = x$.
            \par Như vậy, $\exists a,b\ne 0:\ \forall x\in R, ax = x, xb = x$.
            \par Áp dụng cho hai phần tử $a, b$, ta được $b = ab = a$. Do đó $x = ax = xa$ nên $a$ là phần tử đơn vị của $R$, kéo theo $R$ có đơn vị.
            \par ---
            \par $c$ là một phần tử khác không của $R$.
            \par $\{ ca_{1}, ca_{2}, \ldots, ca_{n} \}$ là một hoán vị của $R\setminus\{0\}$.
            \par $\{ a_{1}c, a_{2}c, \ldots, a_{n}c \}$ là một hoán vị của $R\setminus\{0\}$.
            \par $R$ có đơn vị, do đó tồn tại $a_{i}, a_{j}$ sao cho $ca_{i} = a_{j}c = 1$. Như vậy $c$ khả nghịch bên phải lẫn bên trái nên $c$ khả nghịch.
            \par Vậy mọi phần tử khác không của $R$ đều khả nghịch.
        \item Ta sẽ chứng minh nếu $ab = 1$ thì $ba = 1$.
            \par Xét ánh xạ $f: R\rightarrow R, x\mapsto bx$
            \par $x = y$ thì $f(x) = f(y)$
            \par $f(x) = f(y)$ thì $bx = by$, suy ra $abx = aby \Leftrightarrow x = y$.
            \par Như vậy, $f$ là đơn ánh. Nhưng vì $R$ hữu hạn nên $f$ cũng là song ánh, do đó tồn tại $z$ sao cho $f(z) = 1$, tức là $bz = 1$ mà $ab = 1$ nên $a = abz = z$, tức là $z = a$. Do đó $ab = ba = 1$.
            \par Điều trên có nghĩa là $a, b$ khả nghịch.
    \end{enumerate}
\end{proof}

\begin{exercise}Chứng minh rằng tập hợp các số thực
    \[ \mathbb{Q}(\sqrt{2}) = \{ a + b\sqrt{2}\ |\ a, b\in\mathbb{Q} \} \]
    \par lập nên một trường với các phép toán cộng và nhân thông thường.
\end{exercise}

\begin{proof}
    \begin{enumerate}[label = (\roman*)]
        \item $(a_{1} + b_{1}\sqrt{2}) + (a_{2} + b_{2}\sqrt{2}) = (a_{1} + a_{2}) + (b_{1} + b_{2})\sqrt{2}$. Do đó, $\mathbb{Q}(\sqrt{2})$ đóng kín với phép cộng thông thường.
        \item Phép cộng trên $\mathbb{Q}(\sqrt{2})$ có tính kết hợp.
        \begin{gather*}
            ((a_{1} + b_{1}\sqrt{2}) + (a_{2} + b_{2}\sqrt{2})) + (a_{3} + b_{3}\sqrt{2}) \\
            = (a_{1} + b_{1}\sqrt{2}) + ((a_{2} + b_{2}\sqrt{2}) + (a_{3} + b_{3}\sqrt{2}))
        \end{gather*}
        \item Phép cộng trên $\mathbb{Q}(\sqrt{2})$ có phần tử trung lập.
        \begin{gather*}
            (a + b\sqrt{2}) + 0 = 0 + (a + b\sqrt{2}) = a + b\sqrt{2}
        \end{gather*}
        \item Mọi phần tử đều có phần tử đối:
        \begin{gather*}
            (a + b\sqrt{2}) + (-a - b\sqrt{2}) = (-a - b\sqrt{2}) + (a + b\sqrt{2}) = 0
        \end{gather*}
        \item Phép cộng trên $\mathbb{Q}(\sqrt{2})$ có tính giao hoán.
        \begin{gather*}
            (a_{1} + b_{1}\sqrt{2}) + (a_{2} + b_{2}\sqrt{2}) \\
            = (a_{2} + b_{2}\sqrt{2}) + (a_{1} + b_{1}\sqrt{2}) \\
            = (a_{1} + a_{2}) + (b_{1} + b_{2})\sqrt{2}
        \end{gather*}
        \item $\mathbb{Q}(\sqrt{2})$ đóng kín với phép nhân thông thường.
        \item Phép nhân trên $\mathbb{Q}(\sqrt{2})$ có tính kết hợp.
        \begin{gather*}
            ((a_{1} + b_{1}\sqrt{2}) \cdot (a_{2} + b_{2}\sqrt{2}))\cdot(a_{3} + b_{3}\sqrt{2}) \\
            = (a_{1} + b_{1}\sqrt{2}) \cdot ((a_{2} + b_{2}\sqrt{2})\cdot(a_{3} + b_{3}\sqrt{2}))
        \end{gather*}
        \item Phép nhân trên $\mathbb{Q}(\sqrt{2})$ có tính phân phối từ cả hai phía với phép cộng.
        \begin{gather*}
            (a_{1} + b_{1}\sqrt{2})((a_{2} + b_{2}\sqrt{2}) + (a_{3} + b_{3}\sqrt{2})) \\
            = (a_{1} + b_{1}\sqrt{2})(a_{2} + b_{2}\sqrt{2}) + (a_{1} + b_{1}\sqrt{2})(a_{3} + b_{3}\sqrt{2})
        \end{gather*}
        \begin{gather*}
            ((a_{1} + b_{1}\sqrt{2}) + (a_{2} + b_{2}\sqrt{2}))(a_{3} + b_{3}\sqrt{2}) \\
            = (a_{1} + b_{1}\sqrt{2})(a_{3} + b_{3}\sqrt{2}) + (a_{2} + b_{2}\sqrt{2})(a_{3} + b_{3}\sqrt{2})
        \end{gather*}
        \item Phép nhân trên $\mathbb{Q}(\sqrt{2})$ có phần tử đơn vị.
        \begin{gather*}
            (a + b\sqrt{2})\cdot 1 = 1\cdot (a + b\sqrt{2}) = a + b\sqrt{2}
        \end{gather*}
        \item Phép nhân trên $\mathbb{Q}(\sqrt{2})$ có tính giao hoán:
        \begin{gather*}
            (a_{1} + b_{1}\sqrt{2})(a_{2} + b_{2}\sqrt{2}) = (a_{2} + b_{2}\sqrt{2})(a_{1} + b_{1}\sqrt{2})
        \end{gather*}
        \item Các phần tử khác không của $\mathbb{Q}$ có phần tử nghịch đảo.
        \begin{gather*}
            (a + b\sqrt{2})\left(\frac{a}{a^{2} - 2b^{2}} + \frac{-b}{a^{2}-2b^{2}}\sqrt{2}\right) \\
            =\left(\frac{a}{a^{2} - 2b^{2}} + \frac{-b}{a^{2}-2b^{2}}\sqrt{2}\right)(a + b\sqrt{2}) = 1
        \end{gather*}
    \end{enumerate}
    \par Như vậy, $\mathbb{Q}(\sqrt{2})$ lập nên một trường với hai phép toán cộng và nhân thông thường.
\end{proof}

\begin{exercise}Chứng minh rằng các trường $\mathbb{Q}(\sqrt{2})$ và $\mathbb{Q}(\sqrt{3})$ không đẳng cấu với nhau.
\end{exercise}

\begin{proof}Giả sử ngược lại là hai trường này đẳng cấu. Khi đó tồn tại một ánh xạ đẳng cấu $\phi$:
    \begin{gather*}
        \phi: \mathbb{Q}(\sqrt{2})\rightarrow \mathbb{Q}(\sqrt{3}) \\
        \phi(x) + \phi(y) = \phi(x + y) \\
        \phi(x)\phi(y) = \phi(xy)
    \end{gather*}
    \par Từ điều này ta suy ra $\phi$ biến phần tử trung lập, phần tử đơn vị của $\mathbb{Q}(\sqrt{2})$, lần lượt thành phần tử trung lập, phần tử đơn vị của $\mathbb{Q}(\sqrt{3})$ tức là $\phi(0) = 0, \phi(1) = 1$.
    \par $\phi(1) = 1$. Mà $\phi(x + 1) = \phi(x) + \phi(1) = \phi(x) + 1$ và $\phi(0) = 0$ nên bằng nguyên lý quy nạp, ta suy ra $\phi(n) = n$ với $n$ là một số nguyên.
    \bigskip
    \par Chọn lấy một số hữu tỷ không nguyên, đặt là $\frac{p}{q}$, trong đó $(p, q) = 1, q\ne 0$.
    \par $\phi$ cũng là một đẳng cấu vành nên $\phi\left(\frac{p}{q}\right)\phi(q) = \phi(p)$, tức là $\phi(\frac{p}{q}) = \frac{p}{q}$.
    \par Do đó, với mọi $x\in\mathbb{Q}$, $\phi(x) = x$.
    \bigskip
    \begin{align*}
        \phi(\sqrt{2})\phi(\sqrt{2})&= \phi(2) \\
        \Leftrightarrow \phi(\sqrt{2})^{2}&= 2
    \end{align*}
    \par Suy ra $\phi(\sqrt{2}) = \pm\sqrt{2}$. Nhưng cho dù $\phi(\sqrt{2})$ bằng $\sqrt{2}$ hay $-\sqrt{2}$ thì $\phi(\sqrt{2})$ cũng không thể thuộc trường $\mathbb{Q}(\sqrt{3})$.
    \par Như vậy giả sử phản chứng là sai. Hai trường $\mathbb{Q}(\sqrt{2})$ và $\mathbb{Q}(\sqrt{3})$ không đẳng cấu.
\end{proof}

\begin{exercise}Chứng minh rằng nếu số phức $z\not\in\mathbb{R}$ thì trường gồm các phần tử có dạng
    \[ \mathbb{R}(z) = \{ a + bz\ |\ a, b\in\mathbb{R} \} \]
    \par trùng với trường số phức $\mathbb{C}$.
\end{exercise}

\begin{proof}Trước tiên, ta chứng minh $\mathbb{R}(z)$ là một trường.
    \begin{enumerate}[label = (\roman*)]
        \item $\mathbb{R}(z)$ đóng với phép cộng thông thường:
        \[ (a_{1} + b_{1}z) + (a_{2} + b_{2}z) = (a_{1} + a_{2}) + (b_{1} + b_{2})z \]
        \item $\mathbb{R}(z)$ đóng với phép nhân thông thường:
        \begin{align*}
            (a_{1} + b_{1}z)(a_{2} + b_{2}z) &= (a_{1}a_{2} + b_{1}b_{2}z^{2}) + (a_{1}b_{2} + a_{2}b_{1})z \\
                                             &= (a_{1}a_{2} - b_{1}b_{2}z\overline{z}) + (a_{1}b_{2} + a_{2}b_{1} + b_{1}b_{2}(z + \overline{z}))z
        \end{align*}
        \item $\mathbb{R}(z) \subset \mathbb{C}$ nên phép cộng trên $\mathbb{R}(z)$ cũng có tính kết hợp, giao hoán, có phần tử trung lập, mọi phần tử đều có phần tử đối; phép nhân có tính kết hợp, phân phối với phép cộng, giao hoán, có đơn vị, phần tử khác 0 có nghịch đảo.
    \end{enumerate}
    \par Do đó $\mathbb{R}(z)$ là một trường với phép toán cộng và nhân thông thường.
    \par Đặt $z = p + q\iota \quad (q\ne 0)$.
    \par Ta sẽ chứng minh mọi số phức $a + b\iota$ đều biểu diễn được duy nhất dưới dạng $x + yz$.
    \begin{align*}
        &x + yz = a + b\iota \\
        \Leftrightarrow\quad& x + y(p + q\iota) = a + b\iota \\
        \Leftrightarrow\quad& (x + py) + qy\iota = a + b\iota \\
        \Leftrightarrow\quad&
        \begin{cases}
            x + py = a \\
            qy = b
        \end{cases}
        \Leftrightarrow\quad
        \begin{cases}
            y = \frac{b}{q} \\
            x = a - \frac{bp}{q}
        \end{cases}
    \end{align*}
    \par Như vậy, mọi số phức đều biểu diễn được dưới dạng $x + yz$, giả sử $x_{1} + y_{1}z = x_{2} + y_{2}z$.
    \begin{align*}
        &x_{1} + y_{1}z = x_{2} + y_{2}z \\
        \Leftrightarrow\quad&(x_{1} + y_{1}p) + y_{1}q\iota = (x_{2} + y_{2}p) + y_{2}q\iota \\
        \Leftrightarrow\quad&
        \begin{cases}
            qy_{1} = qy_{2}\quad\text{($q\ne 0$)} \\
            x_{1} + y_{1}p = x_{2} + y_{2}p
        \end{cases}
        \Leftrightarrow\quad
        \begin{cases}
            y_{1} = y_{2} \\
            x_{1} = x_{2}
        \end{cases}
    \end{align*}
    \par Do đó, biểu diễn $x + yz$ là duy nhất với mọi số phức.

    \par Từ những điều trên, ta kết luận $\mathbb{R}(z) = \mathbb{C}$.
\end{proof}

\begin{exercise}Chứng minh rằng các trường $\mathbb{C}$ và $\mathbb{Z}/p$, với $p$ nguyên tố, không là trường được sắp toàn phần đối với bất kỳ thứ tự nào.
\end{exercise}

\begin{proof}\textit{Đối với trường $\mathbb{C}$.}
    \par Giả sử phản chứng, $\mathbb{C}$ được sắp toàn phần.
    \par Ta xét các phần tử $0, 1, \iota, -1$.
    \begin{itemize}
        \item Trường hợp 1. $0\le 1 \le \iota$ (hiểu là $1$ bị kẹp bởi $0$ và $\iota$, nên việc đảo thứ tự sẽ tương tự).
            \par $0\le 1\Rightarrow -1\le 0$. Mà $\iota\ge 0$ nên $1\cdot\iota \le \iota^{2} \Rightarrow \iota \le -1$. Điều này mâu thuẫn với $-1 \le 0 \le \iota$.
        \item Trường hợp 2. $0\le\iota\le 1$
            \par $1\ge\iota\Rightarrow 1 - \iota\ge 0$.
            \par $1\ge 0, \iota\ge 0\Rightarrow 1 + \iota \ge 0$
            \par $\Rightarrow \iota(1+\iota)\ge 0\Rightarrow \iota - 1\ge 0$. Điều này mâu thuẫn với $1 - \iota \ge 0$.
        \item Trường hợp 3. $\iota \le 0 \le 1$
            \par $\Rightarrow-\iota \ge 0\Rightarrow (-\iota)(-\iota)\ge (-\iota)\cdot (-1) \Rightarrow -1\ge\iota \Rightarrow 1 + \iota \le 0\Rightarrow\iota\le -1$
            \par $-\iota \ge 1\Rightarrow (-\iota)(-\iota) \ge -\iota \Rightarrow -1 \ge -\iota \Rightarrow \iota\ge 1$
            \par $1\le -1$, điều này mâu thuẫn với giả thiết.
    \end{itemize}
    \par Vậy $\mathbb{C}$ không được sắp toàn phần với bất kỳ thứ tự nào.
    \bigskip
    \par\textit{Đối với trường $\mathbb{Z}/p$.}
    \par $\mathbb{Z}/p$ là trường hữu hạn. Ta đặt $\mathbb{Z}/p = \{ 0, 1, \ldots, p - 1 \}$.
    \par Giả sử phản chứng, $\mathbb{Z}/p$ được sắp toàn phần.
    \begin{itemize}
        \item Trường hợp 1. $0\le 1$.
        \begin{align*}
            \Rightarrow\quad& 1\le 2 \\
            \Rightarrow\quad& 2\le 3 \\
                            & \vdots \\
            \Rightarrow\quad& p-2\le p-1 \\
            \Rightarrow\quad& p-1\le 0
        \end{align*}
        \par Điều cuối cùng mâu thuẫn với giả thiết.
        \item Trường hợp 2. $1\le 0$.
        \begin{align*}
            \Rightarrow\quad& 2\le 1 \\
            \Rightarrow\quad& 3\le 2 \\
                            & \vdots \\
            \Rightarrow\quad& p-1\le p-2 \\
            \Rightarrow\quad& 0\le p-1
        \end{align*}
        \par Điều cuối cùng mâu thuẫn với giả thiết.
    \end{itemize}
    \par Vậy $\mathbb{Z}/p$ không được sắp toàn phần với bất kỳ thứ tự nào.
\end{proof}

\begin{exercise}Chứng minh rằng ánh xạ $\phi: \mathbb{R}\rightarrow\mathbb{C}^{*}$ xác định bởi $\phi(x) = \cos x + \iota\sin x$ là một đồng cấu từ nhóm $\mathbb{R}$ với phép cộng vào nhóm $\mathbb{C}^{*}$ với phép nhân. Tìm tập giá trị của $\phi$. Đồng cấu $\phi$ có phải là một toàn cấu hay một đơn cấu không?
\end{exercise}

\begin{proof}
    \begin{align*}
        \phi(x)\phi(y) &= (\cos x + \iota\sin x)(\cos y + \iota\sin y) \\
                       &= (\cos x\cos y - \sin x\sin y) + (\sin x\cos y + \cos x\sin y)\iota \\
                       &= \cos (x + y) + \iota\sin(x + y) \\
                       &= \phi(x + y)
    \end{align*}
    \par Do đó $\phi$ là một đồng cấu từ $(\mathbb{R},+)$ vào $(\mathbb{C}^{*},\cdot)$.
    \par Tập giá trị của $\phi$ là các số phức có module bằng 1.
    \par $\phi$ không phải đơn cấu vì $\phi(x) = \phi(x + 2k\pi)$.
    \par $\phi$ không phải toàn cấu vì không tồn tại $x\in\mathbb{R}$ sao cho $\phi(x) = 0$.
\end{proof}

\begin{exercise}Chứng minh rằng đối với số phức $z$:
    \begin{align*}
        z = \overline{z} &\Longleftrightarrow z\in\mathbb{R} \\
        z = -\overline{z}&\Longleftrightarrow \text{$z$ là thuần ảo}
    \end{align*}
\end{exercise}

\begin{proof}Đặt $z = a + b\iota$.
    \par $\Rightarrow\overline{z} = a - b\iota$.
    \par $\Rightarrow a = \frac{1}{2}(z + \overline{z}), b = \frac{1}{2\iota}(z - \overline{z})$.
    \begin{align*}
        &z \in \mathbb{R} \Leftrightarrow b = 0 \Leftrightarrow z = \overline{z} \\
        &\text{$z$ là thuần ảo}\Leftrightarrow a = 0 \Leftrightarrow z = -\overline{z}
    \end{align*}
\end{proof}

\begin{exercise}Khi nào thì tích hai số phức là một số thực? Khi nào thì tổng và tích hai số phức đều là số thực?
\end{exercise}

\begin{proof}Đặt $z_{1} = r_{1}(\cos\phi_{1} + \iota\sin\phi_{1}), z_{2} = r_{2}(\cos\phi_{2} + \iota\sin\phi_{2})$.
    \par $z_{1}z_{2} = r_{1}r_{2}(\cos(\phi_{1} + \phi_{2}) + \iota\sin(\phi_{1} + \phi_{2}))$.
    \par $z_{1}z_{2}\in\mathbb{R}$ khi và chỉ khi $\sin(\phi_{1} + \phi_{2}) = 0\Leftrightarrow\phi_{1} + \phi_{2}\equiv 0\pmod\pi$
    \par Tức là $\arg(z_{1}) + \arg(z_{2}) = k\pi\quad (k\in\mathbb{Z})$.
    \par Vậy tích hai số phức là một số thực khi và chỉ khi tổng argument của chúng có dạng $k\pi\ (k\in\mathbb{Z})$.

    \bigskip
    \par $z_{1} + z_{2}, z_{1}z_{2}\in\mathbb{Z}$ nên $z_{1}, z_{2}$ là nghiệm của một đa thức bậc 2, hệ số thực: $aX^{2} + bX + c$.
    \par Theo công thức nghiệm phương trình bậc hai:
    \[ z_{1} = \frac{-b + \sqrt{b^{2} - 4ac}}{2a}\qquad z_{2} = \frac{-b - \sqrt{b^{2} - 4ac}}{2a} \]
    \par Theo đó, $z_{1}$ và $z_{2}$ là hai số phức liên hợp.
    \par Ngược lại, tổng và tích của hai số phức liên hợp là các số thực.
    \par Vậy tổng và tích hai số phức là một số thực khi và chỉ khi chúng là hai số phức liên hợp.
\end{proof}

\begin{exercise}Tính $\iota^{77}, \iota^{99}, \iota^{-57}, \iota^{n}, (1 + \iota){}^{n}$
\end{exercise}

\begin{proof}
    \begin{align*}
        \iota^{77}&= \iota^{4\cdot 19 + 1} = (\iota^{4}){}^{19}\cdot\iota = \iota
    \end{align*}
    \begin{align*}
        \iota^{99}&= \iota^{4\cdot 24 + 3} = (\iota^{4}){}^{24}\cdot\iota^{3} = \iota^{3} = -\iota
    \end{align*}
    \begin{align*}
        \iota^{-57}&= \iota^{4\cdot(-14) - 1} = (\iota^{4}){}^{-14}\cdot\iota^{-1} = \iota^{-1} = -\iota
    \end{align*}
    \begin{align*}
        \iota^{n}&= \left(\cos\frac{\pi}{2} + \iota\sin\frac{\pi}{2}\right){}^{n}= \cos\frac{n\pi}{2} + \iota\sin\frac{n\pi}{2}
    \end{align*}
    \begin{align*}
        (1+\iota){}^{n}&= \left(\sqrt{2}\left(\cos\frac{\pi}{4} + \iota\sin\frac{\pi}{4}\right)\right){}^{n} \\
                     &= \sqrt{2^{n}}\left(\cos\frac{n\pi}{4} + \iota\sin\frac{n\pi}{4}\right)
    \end{align*}
\end{proof}

\begin{exercise}Chứng minh các đẳng thức
    \begin{align*}
        (1+\iota){}^{8n} &= 2^{4n} \\
        (1+\iota){}^{4n} &= (-1){}^{n}2^{2n},\quad(n\in\mathbb{Z})
    \end{align*}
\end{exercise}

\begin{proof}Sử dụng đẳng thức ở bài tập trước.
    \begin{align*}
        (1+\iota){}^{8n} &= \sqrt{2^{8n}}\left(\cos\frac{8n\pi}{4} + \sin\frac{8n\pi}{4}\right) \\
                         &= 2^{4n}\left(\cos 2n\pi + \iota\sin 2n\pi\right) \\
                         &= 2^{4n}
    \end{align*}

    \begin{align*}
        (1+\iota){}^{4n} &= \sqrt{2^{4n}}\left(\cos\frac{4n\pi}{4} + \sin\frac{4n\pi}{4}\right) \\
                         &= 2^{2n}\left(\cos n\pi + \iota\sin n\pi\right) \\
                         &= (-1){}^{n}2^{2n}
    \end{align*}
\end{proof}

\begin{exercise}Chứng minh rằng nếu $z + z^{-1} = 2\cos\phi$ trong đó $\phi\in\mathbb{R}$ thì $z^{n} + z^{-n} = 2\cos n\phi$, với mọi $n\in\mathbb{N}$.
\end{exercise}

\begin{proof}Ta chứng minh bằng quy nạp. Mệnh đề $z^{n} + z^{-n} = 2\cos n\phi$ đúng với $n = 1$.
    \par Giả sử mệnh đề đúng với $n = k$.
    \begin{align*}
        z^{k+1} + z^{-k-1} &= (z^{k} + z^{-k})(z + z^{-1}) - (z^{k-1} + z^{1-k}) \\
                           &= 2\cos k\phi \cdot 2\cos\phi - 2\cos (k-1)\phi \\
                           &= 2\cos(k+1)\phi + 2\cos(k-1)\phi - 2\cos(k-1)\phi \\
                           &= 2\cos(k+1)\phi
    \end{align*}
    \par Vậy mệnh đề đúng với $n = k + 1$, do đó đúng với mọi $n\in\mathbb{N}$.
\end{proof}

\begin{exercise}Tính
    \begin{enumerate}[label = (\alph*)]
        \item $\dfrac{1 - 2\iota}{4 + 3\iota}$
        \item $\dfrac{(1 - \iota){}^{n}}{(1 - \sqrt{3}\iota){}^{n}}$
        \item $\dfrac{(1 + \sqrt{3}\iota){}^{n}}{(1 + \iota){}^{n + 1}}$
    \end{enumerate}
\end{exercise}

\begin{proof}
    \begin{enumerate}[label = (\alph*)]
        \item \begin{align*}
            \frac{1 - 2\iota}{4 + 3\iota} &= \frac{(1 - 2\iota)(4 - 3\iota)}{25} \\
                                              &= \frac{-2 - 11\iota}{25}
        \end{align*}
        \item \begin{align*}
            \frac{(1 - \iota){}^{n}}{(1 - \sqrt{3}\iota){}^{n}} &= \frac{\sqrt{2^{n}}\left(\cos\dfrac{-\pi}{4} +\iota\sin\dfrac{-\pi}{4}\right){}^{n}}{2^{n}\left(\cos\dfrac{-\pi}{3} + \iota\sin\dfrac{-\pi}{3}\right){}^{n}} \\
                &=\frac{1}{\sqrt{2^{n}}}\left(\cos\frac{n\pi}{12}+\iota\sin\frac{n\pi}{12}\right)
        \end{align*}
        \item \begin{align*}
            \frac{(1 + \sqrt{3}\iota){}^{n}}{(1+\iota){}^{n+1}} &= \frac{2^{n}\left(\cos\dfrac{\pi}{3} + \iota\sin\dfrac{\pi}{3}\right){}^{n}}{\sqrt{2^{n}}\left(\cos\dfrac{\pi}{4} + \iota\sin\dfrac{\pi}{4}\right){}^{n+1}} \\
            &= \sqrt{2^{n}}\left(\cos\left(\frac{n\pi}{12}-\frac{\pi}{4}\right) + \iota\sin\left(\frac{n\pi}{12} - \frac{\pi}{4}\right)\right)
        \end{align*}
    \end{enumerate}
\end{proof}

\begin{exercise}
    \begin{enumerate}[label = (\alph*)]
        \item Tìm dạng lượng giác của số phức $(1 + \iota\tan\phi)/(1 - \iota\tan\phi)$,
        \item Trên mặt phẳng phức, tìm tập hợp các điểm tương ứng với
            \[ \{ z = (1 + \iota t)/(1 - \iota t)\ |\ t\in\mathbb{R} \} \]
    \end{enumerate}
\end{exercise}

\begin{proof}
    \begin{enumerate}[label = (\alph*)]
        \item
            \begin{align*}
                \frac{1 + \iota\tan\phi}{1 - \iota\tan\phi} &= \frac{\cos\phi + \iota\sin\phi}{\cos\phi - \iota\sin\phi} \\
                                                            &= \frac{\cos\phi + \iota\sin\phi}{\cos(-\phi) + \iota\sin(-\phi)} \\
                                                            &= \cos(2\phi) + \iota\sin(2\phi)
            \end{align*}
        \item Vì $t\in\mathbb{R}$ nên tồn tại $\phi\in\left(-\frac{\pi}{2},\frac{\pi}{2}\right)$ sao cho $\tan\phi = t$.
            \par Theo ý (a), $z = \cos (2\phi) + \iota(2\phi)$. Do đó, tập hợp điểm $z$ là đường tròn đơn vị, loại đi hai điểm $-1$.
    \end{enumerate}
\end{proof}

\begin{exercise}Đẳng thức sau đây có đúng không: $\sqrt[ns]{z^{s}} = \sqrt[n]{z}$, trong đó $z\in\mathbb{C}, n, s\in\mathbb{N}$?
\end{exercise}

\begin{proof}Với $s = 1$, đẳng thức luôn đúng.
    \par Với $s > 1$, ta chỉ cần xét trường hợp $z = 1$. Căn bậc $ns$ của $1$ gồm $ns$ giá trị, trong khi đó, căn bậc $n$ của $1$ chỉ có $n$ giá trị. $ns$ giá trị trên khác nhau. Do đó khi $s > 1$ thì đẳng thức trên không đúng.
\end{proof}

\begin{exercise}
    \begin{enumerate}[label = (\alph*)]
        \item Tìm các căn bậc ba của $1 + \iota$, $1 - \sqrt{3}\iota$.
        \item Tìm các căn bậc $n$ của $\iota$, $1 - \iota$, $1 + \sqrt{3}\iota$.
    \end{enumerate}
\end{exercise}

\begin{proof}
    \begin{enumerate}[label = (\alph*)]
        \item $1 + \iota = \sqrt{2}\left(\cos\frac{\pi}{4} + \iota\sin\frac{\pi}{4}\right)$. Các căn bậc ba của $1 + \iota$ bao gồm
            \[
                \begin{cases}
                    \sqrt[6]{2}\left(\cos\frac{\pi}{12} + \iota\sin\frac{\pi}{12}\right) \\
                    \sqrt[6]{2}\left(\cos\frac{3\pi}{4} + \iota\sin\frac{3\pi}{4}\right) \\
                    \sqrt[6]{2}\left(\cos\frac{17\pi}{12} + \iota\sin\frac{17\pi}{12}\right)
                \end{cases}
            \]
            \par $1 - \sqrt{3}\iota = 2\left(\cos\frac{-\pi}{3} + \iota\sin\frac{-\pi}{3}\right)$. Các căn bậc ba của $1 - \sqrt{3}\iota$ bao gồm
            \[
                \begin{cases}
                    \sqrt[3]{2}\left(\cos\frac{-\pi}{9} + \iota\sin\frac{-\pi}{9}\right) \\
                    \sqrt[3]{2}\left(\cos\frac{5\pi}{9} + \iota\sin\frac{5\pi}{9}\right) \\
                    \sqrt[3]{2}\left(\cos\frac{11\pi}{9} + \iota\sin\frac{11\pi}{9}\right)
                \end{cases}
            \]
        \item
            \begin{align*}
                \sqrt[n]{\iota} &= \left\{ \cos\left(\frac{\pi}{2n} + \frac{2k\pi}{n}\right) + \iota\sin\left(\frac{\pi}{2n} + \frac{2k\pi}{n}\right)\ \Big{|} \ k = \overline{0, n-1} \right\} \\
                \sqrt[n]{1-\iota} &= \left\{ \sqrt[2n]{2}\left(\cos\left(\frac{-\pi}{4n} + \frac{2k\pi}{n}\right) + \iota\sin\left(\frac{-\pi}{4n} + \frac{2k\pi}{n}\right)\right)\ \Big{|} \ k = \overline{0, n-1} \right\} \\
                \sqrt[n]{1 + \sqrt{3}\iota} &= \left\{ \sqrt[n]{2}\left(\cos\left(\frac{\pi}{3n} + \frac{2k\pi}{n}\right) + \iota\sin\left(\frac{\pi}{3n} + \frac{2k\pi}{n}\right)\right)\ \Big{|}\ k = \overline{0, n-1} \right\}
            \end{align*}
    \end{enumerate}
\end{proof}

\begin{exercise}Chứng minh rằng tổng các căn bậc $n$ của một số phức bất kỳ đều bằng $0$.
\end{exercise}

\begin{proof}Với số phức $z = 0$, chỉ có một căn bậc $n$ là chính số $0$.
    \par Xét trường hợp $z\ne 0$. $z = |z|(\cos\phi + \iota\sin\phi)$. Các căn bậc $n$ của $z$ bao gồm
    \[ \sqrt[n]{|z|}\left(\cos\left(\frac{\phi}{n} + \frac{2k\pi}{n}\right) + \iota\sin\left(\frac{\phi}{n} + \frac{2k\pi}{n}\right)\right),\quad k = \overline{0, n-1} \]
    \par Các điểm biểu diễn căn bậc $n$ của $z$ trên mặt phẳng phức, theo công thức trên là các đỉnh của một đa giác đều $n$ đỉnh. Đặt các đỉnh đó là
    \[ A_{0}(a_{0}), A_{1}(a_{1}), \ldots, A_{n-1}(a_{n-1}) \]
    \par Sử dụng phép quay tâm $O$, góc $\frac{2\pi}{n}$:
    \begin{align*}
        A_{0}&\mapsto A_{1} \\
        A_{1}&\mapsto A_{2} \\
        &\vdots \\
        A_{n-1}&\mapsto A_{0}
    \end{align*}
    \par Như vậy, phép quay trên biến vector $\sum\limits^{n-1}_{i=0}\overrightarrow{OA_{i}}$ thành chính nó. Mà góc quay của phép quay này $\not\equiv 0\pmod{2\pi}$ nên vector này là vector không.
    \par Do đó $\sum\limits^{n-1}_{i=0}a_{i} = 0$. Đây là điều cần chứng minh.
\end{proof}

\begin{exercise}Phân tích các đa thức sau thành nhân tử bất khả quy trong các vành $\mathbb{R}[X]$ và $\mathbb{C}[X]$:
    \begin{enumerate}[label = (\alph*)]
        \item $X^{3} + 3X^{2} + 5X + 3$
        \item $X^{3} - X^{2} - X - 2$
    \end{enumerate}
\end{exercise}

\begin{proof}
    \begin{enumerate}[label = (\alph*)]
        \item
            \begin{align*}
                X^{3} + 3X^{2} + 5X + 3 &= X^{3} + X^{2} + 2X^{2} + 2X + 3X + 3 \\
                                        &= X^{2}(X + 1) + 2X(X + 1) + 3(X + 1) \\
                                        &= (X + 1)(X^{2} + 2X + 3) \\
                                        &= (X + 1)(X + 1 + \sqrt{2}\iota)(X + 1 - \sqrt{2}\iota)
            \end{align*}
        \item
            \begin{align*}
                X^{3} - X^{2} - X - 2 &= X^{3} - 2X^{2} + X^{2} - 2X + X - 2 \\
                                      &= X^{2}(X - 2) + X(X - 2) + (X - 2) \\
                                      &= (X - 2)(X^{2} + X + 1) \\
                                      &= (X - 2)(X - \frac{-1 - \sqrt{3}\iota}{2})(X - \frac{-1 + \sqrt{3}\iota}{2})
            \end{align*}
    \end{enumerate}
\end{proof}

\begin{exercise}Chứng minh rằng đa thức $X^{3m} + X^{3n + 1} + X^{3p  + 2}$ chia hết cho đa thức $X^{2} + X + 1$, với mọi $m, n, p$ nguyên dương.
\end{exercise}

\begin{proof}Đặt $j$ là nghiệm của $X^{2} + X + 1$, khi đó $j^{3} = 1$
    \begin{align*}
        j^{3m} + j^{3n + 1} + j^{3p + 2} &= 1 + j + j^{2} \\
                                         &= 0
    \end{align*}
    \par Như vậy, nghiệm của $X^{2} + X + 1$ cũng là nghiệm của $X^{3m} + X^{3n + 1} + X^{3p + 2}$.
    \par Do đó $X^{3m} + X^{3n + 1} + X^{3p +2}$ chia hết cho $X^{2} + X + 1$, với mọi $m, n, p$ nguyên dương.
\end{proof}

\begin{exercise}Tìm tất cả các bộ ba nguyên dương $m, n, p$ sao cho đa thức $X^{3m} - X^{3n + 1} + X^{3p + 2}$ chia hết cho đa thức $X^{2} - X + 1$.
\end{exercise}

\begin{proof}Đa thức $X^{3m} - X^{3n + 1} + X^{3p + 2}$ chia hết cho đa thức $X^{2} - X + 1$ khi và chỉ khi nghiệm của $X^{2} - X + 1$ cũng là nghiệm của $X^{3m} - X^{3n + 1} + X^{3p + 2}$.
    \par Đặt $j$ là nghiệm của đa thức $X^{2} - X + 1$ thì $j^{3} = -1$.
    \begin{align*}
        j^{3m} - j^{3n + 1} + j^{3p + 2} &= (-1){}^{m} - (-1){}^{n}j + (-1){}^{p}j^{2}
    \end{align*}
    \par Nếu $m, n, p$ cùng tính chẵn lẻ thì biểu thức trên bằng không.
    \par Nếu $m, n, p$ không cùng tính chẵn lẻ thì biểu thức trên khác không.
    \par Bộ số $m, n, p$ nguyên dương cần tìm là các số nguyên dương cùng tính chẵn lẻ.
\end{proof}

\end{document}
