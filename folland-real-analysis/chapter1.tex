\chapter{Measures}

\section{Introduction}

\section{\(\sigma\)-algebras}

\subsection*{Exercises}

\begin{enumerate}[itemsep=0pt]
	\item A family of set \( \mathcal{R} \subseteq \mathcal{P}(X) \) is called a \textbf{ring} if it is closed under finite unions and differences. A ring that is closed under countable unions is called a \textbf{\(\sigma\)-ring}.
	      \begin{enumerate}[itemsep=0pt,label={\alph*.}]
		      \item Rings (resp\@. \(\sigma\)-rings) are closed under finite (resp\@. countable) intersections.
		      \item If \( \mathcal{R} \) is a ring (resp.\@ \(\sigma\)-ring), then \( \mathcal{R} \) is an algebra (resp.\@ \(\sigma\)-algebra) iff \( X \in \mathcal{R} \).
		      \item If \( \mathcal{R} \) is a \( \sigma \)-ring, then \( \left\{ E \subseteq X : E \in \mathcal{R} \text{ or } E^{\complement} \in \mathcal{R} \right\} \) is a \( \sigma \)-algebra.
		      \item If \( \mathcal{R} \) is a \( \sigma \)-ring, then \( \left\{ E \subseteq X: E \cap F \in \mathcal{R} \text{ for all } F \in \mathcal{R} \right\} \) is a \( \sigma \)-algebra.
	      \end{enumerate}

	      \begin{proof}
		      \begin{enumerate}[itemsep=0pt,label={\alph*.}]
			      \item Assume \( \mathcal{R} \) is a ring.

			            If \( A_{1}, A_{2} \in \mathcal{R} \) then \( A_{1} \cup A_{2}, A_{1} - A_{2}, A_{2} - A_{1}, \varnothing \in \mathcal{R} \) for \( \mathcal{R} \) is closed under finite unions and differences. Since
			            \[
				            A_{1} \cap A_{2} = A_{1} \cup A_{2} \setminus [(A_{1} \setminus A_{2}) \cup (A_{2} \setminus A_{1})]
			            \]

			            then \( A_{1} \cap A_{2} \in \mathcal{R} \). Thus \( \mathcal{R} \) is closed under finite intersections.

			            Assume \( \mathcal{R} \) is a \( \sigma \)-ring.

			            If \( A_{n} \in \mathcal{R} \) for every \( n \in \mathbb{N} \) then \( A = \bigcup_{n=1}^{\infty} A_{n} \in \mathcal{R} \). According to the De Morgan's formulae
			            \[
				            A \setminus \bigcap_{n=1}^{\infty} A_{n} = \bigcup_{n=1}^{\infty} A \setminus A_{n}
			            \]

			            is an element of \( \mathcal{R} \), so the countable intersection \( \bigcap_{n=1}^{\infty} A_{n} \) is in \( \mathcal{R} \). Thus \( \mathcal{R} \) is closed under countable intersections.
			      \item Assume \( \mathcal{R} \) is a ring (resp\@. \(\sigma\)-ring).

			            If \( \mathcal{R} \) is an algebra (resp\@. \(\sigma\)-algebra) then it is closed under complements. Since \( \varnothing \in \mathcal{R} \), it follows that \( X \in \mathcal{R} \).

			            Conversely, if \( X \in \mathcal{R} \) then \( \mathcal{R} \) is closed under complements since it contains \( X \) and is closed under differences. So \( \mathcal{R} \) is an algebra (resp\@. \(\sigma\)-algebra).
			      \item Let \( \mathcal{S} = \left\{ E \subseteq X : E \in \mathcal{R} \text{ or } E^{\complement} \in \mathcal{R} \right\} \).

			            \( \varnothing \in \mathcal{R} \) so \( \varnothing, X \in \mathcal{S} \).

			            If \( E \in \mathcal{S} \) then \( E \in \mathcal{R} \) (this implies \( E^{\complement} \in \mathcal{S} \) because \( E = {(E^{\complement})}^{\complement} \)) or \( E^{\complement} \in \mathcal{R} \) (so \( E^{\complement} \in \mathcal{S} \)). Therefore \( \mathcal{S} \) is closed under complements.

			            If \( E, F \in \mathcal{S} \), there are precisely 4 following cases.
			            \begin{enumerate}[itemsep=0pt,label={\textbf{Case \arabic*.}},itemindent=1cm]
				            \item \( E, F \in \mathcal{R} \) then \( E \cup F \in \mathcal{R} \), so \( E \cup F \in \mathcal{S} \).
				            \item \( E^{\complement}, F^{\complement} \in \mathcal{R} \) then \( {(E \cup F)}^{\complement} = E^{\complement} \cap F^{\complement} \in \mathcal{R} \), so \( E \cup F \in \mathcal{S} \).
				            \item \( E \in \mathcal{R}, F^{\complement} \in \mathcal{R} \) then \( {(E \cup F)}^{\complement} = E^{\complement} \cap F^{\complement} = (X \setminus E) \cap F^{\complement} = F^{\complement} \setminus E \in \mathcal{R} \), so \( E \cup F \in \mathcal{S} \).
				            \item \( E^{\complement} \in \mathcal{R}, F \in \mathcal{R} \) then \( {(E \cup F)}^{\complement} = E^{\complement} \cap F^{\complement} = E^{\complement} \cap (X \setminus F) = E^{\complement} \setminus F \in \mathcal{R} \), so \( E \cup F \in \mathcal{S} \).
			            \end{enumerate}

			            Let \( {\left\{ E_{n} \right\}}_{1}^{\infty} \) be a sequence of elements of \( \mathcal{S} \). Let \( W = \left\{ n \in \mathbb{N} : E_{n} \in \mathcal{R} \right\} \) then
			            \[
				            \bigcup_{n\in\mathbb{N}} E_{n} = \bigcup_{n \in W} E_{n} \cup \bigcup_{n \in \mathbb{N}\setminus W} E_{n}.
			            \]

			            Since \( \mathcal{R} \) is closed under countable unions and intersections, then
			            \[
				            \bigcup_{n \in W} E_{n} \in \mathcal{R}\qquad\text{and}\qquad {\left(\bigcup_{n \in \mathbb{N}\setminus W} E_{n}\right)}^{\complement} = \bigcap_{n\in \mathbb{N}\setminus W} E_{n}^{\complement} \in \mathcal{R}.
			            \]

			            Therefore
			            \[
				            \bigcup_{n \in W} E_{n} \in \mathcal{S} \qquad\text{and}\qquad \bigcup_{n \in \mathbb{N}\setminus W} E_{n} \in \mathcal{S}
			            \]

			            from which we conclude that \( \bigcup_{n\in\mathbb{N}} E_{n} \) according to Case 3.

			            \( \mathcal{S} \) is closed under countable unions and complements so it is a \( \sigma \)-algebra.
			      \item Let \( \mathcal{S} = \left\{ E \subseteq X: E \cap F \in \mathcal{R} \text{ for all } F \in \mathcal{R} \right\} \).

			            \( \varnothing \cap F = \varnothing \in \mathcal{R} \) for every \( F \in \mathcal{R} \) so \( \varnothing \in \mathcal{S} \).

			            \( X \cap F = F \in \mathcal{R} \) for every \( F \in \mathcal{R} \) so \( X \in \mathcal{S} \).

			            Let \( {\left\{ E_{n} \right\}}_{1}^{\infty} \) be a sequence of elements of \( \mathcal{S} \).
			            \[
				            \left(\bigcup_{n\in\mathbb{N}} E_{n}\right) \cap F = \bigcup_{n\in\mathbb{N}} (E_{n} \cap F) \in \mathcal{R}
			            \]

			            for every \( F \in \mathcal{R} \) because \( \mathcal{R} \) is a \( \sigma \)-ring and \( E_{n} \cap F \in \mathcal{R} \) for every \( n \in \mathbb{N} \). Therefore \( \bigcup_{n\in\mathbb{N}} E_{n} \in \mathcal{S} \). So \( \mathcal{S} \) is closed under countable unions.

			            If \( E \in \mathcal{S} \) then
			            \[
				            E^{\complement} \cap F = F - (E \cap F) \in \mathcal{R}
			            \]

			            for every \( F \in \mathcal{R} \), so \( \mathcal{S} \) is closed under complements.

			            Hence \( \mathcal{S} \) is a \( \sigma \)-algebra.
		      \end{enumerate}
	      \end{proof}
	\item Complete the proof of Proposition 1.2.
	      \begin{proof}
		      \( \mathcal{B}_{\mathbb{R}} \) is generated by each of the following:
		      \begin{enumerate}[itemsep=0pt]
			      \item[(b)] \( \mathcal{E}_{2} = \left\{ \left\lbrack a, b \right\rbrack : a < b \right\} \),
			      \item[(c)] \( \mathcal{E}_{3} = \left\{ \left\rbrack a, b\right\rbrack : a < b \right\} \) or \( \mathcal{E}_{4} = \left\{ \left\lbrack a, b\right\lbrack : a < b \right\} \),
			      \item[(d)] \( \mathcal{E}_{5} = \left\{ \left\rbrack a, \infty \right\lbrack: a \in \mathbb{R} \right\} \) or \( \mathcal{E}_{6} = \left\{ \left\rbrack -\infty, a \right\lbrack : a \in \mathbb{R} \right\} \),
			      \item[(e)] \( \mathcal{E}_{7} = \left\{ \left\lbrack a, \infty \right\lbrack: a \in \mathbb{R} \right\} \) or \( \mathcal{E}_{8} = \left\{ \left\rbrack -\infty, a \right\rbrack : a \in \mathbb{R} \right\} \).
		      \end{enumerate}

		      \begin{enumerate}[itemsep=0pt]
			      \item[(b)] Because \( {\left\lbrack a, b \right\rbrack}^{\complement} = \left\rbrack -\infty, a \right\lbrack \cup \left\rbrack b, \infty \right\lbrack \) then \( \left\lbrack a, b \right\rbrack \in \mathcal{B}_{\mathbb{R}} \). Therefore \( \mathcal{M}(\mathcal{E}_{2}) \subseteq \mathcal{B}_{\mathbb{R}} \).
			            \[
				            \left\rbrack a, b \right\lbrack = \bigcup_{n \in \mathbb{N}} \left\lbrack a + 1/n, b - 1/n \right\rbrack
			            \]

			            so \( \mathcal{E}_{1} \subseteq \mathcal{M}(\mathcal{E}_{2}) \), which means \( \mathcal{B}_{\mathbb{R}} = \mathcal{M}(\mathcal{E}_{1}) \subseteq \mathcal{M}(\mathcal{E}_{2}) \). Thus \( \mathcal{B}_{\mathbb{R}} = \mathcal{M}(\mathcal{E}_{2}) \).
			      \item[(c)] Because
			            \begingroup
			            \allowdisplaybreaks%
			            \begin{align*}
				            \left\rbrack a, b \right\rbrack & = \bigcup_{n \in \mathbb{N}} \left\lbrack a + 1/n, b \right\rbrack, \\
				            \left\lbrack a, b \right\lbrack & = \bigcup_{n \in \mathbb{N}} \left\lbrack a, b - 1/n \right\rbrack,
			            \end{align*}
			            \endgroup

			            then \( \mathcal{M}(\mathcal{E}_{2}) \subseteq \mathcal{M}(\mathcal{E}_{3}) \) and \( \mathcal{M}(\mathcal{E}_{2}) \subseteq \mathcal{M}(\mathcal{E}_{4}) \) and \( \mathcal{M}(\mathcal{E}_{2}) \supseteq \mathcal{M}(\mathcal{E}_{3}) \) and \( \mathcal{M}(\mathcal{E}_{2}) \supseteq \mathcal{M}(\mathcal{E}_{4}) \).

			            Thus \( \mathcal{B}_{\mathbb{R}} = \mathcal{M}(\mathcal{E}_{3}) = \mathcal{M}(\mathcal{E}_{4}) \).
			      \item[(d)] Because
			            \begingroup
			            \allowdisplaybreaks%
			            \begin{align*}
				            \left\rbrack a, \infty \right\lbrack  & = \bigcup_{n \in \mathbb{N}} \left\rbrack a, a + n \right\lbrack, \\
				            \left\rbrack -\infty, a \right\lbrack & = \bigcup_{n \in \mathbb{N}} \left\rbrack a - n, a \right\lbrack,
			            \end{align*}
			            \endgroup

			            then \( \mathcal{E}_{5} \subseteq \mathcal{M}(\mathcal{E}_{1}) \) and \( \mathcal{E}_{6} \subseteq \mathcal{M}(\mathcal{E}_{1}) \). Therefore \( \mathcal{M}(\mathcal{E}_{5}), \mathcal{M}(\mathcal{E}_{6}) \subseteq \mathcal{B}_{\mathbb{R}} \).

			            Moreover, the two formulas also mean \( \mathcal{M}(\mathcal{E}_{1}) \subseteq \mathcal{M}(\mathcal{E}_{5}) \) and \( \mathcal{M}(\mathcal{E}_{1}) \subseteq \mathcal{M}(\mathcal{E}_{6}) \).

			            Thus \( \mathcal{B}_{\mathbb{R}} = \mathcal{M}(\mathcal{E}_{5}) = \mathcal{M}(\mathcal{E}_{6}) \).
			      \item[(e)] Because
			            \begingroup
			            \allowdisplaybreaks%
			            \begin{align*}
				            \left\lbrack a, \infty \right\lbrack  & = \bigcup_{n \in \mathbb{N}} \left\lbrack a, a + n \right\rbrack, \\
				            \left\rbrack -\infty, a \right\rbrack & = \bigcup_{n \in \mathbb{N}} \left\lbrack a - n, a \right\rbrack,
			            \end{align*}
			            \endgroup

			            then \( \mathcal{E}_{7} \subseteq \mathcal{M}(\mathcal{E}_{2}) \) and \( \mathcal{E}_{8} \subseteq \mathcal{M}(\mathcal{E}_{2}) \). Therefore \( \mathcal{M}(\mathcal{E}_{7}), \mathcal{M}(\mathcal{E}_{8}) \subseteq \mathcal{B}_{\mathbb{R}} \).

			            Moreover, the two formulas also mean \( \mathcal{M}(\mathcal{E}_{2}) \subseteq \mathcal{M}(\mathcal{E}_{7}) \) and \( \mathcal{M}(\mathcal{E}_{2}) \subseteq \mathcal{M}(\mathcal{E}_{8}) \).

			            Thus \( \mathcal{B}_{\mathbb{R}} = \mathcal{M}(\mathcal{E}_{7}) = \mathcal{M}(\mathcal{E}_{8}) \).
		      \end{enumerate}
	      \end{proof}
	\item Let \( \mathcal{M} \) be an infinite \( \sigma \)-algebra.
	      \begin{enumerate}[itemsep=0pt,label={\alph*.}]
		      \item \( \mathcal{M} \) contains an infinite sequence of disjoint nonempty sets.
		      \item \( \operatorname{card}(\mathcal{M}) \ge \mathfrak{c} \).
	      \end{enumerate}
	      \begin{proof}
		      \( \mathcal{M} \) is a \( \sigma \)-algebra on some set \( X \).
		      \begin{enumerate}[itemsep=0pt,label={\alph*.}]
			      \item For every \( E \in \mathcal{M} \)
			            \[
				            \mathcal{M}_{E} = \left\{ E \cap S : S \in \mathcal{M} \right\}
			            \]

			            is a \( \sigma \)-algebra on \( E \). Moreover, \( \mathcal{M}_{E} \subseteq \mathcal{M} \).

			            Assume that there doesn't exist \( E \in \mathcal{M} \) such that \( \mathcal{M}_{E^{\complement}} \) is infinite and \( \varnothing \subsetneq E \subsetneq X \).

			            Pick \( E \in \mathcal{M} \) such that \( \varnothing \subsetneq E \subsetneq X \) then \( \mathcal{M}_{E^{\complement}} \) is finite. Also, \( \mathcal{M}_{E} \) is finite, for \( E^{\complement} \in \mathcal{M} \). Therefore \( \mathcal{M} \) is finite, which is a contradiction.

			            Hence there exists \( A_{1} \in \mathcal{M} \) such that \( \mathcal{M}_{A_{1}^{\complement}} \) is infinite.

			            For each \( n > 1 \), there exists \( A_{n} \in \mathcal{M}_{A_{n-1}^{\complement}} \) such that \( \mathcal{M}_{A_{n-1}^{\complement} \cap A_{n}^{\complement}} \) is infinite.

			            Hence \( {(A_{n})}_{1}^{\infty} \) is an infinite sequence of disjoint nonempty sets in \( \mathcal{M} \).
			      \item Let \( {(A_{n})}_{1}^{\infty} \) be an infinite sequence of disjoint nonempty sets in \( \mathcal{M} \) then \( \bigcup_{i} A_{i \in S} \) where \( S \subseteq \mathbb{N} \) are distinct sets in \( \mathcal{M} \). Because the cardinality of \( \mathcal{P}(\mathbb{N}) \) is \( \mathfrak{c} \), we conclude that \( \operatorname{card}(\mathcal{M}) \ge \mathfrak{c} \).
		      \end{enumerate}
	      \end{proof}
	\item An algebra \( \mathcal{A} \) is a \( \sigma \)-algebra iff \( \mathcal{A} \) is closed under countable increasing unions.
	      \begin{proof}
		      If the algebra \( \mathcal{A} \) is a \( \sigma \)-algebra then \( \mathcal{A} \) is closed under countable increasing unions.

		      Conversely, suppose the algebra \( \mathcal{A} \) is closed under countable increasing unions.

		      Let \( {(E_{n})}_{1}^{\infty} \) be a sequence of sets in \( \mathcal{A} \). For every \( n \in \mathbb{N} \), \( A_{n} = \bigcup_{i=1}^{n} E_{i} \in \mathcal{A} \). Moreover, \( A_{1} \subseteq A_{2} \subseteq A_{3} \subseteq \cdots \) so \( {(A_{n})}_{1}^{\infty} \) is an increasing sequence, which means \( \bigcup_{n \in \mathbb{N}} A_{n} \in \mathcal{A} \). On the other hand, \( \bigcup_{n \in \mathbb{N}} E_{n} = \bigcup_{n \in \mathbb{N}} A_{n} \), so \( \mathcal{A} \) is closed under countable unions. Thus \( \mathcal{A} \) is a \( \sigma \)-algebra.
	      \end{proof}
	\item If \( \mathcal{M} \) is the \( \sigma \)-algebra generated by \( \mathcal{E} \), then \( \mathcal{M} \) is the union of the \( \sigma \)-algebras generated by \( \mathcal{F} \) as \( \mathcal{F} \) ranges over all countable subsets of \( \mathcal{E} \). (Hint: Show that the latter object is a \( \sigma \)-algebra.)
	      \begin{proof}
		      Consider \( \bigcup_{\mathcal{F} \subseteq \mathcal{E}} \mathcal{M}(\mathcal{F}) \) where \( \mathcal{F} \) is countable. Evidently, \( \bigcup_{\mathcal{F} \subseteq \mathcal{E}} \mathcal{M}(\mathcal{F}) \subseteq \mathcal{M} \).

		      If \( E \in \bigcup_{\mathcal{F} \subseteq \mathcal{E}} \mathcal{M}(\mathcal{F}) \) then there exists \( \mathcal{F} \subseteq \mathcal{E} \) such that \( E \in \mathcal{M}(\mathcal{F}) \), which means \( E^{\complement} \in \mathcal{M}(\mathcal{F}) \subseteq \bigcup_{\mathcal{F} \subseteq \mathcal{E}} \mathcal{M}(\mathcal{F}) \). Therefore \( \bigcup_{\mathcal{F} \subseteq \mathcal{E}} \mathcal{M}(\mathcal{F}) \) is closed under complements.

		      Let \( {(E_{n})}_{1}^{\infty} \) be an infinite sequence of sets in \( \bigcup_{\mathcal{F} \subseteq \mathcal{E}} \mathcal{M}(\mathcal{F}) \). For every \( E_{n} \), there exists \( \mathcal{F}_{n} \subseteq \mathcal{E} \) such that \( E_{n} \in \mathcal{M}(\mathcal{F}_{n}) \). Since every \( \mathcal{F}_{n} \) is a countable subset of \( \mathcal{E} \), the union \( \bigcup_{n \in \mathbb{N}} \mathcal{F}_{n} \) is also a countable subset of \( \mathcal{E} \). Therefore \( E_{n} \in \mathcal{M}\left( \bigcup_{n \in \mathbb{N}} \mathcal{F}_{n} \right) \) for every \( n \in \mathcal{N} \). Hence \( \bigcup_{n \in \mathbb{N}} E_{n} \in \mathcal{M}\left( \bigcup_{n \in \mathbb{N}} \mathcal{F}_{n} \right) \subseteq \bigcup_{\mathcal{F} \subseteq \mathcal{E}} \mathcal{M}(\mathcal{F}) \). So \( \bigcup_{\mathcal{F} \subseteq \mathcal{E}} \mathcal{M}(\mathcal{F}) \) is closed under countable unions.

		      Hence \( \bigcup_{\mathcal{F} \subseteq \mathcal{E}} \mathcal{M}(\mathcal{F}) \) is a \( \sigma \)-algebra containing \( \mathcal{E} \). Since \( \mathcal{E} \) generates \( \mathcal{M} \) and \( \bigcup_{\mathcal{F} \subseteq \mathcal{E}} \mathcal{M}(\mathcal{F}) \subseteq \mathcal{M} \), we conclude that \( \bigcup_{\mathcal{F} \subseteq \mathcal{E}} \mathcal{M}(\mathcal{F}) = \mathcal{M} \), where \( \mathcal{F} \) ranges over all countable subsets of \( \mathcal{E} \).
	      \end{proof}
\end{enumerate}

\section{Measures}

\subsection*{Exercises}

\begin{enumerate}[itemsep=0pt,label=\textbf{\arabic*.}]
	\setcounter{enumi}{5}
	\item Complete the proof of Theorem 1.9.
	      \begin{proof}
		      It remains to show that \( \overline{\mu} \) is a complete measure and \( \overline{\mu} \) is the only measure on \( \overline{\mathcal{M}} \) that extends \( \mu \).

		      Because \( \overline{\mathcal{M}} = \left\{ E \cup F : E \in \mathcal{M}, F \subseteq N \text{ for some } N \in \mathcal{N} \right\} \) where \( \mathcal{N} = \left\{ N \in \mathcal{M} : \mu(N) = 0 \right\} \), it follows that \( F = \varnothing \cup F \) where \( F \subseteq N \) for some \( N \in \mathcal{N} \) is in \( \overline{\mathcal{M}} \). Hence \( \overline{\mu} \) is a complete measure as its domain includes all subsets of null sets.

		      Assume that \( \mu^{\ast} \) is a measure on \( \overline{\mathcal{M}} \) that extends \( \mu \). For every \( E \cup F \in \overline{\mathcal{M}} \), where \( E \in \mathcal{M} \) and \( F \subseteq N \) for some \( N \in \mathcal{N} \)
		      \begingroup
		      \allowdisplaybreaks%
		      \begin{align*}
			      \overline{\mu}(E \cup F) & = \mu(E) = \mu^{\ast}(E) \le \mu^{\ast}(E \cup F) \le \mu^{\ast}(E \cup N) \le \mu^{\ast}(E) + \mu^{\ast}(N) \\
			                               & = \mu(E) + \mu(N) = \mu(E) = \overline{\mu}(E \cup F).
		      \end{align*}
		      \endgroup

		      So \( \mu^{\ast} = \overline{\mu} \), hence the result follows.
	      \end{proof}
	\item If \( \mu_{1}, \ldots, \mu_{n} \) are measures on \( (X, \mathcal{M}) \) and \( a_{1}, \ldots, a_{n} \in \left\lbrack 0, \infty \right\lbrack \), then \( \sum_{1}^{n} a_{j}\mu_{j} \) is a measure on \( (X, \mathcal{M}) \).
	      \begin{proof}
		      Let \( \mu = \sum_{1}^{n} a_{j}\mu_{j} \)

		      If \( {\left\{E_{j}\right\}}_{1}^{\infty} \) are pairwise disjoint measurable sets then
		      \begingroup
		      \allowdisplaybreaks%
		      \begin{align*}
			      \mu(\varnothing)                              & = \sum_{1}^{n} a_{j}\mu_{j}(\varnothing) = \sum_{1}^{n} 0 = 0,                                                                                          \\
			      \mu \left(\bigcup_{k=1}^{\infty} E_{k}\right) & = \sum_{j=1}^{n} a_{j}\mu_{j}\left( \sum_{k=1}^{\infty} E_{k} \right) = \sum_{j=1}^{n} a_{j}\sum_{k=1}^{\infty}\mu_{j}(E_{k})                           \\
			                                                    & = \sum_{j=1}^{n}\sum_{k=1}^{\infty} a_{j}\mu_{j}(E_{k}) = \sum_{j=1}^{n} \lim\limits_{m \to \infty} \sum_{k=1}^{m} a_{j}\mu_{j}(E_{k})                  \\
			                                                    & = \lim\limits_{m\to\infty} \sum_{j=1}^{n}\sum_{k=1}^{m} a_{j}\mu_{j}(E_{k}) = \lim\limits_{m\to\infty} \sum_{k=1}^{m}\sum_{j=1}^{n} a_{j}\mu_{j}(E_{k}) \\
			                                                    & = \lim\limits_{m\to\infty} \sum_{k=1}^{m} \mu(E_{k}) = \sum_{k=1}^{\infty} \mu(E_{k}).
		      \end{align*}
		      \endgroup

		      Thus \( \mu \) is a measure on \( (X, \mathcal{M}) \).
	      \end{proof}
	\item If \( (X, \mathcal{M}, \mu) \) is a measure space and \( {\left\{ E_{j} \right\}}_{1}^{\infty} \subseteq \mathcal{M} \), then \( \mu(\liminf E_{j}) \le \liminf \mu(E_{j}) \). Also, \( \mu(\limsup E_{j}) \ge \limsup \mu(E_{j}) \) provided that \( \mu\left( \bigcup_{1}^{\infty} E_{j} \right) < \infty \).
	      \begin{proof}
		      \begingroup
		      \allowdisplaybreaks%
		      \begin{align*}
			      \mu(\liminf E_{j}) & = \mu\left( \bigcup_{j=1}^{\infty} \bigcap_{k\ge j} E_{k} \right) = \lim\limits_{j\to\infty} \mu\left(\bigcap_{k\ge j} E_{k}\right) & \text{continuity from below}                                               \\
			                         & \le \lim\limits_{j\to \infty} \inf\limits_{k\ge j} \mu(E_{k}) = \liminf \mu(E_{j})                                                  & \mu\left(\bigcap_{k\ge j} E_{k}\right) \le \inf\limits_{k\ge j} \mu(E_{k})
		      \end{align*}
		      \endgroup

		      \begingroup
		      \allowdisplaybreaks%
		      \begin{align*}
			      \mu(\limsup E_{j}) & = \mu\left( \bigcap_{j=1}^{\infty} \bigcup_{k \ge j} E_{k} \right) = \lim\limits_{j\to\infty} \mu\left(\bigcup_{k\ge j} E_{k}\right) & \text{continuity from above}                                               \\
			                         & \ge \lim\limits_{j\to\infty} \sup\limits_{k\ge j} \mu(E_{k}) = \limsup \mu(E_{j})                                                    & \mu\left(\bigcup_{k\ge j} E_{k}\right) \ge \sup\limits_{k\ge j} \mu(E_{k})
		      \end{align*}
		      \endgroup
	      \end{proof}
	\item If \( (X, \mathcal{M}, \mu) \) is a measure space and \( E, F \in \mathcal{M} \), then \( \mu(E) + \mu(F) = \mu(E \cup F) + \mu(E \cap F) \).
	      \begin{proof}
		      \( F \) is the disjoint union of \( F \setminus E \) and \( E \cap F \) so \( \mu(F) = \mu(F \setminus E) + \mu(E \cap F) \).

		      \( E \cup F \) is the disjoint union of \( E \) and \( F \setminus E \) so \( \mu(E \cup F) = \mu(E) + \mu(F \setminus E) \).

		      Hence
		      \[ \mu(E) + \mu(F) = \mu(E) + \mu(F \setminus E) + \mu(E \cap F) = \mu(E \cup F) + \mu(E \cap F). \qedhere \]
	      \end{proof}
	\item Given a measure space \( (X, \mathcal{M}, \mu) \) and \( E \in \mathcal{M} \), define \( \mu_{E}(A) = \mu(A \cap E) \) for \( A \in \mathcal{M} \). Then \( \mu_{E} \) is a measure.
	      \begin{proof}
		      From the definition of \( \mu_{E} \), \( \mu_{E}(\varnothing) = \mu(\varnothing \cap E) = \mu(\varnothing) = 0 \).

		      Let \( { \left\{ A_{n} \right\} }_{1}^{\infty} \) be an infinite sequence of pairwise disjoint measurable sets. Then \( { \left\{ E \cap A_{n} \right\} }_{1}^{\infty} \) is also n infinite sequence of pairwise disjoint measurable sets.
		      \begingroup
		      \allowdisplaybreaks%
		      \begin{align*}
			      \mu_{E}\left(\bigcup_{1}^{\infty} A_{n}\right) & = \mu\left( E \cap \bigcup_{1}^{\infty} A_{n} \right) = \mu\left(\bigcup_{1}^{\infty} (E \cap A_{n})\right) \\
			                                                     & = \sum_{1}^{\infty} \mu(E \cap A_{n}) = \sum_{1}^{\infty} \mu_{E}(A_{n}).
		      \end{align*}
		      \endgroup

		      Thus \( \mu_{E} \) is also a measure on the measurable space \( (X, \mathcal{M}) \).
	      \end{proof}
	\item A finitely additive measure \( \mu \) is a measure iff it is continuous from below as in Theorem 1.8c. If \( \mu(X) < \infty \), \( \mu \) is a measure iff it is continuous from above as in Theorem 1.8d.
	      \begin{proof}
		      If the finitely additive measure \( \mu \) is a measure then it is continuous from below.

		      Conversely, assume the finitely additive measure \( \mu \) is continuous from below.

		      Let \( {\left\{ E_{n} \right\}}_{1}^{\infty} \) be an infinite sequence of pairwise disjoint measurable sets in the domain of \( \mu \).
		      \begingroup
		      \allowdisplaybreaks%
		      \begin{align*}
			      \mu\left(\bigcup_{1}^{\infty} E_{j}\right) & = \lim\limits_{n\to \infty} \mu\left(\bigcup_{1}^{n} E_{j}\right) & \text{continuity from below} \\
			                                                 & = \lim\limits_{n\to\infty} \sum_{1}^{n} \mu(E_{j})                & \text{finite additivity}     \\
			                                                 & = \sum_{1}^{\infty} \mu(E_{j}).
		      \end{align*}
		      \endgroup

		      Hence \( \mu \) is a measure.

		      \hrulefill%

		      If the finitely additive measure \( \mu \) is a measure then it is continuous from above.

		      Conversely, assume the finitely additive measure \( \mu \) is continuous from above.

		      Because \( \mu(X) \) is finite, the measure of any measurable set in the domain of \( \mu \) is finite.

		      Let \( {\left\{ E_{n} \right\}}_{1}^{\infty} \) be an infinite sequence of pairwise disjoint measurable sets in the domain of \( \mu \).
		      \begingroup
		      \allowdisplaybreaks%
		      \begin{align*}
			      \mu\left(X \setminus \bigcup_{1}^{\infty} E_{j}\right) & = \mu\left(\bigcap_{1}^{\infty} X \setminus E_{j}\right) = \lim\limits_{n\to\infty} \mu\left(\bigcap_{j=1}^{n} X\setminus E_{j}\right)                         & \text{continuity from above} \\
			                                                             & = \lim\limits_{n\to\infty} \mu\left( X \setminus \bigcup_{j=1}^{n} E_{j} \right) = \mu(X) - \lim\limits_{n\to\infty} \mu\left( \bigcup_{j=1}^{n} E_{j} \right)                                \\
			                                                             & = \mu(X) - \lim\limits_{n\to\infty} \sum_{j=1}^{n} \mu(E_{j})                                                                                                  & finite additivity            \\
			                                                             & = \mu(X) - \sum_{1}^{\infty} \mu(E_{j})
		      \end{align*}
		      \endgroup

		      so \( \mu\left(\bigcup_{1}^{\infty} E_{j}\right) = \sum_{1}^{\infty} \mu(E_{j}) \). Hence \( \mu \) is a measure.
	      \end{proof}
	\item Let \( (X, \mathcal{M}, \mu) \) be a finite measure space.
	      \begin{enumerate}[itemsep=0pt,label={\textbf{\alph*.}}]
		      \item If \( E, F \in \mathcal{M} \) and \( \mu(E \Delta F) = 0 \), then \( \mu(E) = \mu(F) \).
		      \item Say that \( E \sim F \) if \( \mu(E \Delta F) = 0 \); then \( \sim \) is an equivalence relation on \( \mathcal{M} \).
		      \item For \( E, F \in \mathcal{M} \), define \( \rho(E, F) = \mu(E \Delta F) \). Then \( \rho(E, G) \le \rho(E, F) + \rho(F, G) \), and hence \( \rho \) defines a metric on the space \( \mathcal{M}/{\sim} \) of equivalence classes.
	      \end{enumerate}
	      \begin{proof}
		      The finiteness is implicitly used in \textbf{c.}
		      \begin{enumerate}[itemsep=0pt,label={\textbf{\alph*.}}]
			      \item \( E \Delta F \) is the disjoint union of \( E \setminus F \) and \( F \setminus E \) so \( \mu(E \Delta F) = \mu(E \setminus F) + \mu(F \setminus E) \).

			            If \( \mu(E \Delta F) = 0 \), then \( \mu(E \setminus F) = \mu(F \setminus E) = 0 \) and
			            \[
				            \mu(E) = \mu(E \cap F) + \mu(E \setminus F) = \mu(E \cap F) + \mu(F \setminus E) = \mu(F).
			            \]
			      \item \( \mu(E \Delta E) = \mu(\varnothing) = 0 \) so \( E \sim E \) for all \( E \in \mathcal{M} \).

			            \( \mu(E \Delta F) = 0 \iff \mu(F \Delta E) = 0 \) because \( E \Delta F = F \Delta E \). Therefore \( \sim \) is symmetric.

			            If \( \mu(E \Delta F) = 0 \) and \( \mu(F \Delta G) = 0 \) then
			            \begingroup
			            \allowdisplaybreaks%
			            \begin{align*}
				            \mu(E \Delta G) & = \mu(E \setminus G) + \mu(G \setminus E)                                             \\
				                            & \le \mu(E \setminus F) + \mu(F \setminus G) + \mu(G \setminus F) + \mu(F \setminus E) \\
				                            & = \mu(E \Delta F) + \mu(F \Delta G) = 0
			            \end{align*}
			            \endgroup

			            which means \( E \sim G \), so \( \sim \) is transitive.

			            Hence \( \sim \) is an equivalence relation on \( \mathcal{M} \).
			      \item \( E \setminus G \subseteq (E \setminus F) \cup (F \setminus G) \) and \( G \setminus E \subseteq (G \setminus F) \cup (F \setminus E) \).
			            \begingroup
			            \allowdisplaybreaks%
			            \begin{align*}
				            \rho(E, G) & = \mu(E \Delta G) = \mu(E \setminus G) + \mu(G \setminus E)                               \\
				                       & \le \mu((E \setminus F) \cup (F \setminus G)) + \mu((G \setminus F) \cup (F \setminus E)) \\
				                       & = \mu(E \setminus F) + \mu(F \setminus G) + \mu(G \setminus F) + \mu(F \setminus E)       \\
				                       & = \mu(E \Delta F) + \mu(F \Delta G)                                                       \\
				                       & = \rho(E, F) + \rho(F, G).
			            \end{align*}
			            \endgroup
		      \end{enumerate}
	      \end{proof}
	\item Every \( \sigma \)-finite measure is semifinite.
	      \begin{proof}
		      Let \( (X, \mathcal{M}, \mu) \) be a measure space where \( \mu \) is \( \sigma \)-finite.

		      Because \( \mu \) is \( \sigma \)-finite, there exists an infinite sequence \( {\left\{ E_{j} \right\}}_{1}^{\infty} \) of measurable sets in \( \mathcal{M} \) such that \( \mu(E_{j}) \) is finite for all \( j \) and \( X = \bigcup_{1}^{\infty} E_{j} \).

		      If \( E \in \mathcal{M} \) and \( \mu(E) = \infty \). If \( \mu(E \cap E_{j}) = 0 \) for all \( j \) then \( \mu(E) = \mu\left( E \cap \bigcup_{1}^{\infty} E_{j} \right) = \mu\left( \bigcup_{1}^{\infty} (E \cap E_{j}) \right) \le \sum_{1}^{\infty} \mu(E \cap E_{j}) = 0 \), which is a contradiction. Hence there exists \( j \in \mathbb{N} \) such that \( 0 < \mu(E \cap E_{j}) \). Because \( \mu(E \cap E_{j}) \le \mu(E_{j}) \) and \( \mu(E_{j}) \) is finite, it follows that \( \mu(E \cap E_{j}) \) is finite. Choose \( F = E \cap E_{j} \) then \( F \subseteq E \) and \( 0 < \mu(F) < \infty \).

		      Thus \( \mu \) is semifinite.
	      \end{proof}
	\item If \( \mu \) is a semifinite measure and \( \mu(E) = \infty \), for any \( C > 0 \), there exists \( F \subseteq E \) with \( C < \mu(F) < \infty \).
	      \begin{proof}
		      Let \( \mathcal{A} = \left\{ F : F \subseteq E, 0 < \mu(F) < \infty \right\} \). Since \( \mu \) is a semifinite measure, the set \( A \) is nonempty. Let \( s = \sup \left\{ \mu(F) : F \in \mathcal{A} \right\} \).

		      For \( s \) being the supremum of \( A \), there exists a sequence \( {\left\{ E_{n} \right\}}_{1}^{\infty} \) of measurable sets \( E_{n} \in \mathcal{A} \) such that \( \lim\limits_{n\to\infty} \mu(E_{n}) = s \).

		      For every \( n \in \mathbb{N} \), let \( G_{n} = \bigcup_{j=1}^{n} E_{j} \) then \( \mu(G_{n}) \le \sum_{j=1}^{n} \mu(E_{j}) \) is finite, so \( G_{n} \in \mathcal{A} \). The infinite sequence \( {\left\{ F_{n} \right\}}_{1}^{\infty} \) is nondecreasing. Let \( G = \bigcup_{j=1}^{\infty} G_{j} \).

		      From the ``continuity from below'' property
		      \[
			      \mu(G) = \lim\limits_{n \to \infty} \mu(G_{n}) \ge \lim\limits_{n\to\infty} \mu(E_{n}) = s.
		      \]

		      Moreover, \( \mu(G_{n}) \le s \) for every \( n \in \mathbb{N} \) because \( G_{n} \in \mathcal{A} \) so \( \lim\limits_{n \to \infty} \mu(G_{n}) \le s \). Therefore \( \mu(F) = s \).

		      Assume \( s \) is finite, then \( \mu(E \setminus G) = \infty \). Due to semifiniteness of \( \mu \), there exists \( G^{\prime} \subseteq E \setminus G \) such that \( 0 < \mu(G^{\prime}) < \infty \), which implies \( \mu(G \cup G^{\prime}) = \mu(G) + \mu(G^{\prime}) \) is finite and exceeds \( s \), which is a contradiction to \( s \) being the supremum. Hence \( s = \infty \).

		      From the definition of a supremum, for any \( C > 0 \), there exists \( F \subseteq E \) such that \( C < \mu(F) < \infty \).
	      \end{proof}
	\item Given a measure \( \mu \) on \( (X, \mathcal{M}) \), define \( \mu_{0} \) on \( \mathcal{M} \) by \( \mu_{0}(E) = \sup\left\{ \mu(F): F \subseteq E \text{ and } \mu(F) < \infty \right\} \).
	      \begin{enumerate}[itemsep=0pt,label=\textbf{\alph*.}]
		      \item \( \mu_{0} \) is a semifinite measure. It is called the \textbf{semifinite part} of \( \mu \).
		      \item If \( \mu \) is semifinite, then \( \mu = \mu_{0} \). (Use Exercise 14.)
		      \item There is a measure \( \nu \) on \( \mathcal{M} \) (in general, not unique) which assumes only the values \( 0 \) and \( \infty \) such that \( \mu = \mu_{0} + \nu \).
	      \end{enumerate}
	      \begin{proof}
		      See \href{https://math.stackexchange.com/questions/1788830/real-analysis-folland-problem-1-3-15-measures}{Ramiro's answer}.

		      \begin{enumerate}[itemsep=0pt,label=\textbf{\alph*.}]
			      \item From the definition of \( \mu_{0} \), \( \mu_{0}(E) = \mu(E) \) if \( \mu(E) \) is finite. Therefore \( \mu_{0}(\varnothing) = 0 \).

			            Let \( {\left\{ E_{j} \right\}}_{1}^{\infty} \) be an infinite sequence of pairwise disjoint sets in \( \mathcal{M} \).

			            Let \( F \subseteq \bigcup_{1}^{\infty} E_{j} \) such that \( \mu(F) < \infty \) then \( \mu(F \cap E_{j}) < \infty \) for every \( j \in \mathbb{N} \).
			            \[
				            \mu(F) = \mu\left(\sum_{1}^{\infty} F \cap E_{j} \right) = \sum_{1}^{\infty} \mu(F \cap E_{j}) \le \sum_{1}^{\infty} \mu_{0}(E_{j})
			            \]

			            so
			            \[
				            \mu_{0}\left( \bigcup_{1}^{\infty} E_{j} \right) \le \sum_{1}^{\infty} \mu_{0}(E_{j}).
			            \]

			            If \( \mu_{0}\left( \bigcup_{1}^{\infty} E_{j} \right) = \infty \)  then \( \mu_{0}\left( \bigcup_{1}^{\infty} E_{j} \right) = \sum_{1}^{\infty} \mu_{0}(E_{j}) \).

			            If \( \mu_{0}\left( \bigcup_{1}^{\infty} E_{j} \right) < \infty \), then for every subset of finite measure of \( E_{j} \) is also a subset of finite measure of \( \bigcup_{1}^{\infty} E_{j} \). Therefore \( \mu_{0}(E_{n}) \le \mu_{0}\left( \bigcup_{1}^{\infty} E_{j} \right) \) for every \( n \in \mathbb{N} \), hence \( \mu_{0}(E_{n}) < \infty \) for every \( n \in \mathbb{N} \).

			            Let \( \varepsilon > 0 \).

			            For every \( j \), let \( F_{j} \subseteq E_{j} \) such that \( \mu_{0}(E_{j}) - \dfrac{\varepsilon}{2^{j}} \le \mu(F_{j}) \) (possible due to a property of supremum), then
			            \[
				            \sum_{1}^{n} \mu_{0}(E_{j}) \le \varepsilon + \sum_{1}^{n} \mu(F_{j}) = \varepsilon + \mu\left(\bigcup_{1}^{n} F_{j}\right) = \varepsilon + \mu\left(\bigcup_{1}^{n} F_{j}\right) \le \varepsilon + \mu_{0}\left(\bigcup_{1}^{\infty} E_{j}\right).
			            \]

			            Therefore
			            \[
				            \sum_{1}^{\infty} \mu_{0}(E_{j}) \le \varepsilon + \mu_{0}\left(\bigcup_{1}^{\infty} E_{j}\right)
			            \]

			            which means
			            \[
				            \sum_{1}^{\infty} \mu_{0}(E_{j}) \le \mu_{0}\left(\bigcup_{1}^{\infty} E_{j}\right).
			            \]

			            Hence \( \sum_{1}^{\infty} \mu_{0}(E_{j}) = \mu_{0}\left(\bigcup_{1}^{\infty} E_{j}\right) \), so \( \mu_{0} \) is a measure on \( (X, \mathcal{M}) \).

			            Suppose \( E \in \mathcal{M} \) and \( \mu_{0}(E) = \infty \) then \( \sup\left\{ \mu(F) : F \subseteq E \text{ and } \mu(F) < \infty \right\} = \infty \). So there is a subset \( F \subseteq E \) such that \( 0 < \mu(F) < \infty \). By the definition of \( \mu_{0} \), \( 0 < \mu_{0}(F) = \mu(F) < \infty \). Therefore \( \mu_{0} \) is semifinite.
			      \item Suppose \( \mu \) is semifinite.

			            If \( \mu(E) = \infty \), for every \( C > 0 \), there is \( F \subseteq E \) such that \( C < \mu(F) < \infty \) (Exercise 14). Hence \( \sup\left\{ \mu(F) : F \subseteq E, 0 < \mu(F) < \infty \right\} = \infty \), which means \( \mu_{0}(E) = \infty = \mu(E) \).

			            If \( \mu(E) < \infty \) then \( \mu_{0}(E) = \mu(E) \). Therefore \( \mu_{0}(E) = \mu(E) \) for every \( E \in \mathcal{M} \).

			            Hence \( \mu = \mu_{0} \).
			      \item Let \( E \in \mathcal{M} \). From the definition of \( \mu_{0} \), \( \mu_{0}(E) \le \mu(E) \). Moreover, if \( \mu(E) \) is finite then \( \mu_{0}(E) = \mu(E) \).

			            We define a function \( \nu \) on \( \mathcal{M} \) as follows:
			            \[
				            \nu(E) = \begin{cases}
					            0      & \text{\( E \) is \( \sigma \)-finite with respect to \( \mu \)}, \\
					            \infty & \text{otherwise.}
				            \end{cases}
			            \]

			            From this definition, \( \nu(\varnothing) = 0 \) and \( \nu(E) \ge 0 \) for every \( E \in \mathcal{M} \).

			            Let \( {\left\{ E_{j} \right\}}_{1}^{\infty} \) be an infinite sequence of pairwise disjoint measurable sets.

			            If \( \bigcup_{1}^{\infty} E_{j} \) is \( \sigma \)-finite then \( E_{j} \) is \( \sigma \)-finite for every \( j \in \mathbb{N} \), so
			            \[
				            \nu\left(\bigcup_{1}^{\infty} E_{j}\right) = 0 = \sum_{1}^{\infty} \nu(E_{j}).
			            \]

			            If \( \bigcup_{1}^{\infty} E_{j} \) is not \( \sigma \)-finite then there exists \( j \in \mathbb{N} \) such that \( E_{j} \) is not \( \sigma \)-finite (otherwise, \( \bigcup_{1}^{\infty} E_{j} \) is \(\sigma\)-finite), hence
			            \[
				            \nu\left(\bigcup_{1}^{\infty} E_{j}\right) = \infty = \sum_{1}^{\infty} \nu(E_{j}).
			            \]

			            Therefore \( \nu \) is a measure on \( (X, \mathcal{M}) \).

			            If \( E \in \mathcal{M} \) is \( \sigma \)-finite then \( E \) is decomposed into the union of countably many disjoint measurable sets \( E_{j} \) of finite  \( \mu \)-measure.
			            \[
				            \mu(E) = \mu\left( \bigcup_{1}^{\infty}E_{j} \right) = \sum_{1}^{\infty} \mu(E_{j}) = \sum_{1}^{\infty} \mu_{0}(E_{j}) = \mu_{0}\left(\bigcup_{1}^{\infty} E_{j}\right) = \mu_{0}(E) = \mu_{0}(E) + \nu(E).
			            \]

			            If \( E \in \mathcal{M} \) is not \( \sigma \)-finite then \( \mu(E) = \infty \). Therefore \( \mu(E) = \mu_{0}(E) + \nu(E) \).

			            Thus \( \mu = \mu_{0} + \nu \).
		      \end{enumerate}
	      \end{proof}
	\item Let \( (X, \mathcal{M}, \mu) \) be a measure space. A set \( E \subseteq X \) is called \textbf{locally measurable} if \( E \cap A \in \mathcal{M} \) for all \( A \in \mathcal{M} \) such that \( \mu(A) < \infty \). Let \( \widetilde{\mathcal{M}} \) be the collection of all locally measurable sets. Clearly \( \mathcal{M} \subseteq \widetilde{\mathcal{M}} \); if \( \mathcal{M} = \widetilde{\mathcal{M}} \), then \( \mu \) is called \textbf{saturated}.
	      \begin{enumerate}[itemsep=0pt,label=\textbf{\alph*.}]
		      \item If \( \mu \) is \( \sigma \)-finite, then \( \mu \) is saturated.
		      \item \( \widetilde{\mathcal{M}} \) is a \( \sigma \)-algebra.
		      \item Define \( \widetilde{\mu} \) on \( \widetilde{\mathcal{M}} \) by \( \widetilde{\mu}(E) = \mu(E) \) if \( E \in \mathcal{M} \) and \( \widetilde{\mu}(E) = \infty \) otherwise. Then \( \widetilde{\mu} \) is a saturated measure on \( \widetilde{\mathcal{M}} \), called the \textbf{saturation} of \( \mu \).
		      \item If \( \mu \) is complete, so is \( \widetilde{\mu} \).
		      \item Suppose that \( \mu \) is semifinite. For \( E \in \widetilde{\mathcal{M}} \), define \( \underline{\mu}(E) = \sup\left\{ \mu(A) : A \in \mathcal{M}, A \subseteq E \right\} \). Then \( \underline{\mu} \) is a saturated measure on \( \widetilde{\mathcal{M}} \) that extends \( \mu \).
		      \item Let \( X_{1}, X_{2} \) be disjoint uncountable sets, \( X = X_{1} \cup X_{2} \), and \( \mathcal{M} \) the \( \sigma \)-algebra of countable or co-countable sets in \( X \). Let \( \mu_{0} \) be counting measure on \( \mathcal{P}(X_{1}) \) and define \( \mu \) on \( \mathcal{M} \) by \( \mu(E) = \mu_{0}(E \cap X_{1}) \). Then \( \mu \) is a measure on \( \mathcal{M} \), \( \widetilde{\mathcal{M}} = \mathcal{P}(X) \), and in the notation of parts (c) and (e), \( \widetilde{\mu} \ne \underline{\mu} \).
	      \end{enumerate}
	      \begin{proof}
		      \begin{enumerate}[itemsep=0pt,label=\textbf{\alph*.}]
			      \item Every measurable set is locally measurable so \( \mathcal{M} \subseteq \widetilde{\mathcal{M}} \).

			            Suppose \( \mu \) is \( \sigma \)-finite and \( E \) is a locally measurable set.

			            Because \( \mu \) is \( \sigma \)-finite, there exists an infinite sequence \( {\left\{ E_{j} \right\}}_{1}^{\infty} \) of finite measure measurable sets with \( X = \bigcup_{1}^{\infty} E_{j} \). Since \( E \) is locally measurable, \( E \cap E_{j} \in \mathcal{M} \) for every \( j \), so
			            \[
				            E = E \cap X = E \cap \bigcup_{1}^{\infty} E_{j} = \bigcup_{1}^{\infty} E \cap E_{j}
			            \]

			            is measurable. Hence \( \mathcal{M} = \widetilde{\mathcal{M}} \), which means \( \mu \) is saturated.
			      \item Suppose \( E \) is locally measurable.

			            Let \( A \in \mathcal{M} \) with \( \mu(A) < \infty \). From the definition of locally measurability, \( E \cap A \in \mathcal{M} \). Therefore \( E^{\complement} \cap A = A \setminus (E \cap A) \in \mathcal{M} \). Hence \( E^{\complement} \) is also locally measurable.

			            Suppose \( {\left\{ E_{j} \right\}}_{1}^{\infty} \) is an infinite sequence of locally measurable sets.

			            Let \( A \in \mathcal{M} \) with \( \mu(A) < \infty \). From the definition of locally measurability, \( E_{j} \cap A \in \mathcal{M} \) for every \( j \in \mathbb{N} \). Therefore \( \left(\bigcup_{1}^{\infty} E_{j}\right) \cap A = \bigcup_{1}^{\infty} (E_{j} \cap A) \in \mathcal{M} \) for every \( A \) of finite measure. Hence \( \bigcup_{1}^{\infty} E_{j} \) is locally measurable.

			            \( \widetilde{\mathcal{M}} \) is closed under complements and countably unions so it is a \( \sigma \)-algebra.
			      \item Suppose \( E \subseteq X \) is locally measurable with respect to \( \widetilde{\mu} \).

			            Let \( A \) be a measurable set with respect to \( \mu \) with \( \mu(A) < \infty \), then \( A \) is locally measurable with respect to \( \mu \) and \( \widetilde{\mu} \). From the definition of \( \widetilde{\mu} \), \( \widetilde{\mu}(A) = \mu(A) \).

			            From the local measurability of \( E \) with respect to \( \widetilde{\mu} \), \( E \cap A \in \widetilde{\mathcal{M}} \). Because \( E \cap A \subseteq A \), \( \widetilde{\mu}(E \cap A) \le \widetilde{\mu}(A) \). Besides, \( \widetilde{\mu}(A) < \infty \) so \( \widetilde{\mu}(E \cap A) < \infty \). From the defintion of \( \widetilde{\mu} \), it is not the case that \( E \cap A \notin \mathcal{M} \) because if so, then \( \widetilde{\mu}(E \cap A) = \infty \). Hence \( E \cap A \in \mathcal{M} \), which means \( E \in \widetilde{\mathcal{M}} \).

			            Hence every locally measurable set with respect to \( \widetilde{\mathcal{M}} \) is necessarily locally measurable with respect to \( \mathcal{M} \). This means \( \widetilde{\widetilde{M}} = \widetilde{M} \) and \( \widetilde{\mu} \) is saturated.
			      \item Suppose \( \mu \) is a complete measure.

			            Let \( N \in \widetilde{\mathcal{M}} \) such that \( \widetilde{\mu}(N) = 0 \). From the definition of \( \widetilde{\mu} \), \( N \in \mathcal{M} \) (otherwise, \( \widetilde{\mu}(N) = \infty \)) and \( \mu(N) = \widetilde{\mu}(N) = 0 \).

			            \( \mu \) is a complete measure so every subset of \( N \) is a null set with respect to \( \mu \). Again, from the definition of \( \widetilde{\mu} \), every subset of \( N \) is a null set with respect to \( \widetilde{\mu} \). Hence \( \widetilde{\mu} \) is a complete measure.
			      \item Firstly, we prove that \( \underline{\mu} \) is a measure on \( (X, \widetilde{\mathcal{M}}) \).

			            From the definition of \( \underline{\mu} \)
			            \[
				            \underline{\mu}(E) = \sup\left\{ \mu(A) : A \in \mathcal{M}, A \subseteq E \right\} \ge 0
			            \]

			            because \( \varnothing \in \mathcal{M} \) and \( \varnothing \subseteq E \). Moreover, \( \underline{\mu}(\varnothing) = 0 \).

			            If \( E, E^{\prime} \in \widetilde{\mathcal{M}} \) and \( E \subseteq E^{\prime} \) then
			            \[
				            \left\{ A : A \in \mathcal{M}, A \subseteq E \right\} \subseteq \left\{ A : A \in \mathcal{M}, A \subseteq E^{\prime} \right\}
			            \]

			            which implies \( \underline{\mu}(E) \le \underline{\mu}(E^{\prime}) \).

			            Let \( {\left\{ E_{j} \right\}}_{1}^{\infty} \) be an infinite sequence of pairwise disjoint measurable sets in \( \widetilde{\mathcal{M}} \).

			            If \( \underline{\mu}\left( \bigcup_{1}^{\infty} E_{j} \right) = \infty \) then for every \( C > 0 \), there exists \( A \in \mathcal{M} \) such that \( \mu(A) > C \) and \( A \subseteq \bigcup_{1}^{\infty} E_{j} \).

			            If \( \mu(A) < \infty \), \( E_{j} \cap A \in \mathcal{M} \) and
			            \[
				            C < \mu(A) = \mu\left( A \cap \bigcup_{1}^{\infty} E_{j} \right) = \sum_{1}^{\infty} \mu(A \cap E_{j}) \le \sum_{1}^{\infty} \underline{\mu}(E_{j})
			            \]

			            which means \( \sum_{1}^{\infty} \underline{\mu}(E_{j}) = \infty \).

			            Otherwise, \( \mu(A) = \infty \) then there exists \( F \subseteq A \) with \( F \in \mathcal{M} \) and \( 0 < \mu(F) < \infty \), because \( \mu \) is semifinite. Because \( E_{j} \) is locally measurable with respect to \( \mu \), \( F \cap E_{j} \in \mathcal{M} \) and
			            \[
				            \mu(F) = \mu\left( F \cap \bigcup_{1}^{\infty} E_{j} \right) = \sum_{1}^{\infty} \mu(F \cap E_{j}) \le \sum_{1}^{\infty} \underline{\mu}(E_{j})
			            \]


			            According to Exercise 14, we can choose \( F \subseteq A \) such that \( \mu(F) \) is arbitrarily large, so \( \sum_{1}^{\infty} \underline{\mu}(E_{j}) = \infty \).

			            Hence \( \sum_{1}^{\infty} \underline{\mu}(E_{j}) = \infty = \underline{\mu}\left( \bigcup_{1}^{\infty} E_{j} \right) \).

			            \bigskip

			            If \( \underline{\mu}\left( \bigcup_{1}^{\infty} E_{j} \right) < \infty \) then \( \underline{\mu}(E_{j}) < \infty \) for every \( j \in \mathbb{N} \).

			            Let \( A \subseteq \bigcup_{1}^{\infty} E_{j} \) with \( A \in \mathcal{M} \) then \( \mu(A) < \infty \) and \( E_{j} \cap A \in \mathcal{M} \) (because \( E_{j} \) is locally measurable with respect to \( \mu \)).
			            \[
				            \mu(A) = \mu\left( A \cap \bigcup_{1}^{\infty} E_{j} \right) = \sum_{1}^{\infty} \mu(A \cap E_{j}) \le \sum_{1}^{\infty} \underline{\mu}(E_{j}).
			            \]

			            By passage to supremum, we get
			            \[
				            \underline{\mu}\left( \bigcup_{1}^{\infty} E_{j} \right) \le \sum_{1}^{\infty} \underline{\mu}(E_{j}).
			            \]

			            Pick an arbitrary \( \varepsilon > 0 \). For every \( j \in \mathbb{N} \), there exists \( F_{j} \subseteq E_{j} \) with \( F_{j} \in \mathcal{M} \) such that
			            \[
				            \underline{\mu}(E_{j}) - \varepsilon/2^{j} < \mu(F_{j}).
			            \]

			            Therefore
			            \[
				            \sum_{1}^{n} \underline{\mu}(E_{j}) < \sum_{1}^{n}\varepsilon/2^{j} + \sum_{1}^{n} \mu(F_{j}) < \varepsilon + \mu\left( \bigcup_{1}^{n} F_{j} \right) \le \varepsilon + \underline{\mu}\left(\bigcup_{1}^{\infty} E_{j}\right)
			            \]

			            for every \( n \in \mathbb{N} \), so
			            \[
				            \sum_{1}^{\infty} \underline{\mu}(E_{j}) \le \varepsilon + \underline{\mu}\left( \bigcup_{1}^{\infty} E_{j} \right)
			            \]

			            for every \( \varepsilon > 0 \), which means
			            \[
				            \sum_{1}^{\infty} \underline{\mu}(E_{j}) \le \underline{\mu}\left( \bigcup_{1}^{\infty} E_{j} \right).
			            \]

			            Hence \( \sum_{1}^{\infty} \underline{\mu}(E_{j}) = \underline{\mu}\left( \bigcup_{1}^{\infty} E_{j} \right) \).

			            In conclusion, \( \underline{\mu} \) is a measure on \( (X, \widetilde{\mathcal{M}}) \).

			            \bigskip

			            If \( E \in \mathcal{M} \) then according to the definition of \( \underline{\mu} \), \( \underline{\mu}(E) = \mu(E) \), so \( \underline{\mu} \) extends \( \mu \).

			            \bigskip

			            Suppose \( E \) is locally measurable with respect to \( \underline{\mu} \).

			            Let \( A \in \mathcal{M} \) with \( \mu(A) < \infty \) then \( A \in \widetilde{\mathcal{M}} \) for \( \mathcal{M} \subseteq \widetilde{\mathcal{M}} \) and \( \underline{\mu}(A) = \mu(A) < \infty \). Because \( E \) is locally measurable with respect to \( \underline{\mu} \), \( E \cap A \in \widetilde{\mathcal{M}} \). Moreover,  \( E \cap A \in \widetilde{\mathcal{M}} \) implies  \( E \cap A \in \widetilde{\mathcal{M}} \) being locally measurable with respect to \( \mu \), so \( (E \cap A) \cap A \in \mathcal{M} \), which means \( E \cap A \in \mathcal{M} \) for every \( A \in \mathcal{M} \) with finite measure. Hence \( E \) is locally measurable with respect to \( \mu \), and \( E \in \widetilde{\mathcal{M}} \). Therefore \( \underline{\mu} \) is saturated.

			            \bigskip

			            Thus \( \underline{\mu} \) is a saturated measure on \( (X, \widetilde{\mathcal{M}}) \) and \( \underline{\mu} \) extends \( \mu \).
			      \item \begin{itemize}
				            \item Prove that \( \mu \) is a measure on \( \mathcal{M} \).

				                  For every \( E \in \mathcal{M} \), \( \mu(E) = \mu_{0}(E \cap X_{1}) \ge 0 \).

				                  \( \mu(\varnothing) = \mu_{0}(\varnothing \cap X_{1}) = \mu_{0}(\varnothing) = 0 \).

				                  Let \( {\left\{ E_{j} \right\}}_{1}^{\infty} \) be an infinite sequence of pairwise disjoint measurable sets in \( \mathcal{M} \). Then
				                  \[
					                  \mu\left(\bigcup_{1}^{\infty} E_{j}\right) = \mu_{0}\left( \bigcup_{1}^{\infty} E_{j} \cap X_{1} \right) = \sum_{1}^{\infty} \mu_{0}(E_{j} \cap X_{1}) = \sum_{1}^{\infty} \mu(E_{j}).
				                  \]

				                  Hence \( \mu \) is a measure on \( \mathcal{M} \).
				            \item Prove that \( \mu \) is semifinite. (We need this step in order to make sense of \( \underline{\mu} \).)

				                  If \( \mu(E) = \infty \) then \( E \cap X_{1} \) is infinite. Therefore \( E \cap X_{1} \) contains a nonempty measurable subset \( F \) which has finitely many elements, then \( \mu(F) = \mu_{0}(F \cap X_{1}) = \mu_{0}(F) \in \left\rbrack 0, \infty \right\lbrack \). Thus \( \mu \) is semifinite.
				            \item Prove that \( \widetilde{\mathcal{M}} = \mathcal{P}(X) \).

				                  Let \( E \) be a subset of \( X \).

				                  Let \( A \in \mathcal{M} \) such that \( \mu(A) < \infty \), then \( A \cap X_{1} \) is finite and \( A \) is not co-countable. Therefore \( A \) is countable and \( E \cap A \) is countable, which means \( E \cap A \in \mathcal{M} \).

				                  Hence \( E \) is locally measurable with respect to \( \mu \). Due to the arbitrariness of \( E \), we conclude that \( \widetilde{\mathcal{M}} = \mathcal{P}(X) \).
				            \item Prove that \( \widetilde{\mu} \ne \underline{\mu} \).

				                  \( \widetilde{\mu}(X_{2}) = \infty \) because \( X_{2} \notin \mathcal{M} \).

				                  \( \underline{\mu}(X_{2}) = \sup\left\{ \mu(A) : A \in \mathcal{M}, A \subseteq X_{2} \right\} = 0 \).

				                  Thus \( \widetilde{\mu} \ne \underline{\mu} \).
			            \end{itemize}
		      \end{enumerate}
	      \end{proof}
\end{enumerate}

\section{Outer Measures}

\subsection*{Exercises}

\begin{enumerate}[itemsep=0pt,label=\textbf{\arabic*.}]
	\setcounter{enumi}{16}
	\item If \( \mu^{\ast} \) is an outer measure on \( X \) and \( {\left\{ A_{j} \right\}}_{1}^{\infty} \) is a sequence of disjoint \( \mu^{\ast} \)-measurable sets, then \( \mu^{\ast}\left( E \cap \left( \bigcup_{1}^{\infty} A_{j} \right) \right) = \sum_{1}^{\infty} \mu^{\ast}(E \cap A_{j}) \) for any \( E \subseteq X \).
	\item
	\item
	\item
	\item
	\item
	\item
	\item
	\item
\end{enumerate}

\section{Borel Measures on the Real Line}

\subsection*{Exercises}

\begin{enumerate}[itemsep=0pt,label=\textbf{\arabic*.}]
	\setcounter{enumi}{24}
    \item
	\item
	\item
	\item
	\item
	\item
	\item
	\item
	\item
\end{enumerate}
