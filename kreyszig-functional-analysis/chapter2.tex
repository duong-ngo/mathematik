\chapter{Normed Spaces. Banach Spaces}

\section{Vector Space}

% chapter2:section1:exercise1
\begin{exercise}\label{chapter2:section1:exercise1}
    Show that the set of all real numbers, with the usual addition and multiplication, constitutes a one-dimensional real vector space, and the set of all complex numbers constitutes a one-dimensional complex vector space.
\end{exercise}

\begin{proof}
    I skip this exercise.
\end{proof}
\newpage

% chapter2:section1:exercise2
\begin{exercise}\label{chapter2:section1:exercise2}
    Prove that $0x = \theta$, $\alpha\theta = \theta$ and $(-1)x = -x$.
\end{exercise}

\begin{proof}
    I skip this exercise.
\end{proof}

% chapter2:section1:exercise3
\begin{exercise}\label{chapter2:section1:exercise3}
    Describe the span of $M = \{ (1, 1, 1), (0, 0, 2) \}$ in $\mathbb{R}^{3}$.
\end{exercise}

\begin{proof}
    \[
        M = \{ (\xi, \xi, \eta) \mid \xi, \eta\in\mathbb{R} \}
    \]

    $M$ is the plane of which equation is $x - y = 0$.
\end{proof}

% chapter2:section1:exercise4
\begin{exercise}\label{chapter2:section1:exercise4}
    Which of the following subsets of $\mathbb{R}^{3}$ constitute a subspace of $\mathbb{R}^{3}$? [Here, $x = (\xi_{1}, \xi_{2}, \xi_{3})$.]
    \begin{enumerate}[label={(\alph*)}]
        \item All $x$ with $\xi_{1} = \xi_{2}$ and $\xi_{3} = 0$.
        \item All $x$ with $\xi_{1} = \xi_{2} + 1$.
        \item All $x$ with positive $\xi_{1}, \xi_{2}, \xi_{3}$.
        \item All $x$ with $\xi_{1} - \xi_{2} + \xi_{3} = k = \text{const}$.
    \end{enumerate}
\end{exercise}

\begin{proof}
    \begin{enumerate}[label={(\alph*)}]
        \item This is a subspace of $\mathbb{R}^{3}$.
        \item This is not a subspace of $\mathbb{R}^{3}$ (not closed under scalar multiplication).
        \item This is not a subspace of $\mathbb{R}^{3}$ (not closed under scalar multiplication).
        \item This is a subspace of $\mathbb{R}^{3}$ if and only if $k = 0$.
    \end{enumerate}
\end{proof}

% chapter2:section1:exercise5
\begin{exercise}\label{chapter2:section1:exercise5}
    Show that $\{ x_{1}, \ldots, x_{n} \}$, where $x_{j}(t) = t^{j}$, is a linearly independent set in the space $C[a, b]$.
\end{exercise}

\begin{proof}
    Suppose $a_{1}x_{1} + \cdots + a_{n}x_{n} = 0$, then $a_{1}t^{1} + \cdots + a_{n}t^{n} = 0$ for every $t\in \closedinterval{a, b}$.

    Assume $a_{1}, \ldots, a_{n}$ are not all zero, then the equation $a_{1}t^{1} + \cdots + a_{n}t^{n} = 0$ has at most $n$ solutions. However, this equation has infinitely many solutions because every $t\in\closedinterval{a, b}$ satisfies. So the assumption is false.

    Hence $a_{1} = \cdots = a_{n} = 0$. Thus $\{ x_{1}, \ldots, x_{n} \}$ is a linearly independent set in the space $C[a, b]$.
\end{proof}

% chapter2:section1:exercise6
\begin{exercise}\label{chapter2:section1:exercise6}
    Show that in an $n$-dimensional vector space $X$, the representation of any $x$ as a linear combination of given basis vectors $e_{1}, \ldots, e_{n}$ is unique.
\end{exercise}

\begin{proof}
    I skip this exercise.
\end{proof}

% chapter2:section1:exercise7
\begin{exercise}\label{chapter2:section1:exercise7}
    Let $\{ e_{1}, \ldots, e_{n} \}$ be a basis for a complex vector space $X$. Find a basis for $X$ regarded as a real vector space. What is the dimension of $X$ in either case?
\end{exercise}

\begin{proof}
    If $X$ is regarded as a complex vector space, then $\dim X = n$.

    If $X$ is regarded as a real vector space, then
    \[
        \{ e_{1}, \iota e_{1}, \ldots, e_{n}, \iota e_{n} \}
    \]

    is a basis of $X$, then $\dim X = 2n$.
\end{proof}

% chapter2:section1:exercise8
\begin{exercise}\label{chapter2:section1:exercise8}
    If $M$ is a linearly dependent set in a complex vector space $X$, is $M$ linearly dependent in $X$, regarded as a real vector space?
\end{exercise}

\begin{proof}
    Not necessary. Here is an example.

    $X = \mathbb{C}$ and $M = \{ 1, \iota \}$. If $X$ is regarded as a complex vector space then $M$ is a linearly dependent set.

    If $X$ is regarded as a real vector space then $M$ is a linearly independent set.
\end{proof}

% chapter2:section1:exercise9
\begin{exercise}\label{chapter2:section1:exercise9}
    On a fixed interval $\closedinterval{a, b}\subset\mathbb{R}$, consider the set $X$ consisting of all polynomials with real coefficients and of degree not exceeding a given $n$, and the polynomial $x = 0$ (for which a degree is not defined in the usual discussion of degree). Show that $X$, with the usual addition and the usual multiplication by real numbers, is a real vector space of
    dimension $n + 1$. Find a basis for $X$. Show that we can obtain a
    complex vector space $\tilde{X}$ in a similar fashion if we let those coefficients be complex. Is $X$ a subspace of $\tilde{X}$?
\end{exercise}

\begin{proof}
    A basis of $X$ is $\{ 1, t, t^{2}, \ldots, t^{n} \}$, so $\dim X = n + 1$.

    No, $X$ is not a subspace of $\tilde{X}$, because $X$ is not closed under scalar multiplication (using complex numbers).
\end{proof}

% chapter2:section1:exercise10
\begin{exercise}\label{chapter2:section1:exercise10}
    If $Y$ and $Z$ are subspaces of a vector space $X$, show that $Y\cap Z$ is a subspace of $X$, but $Y\cup Z$ need not be one. Give examples.
\end{exercise}

\begin{proof}
    If $v_{1}, v_{2}$ are in $Y\cap Z$, then $v_{1} + v_{2}\in Y$ and $v_{1} + v_{2}\in Z$, so $v_{1} + v_{2} \in Y\cap Z$.

    If $v$ in $Y\cap Z$ then for every $\lambda\in K$, $\lambda v\in Y$ and $\lambda v\in Z$, so $\lambda v\in Y\cap Z$.

    Hence $Y\cap Z$ is a subspace of $X$.
    \bigskip

    $Y\cup Z$ need not be a subspace of $X$. For example: $X = \mathbb{R}^{2}$, $K = \mathbb{R}$, $Y = \operatorname{span} (1, 0)$ and $Z = \operatorname{span} (0, 1)$. However, $Y\cup Z$ is not a subspace of $X$ because it is not closed under vector addition.
\end{proof}

% chapter2:section1:exercise11
\begin{exercise}\label{chapter2:section1:exercise11}
    If $M \ne \varnothing$ is any subset of a vector space $X$, show that $\operatorname{span} M$ is a subspace of $X$.
\end{exercise}

\begin{proof}
    \[
        \operatorname{span} M = \left\{ \sum^{n}_{j=1}a_{j}v_{j} \mid v_{j} \in M \right\}
    \]

    Let $v$ and $w$ be two vectors in $\operatorname{span} M$. Since $v$ and $w$ are linear combinations of (finitely many) vectors in $M$ then $v + w$ is also a linear combination of vectors in $M$. For every $\lambda \in X$, $\lambda v$ is also a linear combination of vectors in $M$. Hence $\operatorname{span} M$ is a subspace of $X$ because it is closed under vector addition and scalar multiplication.
\end{proof}

% chapter2:section1:exercise12
\begin{exercise}\label{chapter2:section1:exercise12}
    Show that the set of all real two-rowed square matrices forms a vector space $X$. What is the zero vector in $X$? Determine $\dim X$. Find a basis for $X$. Give examples of subspaces of $X$. Do the symmetric matrices $x\in X$ form a subspace? The singular matrices?
\end{exercise}

\begin{proof}
    The zero vector in $X$ is $\begin{pmatrix}0 & 0 \\ 0 & 0\end{pmatrix}$. $\dim X = 2\times 2 = 4$. A basis for $X$ is
    \[
        \left\{
        \begin{pmatrix}1 & 0 \\ 0 & 0\end{pmatrix},
        \begin{pmatrix}0 & 1 \\ 0 & 0\end{pmatrix},
        \begin{pmatrix}0 & 0 \\ 1 & 0\end{pmatrix},
        \begin{pmatrix}0 & 0 \\ 0 & 1\end{pmatrix}
        \right\}
    \]

    The symmetric matrices of $X$ do form a subspace of $X$. However, the singular matrices do not form a subspace of $X$, because it is not closed under vector addition, since
    \[
        \begin{pmatrix}1 & 0 \\ 0 & 0\end{pmatrix} +
        \begin{pmatrix}0 & 1 \\ 0 & 0\end{pmatrix} +
        \begin{pmatrix}0 & 0 \\ 1 & 0\end{pmatrix} +
        \begin{pmatrix}0 & 0 \\ 0 & 1\end{pmatrix} =
        \begin{pmatrix}1 & 0 \\ 0 & 1\end{pmatrix}
    \]

    where $\begin{pmatrix}1 & 0 \\ 0 & 1\end{pmatrix}$ is not singular.
\end{proof}

% chapter2:section1:exercise13
\begin{exercise}[Product]\label{chapter2:section1:exercise13}
    Show that the Cartesian product $X = X_{1}\times X_{2}$ of two vector spaces over the same field becomes a vector space if we define the two algebraic operations by
    \begin{align*}
        (x_{1}, x_{2}) + (y_{1}, y_{2}) & = (x_{1} + y_{1}, x_{2} + y_{2}), \\
        \alpha (x_{1}, x_{2})           & = (\alpha x_{1}, \alpha x_{2}).
    \end{align*}
\end{exercise}

\begin{proof}
    I skip this exercise.
\end{proof}

% chapter2:section1:exercise14
\begin{exercise}[Quotient space, codimension]\label{chapter2:section1:exercise14}
    Let $Y$ be a subspace of a vector space $X$. The coset of an element $x\in X$ with respect to $Y$ is denoted by $x + Y$ and is defined to be the set
    \[
        x + Y = \{ v \mid v = x + y, y\in Y \}.
    \]

    Show that the distinct cosets form a partition of $X$. Show that under algebraic operations defined by
    \begin{align*}
        (w + Y) + (x + Y) & = (w + x) + Y  \\
        \alpha (x + Y)    & = \alpha x + Y
    \end{align*}

    these cosets constitute the elements of a vector space. This space is called the quotient space (or sometimes factor space) of $X$ by $Y$ (or modulo $Y$) and is denoted by $X/Y$. Its dimension is called the codimension of $Y$ and is denoted by $\operatorname{codim} Y$, that is,
    \[
        \operatorname{codim} Y = \dim (X/Y).
    \]
\end{exercise}

\begin{proof}
    I skip this exercise.
\end{proof}

% chapter2:section1:exercise15
\begin{exercise}\label{chapter2:section1:exercise15}
    Let $X = \mathbb{R}^{3}$ and $Y = \{ (\xi_{1}, 0, 0) \mid \xi_{1}\in\mathbb{R} \}$. Find $X/Y$, $X/X$, $X/\{0\}$.
\end{exercise}

\begin{proof}
    $X/Y$ is the sets of lines which are parallel to $Y$.

    $X/X = \{ 0 + X \}$ and $X/\{ 0 \} = X$.
\end{proof}

\section{Normed Space. Banach Space}

% chapter2:section2:exercise1
\begin{exercise}\label{chapter2:section2:exercise1}
    Show that the norm $\norm{x}$ of $x$ is the distance from $x$ to $0$.
\end{exercise}

\begin{proof}
    $\norm{x} = \norm{x - 0} = d(x, 0)$.
\end{proof}

% chapter2:section2:exercise2
\begin{exercise}\label{chapter2:section2:exercise2}
    Verify that the usual length of a vector in the plane or in three dimensional space has the properties (Nt) to (N4) of a norm.
\end{exercise}

\begin{proof}
    I skip this exercise.
\end{proof}

% chapter2:section2:exercise3
\begin{exercise}\label{chapter2:section2:exercise3}
    Prove that $\abs{\norm{y} - \norm{x}} \leq \norm{y - x}$.
\end{exercise}

\begin{proof}
    \begin{align*}
        \norm{y} & = \norm{x + (y - x)}\leq \norm{x} + \norm{y - x},                            \\
        \norm{x} & = \norm{y + (x - y)} \leq \norm{y} + \norm{x - y} = \norm{y} + \norm{y - x}.
    \end{align*}

    So $\norm{y} - \norm{x}\leq \norm{y - x}$ and $\norm{x} - \norm{y}\leq \norm{y - x}$. Therefore $\abs{\norm{y} - \norm{x}} \leq \norm{y - x}$.
\end{proof}

% chapter2:section2:exercise4
\begin{exercise}\label{chapter2:section2:exercise4}
    Show that we may replace (N2) by
    \[
        \norm{x} = 0 \implies x = 0
    \]

    without altering the concept of a norm. Show that nonnegativity of a norm also follows from (N3) and (N4).
\end{exercise}

\begin{proof}
    If $x = 0$ then $\norm{x} = \norm{0\cdot x} = \abs{0}\norm{x} = 0$. Therefore
    \[
        \left(\norm{x} = 0 \implies x = 0\right) \Longleftrightarrow \left(\norm{x} = 0 \Leftrightarrow x = 0\right)
    \]

    so (N2) can be replaced by $\norm{x} = 0 \implies x = 0$.

    Nonnegativity follows from (N3) and (N4) because
    \[
        2\norm{x} = \norm{x} + \norm{x} = \norm{x} + \norm{(-1)x} \geq \norm{(1 + (-1))x} = \norm{0} = 0
    \]

    so $\norm{x}\geq 0$ for every $x\in X$.
\end{proof}

% chapter2:section2:exercise5
\begin{exercise}\label{chapter2:section2:exercise5}
    Show that $\norm{x} = {\left(\sum^{n}_{j=1}\abs{\xi_{j}}^{2}\right)}^{1/2}$ defines a norm.
\end{exercise}

\begin{proof}
    I skip this exercise.
\end{proof}

% chapter2:section2:exercise6
\begin{exercise}\label{chapter2:section2:exercise6}
    Let $X$ be the vector space of all ordered pairs $x = (\xi_{1}, \xi_{2})$, $y = (\eta_{1}, \eta_{2})$, \ldots of real numbers. Show that norms on $X$ are defined by
    \begin{align*}
        \norm{x}_{1}      & = \abs{\xi_{1}} + \abs{\xi_{2}}                   \\
        \norm{x}_{2}      & = {(\abs{\xi_{1}}^{2} + \abs{\xi_{2}}^{2})}^{1/2} \\
        \norm{x}_{\infty} & = \max\{ \abs{\xi_{1}}, \abs{\xi_{2}} \}.
    \end{align*}
\end{exercise}

\begin{proof}
    I skip this exercise.
\end{proof}

% chapter2:section2:exercise7
\begin{exercise}\label{chapter2:section2:exercise7}
    Show that $\norm{x} = {\left(\sum^{n}_{j=1}\abs{\xi_{j}}^{p}\right)}^{1/p}$ defines a norm.
\end{exercise}

\begin{proof}
    I skip this exercise.
\end{proof}

% chapter2:section2:exercise8
\begin{exercise}\label{chapter2:section2:exercise8}
    There are several norms of practical importance on the vector space of ordered $n$-tuples of numbers, notably those defined by
    \begin{align*}
        \norm{x}_{1}      & = \abs{\xi_{1}} + \abs{\xi_{2}} + \cdots + \abs{\xi_{n}}                                                      \\
        \norm{x}_{p}      & = {\left(\abs{\xi_{1}}^{p} + \abs{\xi_{2}}^{p} + \cdots + \abs{\xi_{n}}^{p}\right)}^{1/p} & (1 < p < +\infty) \\
        \norm{x}_{\infty} & = \max\{ \abs{\xi_{1}}, \ldots, \abs{\xi_{n}} \}.
    \end{align*}

    In each case, verify that (N1) to (N4) are satisfied.
\end{exercise}

\begin{proof}
    I skip this exercise.
\end{proof}

% chapter2:section2:exercise9
\begin{exercise}\label{chapter2:section2:exercise9}
    Verify that $\norm{x} = \max_{t\in \closedinterval{a, b}}\abs{x(t)}$ defines a norm on $C[a, b]$.
\end{exercise}

\begin{proof}
    For every $x\in C[a, b]$, $\norm{x} = \max\limits_{t\in \closedinterval{a,b}} \abs{x(t)}\geq 0$ because $\abs{x(t)}\geq 0$ for all $t\in \closedinterval{a, b}$. So (N1) is true.

    $\norm{x} = \max\limits_{t\in \closedinterval{a,b}} \abs{x(t)} = 0$ if and only if $x(t) = 0$ for every $t\in \closedinterval{a,b}$, which means $x = 0$. So (N2) is true.

    $\norm{\alpha x} = \max\limits_{t\in \closedinterval{a,b}} \abs{\alpha x(t)} = \abs{\alpha}\max\limits_{t\in \closedinterval{a,b}} \abs{x(t)} = \abs{\alpha}\norm{x}$. So (N3) is true.
    \begin{align*}
        \norm{x + y} & = \max\limits_{t\in \closedinterval{a,b}} \abs{x(t) + y(t)} \\
                     & \leq \max\limits_{t\in \closedinterval{a,b}} \left(\abs{x(t)} + \abs{y(t)}\right) \\
                     & \leq \max\limits_{t\in \closedinterval{a,b}} \abs{x(t)} + \max\limits_{t\in \closedinterval{a,b}} \abs{y(t)} \\
                     & = \norm{x} + \norm{y}.
    \end{align*}

    So (N4) is true. Thus $\norm{x} = \max_{t\in \closedinterval{a, b}}\abs{x(t)}$ defines a norm on $C[a, b]$.
\end{proof}

% chapter2:section2:exercise10
\begin{exercise}[Unit sphere]\label{chapter2:section2:exercise10}
    The sphere
    \[ S(0; 1) = \{ x\in X \mid \norm{x} = 1 \} \]

    in a normed space $X$ is called the unit sphere. Show that for the norms in Exercise~\ref{chapter2:section2:exercise6} and for the norm defined by
    \[
        \norm{x}_{4} = {(\abs{\xi_{1}}^{4} + \abs{\xi_{2}}^{4})}^{1/4}
    \]

    the unit spheres look as shown in Figure 16.
\end{exercise}

\begin{proof}
    I skip this exercise.
\end{proof}

% chapter2:section2:exercise11
\begin{exercise}[Convex set, segment]\label{chapter2:section2:exercise11}
    A subset $A$ of a vector space $X$ is said to be convex if $x, y\in A$ implies
    \[
        M = \{ z\in X \mid z = \alpha x + (1 - \alpha)y,\, 0\leq \alpha\leq 1 \}\subset A.
    \]

    $M$ is called a closed segment with boundary points $x$ and $y$; any other $z\in M$ is called an interior point of $M$. Show that the closed unit ball
    \[
        \tilde{B}(0; 1) = \{ x\in X \mid \norm{x}\leq 1 \}
    \]

    in a normed space $X$ is convex.
\end{exercise}

\begin{proof}
    Let $x$ and $y$ be two points in $\tilde{B}(0; 1)$. For every $\alpha\in \closedinterval{0, 1}$
    \begin{align*}
        \norm{\alpha x + (1 - \alpha)y} & \leq \norm{\alpha x} + \norm{(1 - \alpha)y} \\
                                        & = \alpha\norm{x} + (1 - \alpha)\norm{y} \\
                                        & \leq \alpha + (1 - \alpha) = 1.
    \end{align*}

    Hence $\alpha x + (1 - \alpha)y\in \tilde{B}(0; 1)$ for every $x, y\in \tilde{B}(0; 1)$ and $\alpha\in\closedinterval{0, 1}$. Thus $\tilde{B}(0; 1)$ is convex.
\end{proof}

% chapter2:section2:exercise12
\begin{exercise}\label{chapter2:section2:exercise12}
    Using Exercise~\ref{chapter2:section2:exercise11}, show that
    \[
        \varphi(x) = {(\sqrt{\abs{\xi_{1}}} + \sqrt{\abs{\xi_{2}}})}^{2}
    \]

    does not define a norm on the vector space of all ordered pairs $x = (\xi_{1}, \xi_{2})$, \ldots of real numbers.
\end{exercise}

\begin{proof}
    Assume $\varphi$ defines a norm on $\mathbb{R}^{2}$, then $\varphi(1/2, 1/2)$ is in $\tilde{B}(0; 1)$. However,
    \[
        \varphi(1/2, 1/2) = {\left(1/\sqrt{2} + 1/\sqrt{2}\right)}^{2} = 2 > 1
    \]

    which is a contradiction. Hence $\varphi$ does not define a norm on $\mathbb{R}^{2}$.
\end{proof}

% chapter2:section2:exercise13
\begin{exercise}\label{chapter2:section2:exercise13}
    Show that the discrete metric on a vector space $X\ne \{0\}$ cannot be obtained from a norm.
\end{exercise}

\begin{proof}
    Assume the discrete metric on a vector space $X\ne \{0\}$ cannot be obtained from a norm $\norm{\cdot}$.

    Let $x$ be a nonzero vector in $X$, then $\norm{x} = 1$. Moreover, $\norm{2x} = 2\norm{x} = 2$ but $\norm{2x} = 1$, this is a contradiction. Hence the discrete metric on a vector space $X\ne \{0\}$ cannot be obtained from a norm.
\end{proof}

% chapter2:section2:exercise14
\begin{exercise}\label{chapter2:section2:exercise14}
    If $d$ is a metric on a vector space $X \ne \{0\}$ which is obtained from a norm, and $\tilde{d}$ is defined by
    \[
        \tilde{d}(x, x) = 0, \qquad \tilde{d}(x, y) = d(x, y) + 1 \qquad (x\ne y),
    \]

    show that $\tilde{d}$ cannot be obtained from a norm.
\end{exercise}

\begin{proof}
    If $x\ne y$ then $\tilde{d}(2x, 2y) = d(2x, 2y) + 1 = 2d(x, y) + 1\ne 2d(x, y) + 2 = 2\tilde{d}(x, y)$. Hence $\tilde{d}$ cannot be obtained from a norm.
\end{proof}

% chapter2:section2:exercise15
\begin{exercise}[Bounded set]\label{chapter2:section2:exercise15}
    Show that a subset $M$ in a normed space $M$ is bounded if and only if there is a positive number $c$ such that $\norm{x}\leq c$ for every $x\in M$.
\end{exercise}

\begin{proof}
    $(\Rightarrow)$ $M$ is bounded.

    Then there exists a positive number $a$ such that $\sup\limits_{x, y\in M}d(x, y)\leq a$.

    Let $x, x_{0}$ be vectors in $M$, then $\norm{x} \leq \norm{x - x_{0}} + \norm{x_{0}} \leq a + \norm{x_{0}}$.

    Let $c = a + \norm{x_{0}}$ then for every $x\in M$, $\norm{x}\leq c$.

    $(\Leftarrow)$ There is a positive number $c$ such that $\norm{x}\leq c$ for every $x\in M$.

    For every $x, y\in M$, $d(x, y) = \norm{x - y}\leq \norm{x} + \norm{-y} = \norm{x} + \norm{y} \leq 2c$.

    Therefore $\sup\limits_{x,y\in M}d(x, y)\leq 2c$, so $M$ is bounded.
\end{proof}

\section{Further Properties of Normed Spaces}

\section{Finite Dimensional Normed Spaces and Subspaces}

\section{Compactness and Finite Dimension}

\section{Linear Operators}

\section{Bounded and Continuous Linear Operators}

\section{Linear Functionals}

\section{Linear Operators and Functionals on Finite Dimensional Spaces}

\section{Normed Spaces of Operators. Dual Space}
