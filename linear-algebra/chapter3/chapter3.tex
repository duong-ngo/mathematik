% chktex-file 44
\documentclass[class=linear-algebra,crop=false]{standalone}

\newcommand{\sgn}[1]{\text{sgn}\left({#1}\right)}
\setcounter{lemma}{0}

\begin{document}

\chapter{Định thức và hệ phương trình tuyến tính}

\par Thực hiện các phép nhân sau đây, viết các phép thế thu được thành tích của những xích rời rạc và tính dấu của chúng.

% exercise 3.1
\begin{exercise}
    $
        \begin{pmatrix}
            1 & 2 & 3 & 4 & 5 \\
            2 & 4 & 5 & 1 & 3
        \end{pmatrix}
        \begin{pmatrix}
            1 & 2 & 3 & 4 & 5 \\
            4 & 3 & 5 & 1 & 2
        \end{pmatrix}
    $.
\end{exercise}

\begin{proof}[Lời giải]
    \[
        \begin{pmatrix}
            1 & 2 & 3 & 4 & 5 \\
            2 & 4 & 5 & 1 & 3
        \end{pmatrix}
        \begin{pmatrix}
            1 & 2 & 3 & 4 & 5 \\
            4 & 3 & 5 & 1 & 2
        \end{pmatrix}
        =
        \begin{pmatrix}
            4 & 3 & 5 & 1 & 2 \\
            1 & 5 & 3 & 2 & 4
        \end{pmatrix}
        \begin{pmatrix}
            1 & 2 & 3 & 4 & 5 \\
            4 & 3 & 5 & 1 & 2
        \end{pmatrix}
        =
        \begin{pmatrix}
            1 & 2 & 3 & 4 & 5 \\
            1 & 5 & 3 & 2 & 4
        \end{pmatrix}.
    \]
    \[
        \begin{pmatrix}
            1 & 2 & 3 & 4 & 5 \\
            1 & 5 & 3 & 2 & 4
        \end{pmatrix}
        =
        (1)(2,5,4)(3).
    \]
    \[
        \sgn{
            \begin{matrix}
                1 & 2 & 3 & 4 & 5 \\
                1 & 5 & 3 & 2 & 4
            \end{matrix}
        }
        = \sgn{1}\sgn{2,5,4}\sgn{3}
        = 1.
    \]
\end{proof}

% exercise 3.2
\begin{exercise}
    $
        \begin{pmatrix}
            1 & 2 & 3 & 4 & 5 \\
            3 & 5 & 4 & 1 & 2
        \end{pmatrix}
        \begin{pmatrix}
            1 & 2 & 3 & 4 & 5 \\
            4 & 3 & 1 & 5 & 2
        \end{pmatrix}
    $.
\end{exercise}

\begin{proof}[Lời giải]
    \[
        \begin{pmatrix}
            1 & 2 & 3 & 4 & 5 \\
            3 & 5 & 4 & 1 & 2
        \end{pmatrix}
        \begin{pmatrix}
            1 & 2 & 3 & 4 & 5 \\
            4 & 3 & 1 & 5 & 2
        \end{pmatrix}
        =
        \begin{pmatrix}
            4 & 3 & 1 & 5 & 2 \\
            1 & 4 & 3 & 2 & 5
        \end{pmatrix}
        \begin{pmatrix}
            1 & 2 & 3 & 4 & 5 \\
            4 & 3 & 1 & 5 & 2
        \end{pmatrix}
        =
        \begin{pmatrix}
            1 & 2 & 3 & 4 & 5 \\
            1 & 4 & 3 & 2 & 5
        \end{pmatrix}.
    \]
    \[
        \begin{pmatrix}
            1 & 2 & 3 & 4 & 5 \\
            1 & 4 & 3 & 2 & 5
        \end{pmatrix}
        =
        (1)(2,4)(3)(5).
    \]
    \[
        \sgn{
            \begin{matrix}
                1 & 2 & 3 & 4 & 5 \\
                1 & 4 & 3 & 2 & 5
            \end{matrix}
        }
        = \sgn{1}\sgn{2,4}\sgn{3}\sgn{5}
        = -1.
    \]
\end{proof}

% exercise 3.3
\begin{exercise}
    $(1,2)(2,3)\ldots (n-1,n)$.
\end{exercise}

\begin{lemma}\label{chapter3:cycles-product}
    $(a_{1}, a_{2}, \ldots, a_{k})(a_{k},a_{k+1}) = (a_{1},a_{2},\ldots, a_{k+1})$.
\end{lemma}

\begin{proof}[Chứng minh bổ đề]
    \par Xét dãy
    \[
        a_{1}, a_{2}, \ldots, a_{k-1}, a_{k}, a_{k+1}.
    \]
    \par Sau khi tác động bằng $(a_{k},a_{k+1})$, dãy trên trở thành:
    \[
        a_{1}, a_{2}, \ldots, a_{k-1}, a_{k+1}, a_{k}.
    \]
    \par Sau khi tác động bằng $(a_{1}, a_{2}, \ldots, a_{k})$, dãy trên (liền trên) trở thành:
    \[
        a_{2}, a_{3}, \ldots, a_{k}, a_{k+1}, a_{1}.
    \]
    \par Theo định nghĩa về xích, ta có điều phải chứng minh.
\end{proof}

\begin{proof}[Lời giải]
    \par Theo bổ đề~\ref{chapter3:cycles-product}:
    \[
        (1,2)(2,3)\ldots (n-1,n) = (1,2,\ldots,n)
        =
        \begin{pmatrix}
            1 & 2 & \cdots & n-1 & n \\
            2 & 3 & \cdots & n   & 1
        \end{pmatrix}
    \]
    \par $(1,2,\ldots, n)$ chính là một xích.
    \[
        \sgn{1,2,\ldots,n} = \sgn{1,2}\sgn{2,3}\ldots\sgn{n-1,n} = (-1){}^{n-1}.
    \]
\end{proof}

% exercise 3.4
\begin{exercise}
    $(1,2,3)(2,3,4)(3,4,5)\ldots (n-2,n-1,n)$.
\end{exercise}

\begin{proof}[Lời giải]
    \par Theo bổ đề~\ref{chapter3:cycles-product}, nếu $n > 3$:
    \begin{align*}
        (1,2,3)(2,3,4)(3,4,5)\ldots (n-2,n-1,n)
         & = (1,2)(2,3)(2,3)(3,4)(3,4)(4,5) \ldots (n-2,n-1)(n-1,n)   \\
         & = (1,2)(2,3){}^{2}(3,4){}^{2}\ldots (n-2,n-1){}^{2}(n-1,n) \\
         & = (1,2)(n-1,n)\qquad\text{(đây là 2 xích rời nhau)}        \\
         & =
        \begin{pmatrix}
            1 & 2 & 3 & \cdots & n-2 & n-1 & n   \\
            2 & 1 & 3 & \cdots & n-2 & n   & n-1
        \end{pmatrix}.
    \end{align*}
    \par Nếu $n = 3$:
    \[
        (1,2,3) =
        \begin{pmatrix}
            1 & 2 & 3 \\
            2 & 3 & 1
        \end{pmatrix}.
    \]
    \par Trong cả hai trường hợp, dấu của phép thế (kết quả) là 1.
\end{proof}

% exercise 3.5
\begin{exercise}
    Cho hai cách sắp thành dãy $a_{1}$, $a_{2}$, \ldots, $a_{n}$ và $b_{1}$, $b_{2}$, \ldots, $b_{n}$ của $n$ số tự nhiên đầu tiên. Chứng minh rằng có thể đưa cách sắp này về cách sắp kia bằng cách sử dụng không quá $n-1$ phép thế sơ cấp.
\end{exercise}

\begin{lemma}\label{chapter3:fixed-point}
    $\forall\sigma,\ \forall x\in\{ 1,2,\ldots,n \}, \exists k\in\mathbb{N}, k\le n$ sao cho $\sigma^{k}(x) = x$.
\end{lemma}

\begin{proof}[Chứng minh bổ đề~\ref{chapter3:existence-of-cycle}]
    \par Giả sử phản chứng, $\sigma^{k}(x)\ne x,\ \forall k=\overline{1,n}$.
    \par Có hai trường hợp cần xem xét.
    \begin{enumerate}[label = Trường hợp \arabic*:,itemindent=2cm]
        \item $n$ số $\sigma(x)$, $\sigma^{2}(x)$, \ldots, $\sigma^{n}(x)$ đôi một khác nhau.
              \par Vì $n$ số này đôi một khác nhau và đều là các phần tử của tập hợp $n$ số tự nhiên đầu tiên nên tồn tại một số tự nhiên $k\le n$ sao cho $\sigma^{k}(x) = x$.
        \item Trong $n$ số $\sigma(x)$, $\sigma^{2}(x)$, \ldots, $\sigma^{n}(x)$, có hai số bằng nhau.
              \par Giả sử $\sigma^{a}(x) = \sigma^{a+b}(x)$, trong đó $1\le a < a + b\le n$.
              \[
                  \begin{split}
                      &(\underbrace{\sigma\circ\cdots\circ\sigma}_{a})(x) = (\underbrace{\sigma\circ\cdots\circ\sigma}_{a+b})(x) \\
                      \Leftrightarrow& ((\underbrace{\sigma^{-1}\circ\ldots\circ\sigma^{-1}}_{a})\circ\underbrace{(\sigma\circ\cdots\circ\sigma)}_{a})(x) = ((\underbrace{\sigma^{-1}\circ\ldots\circ\sigma^{-1}}_{a})\circ\underbrace{(\sigma\circ\cdots\circ\sigma)}_{a+b})(x) \\
                      \Leftrightarrow& x = \sigma^{b}(x).
                  \end{split}
              \]
    \end{enumerate}
    \par Bổ đề được chứng minh.
\end{proof}

\begin{lemma}\label{chapter3:existence-of-cycle}
    Mỗi phần tử của $n$ số tự nhiên đầu tiên luôn thuộc một xích nào đó.
\end{lemma}

\begin{proof}
    \par Theo bổ đề~\ref{chapter3:fixed-point}, cùng với nguyên lý sắp thứ tự tốt (well-ordering principle), ta suy ra luôn chọn được số tự nhiên $k$ nhỏ nhất sao cho $\sigma^{k}(x) = x$.
    \par $(x, \sigma(x), \ldots, \sigma^{k-1}(x))$ chính là một xích độ dài $k$.
\end{proof}

\begin{lemma}\label{chapter3:product-of-disjoint-cycles}
    Mọi phép thế đều có thể được viết dưới dạng tích của các xích rời nhau.
\end{lemma}

\begin{proof}[Chứng minh bổ đề~\ref{chapter3:product-of-disjoint-cycles}]
    \par Theo bổ đề~\ref{chapter3:existence-of-cycle}, mỗi phần tử đều thuộc một xích nào đó.
    \par Nếu hai xích cùng chứa một phần tử $x$ thì hai xích đó trùng nhau.
    \par Do đó hai xích bất kì hoặc trùng nhau, hoặc rời nhau.
    \par Ta tiến hành phân tích một phép thế thành các xích rời nhau.
    \begin{enumerate}[label = (\arabic*)]
        \item Chọn lấy một phần tử bất kì của tập hợp $n$ số tự nhiên đầu tiên.
        \item Xác định xích của phần tử đó.
        \item Chọn lấy một phần tử bất kì không thuộc xích, tiếp xúc xác định xích của phần tử mới này.
    \end{enumerate}
    \par Quy trình trên được làm liên tục đến khi không còn phần tử nào để chọn. Quy trình này sẽ dừng lại sau hữu hạn bước vì số phần tử của tập hợp $n$ số tự nhiên là hữu hạn, và độ dài của một xích cũng không vượt quá số phần tử của tập hợp $n$ số tự nhiên đầu tiên.
\end{proof}

\begin{lemma}\label{chapter3:product-of-transpositions}
    Một xích độ dài $k$ ($k > 1$) có thể viết được dưới dạng tích của $k-1$ phép thế sơ cấp.
\end{lemma}

\begin{proof}[Chứng minh bổ đề~\ref{chapter3:product-of-transpositions}]
    \par Một xích độ dài $k$ sẽ tác động lên dãy $x_{1}$, $x_{2}$, \ldots, $x_{k}$ như sau:
    \[
        x_{1}\mapsto x_{2} \mapsto x_{3} \mapsto \cdots \mapsto x_{k} \mapsto x_{1}.
    \]
    \par Xích trên có thể phân tích được thành tích của $k - 1$ phép thế sơ cấp, theo bổ đề~\ref{chapter3:cycles-product}:
    \begin{align*}
        (x_{1}, x_{2}, \ldots, x_{k}) & = (x_{1}, x_{2}, \ldots, x_{k-1})(x_{k-1}, x_{k})                   \\
                                      & = (x_{1}, x_{2}, \ldots, x_{k-2})(x_{k-2}, x_{k-1})(x_{k-1}, x_{k}) \\
                                      & = \ldots                                                            \\
                                      & = (x_{1}, x_{2})(x_{2}, x_{3})\ldots (x_{k-1}, x_{k}).
    \end{align*}
\end{proof}

\begin{proof}
    \par Xét phép thế $\sigma$ trên tập hợp $n$ số tự nhiên đầu tiên.
    \par Theo bổ đề~\ref{chapter3:product-of-disjoint-cycles}, ta phân tích phép thế $\sigma$ thành các xích rời nhau. Tổng độ dài của các xích này bằng $n$. Gọi số lượng xích rời rạc là $\ell$.
    \par Theo bổ đề~\ref{chapter3:product-of-transpositions}, mỗi xích độ dài $k$ đều phân tích được thành tích của $k-1$ phép thế sơ cấp.
    \par Kết hợp hai điều trên, một phép thế có thể phân tích được thành tích của $n - \ell$ phép thế sơ cấp. Mà $n - \ell \le n - 1$ nên ta có điều phải chứng minh.
    \par Điều này tương đương với việc có thể đưa cách sắp dãy $a_{1}$, $a_{2}$, \ldots, $a_{n}$ thành $b_{1}$, $b_{2}$, \ldots, $b_{n}$ và ngược lại bằng cách thực hiện không quá $n - 1$ phép thế sơ cấp.
\end{proof}

% exercise 3.6
\begin{exercise}
    Với giả thiết như bài trên, chứng minh rằng có thể đưa cách sắp này về cách sắp kia bằng cách sử dụng không quá $n(n-1)/2$ phép chuyển vị của hai phần tử đứng kề nhau.
\end{exercise}

\begin{proof}
    \par Ta chỉ cần chứng minh có thể đưa cách sắp thứ nhất thành cách sắp thứ hai bằng cách sử dụng không quá $n(n-1)/2$ phép chuyển vị của hai phần tử đứng kề nhau.
    \par Như giả thiết, cách sắp dãy thứ nhất là $a_{1}$, $a_{2}$, \ldots, $a_{n}$; cách sắp dãy thứ hai là $b_{1}$, $b_{2}$, \ldots, $b_{n}$.
    \par Lưu ý rằng $\{ a_{1}, a_{2}, \ldots, a_{n} \} = \{ b_{1}, b_{2}, \ldots, b_{n} \}$.

    \begin{enumerate}[label = (\arabic*)]
        \item Tồn tại duy nhất $a_{k_{1}}$ sao cho $a_{k_{1}} = b_{n}$.
        \item Ta lần lượt thực hiện cách phép chuyển vị hai phần tử kề nhau: $(a_{k_{1}}, a_{k_{1}+1})$, $(a_{k_{1}}, a_{k_{1}+2})$, \ldots $(a_{k_{1}}, a_{n})$.
        \item Sau khi thực hiện các phép chuyển vị như trên, $a_{k}$ ở vị trí cuối cùng trong dãy.
    \end{enumerate}
    \par Trong quy trình trên, ta thực hiện $n - k_{1}$ phép chuyển vị hai phần tử kề nhau.
    \par Trong $n - 1$ phần tử đầu tiên của dãy $a_{1}$, $a_{2}$, \ldots, $a_{n}$, tồn tại $a_{k_{2}} = b_{n-1}$,  ta thực hiện các phép chuyển vị $(a_{k_{2}}, a_{k_{2}+1})$, $(a_{k_{2}}, a_{k_{2}+2})$, \ldots $(a_{k_{2}}, a_{n-1})$. Ta đã thực hiện $(n - 1) - k_{2}$ phép chuyển vị hai phần tử kề nhau.
    \par Liên tiếp thực hiện việc dồn dãy như trên, ta đưa được cách sắp thứ nhất về cách sắp thứ hai $-$ chỉ bằng các phép chuyển vị hai phần tử kề nhau.
    \par Số phép chuyển vị hai phần tử kề nhau đã được sử dụng không vượt quá
    \[
        (n - 1) + (n - 2) + \cdots + 2 + 1 = \dfrac{n(n-1)}{2}.
    \]
    \par Đó cũng là điều phải chứng minh.
\end{proof}

% exercise 3.7
\begin{exercise}
    Cho ví dụ về một cách sắp dãy $n$ số tự nhiên đầu tiên thành dãy sao cho dãy này không thể đưa về dãy sắp tự nhiên bằng cách dùng ít hơn $n - 1$ phép thế sơ cấp.
\end{exercise}

\begin{proof}[Lời giải]
    \par Cách sắp như vậy được tạo ra từ một xích độ dài $n$.
    \[
        2, 3, \ldots, n, 1.
    \]
\end{proof}

% exercise 3.8
\begin{exercise}
    Biết số nghịch thế của dãy $a_{1}$, $a_{2}$, \ldots, $a_{n}$ là $k$. Hãy tìm số nghịch thế của dãy $a_{n}$, $a_{n-1}$, \ldots, $a_{1}$.
\end{exercise}

\begin{proof}[Lời giải]
    \par Không mất tính tổng quát, ta giả sử $a_{1}$, $a_{2}$, \ldots, $a_{n}$ là một hoán vị của $n$ số tự nhiên đầu tiên.
    \par Nếu cặp $(a_{i}, a_{j})$ trong dãy $a_{1}$, $a_{2}$, \ldots, $a_{n}$ là nghịch thế thì trong dãy $a_{n}$, $a_{n-1}$, \ldots, $a_{1}$ lại không phải nghịch thế.
    \par Nếu cặp $(a_{i}, a_{j})$ trong dãy $a_{1}$, $a_{2}$, \ldots, $a_{n}$ không phỉa nghịch thế thì trong dãy $a_{n}$, $a_{n-1}$, \ldots, $a_{1}$ lại là phải nghịch thế.
    \par Do đó, số nghịch thế trong $a_{n}$, $a_{n-1}$, \ldots, $a_{1}$ là $\dfrac{n(n-1)}{2} - k$.
\end{proof}

% exercise 3.9
\begin{exercise}
    Tính các định thức sau đây
    \begin{enumerate}[label = (\alph*)]
        \item $
                  \begin{vmatrix}
                      2  & -5 & 4  & 3 \\
                      3  & -4 & 7  & 5 \\
                      4  & -9 & 8  & 5 \\
                      -3 & 2  & -5 & 3
                  \end{vmatrix}
              $
        \item $
                  \begin{vmatrix}
                      3 & -3 & -2 & -5 \\
                      2 & 5  & 4  & 6  \\
                      5 & 5  & 8  & 7  \\
                      4 & 4  & 5  & 6
                  \end{vmatrix}
              $
    \end{enumerate}
\end{exercise}

\begin{proof}[Lời giải]
    \begin{enumerate}[label = (\alph*)]
        \item
              \begin{align*}
                  \begin{vmatrix}
                      2  & -5 & 4  & 3 \\
                      3  & -4 & 7  & 5 \\
                      4  & -9 & 8  & 5 \\
                      -3 & 2  & -5 & 3
                  \end{vmatrix}
                   & =
                  \begin{vmatrix}
                      2 & -5 & 4 & 3  \\
                      3 & -4 & 7 & 5  \\
                      0 & 1  & 0 & -1 \\
                      0 & -2 & 2 & 8
                  \end{vmatrix}
                  =
                  \begin{vmatrix}
                      2 & -5  & 4 & 3   \\
                      0 & 3.5 & 1 & 0.5 \\
                      0 & 1   & 0 & -1  \\
                      0 & -2  & 2 & 8
                  \end{vmatrix}
                  =
                  \begin{vmatrix}
                      2 & -5  & 4 & 3   \\
                      0 & 1   & 0 & -1  \\
                      0 & -2  & 2 & 8   \\
                      0 & 3.5 & 1 & 0.5
                  \end{vmatrix} \\
                   & =
                  \begin{vmatrix}
                      2 & -5 & 4 & 3  \\
                      0 & 1  & 0 & -1 \\
                      0 & 0  & 2 & 6  \\
                      0 & 0  & 1 & 4
                  \end{vmatrix}
                  =
                  \begin{vmatrix}
                      2 & -5 & 4 & 3  \\
                      0 & 1  & 0 & -1 \\
                      0 & 0  & 2 & 6  \\
                      0 & 0  & 0 & 1
                  \end{vmatrix}
                  = 2\cdot 1\cdot 2\cdot 1 = 4.
              \end{align*}
        \item
              \begin{align*}
                  \begin{vmatrix}
                      3 & -3 & -2 & -5 \\
                      2 & 5  & 4  & 6  \\
                      5 & 5  & 8  & 7  \\
                      4 & 4  & 5  & 6
                  \end{vmatrix}
                   & =
                  \begin{vmatrix}
                      1 & -8 & -6 & -11 \\
                      2 & 5  & 4  & 6   \\
                      5 & 5  & 8  & 7   \\
                      4 & 4  & 5  & 6
                  \end{vmatrix}
                  =
                  \begin{vmatrix}
                      1 & -8 & -6 & -11 \\
                      0 & 21 & 16 & 28  \\
                      0 & 45 & 38 & 62  \\
                      0 & 36 & 29 & 50
                  \end{vmatrix}
                  =
                  \begin{vmatrix}
                      21 & 16 & 28 \\
                      45 & 38 & 62 \\
                      36 & 29 & 50
                  \end{vmatrix}                                                                                     \\
                   & = 21(38\cdot 50 - 62\cdot 29) + 16(62\cdot 36 - 45\cdot 50) + 28(45\cdot 29 - 36\cdot 38) = 90.
              \end{align*}
    \end{enumerate}
\end{proof}

% exercise 3.10
\begin{exercise}
    \par Tính các định thức sau bằng cách \textit{đưa về dạng tam giác}:
    \begin{enumerate}[label = (\alph*)]
        \item $\begin{vmatrix} 1 & 2 & 3 & \cdots & n \\ -1 & 0 & 3 & \cdots & n \\ -1 & -2 & 0 & \cdots & n \\ \vdots & \vdots & \vdots & \ddots & \vdots \\ -1 & -2 & -3 & \cdots & 0 \end{vmatrix}$,
        \item $\begin{vmatrix} a_{0} & a_{1} & a_{2} & \cdots & a_{n} \\ -x & x & 0 & \cdots & 0 \\ 0 & -x & x & \cdots & 0 \\ \vdots & \vdots & \vdots & \ddots & \vdots \\ 0 & 0 & 0 & \cdots & x \end{vmatrix}$,
        \item $\begin{vmatrix} a_{1} & a_{2} & a_{3} & \cdots & a_{n} \\ -x_{1} & x_{2} & 0 & \cdots & 0 \\ 0 & -x_{2} & x_{3} & \cdots & 0 \\ \vdots & \vdots & \vdots & \ddots & \vdots \\ 0 & 0 & 0 & \cdots & x_{n} \end{vmatrix}$.
    \end{enumerate}
\end{exercise}

\begin{proof}[Lời giải]
    \begin{enumerate}[label = (\alph*)]
        \item $r_{k}:= r_{k} + r_{1}$, $\forall 1 < k\le n$.
              \begin{align*}
                  \begin{vmatrix}
                      1      & 2      & 3      & \cdots & n      \\
                      -1     & 0      & 3      & \cdots & n      \\
                      -1     & -2     & 0      & \cdots & n      \\
                      \vdots & \vdots & \vdots & \ddots & \vdots \\
                      -1     & -2     & -3     & \cdots & 0
                  \end{vmatrix}
                  =
                  \begin{vmatrix}
                      1      & 2      & 3      & \cdots & n      \\
                      0      & 2      & 6      & \cdots & 2n     \\
                      0      & 0      & 3      & \cdots & 2n     \\
                      \vdots & \vdots & \vdots & \ddots & \vdots \\
                      0      & 0      & 0      & \cdots & n
                  \end{vmatrix}
                  = n!
              \end{align*}
        \item Thực hiện lần lượt các biến đổi sơ cấp sau:
              \begin{itemize}
                  \item $c_{0} = \displaystyle\sum^{n}_{k=0}c_{k}$.
                  \item $c_{1} = \displaystyle\sum^{n}_{k=1}c_{k}$.
                  \item $\cdots$
                  \item $c_{n} = \displaystyle\sum^{n}_{k=n}c_{k}$.
              \end{itemize}
              \begingroup{}
              \allowdisplaybreaks{}
              \begin{align*}
                  \begin{vmatrix}
                      a_{0}  & a_{1}  & a_{2}  & \cdots & a_{n}  \\
                      -x     & x      & 0      & \cdots & 0      \\
                      0      & -x     & x      & \cdots & 0      \\
                      \vdots & \vdots & \vdots & \ddots & \vdots \\
                      0      & 0      & 0      & \cdots & x
                  \end{vmatrix}
                   & =
                  \begin{vmatrix}
                      \sum^{n}_{k=0}a_{k} & a_{1}  & a_{2}  & \cdots & a_{n} \\
                      0                   & x      & 0      & \cdots & 0     \\
                      0                   & -x     & x      & \cdots & 0     \\
                      \vdots              & \vdots & \vdots & \ddots & 0     \\
                      0                   & 0      & 0      & \cdots & x
                  \end{vmatrix}
                  =
                  \begin{vmatrix}
                      \sum^{n}_{k=0}a_{k} & \sum^{n}_{k=1}a_{k} & a_{2}  & \cdots & a_{n} \\
                      0                   & x                   & 0      & \cdots & 0     \\
                      0                   & 0                   & x      & \cdots & 0     \\
                      \vdots              & \vdots              & \vdots & \ddots & 0     \\
                      0                   & 0                   & 0      & \cdots & x
                  \end{vmatrix} \\
                   & =
                  \begin{vmatrix}
                      \sum^{n}_{k=0}a_{k} & \sum^{n}_{k=1}a_{k} & \sum^{n}_{k=n}a_{k} & \cdots & a_{n} \\
                      0                   & x                   & 0                   & \cdots & 0     \\
                      0                   & 0                   & x                   & \cdots & 0     \\
                      \vdots              & \vdots              & \vdots              & \ddots & 0     \\
                      0                   & 0                   & 0                   & \cdots & x
                  \end{vmatrix}
                  = \left(\sum^{n}_{k=0}a_{k}\right)x^{n}.
              \end{align*}
              \endgroup{}
        \item Nếu $x_{1} = 0$
              \begin{align*}
                  \begin{vmatrix}
                      a_{1}  & a_{2}  & a_{3}  & \cdots & a_{n}  \\
                      -x_{1} & x_{2}  & 0      & \cdots & 0      \\
                      0      & -x_{2} & x_{3}  & \cdots & 0      \\
                      \vdots & \vdots & \vdots & \ddots & \vdots \\
                      0      & 0      & 0      & \cdots & x_{n}
                  \end{vmatrix}
                   & =
                  \begin{vmatrix}
                      a_{1}  & a_{2}  & a_{3}  & \cdots & a_{n}  \\
                      0      & x_{2}  & 0      & \cdots & 0      \\
                      0      & -x_{2} & x_{3}  & \cdots & 0      \\
                      \vdots & \vdots & \vdots & \ddots & \vdots \\
                      0      & 0      & 0      & \cdots & x_{n}
                  \end{vmatrix}             \\
                   & = a_{1} \begin{vmatrix}
                                 x_{2}  & 0      & \cdots & 0      \\
                                 -x_{2} & x_{3}  & \cdots & 0      \\
                                 \vdots & \vdots & \ddots & \vdots \\
                                 0      & 0      & \cdots & x_{n}
                             \end{vmatrix} = a_{1}x_{2}x_{3}\cdots x_{n}
              \end{align*}

              \par Nếu $x_{k} = 0$
              \begin{align*}
                  \begin{vmatrix}
                      a_{1}  & a_{2}  & a_{3}  & \cdots & a_{n}  \\
                      -x_{1} & x_{2}  & 0      & \cdots & 0      \\
                      0      & -x_{2} & x_{3}  & \cdots & 0      \\
                      \vdots & \vdots & \vdots & \ddots & \vdots \\
                      0      & 0      & 0      & \cdots & x_{n}
                  \end{vmatrix}
                   & = a_{k}\prod^{n}_{i=1,i\ne k} x_{i}
              \end{align*}
              \par Nếu $x_{k} \ne 0, \forall k=\overline{1, n}$.
              \begingroup{}
              \allowdisplaybreaks{}
              \begin{align*}
                  \begin{vmatrix}
                      a_{1}  & a_{2}  & a_{3}  & \cdots & a_{n}  \\
                      -x_{1} & x_{2}  & 0      & \cdots & 0      \\
                      0      & -x_{2} & x_{3}  & \cdots & 0      \\
                      \vdots & \vdots & \vdots & \ddots & \vdots \\
                      0      & 0      & 0      & \cdots & x_{n}
                  \end{vmatrix}
                   & =
                  \frac{1}{x_{1}x_{2}}
                  \begin{vmatrix}
                      a_{1}x_{2}  & a_{2}x_{1}  & a_{3}  & \cdots & a_{n}  \\
                      -x_{1}x_{2} & x_{1}x_{2}  & 0      & \cdots & 0      \\
                      0           & -x_{1}x_{2} & x_{3}  & \cdots & 0      \\
                      \vdots      & \vdots      & \vdots & \ddots & \vdots \\
                      0           & 0           & 0      & \cdots & x_{n}
                  \end{vmatrix}                                                                                                \\
                   & = \frac{1}{x_{1}x_{2}}
                  \begin{vmatrix}
                      a_{1}x_{2} + a_{2}x_{1} & a_{2}x_{1}  & a_{3}  & \cdots & a_{n}  \\
                      0                       & x_{1}x_{2}  & 0      & \cdots & 0      \\
                      -x_{1}x_{2}             & -x_{1}x_{2} & x_{3}  & \cdots & 0      \\
                      \vdots                  & \vdots      & \vdots & \ddots & \vdots \\
                      0                       & 0           & 0      & \cdots & x_{n}
                  \end{vmatrix}                                                                                    \\
                   & = \frac{1}{x^{2}_{1}x^{2}_{2}x^{2}_{3}}
                  \begin{vmatrix}
                      a_{1}x_{2}x_{3} + a_{2}x_{1}x_{3} & a_{2}x_{1}x_{3}  & a_{3}x_{1}x_{2} & \cdots & a_{n}  \\
                      0                                 & x_{1}x_{2}x_{3}  & 0               & \cdots & 0      \\
                      -x_{1}x_{2}x_{3}                  & -x_{1}x_{2}x_{3} & x_{1}x_{2}x_{3} & \cdots & 0      \\
                      \vdots                            & \vdots           & \vdots          & \ddots & \vdots \\
                      0                                 & 0                & 0               & \cdots & x_{n}
                  \end{vmatrix}                                                            \\
                   & = \frac{1}{x^{2}_{1}x^{2}_{2}x^{2}_{3}}
                  \begin{vmatrix}
                      a_{1}x_{2}x_{3} + a_{2}x_{1}x_{3} + a_{3}x_{1}x_{2} & a_{2}x_{1}x_{3}  & a_{3}x_{1}x_{2} & \cdots & a_{n}  \\
                      0                                                   & x_{1}x_{2}x_{3}  & 0               & \cdots & 0      \\
                      0                                                   & -x_{1}x_{2}x_{3} & x_{1}x_{2}x_{3} & \cdots & 0      \\
                      \vdots                                              & \vdots           & \vdots          & \ddots & \vdots \\
                      0                                                   & 0                & 0               & \cdots & x_{n}
                  \end{vmatrix}                                          \\
                   & = \frac{1}{(x_{1}x_{2}\ldots x_{n}){}^{n-1}}(a_{1}x_{2}x_{3}\ldots x_{n} + a_{2}x_{1}x_{3}\ldots x_{n} + \cdots + a_{n}x_{1}x_{2}\ldots x_{n-1}) \\
                   & \times (x_{2}x_{3}\ldots x_{n}) (x_{1}x_{3}\ldots x_{n}) \cdots (x_{1}x_{2}\ldots x_{n-1})                                                       \\
                   & = a_{1}x_{2}x_{3}\ldots x_{n} + a_{2}x_{1}x_{3}\ldots x_{n} + \cdots + a_{n}x_{1}x_{2}\ldots x_{n-1}                                             \\
                   & = \sum^{n}_{k = 1}\left(a_{k}\prod^{n}_{i\ne k}x_{i}\right)
              \end{align*}
              \endgroup{}
    \end{enumerate}
\end{proof}

% exercise 3.11
\begin{exercise}
    Tính định thức của ma trận vuông cỡ $n$ với yếu tố nằm ở hàng $i$ cột $j$ bằng $\abs{i - j}$.
\end{exercise}

\begin{proof}[Lời giải]
    \begingroup{}
    \allowdisplaybreaks{}
    \begin{align*}
        \begin{vmatrix}
            0      & 1      & 2      & \cdots & n - 1  \\
            1      & 0      & 1      & \cdots & n - 2  \\
            2      & 1      & 0      & \cdots & n - 3  \\
            \vdots & \vdots & \vdots & \ddots & \vdots \\
            n - 1  & n - 2  & n - 3  & \cdots & 0
        \end{vmatrix}
         & =
        \begin{vmatrix}
            0      & 1      & 2      & \cdots & n - 1  \\
            1      & -1     & -1     & \cdots & -1     \\
            1      & 1      & -1     & \cdots & -1     \\
            \vdots & \vdots & \vdots & \ddots & \vdots \\
            1      & 1      & 1      & \cdots & -1
        \end{vmatrix} (r_{k}:= r_{k} - r_{k - 1}) \\
         & =
        \begin{vmatrix}
            n - 1  & 1      & 2      & \cdots & n - 1  \\
            0      & -1     & -1     & \cdots & -1     \\
            0      & 1      & -1     & \cdots & -1     \\
            \vdots & \vdots & \vdots & \ddots & \vdots \\
            0      & 1      & 1      & \cdots & -1
        \end{vmatrix} (c_{1}:= c_{1} + c_{n}) \\
         & =
        \begin{vmatrix}
            n - 1  & 1      & 2      & \cdots & n - 1  \\
            0      & -1     & -1     & \cdots & -1     \\
            0      & 0      & -2     & \cdots & -2     \\
            \vdots & \vdots & \vdots & \ddots & \vdots \\
            0      & 0      & 0      & \cdots & -2
        \end{vmatrix} (r_{k}:= r_{k} + r_{k-1}) \\
         & = (-2){}^{n-1}(n - 1).
    \end{align*}
    \endgroup{}
\end{proof}

% exercise 3.12
\begin{exercise}
    \par Tính các định thức sau đây bằng \textit{phương pháp rút ra các nhân tử tuyến tính}:
    \begin{enumerate}[label = (\alph*)]
        \item $\begin{vmatrix} 1 & 2 & 3 & \cdots & n \\ 1 & x + 1 & 3 & \cdots & n \\ 1 & 2 & x + 1 & \cdots & n \\ \vdots & \vdots & \vdots & \ddots & \vdots \\ 1 & 2 & 3 & \cdots & x + 1 \end{vmatrix}$,
        \item $\begin{vmatrix} 1 + x & 1 & 1 & 1 \\ 1 & 1 - x & 1 & 1 \\ 1 & 1 & 1 + y & 1 \\ 1 & 1 & 1 & 1 - y \end{vmatrix}$.
    \end{enumerate}
\end{exercise}

\begin{proof}[Lời giải]
    \begin{enumerate}[label = (\alph*)]
        \item
              \begin{align*}
                  \begin{vmatrix}
                      1      & 2      & 3      & \cdots & n      \\
                      1      & x + 1  & 3      & \cdots & n      \\
                      1      & 2      & x + 1  & \cdots & n      \\
                      \vdots & \vdots & \vdots & \ddots & \vdots \\
                      1      & 2      & 3      & \cdots & x + 1
                  \end{vmatrix}
                   & =
                  \begin{vmatrix}
                      1      & 0      & 0      & \cdots & 0           \\
                      1      & x - 1  & 0      & \cdots & 0           \\
                      1      & 0      & x - 2  & \cdots & 0           \\
                      \vdots & \vdots & \vdots & \ddots & \vdots      \\
                      1      & 0      & 0      & \cdots & x - (n - 1) \\
                  \end{vmatrix} (c_{k}:= c_{k} - k\times c_{1}) \\
                   & =
                  \begin{vmatrix}
                      1      & 0      & 0      & \cdots & 0           \\
                      0      & x - 1  & 0      & \cdots & 0           \\
                      0      & 0      & x - 2  & \cdots & 0           \\
                      \vdots & \vdots & \vdots & \ddots & \vdots      \\
                      0      & 0      & 0      & \cdots & x - (n - 1) \\
                  \end{vmatrix} (r_{k}:= r_{k} - r_{1}) \\
                   & = (x - 1)(x - 2)\cdots (x - n + 1).
              \end{align*}
        \item Đặt $\varepsilon_{1} = \begin{pmatrix}1 \\ 0 \\ 0 \\ 0\end{pmatrix}$, $\varepsilon_{2} = \begin{pmatrix}0 \\ 1 \\ 0 \\ 0\end{pmatrix}$, $\varepsilon_{3} = \begin{pmatrix}0 \\ 0 \\ 1 \\ 0\end{pmatrix}$, $\varepsilon_{4} = \begin{pmatrix}0 \\ 0 \\ 0 \\ 1\end{pmatrix}$, $\varepsilon = \begin{pmatrix}1 \\ 1 \\ 1 \\ 1\end{pmatrix}$.
              \begin{align*}
                  \begin{vmatrix}
                      1 + x & 1     & 1     & 1     \\
                      1     & 1 - x & 1     & 1     \\
                      1     & 1     & 1 + y & 1     \\
                      1     & 1     & 1     & 1 - y \\
                  \end{vmatrix}
                   & = \det(\varepsilon + x\varepsilon_{1}, \varepsilon - x\varepsilon_{2}, \varepsilon + y\varepsilon_{3}, \varepsilon - y\varepsilon_{4})                       \\
                   & = \det(\varepsilon, -x\varepsilon_{2}, y\varepsilon_{3}, -y\varepsilon_{4}) + \det(x\varepsilon_{1}, \varepsilon, y\varepsilon_{3}, -y\varepsilon_{4})       \\
                   & \quad + \det(x\varepsilon_{1}, -x\varepsilon_{2}, \varepsilon, -y\varepsilon_{4}) + \det(x\varepsilon_{1}, -x\varepsilon_{2}, y\varepsilon_{3}, \varepsilon) \\
                   & \quad + \det(x\varepsilon_{1}, -x\varepsilon_{2}, y\varepsilon_{3}, -y\varepsilon_{4})                                                                       \\
                   & = xy^{2} - xy^{2} + x^{2}y - x^{2}y + x^{2}y^{2}                                                                                                             \\
                   & = x^{2}y^{2}.
              \end{align*}
    \end{enumerate}
\end{proof}

% exercise 3.13
\begin{exercise}
    \par Tính các định thức sau đây bằng cách \textit{sử dụng các quan hệ hồi quy}:
    \begin{enumerate}[label = (\alph*)]
        \item $\begin{vmatrix}
                      a_{1}b_{1} & a_{1}b_{2} & a_{1}b_{3} & \cdots & a_{1}b_{n} \\
                      a_{1}b_{2} & a_{2}b_{2} & a_{2}b_{3} & \cdots & a_{2}b_{n} \\
                      a_{1}b_{3} & a_{2}b_{3} & a_{3}b_{3} & \cdots & a_{3}b_{n} \\
                      \vdots     & \vdots     & \vdots     & \ddots & \vdots     \\
                      a_{1}b_{n} & a_{2}b_{n} & a_{3}b_{n} & \cdots & a_{n}b_{n}
                  \end{vmatrix}$
        \item $\begin{vmatrix}
                      a_{0}  & a_{1}  & a_{2}  & \cdots & a_{n}  \\
                      -y_{1} & x_{1}  & 0      & \cdots & 0      \\
                      0      & -y_{2} & x_{2}  & \cdots & 0      \\
                      \vdots & \vdots & \vdots & \ddots & \vdots \\
                      0      & 0      & 0      & \cdots & x_{n}
                  \end{vmatrix}$
    \end{enumerate}
\end{exercise}

\begin{proof}[Lời giải]
    \begin{enumerate}[label = (\alph*)]
        \item
              \begin{align*}
                  \begin{vmatrix}
                      a_{1}b_{1} & a_{1}b_{2} & a_{1}b_{3} & \cdots & a_{1}b_{n} \\
                      a_{1}b_{2} & a_{2}b_{2} & a_{2}b_{3} & \cdots & a_{2}b_{n} \\
                      a_{1}b_{3} & a_{2}b_{3} & a_{3}b_{3} & \cdots & a_{3}b_{n} \\
                      \vdots     & \vdots     & \vdots     & \ddots & \vdots     \\
                      a_{1}b_{n} & a_{2}b_{n} & a_{3}b_{n} & \cdots & a_{n}b_{n}
                  \end{vmatrix}
                   & =
                  a_{1}b_{n}
                  \begin{vmatrix}
                      b_{1}  & a_{1}b_{2} & a_{1}b_{3} & \cdots & a_{1}b_{n} \\
                      b_{2}  & a_{2}b_{2} & a_{2}b_{3} & \cdots & a_{2}b_{n} \\
                      b_{3}  & a_{2}b_{3} & a_{3}b_{3} & \cdots & a_{3}b_{n} \\
                      \vdots & \vdots     & \vdots     & \ddots & \vdots     \\
                      1      & a_{2}      & a_{3}      & \cdots & a_{n}
                  \end{vmatrix}                                            \\
                   & =
                  a_{1}b_{n}
                  \begin{vmatrix}
                      0      & a_{1}b_{2} - a_{2}b_{1} & a_{1}b_{3} - a_{3}b_{1} & \cdots & a_{1}b_{n} - a_{n}b_{1} \\
                      0      & 0                       & a_{2}b_{3} - a_{3}b_{2} & \cdots & a_{2}b_{n} - a_{n}b_{2} \\
                      0      & 0                       & 0                       & \cdots & a_{3}b_{n} - a_{n}b_{3} \\
                      \vdots & \vdots                  & \vdots                  & \ddots & \vdots                  \\
                      1      & a_{2}                   & a_{3}                   & \cdots & a_{n}
                  \end{vmatrix} (r_{k}:= r_{k} - b_{k}r_{n})     \\
                   & =
                  (-1){}^{n-1}a_{1}b_{n}
                  \begin{vmatrix}
                      1      & a_{2}                   & a_{3}                   & \cdots & a_{n}                       \\
                      0      & a_{1}b_{2} - a_{2}b_{1} & a_{1}b_{3} - a_{3}b_{1} & \cdots & a_{1}b_{n} - a_{n}b_{1}     \\
                      0      & 0                       & a_{2}b_{3} - a_{3}b_{2} & \cdots & a_{2}b_{n} - a_{n}b_{2}     \\
                      0      & 0                       & 0                       & \cdots & a_{3}b_{n} - a_{n}b_{3}     \\
                      \vdots & \vdots                  & \vdots                  & \ddots & \vdots                      \\
                      0      & 0                       & 0                       & \cdots & a_{n-1}b_{n} - a_{n}b_{n-1}
                  \end{vmatrix} \\
                   & =
                  (-1){}^{n-1}a_{1}b_{n}(a_{1}b_{2} - a_{2}b_{1})(a_{2}b_{3} - a_{3}b_{2})\cdots (a_{n-1}b_{n} - a_{n}b_{n-1}).
              \end{align*}
        \item
              \begin{align*}
                  \begin{vmatrix}
                      a_{0}  & a_{1}  & a_{2}  & \cdots & a_{n}  \\
                      -y_{1} & x_{1}  & 0      & \cdots & 0      \\
                      0      & -y_{2} & x_{2}  & \cdots & 0      \\
                      \vdots & \vdots & \vdots & \ddots & \vdots \\
                      0      & 0      & 0      & \cdots & x_{n}
                  \end{vmatrix}
                   & =
                  \begin{vmatrix}
                      a_{0}  & a_{1}  & a_{2}  & \cdots & a_{n}  \\
                      0      & x_{1}  & 0      & \cdots & 0      \\
                      0      & -y_{2} & x_{2}  & \cdots & 0      \\
                      \vdots & \vdots & \vdots & \ddots & \vdots \\
                      0      & 0      & 0      & \cdots & x_{n}
                  \end{vmatrix}
                  +
                  \begin{vmatrix}
                      0      & a_{1}  & a_{2}  & \cdots & a_{n}  \\
                      -y_{1} & x_{1}  & 0      & \cdots & 0      \\
                      0      & -y_{2} & x_{2}  & \cdots & 0      \\
                      \vdots & \vdots & \vdots & \ddots & \vdots \\
                      0      & 0      & 0      & \cdots & x_{n}
                  \end{vmatrix}                                                         \\
                   & =
                  a_{0}x_{1}x_{2}\cdots x_{n}
                  +
                  \begin{vmatrix}
                      y_{1}  & -x_{1} & 0      & \cdots & 0      \\
                      0      & a_{1}  & a_{2}  & \cdots & a_{n}  \\
                      0      & -y_{2} & x_{2}  & \cdots & 0      \\
                      \vdots & \vdots & \vdots & \ddots & \vdots \\
                      0      & 0      & 0      & \cdots & x_{n}
                  \end{vmatrix}                                                         \\
                   & =
                  a_{0}x_{1}x_{2}\cdots x_{n}
                  +
                  y_{1}
                  \begin{vmatrix}
                      a_{1}  & a_{2}  & \cdots & a_{n}  \\
                      -y_{2} & x_{2}  & \cdots & 0      \\
                      \vdots & \vdots & \ddots & \vdots \\
                      0      & 0      & \cdots & x_{n}
                  \end{vmatrix} \text{(hệ thức truy hồi)}                                                            \\
                   & = a_{0}x_{1}x_{2}\cdots x_{n}
                  + y_{1} (a_{1}x_{2}\cdots x_{n})
                  + \cdots
                  + y_{1}y_{2}\cdots y_{n} (a_{n})                                                                   \\
                   & = \sum^{n}_{k=0} a_{k}\left(\prod^{k}_{i=1}y_{i}\right)\left(\prod^{n-1}_{i=k}x_{i + 1}\right).
              \end{align*}
    \end{enumerate}
\end{proof}

% exercise 3.14
\begin{exercise}
    Tính các định thức sau đây bằng cách biểu diễn chúng thành tổng của các định thức nào đó:
    \begin{enumerate}[label = (\alph*)]
        \item $\begin{vmatrix}
                      x + a_{1} & a_{2}     & \cdots & a_{n}     \\
                      a_{1}     & x + a_{2} & \cdots & a_{n}     \\
                      \vdots    & \vdots    & \ddots & \vdots    \\
                      a_{1}     & a_{2}     & \cdots & x + a_{n}
                  \end{vmatrix}$,
        \item $\begin{vmatrix}
                      x_{1}  & a_{2}  & \cdots & a_{n}  \\
                      a_{1}  & x_{2}  & \cdots & a_{n}  \\
                      \vdots & \vdots & \ddots & \vdots \\
                      a_{1}  & a_{2}  & \cdots & x_{n}
                  \end{vmatrix}$.
    \end{enumerate}
\end{exercise}

\begin{proof}[Lời giải]
    \par Đặt $\varepsilon_{1} = (1, 0, \ldots, 0)$, $\varepsilon_{2} = (0, 1, \ldots, 0)$, \ldots $\varepsilon_{n} = (0, 0, \ldots, 1)$, $\varepsilon = \displaystyle\sum^{n}_{i}\varepsilon_{i}$.
    \begin{enumerate}[label = (\alph*)]
        \item
              \begin{align*}
                  \begin{vmatrix}
                      x + a_{1} & a_{2}     & \cdots & a_{n}     \\
                      a_{1}     & x + a_{2} & \cdots & a_{n}     \\
                      \vdots    & \vdots    & \ddots & \vdots    \\
                      a_{1}     & a_{2}     & \cdots & x + a_{n}
                  \end{vmatrix}
                   & = \det(x\varepsilon_{1} + a_{1}\varepsilon, x\varepsilon_{2} + a_{2}\varepsilon, \ldots, x\varepsilon_{n} + a_{n}\varepsilon)             \\
                   & = \det(x\varepsilon_{1}, x\varepsilon_{2}, \ldots, x\varepsilon_{n})                                                                      \\
                   & + \det(a_{1}\varepsilon, x\varepsilon_{2}, \ldots, x\varepsilon_{n}) + \det(x\varepsilon_{1}, a_{2}\varepsilon, \ldots, x\varepsilon_{n}) \\
                   & + \cdots + \det(x\varepsilon_{1}, x\varepsilon_{2}, \ldots, a_{n}\varepsilon)                                                             \\
                   & = x^{n} + x^{n-1}\sum^{n}_{i=1}a_{i}.
              \end{align*}
        \item
              \begin{align*}
                  \begin{vmatrix}
                      x_{1}  & a_{2}  & \cdots & a_{n}  \\
                      a_{1}  & x_{2}  & \cdots & a_{n}  \\
                      \vdots & \vdots & \ddots & \vdots \\
                      a_{1}  & a_{2}  & \cdots & x_{n}
                  \end{vmatrix}
                   & =
                  \begin{vmatrix}
                      (x_{1} - a_{1}) + a_{1} & a_{2}                   & \cdots & a_{n}                   \\
                      a_{1}                   & (x_{2} - a_{2}) + a_{2} & \cdots & a_{n}                   \\
                      \vdots                  & \vdots                  & \ddots & \vdots                  \\
                      a_{1}                   & a_{2}                   & \cdots & (x_{n} - a_{n}) + a_{n}
                  \end{vmatrix}                                                                                       \\
                   & = \det((x_{1} - a_{1})\varepsilon_{1} + a_{1}\varepsilon, (x_{2} - a_{2})\varepsilon_{2} + a_{2}\varepsilon, \ldots, (x_{n} - a_{n})\varepsilon_{n} + a_{n}\varepsilon) \\
                   & = (x_{1} - a_{1})\cdots (x_{n} - a_{n})\det(\varepsilon_{1}, \ldots, \varepsilon_{n})                                                                                   \\
                   & + \sum^{n}_{i = 1}a_{i}\left(\prod^{n}_{\stackrel{j=1}{j\ne i}}(x_{j} - a_{j})\right)\det(\ldots, \varepsilon_{i-1}, \varepsilon, \varepsilon_{i+1}, \ldots)            \\
                   & = \prod^{n}_{i=1}(x_{i} - a_{i}) + \sum^{n}_{i=1}a_{i}\left(\prod^{n}_{\stackrel{j=1}{j\ne i}}(x_{j} - a_{j})\right).
              \end{align*}
    \end{enumerate}
\end{proof}

\par Kí hiệu định thức Vandermonde cỡ $n$ với $n$ biến:

\[
    D_{n} = D_{n}(x_{1}, \ldots, x_{n}) =
    \begin{vmatrix}
        1      & x_{1}  & x_{1}^{2} & \cdots & x_{1}^{n-1} \\
        1      & x_{2}  & x_{2}^{2} & \cdots & x_{2}^{n-1} \\
        \vdots & \vdots & \vdots    & \ddots & \vdots      \\
        1      & x_{n}  & x_{n}^{2} & \cdots & x_{1}^{n-1}
    \end{vmatrix}.
\]

\par Kí hiệu đa thức đối xứng sơ cấp:

\[
    e_{k}(x_{1}, \ldots, x_{n}) = \sum_{1\le i_{1} < \cdots < i_{k} \le n}\left(\prod^{k}_{j=1}x_{i_{j}}\right).
\]

\par Ví dụ
\begin{gather*}
    e_{1}(x_{1}, x_{2}, x_{3}) = x_{1} + x_{2} + x_{3} \\
    e_{2}(x_{1}, x_{2}, x_{3}, x_{4}) = x_{1}x_{2} + x_{1}x_{3} + x_{1}x_{4} + x_{2}x_{3} + x_{3}x_{4} + x_{2}x_{4} \\
    e_{3}(x_{1}, x_{2}, x_{3}) = x_{1}x_{2}x_{3}.
\end{gather*}

\par Kí hiệu đa thức đối xứng thuần nhất đầy đủ:
\[
    h_{k}(x_{1}, \ldots, x_{n}) = \sum_{i_{1}+\cdots+i_{n}=k}\prod^{n}_{j=1}x^{i_{j}}_{j}
\]

\par Ví dụ
\begin{gather*}
    h_{1}(x_{1}, x_{2}, x_{3}) = x_{1} + x_{2} + x_{3} \\
    h_{2}(x_{1}, x_{2}, x_{3}) = x_{1}^{2} + x_{2}^{2} + x_{3}^{2} + x_{1}x_{2} + x_{2}x_{3} + x_{1}x_{3} \\
    h_{3}(x_{1}, x_{2}, x_{3}) = x_{1}^{3} + x_{2}^{3} + x_{3}^{3} + x_{1}^{2}x_{2} + x_{1}^{2}x_{3} + x_{2}^{2}x_{1} + x_{2}^{2}x_{3} + x_{3}^{2}x_{1} + x_{3}^{2}x_{2} + x_{1}x_{2}x_{3}.
\end{gather*}

\par Tính các định thức sau đây:

% exercise 3.15
\begin{exercise}
    $\begin{vmatrix}
            a_{1}  & x_{1}  & x_{1}^{2} & \cdots & x_{1}^{n-1} \\
            a_{2}  & x_{2}  & x_{2}^{2} & \cdots & x_{2}^{n-1} \\
            \vdots & \vdots & \vdots    & \ddots & \vdots      \\
            a_{n}  & x_{n}  & x_{n}^{2} & \cdots & x_{n}^{n-1}
        \end{vmatrix}$.
\end{exercise}

\begin{proof}[Lời giải]
    \par Khai triển Laplace theo cột thứ nhất:
    \begin{align*}
        \begin{vmatrix}
            a_{1}  & x_{1}  & x_{1}^{2} & \cdots & x_{1}^{n-1} \\
            a_{2}  & x_{2}  & x_{2}^{2} & \cdots & x_{2}^{n-1} \\
            \vdots & \vdots & \vdots    & \ddots & \vdots      \\
            a_{n}  & x_{n}  & x_{n}^{2} & \cdots & x_{n}^{n-1}
        \end{vmatrix}
         & =
        \sum^{n}_{i=1}(-1){}^{1+i}a_{i}
        \begin{vmatrix}
            \vdots  & \vdots      & \vdots      & \ddots & \vdots        \\
            x_{i-1} & x_{i-1}^{2} & x_{i-1}^{3} & \cdots & x_{i-1}^{n-1} \\
            x_{i+1} & x_{i+1}^{2} & x_{i+1}^{3} & \cdots & x_{i+1}^{n-1} \\
            \vdots  & \vdots      & \vdots      & \ddots & \vdots
        \end{vmatrix} \\
         & =
        \sum^{n}_{i=1}(-1){}^{1+i}a_{i}\prod^{n}_{j\ne i}
        \begin{vmatrix}
            \vdots & \vdots  & \vdots      & \ddots & \vdots        \\
            1      & x_{i-1} & x_{i-1}^{2} & \cdots & x_{i-1}^{n-2} \\
            1      & x_{i+1} & x_{i+1}^{2} & \cdots & x_{i+1}^{n-2} \\
            \vdots & \vdots  & \vdots      & \ddots & \vdots
        \end{vmatrix}      \\
         & =
        \sum^{n}_{i=1}(-1){}^{1+i}a_{i}\left(\prod^{n}_{j\ne i}x_{j}\right)
        D_{n-1}(\ldots, x_{i-1}, x_{i+1}, \ldots).
    \end{align*}
\end{proof}

% exercise 3.16
\begin{exercise}\label{chapter3:vandermonde-and-symmetric-polynomials}
    \begin{enumerate}[label = (\alph*)]
        \item $D^{(1)}_{n} = \begin{vmatrix}
                      1      & x_{1}^{2} & x_{1}^{3} & \cdots & x_{1}^{n} \\
                      1      & x_{2}^{2} & x_{2}^{3} & \cdots & x_{2}^{n} \\
                      \vdots & \vdots    & \vdots    & \ddots & \vdots    \\
                      1      & x_{n}^{2} & x_{n}^{3} & \cdots & x_{n}^{n}
                  \end{vmatrix}$,
        \item $D^{(s)}_{n} = \begin{vmatrix}
                      1      & x_{1}  & x_{1}^{2} & \cdots & x_{1}^{s-1} & x_{1}^{s+1} & \cdots & x_{1}^{n} \\
                      1      & x_{2}  & x_{2}^{2} & \cdots & x_{2}^{s-1} & x_{2}^{s+1} & \cdots & x_{2}^{n} \\
                      \vdots & \vdots & \vdots    & \ddots & \vdots      & \vdots      & \ddots & \vdots    \\
                      1      & x_{n}  & x_{n}^{2} & \cdots & x_{n}^{s-1} & x_{n}^{s+1} & \cdots & x_{n}^{n}
                  \end{vmatrix}$.
    \end{enumerate}
\end{exercise}

\begin{proof}[Lời giải]
    \begin{enumerate}[label = (\alph*)]
        \item
              \begin{align*}
                  \begin{vmatrix}
                      1      & x_{1}^{2} & x_{1}^{3} & \cdots & x_{1}^{n} \\
                      1      & x_{2}^{2} & x_{2}^{3} & \cdots & x_{2}^{n} \\
                      \vdots & \vdots    & \vdots    & \ddots & \vdots    \\
                      1      & x_{n}^{2} & x_{n}^{3} & \cdots & x_{n}^{n}
                  \end{vmatrix}
                   & =
                  \begin{vmatrix}
                      1      & x_{1}^{2} & x_{1}^{3} & \cdots & x_{1}^{n-1}(x_{1} - x_{n}) \\
                      1      & x_{2}^{2} & x_{2}^{3} & \cdots & x_{2}^{n-1}(x_{2} - x_{n}) \\
                      \vdots & \vdots    & \vdots    & \ddots & \vdots                     \\
                      1      & x_{n}^{2} & x_{n}^{3} & \cdots & 0
                  \end{vmatrix}\quad(c_{n}:= c_{n} - x_{n}c_{n-1})                                     \\
                   & = \vdots                                                                                              \\
                   & =
                  \begin{vmatrix}
                      1      & x_{1}^{2} & x_{1}^{2}(x_{1} - x_{n}) & \cdots & x_{1}^{n-1}(x_{1} - x_{n}) \\
                      1      & x_{2}^{2} & x_{2}^{2}(x_{2} - x_{n}) & \cdots & x_{2}^{n-1}(x_{2} - x_{n}) \\
                      \vdots & \vdots    & \vdots                   & \ddots & \vdots                     \\
                      1      & x_{n}^{2} & 0                        & \cdots & 0
                  \end{vmatrix}\quad(c_{3}:=c_{3} - x_{n}c_{2})                      \\
                   & =
                  \begin{vmatrix}
                      1      & (x_{1} + x_{n})(x_{1} - x_{n}) & x_{1}^{2}(x_{1} - x_{n}) & \cdots & x_{1}^{n-1}(x_{1} - x_{n}) \\
                      1      & (x_{2} + x_{n})(x_{2} - x_{n}) & x_{2}^{2}(x_{2} - x_{n}) & \cdots & x_{2}^{n-1}(x_{2} - x_{n}) \\
                      \vdots & \vdots                         & \vdots                   & \ddots & \vdots                     \\
                      1      & 0                              & 0                        & \cdots & 0
                  \end{vmatrix}\quad(c_{2}:=c_{2} - x_{n}^{2}c_{1}) \\
              \end{align*}
              \par Khai triển Laplace theo cột thứ nhất
              \begin{align*}
                   & = (-1){}^{n+1}(x_{1} - x_{n})(x_{2} - x_{n})\cdots (x_{n-1} - x_{n})
                  \begin{vmatrix}
                      x_{1} + x_{n}   & x_{1}^{2}   & \cdots & x_{1}^{n-1}   \\
                      x_{2} + x_{n}   & x_{2}^{2}   & \cdots & x_{2}^{n-1}   \\
                      \vdots          & \vdots      & \ddots & \vdots        \\
                      x_{n-1} + x_{n} & x_{n-1}^{2} & \cdots & x_{n-1}^{n-1}
                  \end{vmatrix}                  \\
                   & = \prod^{n-1}_{i=1}(x_{n} - x_{i})\left(
                  \begin{vmatrix}
                          x_{1}   & x_{1}^{2}   & \cdots & x_{1}^{n-1}   \\
                          x_{2}   & x_{2}^{2}   & \cdots & x_{2}^{n-1}   \\
                          \vdots  & \vdots      & \ddots & \vdots        \\
                          x_{n-1} & x_{n-1}^{2} & \cdots & x_{n-1}^{n-1}
                      \end{vmatrix}
                  +
                  \begin{vmatrix}
                          x_{n}  & x_{1}^{2}   & \cdots & x_{1}^{n-1}   \\
                          x_{n}  & x_{2}^{2}   & \cdots & x_{2}^{n-1}   \\
                          \vdots & \vdots      & \ddots & \vdots        \\
                          x_{n}  & x_{n-1}^{2} & \cdots & x_{n-1}^{n-1}
                      \end{vmatrix}
                  \right)
              \end{align*}
              \begingroup{}
              \allowdisplaybreaks{}
              \begin{align*}
                  D^{(1)}_{n} & = (x_{n} - x_{1})\cdots(x_{n} - x_{n-1})\left( x_{1}x_{2}\cdots x_{n-1} D_{n-1} + x_{n}D^{(1)}_{n-1}\right)                                                                                        \\
                              & = x_{n}(x_{n} - x_{1})\cdots(x_{n} - x_{n-1})D^{(1)}_{n-1} + (x_{n} - x_{1})\cdots (x_{n} - x_{n-1}) x_{1}x_{2}\cdots x_{n-1}D_{n-1}                                                               \\
                              & = x_{n}\prod^{n-1}_{i=1}(x_{n} - x_{i}) \cdot D^{(1)}_{n-1} + \prod^{n-1}_{i=1}x_{i} \cdot D_{n}                                                                                                   \\
                              & = x_{n}\prod^{n-1}_{i=1}(x_{n} - x_{i}) \cdot \left(x_{n-1}\prod^{n-2}_{i=1}(x_{n-1} - x_{i})\cdot D^{(1)}_{n-2} + \prod^{n-2}_{i=1}x_{i}\cdot D_{n-1}\right) + \prod^{n-1}_{i=1}x_{i} \cdot D_{n} \\
                              & = x_{n}x_{n-1}\prod^{n-1}_{i=1}(x_{n} - x_{i})\prod^{n-2}_{i=1}(x_{n-1} - x_{i})\cdot D^{(1)}_{n-2} + \prod^{n}_{1\le i\ne n-1}x_{i}\cdot D_{n} + \prod^{n}_{1\le i\ne n}x_{i}\cdot D_{n}          \\
                              & = \cdots                                                                                                                                                                                           \\
                              & = D_{n}x_{1}x_{2}\cdots x_{n}\sum^{n}_{i=1}\frac{1}{x_{i}}                                                                                                                                         \\
                              & = D_{n}\sum^{n}_{i=1}\left(\prod^{n}_{1\le j\ne i}x_{j}\right)                                                                                                                                     \\
                              & = D_{n}e_{n-1}(x_{1},\ldots, x_{n}).
              \end{align*}
              \endgroup{}
        \item
              \par Ta xét trường hợp đặc biệt
              \begin{align*}
                   & D^{(n-1)}_{n} =
                  \begin{vmatrix}
                      1      & x_{1}  & \cdots & x_{1}^{n-2} & x_{1}^{n} \\
                      1      & x_{2}  & \cdots & x_{2}^{n-2} & x_{2}^{n} \\
                      \vdots & \vdots & \ddots & \cdots      & \vdots    \\
                      1      & x_{n}  & \cdots & x_{n}^{n-2} & x_{n}^{n}
                  \end{vmatrix}                                                                                                                                                                                 \\
                   & =
                  \begin{vmatrix}
                      1      & x_{1} - x_{n} & \cdots & x_{1}^{n-3}(x_{1} - x_{n}) & x_{1}^{n-2}(x_{1} + x_{n})(x_{1} - x_{n}) \\
                      1      & x_{2} - x_{n} & \cdots & x_{2}^{n-3}(x_{2} - x_{n}) & x_{2}^{n-2}(x_{2} + x_{n})(x_{2} - x_{n}) \\
                      \vdots & \vdots        & \ddots & \vdots                     & \vdots                                    \\
                      1      & 0             & \cdots & 0                          & 0
                  \end{vmatrix}                                                                                                                           \\
                   & = \prod^{n}_{1\le i\ne n}(x_{n} - x_{i})
                  \begin{vmatrix}
                      1      & x_{1}   & \cdots & x_{1}^{n-3}   & x_{1}^{n-1} + x_{1}^{n-2}x_{n}     \\
                      1      & x_{2}   & \cdots & x_{2}^{n-3}   & x_{2}^{n-1} + x_{2}^{n-2}x_{n}     \\
                      \vdots & \vdots  & \ddots & \cdots        & \vdots                             \\
                      1      & x_{n-1} & \cdots & x_{n-1}^{n-3} & x_{n-1}^{n-1} + x_{n-1}^{n-2}x_{n}
                  \end{vmatrix}                                                                                                                                                     \\
                   & = \prod^{n}_{1\le i\ne n}(x_{n} - x_{i})\cdot D^{(n-2)}_{n-1} + x_{n}\prod^{n}_{1\le i\ne n}(x_{n} - x_{i})\cdot D_{n-1}                                                                                                        \\
                   & = \prod^{n}_{1\le i\ne n}(x_{n} - x_{i})\cdot D^{(n-2)}_{n-1} + x_{n}D_{n}                                                                                                                                                      \\
                   & = \prod^{n}_{1\le i\ne n}(x_{n} - x_{i})\prod^{n-1}_{1\le i\ne n-1}(x_{n-1} - x_{i})\cdot D^{(n-3)}_{n-2} + \prod^{n}_{1\le i\ne n}(x_{n} - x_{i})\prod^{n-1}_{1\le i\ne n-1}(x_{n-1} - x_{i})\cdot x_{n-1}D_{n-2} + x_{n}D_{n} \\
                   & = \prod^{n}_{1\le i\ne n}(x_{n} - x_{i})\prod^{n-1}_{1\le i\ne n-1}(x_{n-1} - x_{i})\cdot D^{(n-3)}_{n-2} + (x_{n-1} + x_{n})D_{n}                                                                                              \\
                   & = \prod^{n}_{1\le i\ne n}(x_{n} - x_{i})\cdots \prod^{3}_{1\le i\ne 3}(x_{3} - x_{i})\cdot D^{(1)}_{2} + (x_{3} + \cdots + x_{n})D_{n}                                                                                          \\
                   & = \prod^{n}_{1\le i\ne n}(x_{n} - x_{i})\cdots \prod^{3}_{1\le i\ne 3}(x_{3} - x_{i})(x_{2} - x_{1})(x_{1} + x_{2}) + (x_{3} + \cdots + x_{n})D_{n}                                                                             \\
                   & = D_{n}\sum^{n}_{i=1}x_{i}                                                                                                                                                                                                      \\
                   & = D_{n}e_{1}(x_{1},\ldots, x_{n}).
              \end{align*}

              \par Ta sẽ chứng minh $D_{n}^{(s)} = D_{n}e_{n-s}(x_{1},\ldots, x_{n})\ \forall n, 0\le s\le n$.
              \par Khẳng định trên đúng với $n = 1, 2, 3$.
              \par Nếu khẳng định này đúng với $n - 1, \forall\ 0\le s\le n-1$, ta cần chứng minh khẳng định vẫn đúng với $n, \forall\ 0\le s\le n$.
              \begin{align*}
                   & \phantom{=}\begin{vmatrix}
                                    1      & x_{1}  & x_{1}^{2} & \cdots & x_{1}^{s-1} & x_{1}^{s+1} & \cdots & x_{1}^{n} \\
                                    1      & x_{2}  & x_{2}^{2} & \cdots & x_{2}^{s-1} & x_{2}^{s+1} & \cdots & x_{2}^{n} \\
                                    \vdots & \vdots & \vdots    & \ddots & \vdots      & \vdots      & \ddots & \vdots    \\
                                    1      & x_{n}  & x_{n}^{2} & \cdots & x_{n}^{s-1} & x_{n}^{s+1} & \cdots & x_{n}^{n}
                                \end{vmatrix}                                          \\
                   & =
                  \begin{vmatrix}
                      1      & x_{1}  & x_{1}^{2} & \cdots & x_{1}^{s-1} & x_{1}^{s+1} & \cdots & x_{1}^{n-1}(x_{1} - x_{n}) \\
                      1      & x_{2}  & x_{2}^{2} & \cdots & x_{2}^{s-1} & x_{2}^{s+1} & \cdots & x_{2}^{n-1}(x_{2} - x_{n}) \\
                      \vdots & \vdots & \vdots    & \ddots & \vdots      & \vdots      & \ddots & \vdots                     \\
                      1      & x_{n}  & x_{n}^{2} & \cdots & x_{n}^{s-1} & x_{n}^{s+1} & \cdots & 0
                  \end{vmatrix}\quad(c_{s}:=c_{s} - x_{n}c_{s-1})                                       \\
                   & = \cdots                                                                                                                                  \\
                   & =
                  \begin{vmatrix}
                      1      & x_{1}  & x_{1}^{2} & \cdots & x_{1}^{s-1} & x_{1}^{s-1}(x_{1}^{2} - x_{n}^{2}) & \cdots & x_{1}^{n-1}(x_{1} - x_{n}) \\
                      1      & x_{2}  & x_{2}^{2} & \cdots & x_{2}^{s-1} & x_{2}^{s-1}(x_{2}^{2} - x_{n}^{2}) & \cdots & x_{2}^{n-1}(x_{2} - x_{n}) \\
                      \vdots & \vdots & \vdots    & \ddots & \vdots      & \vdots                             & \ddots & \vdots                     \\
                      1      & x_{n}  & x_{n}^{2} & \cdots & x_{n}^{s-1} & 0                                  & \cdots & 0
                  \end{vmatrix}\quad(c_{s}:=c_{s} - x_{n}c_{s-1})                \\
                   & =
                  \begin{vmatrix}
                      1      & x_{1}  & x_{1}^{2} & \cdots & x_{1}^{s-2}(x_{1} - x_{n}) & x_{1}^{s-1}(x_{1}^{2} - x_{n}^{2}) & \cdots & x_{1}^{n-1}(x_{1} - x_{n}) \\
                      1      & x_{2}  & x_{2}^{2} & \cdots & x_{2}^{s-2}(x_{2} - x_{n}) & x_{2}^{s-1}(x_{2}^{2} - x_{n}^{2}) & \cdots & x_{2}^{n-1}(x_{2} - x_{n}) \\
                      \vdots & \vdots & \vdots    & \ddots & \vdots                     & \vdots                             & \ddots & \vdots                     \\
                      1      & x_{n}  & x_{n}^{2} & \cdots & 0                          & 0                                  & \cdots & 0
                  \end{vmatrix} \\
                   & =
                  \begin{vmatrix}
                      1      & x_{1} - x_{n} & x_{1}(x_{1} - x_{n}) & \cdots & x_{1}^{s-2}(x_{1} - x_{n}) & x_{1}^{s-1}(x_{1}^{2} - x_{n}^{2}) & \cdots & x_{1}^{n-1}(x_{1} - x_{n}) \\
                      1      & x_{2} - x_{n} & x_{2}(x_{2} - x_{n}) & \cdots & x_{2}^{s-2}(x_{2} - x_{n}) & x_{2}^{s-1}(x_{2}^{2} - x_{n}^{2}) & \cdots & x_{2}^{n-1}(x_{2} - x_{n}) \\
                      \vdots & \vdots        & \vdots               & \ddots & \vdots                     & \vdots                             & \ddots & \vdots                     \\
                      1      & 0             & 0                    & \cdots & 0                          & 0                                  & \cdots & 0
                  \end{vmatrix}
              \end{align*}
              \par Khai triển Laplace theo dòng thứ $n$:
              \begin{align*}
                   & = {(-1)}^{n+1}(x_{1} - x_{n})(x_{2} - x_{n})\cdots (x_{n-1} - x_{n})
                  \begin{vmatrix}
                      1      & x_{1}   & \cdots & x_{1}^{s-2}   & x_{1}^{s-1}(x_{1} + x_{n})     & \cdots & x_{1}^{n-1}   \\
                      1      & x_{2}   & \cdots & x_{2}^{s-2}   & x_{2}^{s-1}(x_{2} + x_{n})     & \cdots & x_{2}^{n-1}   \\
                      \vdots & \vdots  & \ddots & \vdots        & \vdots                         & \ddots & \vdots        \\
                      1      & x_{n-1} & \cdots & x_{n-1}^{s-2} & x_{n-1}^{s-1}(x_{n-1} + x_{n}) & \cdots & x_{n-1}^{n-1}
                  \end{vmatrix}                                                    \\
                   & = (x_{n} - x_{1})\cdots (x_{n} - x_{n-1})
                  \begin{vmatrix}
                      1      & x_{1}   & \cdots & x_{1}^{s-2}   & x_{1}^{s}   & \cdots & x_{1}^{n-1} \\
                      1      & x_{2}   & \cdots & x_{2}^{s-2}   & x_{2}^{s}   & \cdots & x_{2}^{n-1} \\
                      \vdots & \vdots  & \ddots & \vdots        & \vdots      & \ddots & \vdots      \\
                      1      & x_{n-1} & \cdots & x_{n-1}^{s-2} & x_{n-1}^{s} & \cdots & x_{s}^{n-1}
                  \end{vmatrix}                                                                         \\
                   & + (x_{n} - x_{1})\cdots (x_{n} - x_{n-1})
                  x_{n}\begin{vmatrix}
                           1      & x_{1}   & \cdots & x_{1}^{s-1}   & x_{1}^{s+1}   & \cdots & x_{1}^{n-1} \\
                           1      & x_{2}   & \cdots & x_{2}^{s-1}   & x_{2}^{s+1}   & \cdots & x_{2}^{n-1} \\
                           \vdots & \vdots  & \ddots & \vdots        & \vdots        & \ddots & \vdots      \\
                           1      & x_{n-1} & \cdots & x_{n-1}^{s-1} & x_{n-1}^{s+1} & \cdots & x_{s}^{n-1}
                       \end{vmatrix}                                                                  \\
                   & = \prod^{n}_{1\le i\ne n}(x_{n} - x_{i})\cdot \left(D^{(s-1)}_{n-1} + x_{n}D^{(s)}_{n-1}\right)                                                     \\
                   & = \prod^{n}_{1\le i\ne n}(x_{n} - x_{i})\cdot \left( D_{n-1}e_{n-s}(x_{1}, \ldots, x_{n-1}) + x_{n}D_{n-1}e_{n-1-s}(x_{1}, \ldots, x_{n-1}) \right) \\
                   & = \prod^{n}_{1\le i\ne n}(x_{n} - x_{i})\cdot D_{n-1} \left( e_{n-s}(x_{1}, \ldots, x_{n-1}) + x_{n}e_{n-1-s}(x_{1}, \ldots, x_{n-1}) \right)       \\
                   & = D_{n} e_{n-s}(x_{1}, \ldots, x_{n-1}, x_{n}).
              \end{align*}
              \par Phép chứng minh quy nạp hoàn tất.
              \par Vậy $D^{(s)}_{n} = D_{n}e_{n-s}(x_{1}, \ldots, x_{n})$.
    \end{enumerate}
\end{proof}

\par Ta kí hiệu các vector cột gồm $n$ thành phần:
\[
    \alpha_{0} = \begin{pmatrix}
        1 \\ 1 \\ \vdots \\ 1
    \end{pmatrix}\qquad
    \alpha_{k} = \begin{pmatrix}
        x_{1}^{k} \\ x_{2}^{k} \\ \vdots \\ x_{n}^{k}
    \end{pmatrix}
\]

% exercise 3.17
\begin{exercise}
    $\begin{vmatrix}
            1      & x_{1}(x_{1} - 1) & x_{1}^{2}(x_{1} - 1) & \cdots & x_{1}^{n-1}(x_{1} - 1) \\
            1      & x_{2}(x_{2} - 1) & x_{2}^{2}(x_{2} - 1) & \cdots & x_{2}^{n-1}(x_{2} - 1) \\
            \vdots & \vdots           & \vdots               & \ddots & \vdots                 \\
            1      & x_{n}(x_{n} - 1) & x_{n}^{2}(x_{n} - 1) & \cdots & x_{n}^{n-1}(x_{n} - 1)
        \end{vmatrix}$.
\end{exercise}

\begin{proof}[Lời giải]
    \par Ta lần lượt thực hiện các biến đổi sau:
    \[
        \begin{cases}
            c_{n}   & := c_{n} + c_{n-1} + \cdots + c_{2}   \\
            c_{n-1} & := c_{n-1} + c_{n-2} + \cdots + c_{2} \\
                    & \vdots                                \\
            c_{3}   & := c_{3} + c_{2}
        \end{cases}
    \]
    \begingroup{}
    \allowdisplaybreaks{}
    \begin{align*}
          & \begin{vmatrix}
                1      & x_{1}(x_{1} - 1) & x_{1}^{2}(x_{1} - 1) & \cdots & x_{1}^{n-1}(x_{1} - 1) \\
                1      & x_{2}(x_{2} - 1) & x_{2}^{2}(x_{2} - 1) & \cdots & x_{2}^{n-1}(x_{2} - 1) \\
                \vdots & \vdots           & \vdots               & \ddots & \vdots                 \\
                1      & x_{n}(x_{n} - 1) & x_{n}^{2}(x_{n} - 1) & \cdots & x_{n}^{n-1}(x_{n} - 1)
            \end{vmatrix}                                                                                                                                               \\
        = &
        \begin{vmatrix}
            1      & x_{1}^{2} - x_{1} & x_{1}^{3} - x_{1}^{2} & \cdots & x_{1}^{n} - x_{1}^{n-1} \\
            1      & x_{2}^{2} - x_{2} & x_{2}^{3} - x_{2}^{2} & \cdots & x_{2}^{n} - x_{2}^{n-1} \\
            \vdots & \vdots            & \vdots                & \ddots & \vdots                  \\
            1      & x_{n}^{2} - x_{n} & x_{n}^{3} - x_{n}^{2} & \cdots & x_{n}^{n} - x_{n}^{n-1}
        \end{vmatrix}
        =
        \begin{vmatrix}
            1      & x_{1}^{2} - x_{1} & x_{1}^{3} - x_{1} & \cdots & x_{1}^{n} - x_{1} \\
            1      & x_{2}^{2} - x_{2} & x_{2}^{3} - x_{2} & \cdots & x_{2}^{n} - x_{2} \\
            \vdots & \vdots            & \vdots            & \ddots & \vdots            \\
            1      & x_{n}^{2} - x_{n} & x_{n}^{3} - x_{n} & \cdots & x_{n}^{n} - x_{n}
        \end{vmatrix}                                                                                                                                                          \\
        = & \det(\alpha_{0}, \alpha_{2} - \alpha_{1}, \alpha_{3} - \alpha_{1}, \ldots, \alpha_{n} - \alpha_{1})                                                                                                                              \\
        = & \det(\alpha_{0}, \alpha_{2}, \alpha_{3}, \ldots, \alpha_{n}) + \left(\det(\alpha_{0}, -\alpha_{1}, \alpha_{3}, \ldots, \alpha_{n}) + \cdots + \det(\alpha_{0}, \alpha_{2}, \alpha_{3}, \ldots, \alpha_{n-1}, -\alpha_{1})\right) \\
        = & D_{n}e_{n-1}(x_{1}, \ldots, x_{n}) + \sum^{n}_{s=2}{(-1)}^{s+1}D^{(s)}_{n}                                                                                                                                                       \\
        = & D_{n}e_{n-1}(x_{1}, \ldots, x_{n}) + \sum^{n}_{s=2}{(-1)}^{s+1}D_{n}e_{n-s}(x_{1}, \ldots, x_{n})                                                                                                                                \\
        = & D_{n}\sum^{n}_{s=1}{(-1)}^{s+1}e_{n-s}(x_{1}, \ldots, x_{n})                                                                                                                                                                     \\
        = & D_{n}\sum^{n}_{s=0}{(-1)}^{s+1}e_{n-s}(x_{1}, \ldots, x_{n}) + D_{n}\prod^{n}_{i=1}x_{i}                                                                                                                                         \\
        = & D_{n}\prod^{n}_{i=1}x_{i} - D_{n}\prod^{n}_{i=1}(x_{i} - 1).
    \end{align*}
    \endgroup{}
\end{proof}

% exercise 3.18
\begin{exercise}
    $\begin{vmatrix}
            1 + x_{1} & 1 + x_{1}^{2} & \cdots & 1 + x_{1}^{n} \\
            1 + x_{2} & 1 + x_{2}^{2} & \cdots & 1 + x_{2}^{n} \\
            \vdots    & \vdots        & \ddots & \vdots        \\
            1 + x_{n} & 1 + x_{n}^{2} & \cdots & 1 + x_{n}^{n}
        \end{vmatrix}$.
\end{exercise}

\begin{proof}[Lời giải]
    \begingroup{}
    \allowdisplaybreaks{}
    \begin{align*}
        \begin{vmatrix}
            1 + x_{1} & 1 + x_{1}^{2} & \cdots & 1 + x_{1}^{n} \\
            1 + x_{2} & 1 + x_{2}^{2} & \cdots & 1 + x_{2}^{n} \\
            \vdots    & \vdots        & \ddots & \vdots        \\
            1 + x_{n} & 1 + x_{n}^{2} & \cdots & 1 + x_{n}^{n}
        \end{vmatrix}
         & =
        \begin{vmatrix}
            1 + x_{1} & x_{1}^{2} - x_{1} & \cdots & x_{1}^{n} - x_{1} \\
            1 + x_{2} & x_{2}^{2} - x_{2} & \cdots & x_{2}^{n} - x_{2} \\
            \vdots    & \vdots            & \ddots & \vdots            \\
            1 + x_{n} & x_{n}^{2} - x_{n} & \cdots & x_{n}^{n} - x_{n}
        \end{vmatrix}                                   \\
         & =
        \begin{vmatrix}
            1      & x_{1}^{2} - x_{1} & \cdots & x_{1}^{n} - x_{1} \\
            1      & x_{2}^{2} - x_{2} & \cdots & x_{2}^{n} - x_{2} \\
            \vdots & \vdots            & \ddots & \vdots            \\
            1      & x_{n}^{2} - x_{n} & \cdots & x_{n}^{n} - x_{n}
        \end{vmatrix}
        +
        \begin{vmatrix}
            x_{1}  & x_{1}^{2} - x_{1} & \cdots & x_{1}^{n} - x_{1} \\
            x_{2}  & x_{2}^{2} - x_{2} & \cdots & x_{2}^{n} - x_{2} \\
            \vdots & \vdots            & \ddots & \vdots            \\
            x_{n}  & x_{n}^{2} - x_{n} & \cdots & x_{n}^{n} - x_{n}
        \end{vmatrix}                                      \\
         & =
        \begin{vmatrix}
            1      & x_{1}^{2} - x_{1} & \cdots & x_{1}^{n} - x_{1} \\
            1      & x_{2}^{2} - x_{2} & \cdots & x_{2}^{n} - x_{2} \\
            \vdots & \vdots            & \ddots & \vdots            \\
            1      & x_{n}^{2} - x_{n} & \cdots & x_{n}^{n} - x_{n}
        \end{vmatrix}
        +
        \begin{vmatrix}
            x_{1}  & x_{1}^{2} & \cdots & x_{1}^{n} \\
            x_{2}  & x_{2}^{2} & \cdots & x_{2}^{n} \\
            \vdots & \vdots    & \ddots & \vdots    \\
            x_{n}  & x_{n}^{2} & \cdots & x_{n}^{n}
        \end{vmatrix}                                                      \\
         & = D_{n}\prod^{n}_{i=1}x_{i} - D_{n}\prod^{n}_{i=1}(x_{i} - 1) + D_{n}\prod^{n}_{i=1}x_{i} \\
         & = 2D_{n}\prod^{n}_{i=1}x_{i} - D_{n}\prod^{n}_{i=1}(x_{i} - 1).
    \end{align*}
    \endgroup{}
\end{proof}

% exercise 3.19
\begin{exercise}
    $\begin{vmatrix}
            1      & \cos(\varphi_{1}) & \cos(2\varphi_{1}) & \cdots & \cos((n-1)\varphi_{1}) \\
            1      & \cos(\varphi_{2}) & \cos(2\varphi_{2}) & \cdots & \cos((n-1)\varphi_{2}) \\
            \vdots & \vdots            & \vdots             & \ddots & \vdots                 \\
            1      & \cos(\varphi_{n}) & \cos(2\varphi_{n}) & \cdots & \cos((n-1)\varphi_{n})
        \end{vmatrix}$.
\end{exercise}

\begin{proof}[Lời giải]
    \par Áp dụng các công thức cộng, trừ cung:
    \begin{align*}
        \cos(n\varphi) & = \cos(\varphi + (n-1)\varphi) = \cos(\varphi)\cos((n-1)\varphi) - \sin(\varphi)\sin((n-1)\varphi) \\
                       & = 2\cos(\varphi)\cos((n-1)\varphi) - \cos((n-2)\varphi).
    \end{align*}
    \par Dựa vào liên hệ này và nhận xét rằng $\cos(0\varphi) = 1$, $\cos(\varphi) = \cos(\varphi)$, ta có thể chứng minh quy nạp được cho khẳng định: $\cos(n\varphi)$ là một đa thực bậc $n$ với biến $\cos(\varphi)$, hệ số cao nhất là $2^{n-1}$.
    \begingroup{}
    \allowdisplaybreaks{}
    \begin{align*}
        \begin{vmatrix}
            1      & \cos(\varphi_{1}) & \cos(2\varphi_{1}) & \cdots & \cos((n-1)\varphi_{1}) \\
            1      & \cos(\varphi_{2}) & \cos(2\varphi_{2}) & \cdots & \cos((n-1)\varphi_{2}) \\
            \vdots & \vdots            & \vdots             & \ddots & \vdots                 \\
            1      & \cos(\varphi_{n}) & \cos(2\varphi_{n}) & \cdots & \cos((n-1)\varphi_{n})
        \end{vmatrix}
         & =
        \begin{vmatrix}
            1      & \cos(\varphi_{1}) & 2{\cos(\varphi_{1})}^{2} & \cdots & \cos((n-1)\varphi_{1}) \\
            1      & \cos(\varphi_{2}) & 2{\cos(\varphi_{2})}^{2} & \cdots & \cos((n-1)\varphi_{2}) \\
            \vdots & \vdots            & \vdots                   & \ddots & \vdots                 \\
            1      & \cos(\varphi_{n}) & 2{\cos(\varphi_{n})}^{2} & \cdots & \cos((n-1)\varphi_{n})
        \end{vmatrix}                     \\
         & =
        \begin{vmatrix}
            1      & \cos(\varphi_{1}) & 2{\cos(\varphi_{1})}^{2} & \cdots & 2^{n-2}\cos((n-1)\varphi_{1}) \\
            1      & \cos(\varphi_{2}) & 2{\cos(\varphi_{2})}^{2} & \cdots & 2^{n-2}\cos((n-1)\varphi_{2}) \\
            \vdots & \vdots            & \vdots                   & \ddots & \vdots                        \\
            1      & \cos(\varphi_{n}) & 2{\cos(\varphi_{n})}^{2} & \cdots & 2^{n-2}\cos((n-1)\varphi_{n})
        \end{vmatrix}              \\
         & = \frac{1}{2^{n-1}}
        \begin{vmatrix}
            1      & 2\cos(\varphi_{1}) & 2^{2}{\cos(\varphi_{1})}^{2} & \cdots & 2^{n-1}{\cos(\varphi_{1})}^{n-1} \\
            1      & 2\cos(\varphi_{2}) & 2^{2}{\cos(\varphi_{2})}^{2} & \cdots & 2^{n-1}{\cos(\varphi_{2})}^{n-2} \\
            \vdots & \vdots             & \vdots                       & \ddots & \vdots                           \\
            1      & 2\cos(\varphi_{n}) & 2^{2}{\cos(\varphi_{n})}^{2} & \cdots & 2^{n-1}{\cos(\varphi_{n})}^{n-1}
        \end{vmatrix} \\
         & = \frac{1}{2^{n-1}}\prod_{i>j}2(\cos(\varphi_{i}) - \cos(\varphi_{j}))                                        \\
         & = 2^{(n-1)(n-2)/2}\prod_{i>j}(\cos(\varphi_{i}) - \cos(\varphi_{j})).                                         \\
    \end{align*}
    \endgroup{}
\end{proof}

% exercise 3.20
\begin{exercise}
    $\begin{vmatrix}
            x_{1}y_{1}     & 1 + x_{1}y_{2} & \cdots & 1 + x_{1}y_{n} \\
            1 + x_{2}y_{1} & x_{2}y_{2}     & \cdots & 1 + x_{2}y_{n} \\
            \vdots         & \vdots         & \ddots & \vdots         \\
            1 + x_{n}y_{1} & 1 + x_{n}y_{2} & \cdots & x_{n}y_{n}
        \end{vmatrix}$.
\end{exercise}

\begin{lemma}
    \begin{enumerate}[label = (\roman*)]
        \item
              \[
                  \underbrace{\begin{vmatrix}
                          0      & 1      & 1      & \cdots & 1      \\
                          1      & 0      & 1      & \cdots & 1      \\
                          1      & 1      & 0      & \cdots & 1      \\
                          \vdots & \vdots & \vdots & \ddots & \vdots \\
                          1      & 1      & 1      & \cdots & 0
                      \end{vmatrix}}_{n\times n}
                  = (-1){}^{n-1}(n-1).
              \]
        \item Thay cột thứ $k$ của định thức trên bởi cột $\begin{pmatrix}x_{1} & x_{2} & \cdots & x_{n} \end{pmatrix}^{T}$ thì định thức mới bằng:
              \[
                  {(-1)}^{n-1}\left((1-n)x_{k} + \sum^{n}_{i=1}x_{i}\right)
              \]
    \end{enumerate}
\end{lemma}

\begin{proof}[Chứng minh bổ đề]
    \begin{enumerate}[label = (\roman*)]
        \item
              \begingroup{}
              \allowdisplaybreaks{}
              \begin{align*}
                  \begin{vmatrix}
                      0      & 1      & 1      & \cdots & 1      \\
                      1      & 0      & 1      & \cdots & 1      \\
                      1      & 1      & 0      & \cdots & 1      \\
                      \vdots & \vdots & \vdots & \ddots & \vdots \\
                      1      & 1      & 1      & \cdots & 0
                  \end{vmatrix}
                   & =
                  \begin{vmatrix}
                      0      & 1      & 1      & \cdots & 1      \\
                      1      & -1     & 0      & \cdots & 0      \\
                      1      & 0      & -1     & \cdots & 0      \\
                      \vdots & \vdots & \vdots & \ddots & \vdots \\
                      1      & 0      & 0      & \cdots & -1
                  \end{vmatrix}\quad(r_{i}:= r_{i} - r_{1}) \\
                   & =
                  \begin{vmatrix}
                      n-1    & 1      & 1      & \cdots & 1      \\
                      0      & -1     & 0      & \cdots & 0      \\
                      0      & 0      & -1     & \cdots & 0      \\
                      \vdots & \vdots & \vdots & \ddots & \vdots \\
                      0      & 0      & 0      & \cdots & -1
                  \end{vmatrix}
                  = {(-1)}^{n-1}(n-1).
              \end{align*}
              \endgroup{}
        \item
              \par Trường hợp $k = 1$:
              \begingroup{}
              \allowdisplaybreaks{}
              \begin{align*}
                  \begin{vmatrix}
                      x_{1}  & 1      & 1      & \cdots & 1      \\
                      x_{2}  & 0      & 1      & \cdots & 1      \\
                      x_{3}  & 1      & 0      & \cdots & 1      \\
                      \vdots & \vdots & \vdots & \ddots & \vdots \\
                      x_{n}  & 1      & 1      & \cdots & 0
                  \end{vmatrix}
                   & =
                  \begin{vmatrix}
                      x_{1}         & 1      & 1      & \cdots & 1      \\
                      x_{2} - x_{1} & -1     & 0      & \cdots & 0      \\
                      x_{3} - x_{1} & 0      & -1     & \cdots & 0      \\
                      \vdots        & \vdots & \vdots & \ddots & \vdots \\
                      x_{n} - x_{1} & 0      & 0      & \cdots & -1
                  \end{vmatrix}\quad(r_{i}:= r_{i} - r_{1})                                 \\
                   & =
                  \begin{vmatrix}
                      (1-n)x_{1} + \displaystyle\sum^{n}_{i=1}x_{i} & 0      & 0      & \cdots & 0      \\
                      x_{2} - x_{1}                                 & -1     & 0      & \cdots & 0      \\
                      x_{3} - x_{1}                                 & 0      & -1     & \cdots & 0      \\
                      \vdots                                        & \vdots & \vdots & \ddots & \vdots \\
                      x_{n} - x_{1}                                 & 0      & 0      & \cdots & -1
                  \end{vmatrix}\quad(r_{1}:= r_{1} + \sum^{n}_{i=2}r_{i}) \\
                   & = {(-1)}^{n-1}\left((1-n)x_{1} + \sum^{n}_{i=1}x_{i}\right).
              \end{align*}
              \endgroup{}

              \par Trường hợp $k\ne 1$:
              \begingroup{}
              \allowdisplaybreaks{}
              \begin{align*}
                  \begin{vmatrix}
                      0      & 1      & \cdots & x_{1}  & \cdots & 1      \\
                      1      & 0      & \cdots & x_{2}  & \cdots & 1      \\
                      \vdots & \vdots & \ddots & \vdots & \ddots & 1      \\
                      1      & 1      & \cdots & x_{k}  & \cdots & 1      \\
                      \vdots & \vdots & \ddots & \vdots & \ddots & \vdots \\
                      1      & 1      & \cdots & x_{n}  & \cdots & 0
                  \end{vmatrix}
                   & =
                  \begin{vmatrix}
                      x_{1}  & 1      & \cdots & 0      & \cdots & 1      \\
                      x_{2}  & 0      & \cdots & 1      & \cdots & 1      \\
                      \vdots & \vdots & \ddots & \vdots & \ddots & 1      \\
                      x_{k}  & 1      & \cdots & 1      & \cdots & 1      \\
                      \vdots & \vdots & \ddots & \vdots & \ddots & \vdots \\
                      x_{n}  & 1      & \cdots & 1      & \cdots & 0
                  \end{vmatrix}\quad(c_{1} \leftrightarrow c_{k}) \\
                   & =
                  \begin{vmatrix}
                      x_{k}  & 1      & \cdots & 1      & \cdots & 1      \\
                      x_{2}  & 0      & \cdots & 1      & \cdots & 1      \\
                      \vdots & \vdots & \ddots & \vdots & \ddots & 1      \\
                      x_{1}  & 1      & \cdots & 0      & \cdots & 1      \\
                      \vdots & \vdots & \ddots & \vdots & \ddots & \vdots \\
                      x_{n}  & 1      & \cdots & 1      & \cdots & 0
                  \end{vmatrix}\quad(r_{1} \leftrightarrow r_{k}) \\
                   & =
                  {(-1)}^{n-1}\left((1-n)x_{k} + \sum^{n}_{i=1}x_{i}\right) \quad\text{(áp dụng trường hợp $k=1$)}
              \end{align*}
              \endgroup{}
    \end{enumerate}
\end{proof}

\begin{proof}[Lời giải]
    \begingroup{}
    \allowdisplaybreaks{}
    \begin{align*}
          & \begin{vmatrix}
                x_{1}y_{1}     & 1 + x_{1}y_{2} & \cdots & 1 + x_{1}y_{n} \\
                1 + x_{2}y_{1} & x_{2}y_{2}     & \cdots & 1 + x_{2}y_{n} \\
                \vdots         & \vdots         & \ddots & \vdots         \\
                1 + x_{n}y_{1} & 1 + x_{n}y_{2} & \cdots & x_{n}y_{n}
            \end{vmatrix}                                                                                     \\
        = &
        \det\begin{pmatrix}
                \begin{pmatrix}
                0      \\
                1      \\
                \vdots \\
                1
            \end{pmatrix}
                +
                y_{1}\begin{pmatrix}
                     x_{1}  \\
                     x_{2}  \\
                     \vdots \\
                     x_{n}
                 \end{pmatrix},
                \ldots,
                \begin{pmatrix}
                1      \\
                1      \\
                \vdots \\
                0
            \end{pmatrix}
                +
                y_{n}\begin{pmatrix}
                     x_{1}  \\
                     x_{2}  \\
                     \vdots \\
                     x_{n}
                 \end{pmatrix}
            \end{pmatrix}                                                                                                              \\
        = &
        \begin{vmatrix}
            0      & 1      & \cdots & 1      \\
            1      & 0      & \cdots & 1      \\
            \vdots & \vdots & \ddots & \vdots \\
            1      & 1      & \cdots & 0
        \end{vmatrix}
        +
        \sum^{n}_{k=1}y_{k}
        \begin{vmatrix}
            0      & 1      & \cdots & x_{1}  & \cdots & 1      \\
            1      & 0      & \cdots & x_{2}  & \cdots & 1      \\
            \vdots & \vdots & \ddots & \vdots & \ddots & \vdots \\
            1      & 1      & \cdots & x_{k}  & \cdots & 1      \\
            \vdots & \vdots & \ddots & \vdots & \ddots & \vdots \\
            1      & 1      & \cdots & x_{n}  & \cdots & 0
        \end{vmatrix}                                                                                               \\
        = &
        {(-1)}^{n-1}(n-1) + {(-1)}^{n-1}\sum^{n}_{k=1}y_{k}\left((1-n)x_{k} + \sum^{n}_{i=1}x_{i}\right)                                                  \\
        = & {(-1)}^{n-1}(n-1) + {(-1)}^{n-1}\left(\sum^{n}_{i=1}x_{i}\right)\left(\sum^{n}_{i=1}y_{i}\right) - {(-1)}^{n-1}(n-1)\sum^{n}_{i=1}x_{i}y_{i}.
    \end{align*}
    \endgroup{}
\end{proof}

% exercise 3.21
\begin{exercise}
    $C_{n} = \begin{vmatrix}
            (a_{1} + b_{1}){}^{-1} & (a_{1} + b_{2}){}^{-1} & \cdots & (a_{1} + b_{n}){}^{-1} \\
            (a_{2} + b_{1}){}^{-1} & (a_{2} + b_{2}){}^{-1} & \cdots & (a_{2} + b_{n}){}^{-1} \\
            \vdots                 & \vdots                 & \ddots & \vdots                 \\
            (a_{n} + b_{1}){}^{-1} & (a_{n} + b_{2}){}^{-1} & \cdots & (a_{n} + b_{n}){}^{-1}
        \end{vmatrix}$.
\end{exercise}

\begin{proof}[Lời giải]
    \begingroup{}
    \allowdisplaybreaks{}
    \begin{align*}
          & \begin{vmatrix}
                (a_{1} + b_{1}){}^{-1} & (a_{1} + b_{2}){}^{-1} & \cdots & (a_{1} + b_{n}){}^{-1} \\
                (a_{2} + b_{1}){}^{-1} & (a_{2} + b_{2}){}^{-1} & \cdots & (a_{2} + b_{n}){}^{-1} \\
                \vdots                 & \vdots                 & \ddots & \vdots                 \\
                (a_{n} + b_{1}){}^{-1} & (a_{n} + b_{2}){}^{-1} & \cdots & (a_{n} + b_{n}){}^{-1}
            \end{vmatrix}                                                                 \\
        = &
        \begin{vmatrix}
            (b_{n}-b_{1})(a_{1}+b_{1}){}^{-1}(a_{1}+b_{n}){}^{-1} & (b_{n} - b_{2})(a_{1} + b_{2}){}^{-1}(a_{1} + b_{n}){}^{-1} & \cdots & (a_{1} + b_{n}){}^{-1} \\
            (b_{n}-b_{1})(a_{2}+b_{1}){}^{-1}(a_{2}+b_{n}){}^{-1} & (b_{n} - b_{2})(a_{2} + b_{2}){}^{-1}(a_{2} + b_{n}){}^{-1} & \cdots & (a_{2} + b_{n}){}^{-1} \\
            \vdots                                                & \vdots                                                      & \ddots & \vdots                 \\
            (b_{n}-b_{1})(a_{n}+b_{1}){}^{-1}(a_{n}+b_{n}){}^{-1} & (b_{n} - b_{2})(a_{n} + b_{2}){}^{-1}(a_{n} + b_{n}){}^{-1} & \cdots & (a_{n} + b_{n}){}^{-1} \\
        \end{vmatrix}\quad (c_{i}:= c_{i} - c_{n}) \\
        = &
        \prod^{n}_{1\le i<n}(b_{n} - b_{i})\prod^{n}_{i=1}{(a_{i}+b_{n})}^{-1}
        \begin{vmatrix}
            (a_{1}+b_{1}){}^{-1} & (a_{1} + b_{2}){}^{-1} & \cdots & 1      \\
            (a_{2}+b_{1}){}^{-1} & (a_{2} + b_{2}){}^{-1} & \cdots & 1      \\
            \vdots               & \vdots                 & \ddots & \vdots \\
            (a_{n}+b_{1}){}^{-1} & (a_{n} + b_{2}){}^{-1} & \cdots & 1      \\
        \end{vmatrix}                                                                                       \\
        = &
        \prod^{n}_{1\le i<n}(b_{n} - b_{i})\prod^{n}_{i=1}{(a_{i}+b_{n})}^{-1}
        \begin{vmatrix}
            (a_{n}-a_{1}){(a_{1}+b_{1})}^{-1}{(a_{n}+b_{1})}^{-1} & (a_{n}-a_{1}){(a_{1}+b_{2})}^{-1}{(a_{n}+b_{2})}^{-1} & \cdots & 0      \\
            (a_{n}-a_{2}){(a_{2}+b_{1})}^{-1}{(a_{n}+b_{1})}^{-1} & (a_{n}-a_{2}){(a_{2}+b_{2})}^{-1}{(a_{n}+b_{2})}^{-1} & \cdots & 0      \\
            \vdots                                                & \vdots                                                & \ddots & \vdots \\
            {(a_{n}+b_{1})}^{-1}                                  & {(a_{n}+b_{2})}^{-1}                                  & \cdots & 1
        \end{vmatrix}
    \end{align*}
    \par Áp dụng khai triển Laplace cho cột thứ $n$ và tách nhân tử chung:
    \begin{align*}
        = &
        \prod^{n}_{1\le i<n}(b_{n} - b_{i})\prod^{n}_{i=1}{(a_{i}+b_{n})}^{-1}\prod^{n}_{1\le i<n}(a_{n} - a_{i})\prod^{n}_{i=1}{(a_{n} + b_{i})}^{-1}
        \begin{vmatrix}
            {(a_{1}+b_{1})}^{-1}   & {(a_{1}+b_{2})}^{-1}   & \cdots & {(a_{1}+b_{n-1})}^{-1}   \\
            {(a_{2}+b_{1})}^{-1}   & {(a_{2}+b_{2})}^{-1}   & \cdots & {(a_{2}+b_{n-1})}^{-1}   \\
            \vdots                 & \vdots                 & \ddots & \vdots                   \\
            {(a_{n-1}+b_{1})}^{-1} & {(a_{n-1}+b_{2})}^{-1} & \cdots & {(a_{n-1}+b_{n-1})}^{-1}
        \end{vmatrix}
    \end{align*}
    \par Từ hệ thức truy hồi trên, và nhận xét $C_{1} = {(a_{1} + b_{1})}^{-1}$, ta được:
    \[
        C_{n} = \prod_{i>j} (a_{i}-a_{j})(b_{i}-b_{j}) \times \prod_{i\ne j}{(a_{i}+b_{j})}^{-1}.
    \]
    \endgroup{}
\end{proof}

% exercise 3.22
\begin{exercise}
    \par Dãy Fibonacci là dãy số bắt đầu với các số hạng 1, 2 và mỗi số hạng, kể từ số hạng thứ ba, đều bằng tổng của hai số hạng đứng ngay trước nó. Chứng minh rằng số hạng thứ $n$ của dãy Fibonacci bằng định thức cỡ $n$ sau đây:
    \[
        \begin{vmatrix}
            1      & 1      & 0      & \cdots & 0      & 0      \\
            -1     & 1      & 1      & \cdots & 0      & 0      \\
            0      & -1     & 1      & \cdots & 0      & 0      \\
            \vdots & \vdots & \vdots & \ddots & \vdots & \vdots \\
            0      & 0      & 0      & \cdots & -1     & 1
        \end{vmatrix}.
    \]
\end{exercise}

\begin{proof}[Lời giải]
    \[
        F_{1} = 1 = \begin{vmatrix}1\end{vmatrix}
    \]
    \[
        F_{2} = 2 = \begin{vmatrix}1 & 1 \\ -1 & 1 \end{vmatrix}
    \]
    \par Ta chứng minh bằng quy nạp, giả sử đẳng thức đúng đến $n-1$.
    \par Với $n > 2$, áp dụng khai triển Laplace cho hàng thứ nhất:
    \begingroup{}
    \allowdisplaybreaks{}
    \begin{align*}
        \underbrace{\begin{vmatrix}
                            1      & 1      & 0      & \cdots & 0      & 0      \\
                            -1     & 1      & 1      & \cdots & 0      & 0      \\
                            0      & -1     & 1      & \cdots & 0      & 0      \\
                            \vdots & \vdots & \vdots & \ddots & \vdots & \vdots \\
                            0      & 0      & 0      & \cdots & -1     & 1
                        \end{vmatrix}}_{n\times n}
         & =
        \underbrace{\begin{vmatrix}
                            1      & 1      & 0      & \cdots & 0      & 0      \\
                            -1     & 1      & 1      & \cdots & 0      & 0      \\
                            0      & -1     & 1      & \cdots & 0      & 0      \\
                            \vdots & \vdots & \vdots & \ddots & \vdots & \vdots \\
                            0      & 0      & 0      & \cdots & -1     & 1
                        \end{vmatrix}}_{(n-1)\times (n-1)}
        +
        {(-1)}^{1+2}
        \underbrace{\begin{vmatrix}
                            -1     & 1      & 0      & \cdots & 0      & 0      \\
                            -1     & 1      & 1      & \cdots & 0      & 0      \\
                            0      & -1     & 1      & \cdots & 0      & 0      \\
                            \vdots & \vdots & \vdots & \ddots & \vdots & \vdots \\
                            0      & 0      & 0      & \cdots & -1     & 1
                        \end{vmatrix}}_{(n-2)\times (n-2)} \\
         & =
        \underbrace{\begin{vmatrix}
                            1      & 1      & 0      & \cdots & 0      & 0      \\
                            -1     & 1      & 1      & \cdots & 0      & 0      \\
                            0      & -1     & 1      & \cdots & 0      & 0      \\
                            \vdots & \vdots & \vdots & \ddots & \vdots & \vdots \\
                            0      & 0      & 0      & \cdots & -1     & 1
                        \end{vmatrix}}_{(n-1)\times (n-1)}
        +
        \underbrace{\begin{vmatrix}
                            1      & 1      & 0      & \cdots & 0      & 0      \\
                            -1     & 1      & 1      & \cdots & 0      & 0      \\
                            0      & -1     & 1      & \cdots & 0      & 0      \\
                            \vdots & \vdots & \vdots & \ddots & \vdots & \vdots \\
                            0      & 0      & 0      & \cdots & -1     & 1
                        \end{vmatrix}}_{(n-2)\times (n-2)} \\
         & = F_{n-1} + F_{n-2}                                          \\
         & = F_{n}.
    \end{align*}
    \endgroup{}
    \par Vậy giả thiết quy nạp đúng, ta có được điều phải chứng minh.
\end{proof}

% exercise 3.23
\begin{exercise}
    \par Tính định thức sau đây bằng cách \textit{viết nó thành tích của hai định thức:}
    \[
        \begin{vmatrix}
            s_{0}  & s_{1}   & s_{2}   & \cdots & s_{n-1}  & 1      \\
            s_{1}  & s_{2}   & s_{3}   & \cdots & s_{n}    & x      \\
            s_{2}  & s_{3}   & s_{4}   & \cdots & s_{n+1}  & x^{2}  \\
            \vdots & \vdots  & \vdots  & \ddots & \vdots   & \vdots \\
            s_{n}  & s_{n+1} & s_{n+2} & \cdots & s_{2n-1} & x^{n}
        \end{vmatrix}.
    \]
    \par trong đó $s_{k} = \displaystyle\sum^{n}_{i=1}x_{i}^{k}$.
\end{exercise}

\begin{proof}[Lời giải]
    \[
        \begin{pmatrix}
            s_{0}  & s_{1}   & s_{2}   & \cdots & s_{n-1}  & 1      \\
            s_{1}  & s_{2}   & s_{3}   & \cdots & s_{n}    & x      \\
            s_{2}  & s_{3}   & s_{4}   & \cdots & s_{n+1}  & x^{2}  \\
            \vdots & \vdots  & \vdots  & \ddots & \vdots   & \vdots \\
            s_{n}  & s_{n+1} & s_{n+2} & \cdots & s_{2n-1} & x^{n}
        \end{pmatrix}
        =
        \begin{pmatrix}
            1           & 1           & 1           & \cdots & 1           & 1      \\
            x_{1}       & x_{2}       & x_{3}       & \cdots & x_{n}       & x      \\
            x_{1}^{2}   & x_{2}^{2}   & x_{3}^{2}   & \cdots & x_{n}^{2}   & x^{2}  \\
            \vdots      & \vdots      & \vdots      & \ddots & \vdots      & \vdots \\
            x_{1}^{n-1} & x_{2}^{n-1} & x_{3}^{n-1} & \cdots & x_{n}^{n-1} & x^{n}
        \end{pmatrix}
        \begin{pmatrix}
            1      & x_{1}  & x_{1}^{2} & \cdots & x_{1}^{n-1} & 0      \\
            1      & x_{2}  & s_{2}^{2} & \cdots & x_{2}^{n-1} & 0      \\
            \vdots & \vdots & \vdots    & \ddots & \vdots      & \vdots \\
            1      & x_{n}  & x_{n}^{2} & \cdots & x_{n}^{n-1} & 0      \\
            0      & 0      & 0         & \cdots & 0           & 1
        \end{pmatrix}
    \]
    \[
        \Rightarrow
        \begin{vmatrix}
            s_{0}  & s_{1}   & s_{2}   & \cdots & s_{n-1}  & 1      \\
            s_{1}  & s_{2}   & s_{3}   & \cdots & s_{n}    & x      \\
            s_{2}  & s_{3}   & s_{4}   & \cdots & s_{n+1}  & x^{2}  \\
            \vdots & \vdots  & \vdots  & \ddots & \vdots   & \vdots \\
            s_{n}  & s_{n+1} & s_{n+2} & \cdots & s_{2n-1} & x^{n}
        \end{vmatrix}
        =
        \prod_{i>j}(x_{i} - x_{j})\prod^{n}_{i=1}(x - x_{i})\prod_{i>j}(x_{i} - x_{j}) = \prod_{i>j}{(x_{i}-x_{j})}^{2}\prod^{n}_{i=1}(x-x_{i}).
    \]
\end{proof}

% exercise 3.24
\begin{exercise}
    \par Chứng minh rằng
    \[
        \begin{vmatrix}
            a_{1}   & a_{2}  & a_{3}  & \cdots & a_{n}   \\
            a_{n}   & a_{1}  & a_{2}  & \cdots & a_{n-1} \\
            a_{n-1} & a_{n}  & a_{1}  & \cdots & a_{n-2} \\
            \vdots  & \vdots & \vdots & \ddots & \vdots  \\
            a_{2}   & a_{3}  & a_{4}  & \cdots & a_{1}
        \end{vmatrix}
        = f(\varepsilon_{1})f(\varepsilon_{2})\cdots f(\varepsilon_{n}),
    \]
    \par trong đó, $f(X) = a_{1} + a_{2}X + \cdots + a_{n}X^{n-1}$ và $\varepsilon_{1}, \varepsilon_{2}, \ldots,\varepsilon_{n}$ là các căn bậc $n$ khác nhau của 1.
\end{exercise}

\begin{proof}
    \begingroup{}
    \allowdisplaybreaks{}
    \begin{align*}
        \begin{pmatrix}
            a_{1}   & a_{2}  & a_{3}  & \cdots & a_{n}   \\
            a_{n}   & a_{1}  & a_{2}  & \cdots & a_{n-1} \\
            a_{n-1} & a_{n}  & a_{1}  & \cdots & a_{n-2} \\
            \vdots  & \vdots & \vdots & \ddots & \vdots  \\
            a_{2}   & a_{3}  & a_{4}  & \cdots & a_{1}
        \end{pmatrix}
        \begin{pmatrix}
            1                   \\
            \varepsilon_{k}     \\
            \varepsilon_{k}^{2} \\
            \vdots              \\
            \varepsilon_{k}^{n-1}
        \end{pmatrix}
         & =
        \begin{pmatrix}
            f(\varepsilon_{k})                    \\
            \varepsilon_{k}f(\varepsilon_{k})     \\
            \varepsilon_{k}^{2}f(\varepsilon_{k}) \\
            \vdots                                \\
            \varepsilon_{k}^{n-1}f(\varepsilon_{k})
        \end{pmatrix}
        = f(\varepsilon_{k})
        \begin{pmatrix}
            1                   \\
            \varepsilon_{k}     \\
            \varepsilon_{k}^{2} \\
            \vdots              \\
            \varepsilon_{k}^{n-1}
        \end{pmatrix}                   \\
        \Rightarrow
        \begin{pmatrix}
            a_{1}   & a_{2}  & a_{3}  & \cdots & a_{n}   \\
            a_{n}   & a_{1}  & a_{2}  & \cdots & a_{n-1} \\
            a_{n-1} & a_{n}  & a_{1}  & \cdots & a_{n-2} \\
            \vdots  & \vdots & \vdots & \ddots & \vdots  \\
            a_{2}   & a_{3}  & a_{4}  & \cdots & a_{1}
        \end{pmatrix}
        \begin{pmatrix}
            1                     & 1                     & \cdots & 1                     \\
            \varepsilon_{1}       & \varepsilon_{2}       & \cdots & \varepsilon_{n}       \\
            \varepsilon_{1}^{2}   & \varepsilon_{2}^{2}   & \cdots & \varepsilon_{n}^{2}   \\
            \vdots                & \vdots                & \ddots & \vdots                \\
            \varepsilon_{1}^{n-1} & \varepsilon_{2}^{n-1} & \cdots & \varepsilon_{n}^{n-1}
        \end{pmatrix}
         & = \prod^{n}_{i=1}f(\varepsilon_{i})
        \begin{pmatrix}
            1                     & 1                     & \cdots & 1                     \\
            \varepsilon_{1}       & \varepsilon_{2}       & \cdots & \varepsilon_{n}       \\
            \varepsilon_{1}^{2}   & \varepsilon_{2}^{2}   & \cdots & \varepsilon_{n}^{2}   \\
            \vdots                & \vdots                & \ddots & \vdots                \\
            \varepsilon_{1}^{n-1} & \varepsilon_{2}^{n-1} & \cdots & \varepsilon_{n}^{n-1}
        \end{pmatrix}
    \end{align*}
    \endgroup{}

    \par Bên cạnh đó
    \[
        \begin{vmatrix}
            1                     & 1                     & \cdots & 1                     \\
            \varepsilon_{1}       & \varepsilon_{2}       & \cdots & \varepsilon_{n}       \\
            \varepsilon_{1}^{2}   & \varepsilon_{2}^{2}   & \cdots & \varepsilon_{n}^{2}   \\
            \vdots                & \vdots                & \ddots & \vdots                \\
            \varepsilon_{1}^{n-1} & \varepsilon_{2}^{n-1} & \cdots & \varepsilon_{n}^{n-1}
        \end{vmatrix}
        = \prod_{i>j}(\varepsilon_{i} - \varepsilon_{j}) \ne 0
    \]
    \par nên ma trận
    \[
        \begin{pmatrix}
            1                     & 1                     & \cdots & 1                     \\
            \varepsilon_{1}       & \varepsilon_{2}       & \cdots & \varepsilon_{n}       \\
            \varepsilon_{1}^{2}   & \varepsilon_{2}^{2}   & \cdots & \varepsilon_{n}^{2}   \\
            \vdots                & \vdots                & \ddots & \vdots                \\
            \varepsilon_{1}^{n-1} & \varepsilon_{2}^{n-1} & \cdots & \varepsilon_{n}^{n-1}
        \end{pmatrix}
    \]
    \par khả nghịch, suy ra
    \begin{align*}
        \begin{vmatrix}
            a_{1}   & a_{2}  & a_{3}  & \cdots & a_{n}   \\
            a_{n}   & a_{1}  & a_{2}  & \cdots & a_{n-1} \\
            a_{n-1} & a_{n}  & a_{1}  & \cdots & a_{n-2} \\
            \vdots  & \vdots & \vdots & \ddots & \vdots  \\
            a_{2}   & a_{3}  & a_{4}  & \cdots & a_{1}
        \end{vmatrix}
        \begin{vmatrix}
            1                     & 1                     & \cdots & 1                     \\
            \varepsilon_{1}       & \varepsilon_{2}       & \cdots & \varepsilon_{n}       \\
            \varepsilon_{1}^{2}   & \varepsilon_{2}^{2}   & \cdots & \varepsilon_{n}^{2}   \\
            \vdots                & \vdots                & \ddots & \vdots                \\
            \varepsilon_{1}^{n-1} & \varepsilon_{2}^{n-1} & \cdots & \varepsilon_{n}^{n-1}
        \end{vmatrix}
         & =
        \prod^{n}_{i=1}f(\varepsilon_{i})
        \begin{vmatrix}
            1                     & 1                     & \cdots & 1                     \\
            \varepsilon_{1}       & \varepsilon_{2}       & \cdots & \varepsilon_{n}       \\
            \varepsilon_{1}^{2}   & \varepsilon_{2}^{2}   & \cdots & \varepsilon_{n}^{2}   \\
            \vdots                & \vdots                & \ddots & \vdots                \\
            \varepsilon_{1}^{n-1} & \varepsilon_{2}^{n-1} & \cdots & \varepsilon_{n}^{n-1}
        \end{vmatrix} \\
        \Leftrightarrow
        \begin{vmatrix}
            a_{1}   & a_{2}  & a_{3}  & \cdots & a_{n}   \\
            a_{n}   & a_{1}  & a_{2}  & \cdots & a_{n-1} \\
            a_{n-1} & a_{n}  & a_{1}  & \cdots & a_{n-2} \\
            \vdots  & \vdots & \vdots & \ddots & \vdots  \\
            a_{2}   & a_{3}  & a_{4}  & \cdots & a_{1}
        \end{vmatrix}
         & =
        \prod^{n}_{i=1}f(\varepsilon_{i}).
    \end{align*}
\end{proof}

% exercise 3.25
\begin{exercise}
    \par Dùng khai triển Laplace chứng minh rằng nếu một định thức cỡ $n$ có các yếu tố nằm trên giao của $k$ hàng và $\ell$ cột xác định nào đó đều bằng 0, trong đó $k + \ell > n$, thì định thức đó bằng 0.
\end{exercise}

\begin{proof}
    \par Chú ý rằng: khi đổi chỗ các hàng, hay đổi chỗ các cột thì giá trị định thức chỉ đổi dấu..
    \par Do đó, không mất tính tổng quát, ta có thể giả sử các yếu tố nằm trên giao của $k$ hàng đầu tiên và $\ell$ cột đầu tiên đều bằng 0.
    \par Ma trận khi đó có dạng sau:
    \[
        A=
        \begin{pmatrix}
            0          & 0          & \cdots & 0             & a_{1(\ell+1)}     & \cdots & a_{1n}     \\
            0          & 0          & \cdots & 0             & a_{2(\ell+1)}     & \cdots & a_{2n}     \\
            \vdots     & \vdots     & \ddots & \vdots        & \vdots            & \ddots & \vdots     \\
            0          & 0          & \cdots & 0             & a_{k(\ell+1)}     & \cdots & a_{kn}     \\
            a_{(k+1)1} & a_{(k+1)2} & \cdots & a_{(k+1)\ell} & a_{(k+1)(\ell+1)} & \cdots & a_{(k+1)n} \\
            \vdots     & \vdots     & \ddots & \vdots        & \vdots            & \ddots & \vdots     \\
            a_{n1}     & a_{n2}     & \cdots & a_{n\ell}     & a_{n(\ell+1)}     & \cdots & a_{nn}
        \end{pmatrix}.
    \]
    \par Áp dụng khai triển Laplace cho $k$ hàng đầu tiên
    \[
        \det A = \sum_{1\le j_{1} < \cdots < j_{k}\le n}(-1){}^{1+\cdots+k+j_{1}+\cdots+j_{k}}D_{1,\ldots,k}^{j_{1},\ldots,j_{k}}\overline{D}_{1,\ldots,k}^{j_{1},\ldots,j_{k}}.
    \]
    \par Giả sử trong các cột $1\le j_{1} < \cdots < j_{k}\le n$, không có cột nào thuộc $\ell$ cột đầu tiên. Như vậy, số cột của ma trận $A$ sẽ lớn hơn hoặc bằng $k + \ell$, tức là $n \ge k + \ell$. Điều này mâu thuẫn với giả thiết $k + \ell > n$.
    \par Do đó, trong các cột $1\le j_{1} < \cdots < j_{k}\le n$, có ít nhất một cột thuộc $\ell$ cột đầu tiên. Điều này dẫn tới việc, bất kể chọn $k$ cột nào thì giá trị của định thức con  $D^{j_{1},\ldots,j_{k}}_{1,\ldots,k}$ bằng không.
    \par Vậy $\det A = 0$.
\end{proof}

% exercise 3.26
\begin{exercise}
    \par Giải hệ phương trình sau đây bằng phương pháp Cramer và phương pháp khử:
    \[
        \begin{array}{ccccccccccc}
            3x_{1} & + & 4x_{2} & + & x_{3}  & + & 2x_{4} & + & 3 & = & 0, \\
            3x_{1} & + & 5x_{2} & + & 3x_{3} & + & 5x_{4} & + & 6 & = & 0, \\
            6x_{1} & + & 8x_{2} & + & x_{3}  & + & 5x_{4} & + & 8 & = & 0, \\
            3x_{1} & + & 5x_{2} & + & 3x_{3} & + & 7x_{4} & + & 8 & = & 0.
        \end{array}
    \]
\end{exercise}

\begin{proof}[Lời giải]
    \par Sử dụng công thức Cramer.
    \par Định thức của ma trận hệ số:
    \begingroup{}
    \allowdisplaybreaks{}
    \begin{align*}
        \det A & =
        \begin{vmatrix}
            3 & 4 & 1 & 2 \\
            3 & 5 & 3 & 5 \\
            6 & 8 & 1 & 5 \\
            3 & 5 & 3 & 7
        \end{vmatrix}
        =
        \begin{vmatrix}
            3 & 4 & 1  & 2 \\
            0 & 1 & 2  & 3 \\
            0 & 0 & -1 & 1 \\
            0 & 1 & 2  & 5
        \end{vmatrix} \\
               & =
        \begin{vmatrix}
            3 & 4 & 1  & 2 \\
            0 & 1 & 2  & 3 \\
            0 & 0 & -1 & 1 \\
            0 & 0 & 0  & 2
        \end{vmatrix}
        = -6.
    \end{align*}
    \endgroup{}
    \par Áp dụng công thức Cramer
    \[
        x_{1} = \dfrac{
            \begin{vmatrix}
                -3 & 4 & 1 & 2 \\
                -6 & 5 & 3 & 5 \\
                -8 & 8 & 1 & 5 \\
                -8 & 5 & 3 & 7
            \end{vmatrix}
        }{\det A} = \dfrac{-12}{-6} = 2,
    \]
    \[
        x_{2} = \dfrac{
            \begin{vmatrix}
                3 & -3 & 1 & 2 \\
                3 & -6 & 3 & 5 \\
                6 & -8 & 1 & 5 \\
                3 & -8 & 3 & 7
            \end{vmatrix}
        }{\det A} = \dfrac{12}{-6} = -2,
    \]
    \[
        x_{3} = \dfrac{
            \begin{vmatrix}
                3 & 4 & -3 & 2 \\
                3 & 5 & -6 & 5 \\
                6 & 8 & -8 & 5 \\
                3 & 5 & -8 & 7
            \end{vmatrix}
        }{\det A} = \dfrac{-6}{-6} = 1,
    \]
    \[
        x_{4} = \dfrac{
            \begin{vmatrix}
                3 & 4 & 1 & -3 \\
                3 & 5 & 3 & -6 \\
                6 & 8 & 1 & -8 \\
                3 & 5 & 3 & -8
            \end{vmatrix}
        }{\det A} = \dfrac{6}{-6} = -1.
    \]
    \bigskip
    \par Sử dụng phương pháp khử.
    \begingroup{}
    \allowdisplaybreaks{}
    \begin{gather*}
        \left(\begin{array}{cccc|c}
                3 & 4 & 1 & 2 & -3 \\
                3 & 5 & 3 & 5 & -6 \\
                6 & 8 & 1 & 5 & -8 \\
                3 & 5 & 3 & 7 & -8
            \end{array}
        \right)
        \Longleftrightarrow{}
        \left(\begin{array}{cccc|c}
                3 & 4 & 1  & 2 & -3 \\
                0 & 1 & 2  & 3 & -3 \\
                0 & 0 & -1 & 1 & -2 \\
                0 & 1 & 2  & 5 & -5
            \end{array}
        \right)
        \Longleftrightarrow{}
        \left(\begin{array}{cccc|c}
                3 & 4 & 1  & 2 & -3 \\
                0 & 1 & 2  & 3 & -3 \\
                0 & 0 & -1 & 1 & -2 \\
                0 & 0 & 0  & 2 & -2
            \end{array}
        \right).
    \end{gather*}
    \endgroup{}
    \par Suy ra $x_{4} = -1$, $x_{3} = 1$, $x_{2} = -2$, $x_{1} = 2$.
\end{proof}

% exercise 3.27
\begin{exercise}
    \par Chứng minh rằng một đa thức bậc $n$ trong $\mathbb{F}[X]$ được hoàn toàn xác định bởi giá trị của nó lại $(n+1)$ điểm khác nhau của trường $\mathbb{F}$. Tìm ví dụ về hai đa thức khác nhau cùng bậc $n$ nhận các giá trị bằng nhau tại mọi điểm của $\mathbb{F}$, nếu số phần tử của $\mathbb{F}$ không vượt quá $n$.
\end{exercise}

\begin{proof}[Lời giải]
    \par Giả sử ta có $n + 1$ cặp giá trị $(x_{i}, y_{i})$, trong đó $i \in \{ 0; 1; \ldots; n \}$ sao cho $x_{i} \ne x_{j}, \forall i\ne j$.
    \par Một đa thức $f(X) = a_{0} + a_{1}X + \cdots + a_{n}X^{n}$ bậc $n$ thỏa mãn $f(x_{i}) = y_{i}, \forall i$ nếu và chỉ nếu hệ phương trình tuyến tính sau có nghiệm
    \begin{align*}
         & a_{0} + a_{1}x_{0} + \cdots + a_{n}x_{0}^{n} = y_{0} \\
         & a_{0} + a_{1}x_{1} + \cdots + a_{n}x_{1}^{n} = y_{1} \\
         & a_{0} + a_{1}x_{2} + \cdots + a_{n}x_{2}^{n} = y_{2} \\
         & \vdots                                               \\
         & a_{0} + a_{1}x_{n} + \cdots + a_{n}x_{n}^{n} = y_{n} \\
    \end{align*}
    \par Hệ phương trình tuyến tính này có $(n+1)$ ẩn $a_{0}, a_{1}, \ldots, a_{n}$ và $(n+1)$ phương trình. Bên cạnh đó, hệ này có ma trận hệ số là ma trận Vandermonde của $(n+1)$ biến đôi một khác nhau, tức là định thức của ma trận hệ số khác không.
    \par Do đó hệ phương trình tuyến tính trên có nghiệm duy nhất. Điều này cũng chứng tỏ đa thức $f(X)$ bậc $n$ được xác định duy nhất bởi giá trị của nó tại $(n+1)$ điểm khác nhau.
    \bigskip
    \bigskip
    \par Nếu $n < 3$, không có ví dụ nào như vậy, vì số phần tử của một trường luôn lớn hơn hoặc bằng 2.
    \par Nếu $n\ge 3$, ta chọn trường $\mathbb{F}_{2}$ và lấy hai đa thức phân biệt:
    \[
        \begin{cases}
            f(X) = X{(X-1)}^{n-1}, \\
            g(X) = X^{n-1}(X-1).
        \end{cases}
    \]
    \par Hai đa thức này luôn cùng nhận giá trị 0 tại mọi điểm của $\mathbb{F}_{2}$.
\end{proof}

\par Giải các hệ phương trình sau đây bằng phương pháp thích hợp:

% exercise 3.28
\begin{exercise}
    \begin{align*}
        \phantom{-}ax + by + cz + dt & = p, \\
        -bx + ay + dz - ct           & = q, \\
        -cx - dy + az + bt           & = r, \\
        -dx + cy - bz + at           & = s.
    \end{align*}
\end{exercise}

\begin{proof}[Lời giải]
    \par Ta tính định thức của ma trận hệ số
    \begingroup{}
    \allowdisplaybreaks{}
    \begin{align*}
        \begin{vmatrix}
            a  & b  & c  & d  \\
            -b & a  & d  & -c \\
            -c & -d & a  & b  \\
            -d & c  & -b & a
        \end{vmatrix}
         & = {(-1)}^{1+2+1+2}
        \begin{vmatrix}
            a  & b \\
            -b & a
        \end{vmatrix}
        \begin{vmatrix}
            a  & b \\
            -b & a
        \end{vmatrix}
        + {(-1)}^{1+2+1+3}
        \begin{vmatrix}
            a  & b  \\
            -c & -d
        \end{vmatrix}
        \begin{vmatrix}
            d  & -c \\
            -b & a
        \end{vmatrix}
        + {(-1)}^{1+2+1+4}
        \begin{vmatrix}
            a  & b \\
            -d & c
        \end{vmatrix}
        \begin{vmatrix}
            d & -c \\
            a & b
        \end{vmatrix}                                                                                                 \\
         & + {(-1)}^{1+2+2+3}
        \begin{vmatrix}
            -b & a  \\
            -c & -d
        \end{vmatrix}
        \begin{vmatrix}
            c  & d \\
            -b & a
        \end{vmatrix}
        + {(-1)}^{1+2+2+4}
        \begin{vmatrix}
            -b & a \\
            -d & c
        \end{vmatrix}
        \begin{vmatrix}
            c & d \\
            a & b
        \end{vmatrix}
        + {(-1)}^{1+2+3+4}
        \begin{vmatrix}
            -c & -d \\
            -d & c
        \end{vmatrix}
        \begin{vmatrix}
            c & d  \\
            d & -c
        \end{vmatrix}                                                                                                 \\
         & = {(a^{2}+b^{2})}^{2} + {(ad-bc)}^{2} + {(ac+bd)}^{2} + {(ac+bd)}^{2} + {(bc-ad)}^{2} + {(c^{2}+d^{2})}^{2} \\
         & = {(a^{2}+b^{2})}^{2} + {(c^{2}+d^{2})}^{2} + 2{(ac+bd)}^{2} + 2{(ad-bc)}^{2}                               \\
         & = {(a^{2}+b^{2})}^{2} + {(c^{2}+d^{2})}^{2} + 2(a^{2}+b^{2})(c^{2}+d^{2})                                   \\
         & = {(a^{2}+b^{2}+c^{2}+d^{2})}^{2}.
    \end{align*}
    \endgroup{}
    \par Nếu $a^{2} + b^{2} + c^{2} + d^{2} \ne 0$ thì hệ phương trình tuyến tính trên có nghiệm duy nhất:
    \[
        x_{1} = \dfrac{1}{(a^{2} + b^{2} + c^{2} + d^{2}){}^{2}}
        \begin{vmatrix}
            p & b  & c  & d  \\
            q & a  & d  & -c \\
            r & -d & a  & b  \\
            s & c  & -b & a
        \end{vmatrix},
    \]
    \[
        x_{2} = \dfrac{1}{(a^{2} + b^{2} + c^{2} + d^{2}){}^{2}}
        \begin{vmatrix}
            a  & p & c  & d  \\
            -b & q & d  & -c \\
            -c & r & a  & b  \\
            -d & s & -b & a
        \end{vmatrix},
    \]
    \[
        x_{3} = \dfrac{1}{(a^{2} + b^{2} + c^{2} + d^{2}){}^{2}}
        \begin{vmatrix}
            a  & b  & p & d  \\
            -b & a  & q & -c \\
            -c & -d & r & b  \\
            -d & c  & s & a
        \end{vmatrix},
    \]
    \[
        x_{4} = \dfrac{1}{(a^{2} + b^{2} + c^{2} + d^{2}){}^{2}}
        \begin{vmatrix}
            a  & b  & c  & p \\
            -b & a  & d  & q \\
            -c & -d & a  & r \\
            -d & c  & -b & s
        \end{vmatrix}.
    \]
    \par Ngược lại, nếu $a^{2} + b^{2} + c^{2} + d^{2} = 0$ thì $a = b = c = d = 0$. Nếu $(p, q, r, s) \ne (0, 0, 0, 0)$ thì hệ phương trình vô nghiệm, ngược lại, tập nghiệm của hệ phương trình là $\mathbb{R}{}^{4}$.
\end{proof}

% exercise 3.29
\begin{exercise}
    \[
        \begin{array}{ccccccccccccc}
            x_{n}  & +      & a_{1}x_{n-1} & +      & a_{1}^{2}x_{n-2} & +      & \cdots & +      & a_{1}^{n-1}x_{1} & +      & a_{1}^{n} & =      & 0      \\
            x_{n}  & +      & a_{2}x_{n-1} & +      & a_{2}^{2}x_{n-2} & +      & \cdots & +      & a_{2}^{n-1}x_{1} & +      & a_{2}^{n} & =      & 0      \\
            \vdots & \vdots & \vdots       & \vdots & \vdots           & \vdots & \ddots & \vdots & \vdots           & \vdots & \vdots    & \vdots & \vdots \\
            x_{n}  & +      & a_{n}x_{n-1} & +      & a_{n}^{2}x_{n-2} & +      & \cdots & +      & a_{n}^{n-1}x_{1} & +      & a_{n}^{n} & =      & 0      \\
        \end{array}
    \]
\end{exercise}

\begin{proof}[Lời giải]
    \par $A$ là ma trận hệ số của hệ phương trình tuyến tính trên.
    \par Như vậy $A$ là một ma trận Vandermonde.

    \begin{enumerate}[label = \textbf{Trường hợp \arabic*.},itemindent=2cm]
        \item $a_{1}$, $a_{2}$, \ldots $a_{n}$ đôi một khác nhau.
              \par Lúc này, định thức Vandermonde của nó khác không.
              \par Áp dụng công thức Cramer:
              \[
                  x_{n-k} = \dfrac{\det A_{k}}{\det A}
              \]
              \par trong đó $A_{k}$ là ma trận $A$ sau khi thay cột $\begin{pmatrix} a_{1}^{k} \\ a_{2}^{k} \\ \vdots \\ a_{n}^{k} \end{pmatrix}$ bởi $\begin{pmatrix}-a_{1}^{n} \\ -a_{2}^{n} \\ \vdots \\ -a_{n}^{n} \end{pmatrix}$.
              \par Sử dụng kết quả từ bài toán~\ref{chapter3:vandermonde-and-symmetric-polynomials}:
              \[
                  \det A_{k} = (-1)(-1){}^{n-k-1}
                  \begin{vmatrix}
                      1      & a_{1}  & \cdots & a_{1}^{k-1} & a_{1}^{k+1} & \cdots & a_{1}^{n} \\
                      1      & a_{1}  & \cdots & a_{1}^{k-1} & a_{1}^{k+1} & \cdots & a_{1}^{n} \\
                      \vdots & \vdots & \ddots & \vdots      & \vdots      & \ddots & \vdots    \\
                      1      & a_{1}  & \cdots & a_{1}^{k-1} & a_{1}^{k+1} & \cdots & a_{1}^{n} \\
                  \end{vmatrix}
                  = (-1){}^{n-k}D_{n}e_{n-k}.
              \]
              \par Suy ra $x_{n-k} = (-1){}^{n-k}e_{n-k}(a_{1},\ldots, a_{n})$.
              \[
                  \begin{cases}
                      x_{1} = (-1)e_{1}(a_{1},\ldots,a_{n})       \\
                      x_{2} = (-1){}^{2}e_{2}(a_{1},\ldots,a_{n}) \\
                      \vdots                                      \\
                      x_{n} = (-1){}^{n}e_{n}(a_{1},\ldots,a_{n})
                  \end{cases}
              \]
              \par trong đó, nhắc lại rằng $e_{k}$ là đa thức đối xứng sơ cấp bậc $k$.
        \item Trong $n$ hệ số $a_{1}$, $a_{2}$, \ldots $a_{n}$, có ít nhất hai hệ số bằng nhau.
              \par Giả sử rằng, sau khi loại bỏ các hệ số dư thừa, ta còn lại $m$ hệ số ($m < n$). Không giảm tổng quát, có thể đánh số lại các hệ số. $m$ hệ số đôi một khác nhau được đánh số lại là $a_{1}$, $a_{2}$, \ldots $a_{m}$.
              \par Thực hiện các phép biến đổi sơ cấp trên các hàng của ma trận $m\times(n+1)$ sau:
              \begingroup{}
              \allowdisplaybreaks{}
              \begin{align*}
                                      &
                  \begin{pmatrix}
                      1      & a_{1}  & a_{1}^{2} & \cdots & a_{1}^{n-1} & a_{1}^{n} \\
                      1      & a_{2}  & a_{2}^{2} & \cdots & a_{2}^{n-1} & a_{2}^{n} \\
                      \vdots & \vdots & \vdots    & \ddots & \vdots      & \vdots    \\
                      1      & a_{m}  & a_{m}^{2} & \cdots & a_{m}^{n-1} & a_{m}^{n}
                  \end{pmatrix}                                                                                                                         \\
                  \Longleftrightarrow &
                  \begin{pmatrix}
                      1      & a_{1}         & a_{1}^{2}             & \cdots & a_{1}^{n-1}               & a_{1}^{n}             \\
                      0      & a_{2} - a_{1} & a_{2}^{2} - a_{1}^{2} & \cdots & a_{2}^{n-1} - a_{1}^{n-1} & a_{2}^{n} - a_{1}^{n} \\
                      \vdots & \vdots        & \vdots                & \ddots & \vdots                    & \vdots                \\
                      0      & a_{m} - a_{1} & a_{m}^{2} - a_{1}^{2} & \cdots & a_{m}^{n-1} - a_{1}^{n-1} & a_{m}^{n} - a_{m}^{n}
                  \end{pmatrix}                                                                            \\
                  \Longleftrightarrow &
                  \begin{pmatrix}
                      1      & a_{1}  & a_{1}^{2}           & a_{1}^{3}           & \cdots & a_{1}^{n-1}           & a_{1}^{n}             \\
                      0      & 1      & h_{1}(a_{1}, a_{2}) & h_{2}(a_{1}, a_{2}) & \cdots & h_{n-2}(a_{1}, a_{2}) & h_{n-1}(a_{1}, a_{2}) \\
                      0      & 1      & h_{1}(a_{1}, a_{3}) & h_{2}(a_{1}, a_{3}) & \cdots & h_{n-2}(a_{1}, a_{2}) & h_{n-1}(a_{1}, a_{2}) \\
                      \vdots & \vdots & \vdots              & \vdots              & \ddots & \vdots                & \vdots                \\
                      0      & 1      & h_{1}(a_{1}, a_{m}) & h_{2}(a_{1}, a_{m}) & \cdots & h_{n-2}(a_{1}, a_{m}) & h_{n-1}(a_{1}, a_{m})
                  \end{pmatrix}                                                                   \\
                  \Longleftrightarrow &
                  \begin{pmatrix}
                      1      & a_{1}  & a_{1}^{2}           & a_{1}^{3}                                 & \cdots & a_{1}^{n-1}                                 & a_{1}^{n}                                   \\
                      0      & 1      & h_{1}(a_{1}, a_{2}) & h_{2}(a_{1}, a_{2})                       & \cdots & h_{n-2}(a_{1}, a_{2})                       & h_{n-1}(a_{1}, a_{2})                       \\
                      0      & 0      & a_{3} - a_{2}       & (a_{3} - a_{2})h_{1}(a_{1}, a_{2}, a_{3}) & \cdots & (a_{3} - a_{2})h_{n-3}(a_{1}, a_{2}, a_{3}) & (a_{3} - a_{2})h_{n-2}(a_{1}, a_{2}, a_{3}) \\
                      \vdots & \vdots & \vdots              & \vdots                                    & \ddots & \vdots                                      & \vdots                                      \\
                      0      & 0      & a_{m} - a_{2}       & (a_{m} - a_{2})h_{1}(a_{1}, a_{2}, a_{m}) & \cdots & (a_{m} - a_{2})h_{n-3}(a_{1}, a_{2}, a_{m}) & (a_{m} - a_{2})h_{n-2}(a_{1}, a_{m})
                  \end{pmatrix} \\
                  \Longleftrightarrow &
                  \begin{pmatrix}
                      1      & a_{1}  & a_{1}^{2}           & a_{1}^{3}                  & \cdots & a_{1}^{n-1}                  & a_{1}^{n}                    \\
                      0      & 1      & h_{1}(a_{1}, a_{2}) & h_{2}(a_{1}, a_{2})        & \cdots & h_{n-2}(a_{1}, a_{2})        & h_{n-1}(a_{1}, a_{2})        \\
                      0      & 0      & 1                   & h_{1}(a_{1}, a_{2}, a_{3}) & \cdots & h_{n-3}(a_{1}, a_{2}, a_{3}) & h_{n-2}(a_{1}, a_{2}, a_{3}) \\
                      \vdots & \vdots & \vdots              & \vdots                     & \ddots & \vdots                       & \vdots                       \\
                      0      & 0      & 1                   & h_{1}(a_{1}, a_{2}, a_{m}) & \cdots & h_{n-2}(a_{1}, a_{2}, a_{m}) & h_{n-2}(a_{1}, a_{2}, a_{m})
                  \end{pmatrix}                                              \\
                                      & \ddots                                                                                                                                                           \\
                                      & \Longleftrightarrow
                  \begin{pmatrix}
                      1      & a_{1}  & a_{1}^{2}           & \cdots & a_{1}^{m-1}                  & \cdots & a_{1}^{n-1}                  & a_{1}^{n}                      \\
                      0      & 1      & h_{1}(a_{1}, a_{2}) & \cdots & h_{m-2}(a_{1}, a_{2})        & \cdots & h_{n-2}(a_{1}, a_{2})        & h_{n-1}(a_{1}, a_{2})          \\
                      0      & 0      & 1                   & \cdots & h_{m-3}(a_{1}, a_{2}, a_{3}) & \cdots & h_{n-3}(a_{1}, a_{2}, a_{3}) & h_{n-2}(a_{1}, a_{2}, a_{3})   \\
                      \vdots & \vdots & \vdots              & \vdots & \ddots                       & \vdots & \vdots                       & \vdots                         \\
                      0      & 0      & 0                   & \cdots & 1                            & \cdots & h_{n-m}(a_{1},\ldots, a_{m}) & h_{n-m+1}(a_{1},\ldots, a_{m})
                  \end{pmatrix}                                 \\
              \end{align*}
              \endgroup{}
              \par (trong đó, nhắc lại $h_{k}$ là đa thức đối xứng thuần nhất đầy đủ).
              \par Như vậy, hệ phương trình tuyến tính có nghiệm.
              \par $x_{n-m}$, $x_{n-m-1}$, \ldots, $x_{0}$ có thể nhận giá trị bất kì.
              \par $x_{n-m+1}$ được xác định từ phương trình thứ $m$.
              \par $x_{n-m+2}$ được xác định từ phương trình thứ $m-1$.
              \par $\ddots$
              \par $x_{n}$ được xác định từ phương trình thứ 1.
    \end{enumerate}
\end{proof}

% exercise 3.30
\begin{exercise}
    \par Đặt $s_{n}(k) = 1^{n} + 2^{n} + \cdots + (k-1){}^{n}$. Hãy thiết lập phương trình
    \[
        k^{n} = 1 + \binom{n}{n-1}s_{n-1}(k) + \cdots + \binom{n}{1}s_{1}(k) + s_{0}(k)
    \]
    \par và chứng minh rằng
    \[
        s_{n-1}(k) = \frac{1}{n!}
        \begin{vmatrix}
            k^{n}   & \binom{n}{n-2}   & \binom{n}{n-3}   & \cdots & \binom{n}{1}   & 1      \\
            k^{n-1} & \binom{n-1}{n-2} & \binom{n-1}{n-3} & \cdots & \binom{n-1}{1} & 1      \\
            k^{n-2} & 0                & \binom{n-2}{n-3} & \cdots & \binom{n-2}{1} & 1      \\
            \vdots  & \vdots           & \vdots           & \ddots & \vdots         & \vdots \\
            k^{2}   & 0                & 0                & \cdots & \binom{2}{1}   & 1      \\
            k       & 0                & 0                & \cdots & 0              & 1
        \end{vmatrix}.
    \]
\end{exercise}

\begin{proof}
    \par Áp dụng định lý nhị thức Newton:
    \begin{align*}
        k^{n}       & = 1 + \binom{n}{1}(k-1) + \cdots + \binom{n}{n-1}(k-1){}^{n-1} + (k-1){}^{n} \\
        (k-1){}^{n} & = 1 + \binom{n}{1}(k-2) + \cdots + \binom{n}{n-1}(k-2){}^{n-1} + (k-2){}^{n} \\
                    & \vdots                                                                       \\
        2^{n}       & = 1 + \binom{n}{1}1     + \cdots + \binom{n}{n-1}1^{n-1} + 1^{n}
    \end{align*}
    \par Cộng vế theo vế của tất cả đẳng thức trên:
    \begin{align*}
        k^{n} + (k-1){}^{n} + \cdots + 2^{n} & = s_{0}(k) + \binom{n}{1}s_{1}(k) + \cdots + \binom{n}{n-1}s_{n-1}(k) + (k-1){}^{n} + \cdots + 1 \\
        \Longleftrightarrow k^{n}            & = s_{0}(k) + \binom{n}{1}s_{1}(k) + \cdots + \binom{n}{n-1}s_{n-1}(k) + 1                        \\
        \Longleftrightarrow k^{n}            & = 1 + \binom{n}{n-1}s_{n-1}(k) + \cdots + \binom{n}{1}s_{1}(k) + s_{0}(k).
    \end{align*}

    \par Áp dụng công thức trên với các giá trị $n$ nhỏ hơn:
    \begin{align*}
        k^{n}   & = 1 + \binom{n}{n-1}s_{n-1}(k) + \cdots + \binom{n}{1}s_{1}(k) + s_{0}(k)     \\
        k^{n-1} & = 1 + \binom{n-1}{n-2}s_{n-2}(k) + \cdots + \binom{n-1}{1}s_{1}(k) + s_{0}(k) \\
                & \ddots                                                                        \\
        k       & = 1 + s_{0}(k)
    \end{align*}
    \par $s_{n-1}(k)$, $s_{n-2}(k)$, \ldots, $s_{0}(k)$ là nghiệm của hệ phương trình tuyến tính:
    \[
        \begin{array}{ccccccccccc}
            \binom{n}{n-1}x_{n-1} & + & \binom{n}{n-2}x_{n-2}   & + & \cdots & + & \binom{n}{1}x_{1}   & + & x_{0} & =      & k^{n} - 1   \\
                                  &   & \binom{n-1}{n-2}x_{n-2} & + & \cdots & + & \binom{n-1}{1}x_{1} & + & x_{0} & =      & k^{n-1} - 1 \\
                                  &   &                         &   & \ddots &   &                     &   &       & \vdots &             \\
                                  &   &                         &   &        &   &                     &   & x_{0} & =      & k - 1
        \end{array}
    \]
    \par Định thức của ma trận hệ số bằng:
    \[
        \begin{vmatrix}
            \binom{n}{n-1} & \binom{n}{n-2}   & \cdots & \binom{n}{1}   & 1      \\
            0              & \binom{n-1}{n-2} & \cdots & \binom{n-1}{1} & 1      \\
            \vdots         & \vdots           & \ddots & \vdots         & \vdots \\
            0              & 0                & \cdots & 0              & 1
        \end{vmatrix}
        = \binom{n}{n-1}\binom{n-1}{n-2}\cdots\binom{2}{1}
        = n!
    \]
    \par Áp dụng công thức Cramer:
    \[
        s_{n-1}(k) = x_{n-1} = \dfrac{1}{n!}
        \begin{vmatrix}
            k^{n} - 1   & \binom{n}{n-2} & \binom{n}{n-3}   & \cdots & \binom{n}{1}   & 1      \\
            k^{n-1} - 1 & 0              & \binom{n-1}{n-3} & \cdots & \binom{n-1}{1} & 1      \\
            k^{n-2} - 1 & 0              & 0                & \cdots & \binom{n-2}{1} & 1      \\
            \vdots      & \vdots         & \vdots           & \ddots & \vdots         & \vdots \\
            k - 1       & 0              & 0                & \cdots & 0              & 1
        \end{vmatrix}
        = \dfrac{1}{n!}
        \begin{vmatrix}
            k^{n}   & \binom{n}{n-2} & \binom{n}{n-3}   & \cdots & \binom{n}{1}   & 1      \\
            k^{n-1} & 0              & \binom{n-1}{n-3} & \cdots & \binom{n-1}{1} & 1      \\
            k^{n-2} & 0              & 0                & \cdots & \binom{n-2}{1} & 1      \\
            \vdots  & \vdots         & \vdots           & \ddots & \vdots         & \vdots \\
            k       & 0              & 0                & \cdots & 0              & 1
        \end{vmatrix}.
    \]
\end{proof}

% exercise 3.31
\begin{exercise}
    \par Xét khai triển $\frac{x}{e^{x} - 1} = 1 + b_{1}x + b_{2}x^{2} + b_{3}x^{3} + \cdots$. Ta đặt $b_{2n} = \frac{(-1){}^{n-1}B_{n}}{(2n)!}$, trong đó $B_{n}$ được gọi là số Bernoullia thứ $n$. Chứng minh rằng
    \[
        B_{n} = (-1){}^{n-1}(2n)!
        \begin{vmatrix}
            \frac{1}{2!}      & 1               & 0                 & 0                 & \cdots & 0            \\
            \frac{1}{3!}      & \frac{1}{2!}    & 1                 & 0                 & \cdots & 0            \\
            \frac{1}{4!}      & \frac{1}{3!}    & \frac{1}{2!}      & 1                 & \cdots & 0            \\
            \vdots            & \vdots          & \vdots            & \vdots            & \ddots & \vdots       \\
            \frac{1}{(2n+1)!} & \frac{1}{(2n)!} & \frac{1}{(2n-1)!} & \frac{1}{(2n-2)!} & \cdots & \frac{1}{2!}
        \end{vmatrix},
    \]
    \par và chỉ ra rằng
    \[
        b_{2n-1} =
        \begin{vmatrix}
            \frac{1}{2!}    & 1                 & 0                 & 0                 & \cdots & 0            \\
            \frac{1}{3!}    & \frac{1}{2!}      & 1                 & 0                 & \cdots & 0            \\
            \frac{1}{4!}    & \frac{1}{3!}      & \frac{1}{2!}      & 1                 & \cdots & 0            \\
            \vdots          & \vdots            & \vdots            & \vdots            & \ddots & \vdots       \\
            \frac{1}{(2n)!} & \frac{1}{(2n-1)!} & \frac{1}{(2n-2)!} & \frac{1}{(2n-3)!} & \cdots & \frac{1}{2!}
        \end{vmatrix}
        = 0
    \]
    \par với mọi $n > 1$.
\end{exercise}

\begin{proof}
    \begingroup{}
    \allowdisplaybreaks{}
    \begin{align*}
        \dfrac{x}{e^{x}-1} & = 1 + b_{1}x + b_{2}x^{2} + b_{3}x^{3} + \cdots                                                                                  \\
        \Leftrightarrow 1  & = (1 + b_{1}x + b_{2}x^{2} + b_{3}x^{3} + \cdots)\left(1 + \dfrac{x}{2!} + \dfrac{x^{2}}{3!} + \dfrac{x^{3}}{4!} + \cdots\right)
    \end{align*}
    \endgroup{}
    \par Đồng nhất hệ số của $x$, $x^{2}$, $x^{3}$, \ldots, $x^{2n}$ trong khai triển trên, ta được hệ phương trình tuyến tính gồm $(2n)$ phương trình:
    \begin{align*}
         & b_{1}                                                               & = \dfrac{-1}{2!}      \\
         & \dfrac{b_{1}}{2!} + b_{2}                                           & = \dfrac{-1}{3!}      \\
         & \dfrac{b_{1}}{3!} + \dfrac{b_{2}}{2!} + b_{3}                       & = \dfrac{-1}{4!}      \\
         &                                                                     & \vdots                \\
         & \dfrac{b_{1}}{(2n-1)!} + \dfrac{b_{2}}{(2n-2)!} + \cdots + b_{2n-1} & = \dfrac{-1}{(2n)!}   \\
         & \dfrac{b_{1}}{(2n)!} + \dfrac{b_{2}}{(2n-1)!} + \cdots + b_{2n}     & = \dfrac{-1}{(2n+1)!}
    \end{align*}
    \par Định thức của ma trận hệ số của hệ $(2n)$ phương trình tuyến tính này bằng:
    \[
        \begin{vmatrix}
            1               & 0                 & 0                 & \cdots & 0            & 0      \\
            \frac{1}{2!}    & 1                 & 0                 & \cdots & 0            & 0      \\
            \frac{1}{3!}    & \frac{1}{2!}      & 1                 & \cdots & 0            & 0      \\
            \vdots          & \vdots            & \vdots            & \ddots & \vdots       & \vdots \\
            \frac{1}{(2n)!} & \frac{1}{(2n-1)!} & \frac{1}{(2n-2)!} & \cdots & \frac{1}{2!} & 1
        \end{vmatrix} = 1 \ne 0.
    \]
    \par Áp dụng công thức Cramer
    \begingroup{}
    \allowdisplaybreaks{}
    \begin{align*}
        b_{2n} & =
        \begin{vmatrix}
            1               & 0                 & 0                 & \cdots & 0            & \frac{-1}{2!}      \\
            \frac{1}{2!}    & 1                 & 0                 & \cdots & 0            & \frac{-1}{3!}      \\
            \frac{1}{3!}    & \frac{1}{2!}      & 1                 & \cdots & 0            & \frac{-1}{4!}      \\
            \vdots          & \vdots            & \vdots            & \ddots & \vdots       & \vdots             \\
            \frac{1}{(2n)!} & \frac{1}{(2n-1)!} & \frac{1}{(2n-2)!} & \cdots & \frac{1}{2!} & \frac{-1}{(2n+1)!}
        \end{vmatrix} \\
               & = (-1){}^{2n-1}
        \begin{vmatrix}
            \frac{-1}{2!}      & 1               & 0                 & 0                 & \cdots & 0            \\
            \frac{-1}{3!}      & \frac{1}{2!}    & 1                 & 0                 & \cdots & 0            \\
            \frac{-1}{4!}      & \frac{1}{3!}    & \frac{1}{2!}      & 1                 & \cdots & 0            \\
            \vdots             & \vdots          & \vdots            & \vdots            & \ddots & \vdots       \\
            \frac{-1}{(2n+1)!} & \frac{1}{(2n)!} & \frac{1}{(2n-1)!} & \frac{1}{(2n-2)!} & \cdots & \frac{1}{2!}
        \end{vmatrix} \\
               & =
        \begin{vmatrix}
            \frac{1}{2!}      & 1               & 0                 & 0                 & \cdots & 0            \\
            \frac{1}{3!}      & \frac{1}{2!}    & 1                 & 0                 & \cdots & 0            \\
            \frac{1}{4!}      & \frac{1}{3!}    & \frac{1}{2!}      & 1                 & \cdots & 0            \\
            \vdots            & \vdots          & \vdots            & \vdots            & \ddots & \vdots       \\
            \frac{1}{(2n+1)!} & \frac{1}{(2n)!} & \frac{1}{(2n-1)!} & \frac{1}{(2n-2)!} & \cdots & \frac{1}{2!}
        \end{vmatrix}  \\
    \end{align*}
    \endgroup{}
    \par Do đó
    \[
        B_{n} = (-1){}^{n-1}(2n)!
        \begin{vmatrix}
            \frac{1}{2!}      & 1               & 0                 & 0                 & \cdots & 0            \\
            \frac{1}{3!}      & \frac{1}{2!}    & 1                 & 0                 & \cdots & 0            \\
            \frac{1}{4!}      & \frac{1}{3!}    & \frac{1}{2!}      & 1                 & \cdots & 0            \\
            \vdots            & \vdots          & \vdots            & \vdots            & \ddots & \vdots       \\
            \frac{1}{(2n+1)!} & \frac{1}{(2n)!} & \frac{1}{(2n-1)!} & \frac{1}{(2n-2)!} & \cdots & \frac{1}{2!}
        \end{vmatrix}.
    \]

    \par Áp dụng công thức Cramer cho $2n-1$ phương trình tuyến tính đầu tiên:
    \begin{align*}
        b_{2n-1} & =
        \begin{vmatrix}
            1                 & 0                 & 0                 & \cdots & 0            & \frac{-1}{2!}    \\
            \frac{1}{2!}      & 1                 & 0                 & \cdots & 0            & \frac{-1}{3!}    \\
            \frac{1}{3!}      & \frac{1}{2!}      & 1                 & \cdots & 0            & \frac{-1}{4!}    \\
            \vdots            & \vdots            & \vdots            & \ddots & \vdots       & \vdots           \\
            \frac{1}{(2n-1)!} & \frac{1}{(2n-2)!} & \frac{1}{(2n-3)!} & \cdots & \frac{1}{2!} & \frac{-1}{(2n)!}
        \end{vmatrix} \\
                 & = -
        \begin{vmatrix}
            \frac{1}{2!}    & 1                 & 0                 & \cdots & 0            & 0            \\
            \frac{1}{3!}    & \frac{1}{2!}      & 1                 & \cdots & 0            & 0            \\
            \frac{1}{4!}    & \frac{1}{3!}      & \frac{1}{2!}      & \cdots & 0            & 0            \\
            \vdots          & \vdots            & \vdots            & \ddots & \vdots       & \vdots       \\
            \frac{1}{(2n)!} & \frac{1}{(2n-1)!} & \frac{1}{(2n-2)!} & \cdots & \frac{1}{3!} & \frac{1}{2!}
        \end{vmatrix}
    \end{align*}
    \begin{align*}
        \dfrac{x}{e^{x}-1}                                             & = 1 + b_{1}x + b_{2}x^{2} + b_{3}x^{3} + \cdots \\
        \dfrac{-x}{e^{-x}-1}                                           & = 1 - b_{1}x + b_{2}x^{2} - b_{3}x^{3} + \cdots \\
        \Longleftrightarrow \dfrac{x}{e^{x} - 1} + \dfrac{x}{e^{-x}-1} & = 2b_{1}x + 2b_{3}x^{3} + \cdots                \\
        \Longleftrightarrow -x                                         & = 2b_{1}x + 2b_{3}x^{3} + \cdots
    \end{align*}
    \par Đồng nhất hệ số của hai đa thức, ta được $b_{1} = \dfrac{-1}{2}$, $b_{3} = b_{5} = \cdots = b_{2n-1} = \cdots = 0$.
    \par Vậy với $n > 1$
    \[
        b_{2n-1} =
        \begin{vmatrix}
            \frac{1}{2!}    & 1                 & 0                 & \cdots & 0            & 0            \\
            \frac{1}{3!}    & \frac{1}{2!}      & 1                 & \cdots & 0            & 0            \\
            \frac{1}{4!}    & \frac{1}{3!}      & \frac{1}{2!}      & \cdots & 0            & 0            \\
            \vdots          & \vdots            & \vdots            & \ddots & \vdots       & \vdots       \\
            \frac{1}{(2n)!} & \frac{1}{(2n-1)!} & \frac{1}{(2n-2)!} & \cdots & \frac{1}{3!} & \frac{1}{2!}
        \end{vmatrix}
        = 0.
    \]
\end{proof}

% exercise 3.32
\begin{exercise}
    \par Diễn đạt hệ số $a_{n}$ trong khai triển
    \[
        e^{-x} = 1 - a_{1}x + a_{2}x^{2} - a_{3}x^{3} + \cdots ,
    \]
    \par như một định thức cỡ $n$, từ đó tính định thức thu được.
\end{exercise}

\begin{proof}[Lời giải]
    \begingroup{}
    \allowdisplaybreaks{}
    \begin{align*}
        e^{-x} & = 1 - a_{1}x + a_{2}x^{2} - a_{3}x^{3} + \cdots                                                                      \\
        1      & = (1 - a_{1}x + a_{2}x^{2} - a_{3}x^{3} + \cdots)\left(1 + x + \dfrac{x^{2}}{2!} + \dfrac{x^{3}}{3!} + \cdots\right) \\
    \end{align*}
    \endgroup{}
    \par Đồng nhất hệ số của các hạng tử $x$, $x^{2}$, $x^{3}$, \ldots, $x^{n}$, ta thu được hệ phương trình tuyến tính:
    \begin{align*}
        (-1){}^{1}a_{1} + \frac{1}{1!}                                                                              & = 0    \\
        (-1){}^{2}a_{2} + (-1){}^{1}a_{1}\frac{1}{1!} + \frac{1}{2!}                                                & = 0    \\
        (-1){}^{3}a_{3} + (-1){}^{2}a_{2}\frac{1}{1!} + (-1){}^{1}a_{1}\frac{1}{2!} + \frac{1}{3!}                  & = 0    \\
                                                                                                                    & \ddots \\
        (-1){}^{n}a_{n} + (-1){}^{n-1}a_{n-1}\frac{1}{1!} + \cdots + (-1){}^{1}a_{1}\frac{1}{(n-1)!} + \frac{1}{n!} & = 0
    \end{align*}
    \par $(-1){}^{1}a_{1}$, $(-1){}^{2}a_{2}$, \ldots, $(-1){}^{n}a_{n}$ là nghiệm của hệ phương tình tuyến tính:
    \begin{align*}
         & x_{1}                                                                                & = \frac{-1}{1!} \\
         & \frac{1}{1!}x_{1} + x_{2}                                                            & = \frac{-1}{2!} \\
         & \frac{1}{2!}x_{1} + \frac{1}{1!}x_{2} + x_{3}                                        & = \frac{-1}{3!} \\
         & \ddots                                                                               &                 \\
         & \frac{1}{(n-1)!}x_{1} + \frac{1}{(n-2)!}x_{2} + \cdots + \frac{1}{1!}x_{n-1} + x_{n} & = \frac{-1}{n!}
    \end{align*}
    \par Định thức của ma trận hệ số bằng 1. Áp dụng công thức Cramer:
    \begin{align*}
        (-1){}^{n}a_{n} = x_{n}   & =
        \begin{vmatrix}
            1                & 0                & 0                & \cdots & 0            & \frac{-1}{1!} \\
            \frac{1}{1!}     & 1                & 0                & \cdots & 0            & \frac{-1}{2!} \\
            \frac{1}{2!}     & \frac{1}{1!}     & 1                & \cdots & 0            & \frac{-1}{3!} \\
            \vdots           & \vdots           & \vdots           & \ddots & \vdots       & \vdots        \\
            \frac{1}{(n-1)!} & \frac{1}{(n-2)!} & \frac{1}{(n-3)!} & \cdots & \frac{1}{1!} & \frac{-1}{n!}
        \end{vmatrix} \\
                                  & = {(-1)}^{n-1}
        \begin{vmatrix}
            \frac{-1}{1!} & 1                & 0                & 0                & \cdots & 0            \\
            \frac{-1}{2!} & \frac{1}{1!}     & 1                & 0                & \cdots & 0            \\
            \frac{-1}{3!} & \frac{1}{2!}     & \frac{1}{1!}     & 1                & \cdots & 0            \\
            \vdots        & \vdots           & \vdots           & \vdots           & \ddots & \vdots       \\
            \frac{-1}{n!} & \frac{1}{(n-1)!} & \frac{1}{(n-2)!} & \frac{1}{(n-3)!} & \cdots & \frac{1}{1!}
        \end{vmatrix} \\
                                  & = {(-1)}^{n}
        \begin{vmatrix}
            \frac{1}{1!} & 1                & 0                & 0                & \cdots & 0            \\
            \frac{1}{2!} & \frac{1}{1!}     & 1                & 0                & \cdots & 0            \\
            \frac{1}{3!} & \frac{1}{2!}     & \frac{1}{1!}     & 1                & \cdots & 0            \\
            \vdots       & \vdots           & \vdots           & \vdots           & \ddots & \vdots       \\
            \frac{1}{n!} & \frac{1}{(n-1)!} & \frac{1}{(n-2)!} & \frac{1}{(n-3)!} & \cdots & \frac{1}{1!}
        \end{vmatrix}  \\
        \Longleftrightarrow a_{n} & =
        \begin{vmatrix}
            \frac{1}{1!} & 1                & 0                & 0                & \cdots & 0            \\
            \frac{1}{2!} & \frac{1}{1!}     & 1                & 0                & \cdots & 0            \\
            \frac{1}{3!} & \frac{1}{2!}     & \frac{1}{1!}     & 1                & \cdots & 0            \\
            \vdots       & \vdots           & \vdots           & \vdots           & \ddots & \vdots       \\
            \frac{1}{n!} & \frac{1}{(n-1)!} & \frac{1}{(n-2)!} & \frac{1}{(n-3)!} & \cdots & \frac{1}{1!}
        \end{vmatrix}.
    \end{align*}

    \par Áp dụng khai triển Taylor:
    \[
        e^{-x} = \sum^{+\infty}_{n=0}\frac{(-1){}^{n}x^{n}}{n!}
    \]
    \par Đồng nhất hệ số với $e^{-x} = 1 - a_{1}x + a_{2}x^{2} - a_{3}x^{3} + \cdots$, ta được:
    \[
        a_{n} =
        \begin{vmatrix}
            \frac{1}{1!} & 1                & 0                & 0                & \cdots & 0            \\
            \frac{1}{2!} & \frac{1}{1!}     & 1                & 0                & \cdots & 0            \\
            \frac{1}{3!} & \frac{1}{2!}     & \frac{1}{1!}     & 1                & \cdots & 0            \\
            \vdots       & \vdots           & \vdots           & \vdots           & \ddots & \vdots       \\
            \frac{1}{n!} & \frac{1}{(n-1)!} & \frac{1}{(n-2)!} & \frac{1}{(n-3)!} & \cdots & \frac{1}{1!}
        \end{vmatrix}
        = \frac{1}{n!}.
    \]
\end{proof}

% exercise 3.33
\begin{exercise}
    \par Không dùng ma trận của tự đồng cấu, hãy chứng minh trực tiếp rằng nếu $f$ là một tự đồng cấu của không gian vector hữu hạn chiều $V$ và $f^{*}$ là đồng cấu đối ngẫu của $f$, thì $\det(f^{*}) = \det(f)$. (Gợi ý: Xét định thức của ma trận ${\left(\dotprod{\alpha_{i},\xi_{j}}\right)}_{n\times n}$, trong đó $\alpha_{1},\ldots,\alpha_{n}\in V$, $\xi_{1},\ldots ,\xi_{n}\in V^{*}$.)
\end{exercise}

\begin{proof}
    \par $(\alpha_{1}, \alpha_{2}, \ldots, \alpha_{n})$ là một cơ sở của $V$.
    \par $(\xi_{1}, \xi_{2}, \ldots, \xi_{n})$ là một cơ sở đối ngẫu của $(\alpha_{1}, \alpha_{2}, \ldots, \alpha_{n})$.
    \begingroup{}
    \allowdisplaybreaks{}
    \begin{align*}
        \begin{vmatrix}
            \dotprod{f(\alpha_{1}),\xi_{1}} & \dotprod{f(\alpha_{1}),\xi_{2}} & \cdots & \dotprod{f(\alpha_{1}),\xi_{n}} \\
            \dotprod{f(\alpha_{2}),\xi_{1}} & \dotprod{f(\alpha_{2}),\xi_{2}} & \cdots & \dotprod{f(\alpha_{2}),\xi_{n}} \\
            \vdots                          & \vdots                          & \ddots & \vdots                          \\
            \dotprod{f(\alpha_{n}),\xi_{1}} & \dotprod{f(\alpha_{n}),\xi_{2}} & \cdots & \dotprod{f(\alpha_{n}),\xi_{n}} \\
        \end{vmatrix}
         & = \det(f)
        \begin{vmatrix}
            \dotprod{\alpha_{1},\xi_{1}} & \dotprod{\alpha_{1},\xi_{2}} & \cdots & \dotprod{\alpha_{1},\xi_{n}} \\
            \dotprod{\alpha_{2},\xi_{1}} & \dotprod{\alpha_{2},\xi_{2}} & \cdots & \dotprod{\alpha_{2},\xi_{n}} \\
            \vdots                       & \vdots                       & \ddots & \vdots                       \\
            \dotprod{\alpha_{n},\xi_{1}} & \dotprod{\alpha_{n},\xi_{2}} & \cdots & \dotprod{\alpha_{n},\xi_{n}} \\
        \end{vmatrix}, \\
        \begin{vmatrix}
            \dotprod{\alpha_{1},f^{*}(\xi_{1})} & \dotprod{\alpha_{1},f^{*}(\xi_{2})} & \cdots & \dotprod{\alpha_{1},f^{*}(\xi_{n})} \\
            \dotprod{\alpha_{2},f^{*}(\xi_{1})} & \dotprod{\alpha_{2},f^{*}(\xi_{2})} & \cdots & \dotprod{\alpha_{2},f^{*}(\xi_{n})} \\
            \vdots                              & \vdots                              & \ddots & \vdots                              \\
            \dotprod{\alpha_{n},f^{*}(\xi_{1})} & \dotprod{\alpha_{n},f^{*}(\xi_{2})} & \cdots & \dotprod{\alpha_{n},f^{*}(\xi_{n})} \\
        \end{vmatrix}
         & = \det(f^{*})
        \begin{vmatrix}
            \dotprod{\alpha_{1},\xi_{1}} & \dotprod{\alpha_{1},\xi_{2}} & \cdots & \dotprod{\alpha_{1},\xi_{n}} \\
            \dotprod{\alpha_{2},\xi_{1}} & \dotprod{\alpha_{2},\xi_{2}} & \cdots & \dotprod{\alpha_{2},\xi_{n}} \\
            \vdots                       & \vdots                       & \ddots & \vdots                       \\
            \dotprod{\alpha_{n},\xi_{1}} & \dotprod{\alpha_{n},\xi_{2}} & \cdots & \dotprod{\alpha_{n},\xi_{n}} \\
        \end{vmatrix}, \\
    \end{align*}
    \par Theo định nghĩa của đồng cấu tuyến tính đối ngẫu, $f^{*}(\varphi) = \varphi\circ f, \forall\varphi\in V^{*}$, suy ra:
    \[
        \det(f)
        \begin{vmatrix}
            \dotprod{\alpha_{1},\xi_{1}} & \dotprod{\alpha_{1},\xi_{2}} & \cdots & \dotprod{\alpha_{1},\xi_{n}} \\
            \dotprod{\alpha_{2},\xi_{1}} & \dotprod{\alpha_{2},\xi_{2}} & \cdots & \dotprod{\alpha_{2},\xi_{n}} \\
            \vdots                       & \vdots                       & \ddots & \vdots                       \\
            \dotprod{\alpha_{n},\xi_{1}} & \dotprod{\alpha_{n},\xi_{2}} & \cdots & \dotprod{\alpha_{n},\xi_{n}} \\
        \end{vmatrix}
        =
        \det(f^{*})
        \begin{vmatrix}
            \dotprod{\alpha_{1},\xi_{1}} & \dotprod{\alpha_{1},\xi_{2}} & \cdots & \dotprod{\alpha_{1},\xi_{n}} \\
            \dotprod{\alpha_{2},\xi_{1}} & \dotprod{\alpha_{2},\xi_{2}} & \cdots & \dotprod{\alpha_{2},\xi_{n}} \\
            \vdots                       & \vdots                       & \ddots & \vdots                       \\
            \dotprod{\alpha_{n},\xi_{1}} & \dotprod{\alpha_{n},\xi_{2}} & \cdots & \dotprod{\alpha_{n},\xi_{n}} \\
        \end{vmatrix}
    \]
    \par Bên cạnh đó, theo định nghĩa của cơ sở đối ngẫu, $\dotprod{\alpha_{i}, \xi_{j}} = \delta_{ij}, \forall i, j\in \{ 1, 2, \ldots, n \}$. Do đó
    \[
        \det(f) = \det(f^{*}).
    \]
    \endgroup{}
\end{proof}

% exercise 3.34
\begin{exercise}
    \par Tính hạng của các ma trận sau đây bằng phương pháp biến đổi sơ cấp và phương pháp dùng định thức con:
    \begin{enumerate}[label = (\alph*)]
        \item $
                  \begin{pmatrix}
                      2 & -1 & 3 & -2 & 4 \\
                      4 & -2 & 5 & 1  & 7 \\
                      2 & -1 & 1 & 8  & 2
                  \end{pmatrix}
              $,
        \item $
                  \begin{pmatrix}
                      3 & -1 & 3  & 2 & 5  \\
                      5 & -3 & 2  & 3 & 4  \\
                      1 & -3 & -5 & 0 & -7 \\
                      7 & -5 & 1  & 4 & 1
                  \end{pmatrix}
              $.
    \end{enumerate}
\end{exercise}

\begin{proof}[Lời giải]
    \begin{enumerate}[label = (\alph*)]
        \item
              \begingroup{}
              \allowdisplaybreaks{}
              \begin{gather*}
                  \rank\begin{pmatrix}
                      2 & -1 & 3 & -2 & 4 \\
                      4 & -2 & 5 & 1  & 7 \\
                      2 & -1 & 1 & 8  & 2
                  \end{pmatrix}
                  =\rank\begin{pmatrix}
                      2 & -1 & 3  & -2 & 4  \\
                      0 & 0  & -1 & 5  & -1 \\
                      0 & 0  & -2 & 10 & -2
                  \end{pmatrix}
                  =\rank\begin{pmatrix}
                      2 & -1 & 3  & -2 & 4  \\
                      0 & 0  & -1 & 5  & -1 \\
                      0 & 0  & 0  & 0  & 0
                  \end{pmatrix}
                  = 2.
              \end{gather*}
              \endgroup{}
        \item
              \begingroup{}
              \allowdisplaybreaks{}
              \begin{gather*}
                  \rank\begin{pmatrix}
                      3 & -1 & 3  & 2 & 5  \\
                      5 & -3 & 2  & 3 & 4  \\
                      1 & -3 & -5 & 0 & -7 \\
                      7 & -5 & 1  & 4 & 1
                  \end{pmatrix} =
                  \rank\begin{pmatrix}
                      1 & -3 & -5 & 0 & -7 \\
                      3 & -1 & 3  & 2 & 5  \\
                      5 & -3 & 2  & 3 & 4  \\
                      7 & -5 & 1  & 4 & 1
                  \end{pmatrix} =
                  \rank\begin{pmatrix}
                      1 & -3 & -5 & 0 & -7 \\
                      0 & 8  & 18 & 2 & 26 \\
                      0 & 12 & 27 & 3 & 39 \\
                      0 & 16 & 36 & 4 & 50
                  \end{pmatrix} \\
                  =\rank\begin{pmatrix}
                      1 & -3 & -5 & 0 & -7 \\
                      0 & 4  & 9  & 1 & 13 \\
                      0 & 4  & 9  & 1 & 13 \\
                      0 & 8  & 18 & 2 & 25
                  \end{pmatrix}
                  =\rank\begin{pmatrix}
                      1 & -3 & -5 & 0 & -7 \\
                      0 & 4  & 9  & 1 & 13 \\
                      0 & 0  & 0  & 0 & 0  \\
                      0 & 0  & 0  & 0 & -1
                  \end{pmatrix}
                  = 3.
              \end{gather*}
              \endgroup{}
    \end{enumerate}
\end{proof}

% exercise 3.35
\begin{exercise}
    \par Tìm giá trị của $\lambda$ sao cho ma trận sau đây có hạng thấp nhất
    \[
        \begin{pmatrix}
            3       & 1 & 1  & 4 \\
            \lambda & 4 & 10 & 1 \\
            1       & 7 & 17 & 3 \\
            2       & 2 & 4  & 3
        \end{pmatrix}.
    \]
\end{exercise}

\begin{proof}[Lời giải]
    \begingroup{}
    \allowdisplaybreaks{}
    \begin{gather*}
        \rank\begin{pmatrix}
            3       & 1 & 1  & 4 \\
            \lambda & 4 & 10 & 1 \\
            1       & 7 & 17 & 3 \\
            2       & 2 & 4  & 3
        \end{pmatrix}
        =\rank\begin{pmatrix}
            \lambda & 4 & 10 & 1 \\
            1       & 7 & 17 & 3 \\
            2       & 2 & 4  & 3 \\
            3       & 1 & 1  & 4 \\
        \end{pmatrix}
        =\rank\begin{pmatrix}
            \lambda & 4   & 10  & 1  \\
            1       & 7   & 17  & 3  \\
            0       & -12 & -30 & -3 \\
            0       & -20 & -50 & -5 \\
        \end{pmatrix} \\
        =\rank\begin{pmatrix}
            \lambda & 4 & 10 & 1 \\
            1       & 7 & 17 & 3 \\
            0       & 4 & 10 & 1 \\
            0       & 4 & 10 & 1 \\
        \end{pmatrix}
        =\rank\begin{pmatrix}
            \lambda & 0 & 0  & 0 \\
            1       & 7 & 17 & 3 \\
            0       & 4 & 10 & 1 \\
            0       & 0 & 0  & 0 \\
        \end{pmatrix}.
    \end{gather*}
    \endgroup{}
    \par Để ma trận trên có hạng thấp nhất, $\lambda = 0$.
\end{proof}

% exercise 3.36
\begin{exercise}
    \par Tìm hạng của ma trận sau đây như một hàm phụ thuộc $\lambda$:
    \[
        \begin{pmatrix}
            1 & \lambda & -1      & 2 \\
            2 & -1      & \lambda & 5 \\
            1 & 10      & -6      & 1
        \end{pmatrix}.
    \]
\end{exercise}

\begin{proof}[Lời giải]
    \par Ma trận trên có định thức con $\begin{vmatrix}2 & -1 \\ 1 & 10\end{vmatrix} = 21 \ne 0$.
    \par Do đó hạng của ma trận trên lớn hơn hoặc bằng 2.
    \par Tính tất cả các định thức con cỡ 3:
    \begin{align*}
         & \begin{vmatrix}
               1 & \lambda & -1      \\
               2 & -1      & \lambda \\
               1 & 10      & -6
           \end{vmatrix} = (\lambda - 3)(\lambda + 5)  \\
         & \begin{vmatrix}
               1 & \lambda & 2 \\
               2 & -1      & 5 \\
               1 & 10      & 1
           \end{vmatrix} = 3(\lambda - 3)              \\
         & \begin{vmatrix}
               1 & -1      & 2 \\
               2 & \lambda & 5 \\
               1 & -6      & 1
           \end{vmatrix} = -(\lambda - 3)              \\
         & \begin{vmatrix}
               \lambda & -1      & 2 \\
               -1      & \lambda & 5 \\
               10      & -6      & 1
           \end{vmatrix} = (\lambda - 3)(\lambda + 13)
    \end{align*}
    \par Vậy hạng của ma trận trên bằng 3 nếu $\lambda \ne 3$, bằng 2 nếu $\lambda = 3$.
\end{proof}

% exercise 3.37
\begin{exercise}
    \par Chứng minh rằng nếu hạng của một ma trận bằng $r$ thì mỗi định thức con nằm trên giao của bất kì $r$ hàng độc lập tuyến tính và $r$ cột độc lập tuyến tính của ma trận đó đều khác 0.
\end{exercise}

\begin{proof}
    \par Giả sử ma trận $A$ đang xét có $m$ hàng và $n$ cột.
    \par Việc thay đổi thứ tự các hàng hay thay đổi thứ tự các cột không làm ảnh hưởng đến sự độc lập tuyến tính/phụ thuộc tuyến tính của các vector hàng, vector cột mà chỉ thay đổi thứ tự các vector đó, cũng như các yếu tố của các vector đó.
    \par Do đó, không mất tính tổng quát, giả sử $r$ hàng đầu tiên của $A$ độc lập tuyến tính và $r$ cột đầu tiên của $A$ độc lập tuyến tính.
    \[
        A =
        \begin{pmatrix}
            a_{11}     & a_{12}     & \cdots & a_{1r}     & a_{1(r+1)}     & \cdots & a_{1n}     \\
            a_{21}     & a_{22}     & \cdots & a_{2r}     & a_{2(r+1)}     & \cdots & a_{2n}     \\
            \vdots     & \vdots     & \ddots & \vdots     & \vdots         & \ddots & \vdots     \\
            a_{r1}     & a_{r2}     & \cdots & a_{rr}     & a_{r(r+1)}     & \cdots & a_{rn}     \\
            a_{(r+1)1} & a_{(r+1)2} & \cdots & a_{(r+1)r} & a_{(r+1)(r+1)} & \cdots & a_{(r+1)n} \\
            \vdots     & \vdots     & \ddots & \vdots     & \vdots         & \ddots & \vdots     \\
            a_{m1}     & a_{m2}     & \cdots & a_{mr}     & a_{m(r+1)}     & \cdots & a_{mn}
        \end{pmatrix}
    \]
    \par Do $\rank(A) = r$ và $r$ hàng đầu tiên của $A$ độc lập tuyến tính nên hàng cột $r+1$, \ldots, $n$ đều biểu thị tuyến tính được theo $r$ hàng đầu tiên, ta đặt:
    \begin{align*}
        (a_{(r+1)1}, a_{(r+1)2}, \ldots, a_{(r+1)r}, \ldots, a_{(r+1)n}) & = \sum^{r}_{k=1}b_{k}^{(r+1)}(a_{k1}, a_{k2}, \ldots, a_{kr}, \ldots, a_{kn}) \\
        (a_{(r+2)1}, a_{(r+2)2}, \ldots, a_{(r+2)r}, \ldots, a_{(r+2)n}) & = \sum^{r}_{k=1}b_{k}^{(r+2)}(a_{k1}, a_{k2}, \ldots, a_{kr}, \ldots, a_{kn}) \\
                                                                         & \ddots                                                                        \\
        (a_{m1}, a_{m2}, \ldots, a_{mr}, \ldots, a_{mn})                 & = \sum^{r}_{k=1}b_{k}^{(m)}(a_{k1}, a_{k2}, \ldots, a_{kr}, \ldots, a_{kn})   \\
    \end{align*}
    \par Từ các biểu thị tuyến tính trên, ta suy ra:
    \[
        \begin{cases}
            (a_{(r+1)1}, a_{(r+1)2}, \ldots, a_{(r+1)r}, 0, \ldots, 0) & = \sum^{r}_{k=1}b_{k}^{(r+1)}(a_{k1}, a_{k2}, \ldots, a_{kr}, 0, \ldots, 0) \\
            (a_{(r+2)1}, a_{(r+2)2}, \ldots, a_{(r+2)r}, 0, \ldots, 0) & = \sum^{r}_{k=1}b_{k}^{(r+2)}(a_{k1}, a_{k2}, \ldots, a_{kr}, 0, \ldots, 0) \\
                                                                       & \ddots                                                                      \\
            (a_{m1}, a_{m2}, \ldots, a_{mr}, 0, \ldots, 0)             & = \sum^{r}_{k=1}b_{k}^{(m)}(a_{k1}, a_{k2}, \ldots, a_{kr}, 0, \ldots, 0)   \\
        \end{cases}
        \tag{$\star$}
    \]
    \par Do $\rank(A) = r$ và $r$ cột đầu tiên của $A$ độc lập tuyến tính nên từng cột $r+1$, \ldots, $n$ đều biểu thị tuyến tính được theo $r$ cột đầu tiên, do đó nếu xóa các cột $r+1$, \ldots, $n$ thì hạng của $A$ không đổi. Nói cách khác:
    \[
        \rank
        \begin{pmatrix}
            a_{11}     & a_{12}     & \cdots & a_{1r}     & a_{1(r+1)}     & \cdots & a_{1n}     \\
            a_{21}     & a_{22}     & \cdots & a_{2r}     & a_{2(r+1)}     & \cdots & a_{2n}     \\
            \vdots     & \vdots     & \ddots & \vdots     & \vdots         & \ddots & \vdots     \\
            a_{r1}     & a_{r2}     & \cdots & a_{rr}     & a_{r(r+1)}     & \cdots & a_{rn}     \\
            a_{(r+1)1} & a_{(r+1)2} & \cdots & a_{(r+1)r} & a_{(r+1)(r+1)} & \cdots & a_{(r+1)n} \\
            \vdots     & \vdots     & \ddots & \vdots     & \vdots         & \ddots & \vdots     \\
            a_{m1}     & a_{m2}     & \cdots & a_{mr}     & a_{m(r+1)}     & \cdots & a_{mn}
        \end{pmatrix}
        =
        \rank
        \begin{pmatrix}
            a_{11}     & a_{12}     & \cdots & a_{1r}     & 0      & \cdots & 0      \\
            a_{21}     & a_{22}     & \cdots & a_{2r}     & 0      & \cdots & 0      \\
            \vdots     & \vdots     & \ddots & \vdots     & \vdots & \ddots & \vdots \\
            a_{r1}     & a_{r2}     & \cdots & a_{rr}     & 0      & \cdots & 0      \\
            a_{(r+1)1} & a_{(r+1)2} & \cdots & a_{(r+1)r} & 0      & \cdots & 0      \\
            \vdots     & \vdots     & \ddots & \vdots     & \vdots & \ddots & \vdots \\
            a_{m1}     & a_{m2}     & \cdots & a_{mr}     & 0      & \cdots & 0
        \end{pmatrix}
    \]
    \par Ma trận mới có các vector cột thứ $r+1$, \ldots, $n$ bằng không.
    \par Mặt khác, theo các đẳng thức $(\star)$, ta có thể xóa các hàng $r+1$, \ldots, $n$ của ma trận mới mà vẫn bảo toàn hạng, điều này kéo theo:
    \[
        \rank
        \begin{pmatrix}
            a_{11} & a_{12} & \cdots & a_{1r} & 0      & \cdots & 0      \\
            a_{21} & a_{22} & \cdots & a_{2r} & 0      & \cdots & 0      \\
            \vdots & \vdots & \ddots & \vdots & \vdots & \ddots & \vdots \\
            a_{r1} & a_{r2} & \cdots & a_{rr} & 0      & \cdots & 0      \\
            0      & 0      & \cdots & 0      & 0      & \cdots & 0      \\
            \vdots & \vdots & \ddots & \vdots & \vdots & \ddots & \vdots \\
            0      & 0      & \cdots & 0      & 0      & \cdots & 0
        \end{pmatrix}
        = \rank(A) = r.
    \]
    \par Giả sử phản chứng rằng:
    \[
        \begin{vmatrix}
            a_{11} & a_{12} & \cdots & a_{1r} \\
            a_{21} & a_{22} & \cdots & a_{2r} \\
            \vdots & \vdots & \ddots & \vdots \\
            a_{r1} & a_{r2} & \cdots & a_{rr}
        \end{vmatrix} = 0.
    \]
    \par Khi đó, mọi định thức con cỡ $r$ của ma trận trên đều bằng không, kéo theo $\rank(A) < r$.
    \par Điều mâu thuẫn này chứng tỏ giả sử phản chứng là sai. Như vậy, định thức con nằm trên giao của $r$ hàng độc lập tuyến tính và $r$ cột độc lập tuyến tính khác không.
\end{proof}

% exercise 3.38
\begin{exercise}
    \par Cho $A$ là một ma trận vuông cỡ $n > 1$ và $\tilde{A}$ là ma trận phụ hợp (gồm những phần bù đại số của các yếu tố) của $A$. Hãy xác định $\rank\tilde{A}$ như một hàm của $\rank A$.
\end{exercise}

\begin{proof}
    \par Đặt $\tilde{a}_{ij}$ là phần bù đại số của $a_{ij}$ trong ma trận $A$. Khi đó, theo định nghĩa của ma trận phụ hợp:
    \[
        \tilde{A} =
        \begin{pmatrix}
            \tilde{a}_{11} & \tilde{a}_{21} & \cdots & \tilde{a}_{n1} \\
            \tilde{a}_{12} & \tilde{a}_{22} & \cdots & \tilde{a}_{n2} \\
            \vdots         & \vdots         & \ddots & \vdots         \\
            \tilde{a}_{1n} & \tilde{a}_{2n} & \cdots & \tilde{a}_{nn}
        \end{pmatrix}
    \]
    \begin{enumerate}[label = \textbf{Trường hợp \arabic*.},itemindent=2cm]
        \item $\rank(A) = n$.
              \par Lúc này, $\det(A)\ne 0$. Mà $\det(A)\cdot\det(\tilde{A}) = 1$ nên $\det(\tilde{A})$, dẫn tới $\rank(\tilde{A}) = n$.
        \item $\rank(A) < n - 1$.
              \par Lúc này, mọi định thức con cỡ $(n-1)$ của $A$ đều bằng không, do đó $\tilde{A}$ là ma trận không. Như vậy $\rank(\tilde{A}) = 0$.
        \item $\rank(A) = n - 1$.
              \par Vì $\rank(A) = n - 1$ nên $A$ sẽ có ít nhất một định thức con cỡ $(n-1)$ với giá trị khác không. Điều này đảm bảo $\rank(\tilde{A})\ge 1$.
              \par Nhận xét rằng:
              \begin{itemize}
                  \item Đổi chỗ hai cột $j$ và $j'$ của $A$ thì ma trận phụ hợp mới là ma trận phụ hợp cũ sau khi đổi chỗ hai hàng $j$ và $j'$.
                  \item Đổi chỗ hai hàng $i$ và $i'$ của $A$ thì ma trận phụ hợp mới là ma trận phụ hợp cũ sau khi đổi chỗ hai cột $i$ và $i'$.
              \end{itemize}
              \par Như vậy, không mất tính tổng quát, giả sử $(n-1)$ cột đầu tiên của $A$ độc lập tuyến tính và $(n-1)$ hàng đầu tiên của $A$ độc lập tuyến tính.
              \par Khi đó, $\tilde{a}_{nn}\ne 0$ và trong ma trận $A$, hàng cuối cùng biểu thị tuyến tính được duy nhất theo $(n-1)$ hàng đầu tiên.
              \par Đặt $\alpha'_{n} = \lambda_{1}\alpha'_{1} + \lambda_{2}\alpha'_{2} + \cdots + \lambda_{n-1}\alpha'_{n-1}$, trong đó $\alpha'_{k}$ và vector hàng thứ $k$ của $A$, $\lambda_{i}$ là các vô hướng thuộc trường $\mathbb{F}$.
              \[
                  \tilde{A} =
                  \begin{pmatrix}
                      \tilde{a}_{11} & \tilde{a}_{21} & \cdots & \tilde{a}_{n1} \\
                      \tilde{a}_{12} & \tilde{a}_{22} & \cdots & \tilde{a}_{n2} \\
                      \vdots         & \vdots         & \ddots & \vdots         \\
                      \tilde{a}_{1n} & \tilde{a}_{2n} & \cdots & \tilde{a}_{nn}
                  \end{pmatrix}.
              \]
              \begingroup{}
              \allowdisplaybreaks{}
              \begin{align*}
                  \tilde{a}_{n\ell} & = (-1){}^{n+\ell}
                  \begin{vmatrix}
                      a_{11}     & \cdots & a_{1(\ell-1)}     & a_{1(\ell+1)}     & \cdots & a_{1n}     \\
                      \vdots     & \ddots & \vdots            & \vdots            & \ddots & \vdots     \\
                      a_{(n-1)1} & \cdots & a_{(n-1)(\ell-1)} & a_{(n-1)(\ell+1)} & \cdots & a_{(n-1)n}
                  \end{vmatrix}                             \\
                  \tilde{a}_{k\ell} & = (-1){}^{k+\ell}
                  \begin{vmatrix}
                      a_{11}     & \cdots & a_{1(\ell-1)}     & a_{1(\ell+1)}     & \cdots & a_{1n}     \\
                      \vdots     & \ddots & \vdots            & \vdots            & \ddots & \vdots     \\
                      a_{(k-1)1} & \cdots & a_{(k-1)(\ell-1)} & a_{(k-1)(\ell+1)} & \cdots & a_{(k-1)n} \\
                      a_{(k+1)1} & \cdots & a_{(k+1)(\ell-1)} & a_{(k+1)(\ell+1)} & \cdots & a_{(k+1)n} \\
                      \vdots     & \ddots & \vdots            & \vdots            & \ddots & \vdots     \\
                      a_{n1}     & \cdots & a_{n(\ell-1)}     & a_{n(\ell+1)}     & \cdots & a_{nn}
                  \end{vmatrix}                             \\
                                    & = (-1){}^{k+\ell}
                  \begin{vmatrix}
                      a_{11}            & \cdots & a_{1(\ell-1)}            & a_{1(\ell+1)}            & \cdots & a_{1n}            \\
                      \vdots            & \ddots & \vdots                   & \vdots                   & \ddots & \vdots            \\
                      a_{(k-1)1}        & \cdots & a_{(k-1)(\ell-1)}        & a_{(k-1)(\ell+1)}        & \cdots & a_{(k-1)n}        \\
                      a_{(k+1)1}        & \cdots & a_{(k+1)(\ell-1)}        & a_{(k+1)(\ell+1)}        & \cdots & a_{(k+1)n}        \\
                      \vdots            & \ddots & \vdots                   & \vdots                   & \ddots & \vdots            \\
                      \lambda_{k}a_{k1} & \cdots & \lambda_{k}a_{k(\ell-1)} & \lambda_{k}a_{k(\ell+1)} & \cdots & \lambda_{k}a_{kn}
                  \end{vmatrix} \\
                                    & = (-1){}^{n+\ell-1}\lambda_{k}
                  \begin{vmatrix}
                      a_{11}     & \cdots & a_{1(\ell-1)}     & a_{1(\ell+1)}     & \cdots & a_{1n}     \\
                      \vdots     & \ddots & \vdots            & \vdots            & \ddots & \vdots     \\
                      a_{(k-1)1} & \cdots & a_{(k-1)(\ell-1)} & a_{(k-1)(\ell+1)} & \cdots & a_{(k-1)n} \\
                      a_{k1}     & \cdots & a_{k(\ell-1)}     & a_{k(\ell+1)}     & \cdots & a_{kn}     \\
                      a_{(k+1)1} & \cdots & a_{(k+1)(\ell-1)} & a_{(k+1)(\ell+1)} & \cdots & a_{(k+1)n} \\
                      \vdots     & \ddots & \vdots            & \vdots            & \ddots & \vdots     \\
                      a_{(n-1)1} & \cdots & a_{(n-1)(\ell-1)} & a_{(n-1)(\ell+1)} & \cdots & a_{(n-1)n}
                  \end{vmatrix}, \quad\forall 1\le\ell\le n
              \end{align*}
              \endgroup{}
              \par Như vậy, trong ma trận phụ hợp $\tilde{A}$, cột thứ $k$ bằng cột thứ $n$ nhân với $\lambda_{k}$, tức là từng cột trong $(n-1)$ cột đầu tiên của $\tilde{A}$ đều biểu thị tuyến tính được theo cột thứ $n$.
              \par Mà $\tilde{a}_{nn}\ne 0$ nên cột thứ $n$ của ma trận phụ hợp khác không.
              \par Hai điều nêu trên kéo theo $\rank(\tilde{A}) = 1$.
    \end{enumerate}

    \par Vậy, với $A\in M(n\times n,\mathbb{F})$, $n > 1$:
    \begin{align*}
        \rank(A) = n   & \Longrightarrow \rank(\tilde{A}) = n, \\
        \rank(A) = n-1 & \Longrightarrow \rank(\tilde{A}) = 1, \\
        \rank(A) < n-1 & \Longrightarrow \rank(\tilde{A}) = 0.
    \end{align*}
\end{proof}

% exercise 3.39
\begin{exercise}
    \par Chứng minh rằng nếu các vector
    \[
        \alpha_{i} = (a_{i1}, a_{i2}, \ldots, a_{in})\in\mathbb{R}_{n}\quad (i = 1, 2, \ldots, s; s\le n),
    \]
    \par thỏa mãn điều kiện $\abs{a_{jj}} > \sum_{i\ne j}\abs{a_{ij}}$, thì chúng độc lập tuyến tính.
\end{exercise}

\par\MakeUppercase{Chưa làm được}.

\begin{proof}
\end{proof}

% exercise 3.40
\begin{exercise}\label{chapter3:rank-of-sum}
    \par Chứng minh rằng nếu $A$ và $B$ là các ma trận cùng số hàng và số cột thì
    \[
        \rank(A + B)\le \rank(A) + \rank(B).
    \]
\end{exercise}

\begin{proof}
    \par $\alpha_{1}$, \ldots, $\alpha_{n}$ là các vector cột của $A$.
    \par $\beta_{1}$, \ldots, $\beta_{n}$ là các vector cột của $B$.
    \par Như vậy, các vector cột của $(A + B)$ là $\alpha_{1} + \beta_{1}$, \ldots, $\alpha_{n} + \beta_{n}$.
    \par Đặt $V_{A} = \text{span}(\alpha_{1}, \ldots, \alpha_{n})$, $V_{B} = \text{span}(\beta_{1}, \ldots, \beta_{n})$ và $V_{A+B} = \text{span}(\alpha_{1}+\beta_{1}, \ldots, \alpha_{n} + \beta_{n})$.
    \[
        \gamma = \underbrace{\sum^{n}_{i=1}c_{i}(\alpha_{i} + \beta_{i})}_{\in V_{A+B}} = \underbrace{\sum^{n}_{i=1}c_{i}\alpha_{i}}_{\in V_{A}} + \underbrace{\sum^{n}_{i=1}c_{i}\beta_{i}}_{\in V_{B}} \in V_{A} + V_{B}.
    \]
    \par Do đó $V_{A+B}$ là không gian con của $V_{A} + V_{B}$.
    \par $\dim V_{A+B}\le \dim(V_{A} + V_{B}) = \dim(V_{A}) + \dim(V_{B}) - \dim(V_{A}\cap V_{B})\le \dim(V_{A}) + \dim(V_{B})$.
    \par Bất đẳng thức $\dim V_{A+B} \le \dim{V_{A}} + \dim{V_{B}}$ tương đương với $\rank(A + B)\le \rank(A) + \rank(B)$.
\end{proof}

% exercise 3.41
\begin{exercise}
    \par Chứng minh rằng mỗi ma trận có hạng $r$ có thể viết thành tổng của $r$ ma trận có hạng 1, nhưng không thể viết thành tổng của một số ít hơn $r$ ma trận có hạng 1.
\end{exercise}

\begin{proof}
    \par Giả sử ma trận $A$ có $n$ cột và các vector cột $\alpha_{1}$, \ldots, $\alpha_{r}$ của $A$ độc lập tuyến tính cực đại.
    \par Khi đó, từng vector cột $\alpha_{r+1}$, \ldots, $\alpha_{n}$ của $A$ có thể biểu thị tuyến tính theo $\alpha_{1}$, \ldots, $\alpha_{r}$. Ta đặt:
    \[
        \begin{cases}
            \alpha_{r+1} = a_{1}^{(r+1)}\alpha_{1} + \cdots + a_{r}^{(r+1)}\alpha_{r}, \\
            \alpha_{r+2} = a_{1}^{(r+2)}\alpha_{1} + \cdots + a_{r}^{(r+2)}\alpha_{r}, \\
            \ddots                                                                     \\
            \alpha_{n} = a_{1}^{(n)}\alpha_{1} + \cdots + a_{r}^{(n)}\alpha_{r}.
        \end{cases}
    \]
    \par Dựa vào những biểu thị tuyến tính này, ta có thể tách $A$ thành tổng của $r$ ma trận như sau:
    \begin{align*}
        A = \begin{pmatrix}
                \alpha_{1} & \cdots & \alpha_{r} & \alpha_{r+1} & \cdots & \alpha_{n}
            \end{pmatrix}
         & =
        \begin{pmatrix}
            \alpha_{1} & \cdots & \alpha_{r} & \sum^{r}_{i=1}a_{i}^{(r+1)}\alpha_{i} & \cdots & \sum^{r}_{i=1}a_{i}^{(n)}\alpha_{i}
        \end{pmatrix} \\
         & =
        \begin{pmatrix}
            \alpha_{1} & \cdots & 0 & a_{1}^{(r+1)}\alpha_{1} & \cdots & a_{1}^{(n)}\alpha_{1}
        \end{pmatrix}                                      \\
         & \phantom{=} + \cdots                                                                                                 \\
         & \phantom{=} + \begin{pmatrix}
                             0 & \cdots & \alpha_{r} & a_{r}^{(r+1)}\alpha_{r} & \cdots & a_{r}^{(n)}\alpha_{r}
                         \end{pmatrix}
    \end{align*}
    \par Như vậy $A$ có thể viết thành tổng của $r$ ma trận có hạng 1.
    \bigskip
    \par Giả sử phản chứng rằng $A$ có thể viết thành tổng của $s$ ma trận $A_{i}$, $i = \overline{1,s}$ có hạng 1, trong đó $s < r$.
    \par Theo bài toán~\ref{chapter3:rank-of-sum}, $\rank(A)\le \sum^{s}_{i=1}\rank(A_{i}) = s < r$.
    \par Bất đẳng thức trên là sai vì $\rank(A) = r$.
    \par Vậy $A$ không thể viết thành tổng của ít hơn $r$ ma trận có hạng 1.
\end{proof}

% exercise 3.42
\begin{exercise}
    \par Chứng minh bất đẳng thức Sylvester cho các ma trận vuông cỡ $n$ bất kì $A$ và $B$:
    \[
        \rank(A) + \rank(B) - n \le \rank(AB) \le \min\{ \rank(A), \rank(B) \}.
    \]
\end{exercise}

\begin{proof}
    \par $\rank(A) = a$, $\rank(B) = b$.
    \par Nhận xét rằng:
    \begin{enumerate}[label = (\roman*)]
        \item Nếu đổi chỗ hai hàng của $A$ thì hai hàng tương ứng của $AB$ đổi chỗ,
        \item Nếu đổi chỗ hai cột của $B$ thì hai cột tương ứng của $AB$ đổi chỗ,
        \item Nếu cộng một hàng của $A$ với tổ hợp tuyến tính của các hàng còn lại thì hàng tương ứng của $AB$ cũng được cộng với tổ hợp tuyến tính của các hàng tương ứng còn lại (cùng hệ số tổ hợp tuyến tính),
        \item Nếu cộng một cột của $B$ với tổ hợp tuyến tính của các hàng còn lại thì cột tương ứng của $AB$ cũng được cộng với tổ hợp tuyến tính của các cột tương ứng còn lại (cùng hệ số tổ hợp tuyến tính).
        \item Nhân một hàng của $A$ với một vô hướng $\lambda$ khác không thì hàng tương ứng của $AB$ cũng được nhân với $\lambda$.
        \item Nhân một cột của $B$ với một vô hướng $\lambda$ khác không thì cột tương ứng của $AB$ cũng được nhân với $\lambda$.
    \end{enumerate}
    \par Do đó, nếu thực hiện các phép biến đổi như trên, $\rank(A)$, $\rank(B)$, $\rank(AB)$ không đổi.
    \bigskip
    \par Vì những lý do trên, không mất tính tổng quát, ta có thể giả sử:
    \begin{itemize}
        \item $a$ hàng đầu tiên của $A$ độc lập tuyến tính (i)
        \item $b$ cột đầu tiên của $B$ độc lập tuyến tính (ii)
        \item $n - a$ hàng cuối của $A$ là các vector không (iii)
        \item $n - b$ cột cuối của $B$ là các vector không (iv)
    \end{itemize}
    \par $\alpha_{1}^{T}$, $\alpha_{2}^{T}$, \ldots, $\alpha_{a}^{T}$ là $a$ vector hàng đầu tiên của $A$.
    \par $\beta_{1}$, $\beta_{2}$, \ldots, $\beta_{b}$ là $b$ vector cột đầu tiên của $B$.
    \[
        \underbrace{\begin{pmatrix}
                \alpha_{1}^{T} \\
                \alpha_{2}^{T} \\
                \vdots         \\
                \alpha_{a}^{T} \\
                0              \\
                \vdots         \\
                0
            \end{pmatrix}}_{A}
        \underbrace{\begin{pmatrix}
                \beta_{1} & \beta_{2} & \cdots & \beta_{b} & 0 & \cdots & 0
            \end{pmatrix}}_{B}
        =
        \underbrace{\begin{pmatrix}
                \alpha_{1}^{T}\beta_{1} & \cdots & \alpha_{1}^{T}\beta_{b} & 0      & \cdots & 0      \\
                \vdots                  & \ddots & \vdots                  & \vdots & \ddots & \vdots \\
                \alpha_{a}^{T}\beta_{1} & \cdots & \alpha_{a}^{T}\beta_{b} & 0      & \cdots & 0      \\
                0                       & \cdots & 0                       & 0      & \cdots & 0      \\
                \vdots                  & \ddots & \vdots                  & \vdots & \ddots & \vdots \\
                0                       & \cdots & 0                       & 0      & \cdots & 0
            \end{pmatrix}}_{AB}
    \]
    \par Do đó, $\rank(AB)\le \min\{ a, b \} = \min\{ \rank(A), \rank(B) \}$.
    \bigskip
    \par Nếu $A = 0$ thì $\rank(A) + \rank(B) - n = \rank(B) - n\le 0 = \rank(AB)$.
    \par Ngược lại:
    \par Ta sử dụng các phép biến đổi sơ cấp khác để đưa ma trận $A$ về dạng đơn giản hơn:
    \begin{enumerate}[label = (\roman*)]
        \setcounter{enumi}{6}
        \item Nhân một \textit{cột} $i$, $j$ của $A$ với vô hướng $c\ne 0$ và \textit{hàng} $i$, $j$ của $B$ với $c^{-1}$. Biến đổi này không làm thay đổi $\rank(A)$, $\rank(B)$ và không làm thay đổi $AB$.
        \item Đổi chỗ hai \textit{cột} $i$, $j$ của $A$ và đổi chỗ hai \textit{hàng} $i$, $j$ của $B$. Biến đổi này không làm thay đổi $\rank(A)$, $\rank(B)$ và không làm thay đổi $AB$.
        \item Cộng thêm cột $j$ vào cột $i$ của $A$ và trừ hàng $i$ khỏi hàng $j$ của $B$. Biến đổi này không làm thay đổi $\rank(A)$, $\rank(B)$ và không làm thay đổi $AB$.
    \end{enumerate}
    \par Như vậy, cả chín phép biến đổi trên sẽ bảo toàn $\rank(A)$, $\rank(B)$, $\rank(AB)$. Ta thực hiện các biến đổi sau, với ma trận $A$ có $a$ hàng đầu tiên độc lập tuyến tính và $n - a$ hàng còn lại bằng không:
    \begin{itemize}
        \item Tồn tại $a_{ij}\ne 0$ (vì $A\ne 0$). Ta đổi chỗ hàng 1 và hàng $i$ của $A$.
        \item Đổi chỗ cột 1 và cột $j$ của $A$, đồng thời đổi chỗ hàng 1 và hàng $j$ của $B$.
        \item Nhân hàng thứ 1 của $A$ với một vô hướng khác không, sao cho yếu tố hàng 1 cột 1 bằng 1.
        \item Với $2\le i\le a$, cộng hàng $i$ với hàng 1 (sau khi nhân hàng 1 với đối của yếu tố hàng $i$, cột 1). Đến lúc này tất cả yêu tố trên cột 1 đều bằng 0, trừ yếu tố hàng 1 cột 1.
        \item Tiếp tục quá trình trên, đến khi $A$ đạt được dạng sau (hàng $i$ có yếu tố hàng $i$, cột $i$ bằng 1, các yếu tố đứng trước bằng 0):
              \[
                  \begin{pmatrix}
                      1      & *      & \cdots & *      & *      & \cdots & *      \\
                      0      & 1      & \cdots & *      & *      & \cdots & *      \\
                      \vdots & \vdots & \ddots & \vdots & \vdots & \ddots & \vdots \\
                      0      & 0      & \cdots & 1      & *      & \cdots & *      \\
                      0      & 0      & \cdots & 0      & 0      & \cdots & 0      \\
                      \vdots & \vdots & \ddots & \vdots & \vdots & \ddots & \vdots \\
                      0      & 0      & \cdots & 0      & 0      & \cdots & 0
                  \end{pmatrix}
              \]
        \item Đối với các cột $i > a$. Bắt đầu từ hàng $a$, rồi đến $a-1$, \ldots, 1, ta áp dụng biến đổi (vii), (ix) để đưa $A$ về dạng:
              \[
                  \begin{pmatrix}
                      1      & *      & \cdots & *      & 0      & \cdots & 0      \\
                      0      & 1      & \cdots & *      & 0      & \cdots & 0      \\
                      \vdots & \vdots & \ddots & \vdots & \vdots & \ddots & \vdots \\
                      0      & 0      & \cdots & 1      & 0      & \cdots & 0      \\
                      0      & 0      & \cdots & 0      & 0      & \cdots & 0      \\
                      \vdots & \vdots & \ddots & \vdots & \vdots & \ddots & \vdots \\
                      0      & 0      & \cdots & 0      & 0      & \cdots & 0
                  \end{pmatrix}
              \]
        \item Đối với các hàng $i < a$. Ta thực hiện biến đổi (iii) để đưa $A$ về dạng:
              \[
                  \begin{pmatrix}
                      1      & 0      & \cdots & 0      & 0      & \cdots & 0      \\
                      0      & 1      & \cdots & 0      & 0      & \cdots & 0      \\
                      \vdots & \vdots & \ddots & \vdots & \vdots & \ddots & \vdots \\
                      0      & 0      & \cdots & 1      & 0      & \cdots & 0      \\
                      0      & 0      & \cdots & 0      & 0      & \cdots & 0      \\
                      \vdots & \vdots & \ddots & \vdots & \vdots & \ddots & \vdots \\
                      0      & 0      & \cdots & 0      & 0      & \cdots & 0
                  \end{pmatrix}
              \]
    \end{itemize}
    \par Như vậy, ta chỉ cần chứng minh bất đẳng thức cho trường hợp $A$ có dạng:
    \[
        \begin{pmatrix}
            1      & 0      & \cdots & 0      & 0      & \cdots & 0      \\
            0      & 1      & \cdots & 0      & 0      & \cdots & 0      \\
            \vdots & \vdots & \ddots & \vdots & \vdots & \ddots & \vdots \\
            0      & 0      & \cdots & 1      & 0      & \cdots & 0      \\
            0      & 0      & \cdots & 0      & 0      & \cdots & 0      \\
            \vdots & \vdots & \ddots & \vdots & \vdots & \ddots & \vdots \\
            0      & 0      & \cdots & 0      & 0      & \cdots & 0
        \end{pmatrix}
    \]
    \par Phần còn lại của chứng minh được triển khai chi tiết hơn từ chứng minh sau: \url{https://math.stackexchange.com/questions/298836/sylvester-rank-inequality-operatornamerank-a-operatornamerankb-leq-o}
    \begin{align*}
        \rank(A) + \rank(B) & = \rank(A) + \rank(AB + B(I_{n} - A))          \\
                            & \le \rank(A) + \rank(AB) + \rank(B(I_{n} - A))
    \end{align*}
    \[
        B(I_{n} - A) =
        \begin{pmatrix}
            b_{11} & b_{12} & \cdots & b_{1n} \\
            b_{21} & b_{22} & \cdots & b_{2n} \\
            \vdots & \vdots & \ddots & \vdots \\
            b_{n1} & b_{n2} & \cdots & b_{nn}
        \end{pmatrix}
        \begin{pmatrix}
            0      & \cdots & 0      & 0      & \cdots & 0      \\
            \vdots & \ddots & \vdots & \vdots & \ddots & \vdots \\
            0      & \cdots & 0      & 0      & \cdots & 0      \\
            0      & \cdots & 0      & 1      & \cdots & 0      \\
            \vdots & \ddots & \vdots & \vdots & \ddots & \vdots \\
            0      & \cdots & 0      & 0      & \cdots & 1
        \end{pmatrix}
        =
        \begin{pmatrix}
            0      & \cdots & 0      & b_{1(a+1)} & \cdots & b_{1n} \\
            0      & \cdots & 0      & b_{2(a+1)} & \cdots & b_{2n} \\
            \vdots & \ddots & \vdots & \vdots     & \ddots & \vdots \\
            0      & \cdots & 0      & b_{a(a+1)} & \cdots & b_{an} \\
            \vdots & \ddots & \vdots & \vdots     & \ddots & \vdots \\
            0      & \cdots & 0      & b_{n(a+1)} & \cdots & b_{nn}
        \end{pmatrix}.
    \]
    \par Đẳng thức trên chứng tỏ $\rank(B(I_{n} - A))\le n - \rank(A)$.
    \begin{align*}
        \rank(A) + \rank(B) & = \rank(A) + \rank(AB + B(I_{n} - A))          \\
                            & \le \rank(A) + \rank(AB) + \rank(B(I_{n} - A)) \\
                            & \text{(tiếp tục)}                              \\
                            & \le \rank(A) + \rank(AB) + n - \rank(A)        \\
                            & = \rank(AB) + n.
    \end{align*}
    \par Vậy $\rank(A) + \rank(B) - n \le \rank(AB)$.
\end{proof}

% exercise 3.43
\begin{exercise}
    \par Chứng minh rằng nếu trường $\mathbb{F}$ có đặc số khác 2 và $A$ là một ma trận vuông cỡ $n$ với các yếu tố trong $\mathbb{F}$ sao cho $A^{2} = E$, thì $\rank(A + E) + \rank(A - E) = n$. Tìm phản ví dụ cho kết luận nói trên khi đặc số của $\mathbb{F}$ bằng 2.
\end{exercise}

\begin{proof}
    \par Nếu $\text{Char}(\mathbb{F})\ne 2$, áp dụng bất đẳng thức Sylvester và bài toán~\ref{chapter3:rank-of-sum}:
    \[
        \rank(A) = \rank(2A) = \rank(A+E + A-E) \le \rank(A+E) + \rank(A-E) \le n + \rank(A^{2} - E) = n.
    \]
    \par Vì $A^{2} = E$ nên $A$ khả nghịch, do đó $\rank(A) = n$.
    \par Nếu $\rank(A + E) + \rank(A - E) < n$ thì $\rank(A) < n$, mâu thuẫn với đẳng thức $\rank(A) = n$.
    \par Do đó, $\rank(A + E) + \rank(A - E) = n$.
    \bigskip
    \par Nếu $\text{Char}(\mathbb{F}) = 2$, chọn $A = E$.
    \par $A = E$ và $\text{Char}(\mathbb{F}) = 2$ kéo theo $A + E = A - E = 0$, suy ra $\rank(A + E) + \rank(A - E) = 0 < n$.
\end{proof}

% exercise 3.44
\begin{exercise}
    \par Tìm ma trận nghịch đảo của các ma trận sau đây bằng phương pháp định thức và phương pháp biến đổi sơ cấp:
    \begin{center}
        \begin{enumerate*}[label = (\alph*)]
            \item $\begin{pmatrix}
                          0 & 1 & 3 \\
                          2 & 3 & 5 \\
                          3 & 6 & 7
                      \end{pmatrix}$,
            \item $\begin{pmatrix}
                          1 & 2 & -1 & -2 \\
                          3 & 8 & 0  & -4 \\
                          2 & 2 & -4 & -3 \\
                          3 & 8 & -1 & -6
                      \end{pmatrix}$.
        \end{enumerate*}
    \end{center}
\end{exercise}

\begin{proof}[Lời giải]
    \begin{enumerate}[label = (\alph*)]
        \item
              \begingroup{}
              \allowdisplaybreaks{}
              \begin{gather*}
                  \left(\begin{array}{ccc|ccc}
                          0 & 1 & 3 & 1 & 0 & 0 \\
                          2 & 3 & 5 & 0 & 1 & 0 \\
                          3 & 6 & 7 & 0 & 0 & 1
                      \end{array}
                  \right)
                  \stackrel{r_{3}:= r_{3}-r_{2}}{\Longleftrightarrow}
                  \left(\begin{array}{ccc|ccc}
                          0 & 1 & 3 & 1 & 0  & 0 \\
                          2 & 3 & 5 & 0 & 1  & 0 \\
                          1 & 3 & 2 & 0 & -1 & 1
                      \end{array}
                  \right) \\
                  \stackrel{
                      \substack{
                          r_{1}:= r_{1} + r_{2} \\
                          r_{2}:= r_{2} {-} 2r_{1}
                      }
                  }{\Longleftrightarrow}
                  \left(\begin{array}{ccc|ccc}
                          1 & 4  & 5 & 1 & -1 & 1  \\
                          0 & -3 & 1 & 0 & 3  & -2 \\
                          1 & 3  & 2 & 0 & -1 & 1
                      \end{array}
                  \right)
                  \stackrel{
                      r_{3}:= r_{3} {-} r_{1}
                  }{\Longleftrightarrow}
                  \left(\begin{array}{ccc|ccc}
                          1 & 4  & 5  & 1  & -1 & 1  \\
                          0 & -3 & 1  & 0  & 3  & -2 \\
                          0 & -1 & -3 & -1 & 0  & 0
                      \end{array}
                  \right) \\
                  \stackrel{
                      r_{3}:= 3r_{3} {-} r_{2}
                  }{\Longleftrightarrow}
                  \left(\begin{array}{ccc|ccc}
                          1 & 4  & 5   & 1  & -1 & 1  \\
                          0 & -3 & 1   & 0  & 3  & -2 \\
                          0 & 0  & -10 & -3 & -3 & 2
                      \end{array}
                  \right)
                  \stackrel{
                      r_{2}:= 10r_{2} + r_{3}
                  }{\Longleftrightarrow}
                  \left(\begin{array}{ccc|ccc}
                          1 & 4   & 5   & 1  & -1 & 1   \\
                          0 & -30 & 0   & -3 & 27 & -18 \\
                          0 & 0   & -10 & -3 & -3 & 2
                      \end{array}
                  \right) \\
                  \stackrel{
                  r_{2}:= \frac{1}{3}r_{2}
                  }{\Longleftrightarrow}
                  \left(\begin{array}{ccc|ccc}
                          1 & 4   & 5   & 1  & -1 & 1  \\
                          0 & -10 & 0   & -1 & 9  & -6 \\
                          0 & 0   & -10 & -3 & -3 & 2
                      \end{array}
                  \right)
                  \stackrel{
                      r_{1}:= 2r_{1} + r_{3}
                  }{\Longleftrightarrow}
                  \left(\begin{array}{ccc|ccc}
                          2 & 8   & 0   & -1 & -5 & 4  \\
                          0 & -10 & 0   & -1 & 9  & -6 \\
                          0 & 0   & -10 & -3 & -3 & 2
                      \end{array}
                  \right) \\
                  \stackrel{
                  r_{1}:= r_{1} + \frac{4}{5}r_{2}
                  }{\Longleftrightarrow}
                  \left(\begin{array}{ccc|ccc}
                          2 & 0   & 0   & \frac{-9}{5} & \frac{11}{5} & \frac{-4}{5} \\
                          0 & -10 & 0   & -1           & 9            & -6           \\
                          0 & 0   & -10 & -3           & -3           & 2
                      \end{array}
                  \right)
                  \stackrel{
                  \substack{
                  r_{1}:= \frac{1}{2}r_{1} \\
                  r_{2}:= \frac{-1}{10}r_{2} \\
                  r_{3}:= \frac{-1}{10}r_{3}
                  }
                  }{\Longleftrightarrow}
                  \left(\begin{array}{ccc|ccc}
                          1 & 0 & 0 & \frac{-9}{10} & \frac{11}{10} & \frac{-2}{5} \\
                          0 & 1 & 0 & \frac{1}{10}  & \frac{-9}{10} & \frac{3}{5}  \\
                          0 & 0 & 1 & \frac{3}{10}  & \frac{3}{10}  & \frac{-1}{5}
                      \end{array}
                  \right)
              \end{gather*}
              \par Vậy
              \[
                  \begin{pmatrix}
                      0 & 1 & 3 \\
                      2 & 3 & 5 \\
                      3 & 6 & 7
                  \end{pmatrix}^{-1}
                  =
                  \begin{pmatrix}
                      \frac{-9}{10} & \frac{11}{10} & \frac{-2}{5} \\
                      \frac{1}{10}  & \frac{-9}{10} & \frac{3}{5}  \\
                      \frac{3}{10}  & \frac{3}{10}  & \frac{-1}{5}
                  \end{pmatrix}.
              \]
              \endgroup{}
        \item
              \begingroup{}
              \allowdisplaybreaks{}
              \begin{gather*}
                  \left(\begin{array}{cccc|cccc}
                          1 & 2 & -1 & -2 & 1 & 0 & 0 & 0 \\
                          3 & 8 & 0  & -4 & 0 & 1 & 0 & 0 \\
                          2 & 2 & -4 & -3 & 0 & 0 & 1 & 0 \\
                          3 & 8 & -1 & -6 & 0 & 0 & 0 & 1
                      \end{array}
                  \right)
                  \stackrel{
                      \substack{
                          r_{2}:= r_{2} {-} 3r_{1} \\
                          r_{3}:= r_{3} {-} 2r_{1} \\
                          r_{4}:= r_{4} {-} 3r_{1}
                      }
                  }{\Longleftrightarrow}
                  \left(\begin{array}{cccc|cccc}
                          1 & 2  & -1 & -2 & 1  & 0 & 0 & 0 \\
                          0 & 2  & 3  & 2  & -3 & 1 & 0 & 0 \\
                          0 & -2 & -2 & 1  & -2 & 0 & 1 & 0 \\
                          0 & 2  & 2  & 0  & -3 & 0 & 0 & 1
                      \end{array}
                  \right) \\
                  \stackrel{
                      \substack{
                          r_{3}:= r_{3} + r_{2} \\
                          r_{4}:= r_{4} {-} r_{2}
                      }
                  }{\Longleftrightarrow}
                  \left(\begin{array}{cccc|cccc}
                          1 & 2 & -1 & -2 & 1  & 0  & 0 & 0 \\
                          0 & 2 & 3  & 2  & -3 & 1  & 0 & 0 \\
                          0 & 0 & 1  & 3  & -5 & 1  & 1 & 0 \\
                          0 & 0 & -1 & -2 & 0  & -1 & 0 & 1
                      \end{array}
                  \right)
                  \stackrel{
                      r_{4}:= r_{4} + r_{3}
                  }{\Longleftrightarrow}
                  \left(\begin{array}{cccc|cccc}
                          1 & 2 & -1 & -2 & 1  & 0 & 0 & 0 \\
                          0 & 2 & 3  & 2  & -3 & 1 & 0 & 0 \\
                          0 & 0 & 1  & 3  & -5 & 1 & 1 & 0 \\
                          0 & 0 & 0  & 1  & -5 & 0 & 1 & 1
                      \end{array}
                  \right) \\
                  \stackrel{
                      \substack{
                          r_{1}:= r_{1} + 2r_{4} \\
                          r_{2}:= r_{2} {-} 2r_{4} \\
                          r_{3}:= r_{3} {-} 3r_{4}
                      }
                  }{\Longleftrightarrow}
                  \left(\begin{array}{cccc|cccc}
                          1 & 2 & -1 & 0 & -9 & 0 & 2  & 2  \\
                          0 & 2 & 3  & 0 & 7  & 1 & -2 & -2 \\
                          0 & 0 & 1  & 0 & 10 & 1 & -2 & -3 \\
                          0 & 0 & 0  & 1 & -5 & 0 & 1  & 1
                      \end{array}
                  \right)
                  \stackrel{
                      \substack{
                          r_{1}:= r_{1} + r_{3} \\
                          r_{2}:= r_{2} {-} 3r_{3}
                      }
                  }{\Longleftrightarrow}
                  \left(\begin{array}{cccc|cccc}
                          1 & 2 & 0 & 0 & 1   & 1  & 0  & -1 \\
                          0 & 2 & 0 & 0 & -23 & -2 & 4  & 7  \\
                          0 & 0 & 1 & 0 & 10  & 1  & -2 & -3 \\
                          0 & 0 & 0 & 1 & -5  & 0  & 1  & 1
                      \end{array}
                  \right) \\
                  \stackrel{
                      r_{1}:= r_{1} {-} r_{2}
                  }{\Longleftrightarrow}
                  \left(\begin{array}{cccc|cccc}
                          1 & 0 & 0 & 0 & 24  & 3  & -4 & -8 \\
                          0 & 2 & 0 & 0 & -23 & -2 & 4  & 7  \\
                          0 & 0 & 1 & 0 & 10  & 1  & -2 & -3 \\
                          0 & 0 & 0 & 1 & -5  & 0  & 1  & 1
                      \end{array}
                  \right)
                  \stackrel{
                  r_{2}:= \frac{1}{2}r_{2}
                  }{\Longleftrightarrow}
                  \left(\begin{array}{cccc|cccc}
                          1 & 0 & 0 & 0 & 24            & 3  & -4 & -8          \\
                          0 & 1 & 0 & 0 & \frac{-23}{2} & -1 & 2  & \frac{7}{2} \\
                          0 & 0 & 1 & 0 & 10            & 1  & -2 & -3          \\
                          0 & 0 & 0 & 1 & -5            & 0  & 1  & 1
                      \end{array}
                  \right)
              \end{gather*}
              \par Vậy
              \[
                  \begin{pmatrix}
                      1 & 2 & -1 & -2 \\
                      3 & 8 & 0  & -4 \\
                      2 & 2 & -4 & -3 \\
                      3 & 8 & -1 & -6
                  \end{pmatrix}^{-1}
                  =
                  \begin{pmatrix}
                      24            & 3  & -4 & -8          \\
                      \frac{-23}{2} & -1 & 2  & \frac{7}{2} \\
                      10            & 1  & -2 & -3          \\
                      -5            & 0  & 1  & 1
                  \end{pmatrix}.
              \]
              \endgroup{}
    \end{enumerate}
\end{proof}

\par Nghiên cứu tính tương thích của các hệ phương trình sau, tìm một nghiệm riêng và nghiệm tổng quát của chúng:

% exercise 3.45
\begin{exercise}
    \[
        \begin{array}{ccccccccc}
            3x & - & 2y & + & 5z & + & 4t & = & 2, \\
            6x & - & 4y & + & 4z & + & 3t & = & 3, \\
            9x & - & 6y & + & 3z & + & 2t & = & 4.
        \end{array}
    \]
\end{exercise}

\begin{proof}[Lời giải]
    \par Thực hiện phép biến đổi sơ cấp trên ma trận hệ số mở rộng:
    \begingroup{}
    \allowdisplaybreaks{}
    \begin{gather*}
        \left(
        \begin{array}{cccc|c}
                3 & -2 & 5 & 4 & 2 \\
                6 & -4 & 4 & 3 & 3 \\
                9 & -6 & 3 & 2 & 4
            \end{array}
        \right)
        \stackrel{
            \substack{
                r_{2}:= r_{2} {-} 2r_{1} \\
                r_{3}:= r_{3} {-} 3r_{1}
            }
        }{\Longleftrightarrow}
        \left(\begin{array}{cccc|c}
                3 & -2 & 5   & 4   & 2  \\
                0 & 0  & -6  & -5  & -1 \\
                0 & 0  & -12 & -10 & -2
            \end{array}
        \right)
        \stackrel{
            r_{3}:= r_{3} {-} 2r_{2}
        }{\Longleftrightarrow}
        \left(\begin{array}{cccc|c}
                3 & -2 & 5  & 4  & 2  \\
                0 & 0  & -6 & -5 & -1 \\
                0 & 0  & 0  & 0  & 0
            \end{array}
        \right)
    \end{gather*}
    \endgroup{}
    \par Theo định lý Kronecker-Capelli, hệ phương trình tuyến tính trên có nghiệm.
    \par Một nghiệm riêng của hệ trên là:
    \[
        (x, y, z, t) = (1, 1, 1, -1).
    \]
    \par Nghiệm của hệ phương trình tuyến tính thuần nhất:
    \[
        (x, y, z, t) = (a, b, -15a + 10b, 18a - 12b).
    \]
    \par Nghiệm tổng quát của hệ phương trình tuyến tính trên là:
    \[
        (x, y, z, t) = (1 + a, 1 + b, 1 - 15a + 10b, -1 + 18a - 12b).
    \]
\end{proof}

% exercise 3.46
\begin{exercise}
    \[
        \begin{array}{ccccccccc}
            8x & + & 6y & + & 5z & + & 2t & = & 21, \\
            3x & + & 3y & + & 2z & + & t  & = & 10, \\
            4x & + & 2y & + & 3z & + & t  & = & 8,  \\
            3x & + & 5y & + & z  & + & t  & = & 15, \\
            7x & + & 4y & + & 5z & + & 2t & = & 18.
        \end{array}
    \]
\end{exercise}

\begin{proof}[Lời giải]
    \par Thực hiện phép biến đổi sơ cấp trên ma trận hệ số mở rộng:
    \begingroup{}
    \allowdisplaybreaks{}
    \begin{gather*}
        \left(
        \begin{array}{cccc|c}
                8 & 6 & 5 & 2 & 21 \\
                3 & 3 & 2 & 1 & 10 \\
                4 & 2 & 3 & 1 & 8  \\
                3 & 5 & 1 & 1 & 15 \\
                7 & 4 & 5 & 2 & 18
            \end{array}
        \right)
        \stackrel{
            \substack{
                r_{2}:= 2r_{2} {-} r_{1} \\
                r_{3}:= 2r_{3} {-} r_{1} \\
                r_{4}:= 2r_{4} {-} r_{1} \\
                r_{5}:= r_{5} {-} r_{1}
            }
        }{\Longleftrightarrow}
        \left(\begin{array}{cccc|c}
                8  & 6  & 5  & 2 & 21 \\
                -2 & 0  & -1 & 0 & -1 \\
                0  & -2 & 1  & 0 & -5 \\
                -2 & 4  & -3 & 0 & 9  \\
                -1 & -2 & 0  & 0 & -3
            \end{array}
        \right)
        \stackrel{
        \substack{
        r_{3}:= r_{3} + r_{2} \\
        r_{4}:= r_{4} {-} 3r_{2} \\
        r_{5}:= -r_{5}
        }
        }{\Longleftrightarrow}
        \left(\begin{array}{cccc|c}
                8  & 6  & 5  & 2 & 21 \\
                -2 & 0  & -1 & 0 & -1 \\
                -2 & -2 & 0  & 0 & -6 \\
                4  & 4  & 0  & 0 & 12 \\
                1  & 2  & 0  & 0 & 3
            \end{array}
        \right) \\
        \stackrel{
            r_{4}:= r_{4} + 2r_{3}
        }{\Longleftrightarrow}
        \left(\begin{array}{cccc|c}
                8  & 6  & 5  & 2 & 21 \\
                -2 & 0  & -1 & 0 & -1 \\
                -2 & -2 & 0  & 0 & -6 \\
                0  & 0  & 0  & 0 & 0  \\
                1  & 2  & 0  & 0 & 3
            \end{array}
        \right)
        \stackrel{
        \substack{
        r_{2}:= -r_{2} \\
        r_{3}:= \frac{-1}{2}r_{3}
        }
        }{\Longleftrightarrow}
        \left(\begin{array}{cccc|c}
                8 & 6 & 5 & 2 & 21 \\
                2 & 0 & 1 & 0 & 1  \\
                1 & 1 & 0 & 0 & 3  \\
                0 & 0 & 0 & 0 & 0  \\
                1 & 2 & 0 & 0 & 3
            \end{array}
        \right)
        \stackrel{
            r_{5}:= r_{5} {-} r_{3}
        }{\Longleftrightarrow}
        \left(\begin{array}{cccc|c}
                8 & 6 & 5 & 2 & 21 \\
                2 & 0 & 1 & 0 & 1  \\
                1 & 1 & 0 & 0 & 3  \\
                0 & 0 & 0 & 0 & 0  \\
                0 & 1 & 0 & 0 & 0
            \end{array}
        \right) \\
        \stackrel{
            r_{3}:= r_{3} {-} r_{5}
        }{\Longleftrightarrow}
        \left(\begin{array}{cccc|c}
                8 & 6 & 5 & 2 & 21 \\
                2 & 0 & 1 & 0 & 1  \\
                1 & 0 & 0 & 0 & 3  \\
                0 & 0 & 0 & 0 & 0  \\
                0 & 1 & 0 & 0 & 0
            \end{array}
        \right)
        \stackrel{
            r_{2}:= r_{2} {-} 2r_{3}
        }{\Longleftrightarrow}
        \left(\begin{array}{cccc|c}
                8 & 6 & 5 & 2 & 21 \\
                0 & 0 & 1 & 0 & -5 \\
                1 & 0 & 0 & 0 & 3  \\
                0 & 0 & 0 & 0 & 0  \\
                0 & 1 & 0 & 0 & 0
            \end{array}
        \right)
        \stackrel{
            r_{1}:= r_{1} {-} 8r_{3} {-} 5r_{2} {-} 6r_{5}
        }{\Longleftrightarrow}
        \left(\begin{array}{cccc|c}
                0 & 0 & 0 & 2 & 22 \\
                0 & 0 & 1 & 0 & -5 \\
                1 & 0 & 0 & 0 & 3  \\
                0 & 0 & 0 & 0 & 0  \\
                0 & 1 & 0 & 0 & 0
            \end{array}
        \right)
    \end{gather*}
    \par Vậy hệ phương trình có nghiệm duy nhất:
    \[
        (x, y, z, t) = (3, 0, -5, 11).
    \]
    \endgroup{}
\end{proof}

\end{document}
