\documentclass[class=linearalgebra,crop=false]{standalone}

\newcommand{\sgn}[1]{\text{sgn}\left({#1}\right)}
\setcounter{lemma}{0}

\begin{document}

\chapter{Định thức và hệ phương trình tuyến tính}

\par Thực hiện các phép nhân sau đây, viết các phép thế thu được thành tích của những xích rời rạc và tính dấu của chúng.

\begin{exercise}
    $
        \begin{pmatrix}
            1 & 2 & 3 & 4 & 5 \\
            2 & 4 & 5 & 1 & 3
        \end{pmatrix}
        \begin{pmatrix}
            1 & 2 & 3 & 4 & 5 \\
            4 & 3 & 5 & 1 & 2
        \end{pmatrix}
    $.
\end{exercise}

\begin{proof}[Lời giải]
    \[
        \begin{pmatrix}
            1 & 2 & 3 & 4 & 5 \\
            2 & 4 & 5 & 1 & 3
        \end{pmatrix}
        \begin{pmatrix}
            1 & 2 & 3 & 4 & 5 \\
            4 & 3 & 5 & 1 & 2
        \end{pmatrix}
        =
        \begin{pmatrix}
            4 & 3 & 5 & 1 & 2 \\
            1 & 5 & 3 & 2 & 4
        \end{pmatrix}
        \begin{pmatrix}
            1 & 2 & 3 & 4 & 5 \\
            4 & 3 & 5 & 1 & 2
        \end{pmatrix}
        =
        \begin{pmatrix}
            1 & 2 & 3 & 4 & 5 \\
            1 & 5 & 3 & 2 & 4
        \end{pmatrix}.
    \]
    \[
        \begin{pmatrix}
            1 & 2 & 3 & 4 & 5 \\
            1 & 5 & 3 & 2 & 4
        \end{pmatrix}
        =
        (1)(2,5,4)(3).
    \]
    \[
        \sgn{
            \begin{matrix}
                1 & 2 & 3 & 4 & 5 \\
                1 & 5 & 3 & 2 & 4
            \end{matrix}
        }
        = \sgn{1}\sgn{2,5,4}\sgn{3}
        = 1.
    \]
\end{proof}

\begin{exercise}
    $
        \begin{pmatrix}
            1 & 2 & 3 & 4 & 5 \\
            3 & 5 & 4 & 1 & 2
        \end{pmatrix}
        \begin{pmatrix}
            1 & 2 & 3 & 4 & 5 \\
            4 & 3 & 1 & 5 & 2
        \end{pmatrix}
    $.
\end{exercise}

\begin{proof}[Lời giải]
    \[
        \begin{pmatrix}
            1 & 2 & 3 & 4 & 5 \\
            3 & 5 & 4 & 1 & 2
        \end{pmatrix}
        \begin{pmatrix}
            1 & 2 & 3 & 4 & 5 \\
            4 & 3 & 1 & 5 & 2
        \end{pmatrix}
        =
        \begin{pmatrix}
            4 & 3 & 1 & 5 & 2 \\
            1 & 4 & 3 & 2 & 5
        \end{pmatrix}
        \begin{pmatrix}
            1 & 2 & 3 & 4 & 5 \\
            4 & 3 & 1 & 5 & 2
        \end{pmatrix}
        =
        \begin{pmatrix}
            1 & 2 & 3 & 4 & 5 \\
            1 & 4 & 3 & 2 & 5
        \end{pmatrix}.
    \]
    \[
        \begin{pmatrix}
            1 & 2 & 3 & 4 & 5 \\
            1 & 4 & 3 & 2 & 5
        \end{pmatrix}
        =
        (1)(2,4)(3)(5).
    \]
    \[
        \sgn{
            \begin{matrix}
                1 & 2 & 3 & 4 & 5 \\
                1 & 4 & 3 & 2 & 5
            \end{matrix}
        }
        = \sgn{1}\sgn{2,4}\sgn{3}\sgn{5}
        = -1.
    \]
\end{proof}

\begin{exercise}
    $(1,2)(2,3)\ldots (n-1,n)$.
\end{exercise}

\begin{lemma}\label{chapter3:cycles-product}
    $(a_{1}, a_{2}, \ldots, a_{k})(a_{k},a_{k+1}) = (a_{1},a_{2},\ldots, a_{k+1})$.
\end{lemma}

\begin{proof}[Chứng minh bổ đề]
    \par Xét dãy
        \[
            a_{1}, a_{2}, \ldots, a_{k-1}, a_{k}, a_{k+1}.
        \]
    \par Sau khi tác động bằng $(a_{k},a_{k+1})$, dãy trên trở thành:
        \[
            a_{1}, a_{2}, \ldots, a_{k-1}, a_{k+1}, a_{k}.
        \]
    \par Sau khi tác động bằng $(a_{1}, a_{2}, \ldots, a_{k})$, dãy trên (liên trên) trở thành:
        \[
            a_{2}, a_{3}, \ldots, a_{k}, a_{k+1}, a_{1}.
        \]
    \par Theo định nghĩa về xích, ta có điều phải chứng minh.
\end{proof}

\begin{proof}[Lời giải]
    \par Theo bổ đề~\ref{chapter3:cycles-product}:
        \[
            (1,2)(2,3)\ldots (n-1,n) = (1,2,\ldots,n)
            =
            \begin{pmatrix}
                1 & 2 & \cdots & n-1 & n \\
                2 & 3 & \cdots & n   & 1
            \end{pmatrix}
        \]
    \par $(1,2,\ldots, n)$ chính là một xích.
        \[
            \sgn{1,2,\ldots,n} = \sgn{1,2}\sgn{2,3}\ldots\sgn{n-1,n} = (-1){}^{n-1}.
        \]
\end{proof}

\begin{exercise}
    $(1,2,3)(2,3,4)(3,4,5)\ldots (n-2,n-1,n)$.
\end{exercise}

\begin{proof}[Lời giải]
    \par Theo bổ đề~\ref{chapter3:cycles-product}, nếu $n > 3$:
    \begin{align*}
        (1,2,3)(2,3,4)(3,4,5)\ldots (n-2,n-1,n)
        & = (1,2)(2,3)(2,3)(3,4)(3,4)(4,5) \ldots (n-2,n-1)(n-1,n) \\
        & = (1,2)(2,3){}^{2}(3,4){}^{2}\ldots (n-2,n-1){}^{2}(n-1,n) \\
        & = (1,2)(n-1,n)\qquad\text{(đây là 2 xích rời nhau)} \\
        & =
        \begin{pmatrix}
            1 & 2 & 3 & \cdots & n-2 & n-1 & n   \\
            2 & 1 & 3 & \cdots & n-2 & n   & n-1
        \end{pmatrix}.
    \end{align*}
    \par Nếu $n = 3$:
    \[
        (1,2,3) =
        \begin{pmatrix}
            1 & 2 & 3 \\
            2 & 3 & 1
        \end{pmatrix}.
    \]
    \par Trong cả hai trường hợp, dấu của phép thế (kết quả) là 1.
\end{proof}

\begin{exercise}
    Cho hai cách sắp thành dãy $a_{1}$, $a_{2}$, \ldots, $a_{n}$ và $b_{1}$, $b_{2}$, \ldots, $b_{n}$ của $n$ số tự nhiên đầu tiên. Chứng minh rằng có thể đưa cách sắp này về cách sắp kia bằng cách sử dụng không quá $n-1$ phép thế sơ cấp.
\end{exercise}

\begin{lemma}\label{chapter3:product-of-disjoint-cycles}
    Mọi phép thế đều có thể được viết dưới dạng tích của các xích rời nhau.
\end{lemma}

\begin{proof}[Chứng minh bổ đề~\ref{chapter3:product-of-disjoint-cycles}]
\end{proof}

\begin{lemma}\label{chapter3:product-of-transpositions}
    Một xích độ dài $k$ ($k > 1$) có thể viết được dưới dạng tích của $k-1$ phép thế sơ cấp.
\end{lemma}

\begin{proof}[Chứng minh bổ đề~\ref{chapter3:product-of-transpositions}]
\end{proof}

\begin{proof}
\end{proof}

\end{document}
