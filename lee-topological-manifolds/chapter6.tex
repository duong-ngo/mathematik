% chktex-file 8
\chapter{Compact Surfaces}

\section{Surfaces}

\begin{exercise}{6.3}
	Prove Proposition 6.2 (b).
\end{exercise}

\begin{proof}
	Define \( f: \mathbb{S}^{1} \to \partial S \) as follows:
	\[
		f(x, y) = \left(\frac{x}{\left\vert x\right\vert + \left\vert y\right\vert}, \frac{y}{\left\vert x\right\vert + \left\vert y\right\vert}\right).
	\]

	This map is bijective, continuous and it is closed by the closed map lemma. Therefore it is a homeomorphism. The map \( f \) extends to the homeomorphism \( F: \overline{\mathbb{B}}^{2} \to S \) which is given by:
	\[
		F(x, y) = \begin{cases}
			\sqrt{x^{2} + y^{2}} f^{-1}\left(\dfrac{(x, y)}{\sqrt{x^{2} + y^{2}}}\right) & \text{if \( (x, y) \ne (0, 0) \)} \\
			(0, 0)                                                                       & \text{if \( (x, y) = (0, 0) \)}
		\end{cases}
	\]

	The constructions of \( f \) and \( F \) are based on Proposition~\ref{prop:5.1}.

	From part (a), there is a quotient map \( q: S \to \mathbb{S}^{2} \) that identifies \( (x, y) \) and \( (-x, -y) \) on \( \partial S \). The composition \( q\circ F^{-1}: S \to \mathbb{P}^{2} \) is a quotient map according to part (a). This composition makes the same identification on \( S \) as the given equivalence relation.

	Thus the square region \( S = \left\{ (x, y): \left\vert x \right\vert + \left\vert y \right\vert \leq 1 \right\} \) modulo the equivalence relation generated by \( (x, y) \sim (-x, -y) \) for each \( (x, y) \in \partial S \) is homeomorphic to the real projective plane \( \mathbb{P}^{2} \).
\end{proof}

\section{Connected Sums of Surfaces}

\section{Polygonal Presentations of Surfaces}

\section{The Classification Theorem}

\section{The Euler Characteristic}

\section{Orientability}

\section*{Problems}\addcontentsline{toc}{section}{Problems}

\begin{problem}{6-1}\label{problem:6-1}
\end{problem}

\begin{problem}{6-2}\label{problem:6-2}
\end{problem}

\begin{problem}{6-3}\label{problem:6-3}
\end{problem}

\begin{problem}{6-4}\label{problem:6-4}
\end{problem}

\begin{problem}{6-5}\label{problem:6-5}
\end{problem}

\begin{problem}{6-6}\label{problem:6-6}
\end{problem}
