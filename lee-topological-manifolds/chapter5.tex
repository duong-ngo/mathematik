% chktex-file 8
\chapter{Cell Complexes}

\section{Cell Complexes and CW Complexes}

\begin{prop}{5.1}\label{prop:5.1}
	If $D\subseteq \mathbb{R}^{n}$ is a compact convex subset with nonempty interior, then $D$ is a closed $n$-cell and its interior is an open $n$-cell. In fact, given any point $p \in \operatorname{Int}D$, there exists a homeomorphism $F: \overline{\mathbb{B}}^{n} \to D$ that sends 0 to $p$, $\mathbb{B}^{n}$ to $\operatorname{Int}D$, and $\mathbb{S}^{n-1}$ to $\partial D$.
\end{prop}

\begin{proof}
	The proposition is true by definition when $n = 0$, so assume that $n > 0$.

	Let $p$ be an interior point of $D$. Since the translation $x\mapsto x - p$ is a homeomorphism of $\mathbb{R}^{n}$ with itself, we can assume $p = 0 \in \operatorname{Int} D$. $p \in \operatorname{Int} D$ so there is $\varepsilon > 0$ such that $B_{\varepsilon}(0) \subseteq \operatorname{Int} D \subseteq D$. The dilation $x\mapsto x/\varepsilon$ is a homeomorphism of $\mathbb{R}^{n}$ with itself, so we can assume $\mathbb{B}^{n} = B_{1}(0) \subseteq D$.

	We will show that each closed ray starting at the origin intersects the boundary $\partial D$ in exactly one point.

	Let $R$ be a closed ray starting at the origin. Because $D$ is compact, its intersection with $R$ is compact. From the extremum value theorem, there is a point $x_{0} \in R\cap D$ that maximizes the distance to the origin. Any point other than $x_{0}$ on the line segment connecting $0$ and $x_{0}$ is of the form $\lambda x_{0}$ for some $0\leq \lambda < 1$. Consider the open ball $B_{1-\lambda}(\lambda x_{0})$ and an arbitrary point $z$ in it. The homothety with center $x_{0}$ and factor $1 - \lambda$ maps $z$ to $y = \dfrac{z - \lambda x_{0}}{1 - \lambda}$. $z \in B_{1 - \lambda}(\lambda x_{0})$ so $\abs{z - \lambda x_{0}} < \abs{1 - \lambda}$, which means $\abs{y} < 1$, so $y \in B_{1}(0) \subseteq D$. Because $y, x_{0} \in D$ and $D$ is convex, $z = (1 - \lambda)y + \lambda x_{0}$ is also in $D$. Therefore $B_{1 - \lambda}(\lambda x_{0}) \subseteq D$, which means $\lambda x_{0}$ is an interior point of $D$.

	We define a map $f: \partial D \to \mathbb{S}^{n-1}$ by $f(x) = x/\abs{x}$. $f$ is a restriction of a continuous map so $f$ is continuous. $\partial D$ is compact and $\mathbb{S}^{n-1}$ is Hausdorff so $f$ is a closed map, according to the closed map lemma. $f$ is bijective due to the previous paragraph. Therefore $f$ is a homeomorphism.

	Define $F: \overline{\mathbb{B}}^{n} \to D$ by
	\begin{align*}
		F(x) = \begin{cases}
			       \abs{x} f^{-1}\left(\frac{x}{\abs{x}}\right), & x\ne 0; \\
			       0                                             & x = 0.
		       \end{cases}
	\end{align*}

	$F$ is continuous on $\overline{\mathbb{B}}^{n}\smallsetminus\set{0}$ because $f^{-1}$ and $x\mapsto \abs{x}$ are continuous. $\partial D$ is compact in $\mathbb{R}^{n}$ so it is bounded, so there exists $R > 0$ such that $d(a, b) < R$ for any $a, b \in \partial D$. For every $x \ne 0$
	\begin{align*}
		\abs{F(x) - 0} = \abs{x}\abs{f^{-1}\left(\frac{x}{\abs{x}}\right)} = \abs{x}\abs{f^{-1}\left(\frac{x}{\abs{x}}\right) - f^{-1}(0)} < R\abs{x}
	\end{align*}

	so for every $\varepsilon > 0$, $\abs{F(x) - 0} < \varepsilon$ whenever $\abs{x - 0} < \dfrac{\varepsilon}{R}$. Hence $F$ is continuous at $0$. Therefore $F$ is continuous. $F$ is closed due to the closed map lemma.

	If $F(x) = F(0)$ then $x = 0$ due to the definition. If $F(x) = F(y)$ in which $x, y\ne 0$ then $\abs{x}f^{-1}(x/\abs{x}) = \abs{y}f^{-1}(y/\abs{y})$. $x, y$ must be on the same closed ray from the origin because $F$ maps distinct closed rays to distinct closed rays. Therefore $x/\abs{x} = y/\abs{y}$ and $\abs{x} = \abs{y}$, which implies $x = y$. Hence $f$ is injective. $F$ is surjective because every $y\in D$ lies on a closed ray from the origin. So $F$ is bijective.

	Therefore $F$ is a homeomorphism.
\end{proof}

\subsection{Cell Decompositions}

A \textbf{cell decomposition of $X$} is a partition $\mathscr{E}$ of $X$ into subspaces that are open cells of various dimensions, such that
\begin{itemize}
	\item for each cell $e \in \mathscr{E}$ of dimension $n\geq 1$, there exists a continuous map $\Phi$ from some closed $n$-cell $D$ into $X$ (called a \textbf{characteristic map for $e$}) that restricts to a homeomorphism from $\operatorname{Int} D$ onto $e$ and maps $\partial D$ into the union of all cells of $\mathscr{E}$ of dimensions strictly less than $n$.
\end{itemize}

A \textbf{cell complex} is a Hausdorff space $X$ together with a specific cell decomposition of $X$.

Each cell $e \in \mathscr{E}$ needs not to be open in $X$.

\subsection{CW Complexes}

Suppose $X$ is a topological space, and $\mathscr{B}$ is any family of subspaces of $X$ whose union is $X$. The topology of $X$ is \textbf{coherent with $\mathscr{B}$} means a subset $U \subseteq X$ is open in $X$ if and only if its intersection with each $B\in\mathscr{B}$ is open in $B$. This definition is equivalent to that $U$ is closed in $X$ if and only if $U\cap B$ is closed in $B$ for each $B \in \mathscr{B}$.

A space is compactly generated if and only if its topology is coherent with the collection of all of its compact subspaces.

\begin{exercise}{5.3}\label{exercise:5.3}
	Prove Proposition 5.2.

	Suppose $X$ is a topological space whose topology is coherent with a family $\mathscr{B}$ of subspaces.
	\begin{enumerate}[label={(\alph*)}]
		\item If $Y$ is another topological space, then a map $f: X\to Y$ is continuous if and only if $f\vert_{B}$ is continuous for every $B\in \mathscr{B}$.
		\item The map $\coprod_{B\in\mathscr{B}}B \to X$ induced by inclusion of each set $B\xhookrightarrow{} X$ is a quotient map.
	\end{enumerate}
\end{exercise}

\begin{proof}
	\begin{enumerate}[label={(\alph*)}]
		\item If $f: X\to Y$ is continuous then the restriction map $f\vert_{B}$ is continuous for every $B \in \mathscr{B}$.

		      Suppose $f\vert_{B}$ is continuous for every $B \in \mathscr{B}$. Let $V$ be an open subset of $Y$. For every $B \in \mathscr{B}$, ${(f\vert_{B})}^{-1}(V) = {(f\vert_{B})}^{-1}(V \cap f(B))$ is open in $B$ because $f\vert_{B}$ is continuous. Moreover, $f^{-1}(V) \cap B = {(f\vert_{B})}^{-1}(V)$ so $f^{-1}(V) \cap B$ is open in $B$ for every $B \in \mathscr{B}$. $X$ is coherent with $\mathscr{B}$ so $f^{-1}(V)$ is open in $X$. Hence $f$ is continuous.
		\item Denote the given map by $q$. $q$ is surjective because for every $x\in X$, there exists $B \in \mathscr{B}$ such that $x \in B$. Let $U$ be a subset of $X$ then
		      \begin{align*}
			      q^{-1}(U) = \coprod_{B\in\mathscr{B}} (U\cap B).
		      \end{align*}

		      If $U$ is open in $X$ then $U\cap B$ is open in $B$ for every $B\in\mathscr{B}$, so $q^{-1}(U)$ is open in $\coprod_{B\in\mathscr{B}}B$. Conversely, if $q^{-1}(U)$ is open in $\coprod_{B\in\mathscr{B}}B$ then $U\cap B$ is open in $B$ for every $B\in\mathscr{B}$, which implies $U$ is open in $X$, according to the definition of coheherent topology.

		      Hence $U$ is open in $X$ if and only if $q^{-1}(U)$ is open in $\coprod_{B\in\mathscr{B}}B$. Together with $q$ being surjective, we conclude that $q$ is a quotient map.
	\end{enumerate}
\end{proof}

A \textbf{CW complex} is cell complex $(X, \mathscr{E})$ such that
\begin{itemize}
	\item [(C)] The closure of each cell is contained in a union of finitely many cells. This property is called \textbf{closure finiteness}.
	\item [(W)] The topology of $X$ is coherent with the family of closed subspaces $\set{ \overline{e} : e \in \mathscr{E} }$. This property is called \textbf{weak topology}.
\end{itemize}

\begin{note}\label{note:characteristic-map-as-a-quotient-map}
	Let $e$ be an open cell of a cell complex $(X, \mathscr{E})$ and $\Phi: D \to X$ be a characteristic map of $e$. Then $\Phi: D \to \overline{e}$ is a quotient map and closed, and $\Phi^{-1}(\overline{e}\smallsetminus e) = \partial D$.
\end{note}

\begin{proof}
	From the definition of closed cell and open cell, we deduce that $D = \overline{\operatorname{Int} D}$.
	\begin{align*}
		\Phi(D) & = \Phi(\overline{\operatorname{Int} D})                                                                                   \\
		        & \subseteq \overline{\Phi(\operatorname{Int} D)} & \text{(Problem~\ref{problem:2-6} or Proposition 2.30)}                  \\
		        & = \overline{e}                                  & \text{($\Phi\vert_{\operatorname{Int} D}$ is a homeomorphism onto $e$)}
	\end{align*}

	$D$ is compact and $\Phi$ is continuous so $\Phi(D)$ is a compact subset of $X$. Moreover, $X$ is Hausdorff (according to the definition of cell complexes) so $\Phi(D)$ is a closed subset of $X$. $\Phi(D)$ is closed in $X$, $e\subseteq \Phi(D) \subseteq \overline{e}$, and $\overline{e}$ is the smallest closed set containing $e$, so $\Phi(D) = \overline{e}$. Therefore $\Phi: D \to \overline{e}$ is a surjective continuous map. By the closed map lemma, $\Phi: D \to \overline{e}$ is a closed map. Hence $\Phi: D \to \overline{e}$ is a quotient map and closed.

	Finally $\Phi^{-1}(\overline{e}\smallsetminus e) = \Phi^{-1}(\overline{e}) \smallsetminus \Phi^{-1}(e) = D \smallsetminus \operatorname{Int} D = \partial D$.
\end{proof}

Locally finite complexes (and thus all finite ones), (C) and (W) are automatic.

\begin{prop}{5.4}
	Let $X$ be a Hausdorff space, and let $\mathscr{E}$ be a cell decomposition of $X$. If $\mathscr{E}$ is locally finite, then it is a CW decomposition.
\end{prop}

\begin{quotation}
	We don't really use the Hausdorff property. We need the Hausdorff property only to follow the definition of cell complexes in the book.
\end{quotation}

\begin{proof}
	Suppose that $(X, \mathscr{E})$ is a cell complex and $\mathscr{E}$ is locally finite.

	For each $e\in \mathscr{E}$, every point of $\overline{e}$ has a neighborhood that intersects only finitely many cells in $\mathscr{E}$. $\overline{e}$ is compact (see Note~\ref{note:characteristic-map-as-a-quotient-map}) so it is covered by finitely many such neighborhoods. Therefore $\overline{e}$ is contained in a union of finitely many cells.

	Let $A \subseteq X$ be a subset whose intersection with $\overline{e}$ is closed in $\overline{e}$ for each $e\in \mathscr{E}$. Let $x$ be a point of $X\smallsetminus A$. The collection $\set{ \overline{e} : e\in\mathscr{E} }$ is locally finite according to Lemma 4.74, so there exists a neighborhood $W$ of $x$ that intersects the closure of only finitely many cells, say $\overline{e}_{1}, \ldots, \overline{e}_{m}$. $A \cap \overline{e}_{i}$ is closed in $\overline{e}_{i}$ and thus in $X$ (Exercise~\ref{exercise:3.6}), then $W\smallsetminus A = W \smallsetminus \bigcup^{m}_{i=1}(A\cap \overline{e}_{i})$ is a neighborhood of $x$ contained in $X\smallsetminus A$. Because $x$ is an arbitrary point of $X\smallsetminus A$, it follows that $X\smallsetminus A$ is open in $X$, so $A$ is closed in $X$. Therefore the topology of $X$ is coherent with $\set{ \overline{e} : e \in \mathscr{E} }$.

	Thus $\mathscr{E}$ is a CW decomposition.
\end{proof}

Suppose $X$ is a CW complex. If there is an integer $n$ such that all the cells of $X$ have dimension at most $n$, then we say $X$ is \textbf{finite-dimensional}; otherwise, it is \textbf{infinite-dimensional}. If it is finite-dimensional, the \textbf{dimension of $X$} is the largest $n$ such that $X$ contains at least one $n$-cell. (The fact that this is well-defined depends on the theorem of invariance of dimension.) Of course, a finite complex is always finite-dimensional.

\begin{prop}{5.5}
	Suppose $X$ is an $n$-dimensional CW complex. Then every $n$-cell of $X$ is an open subset of $X$.
\end{prop}

\begin{proof}
	Suppose $e_{0}$ is an $n$-cell of $X$. Let $\Phi_{0}: D_{0} \to X$ be a characteristic map for $e_{0}$. Consider $\Phi_{0}$ as a quotient map from $D_{0}$ onto $\overline{e_{0}}$ (see Note~\ref{note:characteristic-map-as-a-quotient-map}). Since $\Phi_{0}^{-1}(e_{0}) = \operatorname{Int} D_{0}$ is open in $D_{0}$, it follows that $e_{0}$ is open in $\overline{e_{0}}$ (the definition of quotient topology).

	If $e$ is a cell other than $e_{0}$ (of dimension $k$, then $k \leq n$) of $X$ and $\Phi: D \to X$ is a characteristic map for $e$, then $e_{0} \cap e = \emptyset$, so $e_{0} \cap \overline{e} \subseteq \overline{e} \smallsetminus e$.

	$\overline{e}\smallsetminus e$ is contained in the union of all cells of $\mathscr{E}$ of dimensions less than $n$ because $\Phi^{-1}(\overline{e}\smallsetminus e) = \partial D$ and $\Phi$ maps $\partial D$ into the union of all cells of dimensions strictly less than $k$ (hence less than $n$). On the other hand, $e_{0}$ has dimension $n$, it follows that $e_{0} \cap (\overline{e}\smallsetminus e) = \emptyset$. Moreover, $e_{0} \cap e = \emptyset$, so $e_{0} \cap \overline{e} = \emptyset$.

	Hence the intersection of $e_{0}$ with the closure of any different cell $e$ is open in $\overline{e}$ and the intersection of $e_{0}$ with $\overline{e_{0}}$ is open in $\overline{e_{0}}$. Therefore $e_{0}$ is open in $X$ as the topology of $X$ is coherent with $\set{ \overline{e}: e \in \mathscr{E} }$.
\end{proof}

\begin{note}\label{note:n-skeleton}
	The $n$-skeleton $X_{n}$ of a CW complex $X$ is a subcomplex of $X$ of dimension not exceeding $n$.
\end{note}

\begin{proof}
	$X_{n}$ is the union of all cells of dimensions not exceeding $n$.

	Let $e$ be a cell of $X$ such that $e\subseteq X_{n}$ and $\Phi: D \to X$ be a characteristic map for $e$ then $\Phi(\operatorname{Int} D) = e$ and $\Phi(D) = \overline{e}$. From the definition of cell complexes, $\Phi$ maps $\partial D$ into the union of cells of dimensions strictly less than $n$ (these cells are contained in $X_{n}$), it follows that $\overline{e}$ is contained in $X_{n}$. Therefore $X_{n}$ is a subcomplex of $X$. Moreover, $X_{n}$ is of dimension $\leq n$ due to the definition of the dimension of a finite-dimensional CW complex.
\end{proof}

\section{Topological Properties of CW Complexes}

\subsection{Inductive Construction of CW Complexes}

\subsection{CW Complexes as Manifolds}

\section{Classification of 1-Dimensional Manifolds}

\section{Simplicial Complexes}

\subsection{Simplicial Maps}

\subsection{Abstract Simplicial Complexes}

\section*{Problems}\addcontentsline{toc}{section}{Problems}

\begin{note}\label{note:closed-cell-homeomorphism}
	Suppose $D$ is a closed $n$-cell ($n\geq 1$) and $p\in \operatorname{Int} D$ then there is a homeomorphism $\varphi: D \to \overline{\mathbb{B}}^{n}$ such that $\varphi(\operatorname{Int} D) = \mathbb{B}^{n}$, $\varphi(\partial D) = \partial \overline{\mathbb{B}}^{n}$, and $\varphi(p) = 0$.
\end{note}

\begin{proof}
	Suppose $D$ is a closed $n$-cell. From the definition of closed $n$-cell, there exists a homeomorphism $\psi: D \to \overline{\mathbb{B}}^{n}$. Therefore
	\begin{align*}
		\psi(\operatorname{Int} D) & = \operatorname{Int} \overline{\mathbb{B}}^{n} = \mathbb{B}^{n}, \\
		\psi(\partial D)           & = \partial \overline{\mathbb{B}}^{n},                            \\
		\psi(p)                    & \in \mathbb{B}^{n}.
	\end{align*}

	Since $\overline{\mathbb{B}}^{n}$ is a convex closed subset of $\mathbb{R}^{n}$ with nonempty interior, there is a homeomorphism $f: \overline{\mathbb{B}}^{n} \to \overline{\mathbb{B}}^{n}$ such that $f(\psi(p)) = 0$, according to Proposition~\ref{prop:5.1}. Define $\varphi = f\circ \psi$, we obtain a homeomorphism $\varphi: D \to \overline{\mathbb{B}}^{n}$ such that $\varphi(p) = f(\psi(p)) = 0$, $\varphi(\operatorname{Int} D) = \mathbb{B}^{n}$, $\varphi(\partial D) = \partial \overline{\mathbb{B}}^{n}$.
\end{proof}

\begin{problem}{5-1}\label{problem:5-1}
Suppose $D$ and $D'$ are closed cells (not necessarily of the same dimension).
\begin{enumerate}[label={(\alph*)}]
	\item Show that every continuous map $f: \partial D \to \partial D'$ extends to a continuous map $F: D \to D'$, with $F(\operatorname{Int} D) \subseteq \operatorname{Int} D'$.
	\item Given points $p \in \operatorname{Int} D$ and $p' \in \operatorname{Int} D'$, show that $F$ can be chosen to take $p$ to $p'$.
	\item Show that if $f$ is a homeomorphism, then $F$ can also be chosen to be a homeomorphism.
\end{enumerate}
\end{problem}

\begin{proof}
	\begin{enumerate}[label={(\alph*)}]
		\item We prove a particular case first: $D = \overline{\mathbb{B}}^{n}$ and $D' = \overline{\mathbb{B}}^{m}$. A map $g: D\to D'$ is defined by $g(0) = 0$, and for every $x \in D$ other than 0
		      \begin{align*}
			      g(x) = \abs{x}\cdot f\left(\frac{x}{\abs{x}}\right).
		      \end{align*}

		      $g$ is continuous, $g\vert_{\partial \overline{\mathbb{B}}^{n}} = f$ and $g(\mathbb{B}^{n}) \subseteq \mathbb{B}^{m}$ because $\abs{g(x)} < 1$ for every $x \in \mathbb{B}^{n}$. We will use this construction to prove the general case.

		      Now assume that $D$ is a closed $m$-cell and $D'$ is a closed $n$-cell. Due to the definition of closed cell and the theorem of the invariance of the boundary, there are homeomorphisms $\varphi: \overline{\mathbb{B}}^{n} \to D$ and $\varphi': \overline{\mathbb{B}}^{m} \to D'$ such that $\varphi$ maps $\mathbb{B}^{n}$ to $\operatorname{Int} D$, $\varphi'$ maps $\mathbb{B}^{m}$ to $\operatorname{Int} D'$.

		      ${(\varphi')}^{-1}\circ f\circ \varphi$ is a continuous map from $\partial \overline{\mathbb{B}}^{n}$ to $\partial \overline{\mathbb{B}}^{m}$, so it extends to a continuous map $g: \overline{\mathbb{B}}^{n} \to \overline{\mathbb{B}}^{m}$ such that $g(\mathbb{B}^{n}) \subseteq \mathbb{B}^{m}$. Define $F = {(\varphi')}\circ g\circ \varphi^{-1}$ then $F: D \to D'$ is continuous, $F\vert_{\partial D} = f$ and
		      \begin{align*}
			      F(\operatorname{Int} D) = (\varphi'\circ g\circ \varphi^{-1})(\operatorname{Int} D) = {(\varphi' \circ g)}(\mathbb{B}^{n}) = \varphi'(g(\mathbb{B}^{n})) \subseteq \varphi'(\mathbb{B}^{m}) = \operatorname{Int} D'.
		      \end{align*}
		\item This follows from part (a) and Note~\ref{note:closed-cell-homeomorphism}.
		\item If $f$ is a homeomorphism, we choose $g$ as given in part (a) then $g, F$ are homeomorphisms.
	\end{enumerate}
\end{proof}

\begin{problem}{5-2}\label{problem:5-2}
Suppose $D$ is a closed $n$-cell, $n\geq 1$.
\begin{enumerate}[label={(\alph*)}]
	\item Given any point $p\in \operatorname{Int} D$, show that there is a continuous function $F: D \to \closedinterval{0, 1}$ such that $F^{-1}(1) = \partial D$ and $F^{-1}(0) = \set{p}$.
	\item Given a continuous function $f: \partial D \to \closedinterval{0, 1}$, show that $f$ extends to a continuous function $F: D \to \closedinterval{0, 1}$ that is strictly positive in $\operatorname{Int} D$.
\end{enumerate}
\end{problem}

\begin{proof}
	\begin{enumerate}[label={(\alph*)}]
		\item From Note~\ref{note:closed-cell-homeomorphism}, there is a homeomorphism $\varphi: D \to \overline{\mathbb{B}}^{n}$ such that $\varphi(p) = 0$, $\varphi(\operatorname{Int}D) = \mathbb{B}^{n}$, $\varphi(\partial D) = \partial\overline{\mathbb{B}}^{n}$. Define $f: \overline{\mathbb{B}}^{n} \to \closedinterval{0, 1}$ by $f(x) = \abs{x}$ then $f$ is continuous. Define $F: D \to \closedinterval{0, 1}$ by $F = f\circ \varphi$ then $F$ is continuous and
		      \begin{align*}
			      F^{-1}(1) & = {(f\circ \varphi)}^{-1}(1) = \varphi^{-1}(f^{-1}(1)) = \varphi^{-1}(\partial\overline{\mathbb{B}}^{n}) = \partial D, \\
			      F^{-1}(0) & = {(f\circ \varphi)}^{-1}(0) = \varphi^{-1}(f^{-1}(0)) = \varphi^{-1}(\set{0}) = \set{p}.
		      \end{align*}
		\item Consider this particular case: $D = \overline{\mathbb{B}}^{n}$. Define $F: \overline{\mathbb{B}}^{n} \to \closedinterval{0, 1}$ as follows:
		      \begin{align*}
			      F(x) = \begin{cases}
				             \abs{x}f\left(\frac{x}{\abs{x}}\right) + (1 - \abs{x}) & x \ne 0 \\
				             1                                                      & x = 0
			             \end{cases}
		      \end{align*}

		      $F$ is continuous on $\overline{\mathbb{B}}^{n}\smallsetminus\set{0}$. Let $\varepsilon > 0$, so for every $x$ such that $0 < \abs{x} < \frac{\varepsilon}{2}$
		      \begin{align*}
			      \abs{F(x) - F(0)} = \abs{\abs{x} f\left(\frac{x}{\abs{x}}\right) - \abs{x}} = \abs{x} \abs{f\left(\frac{x}{\abs{x}}\right) - 1} \leq 2\abs{x} < \varepsilon
		      \end{align*}

		      which implies $F$ is continuous at 0. Hence $F$ is continuous. Moreover, for every $x \in \partial \overline{\mathbb{B}}^{n}$, $F(x) = f\left(\frac{x}{\abs{x}}\right) = f(x)$ and $F(x) > 0$ for every $x \in \operatorname{Int} D = \mathbb{B}^{n}$.

		      Consider the general case: $D$ is a closed $n$-cell. By Note~\ref{note:closed-cell-homeomorphism}, there is a homeomorphism $\varphi: D \to \overline{\mathbb{B}}^{n}$ such that $\varphi(p) = 0$, $\varphi(\operatorname{Int} D) = \mathbb{B}^{n}$ and $\varphi(\partial D) = \partial\overline{\mathbb{B}}^{n}$. From the particular case above, $f\circ (\varphi^{-1}\vert_{\partial\overline{\mathbb{B}}^{n}})$ extends to a continuous map $\psi: \overline{\mathbb{B}}^{n} \to \closedinterval{0, 1}$ that is strictly positive in $\mathbb{B}^{n}$. Define $F: D \to \closedinterval{0, 1}$ by $F = \psi \circ \varphi$. If $x \in \partial D$
		      \begin{align*}
			      F(x) & = \psi(\varphi(x)) = (f\circ (\varphi^{-1}\vert_{\partial\overline{\mathbb{B}}^{n}}) \circ \varphi)(x) \\
			           & = (f \circ (\varphi^{-1}\vert_{\partial\overline{\mathbb{B}}^{n}}) \circ \varphi\vert_{\partial D})(x) \\
			           & = f(x).
		      \end{align*}

		      Moreover, from the particular case, $\psi^{-1}(0) \subseteq \partial\overline{\mathbb{B}}^{n}$, so
		      \begin{align*}
			      F^{-1}(0) = {(\psi \circ \varphi)}^{-1}(0) = \varphi^{-1}(\psi^{-1}(0)) \subseteq \varphi^{-1}(\partial\overline{\mathbb{B}}^{n}) = \partial D
		      \end{align*}

		      which means $F(x) > 0$ if $x \in \operatorname{Int} D$. Thus $f$ extends to the continuous map $F: D \to \closedinterval{0, 1}$ that is strictly positive in $\operatorname{Int} D$.
	\end{enumerate}
\end{proof}

\begin{problem}{5-3}\label{problem:5-3}
Recall that a topological space $X$ is said to be topologically homogeneous
if for every pair of points in $X$ there is a homeomorphism of $X$ taking one
point to the other. This problem shows that every connected manifold is
topologically homogeneous.
\begin{enumerate}[label={(\alph*)}]
	\item Given any two points $p, q \in \mathbb{B}^{n}$, show that there is a homeomorphism $\varphi: \overline{\mathbb{B}}^{n} \to \overline{\mathbb{B}}^{n}$ such that $\varphi(p) = q$ and $\varphi\vert_{\partial\mathbb{B}^{n}} = \operatorname{Id}_{\partial\mathbb{B}^{n}}$.
	\item For any topological manifold $X$, show that every point of $X$ has a neighborhood $U$ with the property that for any $p, q \in U$, there is a homeomorphism from $X$ to itself taking $p$ to $q$.
	\item Show that every connected topological manifold is topologically homogeneous.
\end{enumerate}
\end{problem}

\begin{proof}
	\begin{enumerate}[label={(\alph*)}]
		\item $\overline{\mathbb{B}}^{n}$ is a closed convex subset of $\mathbb{R}^{n}$ with nonempty interior so from the proof of Proposition~\ref{prop:5.1}, there exist homeomorphisms $f, g: \overline{\mathbb{B}}^{n} \to \overline{\mathbb{B}}^{n}$ such that $f(p) = 0, g(q) = 0$ and
		      \begin{align*}
			      f(\mathbb{B}^{n}) = g(\mathbb{B}^{n}) = \mathbb{B}^{n} \\
			      f\vert_{\partial\mathbb{B}^{n}} = g\vert_{\partial\mathbb{B}^{n}} = \operatorname{Id}_{\partial\mathbb{B}^{n}}
		      \end{align*}

		      therefore $\varphi = g^{-1}\circ f: \overline{\mathbb{B}}^{n} \to \overline{\mathbb{B}}^{n}$ is a homeomorphism of $\overline{\mathbb{B}}^{n}$ such that $\varphi\vert_{\partial\overline{\mathbb{B}}^{n}} = \operatorname{Id}_{\partial\overline{\mathbb{B}}^{n}}$ and $\varphi(p) = g^{-1}(f(p)) = g^{-1}(0) = q$.
		\item Consider a point of $X$ and a regular coordinate ball $U$ containing it. Let $f: \overline{U} \to \overline{\mathbb{B}}^{n}$ be a homeomorphism (its existence is ensured by the definition of regular coordinate ball). From part (a), it follows that there is a homeomorphism $\varphi: \overline{\mathbb{B}}^{n} \to \overline{\mathbb{B}}^{n}$ such that $\varphi(f(p)) = f(q)$ and $\varphi\vert_{\partial\mathbb{B}^{n}} = \operatorname{Id}_{\partial\mathbb{B}^{n}}$. Define $g: \overline{U} \to \overline{U}$ by $g = f^{-1} \circ \varphi \circ f$ then $g$ is a topological embedding and
		      \begin{align*}
			      g(p) = f^{-1}(\varphi(f(p))) = f^{-1}(f(q)) = q.
		      \end{align*}

		      $h: X\smallsetminus U \to X\smallsetminus U$ defined by $h(x) = x$ is also a topological embedding. Moreover, $g, h$ agree on $\overline{U} \cap (X \smallsetminus U) = \partial \overline{U}$ so there is a continuous map $F: X \to X$ such that $F\vert_{\overline{U}} = g$ and $F\vert_{X\smallsetminus U} = h$ according to the gluing lemma. $F$ is bijective because $g$ and $h$ are bijective. Also by the gluing lemma, there is a continuous map $G: X\to X$ such that $G\vert_{\overline{U}} = g^{-1}$ and $G\vert_{X\smallsetminus U} = h^{-1}$. Moreover, $F \circ G = G \circ F = \operatorname{Id}_{X}$ so $F$ is a homeomorphism. So $F: X\to X$ is a homeomorphism such that $F(p) = g(p) = q$.

		      Hence every point of $X$ has a neighborhood $U$ such that for any $p, q \in U$, there is a homeomorphism from $X$ to itself taking $p$ to $q$.
		\item Suppose $X$ is a connected topological manifold. Define a relation $\sim$ on $X$ as follows: $x \sim y$ if and only if there is a homeomorphism from $X$ to itself taking $x$ to $y$. $\sim$ is an equivalence relation.

		      Let $U$ be a regular coordinate ball of $X$ then part (b) implies that any two points $x, y\in U$, $x \sim y$. Therefore $U$ is contained in the equivalence class ${[x]}_{\sim}$, so ${[x]}_{\sim}$ is open in $X$. All equivalence classes constitute a partition by open subsets of $X$. Since $X$ is connected, it follows that there is exactly one equivalence class. Hence $X$ is topologically homogeneous.
	\end{enumerate}
\end{proof}

\begin{problem}{5-4}\label{problem:5-4}
Generalize the argument of Problem~\ref{problem:5-3} to show that if $M$ is a connected topological manifold, $\dim M > 1$ and $\tuple{p_{1}, \ldots, p_{k}}$ and $\tuple{q_{1}, \ldots, q_{k}}$ are two ordered $k$-tuples of distinct points in $M$, then there is a homeomorphism $F: M \to M$ such that $F(p_{i}) = q_{i}$ for $i = 1, \ldots, k$.
\end{problem}

\begin{quote}
	A connected 0-manifold consists of a single point, so the statement is vacuously true for connected 0-manifolds.

	In general, the statement is not true when $M$ is a connected 1-manifold. Consider $\mathbb{R}$, which is a connected 1-manifold. Assume $p_{1}, p_{2}, p_{3}\in \mathbb{R}$ such that $p_{1} < p_{2} < p_{3}$. Suppose the statement is true for connected 1-manifolds then there is a homeomorphism $f: \mathbb{R} \to \mathbb{R}$ such that $f(p_{1}) = p_{1}, f(p_{2}) = p_{3}, f(p_{3}) = p_{2}$. From the intermediate value theorem, there is $p \in \openinterval{p_{1}, p_{2}}$ such that $f(p) = p_{2}$. This is a contradiction because $f(p) = f(p_{3}) = p_{2}$, $p < p_{3}$ and $f$ is bijective.
\end{quote}

\begin{proof}
	Let $p, q\in M$ and $A$ a finite subset of $M$ not containing $p, q$. We will show that there exists a self-homeomorphism on $M$ that takes $p$ to $q$ and fixes every point of $A$.

	Define $\sim$ on $M\smallsetminus A$ as follows: $x \sim y$ if and only if there exists a self-homeomorphism on $M$ such that takes $x$ to $y$ and fixes every point of $A$. $\sim$ is an equivalence relation. Consider an equivalence class and a point $x$ in it. There exists a regular coordinate ball $B$ containing $x$ and not intersecting $A$ (because $M\smallsetminus A$ is open in $M$). The proof of Problem~\ref{problem:5-3} shows that there is a self-homeomorphism on $M$ that takes $x$ to any point in $B$ and fixes every point of $M\smallsetminus B$. Therefore $x \in B \subseteq {[x]}_{\sim}$, which means ${[x]}_{\sim}$ is an open set. Because $M\smallsetminus A$ is connected (see the note after Problem~\ref{problem:4-18}) and $M\smallsetminus A$ has a partition of equivalence classes induced by $\sim$, it follows that $M\smallsetminus A$ has exactly one equivalence class. Hence there is a self-homeomorphism on $M$ that takes $p$ to $q$ and fixes every point of $A$.

	The statement is true for $k = 1$. Assume that it is true for $k = m \geq 1$. Consider two $(m+1)$-tuples $\tuple{p_{1}, \ldots, p_{m+1}}$ and $\tuple{q_{1}, \ldots, q_{m+1}}$ in which $p_{1}, \ldots, p_{m+1}$ are pairwise distinct and $q_{1}, \ldots, q_{m+1}$ are pairwise distinct. By the inductive hypothesis, there is a homeomorphism $\varphi: M\to M$ that takes $p_{i}$ to $q_{i}$ for $i = 1, \ldots, m$. Because $\varphi$ is bijective, $\varphi(p_{m+1}) \notin A = \set{ q_{1}, \ldots, q_{m} }$. From the above statement, there is a homeomorphism $\psi: M\to M$ that takes $\varphi(p_{m+1})$ to $q_{m+1}$ and fixes every point of $A$. Hence $\psi\circ\varphi$ is a self-homeomorphism on $M$ that takes $p_{i}$ to $q_{i}$ for $i = 1, \ldots, m+1$. Therefore the statement is true due to the principle of mathematical induction.

	Hence there exists a self-homeomorphism $F$ on $M$ such that $F(p_{i}) = q_{i}$ for each $i = 1, \ldots, k$, in which $\tuple{p_{1}, \ldots, p_{k}}$ and $\tuple{q_{1}, \ldots, q_{k}}$ are $k$-tuples of pairwise distinct points.
\end{proof}

\begin{problem}{5-5}\label{problem:5-5}
Suppose $X$ is a topological space and $\set{X_{\alpha}}$ is a family of subspaces whose union is $X$. Show that the topology of $X$ is coherent with the subspaces $\set{X_{\alpha}}$ if and only if it is the finest topology on $X$ for which all of the inclusion maps $X_{\alpha} \hookrightarrow{} X$ are continuous.
\end{problem}

\begin{proof}
	$(\Longrightarrow)$ The topology of $X$ is coherent with the subspaces $\set{X_{\alpha}}$.

	Let $\mathscr{T}$ be the original topology of $X$ and $\mathscr{T}'$ be a topology of $X$ for which all of the inclusion maps $X_{\alpha} \hookrightarrow{} X$ are continuous. Suppose $U \in \mathscr{T}'$. $\iota_{\alpha}: X_{\alpha} \hookrightarrow{} (X, \mathscr{T}')$ is continuous for each $\alpha$, so $\iota_{\alpha}^{-1}(U) = U \cap X_{\alpha}$ is open in $X_{\alpha}$ for each $\alpha$. Therefore $U \in \mathscr{T}$, due to the definition of coherent topology. Hence $\mathscr{T}' \subseteq \mathscr{T}$.

	$(\Longleftarrow)$ The topology of $X$ is the finest topology on $X$ for which all of the inclusion maps $X_{\alpha} \hookrightarrow{} X$ are continuous.

	Denote the topology of $X$ by $\mathscr{T}$. Suppose on the contrary that there is a subset $U$ of $X$ such that $U\cap X_{\alpha}$ is open in $X_{\alpha}$ for each $\alpha$ and $U$ is not open in $X$. Let $\mathscr{T}'$ be a topology on $X$ that contains $\mathscr{T}$ and $U$ then $\mathscr{T} \subsetneq \mathscr{T}'$ and all of the inclusion maps $X_{\alpha} \hookrightarrow{} (X, \mathscr{T}')$ are continuous, which is a contradiction because $\mathscr{T}$ is the finest topology for which all of the inclusion maps $X_{\alpha} \hookrightarrow{} X$ are continuous.
\end{proof}

\begin{problem}{5-6}\label{problem:5-6}
Suppose $X$ is a topological space. Show that the topology of $X$ is coherent with each of the following collections of subspaces of $X$:
\begin{enumerate}[label={(\alph*)}]
	\item Any open cover of $X$.
	\item Any locally finite closed cover of $X$.
\end{enumerate}
\end{problem}

\begin{proof}
	\begin{enumerate}[label={(\alph*)}]
		\item Let $\mathscr{U}$ be an open cover of $X$ and $V$ a subset of $X$.

		      If $V$ is open in $X$ then $V\cap U$ is open in $U$ for each $U\in\mathscr{U}$. Otherwise, if $V\cap U$ is open in $U$ for each $U\in\mathscr{U}$ then $V\cap U$ is open in $X$ for each $U\in\mathscr{U}$ (Exercise~\ref{exercise:3.6}), so $V = \bigcup_{U\in\mathscr{U}}(V\cap U)$ is open in $X$.

		      Hence the topology of $X$ is coherent with any open cover of $X$.
		\item Let $\mathscr{U}$ be a locally finite closed cover of $X$ and $V$ a subset of $X$.

		      If $V$ is closed in $X$ then $V\cap U$ is closed in $U$ for each $U\in\mathscr{U}$. Conversely, suppose that $V\cap U$ is closed in $U$ for each $U\in\mathscr{U}$. $\mathscr{U}$ is locally finite so the collection $\set{ V\cap U : U\in\mathscr{U} }$ is also locally finite.
		      \begin{align*}
			      V & = \bigcup_{U\in\mathscr{U}} (V\cap U)                                                                                                        \\
			        & = \bigcup_{U\in\mathscr{U}} \overline{V\cap U}   & \text{($V\cap U$ is closed in $U$, thus closed in $X$)}                                   \\
			        & = \overline{\bigcup_{U\in\mathscr{U}} (V\cap U)} & \text{(Lemma 4.76 for the locally finite collection $\set{ V\cap U : U\in\mathscr{U} }$)} \\
			        & = \overline{V}.
		      \end{align*}

		      So $V$ is closed in $X$. Hence the topology of $X$ is coherent with any locally finite closed cover of $X$.
	\end{enumerate}
\end{proof}

\begin{problem}{5-7}\label{problem:5-7}
Here is another generalization of the gluing lemma. (Cf.\@ also Problem~\ref{problem:4-30}.) Suppose $X$ is a topology whose topology is coherent with a collection $\set{X_{\alpha}}_{\alpha\in A}$ (this collection covers $X$) of subspaces of $X$, and for each $\alpha\in A$ we are given a continuous map $f_{\alpha}: X_{\alpha} \to Y$ such that $f_{\alpha}\vert_{X_{\alpha} \cap X_{\beta}} = f_{\beta}\vert_{X_{\alpha} \cap X_{\beta}}$ for all $\alpha$ and $\beta$. Show that there exists a unique continuous map $f: X \to Y$ whose restriction to each $X_{\alpha}$ is $f_{\alpha}$.
\end{problem}

\begin{quote}
	The result follows immediately from Exercise~\ref{exercise:5.3}.
\end{quote}

\begin{proof}
	By elementary set theory, there is a unique map $f: X \to Y$ such that $f\vert_{X_{\alpha}} = f_{\alpha}$. It remains to show that $f$ is continuous.

	Let $U$ be an open subset of $Y$ then for each $\alpha\in A$, the set $f^{-1}(U) \cap X_{\alpha} = f_{\alpha}^{-1}(U)$ is an open subset of $X_{\alpha}$. Because the topology of $X$ is coherent with $\set{X_{\alpha}}_{\alpha\in A}$, $f^{-1}(U)$ is an open subset of $X$. So $f$ is continuous.

	Thus there exists a unique continuous map $f: X\to Y$ such that $f\vert_{X_{\alpha}} = f_{\alpha}$ for each $\alpha\in A$.
\end{proof}

\begin{problem}{5-8}\label{problem:5-8}
Prove Proposition 5.7 (the topology of a CW complex is coherent with its
collection of skeleta).
\end{problem}

\begin{proof}
	Let $(X, \mathscr{E})$ be a CW complex and $S \subseteq X$. For each integer $n$, denote by $X_{n}$ the $n$-skeleton of $(X, \mathscr{E})$. $\set{ X_{n} : n\geq 0 }$ is a cover of $X$. $X_{n}$ is a subcomplex of $X$. From Theorem 5.6, $X_{n}$ is closed in $X$ for each $n$.

	Assume that $S \cap X_{n}$ is closed in $X_{n}$ for each integer $n$. Let $e$ be an arbitrary cell of $X$ then there exists a nonnegative integer $n$ such that $e \subseteq X_{n}$. Since $X_{n}$ is a subcomplex of $X$, it follows that $\overline{e} \subseteq X_{n}$, moreover, $\overline{e}$ is closed in $X_{n}$.
	\begin{align*}
		S \cap \overline{e} \subseteq \overline{e} \subseteq X_{n}
	\end{align*}

	In summary, $S\cap X_{n}$ is closed in $X_{n}$, $\overline{e}$ is closed in $X_{n}$, so $S\cap \overline{e} = (S\cap \overline{e}) \cap X_{n}$ is closed in $X_{n}$. From the definition of subspace topology, $S \cap \overline{e}$ is closed in $\overline{e}$. This ($S\cap \overline{e}$ is closed in $\overline{e}$) is true for any cell $e$ of $X$ so $S$ is closed in $X$, due to the (W) property of CW complexes.

	Thus the topology of a CW complex is coherent with its skeleta.
\end{proof}

\begin{problem}{5-9}\label{problem:5-9}
Show that every CW complex is locally path-connected.
\end{problem}

\begin{problem}{5-10}\label{problem:5-10}
Show that every CW complex is compactly generated.
\end{problem}

\begin{problem}{5-11}\label{problem:5-11}
Prove Proposition 5.16 (a CW complex is locally compact if and only if it is locally finite).
\end{problem}

\begin{problem}{5-12}\label{problem:5-12}
\end{problem}

\begin{problem}{5-13}\label{problem:5-13}
\end{problem}

\begin{problem}{5-14}\label{problem:5-14}
\end{problem}

\begin{problem}{5-15}\label{problem:5-15}
\end{problem}

\begin{problem}{5-16}\label{problem:5-16}
\end{problem}

\begin{problem}{5-17}\label{problem:5-17}
\end{problem}

\begin{problem}{5-18}\label{problem:5-18}
\end{problem}
