\documentclass[class=analysis,crop=false]{standalone}

\begin{document}

\setcounter{exercise}{0}

\chapter{Số thực}

\begin{exercise}\label{chapter1:sup-and-inf}
    Chứng minh rằng:
    \begin{enumerate}[label = (\roman*)]
        \item Nếu $A\subset\mathbb{R}$ bị chặn trên thì $z = \sup A$ khi và chỉ khi
            \[
                \begin{cases}
                    x\le z\ \forall x\in A \\
                    \forall\varepsilon > 0\ \exists x_{\varepsilon}\in A : z - \varepsilon < x_{\varepsilon}.
                \end{cases}
            \]
        \item Nếu $A\subset\mathbb{R}$ bị chặn dưới thì $z = \inf A$ khi và chỉ khi
            \[
                \begin{cases}
                    z\le x\ \forall x\in A \\
                    \forall\varepsilon > 0\ \exists x_{\varepsilon}\in A : x_{\varepsilon} < z + \varepsilon.
                \end{cases}
            \]
        \item Giữa cận trên đúng và cận dưới đúng có mối liên hệ sau
            \[
                \inf A = -\sup(-A),\quad \sup A = -\inf(-A).
            \]
    \end{enumerate}
\end{exercise}

\begin{proof}
    \begin{enumerate}[label = (\roman*)]
        \item\textit{Điều kiện cần}.
             \par $z = \sup A$ thì $z$ là cận trên của $A$, do đó $x\le z\ \forall x\in A$.
             \par $z$ là cận trên đúng của $A$ nên $z - \varepsilon$ không phải cận trên của $A$, tức là trong $A$ tồn tại một phần tử $x_{\varepsilon} > z - \varepsilon$.
             \par\textit{Điều kiện đủ}.
             \par $x\le z\ \forall x\in A$ thì $z$ là một cận trên của $A$.
             \par $\forall\varepsilon > 0\ \exists x_{\varepsilon}\in A : z - \varepsilon < x_{\varepsilon}$ nghĩa là với mọi $\varepsilon > 0$, $z - \varepsilon$ không phải là cận trên của $A$. Tức là mọi số bé hơn $z$ đều không phải cận trên của $A$.
             \par Theo định nghĩa \textbf{cận trên đúng} thì $z = \sup A$.
        \item\textit{Điều kiện cần}.
             \par $z = \inf A$ thì $z$ là cận dưới của $A$, do đó $z\le x\ \forall x\in A$.
             \par $z$ là cận dưới đúng của $A$ nên $z + \varepsilon$ không phải cận dưới của $A$, tức là trong $A$ tồn tại một phần tử $x_{\varepsilon} < z + \varepsilon$.
             \par\textit{Điều kiện đủ}.
             \par $z\le x\ \forall x\in A$ thì $z$ là một cận dưới của $A$.
             \par $\forall\varepsilon > 0\ \exists x_{\varepsilon}\in A : x_{\varepsilon} < z + \varepsilon$ nghĩa là với mọi $\varepsilon > 0$, $z + \varepsilon$ không phải cận dưới của $A$. Tức là mọi số lớn hơn $z$ đều không phải cận dưới của $A$.
             \par Theo định nghĩa \textbf{cận dưới đúng} thì $z = \inf A$.
        \item $z = \inf A$.
             \par Theo (ii)
             \[
                 \begin{cases}
                     z\le x\ \forall x\in A \\
                     \forall\varepsilon > 0\ \exists x_{\varepsilon}\in A : x_{\varepsilon} < z + \varepsilon
                 \end{cases}
             \]
             \par Điều này tương đương với:
             \[
                 \begin{cases}
                     -x\le -z\ \forall -x\in -A \\
                     \forall\varepsilon > 0\ \exists -x_{\varepsilon}\in -A : -z - \varepsilon < -x_{\varepsilon}.
                 \end{cases}
             \]
             \par Theo (i) thì $-z = \sup(-A)$. Do đó $\inf A = -\sup(-A)$.
             \par Áp dụng điều này, ta được $-\inf(-A) = -(-\sup A) = \sup A$.
    \end{enumerate}
\end{proof}

\begin{exercise}
    Thế nào là tập không bị chặn trên, không bị chặn dưới, không bị chặn?
\end{exercise}

\begin{proof}[Lời giải]
    Một tập được gọi là không bị chặn trên nếu không tồn tại $z\in\mathbb{R}$ sao cho $z$ lớn hơn hoặc bằng mọi phần tử của tập đó.
    \par Nói theo cách hình thức, tập $A$ không bị chặn trên nếu $\forall\varepsilon > 0$, tồn tại $x\in A$ sao cho $x > \varepsilon$.
    \bigskip
    \par Một tập được gọi là không bị chặn dưới nếu không tồn tại $z\in\mathbb{R}$ sao cho $z$ nhỏ hơn hoặc bằng mọi phần tử của tập đó.
    \par Nói theo cách hình thức, tập $A$ không bị chặn dưới nếu $\forall\varepsilon > 0$, tồn tại $x\in A$ sao cho $x < -\varepsilon$.
    \bigskip
    \par Một tập được gọi là không bị chặn nếu nó không bị chặn trên và cũng không bị chặn dưới.
\end{proof}

\begin{exercise}
    Hãy làm sáng tỏ những điều sau đây:
    \begin{enumerate}[label = (\roman*)]
        \item Không nhất thiết $\sup A\in A$ hoặc $\inf A\in A$.
        \item Nếu $A$ có phần tử lớn nhất thì $\sup A = \max A$. Nếu $A$ có phần tử bé nhất thì $\inf A = \min A$.
        \item Cận trên đúng, cận dưới đúng nếu tồn tại thì duy nhất.
    \end{enumerate}
\end{exercise}

\begin{proof}
    \begin{enumerate}[label = (\roman*)]
        \item $A = \{ \frac{1}{n}\ |\ n\in\mathbb{N}\setminus\{ 0 \} \}$.
            \par $\forall\varepsilon > 0$, chọn $n = \ceil{\frac{1}{\varepsilon}} + 1$ thì $\frac{1}{n} < \varepsilon$.
            \par Do đó $\inf A = 0$. Tuy nhiên $0\not\in A$.
            \par $B = \{ \frac{-1}{n}\ |\ n\in\mathbb{N}\setminus\{ 0 \} \}$.
            \par $\forall\varepsilon > 0$, chọn $n = \ceil{\frac{1}{\varepsilon}} + 1$ thì $\frac{-1}{n} > -\varepsilon$.
            \par Do đó $\sup B = 0$. Tuy nhiên $0\not\in B$.
            \par Nếu $A$, $B$ được bổ sung phần tử $0$ thì $\inf A\in A$ và $\sup B\in B$.
        \item Nếu $A$ có phần tử lớn nhất thì $\max A$ là một cận trên của $A$. Bên cạnh đó, với mọi $x < \max A$ thì phần tử $\max A > x$, nên $x$ không thể là cận trên của $A$. Do đó, theo định nghĩa cận trên đúng, $\sup A = \max A$.
            \par Nếu $A$ có phần tử nhỏ nhất thì $\min A$ là một cận dưới của $A$. Mà với mọi $x > \min A$ thì phần tử $\min A < x$, nên $x$ không thể là cận dưới của $A$. Do đó, theo định nghĩa cận dưới đúng, $\inf A = \min A$.
        \item Cận trên đúng của $A$ tồn tại. Giả sử $A$ có hai cận trên đúng là $x$ và $y$.
            \par Nếu $x < y$ thì theo định nghĩa cận trên đúng, $x$ không phải cận trên của $A$.
            \par Nếu $y < x$ thì theo định nghĩa cận trên đúng, $y$ không phải cận trên của $A$.
            \par Do đó $x = y$, tức là cận trên đúng của $A$ là duy nhất (nếu tồn tại).
            \bigskip
            \par Cận dưới đúng của $A$ tồn tại. Giả sử $A$ có hai cận dưới đúng là $x$ và $y$.
            \par Nếu $x > y$ thì theo định nghĩa cận dưới đúng, $x$ không phải cận dưới của $A$.
            \par Nếu $y > x$ thì theo định nghĩa cận dưới đúng, $y$ không phải cận dưới của $A$.
            \par Do đó $x = y$, tức là cận dưới đúng của $A$ là duy nhất (nếu tồn tại).
     \end{enumerate}
\end{proof}

\begin{exercise}
    Chứng minh $\sqrt{2}$, $\sqrt{3}$ là các số vô tỷ.
\end{exercise}

\begin{proof}
    \begin{itemize}
        \item Giả sử $\sqrt{2}$ là số hữu tỷ, khi đó tồn tại các số nguyên dương $a$, $b$ sao cho $\gcd(a,b) = 1$ và $\frac{a}{b} = \sqrt{2}$.
            \par $\frac{a}{b} = \sqrt{2}$, suy ra $a^{2} = 2b^{2}$.
            \par $2\mid 2b^{2}$ nên $2\mid a^{2}$, suy ra $2\mid a$.
            \par Đặt $a = 2a_{1}$, trong đó $a_{1}$ là số nguyên dương, ta được $b^{2} = 2a^{2}_{1}$. Điều này dẫn tới $2\mid b^{2}$, tức là $2\mid b$. Do đó $\gcd(a, b)\ge 2$, mâu thuận với giả sử rằng $\gcd(a, b) = 1$.
            \par Vậy $\sqrt{2}$ là số vô tỷ.
        \item Giả sử $\sqrt{3}$ là số hữu tỷ, khi đó tồn tại các số nguyên dương $a$, $b$ sao cho $\gcd(a,b) = 1$ và $\frac{a}{b} = \sqrt{3}$.
            \par $\frac{a}{b} = \sqrt{3}$, suy ra $a^{2} = 3b^{2}$.
            \par $3\mid 3b^{2}$ nên $3\mid a^{2}$, suy ra $3\mid a$.
            \par Đặt $a = 3a_{1}$, trong đó $a_{1}$ là số nguyên dương, ta được $b^{2} = 3a^{2}_{1}$. Điều này dẫn tới $3\mid b^{2}$, tức là $3\mid b$. Do đó $\gcd(a, b)\ge 3$, mâu thuẫn với giả sử rằng $\gcd(a, b) = 1$.
            \par Vậy $\sqrt{3}$ là số vô tỷ.
    \end{itemize}
\end{proof}

\begin{exercise}
    Nếu $n\in\mathbb{N}$ không phải là số chính phương thì $\sqrt{n}$ có phải là số vô tỷ không?
\end{exercise}

\begin{proof}
    $n$ không phải là số chính phương nên trong phân tích nguyên tố của $n$, tồn tại ít nhất một ước nguyên tố có số mũ lẻ. Gọi ước nguyên tố đó và số mũ tương ứng của nó trong phân tích nguyên tố của $n$ là $p$ và $2k - 1$.
    \bigskip
    \par Giả sử $\sqrt{n}$ là số hữu tỷ, khi đó tồn tại các số nguyên dương $a$, $b$ sao cho $\gcd(a,b) = 1$ và $\frac{a}{b} = \sqrt{n}$.
    \par $a^{2} = nb^{2}$ nên $n\mid a^{2}$, tức là $p^{2k-1}\mid a^{2}$, suy ra $p^{k}\mid a$.
    \par Đặt $a = p^{k}a_{1}$, trong đó $a_{1}$ là số nguyên dương, ta được $p^{2k}a^{2}_{1} = nb^{2}$, suy ra $pa^{2}_{1} = \frac{n}{p^{2k-1}}b^{2}$.
    \par Vì $p\nmid \frac{n}{p^{2k-1}}$ nên $p\mid b^{2}$, tức là $p\mid b$. Do đó, $\gcd(a,b)\ge p > 1$, mâu thuẫn với giả thiết $\gcd(a,b) = 1$.
    \par Vậy $\sqrt{n}$ là số vô tỷ.
\end{proof}

\begin{exercise}
    Chứng minh $\mathbb{N}$ là tập không bị chặn trên và $\mathbb{Z}$ không bị chặn dưới.
\end{exercise}

\begin{proof}
    \begin{itemize}
        \item Theo định lý Archimedes, với mọi $\varepsilon > 0$, tồn tại $n\in\mathbb{Z}$ sao cho $\varepsilon < n$. Do đó $n > 0$, $n\in\mathbb{N}$. Do đó, $\mathbb{N}$ không bị chặn trên.
        \item Theo định lý Archimedes, với mọi $\varepsilon > 0$, tồn tại $n\in\mathbb{Z}$ sao cho $\varepsilon < n$. Do đó $-n < -\varepsilon$. Do đó, $\mathbb{Z}$ không bị chặn dưới.
    \end{itemize}
\end{proof}

\begin{exercise}
    Chứng minh rằng với mọi $x > 1$, $y > 0$ tồn tại $n\in\mathbb{Z}$ sao cho $x^{n-1}\le y < x^{n}$.
\end{exercise}

\begin{proof}
    Theo định lý Archimedes, tồn tại $n\in\mathbb{Z}$ sao cho $(n-1)\le\log_{x}y\le n$.
    \par Vì $x > 1$ nên $x^{n-1} \le x^{\log_{x}y} \le x^{n}$, tức là $x^{n-1}\le y\le x^{n}$.
\end{proof}

\begin{exercise}
    Chứng minh rằng với mọi $y > 0$ ta có $\inf\{ y/n\mid n\in\mathbb{N} \} = 0$.
\end{exercise}

\begin{proof}
    $A = \{ \frac{y}{n}\mid n\in\mathbb{N} \}$.
    \par $\frac{y}{n} > 0,\ \forall n\in\mathbb{N}$ nên $0$ là một cận dưới của $A$.
    \par $\forall\varepsilon > 0$, $n_{\varepsilon} = \ceil{\frac{y}{\varepsilon}} + 1$ thì $\frac{y}{n_{\varepsilon}} < \varepsilon$.
    \par Do đó, theo Bài~\ref{chapter1:sup-and-inf}, $0$ là cận dưới đúng của $A$.
\end{proof}

\begin{exercise}
    Đặt $N_{x} = \{ s\le x\mid s\in\mathbb{Q} \}$, $M_{x} = \{ r\ge x\mid r\in\mathbb{Q} \}$. Chứng minh rằng $\sup N_{x} = x = \inf M_{x}$.
\end{exercise}

\begin{proof}
    Có hai trường hợp cần xem xét.
    \begin{enumerate}[label = Trường hợp \arabic*.,itemindent=1.5cm]
        \item $x$ là số hữu tỷ.
            \par $x$ là số hữu tỷ thì $x\in N_{x}$ và $x = \max N_{x}$, $x\in M_{x}$ và $x = \min M_{x}$.
            \par Mà $\sup N_{x} = \max N_{x}$, $\inf M_{x} = \min M_{x}$ nên $\sup N_{x} = x = \inf M_{x}$.
        \item $x$ là số vô tỷ.
            \par $x$ là số vô tỷ thì $x\notin N_{x}$ và $x\notin M_{x}$.
            \par Theo định nghĩa của $N_{x}$, $M_{x}$ thì $x$ là cận trên của $N_{x}$ và $x$ là cận dưới của $M_{x}$.
            \par Chọn $y < x$. Vì giữa hai số thực phân biệt, luôn có số hữu tỷ (số này là phần tử của $N_{x}$) nên $y$ không thể là cận trên của $N_{x}$. Nói cách khác, mọi số nhỏ hơn $x$ đều không phải cận trên của $N_{x}$. Mà $x$ là cận trên của $N_{x}$ nên theo định nghĩa, $x = \sup N_{x}$.
            \par Chọn $z > x$. Vì giữa hai số thực phân biệt, luôn có số hữu tỷ (số này là phần tử của $M_{x}$) nên $z$ không thể là cận dưới của $M_{x}$. Nói cách khác, mọi số lớn hơn $x$ đều không phải cận dưới của $M_{x}$. Mà $x$ là cận dưới của $M_{x}$ nên theo định nghĩa, $x = \inf M_{x}$.
    \end{enumerate}
    \par Tóm lại, $\sup N_{x} = x = \inf M_{x}$.
\end{proof}

\par\textbf{Nguyên lý Cantor.} Nếu $\{ A_{i}, i\in I \}$ là họ các đoạn lồng nhau thì
\[
    \bigcap_{i\in I}A_{i} \ne \varnothing
\]
\par tức là, có ít nhất một số thực thuộc tất cả các $A_{i}$, $i\in I$.

\begin{exercise}
    Hãy chỉ ra trong chứng minh của \textbf{Nguyên lý Cantor}, ta có $[a,b] = \bigcap\limits_{i\in I}A_{i}$ và nếu bỏ điều kiện $A_{i}$ là khoảng đóng thì kết luận của định lý trên sẽ dẫn đến sai.
\end{exercise}

\begin{proof}
    $A_{i} = [a_{i}, b_{i}]$, $i\in I$.
    \par Đặt $A = \{ a_{i}, i\in I \}$, $B = \{ b_{i}, i\in I \}$.
    \par $a = \sup A$, $b = \inf B$.
    \par $a_{i}$ là cận dưới của $B$, $\forall i\in I$.
    \par $b_{i}$ là cận trên của $A$, $\forall i\in I$.
    \par Nếu $a$ không phải cận dưới của $B$ thì tồn tại $b_{j}$ sao cho $b_{j} < a$. Mà $b_{j}$ là cận trên của $A$, điều này mâu thuẫn với định nghĩa cận trên đúng. Do đó, $a$ là cận dưới của $B$.
    \par $a$ là cận dưới của $B$, $b$ là cận dưới đúng của $B$ nên $a\le b$.
    \par $x\in \bigcap\limits_{i\in I}A_{i}$ khi và chỉ khi $x$ là cận trên của $A$ và là cận dưới của $B$. Điều này tương đương với $a = \sup A\le x\inf B = b$.
    \par Do đó $[a,b] = \bigcap\limits_{i\in I}A_{i}$.
    \bigskip
    \par Nếu $A_{i}$ là các khoảng mở, chẳng hạn xét họ $X_{n} = (0,\frac{1}{n})$, $n\in\mathbb{N}$.
    \par Giao của tất cả các tập thuộc họ này là rỗng.
    \par Nếu như giao của tất cả các tập này không rỗng, ta chọn lấy một phần tử $x$ thuộc phần giao đó. Do định nghĩa của $X_{n}$ nên $x > 0$. Tuy nhiên $\inf\{ \frac{1}{n} \mid n\in\mathbb{N} \} = 0$ nên với $x > 0$, tồn tại $n_{\varepsilon}$ sao cho $\frac{1}{n_{\varepsilon}} < x$, tức là $x\notin(0, \frac{1}{n_{\varepsilon}})$, mâu thuẫn với định nghĩa của $x$.
    \par Do đó, giao của $X_{n}$, $\forall n\in\mathbb{N}$ là tập rỗng.
    \par Như vậy, nếu bỏ đi điều kiện khoảng đóng thì nguyên lý Cantor không còn đúng.
\end{proof}

\begin{exercise}
    Từ chứng minh của \textbf{Nguyên lý Cantor}, chứng minh rằng $a = b$ khi và chỉ khi $\inf\{ b_{i} - a_{i}\mid i\in I \} = 0$.
\end{exercise}

\begin{proof}
    Kí hiệu $A = \{ a_{i} \mid i\in I \}$, $B = \{ b_{i} \mid i\in I \}$.
    \par Theo Bài~\ref{chapter1:sup-and-inf}, $\inf(-A) = -\sup A = -a$.
    \par $b_{i} - a_{i} = b_{i} + (-a_{i}) \ge b - a$ nên $b - a$ là một cận dưới của $\{ b_{i} - a_{i}\mid i\in I \}$.
    \par Chọn $c = b - a' > b - a$.
    \par $b - a' > b - a$ nên $a' < a = \sup A$. Theo định nghĩa cận trên đúng, tồn tại $a_{j}$ sao cho $a_{j} > a'$, tức là $-a_{j} < -a'$. Suy ra $b_{j} - a_{j} < b - a'$. Do đó, $c = b - a'$ không phải cận dưới của $\{ b_{i} - a_{i} \mid i\in I \}$.
    \par Như vậy, mọi số lớn hơn $b - a$ không phải cận dưới của $\{ b_{i} - a_{i}\mid i\in I \}$. Mà $b - a$ là cận dưới của $\{ b_{i} - a_{i} \mid i\in I \}$ nên theo định nghĩa cận dưới đúng, $b - a = \inf\{ b_{i} - a_{i} \mid i\in I \}$.
    \par Đẳng thức vừa chứng minh cho thấy $a = b$ khi và chỉ khi $\inf\{ b_{i} - a_{i} \mid i\in I \} = 0$.
\end{proof}

\begin{exercise}
    Chứng minh rằng tập $A$ không bị chặn trên khi và chỉ khi với mọi $n_{0}\in\mathbb{N}$, tồn tại phần tử $x_{n}$ trong $A$ sao cho $x_{n} > n_{0}$. Tương tự, tập $A$ không bị chặn dưới khi và chỉ khi với mọi $n_{0}\in\mathbb{N}$, tồn tại phần tử $x_{n}$ trong $A$ sao cho $x_{n} < -n$.
\end{exercise}

\begin{proof}
    \begin{enumerate}[label = (\roman*)]
        \item \textit{Phần thuận.} $A$ không bị chặn trên thì $\forall\varepsilon > 0$, tồn tại $x_{\varepsilon}\in A$ sao cho $x_{\varepsilon} > \varepsilon$. Do đó, với mọi $n_{0}\in\mathbb{N}$, tồn tại $x_{n}\in A$ sao cho $x_{n} > n_{0}$.
            \par \textit{Phần đảo.} Với mọi $n_{0}\in\mathbb{N}$, tồn tại $x_{n}\in A$ sao cho $x_{n} > n_{0}$.
            \par $\forall\varepsilon > 0$, chọn $n_{0} = \ceil{\varepsilon}$ thì tồn tại $x_{n}\in A$ sao cho $x_{n} > n_{0} \ge \varepsilon$. Do đó, $A$ không bị chặn trên.
        \item \textit{Phần thuận.} $A$ không bị chặn dưới thì $\forall\varepsilon > 0$, tồn tại $x_{\varepsilon}\in A$ sao cho $x_{\varepsilon} < -\varepsilon$. Do đó, với mọi $n_{0}\in\mathbb{N}$, tồn tại $x_{n}\in A$ sao cho $x_{} < -n_{0}$.
            \par \textit{Phần đảo.} Với mọi $n_{0}\in\mathbb{N}$, tồn tại $x_{n}\in A$ sao cho $x_{n} < -n_{0}$.
            \par $\forall\varepsilon > 0$, chọn $n_{0} = \floor{\varepsilon}$, tồn tại $x_{\varepsilon}\in A$ sao cho $x_{\varepsilon} < -n_{0} \le -\varepsilon$. Do đó, $A$ không bị chặn dưới.
    \end{enumerate}
\end{proof}

\begin{exercise}
    Thế nào là dãy không phải dãy Cauchy?
\end{exercise}

\begin{proof}[Lời giải]
    \par Dãy $(x_{n})$ không phải dãy Cauchy nếu:
    \par $\exists\varepsilon > 0$, $\forall n_{0}$, $\exists m, n > n_{0}$ sao cho $\vert a_{m} - a_{n}\vert \ge \varepsilon$.
\end{proof}

\begin{exercise}
    Chứng minh rằng các dãy sau không phải dãy Cauchy:
    \[
        a_{n} = (-1){}^{n-1},\quad b_{n} = 1 + \frac{1}{2} + \cdots + \frac{1}{n}.
    \]
\end{exercise}

\begin{proof}
    \par Chọn $\varepsilon = 1$ và $n_{0}$ bất kì. Chọn $n > n_{0}$ và $m = n + 1$ thì $\vert a_{m} - a_{n}\vert = \vert a_{n+1} - a_{n}\vert = 2 > 1 = \varepsilon$. Do đó, dãy $(a_{n})$ không phải dãy Cauchy.
    \bigskip
    \par Chọn $\varepsilon = \frac{1}{2}$ và $n_{0}$ bất kì. Chọn $n > n_{0}$ và $m = 2n$.
    \[
        \vert b_{2n} - b_{n}\vert = \frac{1}{n+1} + \frac{1}{n+2} + \cdots + \frac{1}{2n} > \frac{n}{2n} = \frac{1}{2} = \varepsilon
    \]
    \par Do đó $(b_{n})$ không phải dãy Cauchy.
\end{proof}

\begin{exercise}\label{chapter1:limit-and-operators}
    Giả sử $(x_{n})$ và $(y_{n})$ là các dãy số thực và
    \[
        \lim\limits_{n\to\infty} x_{n} = x,\quad\lim\limits_{n\to\infty} y_{n} = y.
    \]
    \par Chứng minh
    \begin{enumerate}[label = (\roman*)]
        \item $\lim\limits_{n\to\infty}(x_{n} + y_{n}) = x + y$;
        \item $\lim\limits_{n\to\infty} ax_{n} = ax$, với mọi $a\in\mathbb{R}$;
        \item $\lim\limits_{n\to\infty} x_{n}y_{n} = xy$;
        \item $\lim\limits_{n\to\infty} \dfrac{x_{n}}{y_{n}} = \dfrac{x}{y}$, với điều kiện $y\ne 0$;
        \item Nếu $x_{n}\le y_{n}$ đối với $n\ge N$ nào đó thì $x\le y$.
        \item Nếu $x > 0$ thì $x_{n} > 0$ đối với $n\ge N$ nào đó.
    \end{enumerate}
\end{exercise}

\begin{proof}
    \begin{enumerate}[label = (\roman*)]
        \item Theo định nghĩa, $\forall\varepsilon > 0$, $\exists n_{x}, n_{y}$ sao cho $\forall n > n_{x}$ thì $\abs{x_{n} - x} < \dfrac{\varepsilon}{2}$, $\forall n > n_{y}$ thì $\abs{y_{n} - y} < \dfrac{\varepsilon}{2}$.
            \par Do đó, nếu $n > \max\{ n_{x}, n_{y} \}$ thì $\abs{x_{n} - x} < \dfrac{\varepsilon}{2}$, $\abs{y_{n} - y} < \dfrac{\varepsilon}{2}$.
            \par Do đó, $\forall\varepsilon > 0$, $\exists n_{0}$ sao cho $\forall n > n_{0}$ thì
            \[
                \abs{x_{n} + y_{n} - x - y} \le \abs{x_{n} - x} + \abs{y_{n} - y} < \frac{\varepsilon}{2} + \frac{\varepsilon}{2} = \varepsilon
            \]
            \par Theo định nghĩa, $\lim\limits_{n\to+\infty}(x_{n} + y_{n}) = x + y$.
        \item Nếu $a = 0$, mệnh đề hiển nhiên đúng vì $(0x_{n})$ là dãy hằng.
            \par Xét trường hợp $a\ne 0$.
            \par Theo định nghĩa, $\forall\epsilon > 0$, $\exists n_{0}$ sao cho $\forall n > n_{0}$ thì $\abs{x_{n} - x} < \dfrac{\varepsilon}{\abs{a}}$.
            \par Suy ra, $\forall\epsilon > 0$, $\exists n_{0}$ sao cho $\forall n > n_{0}$ thì $\abs{ax_{n} - ax} < \varepsilon$.
        \item Theo định nghĩa, $\forall\varepsilon' > 0$, $\exists n_{x}, n_{y}$ sao cho $\forall n > n_{x}$ thì $\abs{x_{n} - x} < \varepsilon'$, $\forall n > n_{y}$ thì $\abs{y_{n} - y} < \varepsilon'$.
            \par $n_{0} = \max\{ n_{x}, n_{y} \}$ thì $\forall n > n_{0}$, $\abs{x_{n} - x} < \varepsilon'$, $\abs{y_{n} - y} < \varepsilon'$.
            \par $\forall\varepsilon' > 0$, $\exists n_{0}$ sao cho $\forall n > n_{0}$:
            \[
                \abs{x_{n}y_{n} - xy} = \abs{(x_{n} - x)(y_{n} - y)} + \abs{y(x_{n} - x)} + \abs{x(y_{n} - y)} < \varepsilon'^{2} + \varepsilon'(\abs{x} + \abs{y})
            \]
            \par $\forall\epsilon > 0$, chọn $\varepsilon' < -\dfrac{\vert x\vert + \vert y\vert}{2} + \sqrt{\varepsilon + \left(\dfrac{\vert x\vert + \vert y\vert}{2}\right)^{2}}$ thì $\varepsilon'^{2} + \varepsilon'(|x| + |y|) < \varepsilon$.
            \par Do đó, $\forall\varepsilon > 0$, $\exists n_{0}$ sao cho $\forall n > n_{0}$ thì $\abs{x_{n}y_{n} - xy} < \varepsilon$.
        \item $\lim\limits_{n\to+\infty}y_{n} = y$ nên $\forall\varepsilon > 0$, $\exists n_{0}(\varepsilon)$, $\forall n > n_{0}$, $\abs{y_{n} - y} < \varepsilon$.
            \par Chọn $\varepsilon = \dfrac{\abs{y}}{2}$. Tồn tại $n_{1}$ sao cho $\forall n > n_{1}$, $\abs{y_{n} - y} < \dfrac{\abs{y}}{2}$.
            \[
                \Rightarrow \abs{\abs{y_{n}} - \abs{y}} \le \abs{y_{n} - y} < \frac{\abs{y}}{2} \Rightarrow \abs{y_{n}} > \frac{\abs{y}}{2}
            \]
            \par $\forall\varepsilon > 0$, $\dfrac{\abs{y}^{2}\varepsilon}{2} > 0$, $\exists n_{2}$ sao cho $\forall n > n_{2}$ thì $\abs{y_{n} - y} < \dfrac{\abs{y}^{2}\varepsilon}{2}$.
            \par Do đó, nếu $n > \max\{n_{1}, n_{2}\}$ thì $\abs{y_{n} - y} < \dfrac{\abs{y}^{2}\varepsilon}{2}$ và $\abs{y_{n}} > \dfrac{\abs{y}}{2}$.
            \par $\forall\varepsilon > 0$, $\exists n_{3}$ sao cho $\forall n > n_{3}$:
            \[
                \abs{\frac{1}{y_{n}} - \frac{1}{y}} = \frac{\abs{y_{n} - y}}{\abs{y_{n}}\cdot\abs{y}} < \frac{\abs{y}^{2}\varepsilon}{2}\cdot\frac{2}{\abs{y}\cdot\abs{y}} = \varepsilon
            \]
            \par Theo định nghĩa, $\lim\limits_{n\to+\infty}\dfrac{1}{y_{n}} = \dfrac{1}{y}$.
            \par Theo (iii), $\lim\limits_{n\to+\infty}\dfrac{x_{n}}{y_{n}} = \dfrac{x}{y}$.
        \item Theo định nghĩa:
            \[
                \forall\varepsilon > 0, \exists n_{x}, n_{y}: \forall n > n_{x}, \abs{x_{n} - x} < \varepsilon; \forall n > n_{y}, \abs{y_{n} - y} < \varepsilon
            \]
            \par Do đó, $\forall n > \max\{ N, n_{x}, n_{y} \}$ thì:
            \[
                \begin{cases}
                    x_{n} - y_{n} \le 0 \\
                    -\varepsilon < x - x_{n} < \varepsilon \\
                    -\varepsilon < y_{n} - y < \varepsilon
                \end{cases}
                \Rightarrow
                x - y < 2\varepsilon
            \]
            \par Do đó, $\forall\varepsilon > 0$, $x - y < 2\varepsilon$.
            \par Nếu $x > y$ thì ta chọn được $\varepsilon$ sao cho $x - y \ge 2\varepsilon$, dẫn đến giả sử phản chứng là sai.
            \par Vậy $x\le y$.
        \item Chọn $0 < \varepsilon < x$. Theo định nghĩa, tồn tại $N$ sao cho $\forall n \ge N$, $\abs{x_{n} - x} < \varepsilon$.
            \[
                -\varepsilon < x_{n} - x < \varepsilon, \forall n\ge N
            \]
            \[
                \Rightarrow x_{n} > x - \varepsilon > 0, \forall n\ge N
            \]
            \par Đó là điều phải chứng minh.
    \end{enumerate}
\end{proof}

\begin{exercise}
    Cho ba dãy số $(x_{n})$, $(z_{n})$, $(y_{n})$ sao cho $x_{n} \le y_{n} \le z_{n}$, $\forall n\in\mathbb{N}$ và $\lim\limits_{n\to\infty} x_{n} = \lim\limits_{n\to\infty} y_{n} = a$. Chứng minh $\lim\limits_{n\to\infty} z_{n} = a$.
\end{exercise}

\begin{proof}
    \par Theo định nghĩa
    \[
        \forall\varepsilon > 0, \exists n_{x}, n_{y}: \forall n > n_{x}, \abs{x_{n} - a} < \varepsilon; \forall n > n_{y}, \abs{y_{n} - a} < \varepsilon
    \]
    \par Do đó:
    \[
        \begin{cases}
            \forall n > n_{x}, -\varepsilon < x_{n} - a < \varepsilon \\
            \forall n > n_{y}, -\varepsilon < y_{n} - a < \varepsilon
        \end{cases}
    \]
    \par Suy ra, $\forall n > \max\{ n_{x}, n_{y} \}$ thì:
    \[
        -\varepsilon < x_{n} - x \le z_{n} - a \le y_{n} - a < \varepsilon \Longrightarrow \abs{z_{n} - a} < \varepsilon
    \]
    \par Vậy, $\forall\varepsilon > 0$, $\exists n_{0}$: $\forall n > n_{0}$, $\abs{z_{n} - a} < \varepsilon$. Theo định nghĩa, $\lim\limits_{n\to+\infty}z_{n} = a$.
\end{proof}

\begin{exercise}
    Chứng minh rằng nếu $\lim\limits_{n\to\infty} x_{n} = 0$ và $(y_{n})$ bị chặn thì
    \[
        \lim\limits_{n\to\infty} x_{n}y_{n} = 0.
    \]
\end{exercise}

\begin{proof}
    \par $(y_{n})$ bị chặn, do đó tồn tại $M > 0$ sao cho $\abs{y_{n}} < M$, $\forall n\in\mathbb{N}$.
    \par $(x_{n})$ hội tụ đến $x$. $\dfrac{\varepsilon}{M} > 0$.
    \par Theo định nghĩa, $\exists n_{0}$: $\forall n > n_{0}$, $\abs{x_{n}} < \dfrac{\varepsilon}{M}$.
    \par Suy ra $\forall\varepsilon > 0$, $\exists n_{0}$: $\forall n > n_{0}$, $\abs{x_{n}y_{n}} < M\cdot\dfrac{\varepsilon}{M} = \varepsilon$.
    \par Vậy, theo định nghĩa, $\lim\limits_{n\to+\infty} x_{n}y_{n} = 0$.
\end{proof}

\begin{exercise}
    Chứng minh rằng dãy
    \[
        a_{n} = \left(1 + \frac{1}{n}\right)^{n+1}
    \]
    \par là dãy đơn điệu giảm và bị chặn dưới. Từ đó suy ra rằng dãy
    \[
        c_{n} = 1 + \frac{1}{2} + \cdots + \frac{1}{n} - \ln n
    \]
    \par có giới hạn hữu hạn.
\end{exercise}

\begin{proof}
    \[
        a_{n} = \left(1 + \frac{1}{n}\right)^{n+1} = \left(1 + \frac{1}{n}\right)\sum^{n}_{k=0}\binom{n}{k}\frac{1}{n^{k}} = \left(1 + \frac{1}{n}\right)\left(2 + \sum^{n}_{k=2}\binom{n}{k}\frac{1}{n^{k}}\right) \ge 2
    \]
    \par Sử dụng bất đẳng thức $a^{n} - b^{n} > n(a-b)b^{n-1}$ với điều kiện $a > b > 0$.
    \begin{align*}
        a_{n} - a_{n+1} & = \left(1 + \frac{1}{n}\right)^{n+1} - \left(1 + \frac{1}{n+1}\right)^{n+2} \\
                        & = \left(1 + \frac{1}{n}\right)^{n+1} - \left(1 + \frac{1}{n+1}\right)^{n+1} + \left(1 + \frac{1}{n+1}\right)^{n+1} - \left(1 + \frac{1}{n+1}\right)^{n+2} \\
                        & > \frac{n+2}{n(n+1)}\left(1 + \frac{1}{n+1}\right)^{n} - \frac{1}{n+1}\left(1 + \frac{1}{n+1}\right)^{n+1} \\
                        & = \frac{1}{n}\left(1 + \frac{1}{n+1}\right)^{n+1} - \frac{1}{n+1}\left(1 + \frac{1}{n+1}\right)^{n+1} \\
                        & = \frac{1}{n(n+1)}\left(1 + \frac{1}{n+1}\right)^{n+1} > \frac{1}{n(n+1)} > 0
    \end{align*}
    \par Do đó, $(a_{n})$ là dãy giảm và bị chặn dưới, $\inf a_{n} = \lim\limits_{n\to+\infty}a_{n} = e$.
    \par Dãy $b_{n} = \left(1 + \dfrac{1}{n}\right)^{n}$ tăng và bị chặn trên, có giới hạn là $e$ nên $b_{n} \le e$, suy ra $\ln\left(1 + \dfrac{1}{n}\right) < \dfrac{1}{n}$.
    \par $a_{n} \ge e$ nên $\ln a_{n} \ge 1$, tức là $\ln\left(1 + \dfrac{1}{n}\right) > \dfrac{1}{n+1}$.
    \par Mà $c_{n} - c_{n+1} = \ln\left(1 + \dfrac{1}{n}\right) - \dfrac{1}{n+1}$ nên $c_{n} > c_{n+1}$, $\forall n\in\mathbb{N}$. Suy ra $c_{n}$ là dãy giảm.
    \bigskip
    \par Từ các bất đẳng thức trên:
    \[
        0 < c_{n} - c_{n+1} = \ln\left(1 + \frac{1}{n}\right) - \frac{1}{n+1} < \frac{1}{n} - \frac{1}{n+1} = \frac{1}{n(n+1)} < a_{n} - a_{n+1}, \forall n\in\mathbb{N}
    \]
    \[
        \Longrightarrow 0 < c_{n} - c_{n+p} < a_{n} - a_{n+p}, \forall n, p\in\mathbb{N}
    \]
    \par $(a_{n})$ có giới hạn hữu hạn nên, theo định nghĩa:
    \[
        \forall\varepsilon > 0, \exists N_{\varepsilon}: \forall n\ge N_{\varepsilon},\forall p, \abs{a_{n} - a_{n+p}} < \varepsilon
    \]
    \[
        \Rightarrow\varepsilon > 0, \exists N_{\varepsilon}: \forall n\ge N_{\varepsilon},\forall p, \abs{c_{n} - c_{n+p}} < \abs{a_{n} - a_{n+p}} < \varepsilon
    \]
    \par Vậy, $(c_{n})$ có giới hạn hữu hạn.
\end{proof}

\begin{exercise}
    Chứng minh rằng
    \[
        e - s_{n} = \sum^{+\infty}_{k=1}\frac{1}{(n+k)!} < \frac{1}{n!n}
    \]
    \par Từ đó suy ra $e$ là số vô tỷ.
\end{exercise}

\begin{proof}
    \[
        s_{n} = \sum^{n}_{k=0}\frac{1}{k!}\qquad e = \sum^{+\infty}_{k=0}\frac{1}{k!}
    \]
    \[
        \Rightarrow 0 < e - s_{n} = \sum^{+\infty}_{k=1}\frac{1}{(n+k)!}  < \frac{1}{n!}\sum^{+\infty}_{k=1}\frac{1}{(n+1)^{k}} = \frac{1}{n!}\left(\frac{1}{1 - \dfrac{1}{n+1}} - 1\right) = \frac{1}{n!n}, \forall n\in\mathbb{N}
    \]
    \par Giả sử $e$ là số hữu tỷ. Đặt $e = \dfrac{p}{q}$, trong đó, $p, q\in\mathbb{N}$.
    \par Hiển nhiên là $q!e$, $q!s_{q}$ là các số tự nhiên, do đó, $q!e - q!s_{q}$ cũng là số tự nhiên.
    \par Nhưng $0 < q!e - q!s_{q} < \dfrac{q!}{q!q} = \dfrac{1}{q} < 1$, điều này mâu thuẫn với việc $q!e - q!s_{q}$ là số tự nhiên.
    \par Do đó, $e$ là số vô tỷ.
\end{proof}

\end{document}
