\chapter{The Real Number Systems}

\section{Axioms of the real numbers}

Real numbers are defined axiomatically. Let's denote the set of all real numbers by $\mathbb{R}$.
\begin{enumerate}[label={(\roman*)}]
    \item $\mathbb{R}$ is totally ordered (there exists an order relation on $\mathbb{R}$ where any two elements can be compared).
    \item $\mathbb{R}$ is a field under addition and multiplication.
    \item Order relation in $\mathbb{R}$ is compatible to addition and multiplication (with non-negative real number).
    \item Every non-empty set of $\mathbb{R}$ which is bounded above has a least upper bound.
\end{enumerate}

To those who ask ``Are real numbers real?\@'', we can establish a model (a mathematical structure) that satisfies every axiom above. Since 19th century, mathematicians have given several constructions of the real numbers. IMHO, the two most notable and commonlys used constructions are \textit{Dedekind cuts} and \textit{Cauchy sequences}. In the following sections, we will reproduce the two constructions.

\section{Construction of the real numbers by Dedekind cuts of $\mathbb{Q}$}

We will give the definition of Dedekind cuts from which we construct a model that satisfies the axioms of the real numbers.

\subsection{Definition of Dedekind cuts of $\mathbb{Q}$}

\begin{definition}[Dedekind cuts]
    A Dedekind cut $A$ of $\mathbb{Q}$ is a subset of $\mathbb{Q}$ such that
    \begin{enumerate}[label = (DC\arabic*),itemindent=0.3cm]
        \item $A\ne\varnothing$; in other words, $A$ is not empty.
        \item $A\neq\mathbb{Q}$; in other words, $A$ is not the entire set of rational numbers.
        \item $\forall x\left(x\in A \rightarrow \exists y \left( y\in A \wedge x < y \right)\right)$; in other words, $A$ has no greatest element.
        \item $\forall x\in A\left(\forall y( y < x \rightarrow y\in A)\right)$; in other words, $A$ is closed downward.
    \end{enumerate}
\end{definition}

An example of Dedekind cut is
\[
    A = \{ x : x\in\mathbb{Q}\land x < 0 \}.
\]

Let's verify this.
\begin{enumerate}[label={(\roman*)}]
    \item $A$ contains $-1$, hence $A$ is not empty.
    \item $A$ does not contain $0$, hence $A$ is not $\mathbb{Q}$.
    \item Let $q$ be an arbitrary element of $A$. Since $q < 0$, then there exists positive integers $a$ and $b$ such that $q = -\frac{a}{b}$.
          \[
              q = -\frac{a}{b} < -\frac{a}{b + 1} < 0
          \]

          The rational number $-\dfrac{a}{b + 1}$ is in $A$. So if we choose any element of $A$, there always will be another element which is greater than the chosen element.

          Hence $A$ has no greatest element.
    \item Let $q$ be an arbitrary element of $A$. Let $q_{0}$ be a rational number such that $q_{0} < q$. Since ``$<$'' is transitive, then $q_{0} < 0$. Therefore $q_{0}$ is in $A$.

          Hence $A$ is closed downward.
\end{enumerate}

It follows from the definition of Dedekind cuts that $\mathbb{Q}\setminus A$ contains all rational upper bounds of $A$.

We will use Dedekind cuts of $\mathbb{Q}$ to ``cut'' the rational number line.

Sequentially, we will define the following based on Dedekind cuts of $\mathbb{Q}$
\begin{itemize}[itemsep=0pt]
    \item Order relation,
    \item Rational and irrational cut,
    \item Addition,
    \item Subtraction and negation,
    \item Multiplication and division,
\end{itemize}

then we will show that the set of all Dedekind cuts equipped with those operations/order relation will satisfy the axioms of the real numbers.

Proofs in this section will use various properties of rational numbers, including
\begin{itemize}[itemsep=0pt]
    \item Total ordering.
    \item The field structure.
    \item Compatibility of addition and multiplication with the order relation.
    \item There exists a rational number strictly between any two distinct rational numbers.
          \[
              \forall q_{1}\forall q_{2}\left( q_{1}\in\mathbb{Q}\land q_{2}\in\mathbb{Q}\land q_{1} < q_{2} \rightarrow \exists q (q\in\mathbb{Q}\land q_{1} < q\land q < q_{2}) \right).
          \]
\end{itemize}

Denote the set of all Dedekind cuts of $\mathbb{Q}$ by $\mathscr{D}_{\mathbb{Q}}$.

\subsection{Total order}

\begin{theorem}
    $\mathscr{D}_{\mathbb{Q}}$ is totally ordered with $\subseteq$ relation.
\end{theorem}

\begin{proof}
    Let $A$ and $B$ be two Dedekind cuts of $\mathbb{Q}$.

    The subset relation $\subseteq$ is reflexive, transitive, and anti-symmetric.
    \bigskip

    If $A = B$, then there is nothing else to prove.

    Otherwise, $A\ne B$. Let's consider the following cases.

    \begin{enumerate}[label={\textbf{Case \arabic*.}},itemindent=0.5cm]
        \item There exists $b\in B$ such that $b\notin A$.

              $b\notin A$, then $b$ is an upper bound of $A$.

              Let $a$ be an arbitrary element of $A$, then $a < b$. According to (DC4), $a\in B$. Hence $\forall a(a\in A\rightarrow a\in B)$.

              Therefore, $A$ is a proper subset of $B$.
        \item There exists $a\in A$ such that $a\notin B$.

              $a\notin B$, then $a$ is an upper bound of $B$.

              Let $b$ be an arbitrary element of $B$, then $b < a$. According to (DC4), $b\in A$. Hence $\forall b(b\in B\rightarrow b\in A)$.

              Therefore, $B$ is a proper subset of $A$.
        \item There exists $a\in A$ such that $a\not B$ and there exists $b\in B$ such that $b\notin A$.

              According to the two previous cases, $A\subset B$ and $B\subset A$. Thereby $A\subset A$, which is false.

              Hence, this case never occurs.
    \end{enumerate}

    So for every two Dedekind cuts $A$, $B$, one of the following holds: $A\subset B$, $A = B$, $A\supset B$.

    Thus, the set of all Dedekind cuts of $\mathbb{Q}$ is totally ordered with $\subseteq$ relation.
\end{proof}

We define the relation $\le$ on $\mathscr{D}_{\mathbb{Q}}$ as follow:
\[
    A\le B \Longleftrightarrow A\subseteq B.
\]

For convenience, in this section, we use the following notation:
\[
    \begin{split}
        {0}^{*} & = \{ x : x\in\mathbb{Q} \land x < 0 \}, \\
        {q}^{*} & = \{ x : x\in\mathbb{Q} \land x < q \}\qquad\text{($q\in\mathbb{Q}$)}.
    \end{split}
\]

\begin{definition}
    A Dedekind cut $A$ of $\mathbb{Q}$ is called:
    \begin{enumerate}[label={(\roman*)},itemsep=0pt]
        \item positive if $A$ is a proper superset of ${0}^{*}$,
        \item negative if $A$ is a proper subset of ${0}^{*}$,
        \item non-positive if $A\le {0}^{*}$,
        \item non-negative if $A\ge {0}^{*}$.
    \end{enumerate}
\end{definition}

\subsection{Rational and Irrational cuts}

Dedekind cuts are used to define rational and irrational cuts. These notions correspond to rational numbers and irrational numbers, respectively.

\begin{definition}[Rational and irrational]
    A Dedekind cut $A$ of $\mathbb{Q}$ is called:
    \begin{enumerate}[label={(\roman*)},itemsep=0pt]
        \item rational if $\mathbb{Q}\setminus A$ has a least element,
        \item irrational if $\mathbb{Q}\setminus A$ has no least element.
    \end{enumerate}
\end{definition}

In a constructive approach, we must provide examples of rational cuts and irrational cuts.

\begin{example}
    \[
        A = \{ x : x\in\mathbb{Q}\land x < q \}
    \]

    where $q$ is a rational number, is a rational cut.
    \[
        B = \{ x\in\mathbb{Q}: {x}^{2} < 2 \} \cup \mathbb{Q}_{-}
    \]

    is an irrational cut.
\end{example}

\begin{lemma}
    There is no rational number whose square equals $2$.
\end{lemma}
\begin{proof}[Lemma's proof]
    Assume that there exists a rational number $q$ such that ${q}^{2} = 2$.

    Since $q$ is a rational number, then there exists coprime integers $a$ and $b$ such that $q = \dfrac{a}{b}$. Then ${a}^{2} = 2{b}^{2}$.

    $2$ divides $2{b}^{2}$ so $2$ also divides ${a}^{2}$. Thereby, $a$ is even, and there exists an integer $a_{1}$ such that $a = 2\cdot a_{1}$. Substitute $a = 2\cdot a_{1}$, we obtain that $2{a_{1}}^{2} = b^{2}$. $2$ divides $2{a_{1}}^{2}$ so $2$ also divides ${b}^{2}$, which implies that $b$ is even.

    So $a$ and $b$ are even integers, which contradicts $a$ and $b$ are coprime. Hence the initial assumption is false.

    Thus, there does not exist any rational number whose square equals $2$.
\end{proof}

\begin{proof}
    \begin{itemize}[topsep=0pt]
        \item Rational cut.
              \begin{enumerate}[label={(\roman*)},topsep=0pt]
                  \item $A$ is not empty since there are rational numbers which are less than $q$.
                  \item $A\ne\mathbb{Q}$ since $A$ is bounded above by $q$.
                  \item Let $x\in A$.

                        If $x = \dfrac{a}{b} > 0$ and $q = \dfrac{c}{d} > 0$, where $a, b, c, d > 0$, then
                        \[
                            x = \frac{a}{b} < \frac{a + c}{b + d} < \frac{c}{d} = q
                        \]

                        so there exists an element of $A$ which is greater than $x$.

                        If $x < 0$ and $q > 0$, then $0$ is an element of $A$ which is greater than $x$.

                        If $x = -\dfrac{a}{b} < 0$ and $q = -\dfrac{c}{d} < 0$ where $a, b, c, d > 0$, then
                        \[
                            x = -\frac{a}{b} < -\frac{a + c}{b + d} < -\frac{c}{d} = q
                        \]

                        so there exists an element of $A$ which is greater than $x$.

                        Therefore, $A$ does not have a greatest element.
                  \item Let $x\in A$.

                        Let $y$ be a rational number such that $y < x$. Since $x < q$, then $y < q$. Therefore, $y\in A$.

                        So $A$ is closed downward.
              \end{enumerate}

              $\mathbb{Q}\setminus A = \{ x\in\mathbb{Q}: x\ge q \}$ has least element, which is $q$. So $A$ is a rational cut.
        \item Irrational cut.
              \begin{enumerate}[label={(\roman*)},topsep=0pt]
                  \item $B$ contains $0$, so $B$ is not empty.
                  \item $B$ does not contain $2$, so $B\ne\mathbb{Q}$.
                  \item Let $x\in B$.

                        If $x\le 0$, then there exists elements of $B$ which are greater than $x$. For example, $1$.

                        Otherwise, $x > 0$. Let $y = \dfrac{4\cdot x}{{x}^{2} + 2}$.
                        \begin{align*}
                            y & = \frac{4\cdot x}{{x}^{2} + 2} - x + x                  \\
                              & = \frac{4\cdot x - {x}^{3} - 2\cdot x}{{x}^{2} + 2} + x \\
                              & = \frac{2\cdot x - {x}^{3}}{{x}^{2} + 2} + x            \\
                              & = \frac{x\cdot (2 - {x}^{2})}{{x}^{2} + 2} + x > x
                        \end{align*}

                        On the other hand
                        \begin{align*}
                            {y}^{2} & = \frac{16\cdot {x}^{2}}{{({x}^{2} + 2)}^{2}} - 2 + 2                                  \\
                                    & = \frac{16\cdot {x}^{2} - 2\cdot{x}^{4} - 8 - 8\cdot {x}^{2}}{{({x}^{2} + 2)}^{2}} + 2 \\
                                    & = \frac{-2\cdot {({x}^{2} - 2)}^{2}}{{({x}^{2} + 2)}^{2}} + 2 < 2
                        \end{align*}

                        Therefore, for every $x$, there exists an element $y$ of $A$ such that $x < y$, which implies that $A$ does not have a greatest element.
                  \item Let $x\in B$.

                        Let $y$ be a rational number such that $y < x$.

                        If $y\le 0$, then $y\in B$, since $B$ is a superset of $\mathbb{Q}$.

                        Otherwise, $y > 0$, then
                        \[
                            {y}^{2} = {y}^{2} - {x}^{2} + {x}^{2} = (y - x)\cdot(y + x) + {x}^{2} < 0 + 2 = 2
                        \]

                        so $y\in B$. Therefore, $B$ is closed downward.
              \end{enumerate}

              Hence, $B$ is a Dedekind cut. Now I will prove that $B$ is an irrational cut.

              $\mathbb{Q}\setminus B = \{ x\in\mathbb{Q}: {x}^{2}\ge 2 \wedge x > 0 \}$.

              Since there is no rational number $r$ of which square equals $2$, then $\mathbb{Q}\setminus B = \{ x\in\mathbb{Q}: {x}^{2} > 2 \wedge x > 0 \}$ (change from $\ge$ to $>$).

              Let $q\in\mathbb{Q}\setminus B$. Consider $r = \dfrac{q}{2} + \dfrac{1}{q}$.
              \begin{align*}
                  r = \frac{q}{2} + \frac{1}{q} & = -\frac{q}{2} + \frac{1}{q} + q                       \\
                                                & = \frac{2 - {q}^{2}}{2q} + q                           \\
                                                & < q \quad\text{(Since $q > 0$ and $2 - {q}^{2} < 0$)}.
              \end{align*}
              \begin{align*}
                  {r}^{2} & = {\left(\frac{q}{2} + \frac{1}{q}\right)}^{2} = \frac{q^{2}}{4} + \frac{1}{q^{2}} + 1                                                         \\
                          & = \frac{q^{2}}{4} + \frac{1}{q^{2}} - 1 + 2 = {\left(\frac{q}{2} - \frac{1}{q}\right)}^{2} + 2 = {\left( \frac{q^{2} - 2}{2q} \right)}^{2} + 2 \\
                          & > 2
              \end{align*}

              So, for every $q$ in $\mathbb{Q}\setminus B$, there exists a rational number $r$ in $\mathbb{Q}\setminus B$ which is smaller than $q$.

              Hence $\mathbb{Q}\setminus B$ does not have a least element. Thus, $B$ is an irrational cut.
    \end{itemize}
\end{proof}

\subsection{Addition}

In this subsection, we define addition (denoted by $+$), subtraction, negation, and prove that $\mathscr{D}_{\mathbb{Q}}$ with addition is a commutative group.

\begin{theorem}[Addition]
    $A, B$ are Dedekind cuts. Then
    \[
        A + B = \{ x + y : x\in A \wedge y\in B \}
    \]
    is also a Dedekind cut.
\end{theorem}

\begin{proof}
    \begin{enumerate}[label={(\roman*)},itemsep=0pt]
        \item Since $A\ne\varnothing$ and $B\ne\varnothing$, then there exists $a\in A$ and $b\in B$. By definition of $A + B$, we obtain that $a + b \in A + B$. This implies that $A + B$ is not empty.
        \item Let $a$ be an element of $\mathbb{Q}\setminus A$, $b$ be an element of $\mathbb{Q}\setminus B$.

              Then $a$ is an upper bound of $A$, $b$ is an upper bound of $B$.

              $\forall x\in A\forall y\in B$, then $x + y \le a + b$, which means $A + B$ is bounded above.

              Hence $A + B\ne\mathbb{Q}$.
        \item Let $c$ be an element of $A + B$. According to the definition of $A + B$, there exists $a\in A$ and $b\in B$ such that $a + b = c$.

              According to (DC3), there exists $a_{0}\in A$ such that $a < a_{0}$, and there exists $b_{0}\in B$ such that $b < b_{0}$.

              $c = a + b < a_{0} + b_{0}$. According to the definition of $A + B$, $a_{0} + b_{0} \in A + B$. Hence $A + B$ has no greatest element.
        \item Let $c$ be an element of $A + B$. According to the definition of $A + B$, there exists $a\in A$ and $b\in B$ such that $a + b = c$.

              Let $c_{1}$ be a rational number such that $c_{1} < c$.

              According to (DC4), $b + (c_{1} - c)\in B$.
              \[
                  c_{1} = \underbrace{a}_{\in A} + \underbrace{b + (c_{1} - c)}_{\in B} \in A + B.
              \]

              Hence $A + B$ is downward closed.
    \end{enumerate}

    In conclusion, $A + B$ is a Dedekind cut of $\mathbb{Q}$.
\end{proof}

Unlike addition, we cannot define subtraction in the element-wise style. I recommend readers to verify this with some specific rational cuts/rational numbers.

\begin{theorem}[Subtraction]
    Let $A$ and $B$ be Dedekind cuts of $\mathbb{Q}$. The set
    \[
        A - B = \{ a - b': a\in A\land b'\in\mathbb{Q}\setminus B \}
    \]

    is also a Dedekind cut.
\end{theorem}

\begin{proof}
    \begin{enumerate}[label={(\roman*)}]
        \item Since $A\ne\varnothing$, there exists $a\in A$. Since $B\ne\mathbb{Q}$, there exists $b\in\mathbb{Q}\setminus B$. So $a - b\in A - B$, which implies that $A - B$ is not empty.
        \item Let $a, b, a', b'$ be arbitrary elements of $A, B, \mathbb{Q}\setminus A, \mathbb{Q}\setminus B$, respectively. Then $a < a'$ and $b < b'$.

              According to the definition of $A - B$, $a - b'$ is in $A - B$.
              \[
                  a - b' < a' - b
              \]

              Therefore, $A - B$ is bounded above, so $A - B\ne\mathbb{Q}$.
        \item Let $c$ be an arbitrary element of $A - B$. According to the definition of $A - B$, there exists $a\in A$ and $b'\in\mathbb{Q}\setminus B$ such that $c = a - b'$.

              Due to (DC3), there exists $d\in A$ such that $a < d$.

              Then $c = a - b' < d - b' \in A - B$.

              So $A - B$ does not have greatest element.
        \item Let $c$ be an arbitrary element of $A - B$. According to the definition of $A - B$, there exists $a\in A$ and $b'\in\mathbb{Q}\setminus B$ such that $c = a - b'$.

              Let $d$ be a rational number such that $d < c$.
              \[
                  d = (d - c) + c = (d - c) + a - b = \underbrace{(d - c + a)}_{\in A} - \underbrace{b'}_{\in\mathbb{Q}\setminus B}
              \]

              $(d - c + a) < a$ implies $(d - c + a)\in A$, according to (DC4). Hence $d\in A - B$.

              So $A - B$ is downward closed.
    \end{enumerate}

    Thus, $A - B$ is a Dedekind cut.
\end{proof}

We define negation as a particular case of subtraction.

\begin{theorem}[Additive inverse/Negation]
    Let $A$ be a Dedekind cut.
    \[
        -A = \{ w - a : w\in{0}^{*} \wedge a\in\mathbb{Q}\setminus A \}
    \]

    is also a Dedekind cut.
\end{theorem}

Also, the relation between subtraction and negation must not be taken for granted.

\begin{theorem}\label{theorem:chapter1:negation-and-subtraction}
    Let $A$ and $B$ be Dedekind cuts of $\mathbb{Q}$, then
    \[
        A - B = A + (-B).
    \]
\end{theorem}

\begin{proof}
    According to the definitions of addition, subtraction, and negation
    \begin{align*}
        A - B    & = \{ a - b' : a\in A\land b'\in\mathbb{Q}\setminus B \},                      \\
        A + (-B) & = \{ a + w - b' : a\in A\land w\in{0}^{*}\land b'\in\mathbb{Q}\setminus B \}.
    \end{align*}

    We will prove that these two sets are identical.

    \begin{enumerate}[label={\textbf{Step \arabic*.}},itemindent=0.5cm]
        \item Prove that $A - B\subseteq A + (-B)$.

              Let $c = a - b'$ be an arbitrary element in $A - B$, where $a\in A$ and $b'\in\mathbb{Q}\setminus B$.

              According to (DC3), there exists $a_{0}\in A$ such that $a < a_{0}$.
              \[
                  c = a - b = \underbrace{a_{0}}_{\in A} + \underbrace{(a - a_{0})}_{< 0} - \underbrace{b'}_{\in\mathbb{Q}\setminus B}
              \]

              so $c\in A + (-B)$. Hence $A - B\subseteq A + (-B)$.
        \item Prove that $A + (-B)\subseteq A - B$.

              Let $c = a + (w - b')$ be an arbitrary element of $A + (-B)$, where $a\in A$, $w < 0$, and $b'\in\mathbb{Q}\setminus B$.

              $a + w < a$. According to (DC4), $a + w\in A$.
              \[
                  c = a + (w - b) = \underbrace{(a + w)}_{\in A} - \underbrace{b'}_{\in\mathbb{Q}\setminus B}
              \]

              so $c\in A - B$. Hence $A + (-B)\subseteq A - B$.
    \end{enumerate}

    Thus, $A - B = A + (-B)$.
\end{proof}

I need the following property to verify the group structure of $\mathscr{D}_{\mathbb{Q}}$.

\begin{theorem}[Archimedean property for rational numbers]
    Let $a, b$ be rational numbers where $b > 0$. Then there exists an integer $n$ such that
    \[
        (n - 1)\cdot b \le a < n\cdot b.
    \]
\end{theorem}

\begin{proof}
    Since $a, b$ are rational numbers, then $\dfrac{a}{b}$ is also rational number.

    Then there exists a positive integer $q$ and an integer $p$ such that $\dfrac{a}{b} = \dfrac{p}{q}$.

    Apply Euclid division algorithm, there exists two integers $k, r$ such that $0\le r < q$ and $p = k\cdot q + r$.
    \begin{align*}
                         & \frac{p}{q} = \frac{k\cdot q + r}{q} = k + \frac{r}{q} \\
        \Rightarrow\quad & k \le \frac{p}{q} < k + 1.
    \end{align*}

    Hence $k \le \dfrac{a}{b} < k + 1$.
\end{proof}

\begin{theorem}\label{theorem:chapter1:real-field-part-one}
    $\mathscr{D}_{\mathbb{Q}}$ with addition is a commutative group.
    \begin{enumerate}[label={(F\arabic*)},itemsep=0pt]
        \item Addition is associative.
        \item Addition has identity element.
        \item Each element has an additive inverse.
        \item Addition is commutative.
    \end{enumerate}
\end{theorem}

\begin{proof}
    Let $A, B, C$ be arbitrary Dedekind cuts.
    \begin{enumerate}[label = (F\arabic*)]
        \item Addition is associative.
              \begin{align*}
                  (A + B) + C & = \{ (a + b) + c : a\in A\land b\in B\land c\in C \}                                                       \\
                              & = \{ a + (b + c) : a\in A\land b\in B\land c\in C \} \quad\text{(Addition in $\mathbb{Q}$ is associative)} \\
                              & = A + (B + C).
              \end{align*}
        \item Addition has identity element.
              \begin{align*}
                  A + {0}^{*} & = \{ a + w : a\in A\land w < 0 \} \\
                              & = \{ w + a : a\in A\land w < 0 \} \\
                              & = {0}^{*} + A.
              \end{align*}

              \begin{enumerate}[label={\textbf{Step \arabic*.}}]
                  \item Prove that $A \subseteq A + {0}^{*}$.

                        Let $x\in A$. According to (DC3), there exists $y\in A$ such that $x < y$.
                        \[
                            x = \underbrace{y}_{\in A} + \underbrace{(x - y)}_{< 0, \in {0}^{*}}
                        \]

                        So $\forall x(x\in A \rightarrow x\in A + {0}^{*})$, which means $A \subseteq A + {0}^{*}$.
                  \item Prove that $A + {0}^{*} \subseteq A$.

                        Let $a_{0}\in A + {0^{*}}$. According to the definition of $A + {0}^{*}$, there exists $a\in A$ and $w\in {0}^{*}$ such that $a_{0} = a + w$.

                        Since $w < 0$ then $a_{0} < a$. According to (DC4), $a_{0}\in A$.

                        So $\forall a_{0}(a_{0}\in A + {0}^{*} \rightarrow A)$.
              \end{enumerate}

              Hence $A = A + {0}^{*} = {0}^{*} + A$.
        \item Every element has an additive inverse.
              \begin{align*}
                  A + (-A) & = \{ a + (w - a') : a\in A\land w\in {0}^{*}\land a'\in\mathbb{Q}\setminus A \} \\
                           & = \{ (w - a') + a : a\in A\land w\in {0}^{*}\land a'\in\mathbb{Q}\setminus A \} \\
                           & = (-A) + A.
              \end{align*}
              \textbf{Step 1. Prove that $A + (-A)\subseteq {0}^{*}$}.

              Let $x$ be an element of $A + (-A)$. According to the definition of $A + (-A)$, there exists $a\in A, a'\in \mathbb{Q}\setminus A, w\in {0}^{*}$ such that $x = a + (w - a')$.

              Since $a'\notin A$ then $a' > a$, so $a - a' < 0$. Therefore
              \[
                  x = a + (w - a') = w + (a - a') < w
              \]
              According to (DC4), $x = a + (w - a')\in {0}^{*}$. So $A + (-A) \subseteq {0}^{*}$.
              \bigskip

              \textbf{Step 2. Prove that ${0}^{*}\subseteq A + (-A)$}.

              Let $w\in {0}^{*}$.

              We will prove that the set $S = \{ n : n\in\mathbb{Z} \land n\cdot w\in A \}$ has a least element. In equivalent, $S$ is the set of integers such that $n\cdot w\in A$.
              Let $x\in A, y\in\mathbb{Q}\setminus A$. $w < 0$, then according to the Archimedean property, there exists an integer $m$ such that
              \begin{align*}
                  m - 1    & \le \frac{x}{-w} < m \\
                  (1 - m)w & \le x < -m\cdot w
              \end{align*}

              According to (DC4), $(1 - m)w\in A$. Then $S$ is not empty.

              Since the set $\{ n\cdot w : n\in\mathbb{Z}\land n\cdot w\in A \}$ is bounded above, and $w < 0$, then $S$ is bounded below. On the other hand, $S$ is a set of consecutive integers and $S$ is bounded below, then $S$ has a least element (this is another form of the well-ordering principle).

              Let $n$ be the least integer such that $n\cdot w\in A$. Then $(n - 1)\cdot w\notin A$. Hence $(n - 1)\cdot w\in\mathbb{Q}\setminus A$.
              \[
                  w = \underbrace{n\cdot w}_{\in A} - \underbrace{(n - 1)\cdot w}_{\in\mathbb{Q}\setminus A}
              \]

              so $w\in A - A$. According to Theorem~\ref{theorem:chapter1:negation-and-subtraction}, $A - A = A + (-A)$. Then $w\in A + (-A)$. Therefore, ${0}^{*}\subseteq A + (-A)$.
              \bigskip

              Hence $A + (-A) = (-A) + A = {0}^{*}$.
        \item Addition is commutative.
              \begin{align*}
                  A + B & = \{ a + b : a\in A\land b\in B \}                                                      \\
                        & = \{ b + a : b\in B\land a\in A \}\qquad\text{(Addition n $\mathbb{Q}$ is commutative)} \\
                        & = B + A.\qedhere
              \end{align*}
    \end{enumerate}
\end{proof}

\begin{theorem}\label{theorem:chapter1:negation-is-an-involution}
    Let $A$ be a Dedekind cut, then
    \[
        A = -(-A).
    \]
\end{theorem}

\begin{proof}
    Firstly, we prove that there exist a unique Dedekind cut $A'$ such that $A + A' = A' + A = {0}^{*}$.

    According to Theorem~\ref{theorem:chapter1:real-field-part-one}, $A + (-A) = (-A) + A = {0}^{*}$.

    Suppose that $A + A' = A' + A = {0}^{*}$.
    \begin{align*}
        A' & = A' + {0}^{*} = A' + (A + (-A))   \\
           & = (A' + A) + (-A) = {0}^{*} + (-A) \\
           & = -A.
    \end{align*}

    Hence there exists unique Dedekind cut $A'$ of $\mathbb{Q}$ such that $A + A' = A' + A = {0}^{*}$.
    \begin{align*}
         & A + (-A) = (-A) + A = {0}^{*}             \\
         & (-(-A)) + (-A) = (-A) + (-(-A)) = {0}^{*}
    \end{align*}

    Thus, $A = -(-A)$.
\end{proof}

\begin{theorem}
    In $\mathscr{D}_{\mathbb{Q}}$, addition is compatible with $\subseteq$
    \[
        \forall A, B, C\in\mathscr{D}_{\mathbb{Q}}(A\subseteq B \rightarrow A + C\subseteq B + C).
    \]

    Equality holds if and only if $A = B$.
\end{theorem}

\begin{proof}
    Let $A, B, C$ be Dedekind cuts of $\mathbb{Q}$ such that $A\subseteq B$

    Let $a\in A, c\in C$, then $a + c\in A + C$. Since $A\subset B$, then $a\in B$, so $a + c\in B + C$. Therefore $A + C\subseteq B + C$.

    If $A = B$ then $A + C = B + C$.

    Otherwise, $A\subset B$, then there exists $b\in B$ such that $b\notin A$. Hence, for all $c\in C$, $b + c\notin A + C$, then $A + C\ne B + C$. So $A + C\subset B + C$.

    Thus $A + C\subseteq B + C$. Equality holds if and only if $A = B$.
\end{proof}

\begin{theorem}\label{theorem:chapter1:negation-and-sign}
    Let $A$ be a Dedekind cut of $\mathbb{Q}$, then
    \[
        A\subset {0}^{*} \Longleftrightarrow -A\supset {0}^{*}.
    \]
\end{theorem}

\begin{proof}
    $(\Rightarrow)$ $A\subset {0}^{*}\Longrightarrow -A\supset {0}^{*}$.

    Since $A$ is a proper subset of ${0}^{*}$, then there exists a rational number $q$ such that $q\in {0}^{*}$ and $q\notin A$. Apply (DC3) to ${0}^{*}$, there exists a rational number $q_{0}\in {0}^{*}$ and $q < q_{0}$

    According to the definition of negation, $q_{0} - q\in -A$. On the other hand, $q_{0} - q > 0$, so $-A \supset {0}^{*}$.

    \bigskip
    $(\Leftarrow)$ $-A\supset {0}^{*}\Longrightarrow A\subset {0}^{*}$.

    $-A$ is a proper superset of ${0}^{*}$, then there exists a rational number $q$ such that $q\in -A$ and $q\notin {0}^{*}$. In other words, $q$ is a positive rational number.

    According to Theorem~\ref{theorem:chapter1:negation-is-an-involution} and the definition of negation
    \[
        A = -(-A) = \{ w - c : w\in {0}^{*}\land c\in\mathbb{Q}\setminus -A \}
    \]

    For every $w\in {0}^{*}$ and $c\in\mathbb{Q}\setminus - A$, $w - c < w - q < 0 - q = -q$. So $A\subseteq {0}^{*}$.

    Furthermore, $-q$ is an upper bound of $A$, so $-q$ is not in $A$ (otherwise, it contradicts (DC3)).

    Therefore, $A\subset {0}^{*}$.
\end{proof}

\begin{corollary}
    $A$ is a Dedekind cut, then
    \[
        A\subseteq {0}^{*} \Longleftrightarrow -A\supseteq {0}^{*}.
    \]

    Equality holds if and only if $A = {0}^{*}$.
\end{corollary}

\subsection{Multiplication}

\begin{theorem}[Multiplication]\label{theorem:chapter1:multiplication}
    Let $A, B$ be Dedekind cuts of $\mathbb{Q}$. The set $A\cdot B$ is defined as the following.

    If $A\supseteq{0}^{*}$ and $B\supseteq{0}^{*}$
    \[
        A\cdot B = \{ a\cdot b : a\in A\wedge a\ge 0 \wedge b\in B\wedge b\ge 0 \} \cup \mathbb{Q}_{-}.
    \]

    If $A\subseteq{0}^{*}$ and $B\subseteq{0}^{*}$
    \[
        A\cdot B = (-A)\cdot (-B).
    \]

    If $A\subseteq{0}^{*}$ and $B\supseteq{0}^{*}$
    \[
        A\cdot B = -\left((-A)\cdot B\right).
    \]

    If $A\supseteq{0}^{*}$ and $B\subseteq{0}^{*}$
    \[
        A\cdot B = -\left(A\cdot (-B)\right).
    \]

    $A\cdot B$ is also a Dedekind cut.
\end{theorem}

\begin{proof}
    \begin{enumerate}[label={\textbf{Case \arabic*.}},itemindent={0.5cm}]
        \item $A\supseteq {0}^{*}\land B\supseteq {0}^{*}$.

              Let's consider the following three sub-cases.
              \begin{enumerate}
                  \item $A = {0}^{*}$.

                        Since $A = {0}^{*}$, then $\nexists a\in A$ such that $a\ge 0$. Hence
                        \[
                            A\cdot B = \{ a\cdot b: a\in A\land a\ge 0\land b\in B\land b\ge 0 \} \cup\mathbb{Q}_{-} = \varnothing\cup\mathbb{Q}_{-} = \mathbb{Q}_{-} = {0}^{*}.
                        \]
                  \item $B = {0}^{*}$.

                        Since $B = {0}^{*}$, then $\nexists b\in B$ such that $b\ge 0$. Hence
                        \[
                            A\cdot B = \{ a\cdot b: a\in A\land a\ge 0\land b\in B\land b\ge 0 \} \cup\mathbb{Q}_{-} = \varnothing\cup\mathbb{Q}_{-} = \mathbb{Q}_{-} = {0}^{*}.
                        \]
                  \item $A\supset{0}^{*}$ and $B\supset{0}^{*}$.
                        \begin{enumerate}[label = (\roman*)]
                            \item Since $A\cdot B$ is a superset of $\mathbb{Q}_{-}$, then $A\cdot B$ is not empty.
                            \item Let $a_{0}$ be an upper bound of $A$, $b_{0}$ be an upper bound of $B$.

                                  Since $A\supset{0}^{*}$ and $B\supset{0}^{*}$, then $a_{0}\ge 0$ and $b_{0}\ge 0$.

                                  Then for any non-negative elements $a$ and $b$ of $A$ and $B$, $a\cdot b \le a_{0}\cdot b_{0}$.

                                  Hence $a_{0}\cdot b_{0}$ is an upper bound of $A\cdot B$, which implies that $A\cdot B\ne\mathbb{Q}$.
                            \item Let $c$ be an arbitrary element of $A\cdot B$.

                                  If $c$ is negative or zero, then there exists an element which is greater than $c$, since $A\supset {0}^{*}$ and $B\supset {0}^{*}$ (zero is not their greatest element).

                                  Otherwise, $c$ is positive, then there exists $a\in A$ and $a > 0$, $b\in B$ and $b > 0$ such that $a\cdot b = c$. Due to (DC3), there exists $a_{0} > a > 0$ and $a_{0}\in A$, $b_{0} > b > 0$ and $b_{0}\in B$.

                                  Furthermore, $a_{0}\cdot b_{0} > a\cdot b$ and $a_{0}\cdot b_{0}\in A\cdot B$ according to the definition of $A\cdot B$.

                                  So $A\cdot B$ has no greatest element.
                            \item Let $c$ be an arbitrary element of $A\cdot B$.

                                  Let $d$ be a rational number such that $d < c$.

                                  If $d$ is non-positive, then $d\in A\cdot B$, since $A\cdot B$ contains $0$ and is a superset of $\mathbb{Q}_{-}$.

                                  Otherwise, $d$ is positive, then $c$ is also positive. Since $c$ is positive, there exists $a\in A$ and $a > 0$, $b\in B$ and $b > 0$ such that $c = a\cdot b$.
                                  \[
                                      d = c - (c - d) = a\cdot b - (c - d) = a\cdot\left(b - \frac{c - d}{a}\right)
                                  \]

                                  Since $a\in A$ and $a > 0$, $b - \dfrac{c - d}{a}\in B$ (due to (DC4)) and $0 < b - \dfrac{c - d}{a} < b$, then $d \in A\cdot B$.

                                  Hence $A\cdot B$ is downward closed.
                        \end{enumerate}
              \end{enumerate}
        \item $A\subseteq {0}^{*}, B\subseteq {0}^{*}$.

              Then $-A\supseteq {0}^{*}$, $-B\supseteq {0}^{*}$. According to 1st case, $(-A)\cdot (-B)$ is a Dedekind cut.
        \item $A\subseteq {0}^{*}, B\supseteq {0}^{*}$.

              Then $-A\supseteq {0}^{*}$. According to 1st case, $(-A)\cdot B$ is a Dedekind cut. So $-((-A)\cdot B)$ is a Dedekind cut.
        \item $A\supseteq {0}^{*}, B\subseteq {0}^{*}$.

              Then $-B\supseteq {0}^{*}$. According to 1st case, $A\cdot (-B)$ is a Dedekind cut. So $-(A\cdot (-B))$ is a Dedekind cut.
    \end{enumerate}

    Thus, $A\cdot B$ is a Dedekind cut.
\end{proof}

\begin{theorem}\label{theorem:chapter1:multiplication-and-negation}
    Let $A, B$ be Dedekind cuts. Then
    \[
        \begin{cases}
            (-A)\cdot B = A\cdot (-B) = - A\cdot B, \\
            (-A)\cdot (-B) = A\cdot B.
        \end{cases}
    \]
\end{theorem}

\begin{proof}
    \noindent\textbf{Step 1. Prove that $A\cdot (-B) = (-A)\cdot B = -A\cdot B$.}

    Let's apply the definition of multiplication and Theorem~\ref{theorem:chapter1:negation-is-an-involution}.

    \begin{enumerate}[label={\textbf{Case \arabic*.}},itemsep=0pt,itemindent=0.5cm]
        \item $A\supseteq {0}^{*}, B\supseteq {0}^{*}$.
              \begin{align*}
                  A\cdot (-B) & = -A\cdot (-(-B)) = -A\cdot B, \\
                  (-A)\cdot B & = -(-(-A))\cdot B = -A\cdot B.
              \end{align*}
        \item $A\supseteq {0}^{*}, B\subseteq {0}^{*}$.
              \begin{align*}
                  A\cdot (-B) & = -A\cdot (-(-B)) = -A\cdot B,     \\
                  (-A)\cdot B & = (-(-A))\cdot (-B) = A\cdot (-B).
              \end{align*}
        \item $A\subseteq {0}^{*}, B\supseteq {0}^{*}$.
              \begin{align*}
                  A\cdot (-B) & = (-A)\cdot (-(-B)) = (-A)\cdot B, \\
                  -A\cdot B   & = -(-(-A)\cdot B) = (-A)\cdot B.
              \end{align*}
        \item $A\subseteq {0}^{*}, B\subseteq {0}^{*}$.
              \begin{align*}
                  -A\cdot B   & = -(-A)\cdot (-B), \\
                  (-A)\cdot B & = -(-A)\cdot (-B), \\
                  A\cdot (-B) & = -(-A)\cdot (-B).
              \end{align*}
    \end{enumerate}

    Thus, $(-A)\cdot B = A\cdot (-B) = -A\cdot B$.

    \noindent\textbf{Step 2. Prove that $(-A)\cdot (-B) = A\cdot B$}

    \noindent Apply the result in Step 1, $(-A)\cdot (-B) = -A\cdot (-B) = -(-(A\cdot B)) = A\cdot B$.
\end{proof}

\begin{theorem}\label{theorem:chapter1:multiplication-and-order}
    Multiplication in $\mathscr{D}_{\mathbb{Q}}$ is compatible with $\subseteq$.
    \[
        \forall A, B\in\mathscr{D}_{\mathbb{Q}}(A\supseteq {0}^{*}\land B\supseteq {0}^{*}\rightarrow A\cdot B\supseteq {0}^{*}).
    \]
\end{theorem}

\begin{proof}
    \textbf{Case 1. $A = {0}^{*}$ or $B = {0}^{*}$.}

    According to the proof of Theorem~\ref{theorem:chapter1:multiplication} and Theorem~\ref{theorem:chapter1:multiplication-and-negation}, $A\cdot B = {0}^{*}$.
    \bigskip

    \textbf{Case 2. $A\supset {0}^{*}$ and $B\supset {0}^{*}$.}

    Since $A\supset {0}^{*}$, then there exists $a\in A$ such that $a > 0$. $B\supset {0}^{*}$, then there exists $b\in B$ such that $b > 0$.

    According to the definition of multiplication, $a\cdot b\in A\cdot B$. On the other hand, $a\cdot b > 0$. Therefore, $A\cdot B\supset {0}^{*}$.

    \bigskip
    Thus $A\cdot B\supseteq {0}^{*}$. Equality holds if and only if $A = {0}^{*}$ or $B\cdot {0}^{*}$.
\end{proof}

The following result follows Theorem~\ref{theorem:chapter1:negation-and-sign} Theorem~\ref{theorem:chapter1:multiplication}, and Theorem~\ref{theorem:chapter1:multiplication-and-order}.

\begin{corollary}
    Let $A, B$ be Dedekind cuts of $\mathbb{Q}$.
    \[
        \begin{split}
            A\supset {0}^{*}\land B\supset {0}^{*}\rightarrow A\cdot B\supset {0}^{*}, \\
            A\supset {0}^{*}\land B\subset {0}^{*}\rightarrow A\cdot B\subset {0}^{*}, \\
            A\subset {0}^{*}\land B\supset {0}^{*}\rightarrow A\cdot B\subset {0}^{*}, \\
            A\subset {0}^{*}\land B\subset {0}^{*}\rightarrow A\cdot B\supset {0}^{*}.
        \end{split}
    \]
\end{corollary}

\begin{theorem}
    $\mathscr{D}_{\mathbb{Q}}$ with addition and multiplication is a commutative ring.
    \begin{enumerate}[label={(F\arabic*)},itemsep=0pt]
        \item Multiplication is associative.
        \item Multiplication is distributive over addition.
        \item Multiplication has identity element.
        \item Multiplication is commutative.
    \end{enumerate}
\end{theorem}

\begin{proof}
    Let $A, B, C$ be arbitrary Dedekind cuts of $\mathbb{Q}$.
    \begin{enumerate}[label={(F\arabic*)}, start=5]
        \item Multiplication is associative.
              \begin{enumerate}[label={\textbf{Case \arabic*.}},topsep=0pt,itemsep=0pt]
                  \item $A\supseteq {0}^{*}, B\supseteq {0}^{*}, C\supseteq {0}^{*}$, then
                        \begin{align*}
                            (A\cdot B)\cdot C & = \{ (a\cdot b)\cdot c : a\in A\land b\in B\land c\in C\land a\ge 0\land b\ge 0\land c\ge 0 \} \\
                                              & = \{ a\cdot (b\cdot c) : a\in A\land b\in B\land c\in C\land a\ge 0\land b\ge 0\land c\ge 0 \} \\
                                              & = A\cdot (B\cdot C).
                        \end{align*}
                  \item $A\supseteq {0}^{*}, B\supseteq {0}^{*}, C\subseteq {0}^{*}$, then $-C\supseteq {0}^{*}$ and
                        \begin{align*}
                            (A\cdot B)\cdot C & = -\left( (A\cdot B)\cdot (-C) \right)  \\
                                              & = -\left( A\cdot (B \cdot (-C)) \right) \\
                                              & = A\cdot (-(B\cdot (-C)))               \\
                                              & = A\cdot (B\cdot C).
                        \end{align*}
                  \item $A\supseteq {0}^{*}, B\subseteq {0}^{*}, C\supseteq {0}^{*}$, then $-B\supseteq {0}^{*}$ and
                        \begin{align*}
                            (A\cdot B)\cdot C & = (-(A\cdot (-B)))\cdot C \\
                                              & = -((A\cdot (-B))\cdot C) \\
                                              & = -(A\cdot ((-B)\cdot C)) \\
                                              & = A\cdot (-((-B)\cdot C)) \\
                                              & = A\cdot (B\cdot C).
                        \end{align*}
                  \item $A\supseteq {0}^{*}, B\subseteq {0}^{*}, C\subseteq {0}^{*}$, then $-B\supseteq {0}^{*}$, $-C\supseteq {0}^{*}$, and
                        \begin{align*}
                            (A\cdot B)\cdot C & = (-(A\cdot (-B)))\cdot C       \\
                                              & = (-(-(A\cdot (-B))))\cdot (-C) \\
                                              & = (A\cdot (-B))\cdot (-C)       \\
                                              & = A\cdot ((-B)\cdot (-C))       \\
                                              & = A\cdot (B\cdot C).
                        \end{align*}
                  \item $A\subseteq {0}^{*}, B\supseteq {0}^{*}, C\supseteq {0}^{*}$, then $-A\supseteq {0}^{*}$, and
                        \begin{align*}
                            (A\cdot B)\cdot C & = (-((-A)\cdot B))\cdot C \\
                                              & = -(((-A)\cdot B)\cdot C) \\
                                              & = -((-A)\cdot (B\cdot C)) \\
                                              & = A\cdot (B\cdot C).
                        \end{align*}
                  \item $A\subseteq {0}^{*}, B\supseteq {0}^{*}, C\subseteq {0}^{*}$, then $-A\supset {0}^{*}$, $-C\supseteq {0}^{*}$, and
                        \begin{align*}
                            (A\cdot B)\cdot C & = (-(A\cdot B))\cdot (-C) \\
                                              & = ((-A)\cdot B)\cdot (-C) \\
                                              & = (-A)\cdot (B\cdot (-C)) \\
                                              & = (-A)\cdot (-(B\cdot C)) \\
                                              & = A\cdot (B\cdot C).
                        \end{align*}
                  \item $A\subseteq {0}^{*}, B\subseteq {0}^{*}, C\supseteq {0}^{*}$, then $-A\supseteq {0}^{*}, -B\supset {0}^{*}$, and
                        \begin{align*}
                            (A\cdot B)\cdot C & = ((-A)\cdot (-B))\cdot C \\
                                              & = (-A)\cdot ((-B)\cdot C) \\
                                              & = (-A)\cdot (-(B\cdot C)) \\
                                              & = A\cdot (B\cdot C).
                        \end{align*}
                  \item $A\subseteq {0}^{*}, B\subseteq {0}^{*}, C\subseteq {0}^{*}$, then $-A\supseteq {0}^{*}$, $-B\supseteq {0}^{*}$, $-C\supseteq {0}^{*}$, and
                        \begin{align*}
                            (A\cdot B)\cdot C & = -((A\cdot B)\cdot (-C))       \\
                                              & = -(((-A)\cdot (-B))\cdot (-C)) \\
                                              & = -((-A)\cdot ((-B)\cdot (-C))) \\
                                              & = -((-A)\cdot (B\cdot C))       \\
                                              & = A\cdot (B\cdot C).
                        \end{align*}
              \end{enumerate}
        \item Multiplication is distributive over addition.

              We will prove that
              \[
                  \begin{cases}
                      (A + B)\cdot C = A\cdot C + B\cdot C, \\
                      C\cdot (A + B) = C\cdot A + C\cdot B.
                  \end{cases}
              \]

              \begin{enumerate}[label={\textbf{Case \arabic*.}}]
                  \item $A = {0}^{*}$.
                        \[
                            \begin{split}
                                (A + B)\cdot C = B\cdot C = {0}^{*} + B\cdot C = A\cdot C + B\cdot C, \\
                                C\cdot (A + B) = C\cdot B = {0}^{*} + C\cdot B = C\cdot A + C\cdot B.
                            \end{split}
                        \]
                  \item $B = {0}^{*}$.
                        \[
                            \begin{split}
                                (A + B)\cdot C = A\cdot C = A\cdot C + {0}^{*} = A\cdot C + B\cdot C, \\
                                C\cdot (A + B) = C\cdot A = C\cdot A + {0}^{*} = C\cdot A + C\cdot B.
                            \end{split}
                        \]
                  \item $C = {0}^{*}$.
                        \[
                            \begin{split}
                                (A + B)\cdot C = {0}^{*} = {0}^{*} + {0}^{*} = A\cdot C + B\cdot C, \\
                                C\cdot (A + B) = {0}^{*} = {0}^{*} + {0}^{*} = C\cdot A + C\cdot B.
                            \end{split}
                        \]
                  \item $C\supset {0}^{*}, A\supset {0}^{*}, B\supset {0}^{*}$.
                        \begin{align*}
                            (A + B)\cdot C      & = \{ (a + b)\cdot c : a\in A\land b\in B\land c\in C\land a+b\ge 0\land c\ge 0 \} \cup\mathbb{Q}_{-}                                                              \\
                                                & = \{ a\cdot c + b\cdot c : a\in A\land b\in B\land c\in C\land a+b\ge 0\land c\ge 0 \} \cup\mathbb{Q}_{-}                                                         \\
                            \\
                            A\cdot C + B\cdot C & = \{ a\cdot c : a\in A\land c\in C\land a\ge 0\land c\ge 0 \} \cup\mathbb{Q}_{-} + \{ b\cdot c : b\in B\land c\in C\land b\ge 0\land c\ge 0 \} \cup\mathbb{Q}_{-}
                        \end{align*}
              \end{enumerate}

              \textbf{Part 1. Prove that $(A + B)\cdot C = A\cdot C + B\cdot C$.}

              \textbf{Step 1. Prove that $(A + B)\cdot C\subseteq A\cdot C + B\cdot C$.}

              Let $x\in (A + B)\cdot C$.

              If $x\le 0$, then $x\in A\cdot C + B\cdot C$, since both $A\cdot C + B\cdot C$ are supersets of $\{0\}\cup\mathbb{Q}$.

              \bigskip

              Otherwise $x > 0$, then there exists $a\in A, b\in B, c\in C$ such that $a + b > 0, c > 0$ and $(a + b)\cdot c = x$.

              $a\in A, c\in C$ and $c > 0$, then $a\cdot c\in A\cdot C$, no matter whether $a$ is greater than zero or not.

              $b\in B, c\in C$ and $c > 0$, then $b\cdot c\in B\cdot C$, no matter whether $b$ is greater than zero or not.

              So $x\in A\cdot C + B\cdot C$. Hence $(A + B)\cdot C\subseteq A\cdot C + B\cdot C$.

              \textbf{Step 2. Prove that $A\cdot C + B\cdot C\subseteq (A + B)\cdot C$.}

              Let $x\in A\cdot C + B\cdot C$.

              If $x\le 0$, then $x$ is also in $(A + B)\cdot C$ since $(A + B)\cdot C \supset \{0\}\cup\mathbb{Q}_{-}$.

              Otherwise, $x > 0$. I will show that here exists $y\in A\cdot C$ and $z\in B\cdot C$ such that $y > 0$, $z > 0$ and $y + z = x$.

              If $A\cdot C = B\cdot C$, then we choose $y = \dfrac{x}{2}, z = \dfrac{x}{2}$.

              Without loss of generality, suppose that $A$ is a proper subset of $B$.

              \begin{itemize}
                  \item $x\in A\cdot C$. Choose $y = \dfrac{x}{2}$ and $z = \dfrac{x}{2}$.
                  \item $x\notin A\cdot C\land x\in B\cdot C$.
                        Choose an element $y\in A\cdot C$ such that $y > 0$.

                        $x - y < x$. According to (DC4), $x - y\in B\cdot C$. Since $x\notin A\cdot C$ and $y\in A\cdot C$, then $x - y > 0$.

                        We choose $z = x - y$.
                  \item $x\notin B\cdot C$. Then $x$ is greater than any elements of $A\cdot C$ and $B\cdot C$. According to the definition of addition, there exists $y\in A\cdot C$ and $z\in B\cdot C$ such that $y + z = x$.

                        Since $x > y$ and $x > z$, then $y > 0$ and $z > 0$.
              \end{itemize}

              Now, for every positive rational number $x\in A\cdot C + B\cdot C$, there exists positive rational numbers $y\in A\cdot C$ and $z\in B\cdot C$ such that $y + z = x$.

              According to the definition of multiplication
              \begin{itemize}
                  \item $y\in A\cdot C$ and $y > 0$, then there exists $a\in A$, $c_{1}\in C$ such that $a > 0, c_{1} > 0$ and $a\cdot c_{1} = y$.
                  \item $z\in B\cdot C$ and $z > 0$, then there exists $b\in B$, $c_{2}\in C$ such that $b > 0, c_{2} > 0$ and $b\cdot c_{2} = z$.
              \end{itemize}

              (This choice of $c$ is from Hayden\footnote{\url{https://math.stackexchange.com/questions/1205640/proof-that-real-multiplication-distributes-over-addition-using-dedekind-cuts}} on MathOverflow) Let $c = \dfrac{a\cdot c_{1} + b\cdot c_{2}}{a + b}$. $c$ is between $c_{1}$ and $c_{2}$ so $c\in C$ (according to (DC4)).
              \begin{align*}
                  (a + b)\cdot c & = (a + b)\cdot\dfrac{a\cdot c_{1} + b\cdot c_{2}}{a + b} \\
                                 & = a\cdot c_{1} + b\cdot c_{2}                            \\
                                 & = y + z                                                  \\
                                 & = x.
              \end{align*}

              So $x\in (A + B)\cdot C$. Hence $A\cdot C + B\cdot C\subseteq (A + B)\cdot C$.

              Thus, $(A + B)\cdot C = A\cdot C + B\cdot C$.

              \bigskip

              \textbf{Part 2. $C\cdot (A + B) = C\cdot A + C\cdot B$}
              \begin{align*}
                  C\cdot (A + B) & = \{ c\cdot (a + b) : c\ge 0\land c\in C\land a\in A\land b\in B\land (a+b)\ge 0 \}\cup\mathbb{Q}_{-} \\
                                 & = \{ (a + b)\cdot c : c\ge 0\land c\in C\land a\in A\land b\in B\land (a+b)\ge 0 \}\cup\mathbb{Q}_{-} \\
                                 & = (A + B)\cdot C.
              \end{align*}
              On the other hand
              \begin{align*}
                  C\cdot A + C\cdot B & = \{ c\cdot a : c\in C\land c\ge 0\land a\in A\land a\ge 0 \}\cup\mathbb{Q}_{-} + \{ c\cdot b :  c\in C\land c\ge 0\land b\in B\land b\ge 0 \}\cup\mathbb{Q}_{-} \\
                                      & = \{ a\cdot c : c\in C\land c\ge 0\land a\in A\land a\ge 0 \}\cup\mathbb{Q}_{-} + \{ b\cdot c :  c\in C\land c\ge 0\land b\in B\land b\ge 0 \}\cup\mathbb{Q}_{-} \\
                                      & = A\cdot C + B\cdot C.
              \end{align*}
              According to step 1, $(A + B)\cdot C = A\cdot C + B\cdot C$.

              Thus, $C\cdot (A + B) = C\cdot A + C\cdot B$.

              \textbf{Case 5.} $C\supset {0}^{*}, A\subset {0}^{*}, B\subset {0}^{*}$.
              \begin{align*}
                  ((-B) + (-A)) + (A + B) & = ((-B) + ((-A) + A)) + B \\
                                          & = ((-B) + {0}^{*}) + B    \\
                                          & = (-B) + B                \\
                                          & = {0}^{*},                \\
                  (A + B) + ((-B) + (-A)) & = (A + (B + (-B))) + (-A) \\
                                          & = (A + {0}^{*}) + (-A)    \\
                                          & = A + (-A)                \\
                                          & = {0}^{*}.
              \end{align*}

              On the other hand, the addition over Dedekind cuts of $\mathbb{Q}$ is commutative, it follows that $(-B) + (-A) = (-A) + (-B)$. So $-(A + B) = (-A) + (-B)$.

              $A\subset {0}^{*}$ and $B\subset {0}^{*}$ implies that $-A\supset {0}^{*}$ and $-B\supset {0}^{*}$. Apply Case 4 and Theorem~\ref{theorem:chapter1:multiplication-and-negation}
              \begin{align*}
                  (A + B)\cdot C & = -((-A) + (-B))\cdot C           \\
                                 & = -((-A)\cdot C + (-B)\cdot C)    \\
                                 & = (-(-A)\cdot C) + (-(-B)\cdot C) \\
                                 & = A\cdot C + B\cdot C,            \\
                  C\cdot (A + B) & = -C\cdot ((-A) + (-B))           \\
                                 & = -(C\cdot (-A) + C\cdot (-B))    \\
                                 & = (-C\cdot (-A)) + (-C\cdot (-B)) \\
                                 & = C\cdot A + C\cdot B
              \end{align*}

              \textbf{Case 6.} $C\supset {0}^{*}, A\supset {0}^{*}, B\subset {0}^{*}$.

              \textbf{Case 6.1.} $A + B = {0}^{*}$.
              \[
                  \begin{split}
                      (A + B)\cdot C = {0}^{*} = A\cdot C + (-A)\cdot C = A\cdot C + B\cdot C, \\
                      C\cdot (A + B) = {0}^{*} = C\cdot A + C\cdot (-A) = C\cdot A + C\cdot B.
                  \end{split}
              \]
              \textbf{Case 6.2.} $A + B > {0}^{*}$.

              I apply Case 4 and Theorem~\ref{theorem:chapter1:multiplication-and-negation}
              {\allowdisplaybreaks{}
                  \begin{align*}
                      (A + B)\cdot C + (-B)\cdot C   & = ((A + B) + (-B))\cdot C   \\
                                                     & = (A + (B + (-B)))\cdot C   \\
                                                     & = A\cdot C,                 \\
                      \Longrightarrow (A + B)\cdot C & = A\cdot C - (-B)\cdot C    \\
                                                     & = A\cdot C + (-(-B)\cdot C) \\
                                                     & = A\cdot C + B\cdot C.      \\
                      \bigskip
                      C\cdot (A + B) + C\cdot (-B)   & = C\cdot ((A + B) + (-B))   \\
                                                     & = C\cdot (A + (B + (-B)))   \\
                                                     & = C\cdot A,                 \\
                      \Longrightarrow C\cdot (A + B) & = C\cdot A - C\cdot (-B)    \\
                                                     & = C\cdot A + C\cdot B.
                  \end{align*}}

              \textbf{Case 6.3.} If $A + B < {0}^{*}$.
                  {\allowdisplaybreaks{}
                      \begin{align*}
                          (A + B)\cdot C & = -((-A) + (-B))\cdot C                                                                                   \\
                                         & = -((-A)\cdot C + (-B)\cdot C) & \quad\text{(Case 6.2)}                                                   \\
                                         & = A\cdot C + B\cdot C,         & \quad\text{(Theorem~\ref{theorem:chapter1:multiplication-and-negation})} \\
                          C\cdot (A + B) & = -C\cdot ((-A) + (-B))                                                                                   \\
                                         & = -(C\cdot (-A) + C\cdot (-B)) & \quad\text{(Case 6.2)}                                                   \\
                                         & = C\cdot A + C\cdot B.         & \quad\text{(Theorem~\ref{theorem:chapter1:multiplication-and-negation})}
                      \end{align*}}

              \textbf{Case 7.} $C\supset {0}^{*}, A\subset {0}^{*}, B\supset {0}^{*}$.

              This case is similar to Case 6.

              \textbf{Case 7.1.} $A + B = {0}^{*}$.
              \[
                  \begin{split}
                      (A + B)\cdot C = {0}^{*} = A\cdot C + (-A)\cdot C = A\cdot C + B\cdot C, \\
                      C\cdot (A + B) = {0}^{*} = C\cdot A + C\cdot (-A) = C\cdot A + C\cdot B.
                  \end{split}
              \]
              \textbf{Case 7.2.} $A + B > {0}^{*}$.
              \begin{align*}
                  (-A)\cdot C + (A + B)\cdot C   & = ((-A) + (A + B))\cdot C \\
                                                 & = (((-A) + A) + B)\cdot C \\
                                                 & = B\cdot C                \\
                  \Longrightarrow (A + B)\cdot C & = -(-A)\cdot C + B\cdot C \\
                                                 & = A\cdot C + B\cdot C.    \\
                  \bigskip
                  C\cdot (-A) + C\cdot (A + B)   & = C\cdot ((-A) + (A + B)) \\
                                                 & = C\cdot (((-A) + A) + B) \\
                                                 & = C\cdot B                \\
                  \Longrightarrow C\cdot (A + B) & = -C\cdot (-A) + C\cdot B \\
                                                 & = C\cdot A + C\cdot B.
              \end{align*}

              \textbf{Case 7.3.} $A + B < {0}^{*}$.
              \begin{align*}
                  (A + B)\cdot C & = -((-A) + (-B))\cdot C                                                                                    \\
                                 & = -((-A)\cdot C + (-B)\cdot C) & \quad\text{(Case 7.2)}                                                    \\
                                 & = A\cdot C + B\cdot C,         & \quad\text{(Theorem~\ref{theorem:chapter1:multiplication-and-negation})}  \\
                  C\cdot (A + B) & = -C\cdot ((-A) + (-B))                                                                                    \\
                                 & = -(C\cdot (-A) + C\cdot (-B)) & \quad\text{(Case 7.2)}                                                    \\
                                 & = C\cdot A + C\cdot B.         & \quad\text{(Theorem~\ref{theorem:chapter1:multiplication-and-negation})}.
              \end{align*}

              \textbf{Case 8.} $C\subset {0}^{*}$.

              Apply Case 4, 5, 6, 7 and Theorem~\ref{theorem:chapter1:multiplication-and-negation}
              \[
                  \begin{split}
                      (A + B)\cdot C = -(A + B)\cdot (-C) = -(A\cdot (-C) + B\cdot (-C)) = A\cdot C + B\cdot C, \\
                      C\cdot (A + B) = -(-C)\cdot (A + B) = -((-C)\cdot A + (-C)\cdot B) = C\cdot A + C\cdot B.
                  \end{split}
              \]
        \item Multiplication has identity element.

              I will prove that
              \[
                  A\cdot {1}^{*} = {1}^{*}\cdot A = A.
              \]

              \begin{enumerate}[label={\textbf{Case \arabic*.}}]
                  \item $A = {0}^{*}$.

                        According to the proof of Theorem~\ref{theorem:chapter1:multiplication}
                        \[
                            A\cdot {1}^{*} = {0}^{*}\cdot {1}^{*} = {0}^{*} = {1}^{*}\cdot {0}^{*} = {1}^{*}\cdot A.
                        \]

                  \item $A > {0}^{*}$.
                        \begin{align*}
                            A\cdot {1}^{*} & = \{ a\cdot b : a\in A\land a\ge 0\land b\ge 0\land b < 1 \}\cup\mathbb{Q}_{-} \\
                                           & = \{ b\cdot a : a\in A\land a\ge 0\land b\ge 0\land b < 1 \}\cup\mathbb{Q}_{-} \\
                                           & = {1}^{*}\cdot A.
                        \end{align*}
                        Let $x\in A$.

                        If $x\le 0$, then $x\in A\cdot {1}^{*}$.

                        Otherwise, $x > 0$. According to (DC3), there exists $y\in A$ such that $x < y$. Then
                        \[
                            x = \underbrace{y}_{>0, \in A}\cdot\underbrace{\dfrac{x}{y}}_{>0, <1}
                        \]
                        Hence $x\in A\cdot {1}^{*}$, so $A\subseteq A\cdot {1}^{*}$.

                        \bigskip

                        Let $x\in A\cdot {1}^{*}$.

                        If $x\le 0$, then $x\in A$.

                        Otherwise, $x > 0$. Then there exists $a\in A$ and $b\in {1}^{*}$ such that $a > 0, b > 0$ and $x = a\cdot b$.

                        Since $0 < b < 1$, then $a\cdot b < a$.

                        According to (DC4), $x = a\cdot b\in A$, so $x\in A$. Therefore $A\cdot {1}^{*}\subseteq A$.

                        \bigskip

                        Thus, $A\cdot {1}^{*} = {1}^{*}\cdot A = A$.

                  \item $A < 0^{*}$

                        Apply Case 2 and Theorem~\ref{theorem:chapter1:multiplication-and-negation}
                        \[
                            \begin{split}
                                A\cdot {1}^{*} = -(-A)\cdot {1}^{*} = -(-A) = A, \\
                                {1}^{*}\cdot A = -{1}^{*}\cdot (-A) = -(-A) = A.
                            \end{split}
                        \]
              \end{enumerate}
        \item Multiplication is commutative.

              I will prove that $A\cdot B = B\cdot A$.

              If $A = {0}^{*}$ or $B^{*} = {0}^{*}$, then according to the proof of Theorem~\ref{theorem:chapter1:multiplication}, $A\cdot B = {0}^{*} = B\cdot A$.

              Otherwise, let's consider the following four cases.

              \textbf{Case 1.} $A > {0}^{*}, B > {0}^{*}$.
              \begin{align*}
                  A\cdot B & = \{ a\cdot b : a\in A\land b\in B\land a\ge 0\land b\ge 0 \}\cup\mathbb{Q}_{-} \\
                           & = \{ b\cdot a : a\in A\land b\in B\land a\ge 0\land b\ge 0 \}\cup\mathbb{Q}_{-} \\
                           & = B\cdot A.
              \end{align*}

              In the following cases, I use Case 1 and Theorem~\ref{theorem:chapter1:multiplication-and-negation}.

              \textbf{Case 2.} $A > {0}^{*}, B < {0}^{*}$.
              \[
                  A\cdot B = -A\cdot (-B) = -(-B)\cdot A = B\cdot A.
              \]

              \textbf{Case 3.} $A < {0}^{*}, B > {0}^{*}$.
              \[
                  A\cdot B = -(-A)\cdot B = -B\cdot (-A) = B\cdot A.
              \]

              \textbf{Case 4.} $A < {0}^{*}, B < {0}^{*}$.
              \[
                  A\cdot B = (-A)\cdot (-B) = (-B)\cdot (-A) = B\cdot A.
              \]

              Thus, multiplication in $\mathscr{D}_{\mathbb{Q}}$ is commutative.
    \end{enumerate}
\end{proof}

\begin{theorem}\label{theorem:chapter1:division}
    Let $A, B$ be Dedekind cuts such that $B\ne {0}^{*}$. Define the set $A/B$ as the following

    if $A\ge {0}^{*}, B > {0}^{*}$
    \[
        A/B = \left\{ \frac{a}{b} : a\in A\land a\ge 0\land b\in\mathbb{Q}\setminus B \right\}\cup\mathbb{Q}_{-},
    \]

    if $A\le {0}^{*}, B > {0}^{*}$
    \[
        A/B = -(-A)/B,
    \]

    if $A\ge {0}^{*}, B < {0}^{*}$
    \[
        A/B = -A/(-B),
    \]

    if $A\le {0}^{*}, B < {0}^{*}$
    \[
        A/B = (-A)/(-B).
    \]

    Then $A/B$ is a Dedekind cut of $\mathbb{Q}$.
\end{theorem}

\begin{proof}
    It suffices to prove 1st case: $A\supseteq {0}^{*}$ and $B\supset {0}^{*}$.

    \begin{enumerate}[label={\textbf{Case \arabic*.}},itemindent=0.4cm]
        \item $A\ge {0}^{*}, B > {0}^{*}$.
              $A/B$ is a superset of $\mathbb{Q}_{-}$, so $A/B$ is not empty. (DC1) is satisfied.
              \bigskip

              Let $c\in\mathbb{Q}\setminus A$, $d\in B$ such that $d > 0$. Then for all $a\in A, b\in\mathbb{Q}\setminus B$ such that $a \ge 0, b > 0$, we obtain that
              \[
                  \frac{a}{b} < \frac{c}{d}
              \]
              which means $A/B$ is bounded above. (DC2) is satisfied.
              \bigskip

              Let $x\in A/B$. If $x\le 0$, then there exists an element $y\in A/B$ such that $x < y$. For example, zero.

              Otherwise, $x > 0$, then there exists $a\in A, b\in\mathbb{Q}\setminus B$ such that $a > 0, b > 0$ and $x = \dfrac{a}{b}$.

              Since $A$ is a Dedekind cut, then there exists $a_{0}\in A$ such that $a < a_{0}$. So $\dfrac{a}{b} < \dfrac{a_{0}}{b}$.

              According to the definition of $A/B$, $\dfrac{a_{0}}{b}\in A/B$. So $A/B$ does not have a greatest element. (DC3) is satisfied.
              \bigskip

              Let $x\in A/B$, $y$ be a rational number such that $y < x$.

              If $x\le 0$, then $y < 0$. Since $A/B$ is a superset of $\mathbb{Q}_{-}$, then $y\in A/B$.

              Otherwise, $x > 0$, then there exists $a\in A, b\in\mathbb{Q}\setminus B$ such that $a > 0, b > 0$ and $x = \dfrac{a}{b}$.

              Since $y < \dfrac{a}{b}$, then $b\cdot y < a$. According to (DC4), $b\cdot y\in A$.

              If $y\ge 0$, then $y = \dfrac{b\cdot y}{b} \in A/B$. Otherwise, $y < 0$, then $y\in A/B$ since $A/B$ is a superset of $\mathbb{Q}_{-}$.
        \item $A\le {0}^{*}, B > {0}^{*}$.

              According to Case 1, $(-A)/B$ is a Dedekind cut, so $-(-A)/B$ is a Dedekind cut.
        \item $A\ge {0}^{*}, B < {0}^{*}$.

              According to Case 1, $A/(-B)$ is a Dedekind cut, so $-A/(-B)$ is a Dedekind cut.
        \item $A\le {0}^{*}, B < {0}^{*}$.

              According to Case 1, $(-A)/(-B)$ is a Dedekind cut.
    \end{enumerate}

    Thus, $A/B$ is a Dedekind cut.
\end{proof}

(F1) (F2) (F3) (F4) (F5) (F6) (F7) (F8) make $\mathscr{D}_{\mathbb{Q}}$ a commutative ring. Together with the following, $\mathscr{D}_{\mathbb{Q}}$ is a field.

\begin{theorem}\label{theorem:chapter1:multiplicative-inverse}
    \begin{enumerate}[label={(F\arabic*)},start=9]
        \item Each non-zero element of $\mathscr{D}_{\mathbb{Q}}$ has a multiplicative inverse.
    \end{enumerate}
\end{theorem}

\begin{proof}
    Let $A$ be a non-zero element of $\mathscr{D}_{\mathbb{Q}}$. I will show that
    \[
        A\cdot ({1}^{*}/A) = ({1}^{*}/A)\cdot A = {1}^{*}.
    \]

    Since multiplication over $\mathscr{D}_{\mathbb{Q}}$ is commutative, then $A\cdot ({1}^{*}/A) = ({1}^{*}/A)\cdot A$.

    \textbf{Case 1.} $A > {0}^{*}$.
    \[
        \begin{split}
            A\cdot ({1}^{*}/A) & = \left\{ a\cdot\frac{b}{a_{0}} : a\in A\land 0\le b\land b < 1\land a_{0}\in\mathbb{Q}\setminus A\land a\ge 0\land a_{0} > 0 \right\}\cup\mathbb{Q}_{-} \\
            {1}^{*} & = \{ b: b\in\mathbb{Q}\land b < 1 \}
        \end{split}
    \]

    Let $x\in A\cdot ({1}^{*}/A)$.

    If $x\le 0$, then $x\in {1}^{*}$ as well.

    Otherwise, $x > 0$, then there exists $a\in A, b\in {1}^{*}, a_{0}\in \mathbb{Q}\setminus A$ such that $a > 0, b > 0, a_{0} > 0$ and $x = a\cdot\dfrac{b}{a_{0}}$.

    Since $a < a_{0}$ (because $a\in A$ and $a_{0}\in\mathbb{Q}\setminus A$), then
    \[
        x = a\cdot\dfrac{b}{a_{0}} = b\cdot\dfrac{a}{a_{0}} < b < 1.
    \]

    So $x\in {1}^{*}$. Hence $A\cdot ({1}^{*}/A)\subseteq {1}^{*}$.

    \bigskip

    Let $x\in {1}^{*}$.

    If $x\le 0$, then $x\in A\cdot ({1}^{*}/A)$ as well.

    Otherwise, $x > 0$. Then $0 < x < 1$.

    I will show that there exists an integer $n$ such that $x^{n}\in A$ and $x^{n-1}\notin A$.

    Let's consider the following sets:
    \[
        \begin{split}
            S_{1} & = \{ {x}^{n} : n\in\mathbb{Z} \} \cap A, \\
            S_{2} & = \{ {x}^{n} : n\in\mathbb{Z} \} \cap \mathbb{Q}\setminus A, \\
            S & = S_{1}\cup S_{2}
        \end{split}
    \]
    \begin{enumerate}[label={\textbf{Step \arabic*.}},itemindent=1cm]
        \item Prove that $S_{1}$ and $S_{2}$ are not empty.

              Let $y$ be a positive rational number. I will find a natural number $m$ ($m\ge 1$) such that $x^{-m} > y$.

              Due to binomial theorem
              \[
                  x^{-m} = {\left(1 + \left(\frac{1}{x} - 1\right)\right)}^{m} = \sum^{m}_{k=1}\binom{m}{k}{\left(\frac{1}{x} - 1\right)}^{k} \ge 1 + m\cdot\left(\frac{1}{x} - 1\right)
              \]
              So for any natural number $m$ that satisfies $m > \dfrac{y - 1}{\dfrac{1}{x} - 1}$, we will obtain that $x^{-m} > y$.

              Let $a_{1}\in A$ and $a_{2}\in\mathbb{Q}\setminus A$ such that $a_{1} > 0$ and $a_{2} > 0$.

              According to the result that we have just proved, there exists natural numbers $n_{1}$ and $n_{2}$ such that
              \[
                  x^{-{n}_{2}} > a_{2}\qquad\text{and}\qquad x^{-{n}_{1}} > \frac{1}{a_{1}}
              \]

              Hence $S_{1}$ and $S_{2}$ are not empty.

        \item Prove that there exists an integer $n$ such that $x^{n}\in A$ and $x^{n-1}\notin A$.

              Since $0 < x < 1$, then ${x}^{n} < {x}^{n-1}$.

              $\{ {x}^{n} : n\in\mathbb{Z}\land {x}^{n}\in A \}$ is bounded above, then $\{ n : n\in\mathbb{Z}\land {x}^{n}\in A \}$ is bounded below. According to the well-ordering principle, $\{ n : n\in\mathbb{Z}\land {x}^{n}\in A \}$ contains a least element.

              Hence, there exists an integer $n$ such that ${x}^{n}\in A$ and ${x}^{n-1}\notin A$.
    \end{enumerate}

    \[
        x = {x^{n}} / {x^{n-1}}.
    \]

    According to (DC3), there exists $x_{0}\in A$ such that $x^{n} < x_{0}$.
    \[
        x = x_{0}\cdot\frac{x^{n}/x_{0}}{x^{n-1}}.
    \]

    Since $x_{0}\in A, x^{n-1}\in\mathbb{Q}\setminus A, 0 < x^{n}/x_{0} < 1$, then $x\in A\cdot ({1}^{*}/A)$.

    Hence ${1}^{*}\subseteq A\cdot ({1}^{*}/A)$.

    \bigskip
    \textbf{Case 2.} $A < {0}^{*}$.

    According to the definition of division
    \[
        {1}^{*}/A = -{1}^{*}/(-A)
    \]

    Apply Theorem~\ref{theorem:chapter1:multiplication-and-negation} and Case 1
    \[
        A\cdot ({1}^{*}/A) = (-(-A))\cdot (-{1}^{*}/(-A)) = (-A)\cdot ({1}^{*}/(-A)) = {1}^{*}
    \]

    Thus, $A\cdot ({1}^{*}/A) = ({1}^{*}/A)\cdot A = {1}^{*}$.
\end{proof}

\begin{theorem}
    Let $A$ be a Dedekind cut of $\mathbb{Q}$ and $A\ne {0}^{*}$, then
    \[
        {1}^{*}/({1}^{*}/A) = A.
    \]
\end{theorem}

\begin{proof}
    Theorem~\ref{theorem:chapter1:multiplicative-inverse} asserts that there exists a Dedekind cut $B$ such that $A\cdot B = B\cdot A = {1}^{*}$.

    Suppose that $B$ and $C$ are two Dedekind cuts such that
    \[
        \begin{split}
            A\cdot B = B\cdot A = {1}^{*}, \\
            A\cdot C = C\cdot A = {1}^{*}.
        \end{split}
    \]

    Apply (F5) and (F7)
    \[
        B = B\cdot {1}^{*} = B\cdot (A\cdot C) = (B\cdot A)\cdot C = {1}^{*}\cdot C = C.
    \]

    Hence there exists a unique Dedekind cut $B$ such that $A\cdot B = B\cdot A = {1}^{*}$.

    Since this uniqueness, we call $B$ \textit{the multiplicative inverse} of $A$.

    According to Theorem~\ref{theorem:chapter1:multiplicative-inverse}
    \[
        A\cdot ({1}^{*}/A) = ({1}^{*}/A)\cdot A = {1}^{*}.
    \]

    Therefore, $A$ is the multiplicative inverse of ${1}^{*}/A$.

    Thus ${1}^{*}/({1}^{*}/A) = A$.
\end{proof}

\begin{theorem}
    $\mathscr{D}_{\mathbb{Q}}$ is a field and it is characteristic zero.
\end{theorem}

\begin{proof}
    (F1) (F2) (F3) (F4) (F5) (F6) (F7) (F8) (F9) together assert that $\mathscr{D}_{\mathbb{Q}}$ is a field.

    Since ${0}^{*} < {1}^{*}$, then for all positive integer $n$
    \[
        \underbrace{{1}^{*} + \cdots + {1}^{*}}_{n} > {0}^{*}.
    \]
    Therefore, $\mathscr{D}_{\mathbb{Q}}$ is characteristic zero.
\end{proof}

\subsection{Least-upper-bound property}

So far, we prove that the set of Dedekind cuts with addition and multiplication is a totally ordered field. But those properties make no distinction between $\mathscr{D}_{\mathbb{Q}}$ and $\mathbb{Q}$. The property that makes distinction is the least-upper-bound property.

\begin{theorem}
    Let $\mathcal{D}$ be a set of Dedekind cuts of $\mathbb{Q}$ and $\mathcal{D}$ has an upper bound.

    Then $\mathcal{D}$ has a least upper bound.
\end{theorem}

\begin{proof}
    Let $S = \bigcup\limits_{A\in\mathcal{D}} A$. $S$ consists of rational numbers.

    \textbf{Step 1. Prove that $S$ is a Dedekind cut of $\mathbb{Q}$.}

    $S$ is non-empty by definition. (DC1) is satisfied.

    $S$ is bounded above. (DC2) is satisfied.

    Let $a\in S$. Then there exists $A\in\mathcal{D}$ such that $a\in A$. Since $A$ has no greatest element, then there exists $x_{0}\in A$ such that $x < x_{0}$. $x_{0}\in A$ then $x_{0}\in S$. (DC3) is satisfied.

    Let $a\in S$, and $b$ be a rational number such that $b < a$. Since $a\in S$, then there exists $A\in\mathcal{D}$ such that $a\in A$. Since $b < a$, then $b\in A$, so $b\in S$ also. (DC4) is satisfied.

    Hence $S$ is a Dedekind cut of $\mathbb{Q}$.
    \bigskip

    \textbf{Step 2. Prove that $S$ is the least upper bound of $\mathcal{D}$.}

    Let $A\in\mathcal{D}$. Since $A\subseteq S$, then $S$ is an upper bound of $\mathcal{D}$.

    Let $S'$ be an upper bound of $\mathcal{D}$. According to the definition of upper bound, then $\forall A (A\in\mathcal{D}\rightarrow A\subseteq S')$. So $S\subseteq S'$, or equivalently, $S\le S'$. Hence $S$ is the least upper bound of $\mathcal{D}$.

    Thus $\mathcal{D}$ has a least upper bound.
\end{proof}

Now we have proved that $\mathscr{D}_{\mathbb{Q}}$ satisfies all real numbers axioms.

\subsection{Embed $\mathbb{Q}$ into $\mathscr{D}_{\mathbb{Q}}$}

I don't feel so good after proving that $\mathscr{D}_{\mathbb{Q}}$ is a totally ordered field. It is because I haven't showed that the addition and multiplication in $\mathscr{D}_{\mathbb{Q}}$ are ``the same'' as addition and multiplication $\mathbb{Q}$.

\begin{theorem}
    There exists a ring monomorphism $\iota$
    \[
        \begin{split}
            \iota:\quad & \mathbb{Q}\to\mathscr{D}_{\mathbb{Q}}, \\
            & q\mapsto {q}^{*}
        \end{split}
    \]
    which preserves order.
\end{theorem}

\begin{proof}
    We define a mapping $\iota$ as the following:
    \[
        \begin{split}
            \iota:\quad & \mathbb{Q}\to\mathscr{D}_{\mathbb{Q}}, \\
            & q\mapsto \{ x : x\in\mathbb{Q}\land x < q \}.
        \end{split}
    \]

    Let $q_{1}$ and $q_{2}$ be two rational numbers.
    \begin{itemize}
        \item If $q_{1} = q_{2}$, then $\iota(q_{1}) = \iota(q_{2})$.

              Otherwise, $q_{1}\ne q_{2}$. Without loss of generality, suppose that $q_{1} < q_{2}$.

              Since every element of $\iota(q_{1})$ is less than $q_{1}$, then according to (DC4), every element of $\iota(q_{1})$ is an element of $\iota(q_{2})$. Therefore $\iota(q_{1})\subseteq\iota(q_{2})$.

              On the other hand, $q_{1}\notin\iota(q_{1})$ and $q_{1}\in\iota(q_{2})$. Therefore $\iota(q_{1})$ is a proper subset of $\iota(q_{2})$.

              Hence $\iota$ preserves order and $\iota$ is injective.
        \item To prove that $\iota$ is a monomorphism, I will show that
              \[
                  \begin{split}
                      \iota(q_{1} + q_{2}) = \iota(q_{1}) + \iota(q_{2}) \\
                      \iota(q_{1}\cdot q_{2}) = \iota(q_{1})\cdot\iota(q_{2}).
                  \end{split}
              \]
              \begin{itemize}
                  \item Prove that $\iota(q_{1} + q_{2}) = \iota(q_{1}) + \iota(q_{2})$.

                        Let $x\in\iota(q_{1}) + \iota(q_{2})$.

                        According to the definition of addition, there exists $y\in\iota(q_{1})$ and $z\in\iota(q_{2})$ such that $y + z = x$.

                        On the other hand
                        \[
                            x = y + z < q_{1} + q_{2}
                        \]
                        So $x\in\iota(q_{1} + q_{2})$. Therefore, $\iota(q_{1}) + \iota(q_{2})\le\iota(q_{1} + q_{2})$.
                        \bigskip

                        Let $x\in\iota(q_{1} + q_{2})$, then $x < q_{1} + q_{2}$.

                        Choose $y = q_{1} + \dfrac{x - (q_{1} + q_{2})}{2}$ and $z = q_{2} + \dfrac{x - (q_{1} + q_{2})}{2}$, then $y + z = x$.

                        Since $x < q_{1} + q_{2}$, then $y < q_{1}$ and $z < q_{2}$. $y$ and $z$ are also rational numbers, so $y\in\iota(q_{1})$ and $z\in\iota(q_{2})$.

                        Therefore, $\iota(q_{1} + q_{2})\le\iota(q_{1}) + \iota(q_{2})$.

                        Thus $\iota(q_{1}) + \iota(q_{2}) = \iota(q_{1} + q_{2})$.
                  \item Prove that $\iota(-q) = - \iota(q)$.
                        \[
                            \iota(-q) = \iota(0) - \iota(q) = {0}^{*} - \iota(q) = -\iota(q).
                        \]
                  \item Prove that $\iota(q_{1}\cdot q_{2}) = \iota(q_{1})\cdot\iota(q_{2})$.

                        If $q_{1} = 0$ or $q_{2} = 0$, then $\iota(q_{1})\cdot\iota(q_{2}) = {0}^{*} = \iota(0) = \iota(q_{1}\cdot q_{2})$.

                        Otherwise, $q_{1}\ne 0$ and $q_{2}\ne 0$.

                        \textbf{Case 1.} $q_{1} > 0$ and $q_{2} > 0$.
                        \[
                            \begin{split}
                                \iota(q_{1})\cdot\iota(q_{2}) & = \{ a\cdot b : a\ge 0\land a < q_{1}\land b\ge 0\land b < q_{2} \}\cup\mathbb{Q}_{-}, \\
                                \iota(q_{1}\cdot\iota(q_{2})) & = \{ x: x\in\mathbb{Q}\land x < q_{1}\cdot q_{2} \}.
                            \end{split}
                        \]
                        Then $\iota(q_{1})\cdot\iota(q_{2})\subseteq\iota(q_{1}\cdot q_{2})$.

                        Let $x\in\iota(q_{1}\cdot q_{2})$.

                        If $x\le 0$, then $x\in\iota(q_{1})\cdot\iota(q_{2})$.

                        Otherwise, $0 < x < q_{1}\cdot q_{2}$. Choose $y$ such that $\dfrac{x}{q_{1}\cdot q_{2}} < y < 1$, then
                        \[
                            \dfrac{x}{q_{1}q_{2}} = \underbrace{y}_{< 1}\cdot\underbrace{\dfrac{x}{q_{1}\cdot q_{2}\cdot y}}_{< 1}
                        \]
                        Let $a = q_{1}\cdot y$ and $b = q_{2}\cdot\dfrac{x}{q_{1}\cdot q_{2}\cdot y}$, then $a\in\iota(q_{1}), b\in\iota(q_{2})$, and $a\cdot b = x$. This means $x\in\iota(q_{1})\cdot\iota(q_{2})$.

                        So $\iota(q_{1}\cdot q_{2})\subseteq\iota(q_{1})\cdot\iota(q_{2})$.

                        Hence $\iota(q_{1})\cdot\iota(q_{2}) = \iota(q_{1}\cdot q_{2})$.

                        \textbf{Case 2.} $q_{1} < 0$ and $q_{2} < 0$.
                        \begin{align*}
                            \iota(q_{1}\cdot q_{2}) & = \iota((-q_{1})\cdot (-q_{2}))        \\
                                                    & = \iota(-q_{1})\cdot\iota(-q_{2})      \\
                                                    & = (-\iota(q_{1}))\cdot (-\iota(q_{2})) \\
                                                    & = \iota(q_{1})\cdot\iota(q_{2}).
                        \end{align*}
                        \textbf{Case 3.} $q_{1} > 0$ and $q_{2} < 0$.
                        \begin{align*}
                            \iota(q_{1}\cdot q_{2}) & = -\iota(-q_{1}\cdot q_{2})         \\
                                                    & = -\iota(q_{1}\cdot (-q_{2}))       \\
                                                    & = -(\iota(q_{1})\cdot\iota(-q_{2})) \\
                                                    & = \iota(q_{1})\cdot\iota(q_{2}).
                        \end{align*}
                        \textbf{Case 4.} $q_{1} < 0$ and $q_{2} > 0$.
                        \begin{align*}
                            \iota(q_{1}\cdot q_{2}) & = -\iota(-q_{1}\cdot q_{2})         \\
                                                    & = -\iota((-q_{1})\cdot q_{2})       \\
                                                    & = -(\iota(-q_{1})\cdot\iota(q_{2})) \\
                                                    & = \iota(q_{1})\cdot\iota(q_{2}).
                        \end{align*}
                        Thus, $\iota(q_{1}\cdot q_{2}) = \iota(q_{1})\cdot\iota(q_{2})$.
              \end{itemize}
    \end{itemize}

    We have constructed a ring monomorphism $\iota:\mathbb{Q}\to\mathscr{D}_{\mathbb{Q}}$ that preserves order.
\end{proof}

\subsection{Completeness (Dedekind completeness)}

So far, we have successfully constructed a model (by using Dedekind cuts), which satisfies the axioms of the real numbers.

In this subsection, we will show that $\mathbb{R}$ is complete, in the sense that every cut is represented by a real number. To do so, we will use a generalization of Dedekind cut.

\begin{definition}[Dedekind cut]
    Let $S$ be a totally ordered set. A Dedekind cut of $S$ is a partition $(L, R)$ ($L\cap R = \varnothing$ and $L\cup R = S$) such that
    \begin{enumerate}[label={(\roman*)}]
        \item $L$ is not empty.
        \item $R$ is not empty (i.e. $L\ne S$).
        \item $L$ does not contain a greatest element.
        \item $L$ is closed downward.
    \end{enumerate}
\end{definition}

Instead of denoting $(L, R)$, we can use $L$ for short. This is totally fine because $R = S\setminus L$.

\begin{definition}[Dedekind-complete]
    A totally order set $S$ is called Dedekind-complete if every Dedekind cut $A$ of it satisfies: \textit{$S\setminus A$ has a least element.}
\end{definition}

\begin{theorem}
    $\mathbb{R}$ is Dedekind-complete.
\end{theorem}

\begin{proof}
    Let $(L, R)$ be a Dedekind cut of $\mathbb{R}$.

    $L$ is bounded above. According to the least-upper-bound property, $L$ has a least upper bound $\ell$.

    On the other hand, $L$ does not contain a greatest element, then $\ell\in R$.

    Since every element of $R$ is an upper bound of $L$, then $\ell$ is the least element of $R$.

    Therefore, every Dedekind cut $(L, R)$ of $\mathbb{R}$, $R$ has a least element. Thus, $\mathbb{R}$ is Dedekind-complete.
\end{proof}

\section{Construction of the real numbers by Cauchy sequences}

Another way to construct the real numbers from the rational numbers is using Cauchy sequences.

In the upcoming, we consider sequences with only rational terms.

\subsection{Definition of Cauchy sequence}

In this subsection and the upcoming three subsections, we work with Cauchy sequences which contain only rational numbers.

\begin{definition}
    A sequence $\seq{a_{n}}$ is called a Cauchy sequence if and only if
    \begin{quotation}
        \noindent for every positive rational number $\varepsilon$, there exists a natural number $N$ which is a function of $\varepsilon$ such that for all natural numbers $m, n$ which are greater than $N$, $\left\vert a_{m} - a_{n}\right\vert < \varepsilon$.
    \end{quotation}

    Equivalently
    \[
        (\forall\varepsilon > 0)(\exists N=N(\varepsilon))(\forall m, n > N)\left(\abs{a_{m} - a_{n}} < \varepsilon\right),
    \]

    or
    \[
        (\forall\varepsilon > 0)(\exists N=N(\varepsilon))(\forall n > N)(\forall p)\left(\abs{a_{n+p} - a_{n}} < \varepsilon\right).
    \]
\end{definition}

Informally, the differences of any two terms can be arbitrarily small from a certain index.

\bigskip

Such sequences do exist. For example: $\seq{a_{n}}$, where $a_{n} = q$ ($q$ is a given rational number) for every natural number $n$.

A sequence $\seq{b_{n}}$ is \textit{not a Cauchy sequence} if and only if
\[
    (\exists\varepsilon > 0)(\forall N)(\exists m, n > N)(\abs{b_{m} - b_{n}}\ge\varepsilon).
\]

\subsection{Properties of Cauchy sequences}

\begin{definition}
    We say $\seq{a_{n}}$ converges to a rational number $a$ if and only if
    \[
        (\forall\varepsilon > 0)(\exists N=N(\varepsilon))(\forall n > N)(\abs{a_{n} - a} < \varepsilon).
    \]

    Then we say $a$ is a limit point of $\seq{a_{n}}$.
\end{definition}

\begin{theorem}
    If $\seq{a_{n}}$ converges to a rational number $a$, then $\seq{a_{n}}$ is a Cauchy sequence.
\end{theorem}

\begin{proof}
    Let $\varepsilon$ be a positive rational number.

    Since $\seq{a_{n}}$ converges to $a$, there exists a natural number $N = N(\varepsilon)$ such that
    \[
        (\forall n > N)\left(\abs{a_{n} - a} < \frac{\varepsilon}{2}\right).
    \]

    Hence, for all $m, n > N$
    \[
        \abs{a_{m} - a_{n}} = \abs{(a_{m} - a) + (a - a_{n})} \le \abs{a_{m} - a} + \abs{a_{n} - a} < \frac{\varepsilon}{2} + \frac{\varepsilon}{2} = \varepsilon.
    \]

    Thus, $\seq{a_{n}}$ is a Cauchy sequence.
\end{proof}

A Cauchy sequence can converge to a rational number or not.

\begin{example}
    $\seq{a_{n}}$ where $a_{n} = q$, $q$ is a rational number converges to a rational number.
\end{example}

\begin{proof}
    For every rational number $\varepsilon > 0$, let's choose $N = 1$, then $\forall n > N$, $\abs{a_{n} - q} = 0 < \varepsilon$.

    Therefore, $\seq{a_{n}}$ converges to $q$.
\end{proof}

\begin{theorem}
    If a Cauchy sequence converges, then it has a unique limit point.
\end{theorem}

\begin{proof}
    Let $\seq{a_{n}}$ be a Cauchy sequence.

    Assume that two rational numbers $a$ and $b$ are limit points of $\seq{a_{n}}$ and $a\ne b$.

    Pick a positive rational number $\varepsilon$ such that $\abs{a - b}\ge\varepsilon$.
    \begin{itemize}
        \item there exists $N_{1} = N_{1}(\varepsilon)$ such that for all $n > N_{1}$, $\abs{a_{n} - a} < \dfrac{\varepsilon}{2}$,
        \item there exists $N_{2} = N_{2}(\varepsilon)$ such that for all $n > N_{2}$, $\abs{a_{n} - b} < \dfrac{\varepsilon}{2}$.
    \end{itemize}

    Choose $N = \max\{ N_{1}, N_{2} \}$, then for all $n > N$
    \[
        \abs{a - b}\le \abs{a - a_{n}} + \abs{a_{n} - b} < \frac{\varepsilon}{2} + \frac{\varepsilon}{2} = \varepsilon.
    \]

    $\abs{a - b} < \varepsilon$ contradicts $\abs{a - b}\ge\varepsilon$.

    Hence $a = b$.

    Thus if a Cauchy sequence converges, it has a unique limit point.
\end{proof}

\begin{theorem}
    Every Cauchy sequence is bounded.
\end{theorem}

\begin{proof}
    Let $\seq{a_{n}}$ be a Cauchy sequence.

    Then there exists a natural number $N$ such that for all $m, n > N$, $\abs{a_{m} - a_{n}} < 1$.

    For $n > N$
    \[
        \abs{a_{n}} = \abs{a_{n} - a_{N+1} + a_{N+1}}\le \abs{a_{n} - a_{N+1}} + \abs{a_{N+1}} < 1 + \abs{a_{N+1}}
    \]

    For every $n$
    \[
        \abs{a_{n}}\le \max\{ \abs{a_{1}}, \abs{a_{2}}, \ldots, \abs{a_{N}}, 1 + \abs{a_{N+1}} \}.
    \]

    Thus, $\seq{a_{n}}$ is bounded.
\end{proof}

\subsection{Equivalent Cauchy sequences}

\begin{definition}[Equivalent (Cauchy) sequences]
    Two (Cauchy) sequences $\seq{a_{n}}, \seq{b_{n}}$ are called equivalent if and only if
    \[
        (\forall\varepsilon > 0)(\exists N=N(\varepsilon))(\forall n > N)(\left\vert a_{n} - b_{n} \right\vert < \varepsilon).
    \]
\end{definition}

\begin{theorem}
    The relation in the previous definition is an equivalence relation. In other words, this relation is reflexive, symmetric, transitive.
\end{theorem}

\begin{proof}
    \begin{itemize}
        \item Reflexive.

              A Cauchy sequence $\seq{a_{n}}$ is equivalent to itself, since
              \[
                  (\forall\varepsilon > 0)(\forall n > 1)(\left\vert a_{n} - a_{n} \right\vert = 0 < \varepsilon)
              \]
        \item Symmetric.

              Let $\seq{a_{n}}$ and $\seq{b_{n}}$ be two Cauchy sequences.

              $\seq{a_{n}}$ is equivalent to $\seq{b_{n}}$ if and only if
              \[
                  \begin{split}
                      & (\forall\varepsilon > 0)(\exists N=N(\varepsilon))(\forall n > N)(\left\vert a_{n} - b_{n} \right\vert < \varepsilon) \\
                      \Leftrightarrow\quad & (\forall\varepsilon > 0)(\exists N=N(\varepsilon))(\forall n > N)(\left\vert b_{n} - a_{n} \right\vert < \varepsilon)
                  \end{split}
              \]

              which implies that $\seq{b_{n}}$ is equivalent to $\seq{a_{n}}$.
        \item Transitive.

              Assume that among three Cauchy sequences $\seq{a_{n}}$, $\seq{b_{n}}$, $\seq{c_{n}}$, $\seq{a_{n}}$ is equivalent to $\seq{b_{n}}$ and $\seq{b_{n}}$ is equivalent to $\seq{c_{n}}$.

              Let $\varepsilon$ be an arbitrary positive rational number.
              \[
                  \begin{split}
                      (\forall\varepsilon > 0)(\exists N_{1}=N_{1}(\varepsilon))(\forall n > N_{1})\left(\left\vert a_{n} - b_{n} \right\vert < \frac{\varepsilon}{2}\right), \\
                      (\forall\varepsilon > 0)(\exists N_{2}=N_{2}(\varepsilon))(\forall n > N_{2})\left(\left\vert b_{n} - c_{n} \right\vert < \frac{\varepsilon}{2}\right).
                  \end{split}
              \]

              Let's choose $N = \max\{ N_{1}, N_{2} \}$, then for all $n > N$
              \[
                  \abs{a_{n} - c_{n}} \le \abs{a_{n} - b_{n}} + \abs{b_{n} - c_{n}} < \frac{\varepsilon}{2} + \frac{\varepsilon}{2} = \varepsilon.
              \]

              Hence
              \[
                  (\forall\varepsilon > 0)(\forall n > N)(\abs{a_{n} - c_{n}} < \varepsilon).
              \]

              Thus, $\seq{a_{n}}$ and $\seq{c_{n}}$ are equivalent.
    \end{itemize}
\end{proof}

\begin{theorem}
    If the Cauchy sequence $\seq{a_{n}}$ converges to $a$, then every Cauchy sequence which is equivalent to $\seq{a_{n}}$ also converges to $a$.
\end{theorem}

\begin{proof}
    Let $\seq{a_{n}}$ and $\seq{b_{n}}$ be equivalent Cauchy sequences.

    Let $\varepsilon$ be a positive rational number.

    $\seq{a_{n}}$ converges to $a$ means
    \[
        (\exists N_{1}=N_{1}(\varepsilon))(\forall n > N_{1})\left(\abs{a_{n} - a} < \frac{\varepsilon}{2}\right)
    \]

    Since $\seq{a_{n}}$ and $\seq{b_{n}}$ are equivalent
    \[
        (\exists N_{2}=N_{2}(\varepsilon))(\forall n > N_{2})\left(\abs{b_{n} - a_{n}} < \frac{\varepsilon}{2}\right)
    \]

    Let $N = \max\{ N_{1}, N_{2} \}$, then
    \[
        (\forall n > N)\left(\abs{b_{n} - a} \le \abs{b_{n} - a_{n}} + \abs{a_{n} - a} < \frac{\varepsilon}{2} + \frac{\varepsilon}{2} = \varepsilon\right)
    \]

    So $\seq{b_{n}}$ converges to $a$.

    Thus, every Cauchy sequence which is equivalent to $\seq{a_{n}}$ converges to $a$.
\end{proof}

\subsection{A Cauchy sequence which does not converge}

\begin{theorem}
    If the Cauchy sequence $\seq{a_{n}}$ does not converge to any rational number, then every Cauchy sequence which is equivalent to $\seq{a_{n}}$ does not converge to any rational number, either.
\end{theorem}

\begin{proof}
    Let $\seq{a_{n}}$ and $\seq{b_{n}}$ be equivalent Cauchy sequences.

    Assume that $\seq{b_{n}}$ converges to a rational number. Then according to the previous theorem, $\seq{a_{n}}$ converges to the same rational number, which is a contradiction. So $\seq{b_{n}}$ does not converge to a ratioanl number.

    Thus, every Cauchy sequence which is equivalent to $\seq{a_{n}}$ does not converge to any rational number.
\end{proof}

\begin{example}
    $\seq{b_{n}}$ which is defined by
    \[
        b_{n} =
        \begin{cases}
            2                                       & \text{if $n = 1$} \\
            \dfrac{b_{n-1}}{2} + \dfrac{1}{b_{n-1}} & \text{if $n > 1$}
        \end{cases}
    \]

    does not converge to a rational number.
\end{example}

\begin{proof}
    \begin{enumerate}[label={\textbf{Step \arabic*.}},itemindent=1cm]
        \item ${b^{2}_{n}} > 2$.

              For $n = 1$, $b_{1} = 2$, so $b^{2}_{1} = 4 > 2$.

              For $n > 1$
              \begin{align*}
                  {b^{2}_{n}} - 2 & = {\left(\frac{b_{n-1}}{2} + \frac{1}{b_{n-1}}\right)}^{2} - 2         \\
                                  & = {\left(\frac{b^{2}_{n-1}}{4} + \frac{1}{b^{2}_{n-1}} + 1\right)} - 2 \\
                                  & = \frac{b^{2}_{n-1}}{4} + \frac{1}{b^{2}_{n-1}} - 1                    \\
                                  & = {\left(\frac{b_{n-1}}{2} - \frac{1}{b_{n-1}}\right)}^{2} > 0
              \end{align*}

              Hence $b^{2}_{n} > 2$.
        \item Prove that $\seq{b_{n}}$ is a decreasing sequence.

              For every $n$, ${b^{2}_{n}}$ is greater than $2$.

              For $n > 1$
              \[
                  b_{n} - b_{n-1} = \left(\frac{b_{n-1}}{2} + \frac{1}{b_{n-1}}\right) - b_{n-1} = \frac{1}{b_{n-1}} - \frac{b_{n-1}}{2} = \frac{2 - {b}^{2}_{n-1}}{2\cdot b_{n}} < 0
              \]

              Then $\seq{b_{n}}$ is a decreasing sequence.
        \item Prove that $\seq{b_{n}}$ is a Cauchy sequence.

              \begin{align*}
                  \abs{b_{n+1} - b_{n}} & = \abs{\frac{b_{n}}{2} + \frac{1}{b_{n}} - \frac{b_{n-1}}{2} - \frac{1}{b_{n-1}}}   \\
                                        & = \abs{\frac{b_{n} - b_{n-1}}{2} - \frac{b_{n} - b_{n-1}}{b_{n}b_{n-1}}}            \\
                                        & = \frac{1}{2}\abs{b_{n} - b_{n-1}}\cdot\abs{1 - \frac{2}{b_{n}b_{n-1}}}             \\
                                        & = \frac{1}{2}\abs{b_{n} - b_{n-1}}\cdot\frac{\abs{b_{n}b_{n-1} - 2}}{b_{n}b_{n-1}}.
              \end{align*}

              Since ${b_{n}}^{2} > 2$ for all $n$, then
              \[
                  b_{n}b_{n-1} - 2 = \frac{{b_{n}}^{2}{b_{n-1}}^{2} - 4}{b_{n}b_{n-1} + 2} > 0.
              \]

              So that $\dfrac{1}{b_{n}b_{n-1}} < \dfrac{1}{2}$, and
              \[
                  \abs{b_{n+1} - b_{n}} = \frac{1}{2}\abs{b_{n} - b_{n-1}}\cdot\frac{b_{n}b_{n-1} - 2}{b_{n}b_{n-1}} < \frac{1}{4}\abs{b_{n} - b_{n-1}}({b_{1}}^{2} - 2) = \frac{1}{2}\abs{b_{n} - b_{n-1}}
              \]

              Recursively
              \[
                  \abs{b_{n+1} - b_{n}} < \frac{1}{{2}^{n-1}}\abs{b_{2} - b_{1}} = \frac{1}{2^{n}}.
              \]

              For natural number $p$
              \begin{align*}
                  \abs{b_{n+p} - b_{n}} & \le \abs{b_{n+p} - b_{n+p-1}} + \cdots + \abs{b_{n+1} - b_{n}}                              \\
                                        & = \frac{1}{{2}^{n+p-1}} + \cdots + \frac{1}{{2}^{n}}                                        \\
                                        & = \frac{1}{{2}^{n}}\left( \frac{1}{{2}^{p-1}} + \cdots + 1 \right)                          \\
                                        & = \frac{1}{{2}^{n}}\cdot\frac{1 - \frac{1}{{2}^{p}}}{1 - \frac{1}{2}} < \frac{1}{{2}^{n-1}}
              \end{align*}

              For every rational number $\varepsilon > 0$, for all $n > N(\varepsilon) = \ceiling{\dfrac{1}{\varepsilon}}$ and for all $p > 0$
              \[
                  \abs{b_{n+p} - b_{n}} < \frac{1}{{2}^{n-1}} \le \frac{1}{n} < \varepsilon
              \]

              Hence $\seq{b_{n}}$ is a Cauchy sequence.
    \end{enumerate}

    \bigskip
    Assume that $\seq{b_{n}}$ converges to a rational number $b$.

    \begin{enumerate}[label={\textbf{Step \arabic*.}},start=4,itemindent=1cm]
        \item Prove that $b$ is smaller than all terms of $\seq{b_{n}}$.

              Assume that there exists a natural number $n_{1}$ such that $b_{n_{1}}\le b$. Since $\seq{b_{n}}$ is decreasing, then $b_{n} < b_{n_{1}}\le b$ for every $n > n_{1}$.

              Choose $\varepsilon = b - b_{n_{1} + 1}$, then for all $n > n_{1}$, $\abs{b_{n} - b} = b - b_{n} \ge b - b_{n_{1} + 1} = \varepsilon$. Hence, $\seq{b_{n}}$ does not converge to $b$, this contradicts the initial assumption.

              Therefore, $b < b_{n}$, for every natural number $n$.
        \item Prove that $0 < b < 2$.

              Since $b$ is smaller than all terms of $\seq{b_{n}}$, then $b < b_{1} = 2$.

              Suppose that $b\le 0$.

              Choose $\varepsilon = 1$, then for all $n$, $\abs{b_{n} - b} = b_{n} - b \ge b_{n} > \varepsilon$. This means $\seq{b_{n}}$ does not converge to $b$.

              Therefore, $b > 0$.
        \item Prove that ${b}^{2} = 2$.

              Let's consider another sequence $\seq{x_{n}}$, where $x_{n} = {b^{2}_{n}}$.

              Since $b_{n}$ converges to $b$, then
              \[
                  (\forall\varepsilon > 0)(\exists N=N(\varepsilon))(\forall n > N)\left(\abs{b_{n} - b} < \frac{\varepsilon}{4}\right)
              \]

              It follows that
              \[
                  (\forall\varepsilon > 0)(\exists N=N(\varepsilon))(\forall n > N)\left(\abs{b^{2}_{n} - b^{2}} < \abs{b_{n} - b}\cdot\abs{b_{n} + b} < \frac{\varepsilon}{4}\cdot 4 = \varepsilon \right)
              \]

              So $\seq{x_{n}}$ converges to $b^{2}$.

              For $n > 2$
              \begin{align*}
                  {b_{n}}^{2} - 2 & = \frac{{\left( {b_{n-1}}^{2} - 2 \right)}^{2}}{4\cdot b^{2}_{n-1}}                               \\
                                  & \le \frac{({b_{n-1}}^{2} - 2)\cdot ({b_{1}}^{2} - 2)}{4\cdot b^{2}_{n-1}}                         \\
                                  & \le \frac{{b_{n-1}}^{2} - 2}{2\cdot {b_{n-1}}^{2}}                                                \\
                                  & < \frac{{b_{n-1}}^{2} - 2}{4}                                                                     \\
                                  & \le \frac{{b_{1}}^{2} - 2}{4^{n-1}} = \frac{2}{4^{n-1}} = \frac{1}{2^{2n-3}} \le \frac{1}{2^{n}}.
              \end{align*}

              \[
                  (\forall\varepsilon > 0)\left(\forall n > N(\varepsilon) = \abs{\floor{\frac{1}{\varepsilon} - 1}}\right)\left( \frac{1}{2^{n}} < \varepsilon \right).
              \]

              So for every $\varepsilon > 0$, choose $N = \abs{\floor{\dfrac{1}{\varepsilon} - 1}} + 3$ (I plus 3 to make sure $N > 2$), then for all $n > N$, $\abs{{b_{n}}^{2} - 2} < \dfrac{1}{2^{n}} < \varepsilon$.

              Therefore, $\seq{x_{n}}$ converges to $2$.

              Hence $b^{2} = 2$.
    \end{enumerate}

    $b^{2} = 2$ contradicts that there is no rational number whose square equals $2$.

    Hence the initial assumption ($b$ is a rational number) is false. Therefore, $\seq{b_{n}}$ does not converge to a rational number.
\end{proof}

The equivalence relation amongst Cauchy sequences partitions the set of all Cauchy sequences into \textit{equivalance classes}. In the upcoming subsections, I will prove that these equivalence classes satisfy the axioms of the real numbers. Firstly, I have to define the order relation, operations (addition, multiplication) on them.

Let's denote the set of all equivalance classes of Cauchy sequences (with only rational terms) by $\mathscr{C}_{\mathbb{Q}}$.

\subsection{Field structure}

In this subsection, we will define addition and multiplication over Cauchy sequences and classes of Cauchy sequences. Then we will prove that $\mathscr{C}_{\mathbb{Q}}$ is a field.

\subsubsection*{Addition}

\begin{theorem}[Sum of two Cauchy sequences]\label{theorem:chapter1:sum-of-cauchy-sequences}
    Let $\seq{a_{n}}$, $\seq{b_{n}}$ be Cauchy sequences, then $\seq{c_{n}}$ where $c_{n} = a_{n} + b_{n}$ is a Cauchy sequence.
\end{theorem}

\begin{proof}
    Since $\seq{a_{n}}$ and $\seq{b_{n}}$ are Cauchy sequences, then for every positive rational number $\varepsilon$
    \begin{itemize}[itemsep=0pt]
        \item there exists $N_{1} = N_{1}(\varepsilon)$ such that for all $m, n > N_{1}$, $\abs{a_{m} - a_{n}} < \dfrac{\varepsilon}{2}$,
        \item there exists $N_{2} = N_{2}(\varepsilon)$ such that for all $m, n > N_{2}$, $\abs{b_{m} - b_{n}} < \dfrac{\varepsilon}{2}$.
    \end{itemize}

    Hence, for all $m, n > N = \max\{N_{1}, N_{2}\}$
    \[
        \abs{c_{m} - c_{n}} = \abs{(a_{m} + b_{m}) - (a_{n} + b_{n})} = \abs{(a_{m} - a_{n}) + (b_{m} - b_{n})}\le \abs{a_{m} - a_{n}} + \abs{b_{m} - b_{n}} < \frac{\varepsilon}{2} + \frac{\varepsilon}{2} = \varepsilon.
    \]

    Thus, $\seq{c_{n}}$ is a Cauchy sequence.
\end{proof}

$\seq{c_{n}}$ is called the sum of $\seq{a_{n}}$ and $\seq{b_{n}}$. We denote the sum of the two sequences by $\seq{a_{n}} + \seq{b_{n}}$.

\begin{theorem}
    $\seq{a_{n}}$ and $\seq{c_{n}}$ are equivalent Cauchy sequences.

    $\seq{b_{n}}$ and $\seq{d_{n}}$ are equivalent Cauchy sequences.

    Then $\seq{a_{n}} + \seq{b_{n}}$ and $\seq{c_{n}} + \seq{d_{n}}$ are equivalent.
\end{theorem}

\begin{proof}
    Let $\varepsilon$ be a positive rational number.

    Since $\seq{a_{n}}$ and $\seq{c_{n}}$ are equivalent Cauchy sequences, $\seq{b_{n}}$ and $\seq{d_{n}}$ are equivalent Cauchy sequences
    \[
        \begin{split}
            (\exists N_{1} = N_{1}(\varepsilon))(\forall n > N_{1})(\abs{a_{n} - c_{n}} < \frac{\varepsilon}{2}), \\
            (\exists N_{2} = N_{2}(\varepsilon))(\forall n > N_{2})(\abs{b_{n} - d_{n}} < \frac{\varepsilon}{2}).
        \end{split}
    \]

    Let $N = \max\{ N_{1}, N_{2} \}$, then
    \[
        (\forall n > N)\left(\abs{(a_{n} + b_{n}) - (c_{n} + d_{n})} = \abs{(a_{n} - c_{n}) + (b_{n} - d_{n})} < \frac{\varepsilon}{2} + \frac{\varepsilon}{2} = \varepsilon\right).
    \]

    Thus, the two Cauchy sequences $\seq{a_{n}} + \seq{b_{n}}$ and $\seq{c_{n}} + \seq{d_{n}}$ are equivalent.
\end{proof}

The above theorem leads to the definition of addition over classes of Cauchy sequences.

\begin{definition}
    $\alpha, \beta\in\mathscr{C}_{\mathbb{Q}}$.

    Let $\seq{a_{n}}$ and $\seq{b_{n}}$ be members of $\alpha$ and $\beta$, respectively.

    The sum of $\alpha$ and $\beta$, which is denoted by $\alpha + \beta$, is
    \[
        \alpha + \beta = \class{\seq{a_{n}} + \seq{b_{n}}}.
    \]
\end{definition}

\begin{theorem}
    $\mathscr{C}_{\mathbb{Q}}$ with addition is a commutative group.
    \begin{enumerate}[label={(F\arabic*)},itemsep=0pt]
        \item Addition is associative.
        \item Addition has an identity element.
        \item Each element has an additive inverse.
        \item Addition is commutative.
    \end{enumerate}
\end{theorem}

\begin{proof}
    Let $\seq{a_{n}}$, $\seq{b_{n}}$, $\seq{c_{n}}$ be Cauchy sequences.

    \begin{enumerate}[label={(F\arabic*)},itemsep=0pt]
        \item For all $n$, $(a_{n} + b_{n}) + c_{n} = a_{n} + (b_{n} + c_{n})$, then the two Cauchy sequences $\seq{(a_{n} + b_{n}) + c_{n}}$ and $\seq{a_{n} + (b_{n} + c_{n})}$ are equivalent.

              Hence
              \begin{align*}
                  \left(\class{\seq{a_{n}}} + \class{\seq{b_{n}}}\right) + \class{\seq{c_{n}}} & = \class{\seq{a_{n} + b_{n}}} + \class{\seq{c_{n}}}                             \\
                                                                                               & = \class{\seq{(a_{n} + b_{n}) + c_{n}}}                                         \\
                                                                                               & = \class{\seq{a_{n} + (b_{n} + c_{n})}}                                         \\
                                                                                               & = \class{\seq{a_{n}} + \seq{b_{n} + c_{n}}}                                     \\
                                                                                               & = \class{\seq{a_{n}}} + \class{\seq{b_{n} + c_{n}}}                             \\
                                                                                               & = \class{\seq{a_{n}}} + \left(\class{\seq{b_{n}}} + \class{\seq{c_{n}}}\right).
              \end{align*}
        \item $\seq{x_{n}}$ is a sequence where $x_{n} = 0$, then $\seq{x_{n}}$ is a Cauchy sequence.

              For all $n$, $a_{n} + x_{n} = x_{n} + a_{n} = a_{n}$. Therefore
              \[
                  \seq{a_{n}} + \seq{x_{n}} = \seq{x_{n}} + \seq{a_{n}} = \seq{a_{n}}.
              \]

              Hence
              \begin{align*}
                  \class{\seq{a_{n}}} + \class{\seq{x_{n}}} & = \class{\seq{a_{n}} + \seq{x_{n}}}          \\
                                                            & = \class{\seq{a_{n}}}                        \\
                                                            & = \class{\seq{x_{n}} + \seq{a_{n}}}          \\
                                                            & = \class{\seq{x_{n}}} + \class{\seq{a_{n}}}.
              \end{align*}
        \item Define a sequence $\seq{d_{n}}$ as the following: $d_{n} = -a_{n}$.

              Since $a_{n}$ is a Cauchy sequence, then for every positive rational number $\varepsilon$
              \[
                  (\exists N=N(\varepsilon))(\forall m, n > N)(\abs{a_{m} - a_{n}} < \varepsilon)
              \]

              So for all $m, n > N$
              \[
                  \abs{d_{m} - d_{n}} = \abs{a_{n} - a_{m}} < \varepsilon.
              \]

              This implies $\seq{d_{n}}$ is a Cauchy sequence.

              On the other hand, $a_{n} + d_{n} = d_{n} + a_{n} = 0$, therefore
              \[
                  \seq{a_{n}} + \seq{d_{n}} = \seq{0} = \seq{d_{n}} + \seq{a_{n}}
              \]

              Thus
              \begin{align*}
                  \class{\seq{a_{n}}} + \class{\seq{d_{n}}} & = \class{\seq{a_{n}} + \seq{d_{n}}}          \\
                                                            & = \class{\seq{0}}                            \\
                                                            & = \class{\seq{d_{n}} + \seq{a_{n}}}          \\
                                                            & = \class{\seq{d_{n}}} + \class{\seq{a_{n}}}.
              \end{align*}
        \item \begin{align*}
                  \class{\seq{a_{n}}} + \class{\seq{b_{n}}} & = \class{\seq{a_{n}} + \seq{b_{n}}}                  \\
                                                            & = \class{\seq{a_{n} + b_{n}}}                        \\
                                                            & = \class{\seq{b_{n} + a_{n}}}                        \\
                                                            & = \class{\seq{b_{n}} + \seq{a_{n}}}                  \\
                                                            & = \class{\seq{b_{n}}} + \class{\seq{a_{n}}}.\qedhere
              \end{align*}
    \end{enumerate}
\end{proof}

If $\alpha\in\mathscr{C}_{\mathbb{Q}}$, we denote the negation (in other words, the additive inverse) of $\alpha$ by $-\alpha$. Negation is involutive, which means $\alpha = -(-\alpha)$.

We define subtraction as the following: $\alpha - \beta = \alpha + (-\beta)$.

\subsubsection*{Multiplication}

\begin{theorem}[Product of Cauchy sequences]\label{theorem:chapter1:product-of-cauchy-sequences}
    Let $\seq{a_{n}}, \seq{b_{n}}\in\mathscr{C}_{\mathbb{Q}}$, then $\seq{c_{n}}$ where $c_{n} = a_{n} \cdot b_{n}$ is a Cauchy sequence.
\end{theorem}

\begin{proof}
    Since Cauchy sequences are bounded, there exists positive rational numbers $A$ and $B$ such that
    \[
        \begin{split}
            \abs{a_{n}} < A\quad\forall n, \\
            \abs{b_{n}} < B\quad\forall n.
        \end{split}
    \]

    Let $\varepsilon$ be a positive rational number.
    \[
        \begin{split}
            (\exists N_{1}=N_{1}(\varepsilon))(\forall m, n > N_{1})\left(\abs{a_{m} - a_{n}} < \frac{\varepsilon}{2\cdot B}\right), \\
            (\exists N_{2}=N_{2}(\varepsilon))(\forall m, n > N_{2})\left(\abs{b_{m} - b_{n}} < \frac{\varepsilon}{2\cdot A}\right).
        \end{split}
    \]

    Choose $N = \max\{ N_{1}, N_{2} \}$. Then for every $m, n > N$
    \begin{align*}
        \abs{a_{m}\cdot b_{m} - a_{n}\cdot b_{n}} & = \abs{a_{m}\cdot b_{m} - a_{m}\cdot b_{n} + a_{m}\cdot b_{n} - a_{n}\cdot b_{n}}         \\
                                                  & \le \abs{a_{m}\cdot b_{m} - a_{m}\cdot b_{n}} + \abs{a_{m}\cdot b_{n} - a_{n}\cdot b_{n}} \\
                                                  & = \abs{a_{m}}\cdot\abs{b_{m} - b_{n}} + \abs{b_{n}}\cdot\abs{a_{m} - a_{n}}               \\
                                                  & < A\cdot\frac{\varepsilon}{2\cdot A} + B\cdot\frac{\varepsilon}{2\cdot B}                 \\
                                                  & = \varepsilon.
    \end{align*}

    Thus, $\seq{c_{n}}$ is a Cauchy sequence.
\end{proof}

$\seq{c_{n}}$ is called the product of $\seq{a_{n}}$ and $\seq{b_{n}}$. We denote the product of the two sequences by $\seq{a_{n}}\cdot\seq{b_{n}}$.

\begin{theorem}
    $\seq{a_{n}}$ and $\seq{c_{n}}$ are equivalent Cauchy sequences.

    $\seq{b_{n}}$ and $\seq{d_{n}}$ are equivalent Cauchy sequences.

    Then $\seq{a_{n}}\cdot\seq{b_{n}}$ and $\seq{c_{n}}\cdot\seq{d_{n}}$ are equivalent.
\end{theorem}

\begin{proof}
    $\seq{b_{n}}$  and $\seq{c_{n}}$ are Cauchy sequences so they are bounded. Then there exists positive rational numbers $B$ and $C$ such that
    \[
        \begin{split}
            \abs{b_{n}} < B\quad\forall n, \\
            \abs{c_{n}} < C\quad\forall n.
        \end{split}
    \]

    Let $\varepsilon$ be a positive rational number. Since $\seq{a_{n}}$ and $\seq{c_{n}}$ are equivalent, $\seq{b_{n}}$ and $\seq{d_{n}}$ are equivalent, then
    \[
        \begin{split}
            (\exists N_{1}=N_{1}(\varepsilon))(\forall n > N_{1})\left(\abs{a_{n} - c_{n}} < \frac{\varepsilon}{2\cdot B}\right), \\
            (\exists N_{2}=N_{2}(\varepsilon))(\forall n > N_{2})\left(\abs{b_{n} - d_{n}} < \frac{\varepsilon}{2\cdot C}\right).
        \end{split}
    \]

    Let $N = \max\{ N_{1}, N_{2} \}$. For all $n > N$, we obtain that
    \begin{align*}
        \abs{a_{n}\cdot b_{n} - c_{n}\cdot d_{n}} & = \abs{a_{n}\cdot b_{n} - b_{n}\cdot c_{n} + b_{n}\cdot c_{n} - c_{n}\cdot d_{n}} \\
                                                  & = \abs{b_{n}\cdot (a_{n} - c_{n}) + c_{n}\cdot (b_{n} - d_{n})}                   \\
                                                  & \le \abs{b_{n}}\cdot\abs{a_{n} - c_{n}} + \abs{c_{n}}\cdot\abs{b_{n} - d_{n}}     \\
                                                  & < B\cdot\frac{\varepsilon}{2\cdot B} + C\cdot\frac{\varepsilon}{2\cdot C}         \\
                                                  & = \frac{\varepsilon}{2} + \frac{\varepsilon}{2}                                   \\
                                                  & = \varepsilon.
    \end{align*}

    Thus, two Cauchy sequences $\seq{a_{n}}\cdot\seq{b_{n}} = \seq{a_{n}\cdot b_{n}}$ and $\seq{c_{n}}\cdot\seq{d_{n}} = \seq{c_{n}\cdot d_{n}}$ are equivalent.
\end{proof}

Similar to addition, we define product of two Cauchy sequence classes as the following.

\begin{definition}
    $\alpha, \beta\in\mathscr{C}_{\mathbb{Q}}$. Let $\seq{a_{n}}$ and $\seq{b_{n}}$ be members of $\alpha$ and $\beta$ respectively.

    The product of $\alpha$ and $\beta$, which is denoted by $\alpha\cdot\beta$, is
    \[
        \alpha\cdot\beta = \class{\seq{a_{n}}\cdot\seq{b_{n}}}.
    \]
\end{definition}

\begin{theorem}
    $\mathscr{C}_{\mathbb{Q}}$ with addition and multiplication is a commutative ring.
    \begin{enumerate}[label={(F\arabic*)},itemsep=0pt,start=5]
        \item Multiplication is associative.
        \item Multiplication is distributive over addition.
        \item Multiplication has an identity element.
        \item Multiplication is commutative.
    \end{enumerate}
\end{theorem}

The proof would be a lot simpler (shorter, actually) if I prove the commutativity first.

\begin{proof}
    Let $\seq{a_{n}}$, $\seq{b_{n}}$, $\seq{c_{n}}$ be Cauchy sequences.
    \begin{enumerate}[label={(F\arabic*)},itemsep=0pt,start=5]
        \item We will prove that
              \[
                  \left(\class{\seq{a_{n}}}\cdot\class{\seq{b_{n}}}\right)\cdot\class{\seq{c_{n}}} = \class{\seq{a_{n}}}\cdot\left(\class{\seq{b_{n}}}\cdot\class{\seq{c_{n}}}\right).
              \]

              According to Theorem~\ref{theorem:chapter1:product-of-cauchy-sequences}, $\left(\seq{a_{n}}\cdot\seq{b_{n}}\right)\cdot\seq{c_{n}}$ and $\seq{a_{n}}\cdot\left(\seq{b_{n}}\cdot\seq{c_{n}}\right)$ are Cauchy sequences.
              \begin{align*}
                  \left(\class{\seq{a_{n}}}\cdot\class{\seq{b_{n}}}\right)\cdot\class{\seq{c_{n}}} & = \class{\seq{a_{n}}\cdot\seq{b_{n}}}\cdot\class{\seq{c_{n}}}                                                          \\
                                                                                                   & = \class{\seq{a_{n}\cdot b_{n}}}\cdot\class{\seq{c_{n}}}                                                               \\
                                                                                                   & = \class{\seq{a_{n}\cdot b_{n}}\cdot\seq{c_{n}}}                                                                       \\
                                                                                                   & = \class{\seq{(a_{n}\cdot b_{n})\cdot c_{n}}}                                                                          \\
                                                                                                   & = \class{\seq{a_{n}\cdot(b_{n}\cdot c_{n})}}                                        & \text{($\mathbb{Q}$ is a field)} \\
                                                                                                   & = \class{\seq{a_{n}}}\cdot\class{\seq{b_{n}\cdot c_{n}}}                                                               \\
                                                                                                   & = \class{\seq{a_{n}}}\cdot\class{\seq{b_{n}}\cdot\seq{c_{n}}}                                                          \\
                                                                                                   & = \class{\seq{a_{n}}}\cdot\left(\class{\seq{b_{n}}}\cdot\class{\seq{c_{n}}}\right).
              \end{align*}
        \item We will prove that
              \[
                  \begin{split}
                      \left(\class{\seq{a_{n}}} + \class{\seq{b_{n}}}\right)\cdot\class{\seq{c_{n}}} = \class{\seq{a_{n}}}\cdot\class{\seq{c_{n}}} + \class{\seq{b_{n}}}\cdot\class{\seq{c_{n}}}, \\
                      \class{\seq{c_{n}}}\cdot\left(\class{\seq{a_{n}}} + \class{\seq{b_{n}}}\right) = \class{\seq{c_{n}}}\cdot\class{\seq{a_{n}}} + \class{\seq{c_{n}}}\cdot\class{\seq{b_{n}}}.
                  \end{split}
              \]

              According to Theorem~\ref{theorem:chapter1:sum-of-cauchy-sequences} and Theorem~\ref{theorem:chapter1:product-of-cauchy-sequences}, the following sequences are Cauchy sequences:
              \[
                  \begin{split}
                      \left(\seq{a_{n}} + \seq{b_{n}}\right)\cdot\seq{c_{n}} \qquad \seq{c_{n}}\cdot\left(\seq{a_{n}} + \seq{b_{n}}\right), \\
                      \seq{a_{n}}\cdot\seq{c_{n}} + \seq{b_{n}}\cdot\seq{c_{n}} \qquad \seq{c_{n}}\cdot\seq{a_{n}} + \seq{c_{n}}\cdot\seq{b_{n}}.
                  \end{split}
              \]
              \begin{align*}
                  \left(\class{\seq{a_{n}}} + \class{\seq{b_{n}}}\right)\cdot\class{\seq{c_{n}}} & = \class{\seq{a_{n}} + \seq{b_{n}}}\cdot\class{\seq{c_{n}}}                                                                     \\
                                                                                                 & = \class{\seq{a_{n} + b_{n}}}\cdot\class{\seq{c_{n}}}                                                                           \\
                                                                                                 & = \class{\seq{a_{n} + b_{n}}\cdot\seq{c_{n}}}                                                                                   \\
                                                                                                 & = \class{\seq{(a_{n} + b_{n})\cdot c_{n}}}                                                                                      \\
                                                                                                 & = \class{\seq{a_{n}\cdot c_{n} + b_{n}\cdot c_{n}}}                                          & \text{($\mathbb{Q}$ is a field)} \\
                                                                                                 & = \class{\seq{a_{n}\cdot c_{n}} + \seq{b_{n}\cdot c_{n}}}                                                                       \\
                                                                                                 & = \class{\seq{a_{n}\cdot c_{n}}} + \class{\seq{b_{n}\cdot c_{n}}}                                                               \\
                                                                                                 & = \class{\seq{a_{n}}\cdot\seq{c_{n}}} + \class{\seq{b_{n}}\cdot\seq{c_{n}}}                                                     \\
                                                                                                 & = \class{\seq{a_{n}}}\cdot\class{\seq{c_{n}}} + \class{\seq{b_{n}}}\cdot\class{\seq{c_{n}}}.
              \end{align*}
              \begin{align*}
                  \class{\seq{c_{n}}}\cdot\left(\class{\seq{a_{n}}} + \class{\seq{b_{n}}}\right) & = \class{\seq{c_{n}}}\cdot\class{\seq{a_{n}} + \seq{b_{n}}}                                                                     \\
                                                                                                 & = \class{\seq{c_{n}}}\cdot\class{\seq{a_{n} + b_{n}}}                                                                           \\
                                                                                                 & = \class{\seq{c_{n}}\cdot\seq{a_{n} + b_{n}}}                                                                                   \\
                                                                                                 & = \class{\seq{c_{n}\cdot (a_{n} + b_{n})}}                                                                                      \\
                                                                                                 & = \class{\seq{c_{n}\cdot a_{n} + c_{n}\cdot b_{n}}}                                          & \text{($\mathbb{Q}$ is a field)} \\
                                                                                                 & = \class{\seq{c_{n}\cdot a_{n}} + \seq{c_{n}\cdot b_{n}}}                                                                       \\
                                                                                                 & = \class{\seq{c_{n}\cdot a_{n}}} + \class{\seq{c_{n}\cdot b_{n}}}                                                               \\
                                                                                                 & = \class{\seq{c_{n}}\cdot\seq{a_{n}}} + \class{\seq{c_{n}}\cdot\seq{b_{n}}}                                                     \\
                                                                                                 & = \class{\seq{c_{n}}}\cdot\class{\seq{a_{n}}} + \class{\seq{c_{n}}}\cdot\class{\seq{b_{n}}}.
              \end{align*}
        \item We will prove that $\class{\seq{1}}$ is such an element.
              \begin{align*}
                  \class{\seq{a_{n}}}\cdot\class{\seq{1}} & = \class{\seq{a_{n}}\cdot\seq{1}}          \\
                                                          & = \class{\seq{a_{n}\cdot 1}}               \\
                                                          & = \class{\seq{a_{n}}}                      \\
                                                          & = \class{\seq{1\cdot a_{n}}}               \\
                                                          & = \class{\seq{1}\cdot\seq{a_{n}}}          \\
                                                          & = \class{\seq{1}}\cdot\class{\seq{a_{n}}}.
              \end{align*}
        \item We will prove that $\class{\seq{a_{n}}}\cdot\class{\seq{b_{n}}} = \class{\seq{b_{n}}}\cdot\class{\seq{a_{n}}}$.
              \begin{align*}
                  \class{\seq{a_{n}}}\cdot\class{\seq{b_{n}}} & = \class{\seq{a_{n}}\cdot\seq{b_{n}}}          \\
                                                              & = \class{\seq{a_{n}\cdot b_{n}}}               \\
                                                              & = \class{\seq{b_{n}\cdot a_{n}}}               \\
                                                              & = \class{\seq{b_{n}}\cdot\seq{a_{n}}}          \\
                                                              & = \class{\seq{b_{n}}}\cdot\class{\seq{a_{n}}}.
              \end{align*}
    \end{enumerate}
\end{proof}

I have difficulty proving the 9th field axiom. After trying, I found out that I need the following lemma (later, I also improved this lemma).

\begin{lemma}\label{lemma:chapter1:nonzero-sequence}
    If the Cauchy sequence $\seq{a_{n}}$ is not equivalent to $\seq{0}$, then only one of the following proposition holds
    \begin{itemize}[itemsep=0pt]
        \item $(\exists a > 0)(\exists N)(\forall n > N)(a_{n} > a)$,
        \item $(\exists a > 0)(\exists N)(\forall n > N)(a_{n} < -a)$.
    \end{itemize}
\end{lemma}

\begin{proof}[Proof of Lemma~\ref{lemma:chapter1:nonzero-sequence}]
    Since $\seq{a_{n}}$ and $\seq{0}$ are inequivalent, there exists a positive rational number $\varepsilon_{0}$ such that
    \begin{equation*}
        (\forall N)(\exists n > N)\left(\abs{a_{n}}\ge\varepsilon_{0}\right).\tag{$\star$}
    \end{equation*}

    $\seq{a_{n}}$ is a Cauchy sequence, so
    \[
        (\exists N_{0}=N_{0}(\varepsilon_{0}))(\forall m, n > N_{0})\left(\abs{a_{m} - a_{n}} < \frac{\varepsilon_{0}}{2}\right).
    \]

    On the other hand $(\star)\implies (\exists n_{0} > N_{0})(\abs{a_{n_{0}}}\ge\varepsilon_{0})$.

    So, for all $n > N_{0}$
    \[
        -\frac{\varepsilon_{0}}{2} < a_{n} - a_{n_{0}} < \frac{\varepsilon_{0}}{2}.
    \]

    $\abs{a_{n_{0}}}\ge\varepsilon_{0}$ implies that \textit{either} $a_{n_{0}}\ge\varepsilon_{0}$, or $a_{n_{0}}\le-\varepsilon_{0}$.
    \bigskip

    \textbf{Case 1. $a_{n_{0}}\ge\varepsilon_{0}$.}

    For all $n > N_{0}$
    \[
        a_{n} = (a_{n} - a_{n_{0}}) + a_{n_{0}} > -\frac{\varepsilon_{0}}{2} + \varepsilon_{0} > \frac{\varepsilon_{0}}{2}.
    \]

    \textbf{Case 2. $a_{n_{0}}\le-\varepsilon_{0}$.}

    For all $n > N_{0}$
    \[
        a_{n} = (a_{n} - a_{n_{0}}) + a_{n_{0}} < \frac{\varepsilon_{0}}{2} + (-\varepsilon_{0}) = -\frac{\varepsilon_{0}}{2}.
    \]
\end{proof}

\begin{theorem}
    \begin{enumerate}[label={(F\arabic*)},start=9]
        \item Each non-zero element (not equal to $\class{\seq{0}}$) of $\mathscr{C}_{\mathbb{Q}}$ has a multiplicative inverse.
    \end{enumerate}
\end{theorem}

\begin{proof}
    We will prove that if $\seq{a_{n}}$ and $\seq{0}$ are inequivalent, then there exists a sequence $\seq{y_{n}}$ such that
    \[
        \class{\seq{a_{n}}}\cdot\class{\seq{y_{n}}} = \class{\seq{y_{n}}}\cdot\class{\seq{a_{n}}} = \class{\seq{1}}.
    \]

    $\seq{a_{n}}$ and $\seq{0}$ are inequivalent.

    Apply Lemma~\ref{lemma:chapter1:nonzero-sequence}, there exists a natural number $N$ such that for all $n > N$, $a_{n}\ne 0$.

    We define $\seq{y_{n}}$ as the following
    \[
        y_{n} = \begin{cases}
            0                & \text{if $n\le N$}, \\
            \dfrac{1}{a_{n}} & \text{if $n > N$}.
        \end{cases}
    \]

    From the definition of $\seq{y_{n}}$, we obtain that
    \[
        a_{n}\cdot y_{n} = y_{n}\cdot a_{n} = \begin{cases}
            0 & \text{if $n\le N$}, \\
            1 & \text{if $n > N$}.
        \end{cases}
    \]

    Let $\varepsilon$ be a positive rational number. Then
    \[
        (\forall n > N)(\abs{a_{n}\cdot y_{n} - 1} = 0 < \varepsilon).
    \]

    Therefore, $\seq{a_{n}}\cdot \seq{y_{n}}$ converges to $\seq{1}$. Consequently
    \begin{align*}
        \class{\seq{a_{n}}}\cdot\class{\seq{y_{n}}} & = \class{\seq{y_{n}}}\cdot\class{\seq{a_{n}}} \\
                                                    & = \class{\seq{y_{n}}\cdot\seq{a_{n}}}         \\
                                                    & = \class{\seq{y_{n}\cdot a_{n}}}              \\
                                                    & = \class{\seq{1}}.
    \end{align*}

    Hence $\class{\seq{a_{n}}}$ has a multiplicative inverse.
\end{proof}

If $\alpha\in\mathscr{C}_{\mathbb{Q}}$ and $\alpha$ is non-zero, then we denote its multiplicative inverse by ${\alpha}^{-1}$. Like additive inversion, multiplicative inversion is involutive ($\alpha = {\left({\alpha}^{-1}\right)}^{-1}$).

At this point, we have proved that $\mathscr{C}_{\mathbb{Q}}$ with addition and multiplication satisfies the nine field axioms.

\subsection{Preorder relation}

We define preorder relation on the set of all Cauchy sequences first.

\begin{definition}
    Let $\seq{a_{n}}$ and $\seq{b_{n}}$ be two Cauchy sequences.

    \begin{itemize}
        \item We say $\seq{a_{n}}$ precedes $\seq{b_{n}}$ and denote $\seq{a_{n}}\le\seq{b_{n}}$ if they are equivalent or there exists a natural number $N$ such that
              \[
                  (\forall n > N)(a_{n}\le b_{n}).
              \]
        \item We say $\seq{a_{n}}$ strictly precedes $\seq{b_{n}}$ and denote $\seq{a_{n}} < \seq{b_{n}}$ if they are inequivalent and there exists a natural number $N$ such that
              \[
                  (\forall n > N)(a_{n}\le b_{n}).
              \]
    \end{itemize}
\end{definition}

Strict precedence implies precedence but the converse is not generally true.

Those two are just preorder (not order) relations (we haven't proved this yet!), since the anti-symmetric property is not satisfied. I chose to define the two relations this way because they will be ``preserved'' when I define the corresponding relations for equivalence classes of Cauchy sequences.

\begin{theorem}\label{theorem:chapter1:inequivalent-sequences}
    Let $\seq{a_{n}}$ and $\seq{b_{n}}$ be two Cauchy sequences.

    $\seq{a_{n}} < \seq{b_{n}}$ if and only if then there exists a positive rational number $q$ such that $(\exists N)(\forall n > N)(a_{n} - b_{n} < -q)$.
\end{theorem}

\begin{proof}
    $(\Rightarrow)$

    $\seq{a_{n}}$ and $\seq{b_{n}}$ are inequivalent, so that $\seq{a_{n} - b_{n}}$ does not converge to $0$. According to Lemma~\ref{lemma:chapter1:nonzero-sequence}
    \begin{align*}
        \text{either}\quad & (\exists q > 0)(\exists N_{0})(\forall n > N_{0})(a_{n} - b_{n} < -q), \\
        \text{or}\quad     & (\exists q > 0)(\exists N_{0})(\forall n > N_{0})(a_{n} - b_{n} > q).
    \end{align*}

    However, the latter is not the case since $\seq{a_{n}} < \seq{b_{n}}$ implies that $(\exists N)(\forall n > N)(a_{n}\le b_{n})$.

    Hence, $(\exists q > 0)(\exists N_{0})(\forall n > N_{0})(a_{n} - b_{n} < -q)$.

    \bigskip
    $(\Leftarrow)$

    There exists a positive rational number $q$ such that $(\exists N)(\forall n > N)(a_{n} - b_{n} < -q)$.

    This implies $(\exists N)(\forall n > N)(a_{n} < b_{n})$. According to the definition of ``$\le$'', $\seq{a_{n}} \le \seq{b_{n}}$.

    Choose $\varepsilon_{0} = q$, then $(\forall N_{0})(\exists n > \max\{ N, N_{0} \})(\abs{a_{n} - b_{n}} = a_{n} - b_{n} > q = \varepsilon_{0})$. This means $\seq{a_{n}}$ and $\seq{b_{n}}$ are inequivalent.

    Hence $\seq{a_{n}} < \seq{b_{n}}$.
\end{proof}

\begin{theorem}\label{theorem:chapter1:equivalent-cauchy-sequences-and-order}
    Given Cauchy sequences $\seq{a_{n}}, \seq{b_{n}}, \seq{c_{n}}, \seq{d_{n}}$.

    $\seq{a_{n}}$ and $\seq{c_{n}}$ are equivalent, $\seq{b_{n}}$ and $\seq{d_{n}}$ are equivalent.

    \begin{enumerate}[label={(\roman*)}]
        \item $\seq{a_{n}}\le\seq{b_{n}}$ iff $\seq{c_{n}}\le\seq{d_{n}}$.
        \item $\seq{a_{n}} < \seq{b_{m}}$ iff $\seq{c_{n}} < \seq{d_{n}}$.
    \end{enumerate}
\end{theorem}

\begin{proof}
    Because of the symmetry, we only need to prove the ($\Rightarrow$) part.

    \begin{enumerate}[label={\textbf{Case \arabic*.}},itemindent=0.5cm]
        \item $\seq{a_{n}}$ and $\seq{b_{n}}$ are equivalent.

              Due to the transitivity of equivalence relation, $\seq{c_{n}}$ and $\seq{d_{n}}$ are equivalent.

              According to the definition, $\seq{c_{n}}\le\seq{d_{n}}$.
        \item $\seq{a_{n}}$ and $\seq{b_{n}}$ are inequivalent.

              According to Theorem~\ref{theorem:chapter1:inequivalent-sequences}, there exists positive rational number $q$ such that
              \[
                  (\exists N_{0})(\forall n > N_{0})(a_{n} - b_{n} < -q) \implies a_{N_{0} + 1} < b_{N_{0} + 1}
              \]

              Since $\seq{a_{n}}, \seq{b_{n}}$ are Cauchy sequences
              \[
                  \begin{split}
                      (\exists N_{a})(\forall m, n > N_{a})\left(\abs{a_{m} - a_{n}} < \frac{b_{N_{0} + 1} - a_{N_{0} + 1}}{8}\right) \\
                      (\exists N_{b})(\forall m, n > N_{b})\left(\abs{b_{m} - b_{n}} < \frac{b_{N_{0} + 1} - a_{N_{0} + 1}}{8}\right)
                  \end{split}
              \]

              $\seq{a_{n}}$ and $\seq{c_{n}}$ are equivalent, so
              \[
                  (\exists N_{c})(\forall n > N_{c})\left(-\frac{b_{N_{0}+1} - a_{N_{0}+1}}{8} < a_{n} - c_{n} < \frac{b_{N_{0}+1} - a_{N_{0}+1}}{8}\right).
              \]

              $\seq{b_{n}}$ and $\seq{d_{n}}$ are equivalent, so
              \[
                  (\exists N_{d})(\forall n > N_{d})\left(-\frac{b_{N_{0}+1} - a_{N_{0}+1}}{8} < d_{n} - b_{n} < \frac{b_{N_{0}+1} - a_{N_{0}+1}}{8}\right).
              \]

              The idea behind this proof is finding/adjusting epsilon\@(s) to make an inequality correct.

              Let $N = \max\{ N_{0}, N_{a}, N_{b}, N_{c}, N_{d} \}$. For $n > N$
              \begin{align*}
                  d_{n} - c_{n} & = (d_{n} - b_{n}) + (b_{n} - a_{n}) + (a_{n} - c_{n})                                                                                                                                                                          \\
                                & = \abs{b_{n} - a_{n}} + (a_{n} - c_{n}) + (d_{n} - b_{n})                                                                                                                                                                      \\
                                & = \abs{(b_{N_{0}+1} - a_{N_{0}+1}) - (a_{n} - a_{N_{0}+1} + b_{N_{0}+1} - b_{n})} + (a_{n} - c_{n}) + (d_{n} - b_{n})                                                                                                          \\
                                & \ge \abs{b_{N_{0}+1} - a_{N_{0}+1}} - \abs{(a_{n} - a_{N_{0}+1}) + (b_{N_{0}+1} - b_{n})} + (a_{n} - c_{n}) + (d_{n} - b_{n})                                                                                                  \\
                                & \ge (b_{N_{0}+1} - a_{N_{0}+1}) - \left(\abs{a_{n} - a_{N_{0}+1}} + \abs{b_{N_{0}+1} - b_{n}}\right) + (a_{n} - c_{n}) + (d_{n} - b_{n})                                                                                       \\
                                & > (b_{N_{0}+1} - a_{N_{0}+1}) - \left(\frac{b_{N_{0}+1} - a_{N_{0}+1}}{8} + \frac{b_{N_{0}+1} - a_{N_{0}+1}}{8}\right) + \left(-\frac{b_{N_{0}+1} - a_{N_{0}+1}}{8}\right) + \left(-\frac{b_{N_{0}+1} - a_{N_{0}+1}}{8}\right) \\
                                & = \frac{b_{N_{0}+1} - a_{N_{0}+1}}{2} > 0.
              \end{align*}

              So, for all $n > N$, $c_{n} < d_{n}$. Therefore, by definition, $\seq{c_{n}}\le \seq{d_{n}}$.

              On the other hand, $d_{n} - c_{n} > \dfrac{b_{N_{0} + 1} - a_{N_{0} + 1}}{2} > 0$ for all $n > N$ so $\seq{c_{n}}$ and $\seq{d_{n}}$ are inequivalent.
              \bigskip

              Hence, $\seq{c_{n}} < \seq{d_{n}}$.
    \end{enumerate}

    \bigskip
    Thus
    \begin{align*}
        \seq{a_{n}} \le \seq{b_{n}} \implies \seq{c_{n}} \le \seq{d_{n}}, \\
        \seq{a_{n}} < \seq{b_{n}} \implies \seq{c_{n}} < \seq{d_{n}}.
    \end{align*}
\end{proof}

\begin{theorem}\label{theorem:chapter1:preorder-and-cauchy-sequences}
    The ``$\le$'' relation between Cauchy sequences is a total \textbf{preorder} relation.
\end{theorem}

\begin{proof}
    \begin{itemize}
        \item Reflexivity.

              Due to the definition, a Cauchy sequence precedes itself, because a Cauchy sequence is equivalent to itself.
        \item Transitivity.

              \begin{enumerate}[label={\textbf{Case \arabic*.}},itemindent=0.5cm]
                  \item $\seq{a_{n}}$ and $\seq{b_{n}}$ are equivalent, $\seq{b_{n}}$ and $\seq{c_{n}}$ are equivalent.

                        In this case, $\seq{a_{n}}$ and $\seq{c_{n}}$ are equivalent. Hence, $\seq{a_{n}}\le\seq{c_{n}}$.
                  \item $\seq{a_{n}}$ and $\seq{b_{n}}$ are equivalent, $\seq{b_{n}}$ and $\seq{c_{n}}$ are inequivalent.

                        According to Theorem~\ref{theorem:chapter1:inequivalent-sequences}, there exists a positive rational number $q$ such that
                        \[
                            (\exists N_{0})(\forall n > N_{0})(b_{n} - c_{n} < -q)
                        \]

                        $\seq{a_{n}}$ and $\seq{b_{n}}$ are equivalent implies
                        \[
                            (\exists N_{1})(\forall n > N_{1})\left(\abs{a_{n} - b_{n}} < \frac{q}{2}\right).
                        \]

                        So, for all $n > N = \max\{ N_{0}, N_{1} \}$
                        \[
                            a_{n} - c_{n} = (a_{n} - b_{n}) + (b_{n} - c_{n}) < \frac{q}{2} + (-q) = \frac{-q}{2} < 0
                        \]

                        Hence $\seq{a_{n}} < \seq{c_{n}}$.

                  \item $\seq{a_{n}}$ and $\seq{b_{n}}$ are inequivalent, $\seq{b_{n}}$ and $\seq{c_{n}}$ are equivalent.

                        According to Theorem~\ref{theorem:chapter1:inequivalent-sequences}, there exists a positive rational number $q$ such that
                        \[
                            (\exists N_{0})(\forall n > N_{0})(a_{n} - b_{n} < -q)
                        \]

                        $\seq{b_{n}}$ and $\seq{c_{n}}$ are equivalent implies
                        \[
                            (\exists N_{1})(\forall n > N_{1})\left(\abs{b_{n} - c_{n}} < \frac{q}{2}\right).
                        \]

                        So, for all $n > N = \max\{ N_{0}, N_{1} \}$
                        \[
                            a_{n} - c_{n} = (a_{n} - b_{n}) + (b_{n} - c_{n}) < (-q) + \frac{q}{2} = \frac{-q}{2} < 0
                        \]

                        Hence $\seq{a_{n}} < \seq{c_{n}}$.
                  \item $\seq{a_{n}}$ and $\seq{b_{n}}$ are inequivalent, $\seq{b_{n}}$ and $\seq{c_{n}}$ are inequivalent.

                        According to Theorem~\ref{theorem:chapter1:inequivalent-sequences}, there exists positive rational number $q_{1}, q_{2}$ such that
                        \[
                            \begin{split}
                                & (\exists N_{1})(\forall n > N_{1})(a_{n} - b_{n} < -q_{1}), \\
                                & (\exists N_{2})(\forall n > N_{2})(b_{n} - c_{n} < -q_{2}).
                            \end{split}
                        \]

                        So, for all $n > N = \max\{ N_{1}, N_{2} \}$
                        \[
                            a_{n} - c_{n} = (a_{n} - b_{n}) + (b_{n} - c_{n}) < -(q_{1} + q_{2}) < 0.
                        \]

                        Hence $\seq{a_{n}} < \seq{c_{n}}$.
              \end{enumerate}
        \item Totality.
              \begin{enumerate}[label={\textbf{Case \arabic*.}},itemsep=0pt,itemindent=1cm]
                  \item $\seq{a_{n}}$ and $\seq{b_{n}}$ are equivalent.

                        There is nothing to prove in this case.

                  \item $\seq{a_{n}}$ and $\seq{b_{n}}$ are inequivalent.

                        According to Lemma~\ref{lemma:chapter1:nonzero-sequence}
                        \begin{align*}
                            (\exists q > 0)(\exists N)(\forall n > N)(a_{n} - b_{n} > q), \\
                            (\exists q > 0)(\exists N)(\forall n > N)(a_{n} - b_{n} < -q).
                        \end{align*}

                        The 1st scenario implies $\seq{a_{n}} > \seq{b_{n}}$. The 2nd scenario implies $\seq{a_{n}} < \seq{b_{n}}$.
              \end{enumerate}

              Hence, one of these propositions hold: $\seq{a_{n}}\le\seq{b_{n}}$ and $\seq{b_{n}}\le\seq{a_{n}}$.
    \end{itemize}
\end{proof}

\begin{definition}
    $\seq{a_{n}}$ and $\seq{b_{n}}$ are two Cauchy sequences.

    We say $\seq{a_{n}}$ strictly precedes $\seq{b_{n}}$ and denote $\seq{a_{n}} < \seq{b_{n}}$ if $\seq{a_{n}}\le\seq{b_{n}}$ and the two sequences are inequivalent.
\end{definition}

\begin{theorem}
    ``$<$'' relation in the set of all Cauchy sequences is irreflexive and transitive.
\end{theorem}

\begin{proof}
    \begin{itemize}
        \item Irreflexivity.

              Let $\seq{a_{n}}$ be a Cauchy sequence. Since a Cauchy sequence is equivalent to itself, $\seq{a_{n}} < \seq{a_{n}}$ is false.
        \item Transitivity.

              Let $\seq{a_{n}}, \seq{b_{n}}, \seq{c_{n}}$ be Cauchy sequences such that $\seq{a_{n}} < \seq{b_{n}}$ and $\seq{b_{n}} < \seq{c_{n}}$. I will prove that $\seq{a_{n}} < \seq{c_{n}}$.

              $\seq{a_{n}}$ and $\seq{b_{n}}$ are inequivalent implies that $\seq{a_{n} - b_{n}}$ and $\seq{0}$ are inequivalent. $\seq{b_{n}}$ and $\seq{c_{n}}$ are inequivalent implies that $\seq{b_{n} - c_{n}}$ and $\seq{0}$ are inequivalent. According to Lemma~\ref{lemma:chapter1:nonzero-sequence}
              \begin{itemize}
                  \item Either $(\exists p > 0)(\exists N_{p})(\forall n > N_{p})(a_{n} - b_{n} > p)$ or $(\exists p > 0)(\exists N_{p})(\forall n > N_{p})(a_{n} - b_{n} < -p)$.

                        $\seq{a_{n}} < \seq{b_{n}}$ implies $\seq{a_{n}}\le\seq{b_{n}}$, so $(\exists N_{1})(\forall n > N_{1})(a_{n} - b_{n} \le 0)$. This contradicts the first possibility. Hence the first is false, and the second is true.
                  \item Either $(\exists q > 0)(\exists N_{q})(\forall n > N_{q})(b_{n} - c_{n} > q)$ or $(\exists q > 0)(\exists N_{q})(\forall n > N_{q})(b_{n} - c_{n} < -q)$.

                        $\seq{b_{n}} < \seq{c_{n}}$ implies $\seq{b_{n}}\ge\seq{c_{n}}$, so $(\exists N_{2})(\forall n > N_{2})(b_{n} - c_{n} \le 0)$. This contradicts the first possibility. Hence the first is false, and the second is true.
              \end{itemize}

              So for all $n > \max\{ N_{p}, N_{q} \}$
              \[
                  a_{n} - c_{n} = (a_{n} - b_{n}) + (b_{n} - c_{n}) < (-p) + (-q) = -(p+q) < 0.
              \]

              Hence $\seq{a_{n}}$ and $\seq{c_{n}}$ are inequivalent.

              On the other hand, $\seq{a_{n}} < \seq{b_{n}}$ implies $\seq{a_{n}}\le\seq{b_{n}}$; $\seq{b_{n}} < \seq{c_{n}}$ implies $\seq{b_{n}}\le\seq{c_{n}}$. So $\seq{a_{n}}\le\seq{c_{n}}$.

              According to the definition of ``$<$'', $\seq{a_{n}} < \seq{c_{n}}$.
    \end{itemize}
\end{proof}

\begin{theorem}\label{theorem:chapter1:rational-number-between-zero-and-positive-cauchy-sequence}
    If $\seq{0} < \seq{a_{n}}$, there exists a positive rational number $q$ such that
    \[
        \seq{0} < \seq{q} < \seq{a_{n}}.
    \]
\end{theorem}

\begin{proof}
    According to Theorem~\ref{theorem:chapter1:inequivalent-sequences}, there exists a positive rational number $a$ such that
    \[
        (\exists N)(\forall n > N)(a_{n} > a)
    \]

    It follows that $\seq{a} \le \seq{a_{n}}$.

    There exists a positive rational number $q$ such that $0 < q < a$. This implies $\seq{0} < \seq{q} < \seq{a}$.

    According to the proof of Theorem~\ref{theorem:chapter1:preorder-and-cauchy-sequences}, $\seq{0} < \seq{q} < \seq{a_{n}}$.
\end{proof}

The ``$\le$'' relation between Cauchy sequences is not anti-symmetric. However, we have the following result.

\begin{theorem}\label{theorem:chapter1:weak-anti-symmetry}
    Let $\seq{a_{n}}$ and $\seq{b_{n}}$ be Cauchy sequences that satisfy
    \[
        \seq{a_{n}}\le\seq{b_{n}}\qquad\seq{b_{n}}\le\seq{a_{n}}.
    \]

    Then $\seq{a_{n}}$ and $\seq{b_{n}}$ are equivalent.
\end{theorem}

\begin{proof}
    According to Theorem~\ref{theorem:chapter1:preorder-and-cauchy-sequences}, either $\seq{a_{n}}\le\seq{b_{n}}$ or $\seq{b_{n}}\le\seq{a_{n}}$.

    Assume that the two Cauchy sequences are inequivalent. Then $\seq{a_{n}} < \seq{b_{n}}$ and $\seq{b_{n}} < \seq{a_{n}}$, which implies $\seq{a_{n}} < \seq{a_{n}}$.

    Thus, $\seq{a_{n}}$ and $\seq{b_{n}}$ are equivalent.
\end{proof}

This total preorder relation is compatible with addition and multiplication.

\begin{theorem}\label{theorem:chapter1:cauchy-sequence-order-compatibility-with-addition-and-mulitplication}
    Let $\seq{a_{n}}, \seq{b_{n}}, \seq{c_{n}}$ be Cauchy sequences. Then
    \begin{enumerate}[label={(\roman*)}]
        \item ${\seq{a_{n}}\le\seq{b_{n}}} \implies {\seq{a_{n}}+\seq{c_{n}}\le\seq{b_{n}}+\seq{c_{n}}}$.
        \item ${\seq{0}\le\seq{a_{n}}} \land {\seq{0}\le\seq{b_{n}}} \implies {\seq{0}\le\seq{a_{n}}\cdot\seq{b_{n}}}$.
        \item ${\seq{a_{n}} < \seq{b_{n}}} \implies {\seq{a_{n}}+\seq{c_{n}} < \seq{b_{n}}+\seq{c_{n}}}$.
        \item ${\seq{0} < \seq{a_{n}}} \land {\seq{0} < \seq{b_{n}}} \implies {\seq{0} < \seq{a_{n}}\cdot\seq{b_{n}}}$.
    \end{enumerate}
\end{theorem}

\begin{proof}
    \begin{enumerate}[label={(\roman*)}]
        \item $\seq{a_{n}}\le\seq{b_{n}}$ implies that there exists a natural number $N$ such that for all $n > N$, $a_{n}\le b_{n}$.

              On the other hand $a_{n}\le b_{n}\rightarrow a_{n} + c_{n}\le b_{n} + c_{n}$. Therefore
              \[
                  (\exists N)(\forall n > N)(a_{n} + c_{n}\le b_{n} + c_{n}).
              \]

              Hence $\seq{a_{n}} + \seq{c_{n}}\le\seq{b_{n}} + \seq{c_{n}}$.
        \item $\seq{0}\le\seq{a_{n}}$ and $\seq{0}\le\seq{b_{n}}$ imply that
              \[
                  \begin{split}
                      (\exists N_{a})(\forall n > N_{a})(a_{n}\ge 0), \\
                      (\exists N_{b})(\forall n > N_{b})(b_{n}\ge 0).
                  \end{split}
              \]

              Therefore, for all $n > N = \max\{ N_{a}, N_{b} \}$, $a_{n}\cdot b_{n}\ge 0$.

              Thus, $\seq{0}\le\seq{a_{n}}\cdot\seq{b_{n}}$.
        \item According to Theorem~\ref{theorem:chapter1:inequivalent-sequences}, there exists a positive rational number $q$ such that
              \[
                  (\exists N)(\forall n > N)(a_{n} - b_{n} < -q)
              \]

              which is equivalent to
              \[
                  (\exists N)(\forall n > N)((a_{n} + c_{n}) - (b_{n} + c_{n}) < -q).
              \]

              According to Theorem~\ref{theorem:chapter1:inequivalent-sequences}, $\seq{a_{n}} + \seq{c_{n}} < \seq{b_{n}} + \seq{c_{n}}$.
        \item According to Theorem~\ref{lemma:chapter1:nonzero-sequence}
              \[
                  \begin{split}
                      (\exists a > 0)(\exists N_{a})(\forall n > N_{a})(a_{n} > a), \\
                      (\exists b > 0)(\exists N_{b})(\forall n > N_{b})(b_{n} > b),
                  \end{split}
              \]

              Then $(\forall n > N = \max{N_{a}, N_{b}})(a_{n}b_{n} > ab > 0)$. According to Theorem~\ref{theorem:chapter1:inequivalent-sequences}, $\seq{0} < \seq{a_{n}}\cdot\seq{b_{n}}$.\qedhere
    \end{enumerate}
\end{proof}

\subsection{Order relation}

\begin{definition}
    We say
    \begin{itemize}
        \item $\class{\seq{a_{n}}}$ precedes $\class{\seq{b_{n}}}$ if and only if every Cauchy sequence in $\class{\seq{a_{n}}}$ precedes every Cauchy sequence in $\class{\seq{b_{n}}}$, and denote $\class{\seq{a_{n}}} \le \class{\seq{b_{n}}}$.
        \item $\class{\seq{a_{n}}}$ strictly precedes $\class{\seq{b_{n}}}$ if and only if every Cauchy sequence in $\class{\seq{a_{n}}}$ strictly precedes every Cauchy sequence in $\class{\seq{b_{n}}}$, and denote $\class{\seq{a_{n}}} < \class{\seq{b_{n}}}$.
    \end{itemize}
\end{definition}

\begin{theorem}[Total ordering]
    $\mathscr{C}_{\mathbb{Q}}$ with ``$\le$'' relation is totally ordered.
\end{theorem}

\begin{proof}
    Let $\alpha, \beta, \gamma$ be Cauchy sequences equivalence classes.
    \begin{itemize}
        \item Reflexivity ($\alpha\le\alpha$)

              For every two Cauchy sequences $\seq{a_{n}}$ and $\seq{b_{n}}$ of a Cauchy sequences class, $\seq{a_{n}}\le\seq{b_{n}}$. So a Cauchy sequence class precedes itself.
        \item Transitivity ($\alpha\le\beta\land\beta\land\gamma\rightarrow\alpha\le\gamma$).

              Assume that $\alpha\le\beta$ and $\beta\le\gamma$.

              Let $\seq{a_{n}}, \seq{b_{n}}, \seq{c_{n}}$ be members of $\alpha, \beta, \gamma$, respectively.

              According to the definition of ``$\le$'' in $\mathscr{C}_{\mathbb{Q}}$, $\seq{a_{n}}\le\seq{b_{n}}$ and $\seq{b_{n}}\le\seq{c_{n}}$.

              Since ``$\le$'' relation in the set of all Cauchy sequences (with only rational terms) is a total preorder relation, then $\seq{a_{n}}\le\seq{c_{n}}$.

              It follows from Theorem~\ref{theorem:chapter1:equivalent-cauchy-sequences-and-order} that $\class{\seq{a_{n}}}\le\class{\seq{c_{n}}}$. Therefore $\alpha\le\gamma$.

        \item Anti-symmetry ($\alpha\le\beta\land\beta\le\alpha\rightarrow\alpha=\beta$).

              Let $\seq{a_{n}}, \seq{b_{n}}$ be members of $\alpha, \beta$, respectively.

              According to the definition of ``$\le$'' in $\mathscr{C}_{\mathbb{Q}}$, $\seq{a_{n}}\le\seq{b_{n}}$ and $\seq{b_{n}}\le\seq{a_{n}}$.

              Then it follows from Theorem~\ref{theorem:chapter1:weak-anti-symmetry} that $\seq{a_{n}}$ and $\seq{b_{n}}$ are equivalent.

              Therefore, $\class{\seq{a_{n}}} = \class{\seq{b_{n}}}$. In other words, $\alpha = \beta$.
        \item Totality ($\alpha\le\beta\vee\beta\le\alpha$).

              Let $\seq{a_{n}}, \seq{b_{n}}$ be members of $\alpha, \beta$, respectively.

              Since ``$\le$'' in the set of all Cauchy sequences (with only rational terms) is a total preorder relation, then $\seq{a_{n}}\le\seq{b_{n}}$ or $\seq{b_{n}}\le\seq{a_{n}}$.

              Together with Theorem~\ref{theorem:chapter1:equivalent-cauchy-sequences-and-order}, we obtain that either $\alpha = \beta$, or $\alpha < \beta$ or $\beta < \alpha$.\qedhere
    \end{itemize}
\end{proof}

The following theorem is a corollary of Theorem~\ref{theorem:chapter1:preorder-and-cauchy-sequences} and~\ref{theorem:chapter1:equivalent-cauchy-sequences-and-order}.

\begin{theorem}\label{theorem:chapter1:compatibility-of-operation-and-order}
    Let $\alpha, \beta, \gamma\in\mathscr{C}_{\mathbb{Q}}$.
    \begin{enumerate}[label={(\roman*)}]
        \item $\alpha \le \beta \implies \alpha + \gamma \le \beta + \gamma$.
        \item $\class{\seq{0}}\le\alpha\land\class{\seq{0}}\le\beta \implies \seq{0}\le\alpha\cdot\beta$.
        \item $\alpha < \beta \implies \alpha + \gamma < \beta + \gamma$.
        \item $\class{\seq{0}} < \alpha \land \class{\seq{0}} < \beta \implies \class{\seq{0}} < \alpha\cdot\beta$.
    \end{enumerate}
\end{theorem}

\begin{proof}
    \begin{enumerate}[label={(\roman*)}]
        \item Let $\seq{a_{n}}, \seq{b_{n}}, \seq{c_{n}}$ be members of $\alpha, \beta, \gamma$, respectively.

              $\alpha\le\beta$ implies $\seq{a_{n}}\le\seq{b_{n}}$.

              $\seq{a_{n}}\le\seq{b_{n}}$ implies $\seq{a_{n}} + \seq{c_{n}}\le\seq{b_{n}} + \seq{c_{n}}$.

              According to Theorem~\ref{theorem:chapter1:cauchy-sequence-order-compatibility-with-addition-and-mulitplication},~\ref{theorem:chapter1:equivalent-cauchy-sequences-and-order} and the definition of addition

              \begin{align*}
                  \alpha + \gamma & = \class{\seq{a_{n}}} + \class{\seq{c_{n}}} \\
                                  & = \class{\seq{a_{n}} + \seq{c_{n}}}         \\
                                  & \le\class{\seq{b_{n}} + \seq{c_{n}}}        \\
                                  & = \class{\seq{b_{n}}} + \class{\seq{c_{n}}} \\
                                  & = \beta + \gamma.
              \end{align*}
        \item Let $\seq{a_{n}}, \seq{b_{n}}, \seq{c_{n}}$ be members of $\alpha, \beta, \gamma$, respectively.

              $\class{\seq{0}}\le\alpha$ and $\class{\seq{0}}\le\beta$ imply $\seq{0}\le\seq{a_{n}}$ and $\seq{0}\le\seq{b_{n}}$.

              According to Theorem~\ref{theorem:chapter1:cauchy-sequence-order-compatibility-with-addition-and-mulitplication},~\ref{theorem:chapter1:equivalent-cauchy-sequences-and-order} and the definition of multiplication
              \begin{align*}
                  \class{\seq{0}} & \le \class{\seq{a_{n}}\cdot\seq{b_{n}}}       \\
                                  & = \class{\seq{a_{n}}}\cdot\class{\seq{b_{n}}} \\
                                  & = \alpha\cdot\beta.
              \end{align*}
        \item Let $\seq{a_{n}}, \seq{b_{n}}, \seq{c_{n}}$ be members of $\alpha, \beta, \gamma$, respectively.
              \begin{align*}
                                  & \alpha < \beta                                                                                                                                                                                                             \\
                  \Leftrightarrow & (\forall \seq{a_{n}}\in\alpha\land \seq{b_{n}}\in\beta) (\seq{a_{n}} < \seq{b_{n}})                                                        & \text{(Theorem~\ref{theorem:chapter1:equivalent-cauchy-sequences-and-order})} \\
                  \Leftrightarrow & (\forall \seq{a_{n}}\in\alpha\land \seq{b_{n}}\in\beta \land \seq{c_{n}}\in\gamma) (\seq{a_{n}} + \seq{c_{n}} < \seq{b_{n}} + \seq{c_{n}}) & \text{(Theorem~\ref{theorem:chapter1:preorder-and-cauchy-sequences})}         \\
                  \Leftrightarrow & \alpha + \gamma < \beta + \gamma.
              \end{align*}
        \item Let $\seq{a_{n}}, \seq{b_{n}}$ be members of $\alpha, \beta$, respectively.

              $\class{\seq{0}} < \alpha$ and $\class{\seq{0}} < \beta$ implies $\seq{0} < \seq{a_{n}}$ and $\seq{0} < \seq{b_{n}}$.

              According to Theorem~\ref{theorem:chapter1:preorder-and-cauchy-sequences}, $\seq{0} < \seq{a_{n}}\cdot\seq{b_{n}}$.

              Due to Theorem~\ref{theorem:chapter1:equivalent-cauchy-sequences-and-order}, $\class{\seq{0}} < \class{\seq{a_{n}}\cdot\seq{b_{n}}} = \alpha\cdot\beta$.\qedhere
    \end{enumerate}
\end{proof}

At this time, we have proved that $\mathscr{C}_{\mathbb{Q}}$ is a totally ordered field.

\subsection{Positive and Negative}

\begin{definition}
    An element $\alpha$ of $\mathscr{C}_{\mathbb{Q}}$ is called
    \begin{itemize}
        \item \textit{positive} if and only if $\class{\seq{0}}$ strictly precedes $\alpha$.
        \item \textit{negative} if and only if $\alpha$ strictly precedes $\class{\seq{0}}$.
        \item \textit{non-positive} if and only if it is not positive ($\alpha$ precedes $\class{\seq{0}}$).
        \item \textit{non-negative} if and only if it is not negative ($\class{\seq{0}}$ precedes $\alpha$).
    \end{itemize}
\end{definition}

\begin{theorem}\label{theorem:chapter1:sign-and-additive-inverse}
    In $\mathscr{C}_{\mathbb{Q}}$, additive inverse of an element is positive if and only if the element is negative.
\end{theorem}

\begin{proof}
    Let $\alpha$ be an element of $\mathscr{C}_{\mathbb{Q}}$.
    \[
        \alpha + (-\alpha) = \class{\seq{0}}.
    \]

    $(\Rightarrow)$ $-\alpha$ is positive.

    If $\alpha$ is positive, then $\alpha + (-\alpha) > \class{\seq{0}} + \class{\seq{0}} = \class{\seq{0}}$ (contradiction).

    If $\alpha$ is zero, then $-\alpha$ is zero (contradiction).

    Hence $\alpha$ is negative (follows the totality of ``$\le$'').

    \bigskip
    $(\Leftarrow)$ $\alpha$ is negative.

    If $-\alpha$ is negative, then $\alpha + (-\alpha) < \class{\seq{0}} + \class{\seq{0}} = \class{\seq{0}}$ (contradiction).

    If $-\alpha$ is zero, then $\alpha$ is zero (contradiction).

    Hence $-\alpha$ is positive (follows the totality of ``$\le$'').
\end{proof}

\begin{theorem}\label{theorem:chapter1:sign-and-multiplication-cauchy-sequence}
    Let $\alpha$ and $\beta$ be elements of $\mathscr{C}_{\mathbb{Q}}$.
    \[
        \begin{split}
            & \alpha\cdot(-\beta) = (-\alpha)\cdot\beta = -\alpha\cdot\beta, \\
            & \alpha\cdot\beta = (-\alpha)\cdot(-\beta).
        \end{split}
    \]
\end{theorem}

\begin{proof}
    According to (F7)
    \[
        \begin{split}
            \alpha\cdot(-\beta) + \alpha\cdot\beta = \alpha((-\beta) + \beta) = \alpha\cdot\class{\seq{0}} = \class{\seq{0}}, \\
            (-\alpha)\cdot\beta + \alpha\cdot\beta = ((-\alpha) + \alpha)\beta = \class{\seq{0}}\cdot\beta = \class{\seq{0}}.
        \end{split}
    \]

    Hence $\alpha\cdot(-\beta)$ and $(-\alpha)\cdot\beta$ equal the negation of $\alpha\cdot\beta$, which is $-\alpha\cdot\beta$.

    Finally
    \[
        \alpha\cdot\beta = -(-\alpha)\cdot\beta = (-\alpha)\cdot(-\beta).\qedhere
    \]
\end{proof}

\begin{theorem}
    Let $\alpha, \beta$ be elements of $\mathscr{C}_{\mathbb{Q}}$.
    \begin{enumerate}[label={(\roman*)}]
        \item $\alpha > \class{\seq{0}} \land \beta > \class{\seq{0}} \implies \alpha\cdot\beta > \class{\seq{0}}$.
        \item $\alpha > \class{\seq{0}} \land \beta < \class{\seq{0}} \implies \alpha\cdot\beta < \class{\seq{0}}$.
        \item $\alpha < \class{\seq{0}} \land \beta > \class{\seq{0}} \implies \alpha\cdot\beta < \class{\seq{0}}$.
        \item $\alpha < \class{\seq{0}} \land \beta < \class{\seq{0}} \implies \alpha\cdot\beta > \class{\seq{0}}$.
    \end{enumerate}
\end{theorem}

\begin{proof}
    \begin{enumerate}[label={(\roman*)}]
        \item This follows Theorem~\ref{theorem:chapter1:compatibility-of-operation-and-order}.
        \item According to Theorem~\ref{theorem:chapter1:sign-and-additive-inverse} and~\ref{theorem:chapter1:compatibility-of-operation-and-order}, $\alpha\cdot(-\beta) > \class{\seq{0}}$.

              On the other hand, $\alpha\cdot(-\beta) = -\alpha\cdot\beta$. Due to Theorem~\ref{theorem:chapter1:sign-and-additive-inverse}, $\alpha\cdot\beta < \class{\seq{0}}$.
        \item According to Theorem~\ref{theorem:chapter1:sign-and-additive-inverse} and~\ref{theorem:chapter1:compatibility-of-operation-and-order}, $(-\alpha)\cdot\beta > \class{\seq{0}}$.

              On the other hand, $(-\alpha)\cdot\beta = -\alpha\cdot\beta$. Due to Theorem~\ref{theorem:chapter1:sign-and-additive-inverse}, $\alpha\cdot\beta < \class{\seq{0}}$.
        \item According to Theorem~\ref{theorem:chapter1:sign-and-additive-inverse} and~\ref{theorem:chapter1:compatibility-of-operation-and-order}, $(-\alpha)\cdot(-\beta) > \class{\seq{0}}$.

              On the other hand, $(-\alpha)\cdot(-\beta) = \alpha\cdot\beta$. Hence $\alpha\cdot\beta > \class{\seq{0}}$.\qedhere
    \end{enumerate}
\end{proof}

\begin{corollary}
    A non-zero element of $\mathscr{C}_{\mathbb{Q}}$ is
    \begin{itemize}
        \item positive if and only if its multiplicative inverse is positive.
        \item negative if and only if its multiplicative inverse is negative.
    \end{itemize}
\end{corollary}

\subsection{Embed $\mathbb{Q}$ into $\mathscr{C}_{\mathbb{Q}}$}

\begin{theorem}
    There exists an embedding $\mathfrak{c}: \mathbb{Q}\to\mathscr{C}_{\mathbb{Q}}$ which is a ring monomorphism and preserves order.
\end{theorem}

\begin{proof}
    We define such embedding as follows:
    \begin{align*}
        \mathfrak{c}:\quad & \mathbb{Q}\to\mathscr{C}_{\mathbb{Q}} \\
                           & q\mapsto \class{\seq{q}}
    \end{align*}

    This is a ring monomorphism since
    \[
        q_{1}\ne q_{2}\implies \mathfrak{c}(q_{1})\ne\mathfrak{c}(q_{2}) \\
    \]
    \begin{align*}
        \mathfrak{c}(q_{1} + q_{2})    & = \class{\seq{q_{1} + q_{2}}}                  \\
                                       & = \class{\seq{q_{1}} + \seq{q_{2}}}            \\
                                       & = \class{\seq{q_{1}}} + \class{\seq{q_{2}}}    \\
                                       & = \mathfrak{c}(q_{1}) + \mathfrak{c}(q_{2}),   \\
        \mathfrak{c}(q_{1}\cdot q_{2}) & = \class{\seq{q_{1}\cdot q_{2}}}               \\
                                       & = \class{\seq{q_{1}}\cdot\seq{q_{2}}}          \\
                                       & = \class{\seq{q_{1}}}\cdot\class{\seq{q_{2}}}  \\
                                       & = \mathfrak{c}(q_{1})\cdot\mathfrak{c}(q_{2}).
    \end{align*}

    $\mathfrak{c}$ preserves order, since
    \[
        q_{1}\le q_{2}\implies\seq{q_{1}}\le\seq{q_{2}}\implies\class{\seq{q_{1}}}\le\class{\seq{q_{2}}}.
    \]
\end{proof}

\subsection{Density of $\mathbb{Q}$ in $\mathscr{C}_{\mathbb{Q}}$}

So far, the least-upper-bound property has remained untouched. Turns out, it is more difficult to prove this property using Cauchy sequences. I have searched for proofs in several notes, articles, books, appendices. In summary, the authors used some intermediary results, notably \textit{the Archimedean property} and \textit{the density of $\mathbb{Q}$ in $\mathscr{C}_{\mathbb{Q}}$}.

\begin{theorem}[Archimedean property in $\mathscr{C}_{\mathbb{Q}}$]
    Let $\alpha, \beta\in\mathscr{C}_{\mathbb{Q}}$, where $\beta$ is positive. There exists an integer $k$ such that
    \[
        (k-1)\cdot\beta \le \alpha < k\cdot\beta
    \]

    where for $k\in\mathbb{Z}, k\cdot\beta = \class{\seq{k}}\cdot\beta$.
\end{theorem}

I am going to use the well-ordering principle.

\begin{proof}
    Since $\beta$ is positive, then its multiplicative inverse is positive.

    Let $\seq{x_{n}}$ be a member of $\alpha\cdot{\beta}^{-1}$.

    Let $S_{1} = \{ k : k\in\mathbb{Z} \land \seq{k} \le \seq{x_{n}} \}$ and $S_{2} = \{ k : k\in\mathbb{Z}\land \seq{x_{n}} < \seq{k} \}$.

    \begin{enumerate}[label={\textbf{Step \arabic*.}},itemindent=0.5cm]
        \item Prove that $S_{1}$ and $S_{2}$ are not empty.

              Since every Cauchy sequence is bounded, then there exists a positive rational number $q$ such that $\abs{x_{n}}\le q$. Let $c$ be an integer such that $c \le -q$, $d$ be an integer such that $q < d$.

              Then $c \le -q \le x_{n} \le q < d$. This means $c$ is in $S_{1}$ and $d$ is in $S_{2}$.

              Hence $S_{1}$ and $S_{2}$ are not empty.
        \item Prove that $S_{2}$ is bounded below.

              $S_{1}$ and $S_{2}$ together is a partition of $\mathbb{Z}$.

              According to the definition of $S_{1}$ and $S_{2}$, each element of $S_{1}$ is a lower bound of $S_{2}$. Hence $S_{2}$ is bounded below.
        \item Prove that $S_{2}$ has a least element.

              $S_{2}$ is a set of integers and $S_{2}$ is bounded above, then $S_{2}$ has a least element, according to the well-ordering principle.
    \end{enumerate}

    Let $k$ be the least element of $S_{2}$, then $k-1$ is not an element of $S_{2}$. Thereby, $k-1$ is in $S_{1}$.

    Therefore, $k-1 \le \alpha\cdot{\beta}^{-1} < k$. Since ${\beta}^{-1}$ is positive, it follows that $(k-1)\cdot\beta \le \alpha < k\cdot\beta$.
\end{proof}

\begin{theorem}\label{theorem:chapter1:archimedean-property-cauchy-sequence-equivalence-class}
    Let $\alpha, \beta\in\mathscr{C}_{\mathbb{Q}}$, where $\alpha < \beta$. There exists a rational number $q$ such that $\alpha < \class{\seq{q}} < \beta$.
\end{theorem}

\begin{proof}
    $\beta - \alpha > \class{\seq{0}}$.

    According to the Archimedean property, there exists an integer $y$ such that $\class{\seq{y}}(\beta - \alpha) > \class{\seq{1}}$. It follows that $y$ is a positive integer. From the inequality, and the distributivity, we obtain that
    \begin{equation*}
        \class{\seq{y}}\cdot\beta > \class{\seq{1}} + \class{\seq{y}}\cdot\alpha.
        \tag{(1)}
    \end{equation*}

    Once again, due to the Archimedean property, there exists an integer $x$ such that
    \begin{equation*}
        \class{\seq{x}} - \class{\seq{1}} \le \class{\seq{y}}\cdot\alpha < \class{\seq{x}}.
        \tag{(2)}
    \end{equation*}

    From (1) and (2)
    \[
        \class{\seq{y}}\cdot\beta > \class{\seq{1}} + \class{\seq{y}}\cdot\alpha \ge \class{\seq{1}} + (\class{\seq{x}} - \class{\seq{1}}) = \class{\seq{x}}.
    \]

    Hence
    \[
        \class{\seq{y}}\cdot\alpha < \class{\seq{x}} < \class{\seq{y}}\cdot\beta.
    \]

    Multiply by the multiplicative inverse of $\class{\seq{y}}$ (which is positive), we obtain that
    \[
        \alpha < \class{\seq{\frac{x}{y}}} < \beta.
    \]

    $\dfrac{x}{y}$ is the desired rational number.
\end{proof}

\subsection{Least-upper-bound property}

\begin{theorem}\label{theorem:chapter1:cauchy-sequence-class-least-upper-bound}
    Let $S$ be a non-empty subset of $\mathscr{C}_{\mathbb{Q}}$, $S$ is bounded above. Then $S$ has a least upper bound.
\end{theorem}

I will construct such an equivalence class of Cauchy sequences (least upper bound of $S$). The following proof is an extended version of one in Wikipedia. To derive Theorem~\ref{theorem:chapter1:cauchy-sequence-class-least-upper-bound}, I prove the following result first.

\begin{theorem}\label{theorem:chapter1:cauchy-sequence-least-upper-bound}
    Let $S$ be a non-empty set of Cauchy sequences and $S$ is bounded above. Then there exists a Cauchy sequence $\seq{x_{n}}$ such that $\seq{x_{n}}$ is a least upper bound of $S$. In other words
    \begin{itemize}[itemsep=0pt]
        \item $\seq{x_{n}}$ is an upper bound of $X$,
        \item any Cauchy sequence which strictly precedes $\seq{x_{n}}$ is not an upper bound of $X$.
    \end{itemize}
\end{theorem}

\begin{proof}
    Let $\seq{b_{n}}$ be an element of $S$.

    Since $S$ is bounded above, then there exists a sequence $\seq{a_{n}}$ which is an upper bound of $S$.

    Because every Cauchy sequence is bounded, there exists positive rational numbers $a, b$ such that $\abs{a_{n}}\le a$ and $\abs{b_{n}}\le b$.

    Let $u_{0} = a$ and $\ell_{0} = -b$, then $a_{n} \le u_{0}$ and $\ell_{0} \le b_{n}$.

    $\seq{\ell_{0}} \le \seq{s_{n}} \le \seq{a_{n}} \le \seq{u_{0}}$ implies that $\ell_{0} \le u_{0}$.

    From these, we define two sequences $\seq{u_{n}}$ and $\seq{\ell_{n}}$ by induction: If $\seq{\dfrac{1}{2}(u_{k-1} + \ell_{k-1})}$ is an upper bound of $S$, then $u_{k} = \dfrac{1}{2}(u_{k-1} + \ell_{k-1})$ and $\ell_{k} = \ell_{k-1}$. Otherwise, $u_{k} = u_{k-1}$ and $\ell_{k} = \dfrac{1}{2}(u_{k-1} + \ell_{k-1})$. From this definition and by induction, we deduce the following
    \begin{itemize}[itemsep=0pt]
        \item For any fixed non-negative integer $m$, $\seq{u_{m}}$ is an upper bound of $S$ and $\seq{\ell_{m}}$ is not an upper bound of $S$.
        \item $\seq{u_{n}}$ is non-increasing and $\seq{\ell_{n}}$ is non-decreasing.
        \item $u_{n} \ge \ell_{n}$ for every non-negative integer $n$.
        \item $u_{m} \ge \ell_{n}$ for any non-negative integers $m, n$.
        \item $u_{n} - \ell_{n} = \dfrac{1}{2}(u_{n-1} - \ell_{n-1})$ and $u_{n} - \ell_{n} = \dfrac{1}{2^{n}}(u_{0} - \ell_{0})$.
    \end{itemize}

    For every positive rational number $\varepsilon$
    \[
        \begin{cases}
            \left(\forall n > N = \ceiling{\dfrac{u_{0} - \ell_{0}}{\varepsilon}}\right)\left( 0 \le u_{n} - \ell_{n} = \dfrac{1}{2^{n}}(u_{0} - \ell_{0}) \le \dfrac{1}{n}(u_{0} - \ell_{0}) < \varepsilon \right), \\
            \left(\forall n > N = \ceiling{\dfrac{u_{0} - \ell_{0}}{\varepsilon}}\right)(\forall p)\left( 0 \le u_{n} - u_{n+p} \le u_{n} - \ell_{n} < \varepsilon \right).
        \end{cases}
    \]

    Therefore, $\seq{u_{n}}$ is a Cauchy sequence and $\seq{u_{n}}$ is equivalent to $\seq{\ell_{n}}$.

    \begin{enumerate}[label={\textbf{Step \arabic*.}},itemindent=0.5cm]
        \item Prove that $\seq{u_{n}}$ is an upper bound of $S$.

              Let $\seq{s_{n}}$ be an arbitrary Cauchy sequence from $S$. Assume that $\seq{u_{n}}$ strictly precedes $\seq{s_{n}}$.

              According to Theorem~\ref{theorem:chapter1:inequivalent-sequences}, there exists a positive rational number $q$ such that
              \[
                  (\exists N)(\forall n > N)(u_{n} - s_{n} < -q).
              \]

              Since $\seq{u_{N+1}}$ is an upper bound of $S$ and $\seq{u_{n}}$ is non-increasing
              \[
                  (\forall n > N)(u_{N+1} - s_{n} \le u_{n} - s_{n} < -q)
              \]

              which implies that $\seq{u_{N+1}}$ strictly precedes $\seq{s_{n}}$ due to Theorem~\ref{theorem:chapter1:inequivalent-sequences}. This contradicts ``$\seq{u_{N+1}}$ is an upper bound of $S$''.

              Hence the assumption ($\seq{u_{n}}$ strictly precedes $\seq{s_{n}}$) is false. Therefore $\seq{s_{n}} \le \seq{u_{n}}$ for arbitrary $\seq{s_{n}}$ from $S$. Thus $\seq{u_{n}}$ is an upper bound of $S$.
        \item Prove that $\seq{\ell_{n}}$ is an upper bound of $S$.

              Since $\seq{u_{n}}$ is an upper bound of $S$ and $\seq{\ell_{n}}$ is equivalent to $\seq{u_{n}}$.
        \item Prove that $\seq{v_{n}} < \seq{u_{n}}$ implies $\seq{v_{n}}$ is not an upper bound of $S$.

              Since $\seq{u_{n}}$ is equivalent to $\seq{\ell_{n}}$, $\seq{v_{n}} < \seq{\ell_{n}}$. According to the Archimedean property (Theorem~\ref{theorem:chapter1:archimedean-property-cauchy-sequence-equivalence-class}), there exists a positive rational number $q$ such that $\seq{v_{n}} < \seq{q} < \seq{\ell_{n}}$.

              $\seq{q} < \seq{\ell_{n}}$ implies that there exists a natural number $N$ such that for all $n > N$, $q \le \ell_{n}$. Therefore, $q \le \ell_{N+1}$ and $\seq{q} \le \seq{\ell_{N+1}}$. On the other hand, $\seq{\ell_{N+1}}$ is not an upper bound of $S$. Thereby, $\seq{q}$ is not an upper bound of $S$. Furthermore, $\seq{v_{n}} < \seq{q}$, so $\seq{v_{n}}$ is not an upper bound of $S$ either.

              Thus $\seq{v_{n}} < \seq{u_{n}}$ implies $\seq{v_{n}}$ is not an upper bound of $S$.
    \end{enumerate}

    We have constructed a desired Cauchy sequence, which is $\seq{u_{n}}$.
\end{proof}

Back to Theorem~\ref{theorem:chapter1:cauchy-sequence-class-least-upper-bound}.

\begin{proof}[Proof of Theorem~\ref{theorem:chapter1:cauchy-sequence-class-least-upper-bound}]
    Let $S'$ be the set of all Cauchy sequences which are members of some Cauchy sequence equivalence class from $S$.

    $S$ is bounded above, then $S'$ is also bounded above.

    According to Theorem~\ref{theorem:chapter1:cauchy-sequence-least-upper-bound}, there exists a Cauchy sequence $\seq{x_{n}}$ such that $\seq{x_{n}}$ is a least upper bound of $S'$. $\class{\seq{x_{n}}}$ is an upper bound of $S$.

    Let $\alpha$ be an upper bound of $S$, then there exists $\seq{y_{n}}$ within $\alpha$ such that $\seq{y_{n}}$ is an upper bound of $S'$. Since $\seq{x_{n}}$ is a least upper bound of $S'$, then $\seq{x_{n}} \le \seq{y_{n}}$. This implies $\class{\seq{x_{n}}} \le \alpha$.

    Thus $\class{\seq{x_{n}}}$ is the least upper bound of $S$.
\end{proof}

At this point, we have successfully contructed the real numbers by Cauchy sequences.

\subsection{Completeness (Cauchy-completeness)}

In this subsection, we have got the real numbers. From now on, we consider Cauchy sequences which have real-number terms (and call them Cauchy sequences). All prior results of Cauchy sequences with only rational-number terms hold for Cauchy sequences with real-number terms (of course, there are slight changes in statements, from ``rational'' to ``real'').

The main result of this subsection is: every Cauchy sequence is convergent (Cauchy-complete). The following definitions and intermediate results pave the way to the proof of the main result.

\begin{definition}[Supremum and Infimum]
    Let $S$ be a set of real numbers.
    \begin{itemize}
        \item A real number $a$ is called a supremum of $S$ if $a$ is an upper bound of $S$ and every real number which is less than $a$ is not an upper bound of $S$. $a$ is denoted by $\sup S$.
        \item A real number $b$ is called an infimum of $S$ if $b$ is a lower bound of $S$ and every real number which is greater than $b$ is not a lower bound of $S$. $b$ is denoted by $\inf S$.
    \end{itemize}
\end{definition}

From the least-upper-bound property, one can prove that
\begin{itemize}
    \item if a set of real numbers is bounded above, then it has a supremum,
    \item if a set of real numbers is bounded below, then it has an infimum.
\end{itemize}

Supremum and maximum are not synonymous. For a given bounded above set $S$, supremum always exists, but maximum does not necessarily exist. But when maximum exists, it is identical to supremum. Likewise, so are infimum and minimum.

\begin{theorem}[Monotone convergent theorem]
    A non-decreasing sequence which is bounded above is convergent. A non-increasing sequence which is bounded below is convergent.
\end{theorem}

\begin{proof}
    \begin{itemize}
        \item Let $\seq{a_{n}}$ be a non-decreasing sequence which is bounded above.

              Let $a$ be the supremum of $\seq{a_{n}}$ and $\varepsilon$ be an arbitrary positive real number.

              $a = \sup{a_{n}}$ so $a - \varepsilon$ is not an upper bound of $\seq{a_{n}}$. So there exists a natural number $N$ such that $a - \varepsilon < a_{N}$.

              On the other hand, $\seq{a_{n}}$ is non-decreasing, so
              \[
                  (\forall n > N)(\abs{a_{n} - a} = a - a_{n} \le a - a_{N} < \varepsilon)
              \]

              Hence $\seq{a_{n}}$ converges to its supremum.
        \item Let $\seq{a_{n}}$ be a non-increasing sequence which is bounded below.

              Let $a$ be the infimum of $\seq{a_{n}}$ and $\varepsilon$ be an arbitrary positive real number.

              $a = \inf{a_{n}}$ so $a + \varepsilon$ is not a lower bound of $\seq{a_{n}}$. So there exists a natural number $N$ such that $a_{N} < a + \varepsilon$.

              On the other hand, $\seq{a_{n}}$ is non-increasing, so
              \[
                  (\forall n > N)(\abs{a_{n} - a} = a_{n} - a \le a_{N} - a < \varepsilon).
              \]

              Hence $\seq{a_{n}}$ converges to its infimum.
    \end{itemize}
\end{proof}

\begin{theorem}[Nested interval theorem]
    Let $\seq{a_{n}}$ be a non-decreasing sequence, $\seq{b_{n}}$ be a non-increasing sequence, where $a_{n} \le b_{n}$ for every natural number $n$.

    $I_{n} = [a_{n}, b_{n}] = \{ x: x\in\mathbb{R} \land a_{n} \le x \le b_{n} \}$.

    Then $\bigcap\limits_{n\in\mathbb{N}} {I_{n}}$ is not empty.
\end{theorem}

\begin{proof}
    Let $i, j$ be two natural numbers such that $i\le j$. Since $\seq{a_{n}}$ is non-decreasing, $\seq{b_{n}}$ is non-increasing and $a_{n} \le b_{n}$, we deduce that
    \begin{equation*}
        a_{i} \le a_{j} \le b_{j} \le b_{i}
        \tag{$\star$}
    \end{equation*}

    $\seq{a_{n}}$ is a non-decreasing sequence and bounded above by $b_{1}$, $\seq{b_{n}}$ is a non-increasing sequence and bounded below by $a_{1}$.

    According to the Monotone Convergence Theorem, $\seq{a_{n}}$ converges to $\sup{a_{n}}$, $\seq{b_{n}}$ converges to $\inf{b_{n}}$.

    Let $a = \sup{a_{n}}$ and $b = \inf{b_{n}}$.

    Prove that $a\le b$.

    $b_{n}$ is an upper bound of $\seq{a_{n}}$, for every natural number $n$. On the one hand, $a$ is the supremum (the least upper bound) of $\seq{a_{n}}$, so $a$ is a lower bound of $\seq{b_{n}}$. On the other hand, $b$ is the infimum (the greatest lower bound) of $\seq{b_{n}}$, so $a\le b$.

    Let $x$ be an element of $[a, b]$, then $a \le x \le b$. For any natural number $n$, $a_{n}\le a \le x \le b \le b_{n}$. Therefore, $x$ is in $[a_{n}, b_{n}]$, for every natural number $n$. Hence $x\in\bigcap\limits_{n\in\mathbb{N}} {I_{n}}$.

    Thus $\bigcap\limits_{n\in\mathbb{N}} {I_{n}}$ is not empty.
\end{proof}

\begin{theorem}[Cantor's intersection theorem]
    Let $\seq{a_{n}}$ be a non-decreasing sequence, $\seq{b_{n}}$ be a non-increasing sequence, where $a_{n} \le b_{n}$ for every natural number $n$ and $\seq{a_{n}}$ is equivalent to $\seq{b_{n}}$.

    $I_{n} = [a_{n}, b_{n}] = \{ x: x\in\mathbb{R} \land a_{n} \le x \le b_{n} \}$.

    Then $\bigcap\limits_{n\in\mathbb{N}} {I_{n}}$ has a only one element.
\end{theorem}

\begin{proof}
    $\seq{a_{n}}$ is a non-decreasing sequence and bounded above by $b_{1}$, $\seq{b_{n}}$ is a non-increasing sequence and bounded below by $a_{1}$.

    According to the Monotone Convergence Theorem, $\seq{a_{n}}$ converges to $\sup{a_{n}}$, $\seq{b_{n}}$ converges to $\inf{b_{n}}$.

    Let $a = \sup{a_{n}}$ and $b = \inf{b_{n}}$.

    According to the proof of the Nesting Interval Theorem, $a\le b$.

    \begin{enumerate}[label={\textbf{Step \arabic*.}},itemindent=0.5cm]
        \item Prove that $I = \bigcap\limits_{n\in\mathbb{N}} {I_{n}} = [a, b]$.

              From the proof of the Nesting Interval Theorem, we deduce that $[a, b]\subseteq I$.

              Let $x$ be an element of $I$. Due to the definition of $I$, for every natural number $n$, $a_{n} \le x \le b_{n}$. It follows that $x$ is an upper bound of $\seq{a_{n}}$ and a lower bound of $\seq{b_{n}}$. Therefore $a \le x \le b$. Equivalently, $x$ is in $[a, b]$. Hence $I\subseteq [a, b]$.

              Thus $I = [a, b]$.
        \item Prove that $a = b$.

              Assume that $a < b$.

              $\seq{a_{n}}$ and $\seq{b_{n}}$ are equivalent, so there exists a natural number $N$ such that for all $n > N$, we have $\abs{a_{n} - b_{n}} < b - a$.

              On the other hand, for every natural number $n$
              \[
                  \abs{a_{n} - b_{n}} = b_{n} - a_{n} = (b_{n} - b) + (b - a) + (a - a_{n}) \ge b - a > 0,
              \]

              which makes contradiction.

              Therefore, the assumption ($a < b$) is false. Hence $a = b$.
    \end{enumerate}

    Because $a = b$, $[a, b]$ has only one element. Thus, $I$ has only one element.
\end{proof}

\begin{theorem}[Bolzano-Weierstra{\ss}'s theorem]
    A bounded sequence contains at least one convergent subsequence.
\end{theorem}

\begin{proof}
    Let $\seq{x_{n}}$ be a bounded sequence.

    We define $\seq{A_{n}}$, $\seq{B_{n}}$ and contruct the desired subsequence inductively as the following.

    Let $x_{n_{1}} = x_{1}$. Let $A_{1}$ be a lower bound of $\seq{x_{n}}$ and $B_{1}$ be an upper bound of $\seq{x_{n}}$. Then for every natural number $n$, $x_{n}\in [A_{1}, B_{1}]$.

    Among two closed intervals $\left[A_{1}, \frac{1}{2}(A_{1} + B_{1})\right]$ and $\left[\frac{1}{2}(A_{1} + B_{1}), B_{1}\right]$, there is at least one that contains infinite terms of $\seq{x_{n}}$. Let such closed interval be $[A_{2}, B_{2}]$. Thanks to the well-ordering principle, we can choose the least natural number $n_{2}$ such that $n_{2} > n_{1}$ and $x_{n_{2}}\in [A_{2}, B_{2}]$.

    Among two closed intervals $\left[A_{k-1}, \frac{1}{2}(A_{k-1} + B_{k-1})\right]$ and $\left[\frac{1}{2}(A_{k-1} + B_{k-1}, B_{k-1})\right]$, there is at least one that contains infinite terms of $\seq{x_{n}}$. Let such closed interval be $[A_{k}, B_{k}]$. By the well-ordering principle, we can choose the least natural number $n_{k}$ such that $n_{k} > n_{k-1}$ and $x_{n_{k}}\in [A_{k}, B_{k}]$.

    From these constructions, we obtain that
    \begin{itemize}
        \item $\seq{A_{n}}$ is a non-decreasing sequence,
        \item $\seq{B_{n}}$ is a non-increasing sequence,
        \item $A_{n}\le B_{n}$ for every natural number $n$,
        \item $A_{n} - B_{n} = \frac{1}{2}(A_{n-1} - B_{n-1}) = \cdots = \frac{1}{2^{n-1}}(A_{1} - B_{1})$.
    \end{itemize}

    For every positive real number $\varepsilon$, choose $N = \ceiling{\dfrac{B_{1} - A_{1}}{\varepsilon}} + 1$,
    \[
        (\forall n > N)\left(\abs{A_{n} - B_{n}} = B_{n} - A_{n} = \frac{1}{2^{n-1}}(B_{1} - A_{1})\le \frac{1}{n}(B_{1} - A_{1}) < \varepsilon\right).
    \]

    Therefore, $\seq{A_{n}}$ and $\seq{B_{n}}$ are equivalent. According to the Cantor's Intersection Theorem, $\seq{A_{n}}$ and $\seq{B_{n}}$ have the same limit point. Let the limit point be $x$.
    \[
        \abs{x_{n_{k}} - x} \le \abs{A_{k} - B_{k}} = \frac{1}{2^{k-1}} \abs{A_{1} - B_{1}}.
    \]

    For every positive real number $\varepsilon$, choose $M = \ceiling{\dfrac{B_{1} - A_{1}}{\varepsilon}} + 1$,
    \[
        (\forall k > M)\left(\abs{x_{n_{k}} - x} = \frac{1}{2^{k-1}}\abs{A_{1} - B_{1}} \le \frac{1}{k}(B_{1} - A_{1}) < \varepsilon\right).
    \]

    Hence the subsequence $(x_{n_{1}}, x_{n_{2}}, \ldots)$ converges to $x$.

    Thus $\seq{x_{n}}$ contains a convergent subsequence.
\end{proof}

\begin{theorem}
    Every Cauchy sequence converges to a real number.
\end{theorem}

\begin{proof}
    Let $\seq{x_{n}}$ be a Cauchy sequence.

    Every Cauchy sequence is bounded. Apply the Bolzano-Weierstra{\ss}'s theorem, we obtain that $\seq{x_{n}}$ contains a convergent subsequence. Let such subsequence be $(x_{n_{1}}, x_{n_{2}}, \ldots)$ and $x$ be its limit point. We are going to prove that $\seq{x_{n}}$ converges to $x$.

    For every positive real number $\varepsilon$, there exists $N$ and $N_{0}$ such that
    \[
        \begin{split}
            & (\forall m, n > N)\left(\abs{x_{m} - x_{n}} < \frac{\varepsilon}{2}\right), \\
            & (\forall n_{k} > N_{0})\left(\abs{x_{n_{k}} - x} < \frac{\varepsilon}{2}\right).
        \end{split}
    \]

    For all $n, n_{k} > \max\{ N, N_{0} \}$,
    \[
        \abs{x_{n} - x} \le \abs{x_{n} - x_{n_{k}}} + \abs{x_{n_{k}} - x} < \frac{\varepsilon}{2} + \frac{\varepsilon}{2} = \varepsilon.
    \]

    Thus $\seq{x_{n}}$ converges to $x$.
\end{proof}

\section{Compare the two constructions (models)}

\section{Uniqueness of complete ordered field}

