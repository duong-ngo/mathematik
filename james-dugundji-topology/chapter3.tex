\chapter{Topological spaces}

\section{Topological spaces}

\begin{problem}{III.1.1}\label{problem:III.1.1}
\begin{enumerate}[label={(\alph*)},leftmargin=*]
	\item Let \( X \) be an infinite set. Show that \( \mathscr{A}_{0} = \left\{ \varnothing \right\} \cup \left\{ A \mid \mathscr{C}A \text{ is finite} \right\} \) is a topology.
	\item Let \( \aleph(X) \ge \aleph_{0} \). Show that \( \mathscr{A}_{1} = \left\{ \varnothing \right\} \cup \left\{ A \mid \aleph(\mathscr{C}A) < \aleph(X) \right\} \) is a topology.
\end{enumerate}
\end{problem}

\( \mathscr{A}_{0} \) is called the cofinite topology on \( X \).

\begin{proof}
	\begin{enumerate}[label={(\alph*)},leftmargin=*]
		\item \( \varnothing \in \mathscr{A}_{0} \) and \( \mathscr{C}X = \varnothing \) is finite so \( \mathscr{A}_{0} \) contains \( \varnothing \) and \( X \).

		      Assume that \( {(A_{\alpha})}_{\alpha\in\mathscr{A}} \) is a family of elements of \( \mathscr{A}_{0} \). According to the De Morgan's laws
		      \begin{align*}
			      \mathscr{C}\left( \bigcup_{\alpha\in\mathscr{A}} A_{\alpha} \right) = \bigcap_{\alpha\in\mathscr{A}} \mathscr{C}A_{\alpha}
		      \end{align*}

		      If there exists \( \alpha \in \mathscr{A} \) such that \( \mathscr{C}A_{\alpha} \) is finite, then \( \bigcap_{\alpha\in\mathscr{A}} \mathscr{C}A_{\alpha} \) is finite for being a subset of \( \mathscr{C}A_{\alpha} \). Otherwise, \( \bigcap_{\alpha\in\mathscr{A}} \mathscr{C}A_{\alpha} = \bigcap_{\alpha\in\mathscr{A}} X = X \). In either case, it is true that \( \bigcup_{\alpha\in\mathscr{A}} A_{\alpha} \in \mathscr{A}_{0} \).

		      Moreover, if \( \mathscr{A} \) is finite then
		      \begin{align*}
			      \mathscr{C}\left( \bigcap_{\alpha\in\mathscr{A}} A_{\alpha} \right) = \bigcup_{\alpha\in\mathscr{A}} \mathscr{C}A_{\alpha}
		      \end{align*}

		      If there exists \( \alpha \in \mathscr{A} \) such that \( A_{\alpha} = \varnothing \) then \( \bigcap_{\alpha\in\mathscr{A}} A_{\alpha} = \varnothing \in \mathscr{A}_{0} \). Otherwise, \( \mathscr{A} \) is finite and each \( \mathscr{C}A_{\alpha} \) is finite, so their union is also finite, which implies that \( \bigcup_{\alpha\in}\mathscr{C}A_{\alpha} \) is finite. In either case, it is true that \( \bigcap_{\alpha\in\mathscr{A}} A_{\alpha} \in \mathscr{A}_{0} \) for any finite \( \mathscr{A} \).

		      Thus \( \mathscr{A}_{0} \) is a topology.
		\item \( \varnothing \in \mathscr{A}_{1} \) and \( \mathscr{C}X = \varnothing \) so \( \mathscr{A}_{1} \) contains \( \varnothing \) and \( X \).

		      Assume that \( {(A_{\alpha})}_{\alpha\in\mathscr{A}} \) is a family of elements of \( \mathscr{A}_{1} \). According to the De Morgan's laws
		      \begin{align*}
			      \mathscr{C}\left( \bigcup_{\alpha\in\mathscr{A}} A_{\alpha} \right) = \bigcap_{\alpha\in\mathscr{A}} \mathscr{C}A_{\alpha}
		      \end{align*}

		      If there exists \( \alpha \in \mathscr{A} \) such that \( \aleph(\mathscr{C}A_{\alpha}) < \aleph(X) \), then \( \aleph\left(\bigcap_{\alpha\in\mathscr{A}} \mathscr{C}A_{\alpha}\right) \le \aleph(\mathscr{C}A_{\alpha}) < \aleph(X) \). Otherwise, \( \bigcap_{\alpha\in\mathscr{A}} \mathscr{C}A_{\alpha} = \bigcap_{\alpha\in\mathscr{A}} X = X \). In either case, it is true that \( \bigcup_{\alpha\in\mathscr{A}} A_{\alpha} \in \mathscr{A}_{1} \).

		      Moreover, if \( \mathscr{A} \) is finite then
		      \begin{align*}
			      \mathscr{C}\left( \bigcap_{\alpha\in\mathscr{A}} A_{\alpha} \right) = \bigcup_{\alpha\in\mathscr{A}} \mathscr{C}A_{\alpha}
		      \end{align*}

		      If there exists \( \alpha \in \mathscr{A} \) such that \( A_{\alpha} = \varnothing \) then \( \bigcap_{\alpha\in\mathscr{A}} A_{\alpha} = \varnothing \in \mathscr{A}_{1} \). Otherwise, \( \mathscr{A} \) is finite and \( \aleph(\mathscr{C}A_{\alpha}) < \aleph(X) \) for each \( \alpha \).

		      We show that if \( X \) is infinite, \( A, B \subset X \) and \( \aleph(A) < \aleph(X), \aleph(B) < \aleph(X) \) then \( \aleph(A \cup B) < \aleph(X) \). If \( A, B \) are both finite then \( \aleph(A \cup B) < \aleph_{0} < \aleph(X) \). Otherwise, \( \aleph(A \cup B) = \max\left\{\aleph(A); \aleph(B)\right\} < \aleph(X) \).

		      Using mathematical induction, one can show that \( \aleph\left( \bigcup_{\alpha\in\mathscr{A}} \mathscr{C}A_{\alpha}\right) < \aleph(X) \).

		      In either case, it is true that \( \bigcap_{\alpha\in\mathscr{A}} A_{\alpha} \in \mathscr{A}_{1} \) for any finite \( \mathscr{A} \).

		      Thus \( \mathscr{A}_{1} \) is a topology.
	\end{enumerate}
\end{proof}

\begin{problem}{III.1.2}
How many distinct topologies can a set of three elements have? What is their partial ordering?
\end{problem}

\begin{proof}
	There are exactly 29 possible topologies on \( X = \left\{ a, b, c \right\} \).
	\begin{enumerate}
		\item \( \left\{ \varnothing, \left\{ a, b, c \right\} \right\} \)
		\item \( \left\{ \varnothing, \left\{ a \right\}, \left\{ a, b, c \right\} \right\} \)
		\item \( \left\{ \varnothing, \left\{ b \right\}, \left\{ a, b, c \right\} \right\} \)
		\item \( \left\{ \varnothing, \left\{ c \right\}, \left\{ a, b, c \right\} \right\} \)
		\item \( \left\{ \varnothing, \left\{ a, b \right\}, \left\{ a, b, c \right\} \right\} \)
		\item \( \left\{ \varnothing, \left\{ b, c \right\}, \left\{ a, b, c \right\} \right\} \)
		\item \( \left\{ \varnothing, \left\{ a, c \right\}, \left\{ a, b, c \right\} \right\} \)
		\item \( \left\{ \varnothing, \left\{ a \right\}, \left\{ a, b \right\}, \left\{ a, b, c \right\} \right\} \)
		\item \( \left\{ \varnothing, \left\{ b \right\}, \left\{ a, b \right\}, \left\{ a, b, c \right\} \right\} \)
		\item \( \left\{ \varnothing, \left\{ c \right\}, \left\{ a, b \right\}, \left\{ a, b, c \right\} \right\} \)
		\item \( \left\{ \varnothing, \left\{ a \right\}, \left\{ b \right\} \left\{ a, b \right\}, \left\{ a, b, c \right\} \right\} \)
		\item \( \left\{ \varnothing, \left\{ a \right\}, \left\{ b, c \right\}, \left\{ a, b, c \right\} \right\} \)
		\item \( \left\{ \varnothing, \left\{ b \right\}, \left\{ b, c \right\}, \left\{ a, b, c \right\} \right\} \)
		\item \( \left\{ \varnothing, \left\{ c \right\}, \left\{ b, c \right\}, \left\{ a, b, c \right\} \right\} \)
		\item \( \left\{ \varnothing, \left\{ b \right\}, \left\{ c \right\} \left\{ b, c \right\}, \left\{ a, b, c \right\} \right\} \)
		\item \( \left\{ \varnothing, \left\{ a \right\}, \left\{ a, c \right\}, \left\{ a, b, c \right\} \right\} \)
		\item \( \left\{ \varnothing, \left\{ b \right\}, \left\{ a, c \right\}, \left\{ a, b, c \right\} \right\} \)
		\item \( \left\{ \varnothing, \left\{ c \right\}, \left\{ a, c \right\}, \left\{ a, b, c \right\} \right\} \)
		\item \( \left\{ \varnothing, \left\{ a \right\}, \left\{ c \right\}, \left\{ a, c \right\}, \left\{ a, b, c \right\} \right\} \)
		\item \( \left\{ \varnothing, \left\{ b \right\}, \left\{ a, b \right\}, \left\{ b, c \right\}, \left\{ a, b, c \right\} \right\} \)
		\item \( \left\{ \varnothing, \left\{ a \right\}, \left\{ b \right\}, \left\{ a, b \right\}, \left\{ b, c \right\}, \left\{ a, b, c \right\} \right\} \)
		\item \( \left\{ \varnothing, \left\{ b \right\}, \left\{ c \right\}, \left\{ a, b \right\}, \left\{ b, c \right\}, \left\{ a, b, c \right\} \right\} \)
		\item \( \left\{ \varnothing, \left\{ c \right\}, \left\{ b, c \right\}, \left\{ a, c \right\}, \left\{ a, b, c \right\} \right\} \)
		\item \( \left\{ \varnothing, \left\{ a \right\}, \left\{ c \right\}, \left\{ b, c \right\}, \left\{ a, c \right\}, \left\{ a, b, c \right\} \right\} \)
		\item \( \left\{ \varnothing, \left\{ b \right\}, \left\{ c \right\}, \left\{ b, c \right\}, \left\{ a, c \right\}, \left\{ a, b, c \right\} \right\} \)
		\item \( \left\{ \varnothing, \left\{ a \right\}, \left\{ a, b \right\}, \left\{ a, c \right\}, \left\{ a, b, c \right\} \right\} \)
		\item \( \left\{ \varnothing, \left\{ a \right\}, \left\{ b \right\}, \left\{ a, b \right\}, \left\{ a, c \right\}, \left\{ a, b, c \right\} \right\} \)
		\item \( \left\{ \varnothing, \left\{ a \right\}, \left\{ c \right\}, \left\{ a, b \right\}, \left\{ a, c \right\}, \left\{ a, b, c \right\} \right\} \)
		\item \( \left\{ \varnothing, \left\{ a \right\}, \left\{ b \right\}, \left\{ c \right\}, \left\{ a, b \right\}, \left\{ b, c \right\}, \left\{ a, c \right\}, \left\{ a, b, c \right\} \right\} \)
	\end{enumerate}
\end{proof}

\begin{problem}{III.1.3}
Let \( \mathscr{T}_{X}, \mathscr{T}_{Y} \) be topologies in \( X, Y \), respectively. Is
\[ \mathscr{T} = \left\{ A\times B \mid A \in \mathscr{T}_{X}, B \in \mathscr{T}_{Y} \right\} \]

a topology in \( X\times Y \)?
\end{problem}

\begin{proof}
	In general, it is not a topology in \( X\times Y \). Here is a counterexample.

	\( X = Y = \left\{ 0, 1 \right\} \) and \( \mathscr{T}_{X} = \mathscr{T}_{Y} = \left\{ \varnothing, \left\{ 0 \right\}, \left\{ 1 \right\}, \left\{ 0, 1 \right\} \right\} \). Let \( A_{1} = B_{1} = \left\{ 0 \right\} \) and \( A_{2} = B_{2} = \left\{ 1 \right\} \) then \( A_{1} \times B_{1}, A_{2} \times B_{2} \in \mathscr{T} \) and \( A_{1} \times B_{1} \cup A_{2} \times B_{2} = \left\{ (0, 0), (1, 1) \right\} \), which is not a Cartesian product of any two sets as it contains \( (0, 0), (1, 1) \) but not \( (0, 1), (1, 0) \).
\end{proof}

\begin{problem}{III.1.4}\label{problem:III.1.4}
Let \( X \) be a partially ordered set. Define \( U \subset X \) to be open if it satisfies the condition: \( (x \in U) \land (y \prec x) \implies y \in U \). Show that \( \left\{ U \mid U \text{ is open} \right\} \) is a topology.
\end{problem}

This is in fact an Alexandrov topology, in which the intersection of arbitrarily many open sets is open.

\begin{proof}
	Let \( \mathscr{T} = \left\{ U \mid U \text{ is open} \right\} \) then \( \varnothing, X \in \mathscr{T} \) (the first one is vacuously true).

	Assume that \( {(U_{\alpha})}_{\alpha\in\mathscr{A}} \) is a family of open sets and denote \( U = \bigcup_{\alpha\in\mathscr{A}} U_{\alpha}, V = \bigcap_{\alpha\in\mathscr{A}} \).

	If \( x \in U \) and \( y \prec x \) then there exists \( \alpha \in \mathscr{A} \) such that \( x \in U_{\alpha} \). From the definition of open sets, we deduce that \( y \in U_{\alpha} \subset U \). Therefore \( U \) is open.

	If \( x \in V \) and \( y \prec x \) then from the definition of open sets, we deduce that \( y \in U_{\alpha} \subset U \) for every \( \alpha\in\mathscr{A} \). Therefore \( V \) is open.

	Thus \( \left\{ U \mid U \text{ is open} \right\} \) is a topology.
\end{proof}

\begin{problem}{III.1.5}\label{problem:III.1.5}
In \( \mathbb{Z}^{+} \), define \( U \subset \mathbb{Z}^{+} \) to be open if it satisfies the condition: \( n \in U \implies \) every divisor of \( n \) belongs to \( U \). Show that this is a topology in \( \mathbb{Z}^{+} \) and that it is not the discrete topology.
\end{problem}

\begin{proof}
	This is a particular case of the result in Problem~\ref{problem:III.1.4} as ``is a divisor of'' is a partial ordering on \( \mathbb{Z}^{+} \).

	This is not the discrete topology as it doesn't contain the subset of the form \( \left\{ n \right\} \) where \( n > 1 \).
\end{proof}

\begin{problem}{III.1.6}
Prove: \( \mathscr{T} \) is the discrete topology in \( X \) if and only if every point is an open set.
\end{problem}

\begin{proof}
	If \( \mathscr{T} \) is the discrete topology in \( X \) then every point is an open set by the definition of discrete topology.

	Conversely, if every point is an open set then every subset \( A \) of \( X \) is a union of open set, since \( A = \bigcup_{x\in A} \left\{x\right\} \), so \( A \in \mathscr{T} \) for every subset \( A \) of \( X \). Therefore \( \mathscr{T} \) is the discrete topology in \( X \).
\end{proof}

\section{Basis for a given topology}

\section{Topologizing of sets}

\begin{problem}{III.3.1}
Use 3.2 to verify the final statement of section 1, examples 4 and 5.
\end{problem}

\begin{proof}
	The family of open intervals in \( E^{1} \) is a basis for the Euclidean topology in \( E^{1} \).

	The family of open balls in \( E^{n} \) is a basis for the Euclidean topology in \( E^{n} \).

	We prove the latter as it is more general. Let \( \mathscr{B} \) be the collection of open balls in \( E^{n} \) and \( B(x_{1}; r_{1}), B(x_{2}; r_{2}) \in \mathscr{B} \). If \( x \in B_{1}(x_{1}; r_{1}) \cap B_{2}(x_{2}; r_{2}) \), then \( x \in B(x; r) \subset B_{1}(x_{1}; r_{1}) \cap B_{2}(x_{2}; r_{2}) \), in which \( r = \min\left\{ r_{1} - \left\vert x - x_{1} \right\vert, r_{2} - \left\vert x - x_{2} \right\vert \right\} \). Hence \( \mathscr{B} \) is a basis for some topology in \( E^{n} \). Denote this topology by \( \mathscr{T}(\mathscr{B}) \).

	According to Theorem 3.2, \( \mathscr{T}(\mathscr{B}) \) is unique and is the smallest topology containing \( \mathscr{B} \). Hence the Euclidean topology \( \mathscr{T} \) in \( E^{n} \) contains \( \mathscr{T}(\mathscr{B}) \). Moreover, if \( G \in \mathscr{T} \) then for each \( x \in G \), there exists \( r(x) > 0 \) such that \( x \in B(x; r(x)) \subset G \), from which we deduce that \( G = \bigcup_{x\in G} B(x; r(x)) \in \mathscr{T}(\mathscr{B}) \). Hence \( \mathscr{T} = \mathscr{T}(\mathscr{B}) \).
\end{proof}

\begin{problem}{III.3.2}
If, in the plane, all straight lines are taken as subbasis, what is the topology?
\end{problem}

\begin{proof}
	Since all straight lines are taken as subbasis, then the intersections of non-parallel straight lines are open, which means any point is open. Therefore the topology is the discrete topology.
\end{proof}

\begin{problem}{III.3.3}
Describe the open sets if all straight lines in the plane parallel to the \(x\)-axis are used for subbasis.
\end{problem}

\begin{proof}
	Each open set is a union of straight lines parallel to the \(x\)-axis.
\end{proof}

\begin{problem}{III.3.4}
Let \(X\) be the set of all \( (n\times n) \) matrices of real numbers. For each \( a = {(a_{ij})} \) and \( r > 0 \), let \( U_{r}(a) = \left\{ (b_{ij}) \mid \forall i, j: \left\vert a_{ij} - b_{ij} \right\vert < r \right\} \). Show that these sets are the basis for a topology in \( X \).
\end{problem}

\begin{proof}
	Assume that \( a \in U_{r_{1}}(a^{1}) \cap U_{r_{2}}(a^{2}) \) then \( a \in U_{r}(a) \subset U_{r_{1}}(a^{1}) \cap U_{r_{2}}(a^{2}) \) in which
	\[
		r = \min\limits_{i, j}\left\{ r_{1} - \left\vert a_{ij} - a^{1}_{ij} \right\vert, r_{2} - \left\vert a_{ij} - a^{2}_{ij} \right\vert \right\}
	\]

	so these sets are the basis for a topology in \(X\), according to Theorem 3.2.
\end{proof}

\begin{problem}{III.3.5}\label{problem:III.3.5}
Let \(C\) be the set of all continuous real-valued functions on \( [0, 1] \). For each \( f\in C \), each finite set \( x_{1}, \ldots, x_{n} \in [0, 1] \), and \( \varepsilon > 0 \), let \( U_{(x_{1}, \ldots, x_{n}, \varepsilon)}(f) = \left\{ g \mid \left\vert g(x_{i}) - f(x_{i}) \right\vert < \varepsilon, i = 1, \ldots, n \right\} \). Show (a) that these neighborhoods form a basis for some topology \( \mathscr{L} \); (b) that \( \mathscr{L} \subsetneq \mathscr{U} \); and (c) that \( \mathscr{L} \) and \( \mathscr{M} \) are not related in the partial ordering of topologies in \( C \).
\end{problem}

\begin{proof}
	\begin{enumerate}[label={(\alph*)}]
		\item Assume that \( h \in U_{(x_{1}, \ldots, x_{n}, \varepsilon)}(f) \cap U_{(y_{1}, \ldots, y_{m}, \delta)}(g) \). We define
		      \begin{multline*}
			      r = \min\Bigl\{ \varepsilon - \left\vert h(x_{1}) - f(x_{1}) \right\vert, \ldots, \varepsilon - \left\vert h(x_{n}) - f(x_{n}) \right\vert, \\
			      \delta - \left\vert h(y_{1}) - g(y_{1}) \right\vert, \ldots, \delta - \left\vert h(y_{m}) - g(y_{m}) \right\vert \Bigr\}
		      \end{multline*}

		      then \( h \in U_{(x_{1}, \ldots, x_{n}, y_{1}, \ldots, y_{m}, r)}(h) \subset U_{(x_{1}, \ldots, x_{n}, \varepsilon)}(f) \cap U_{(y_{1}, \ldots, y_{m}, \delta)}(g) \).

		      Hence the given neighborhoods form a basis for some topology \( \mathscr{L} \).
		\item Let \( U_{(x_{1}, \ldots, x_{n}, \varepsilon)}(f) \) be a basic open set of \( \mathscr{L} \) and \( g \in U_{(x_{1}, \ldots, x_{n}, \varepsilon)}(f) \) then let
		      \[
			      \delta = \min\limits_{i}\left\{ \varepsilon - \left\vert f(x_{i}) - g(x_{i}) \right\vert \right\}
		      \]

		      then \( g \in U(g, \delta) \subset U_{(x_{1}, \ldots, x_{n}, \varepsilon)}(f) \). Hence \( \mathscr{L} \subset \mathscr{U} \).

		      Choose \( f \equiv 0 \) and consider \( U(f, 1) \). Let \( g \) be an element of \( U(f, 1) \) then \( \sup\limits_{x}\left\vert f(x) - g(x) \right\vert < 1 \). Consider \( x_{1}, \ldots, x_{n} \in [0, 1] \) and \( \varepsilon > 0 \). There exists a continuous function \( h \in U_{(x_{1}, \ldots, x_{n}, \varepsilon)}(g) \) such that \( h(x_{i}) = g(x_{i}) \) for \( i = 1, \ldots, n \) and \( h(x) = 1 \) for some \( x \in [0, 1] \) other than \( x_{1}, \ldots, x_{n} \) (for example, a piecewise linear function). Hence \( h \in U_{(x_{1}, \ldots, x_{n}, \varepsilon)}(f) \) but \( h \notin U(f, 1) \), which means \( U_{(x_{1}, \ldots, x_{n}, \varepsilon)}(g) \not\subset U(f, 1) \) for every \( g \in U(f, 1) \), every finite subset \( \left\{ x_{1}, \ldots, x_{n} \right\} \subset [0, 1] \), every \( \varepsilon > 0 \). This implies that \( U(f, 1) \) doesn't contains any basic open set of \( \mathscr{L} \), which means \( \mathscr{U} \ne \mathscr{L} \).

		      Thus \( \mathscr{L} \) is strictly coarser (smaller) than \( \mathscr{U} \).
		\item Choose \( f \equiv 0 \).

		      Let \( g \in M(f, 1) \) and \( \left\{ x_{1}, \ldots, x_{n} \right\} \subset [0, 1], \varepsilon > 0 \). Without loss of generality, when \( n > 1 \), one can assume \( 0 \le x_{1} < x_{2} < \cdots < x_{n} \le 1 \). There exists a continuous function \( h \in C \) such that \( h(x_{i}) = g(x_{i}) \) for \( i = 1, \ldots, n \) and \( \int_{x_{1}}^{x_{2}}\left\vert h \right\vert = \int_{x_{1}}^{x_{2}}\left\vert h - f \right\vert = 1 \). When \( n = 1 \), we can choose \( h \in C \) such that \( h(x_{1}) = g(x_{1}) \) and \( \int_{x_{1}}^{1}\left\vert h \right\vert = 1 \) (if \( x_{1} \ne 1 \)) or \( \int_{0}^{x_{1}}\left\vert h \right\vert = 1 \) (if \( x_{1} = 1 \)). In either cases, \( h \in U_{(x_{1}, \ldots, x_{n}, \varepsilon)}(g) \) but \( h \notin M(f, 1) \).

		      Hence \( M(f, 1) \) doesn't contain any basic open set in \( \mathscr{L} \).

		      Let \( g \in U_{(x_{0}, 1)}(f) \) in which \( 0 < x_{0} < 1 \) and \( \varepsilon > 0 \). We define a function \( k \in C \). We can pick a real number \( \delta \) such that \( 0 < \delta < \min\left\{ \dfrac{\varepsilon}{1 + \left\vert 1 - g(x_{0}) \right\vert}, x_{0}, 1 - x_{0} \right\} \). Define \( k \in C \) as a piecewise linear function
		      \[
			      k(x) = \begin{cases}
				      0                                                      & 0 \le x < x_{0} - \delta       \\
				      \frac{1 - g(x_{0})}{\delta}(x - x_{0}) + 1 - g(x_{0})  & x_{0} - \delta \le x \le x_{0} \\
				      -\frac{1 - g(x_{0})}{\delta}(x - x_{0}) + 1 - g(x_{0}) & x_{0} \le x \le x_{0} + \delta \\
				      0                                                      & x_{0} + \delta < x \le 1
			      \end{cases}
		      \]

		      then \( \int_{0}^{1}\left\vert k \right\vert = \delta\left\vert 1 - g(x_{0}) \right\vert < \dfrac{\varepsilon}{1 + \left\vert 1 - g(x_{0})\right\vert} \cdot \left\vert 1 - g(x_{0}) \right\vert < \varepsilon \) so if we define \( h = g + k \) then \( h \in M(g, \varepsilon) \). However, \( h(x_{0}) - f(x_{0}) = h(x_{0}) = k(x_{0}) + g(x_{0}) = (1 - g(x_{0})) + g(x_{0}) = 1 \), which means \( h \notin U_{(x_{0}, 1)}(f) \).

		      So \( U_{(x_{0}, 1)}(f) \) doesn't contain any basic open set in \( \mathscr{M} \).

		      Thus \( \mathscr{L} \) and \( \mathscr{M} \) are not related in the partial ordering of topologies in \( C \).
	\end{enumerate}
\end{proof}

\begin{problem}{III.3.6}\label{problem:III.3.6}
Let \( X \) be a partially ordered set {\color{red}with at least two distinct and comparable elements}. Let \( U_{L}(x) = \{y \mid y \prec x\} \) and \( U_{R}(x) = \{y \mid x \prec y\} \). Show:
\begin{enumerate}[label={(\alph*)}]
	\item The families \(\{U_{L}(x)\}\), \(\{U_{R}(x)\}\) form bases for topologies \(\mathscr{T}_{L}\), \(\mathscr{T}_{R}\) in \(X\).
	\item \( G \in \mathscr{T}_{L} \) if and only if it satisfies the condition \( x \in G \implies U_{L}(x) \subset G \).
	\item In \(\mathscr{T}_{L}\), the arbitrary intersection of open sets is an open set.
	\item The discrete topology is the only one larger than \(\mathscr{T}_{L}\) and larger than \(\mathscr{T}_{R}\).
	\item \(\mathscr{T}_{L}\) and \(\mathscr{T}_{R}\) are not related in the partial ordering of the topologies on \(X\).
\end{enumerate}
\end{problem}

\begin{proof}
	\begin{enumerate}[label={(\alph*)}]
		\item If \( U_{L}(x) \) and \( U_{L}(y) \) are not disjoint then there exists \( z \in U_{L}(x) \cap U_{L}(y) \). Because \( \prec \) is transitive, we deduce that \( U_{L}(z) \subset U_{L}(x) \) and \( U_{L}(z) \subset U_{L}(y) \), which implies \( U_{L}(z) \subset U_{L}(x) \cap U_{L}(y) \). Therefore \( \left\{ U_{L}(x) \right\} \) forms a basis for a topology \( \mathscr{T}_{L} \) in \( X \).

		      If \( U_{R}(x) \) and \( U_{R}(y) \) are not disjoint then there exists \( z \in U_{R}(x) \cap U_{R}(y) \). Because \( \prec \) is transitive, we deduce that \( U_{R}(z) \subset U_{R}(x) \) and \( U_{R}(z) \subset U_{R}(y) \), which implies \( U_{R}(z) \subset U_{R}(x) \cap U_{R}(y) \). Therefore \( \left\{ U_{R}(x) \right\} \) forms a basis for a topology \( \mathscr{T}_{R} \) in \( X \).
		\item If \( G \in \mathscr{T}_{L} \) then \( G \) is a union of sets of the form \( U_{L}(y) \). In other words, \( G = \bigcup_{y \in I} U_{L}(y) \). If \( x \in G \) then there exists \( y \in I \) such that \( x \in U_{L}(y) \) so \( U_{L}(x) \subset U_{L}(y) \subset G \).

		      Conversely, assume that \( x \in G \implies U_{L}(x) \subset G \) then \( G = \bigcup_{x\in G} U_{L}(x) \), which means \( G \in \mathscr{T}_{L} \) since \( \left\{ U_{L}(x) \right\} \) is a basis for \( \mathscr{T}_{L} \).
		\item Firstly, we show that the intersection of arbitrary many basic open sets in \( \mathscr{T}_{L} \) is a basic open set in \( \mathscr{T}_{L} \). Let \( S \) be a subset of \( X \) and \( U = \bigcap_{x\in S} U_{L}(x) \). If \( U \) is empty then it is open. Otherwise, for each \( y \in U \), \( U_{L}(y) \subset U_{L}(x) \) for every \( x \in S \) so for each \( y \in U \), \( U_{L}(y) \subset U \). Hence \( U \in \mathscr{T}_{L} \).
		\item Assume that \( \mathscr{T} \) is a topology on \( X \) containing both \( \mathscr{T}_{L} \) and \( \mathscr{T}_{R} \). Let \( x \) be an arbitrary element of \( X \) then \( \left\{ x \right\} = U_{L}(x) \cap U_{R}(x) \) (due to antisymmetry), which means \( \left\{ x \right\} \in \mathscr{T} \) for every \( x \in X \). Therefore \( \mathscr{T} \) is the discrete topology on \( X \). Conversely, the discrete topology on \( X \) contains both \( \mathscr{T}_{L} \) and \( \mathscr{T}_{R} \). Thus the discrete topology is the only one larger than both \( \mathscr{T}_{L} \) and \( \mathscr{L}_{R} \).
		\item Let \( a, b \) be two distinct and comparable elements of \( X \). Without loss of generality, assume that \( a \prec b \). Moreover, \( U_{L}(a) \notin \mathscr{T}_{R} \) and \( U_{R}(b) \notin \mathscr{T}_{L} \) (according to part (b)). Thus \(\mathscr{T}_{L}\) and \(\mathscr{T}_{R}\) are not related in the partial ordering of the topologies on \(X\).
	\end{enumerate}
\end{proof}

\begin{problem}{III.3.7}
In \(\mathbb{Z}\), let \(p\) be a fixed prime. For each integer \(a > 0\), define \(U_{a}(n) = \{n + \lambda p^{a} \mid \lambda \in \mathbb{Z}\}\). Show that \(\{U_{a}(n)\}\) is a basis for some topology.
\end{problem}

\begin{proof}
	Assume that \( U_{a}(n) \cap U_{b}(m) \) is nonempty. Let \( x \in U_{a}(n) \cap U_{b}(m) \) then \( \lambda = \dfrac{x - n}{p^{a}} \) and \( \mu = \dfrac{x - m}{p^{b}} \) are integers. Without loss of generality, assume that \( a \le b \).

	\[ 0 = x - x = (n + \lambda p^{a}) - (m + \mu p^{b}) = (n - m) + (\lambda + \mu p^{b - a}) p^{a} \]

	so
	\[ m = n + (\lambda + \mu p^{b-a})p^{a} \in U_{a}(n) \]

	from which and \( a \le b \), we deduce that \( U_{b}(m) \subset U_{a}(n) \).

	Therefore \( U_{a}(n) \cap U_{b}(m) = U_{b}(m) \). Hence \( \left\{ U_{a}(n) \right\} \) is a basis for some topology.
\end{proof}

\begin{problem}{III.3.8}
Let \(\{\mathscr{T}_\alpha \mid \alpha \in \mathscr{A}\}\) be any family of topologies on \(X\). Define \(\bigvee_\alpha \mathscr{T}_\alpha\) to be the topology having \(\bigcup_\alpha \mathscr{T}_\alpha\) as subbasis.
\begin{enumerate}[label={(\alph*)}]
	\item Prove: \( \bigvee_{\alpha} \mathscr{T}_{\alpha} \) is the smallest of the topologies on \(X\) larger than every \( \mathscr{T}_{\alpha} \).
	\item Show that with the operations \( \mathscr{T}_{1} \vee \mathscr{T}_{2} \) and \( \mathscr{T}_{1} \cap \mathscr{T}_{2} \) that the topologies on \( X \) form a complete lattice.
	\item Is \( \mathscr{T}_{1} \cap (\mathscr{T}_{2} \vee \mathscr{T}_{3}) = (\mathscr{T}_{1} \cap \mathscr{T}_{2}) \vee (\mathscr{T}_{1} \cap \mathscr{T}_{3}) \) always true?
	\item What is \( \mathscr{T}_{L} \vee \mathscr{T}_{R} \) in Problem~\ref{problem:III.3.6}?
\end{enumerate}
\end{problem}

\begin{proof}
	\begin{enumerate}[label={(\alph*)}]
		\item \(\bigvee_\alpha \mathscr{T}_\alpha\) is the topology having \(\bigcup_\alpha \mathscr{T}_\alpha\) as subbasis so it is the smallest topology containing \(\bigcup_\alpha \mathscr{T}_\alpha\).

		      Assume that \( \mathscr{T} \) is a topology larger than every \( \mathscr{T}_{\alpha} \) then every finite intersection of open sets in \( \bigcup_{\alpha}\mathscr{T}_{\alpha} \) is in \( \mathscr{T} \) hence every union of finite intersections of open sets in \( \bigcup_{\alpha} U_{\alpha} \) is in \( \mathscr{T} \). Therefore \( \bigvee_\alpha \mathscr{T}_\alpha \subset \mathscr{T} \).

		      Thus \( \bigvee_\alpha \mathscr{T}_\alpha \) is the smallest of the topologies on \(X\) larger than every \( \mathscr{T}_{\alpha} \).
		\item Let \( T \) be a collection of topologies on \( X \).

		      \( \bigcap_{\mathscr{T} \in T} \mathscr{T} \) is the largest of the topologies on \( X \) smaller than every \( \mathscr{T} \in T \).

		      \( \bigvee_{\mathscr{T} \in T} \mathscr{T} \) is the smallest of the topologies on \( X \) larger than every \( \mathscr{T} \in T \).

		      Hence \( T \) has a greatest lower bound and a least upper bound, so the topologies on \(X\) with the partial ordering by inclusion form a complete lattice.
		\item No. It is not true in general. However, \( \mathscr{T}_{1} \cap (\mathscr{T}_{2} \vee \mathscr{T}_{3}) \supset (\mathscr{T}_{1} \cap \mathscr{T}_{2}) \vee (\mathscr{T}_{1} \cap \mathscr{T}_{3}) \) always holds.

		      For counterexamples, let \( X = \left\{ a, b, c \right\} \) and
		      \begingroup
		      \allowdisplaybreaks%
		      \begin{align*}
			      \mathscr{T}_{1} & = \left\{ \varnothing, X, \left\{ a \right\}, \left\{ a, b \right\} \right\} \\
			      \mathscr{T}_{2} & = \left\{ \varnothing, X, \left\{ a \right\}, \left\{ a, c \right\} \right\} \\
			      \mathscr{T}_{3} & = \left\{ \varnothing, X, \left\{ b \right\}, \left\{ b, c \right\} \right\}
		      \end{align*}
		      \endgroup

		      then
		      \begingroup
		      \allowdisplaybreaks%
		      \begin{align*}
			      \mathscr{T}_{1} \cap (\mathscr{T}_{2} \vee \mathscr{T}_{3})                        & = \mathscr{T}_{1} \cap \left\{ \varnothing, X, \left\{ a \right\}, \left\{ b \right\}, \left\{ c \right\}, \left\{ a, b \right\}, \left\{ b, c \right\}, \left\{ a, c \right\} \right\} \\
			                                                                                         & = \mathscr{T}_{1}                                                                                                                                                                       \\
			                                                                                         & = \left\{ \varnothing, X, \left\{ a \right\}, \left\{ a, b \right\} \right\},                                                                                                           \\
			      (\mathscr{T}_{1} \cap \mathscr{T}_{2}) \vee (\mathscr{T}_{1} \cap \mathscr{T}_{3}) & = \left\{ \varnothing, X, \left\{ a \right\} \right\} \vee \left\{ \varnothing, X \right\}                                                                                              \\
			                                                                                         & = \left\{ \varnothing, X, \left\{ a \right\} \right\}.
		      \end{align*}
		      \endgroup
		\item According to part (e) of Problem~\ref{problem:III.3.6} and part (a) of this problem, \( \mathscr{T}_{L} \vee \mathscr{T}_{R} \) is the discrete topology.
	\end{enumerate}
\end{proof}

\begin{problem}{III.3.9}
In the set \(\mathbb{R}\) of all real numbers, let \( \sigma \) consist of all sets of form \( \left\{ x \mid x > r \right\} \), \( \left\{ x \mid x < r \right\} \), where \( r, s \) are rational. Show that \( \mathscr{T}(\sigma) \) is the Euclidean topology of \( \mathbb{R} \). Is this still true if \( r, s \) are restricted to be numbers of the form \( k/2^{n} \) (\(k\) and \(n\) arbitrary {\color{red}integers})?
\end{problem}

\begin{proof}
	The open sets in \( \mathscr{T}(\sigma) \) are the unions of finite intersections of sets of the forms \( \left\{ x \mid x > r \right\} \) and \( \left\{ x \mid x < s \right\} \) in which \( r, s \) are rational. Hence the open sets in \( \mathscr{T}(\sigma) \) are the unions of sets of the form \( \openinterval{r, s} \), \( \left\{ x \mid x > r \right\} \), \( \left\{ x \mid x < s \right\} \) in which \( r, s \) are rational. So the collection \( \mathscr{C} \) of \( \openinterval{r, s} \), \( \left\{ x \mid x > r \right\} \), \( \left\{ x \mid x < s \right\} \) in which \( r, s \) are rational is a basis for \( \mathscr{T}(\sigma) \). For each \( C \in \mathscr{C} \) and for each \( x \in C \) there are \( a, b \in \mathbb{R} \) such that \( x \in \openinterval{a, b} \subset C \).

	The collection of open intervals is a basis for the Euclidean topology of \( \mathbb{R} \). Consider an open interval \( \openinterval{a, b} \) and an arbitrary \( x \in \openinterval{a, b} \). Since there is a rational number between two distinct real numbers, there exist rational numbers \( r \in \openinterval{a, x} \) and \( s \in \openinterval{x, b} \). Hence \( x \in \openinterval{r, s} \subset \openinterval{a, b} \).

	Thus \( \mathscr{T}(\sigma) \) is the Euclidean topology of \( \mathbb{R} \).

	The only property of rational numbers that we exploit in the above proof is that every open interval contains a rational number.

	We show that if two real numbers \( a, b \) are such that \( a < b \) then there exists a rational number of the form \( \dfrac{k}{2^{n}} \) in \( \openinterval{a, b} \). There exists an integer \( n \) such that \( \dfrac{1}{2^{n}} < b - a \) so \( 2^{n}a < 2^{n}a + 1 < 2^{n}b \). Let \( k = \left\lfloor 2^{n}a \right\rfloor + 1 \) then \( 2^{n}a < k \le 2^{n}a + 1 < 2^{n}b \), which implies \( a < \dfrac{k}{2^{n}} < b \).

	Hence every open interval contains a rational numbers of the form \( k/2^{n} \) (\(k\) and \(n\) are arbitrary integers). Thus the statement still holds if \( r, s \) are restricted to be numbers of the form \( k/2^{n} \) (\(k\) and \(n\) are arbitrary integers).
\end{proof}

\section{Elementary concepts}

\begin{problem}{III.4.1}
Determine the closure, derived set, interior, and boundary, of the following sets: (a) The rationals in \( E^{1} \); (b) the Cantor set in \(E^{1}\); (c) the set
\[
	\left\{ (r_{1}, r_{2}) \mid r_{1}, r_{2} \in \mathbb{Q} \right\} \subset E^{2};
\]

(d) \( \left\{ (x, 0) \mid 0 < x < 1 \right\} \subset E^{2} \).
\end{problem}

\begin{proof}
	\begin{enumerate}[label={(\alph*)},itemsep=0pt]
		\item Let \( U \) be a nonempty open set of \( E^{1} \) then \( U \) contains an interval \( \openinterval{a, b} \) where \( a < b \). Since there is a rational number \( q \in \openinterval{a, b} \), we conclude that \( U \cap \mathbb{Q} \ne \varnothing \). Therefore \( \mathbb{Q} \) is dense in \( E^{1} \), which means the closure of \( \mathbb{Q} \) in \( E^{1} \) is \( \mathbb{R} \).

		      For every \( x \in \mathbb{R} \), for every neighborhood \( U \) of \( x \), there exist \( a, b \in \mathbb{R} \) such that \( x \in \openinterval{a, b} \subset U \). The interval \( \openinterval{x, b} \subset \openinterval{a, b} \) contains a rational number and it is other than \( x \). Hence \( x \) is in the derived set of \( \mathbb{Q} \) in \( E^{1} \). Therefore the derived set of \( \mathbb{Q} \) in \( E^{1} \) is \( \mathbb{R} \).

		      Let \( r \in \mathbb{Q} \). For every neighborhood \( U \) of \( r \), there exist \( a, b \in \mathbb{R} \) such that \( r \in \openinterval{a, b} \subset U \). However, \( \openinterval{a, b} \) intersects both \( \mathbb{Q} \) and \( \mathbb{R} - \mathbb{Q} \), which means \( \mathbb{Q} \) has empty interior. This implies that \( \mathbb{R} - \mathbb{Q} \) is dense in \( E^{1} \).

		      The boundary of \( \mathbb{Q} \) is \( E^{1} \) is \( \overline{\mathbb{Q}} \cap \overline{\mathscr{C}\mathbb{Q}} = \overline{\mathbb{Q}} \cap \overline{\mathbb{R} - \mathbb{Q}} = \mathbb{R} \cap \mathbb{R} = \mathbb{R} \).
		\item We repeat the definition of the Cantor set.

		      Starting with \( \closedinterval{0, 1} \), we remove the middle-third open interval \( M_{1} = \openinterval{\frac{1}{3}, \frac{2}{3}} \).

		      Then we remove the middle-third open intervals from the remaining intervals \( \closedinterval{0, \frac{1}{3}} \) and \( \closedinterval{\frac{2}{3}, 1} \).

		      At the \( n \)-th stage, we remove the union \( M_{n} \) of the middle thirds of the remaining \( 2^{n-1} \) intervals.

		      The Cantor set is \( C = \closedinterval{0, 1} - \bigcup^{\infty}_{1} M_{n} \).

		      Since each \( M_{n} \) is open in \( E^{1} \), it follows that \( \bigcup^{\infty}_{1} M_{n} \) is open in \( E^{1} \), so \( C \) is closed in \( E^{1} \), which means the closure of \( C \) in \( E^{1} \) is \( C \).

		      Because \( C \) is closed in \( E^{1} \), the derived set of \( C \) is contained in \( C \). We will show that every point of \( C \) is a cluster point of \( C \). Let \( x \in C \) and \( U \) be a neighborhood of \(x\). There exists an open interval \( \openinterval{a, b} \) such that \( x \in \openinterval{a, b} \subset U \).
		      \[ C = \closedinterval{0, 1} - \bigcup^{\infty}_{1} M_{n} = \bigcap^{\infty}_{1}\left( \closedinterval{0, 1} - M_{n} \right) \]

		      For each \( n \), there exists a closed interval \( I_{n} \) in \( \closedinterval{0, 1} - M_{n} \) containing \( x \). We can choose a sufficiently large number \( n \) such that the length of \( I_{n} \) is less than \( x - a \) and \( b - x \). Hence \( x \in I_{n} \subset \openinterval{a, b} \). The endpoints of \( I_{n} \) are in \( C \) and other than \( x \) so \( \openinterval{a, b} \cap (C - \left\{x\right\}) \ne \varnothing \). Hence \( x \) is a cluster point of \( C \).

		      Thus the derived set of \( C \) is \( C \) itself.

		      Let \( x \) be a point in \( C \) and \( U \) a neighborhood of \( x \). Since \( U \) is open, there exists an open interval \( \openinterval{a, b} \) such that \( x \in \openinterval{a, b} \subset U \). The total length of closed intervals in \( \closedinterval{0, 1} - M_{n} \) is \( {(2/3)}^{n} \) and can be arbitrarily small, so there exists \( n \) such that \( {(2/3)}^{n} < b - a \), which implies \( \openinterval{a, b} \not\subset \closedinterval{0, 1} - M_{n} \) so \( \openinterval{a, b} \) is not contained in the Cantor set. Therefore \( C \) has empty interior.

		      The boundary of \( C \) is the complement of the interior of \( C \) in the closure of \( C \), so the boundary of \(C\) is \(C\) itself.
		\item The given set is dense in \( E^{2} \) so its closure is \( \mathbb{R}^{2} \).

		      Every point of \( E^{2} \) is a cluster point of the given set so its derived set is \( E^{2} \).

		      The interior of the given set is empty.

		      The boundary of the given set is \( \mathbb{R}^{2} \).
		\item The closure of the given set is \( \left\{ (x, 0) \mid 0 \le x \le 1 \right\} \).

		      The derived set of the given set is \( \left\{ (x, 0) \mid 0 \le x \le 1 \right\} \).

		      The interior of the given set is empty.

		      The boundary of the given set is \( \left\{ (x, 0) \mid 0 \le x \le 1 \right\} \).
	\end{enumerate}
\end{proof}

\begin{problem}{III.4.2}
Show that \( E^{s} \times 0 \subset E^{s} \times E^{t} = E^{s+t} \) is closed.
\end{problem}

\begin{proof}
	Let \( x \in E^{s+t} - (E^{s}\times 0) \) and \( y \) be the orthogonal projection of \( x \) on \( E^{s} \times 0 \), then
	\[
		x \in B(x; \left\vert x - y \right\vert) \subset E^{s+t} - (E^{s}\times 0)
	\]

	which implies that \( E^{s+t} - (E^{s}\times 0) \) is open in \( E^{s+t} \). Hence \( E^{s}\times 0 \) is closed in \( E^{s+t} \).
\end{proof}

\begin{problem}{III.4.3}
Let \( A \subset E^{1} \) be a bounded set. Show \( \sup A \in \overline{A} \). Under what condition is \( \sup A \in A^{\prime} \)\@?
\end{problem}

\begin{proof}
	From the definition of supremum, any real number strictly less than \( \sup A \) is not an upper bound of \( A \). Let \( U \) be a neighborhood of \( \sup A \) then there exists an open interval \( \openinterval{a, b} \) such that \( \sup A \in \openinterval{a, b} \subset U \). Since \( a < \sup A \) then there exists \( x \in A \) such that \( a < x \), from which we deduce that \( a < x \le \sup A < b \). Hence \( U \cap A \ne \varnothing \), which means \( \sup A \in \overline{A} \).

	\( \sup A \in A^{\prime} \) if and only if for every \( x < \sup A \), there exists \( y \in A \) such that \( x < y < \sup A \).
\end{proof}

\begin{problem}{III.4.4}
Prove: \(G\) is open in \(X\) if and only if \( \overline{G \cap \overline{A}} = \overline{G \cap A} \) for every \( A \subset X \).
\end{problem}

\begin{proof}
	\( G \) is open in \(X\) \( \implies \) \( \overline{G \cap \overline{A}} = \overline{G \cap A} \) for every \( A \subset X \).

	\( G \cap A \subset G \cap \overline{A} \) implies \( \overline{G \cap A} \subset \overline{G \cap \overline{A}} \). Assume that there exists \( x \in \overline{G \cap \overline{A}} \) such that \( x \notin \overline{G \cap A} \) then there is a neighborhood \(V\) of \(x\) such that \( V \cap (G \cap \overline{A}) \ne \varnothing \) and \( V \cap (G \cap \overline{A}) = \varnothing \). This is a contradiction, as any open set that intersects \( \overline{A} \) must intersects \( A \). Hence \( \overline{G \cap \overline{A}} = \overline{G \cap A} \).

	\( \overline{G \cap \overline{A}} = \overline{G \cap A} \) for every \( A \subset X \) \( \implies \)  \( G \) is open in \(X\).

	Choose \( A = \mathscr{C}G \) then \( \overline{G \cap \overline{\mathscr{C}G}} = \overline{G \cap (\mathscr{C}G)} = \overline{\varnothing} = \varnothing \). Therefore \( G \cap \overline{\mathscr{C}G} = \varnothing \). Moreover, \( \mathscr{C}\overline{\mathscr{C}G} = \operatorname{Int} G \) so \( G \subset \operatorname{Int} G \) and it follows from the definition of interior that \( G = \operatorname{Int} G \). Hence \( G \) is open in \( X \).
\end{proof}

\begin{problem}{III.4.5}
Show that \( (A^{\prime} = \varnothing) \implies (A \text{ is closed}) \).
\end{problem}

\begin{proof}
	Since \( \overline{A} = A \cup A^{\prime} \) it follows that \( A^{\prime} = \varnothing \) implies \( \overline{A} = A \), which means \( A \) is closed.
\end{proof}

\begin{problem}{III.4.6}
Let \(A = \left\{ 1/m + 1/n \mid m, n \in \mathbb{Z}^{+} \right\} \subset E^{1} \). Show that \( A^{\prime} = \left\{ 1/n \mid n \in \mathbb{Z}^{+} \right\} \cup \left\{ 0 \right\} \), \( A^{\prime\prime} = \left\{0\right\} \).
\end{problem}

\begin{proof}
	Every number in \( \left\{ 1/n \mid n \in \mathbb{Z}^{+} \right\} \cup \left\{0\right\} \) is a cluster point of \(A\).

	Let \(x\) be a real number such that there is no positive integer \(n\) such that \( x = \dfrac{1}{n} \). If \( x < 0 \) or \( x \ge 2 \), \( x \) is not a cluster point of \(A\). Consider \( x \in \openinterval{0, 2} \).

	Let \(p\) be the largest positive integer such that \( x < \dfrac{1}{p} \) then \( \dfrac{1}{p+1} < x < \dfrac{1}{p} \). Hence there exists \( \varepsilon > 0 \) such that \( \openinterval{x - \varepsilon, x + \varepsilon} \) doesn't contain any rational number of the form \( \dfrac{1}{q} \).

	We will show that \( \openinterval{x - \varepsilon/2, x + \varepsilon/2} \) has at most finitely many numbers of the form \( 1/m + 1/n \) in which \( m \le n \). Assume that \( \dfrac{1}{m} + \dfrac{1}{n} \in \openinterval{x - \varepsilon/2, x + \varepsilon/2} \). Because \( \openinterval{x - \varepsilon, x + \varepsilon} \) has no rational number of the form \(1/a\), then
	\[ \dfrac{1}{n} \le \dfrac{1}{m} \le x - \varepsilon \]

	and
	\[ 0 < x - \dfrac{\varepsilon}{2} < \dfrac{1}{m} + \dfrac{1}{n} \le \dfrac{2}{m} \]

	Moreover
	\[
		x - \dfrac{\varepsilon}{2} < \dfrac{1}{m} + \dfrac{1}{n} < x - \varepsilon + \dfrac{1}{n}
	\]

	which implies \( n < \dfrac{2}{\varepsilon} \).

	Hence \( m < \dfrac{2}{x - \dfrac{\varepsilon}{2}} \) and \( n < \dfrac{2}{\varepsilon} \), which means there are at most finitely many rational numbers of the form \( 1/m + 1/n \) in \( \openinterval{x - \dfrac{\varepsilon}{2}, x + \dfrac{\varepsilon}{2}} \). Hence \( x \) is not a cluster point of \(A\), thus \( A^{\prime} = \left\{ 1/n \mid n \in \mathbb{Z}^{+} \right\} \cup \left\{0\right\} \).

	The only cluster point of \(A^{\prime}\) is 0 so \( A^{\prime\prime} = \left\{0\right\} \).
\end{proof}

\begin{problem}{III.4.7}
Let \( \left\{ A_{\alpha} \mid \alpha \in \mathscr{A} \right\} \) be any family of sets in \(X\). Assume that \( \bigcup_{\alpha} \overline{A_{\alpha}} \) is closed. Prove \( \bigcup_{\alpha} \overline{A_{\alpha}} = \overline{\bigcup_{\alpha} A_{\alpha}} \).
\end{problem}

\begin{proof}
	If \( x \in \bigcup_{\alpha} \overline{A_{\alpha}} \) then there exists \( \alpha \in \mathscr{A} \) such that \( x \in \overline{A_{\alpha}} \). From the definition of closure, every neighborhood of \(x\) intersects \( A_{\alpha} \), hence every neighborhood of \(x\) intersects \( \bigcup_{\alpha} A_{\alpha} \). Therefore \( x \in \overline{\bigcup_{\alpha} A_{\alpha}} \), so \( \bigcup_{\alpha} \overline{A_{\alpha}} \subset \overline{\bigcup_{\alpha} A_{\alpha}} \).

	On the one hand, \( \bigcup_{\alpha} A_{\alpha} \) is a subset of \( \bigcup_{\alpha} \overline{A_{\alpha}} \) and \( \overline{\bigcup_{\alpha} A_{\alpha}} \). On the other hand, \( \overline{\bigcup_{\alpha} A_{\alpha}} \) is the smallest closed set containing \( \bigcup_{\alpha} A_{\alpha} \) and \( \bigcup_{\alpha} \overline{A_{\alpha}} \) is a closed set containing \( \bigcup_{\alpha} A_{\alpha} \). Therefore \( \bigcup_{\alpha} \overline{A_{\alpha}} = \overline{\bigcup_{\alpha} A_{\alpha}} \).
\end{proof}

\begin{problem}{III.4.8}
Prove: \( \operatorname{Fr}(A) = \varnothing \) if and only if \(A\) is both open and closed.
\end{problem}

\begin{proof}
	The complement of \( \overline{A} \) is \( \operatorname{Int}(\mathscr{C}A) \).

	\( \operatorname{Fr}(A) = \varnothing \implies A \) is both open and closed.

	From the definition of boundary, \( \varnothing = \operatorname{Fr}(A) = \overline{A} \cap \overline{\mathscr{C}A} \). On the other hand, \( \overline{A} \cup \overline{\mathscr{C}A} \supset A \cup \mathscr{C}A \) so the complement of \( \overline{A} \) is \( \overline{\mathscr{C}A} \). Hence \( \overline{\mathscr{C}A} = \operatorname{Int}(\mathscr{C}A) \) so \( \mathscr{C}A \) is both open and closed as \( \operatorname{Int}(\mathscr{C}A) \subset \mathscr{C}A \subset \overline{\mathscr{C}A} \). Therefore \(A\) is both open and closed.

	\( A \) is both open and closed \( \implies \operatorname{Fr}(A) = \varnothing \).

	Then \( \mathscr{C}A \) is both open and closed. If \( x \in A \) then there exists a neighborhood of \(x\) contained in \(A\) as \(A\) is open and this neighborhood doesn't intersect \(\mathscr{C}A\), so \( x \notin \overline{\mathscr{C}A} \). Otherwise, \( x \in \mathscr{C}A \) then there exists a neighborhood of \(x\) contained in \(\mathscr{C}A\) as \( A \) is closed and this neighborhood doesn't intersect \(A\), so \( x \notin \overline{A} \). In either cases, \( x \notin \overline{A} \cap \overline{\mathscr{C}A} = \operatorname{Fr}(A) \). Thus \( \operatorname{Fr}(A) = \varnothing \).
\end{proof}

\begin{problem}{III.4.9}
Prove the following formulas:
\begin{enumerate}[label={(\alph*)},itemsep=0pt]
	\item \( \operatorname{Fr}\left[ \operatorname{Fr}\left[ \operatorname{Fr}(A) \right] \right] = \operatorname{Fr}\left[ \operatorname{Fr}(A) \right] \);
	\item \( \operatorname{Fr}\left[ \operatorname{Int}(A) \right] \subset \operatorname{Fr}(A) \);
	\item \( \operatorname{Int}(A - B) \subset \operatorname{Int}(A) - \operatorname{Int}(B) \).
\end{enumerate}
\end{problem}

\begin{proof}
	We use the following facts about the boundary of a set \(S\)
	\begin{itemize}[leftmargin=*]
		\item \( \operatorname{Fr}(S) = \overline{S} \cap \overline{\mathscr{C}S} \) so \( \operatorname{Fr}(S) \subset \overline{S} \).
		\item \( \operatorname{Fr}(S) = \overline{S} \cap \overline{\mathscr{C}S} \) so every neighborhood of every point of \( \operatorname{Fr}(S) \) intersects both \( S \) and \( \mathscr{C}S \).
		\item \( \operatorname{Fr}(S) = \overline{S} - \operatorname{Int}(S) = \overline{S} \cap \mathscr{C}\left[ \operatorname{Int}(S) \right] \) so \( \operatorname{Fr}(S) \) is a closed set.
	\end{itemize}
	\begin{enumerate}[label={(\alph*)},itemsep=0pt,leftmargin=*]
		\item \( \operatorname{Fr}\left[\operatorname{Fr}\left[\operatorname{Fr}(A)\right]\right] = \overline{\operatorname{Fr}\left[\operatorname{Fr}(A)\right]} \cap \overline{\mathscr{C}\left[\operatorname{Fr}\left[\operatorname{Fr}(A)\right]\right]} = \operatorname{Fr}\left[\operatorname{Fr}(A)\right] \cap \overline{\mathscr{C}\left[\operatorname{Fr}\left[\operatorname{Fr}(A)\right]\right]} \subset \operatorname{Fr}\left[\operatorname{Fr}(A)\right] \).

		      Let \(x\) be a point of \( \operatorname{Fr}\left[\operatorname{Fr}(A)\right] \) then every neighborhood of \(x\) intersects both \( \operatorname{Fr}(A) \) and \( \mathscr{C}\left[\operatorname{Fr}(A)\right] \).
		      \begingroup
		      \allowdisplaybreaks%
		      \begin{align*}
			      \mathscr{C}\left[ \operatorname{Fr}\left[\operatorname{Fr}(A)\right] \right] & = \mathscr{C}\left[ \overline{\operatorname{Fr}(A)} \cap \overline{\mathscr{C}\left[ \operatorname{Fr}(A) \right]} \right]          \\
			                                                                                   & = \mathscr{C}\left[ \operatorname{Fr}(A) \cap \overline{\mathscr{C}\left[\operatorname{Fr}(A)\right]} \right]                       \\
			                                                                                   & = \mathscr{C}\left[\operatorname{Fr}(A)\right] \cup \mathscr{C}\left[\overline{\mathscr{C}\left[\operatorname{Fr}(A)\right]}\right]
		      \end{align*}
		      \endgroup

		      so every neighborhood of \(x\) intersects both \( \operatorname{Fr}\left[\operatorname{Fr}(A)\right] \) and \( \mathscr{C}\left[\operatorname{Fr}\left[\operatorname{Fr}(A)\right]\right] \). Therefore \( x \in \operatorname{Fr}\left[\operatorname{Fr}\left[\operatorname{Fr}(A)\right]\right] \).

		      Thus \( \operatorname{Fr}\left[\operatorname{Fr}\left[\operatorname{Fr}(A)\right]\right] = \operatorname{Fr}\left[\operatorname{Fr}(A)\right] \).
		\item \( \operatorname{Fr}\left[ \operatorname{Int}(A) \right] = \overline{\operatorname{Int}(A)} - \operatorname{Int}(\operatorname{Int}(A)) = \overline{\operatorname{Int}(A)} - \operatorname{Int}(A) \subset \overline{A} - \operatorname{Int}(A) = \operatorname{Fr}(A) \)
		\item \( \operatorname{Int}(A - B) \cup \operatorname{Int}(B) \subset (A - B) \cup B = A \) so \( \operatorname{Int}(A - B) \cup \operatorname{Int}(B) \) is an open set contained in \( A \). Because \( \operatorname{Int}(A) \) is the largest open set contained in \(A\), it follows that \( \operatorname{Int}(A - B) \cup \operatorname{Int}(B) \subset \operatorname{Int}(A) \).


		      Moreover \( A - B \) and \( B \) are disjoint, \( \operatorname{Int}(A - B) \subset A - B \) and \( \operatorname{Int}(B) \subset B \) so \( \operatorname{Int}(A - B) \) and \( \operatorname{Int}(B) \) are disjoint. Therefore \( \operatorname{Int}(A - B) = (\operatorname{Int}(A - B) \cup \operatorname{Int}(B)) - \operatorname{Int}(B) \subset \operatorname{Int}(A) - \operatorname{Int}(B) \).
	\end{enumerate}
\end{proof}

\begin{problem}{III.4.10}
Assume that \( \operatorname{Fr}(A) \cap \operatorname{Fr}(B) = \varnothing \). Prove: \( \operatorname{Int}(A \cup B) = \operatorname{Int}(A) \cup \operatorname{Int}(B) \) and \( \operatorname{Fr}(A \cap B) = \left[ \overline{A} \cap \operatorname{Fr}(B) \right] \cup \left[ \operatorname{Fr}(A) \cap \overline{B} \right] \).
\end{problem}

\begin{proof}
	We prove that \( \operatorname{Int}(A \cup B) = \operatorname{Int}(A) \cup \operatorname{Int}(B) \iff \operatorname{Fr}(A) \cap \operatorname{Fr}(B) \subset \operatorname{Fr}(A \cup B) \).

	\( \operatorname{Int}(A) \cup \operatorname{Int}(B) \) is an open set contained in \( A \cup B \) and \( \operatorname{Int}(A \cup B) \) is the largest open set contained in \( A \cup B \). Hence \( \operatorname{Int}(A) \cup \operatorname{Int}(B) \subset \operatorname{Int}(A \cup B) \).

	Suppose that \( \operatorname{Int}(A \cup B) = \operatorname{Int}(A) \cup \operatorname{Int}(B) \). Let \( x \in \operatorname{Fr}(A) \cap \operatorname{Fr}(B) \) then every neighborhood of \( x \) intersects \( A, B, \mathscr{C}A, \mathscr{C}B \) hence intersects \( A \cup B, \mathscr{C}\left[A \cup B\right] = \mathscr{C}A \cap \mathscr{C}B \). Therefore \( x \in \operatorname{Fr}(A \cup B) \), which means \( \operatorname{Fr}(A) \cap \operatorname{Fr}(B) \subset \operatorname{Fr}(A \cup B) \).

	Conversely, suppose that \( \operatorname{Fr}(A) \cap \operatorname{Fr}(B) \subset \operatorname{Fr}(A \cup B) \). Let \( x \in \operatorname{Int}(A \cup B) \) then \( x \in \overline{A \cup B} - \operatorname{Fr}(A \cup B) \), which means \( x \notin \operatorname{Fr}(A \cup B) \) then \( x \notin \operatorname{Fr}(A) \cap \operatorname{Fr}(B) \). Without loss of generality, assume that \( x \notin \operatorname{Fr}(A) \) then either \( x \in \operatorname{Int}(A) \) or \( x \in \operatorname{Int}(\mathscr{C}A) \).

	If \( x \in \operatorname{Int}(A) \) then \( x \in \operatorname{Int}(A) \cup \operatorname{Int}(B) \).

	If \( x \in \operatorname{Int}(\mathscr{C}A) \).
	\begingroup
	\allowdisplaybreaks%
	\begin{align*}
		x \in \operatorname{Int}(A \cup B) & \subset A \cup B \subset \overline{A} \cup \overline{B}                                                                          \\
		                                   & = \left(\operatorname{Int}(A) \cup \operatorname{Fr}(A)\right) \cup \left(\operatorname{Int}(B) \cup \operatorname{Fr}(B)\right) \\
		                                   & = \left(\operatorname{Int}(A) \cup \operatorname{Int}(B)\right) \cup \left(\operatorname{Fr}(A) \cup \operatorname{Fr}(B)\right)
	\end{align*}
	\endgroup

	since \( x \notin \operatorname{Fr}(A) \cup \operatorname{Fr}(B) \), we deduce that \( x \in \operatorname{Int}(A) \cup \operatorname{Int}(B) \).

	In either cases, \( x \in \operatorname{Int}(A) \cup \operatorname{Int}(B) \) so \( \operatorname{Int}(A \cup B) \subset \operatorname{Int}(A) \cup \operatorname{Int}(B) \). Hence \( \operatorname{Int}(A \cup B) = \operatorname{Int}(A) \cup \operatorname{Int}(B) \).

	So if \( \operatorname{Fr}(A) \cap \operatorname{Fr}(B) = \varnothing \) then \( \operatorname{Int}(A \cup B) = \operatorname{Int}(A) \cup \operatorname{Int}(B) \).

	\hrulefill%

	If \( \operatorname{Fr}(A) \cap \operatorname{Fr}(B) = \varnothing \) then \( \overline{A \cap B} = \overline{A} \cap \overline{B} \) because
	\begingroup
	\allowdisplaybreaks%
	\begin{align*}
		\overline{A} \cap \overline{B} & = \left( A \cup \operatorname{Fr}(A) \right) \cap \left( B \cup \operatorname{Fr}(B) \right)                                                          \\
		                               & = (A \cap B) \cup (A \cap \operatorname{Fr}(B)) \cup (\operatorname{Fr}(A) \cap B) \cup \left( \operatorname{Fr}(A) \cap \operatorname{Fr}(B) \right) \\
		                               & = \overline{A \cap B} \cup \left( \operatorname{Fr}(A) \cap \operatorname{Fr}(B) \right).
	\end{align*}
	\endgroup

	Therefore
	\begingroup
	\allowdisplaybreaks%
	\begin{align*}
		\operatorname{Fr}(A \cap B) & = \overline{A \cap B} - \operatorname{Int}(A \cap B)                                                                                                                                                         \\
		                            & = \overline{A \cap B} - (\operatorname{Int}(A) \cap \operatorname{Int}(B))                                                                                                                                   \\
		                            & = \overline{A \cap B} \cap \mathscr{C}\left[ \operatorname{Int}(A) \cap \operatorname{Int}(B) \right]                                                                                                        \\
		                            & = \overline{A} \cap \overline{B} \cap \left( \mathscr{C}\left[ \operatorname{Int}(A) \right] \cup \mathscr{C}\left[ \operatorname{Int}(B) \right] \right)                                                    \\
		                            & = \left( \overline{A} \cap \overline{B} \cap \mathscr{C}\left[ \operatorname{Int}(A) \right] \right) \cup \left( \overline{A} \cap \overline{B} \cap \mathscr{C}\left[ \operatorname{Int}(B) \right] \right) \\
		                            & = \left( \operatorname{Fr}(A) \cap \overline{B} \right) \cup \left( \overline{A} \cap \operatorname{Fr}(B) \right).
	\end{align*}
	\endgroup

	Hence \( \operatorname{Fr}(A) \cap \operatorname{Fr}(B) = \varnothing \) implies \( \operatorname{Fr}(A \cap B) = \left( \operatorname{Fr}(A) \cap \overline{B} \right) \cup \left( \overline{A} \cap \operatorname{Fr}(B) \right) \).
\end{proof}

\begin{problem}{III.4.11}
For what spaces \(X\) is the only dense set \(X\) itself?
\end{problem}

\begin{proof}
	Assume that \(X\) is the only dense set in \(X\) then for every \( x \in X \), \( X - \left\{ x \right\} \) is not dense in \( X \), which means the closure of \( X - \left\{ x \right\} \) in \(X\) is \( X - \left\{x\right\} \). Therefore \( X - \left\{x\right\} \) is closed, so \( \left\{ x \right\} \) is open in \(X\) for every \(x\in X\). Hence \(X\) is a discrete space.

	Conversely, if \(X\) is a discrete space then \(X\) is the only dense set in \(X\).
\end{proof}

\begin{problem}{III.4.12}
Let \(E\) and \(G\) be dense in \(X\). Prove: If \(E\) and \(G\) are open, then \(E \cap G\) is also dense in \(X\).
\end{problem}

\begin{proof}
	\( E \) and \( G \) are dense open subsets of \(X\) so \( E \cap G \) is nonempty. Let \( A \) be a nonempty open subset of \( X \). Because \( E \) is dense in \( X \), \( A \cap E \) is nonempty. Moreover, \( A \cap E \) is open and \( G \) is dense so \( (A \cap E) \cap G \) is nonempty. Therefore \( A \cap (E \cap G) \) is nonempty. By the definition of dense set, we conclude that \( E \cap G \) is dense in \( X \).
\end{proof}

\begin{problem}{III.4.13}\label{problem:III.4.13}
Let \(D\) be dense in \(X\). Prove: \( \overline{D \cap G} = \overline{G} \) for every open set \( G \subset X \).
\end{problem}

\begin{proof}
	\( D \cap G \subset G \) so \( \overline{D \cap G} \subset \overline{G} \).

	Let \( x \in \overline{G} \) then every neighborhood \(V\) of \(x\) intersects \( G \). Moreover, \( V \cap G \) is nonempty and it is open because \( V, G \) are open so \( V \cap G \) intersects \( D \) as \( D \) is dense in \( X \). Hence \( V \) intersects \( D \cap G \), which means \( x \in \overline{D \cap G} \). Therefore \( \overline{G} = \overline{D \cap G} \).
\end{proof}

\begin{problem}{III.4.14}
Let \( \mathscr{B} \) be a subbasis for \(X\), and \(D \subset X\) such that \( U \cap D \ne \varnothing \) for each \( U \in \mathscr{B} \). Does this imply that \(D\) is dense in \(X\)?
\end{problem}

\begin{proof}
	No. It is not necessarily true. Here is a counterexample.

	\( \mathbb{Z} \subset \mathbb{R} \). A subbasis for the Euclidean topology on \( \mathbb{R} \) is the collection of set of the forms \( \left\{ x \mid x > b \right\} \) and \( \left\{ x \mid x < a \right\} \). The set \( \mathbb{Z} \) intersects every set in the given subbasis but \( \mathbb{Z} \) is not dense in \( \mathbb{R} \).
\end{proof}

\begin{problem}{III.4.15}
In Problem~\ref{problem:III.3.6}, what is the closure of \( \left\{ x \right\} \) in \( \mathscr{T}_{L} \)? For what points is \( \left\{ x \right\} \) a closed set? An open set?
\end{problem}

\begin{proof}
	The closure of \( \left\{ x \right\} \) in \( \mathscr{T}_{L} \) is the intersection of closed sets containing \( x \), hence the complement of the union of open sets not containing \(x\). Therefore the closure of \( \left\{ x \right\} \) is the complement of the union of \( U_{L}(y) \) in which \( x \notin U_{L}(y) \).

	Assume that \( \left\{ x \right\} \) is a closed set then \( \mathscr{C}(\left\{ x \right\}) \) is open. Hence for every \( y \ne x \), \( x \notin U_{L}(y) \), which means for each such \( U_{L}(y) \), for every \( z \in U_{L}(y) \), either \( z \) and \( x \) are incomparable or \( (z \prec x) \land (z \ne x) \). This means \( x \) is a maximal element of \( X \). Conversely, if \( x \) is a maximal element of \( X \) then \( \left\{ x \right\} \) is a closed set.

	\( \left\{ x \right\} \) is an open set if and only if \( \left\{ x \right\} = U_{L}(x) \), which precisely means \( x \) is a minimal element of \( X \).
\end{proof}

\begin{problem}{III.4.16}
Assume \( (X, \mathscr{T}) \) a space with the properties: (a) The intersection of any family of open sets is open; (b) if \(x\ne y\), then there is at least one open set containing some of of \(x, y\), and not the other. Define \( x\le y \) if \( x \in \overline{\left\{y\right\}} \). Show that this is a partial ordering and that the topology \( \mathscr{T}_{R} \) (Problem~\ref{problem:III.3.6}) is precisely \( \mathscr{T} \).
\end{problem}

\begin{proof}
	\( x \in \overline{\left\{ x \right\}} \) so \( x \le x \).

	If \( x \in \overline{\left\{ y \right\}} \) and \( y \in \overline{ \left\{z\right\} } \) then \( \overline{\left\{ z\right\} } \) is a closed set containing \( \left\{ y \right\} \) so \( \overline{ \left\{ y \right\} } \subset \overline{ \left\{ z \right\} } \). Therefore \( x \in \overline{ \left\{ z \right\} } \) which precisely means \( x \le z \).

	Suppose that \( x \le y \) and \( y \le x \). Assume for the sake of contradiction that \( x \ne y \) then there is a neighborhood \( U_{x} \) of \(x\) not containing \( y \) or there is a neighborhood \( U_{y} \) of \(y\) not containing \( x \). In the former case, \( x \in \operatorname{Ext}(\left\{y\right\}) \) and \( x \in \overline{ \left\{y\right\} } \). In the latter case, \( y \in \operatorname{Ext}(\left\{x\right\}) \) and \( y \in \overline{ \left\{x\right\} } \). In either cases, we obtain a contradiction. Hence \( x = y \).

	Thus \( \le \) is a partial ordering.

	\hrulefill%

	We will show that \( \mathscr{T}_{R} = \mathscr{T} \).

	Let \( G \in \mathscr{T} \) then for every \( x \in G, y \notin G \), one has \( \overline{ \left\{y\right\} } \subset \mathscr{C}(G) \) and \( x \notin \overline{ \left\{y\right\} } \), which means \( x \nleq y \), so \( y \notin U_{R}(x) \). Therefore \( \mathscr{C}(G) \subset \mathscr{C}(U_{R}(x)) \) whenever \( x \in G \), which implies \( U_{R}(x) \subset G \) whenever \( x \in G \).

	Let \( N(x) \) be the intersection of all neighborhoods of \(X\) then \( N(x) \) is open as the intersection of any family of open sets is open. Hence \( N(x) \) is the smallest neighborhood of \(x\). In the above paragraph, we showed that \( U_{R}(x) \subset N(x) \). Conversely let \( y \in N(x) \). Assume that \( x \nleq y \) then \( x \notin \overline{\left\{ y \right\}} \) so there is a neighborhood of \( x \) not containing \( y \), which contradicts \( y \in N(x) \) and the definition of \(N(x)\). Hence \( x \le y \), which implies \( y \in U_{R}(x) \). Therefore \( N(x) \subset U_{R}(x) \). Thus \( U_{R}(x) = N(x) \), so \( U_{R}(x) \) is open in \( \mathscr{T} \).

	\( \left\{ U_{R}(x) \right\} \subset \mathscr{T} \) and for every \( G \in \mathscr{T} \), every \( x \in G \), one has \( x \in U_{R}(x) \subset G \). This means \( \left\{ U_{R}(x) \right\} \) is a basis for \( \mathscr{T} \). Hence \( \mathscr{T} = \mathscr{T}_{R} \).
\end{proof}

\begin{problem}{III.4.17}
In \(X\times Y\), show that \( \overline{A \times B} = \overline{A} \times \overline{B} \); \( \operatorname{Int}(A \times B) = \operatorname{Int}(A) \times \operatorname{Int}(B) \); \( \operatorname{Fr}(A \times B) = (\operatorname{Fr}(A) \times \overline{B}) \cup (\overline{A} \times \operatorname{Fr}(B)) \).
\end{problem}

\begin{proof}
	Suppose that \( (x, y) \in \overline{A \times B} \). Let \( U \) be a neighborhood of \(x\) and \(V\) be a neighborhood of \(y\) then \( U \times V \) is a neighborhood of \( (x, y) \), so \( U \times V \) intersects \( A \times B \). Since \( (U \times V) \cap (A \times B) = (U \cap A) \times (V \cap B) \) so \( U \cap A \ne \varnothing \) and \( V \cap B \ne \varnothing \). Therefore \( x \in \overline{A} \) and \( y \in \overline{B} \), which implies \( (x, y) \in \overline{A} \times \overline{B} \), so \( \overline{A \times B} \subset \overline{A} \times \overline{B} \).

	Suppose that \( (x, y) \in \overline{A} \times \overline{B} \). Let \( S \) be a neighborhood of \( (x, y) \) then there exists a basic open set \( U \times V \) such that \( (x, y) \in U \times V \subset S \). Because \( U \cap A \ne \varnothing \) and \( V \cap B \ne \varnothing \), it follows that \( U \times V \) intersects \( A \times B \), which means \( S \) intersects \( A \times B \). Hence \( (x, y) \in \overline{A \times B} \) and \( \overline{A} \times \overline{B} \subset \overline{A \times B} \).

	Thus \( \overline{A \times B} = \overline{A} \times \overline{B} \).

	\hrulefill%

	Suppose that \( (x, y) \in \operatorname{Int}(A\times B) \) then there exists a neighborhood \(S\) of \( (x, y) \) contained in \( \operatorname{Int}(A\times B) \). There is a basic open set \( U\times V \) such that \( (x, y) \in U \times V \subset S \). Because \( U \times V \subset S \subset A \times B \) and \( U, V \) are nonempty, it follows that \( U \subset A \) and \( V \subset B \). So \( U \) is a neighborhood of \( x \) contained in \( A \) and \( V \) is a neighborhood of \(y\) contained in \(B\), therefore \( (x, y) \in \operatorname{Int}(A) \times \operatorname{Int}(B) \).

	Suppose that \( (x, y) \in \operatorname{Int}(A) \times \operatorname{Int}(B) \) then there is a neighborhood \(U\) of \(x\) and a neighborhood \(V\) of \(y\) such that \( U \subset A \) and \( V \subset B \). Therefore \( (x, y) \in U\times V \subset A\times B \). Moreover, \( U\times V \) is a neighborhood of \( (x, y) \) so \( (x, y) \in U \times V \subset \operatorname{Int}(A \times B) \).

	Thus \( \operatorname{Int}(A\times B) = \operatorname{Int}(A) \times \operatorname{Int}(B) \).

	\hrulefill%
	\begingroup
	\allowdisplaybreaks%
	\begin{align*}
		\operatorname{Fr}(A \times B) & = \overline{A \times B} \cap \overline{\mathscr{C}(A \times B)}                                                                                                                                              \\
		                              & = (\overline{A} \times \overline{B}) \cap \overline{\mathscr{C}(A) \times B \cup A \times \mathscr{C}(B)}                                                                                                    \\
		                              & = (\overline{A} \times \overline{B}) \cap \left( \overline{\mathscr{C}(A) \times B} \cup \overline{A\times \mathscr{C}(B)} \right)                                                                           \\
		                              & = (\overline{A} \times \overline{B}) \cap \left( \overline{\mathscr{C}(A)} \times \overline{B} \cup \overline{A} \times \overline{\mathscr{C}(B)} \right)                                                    \\
		                              & = \left( \overline{A} \times \overline{B} \cap \overline{\mathscr{C}(A)} \times \overline{B} \right) \cup \left( \overline{A} \times \overline{B} \cap \overline{A} \times \overline{\mathscr{C}(B)} \right) \\
		                              & = \left( \operatorname{Fr}(A) \times \overline{B} \right) \cup \left( \overline{A} \times \operatorname{Fr}(B) \right).
	\end{align*}
	\endgroup

	Thus \( \operatorname{Fr}(A \times B) = \left( \operatorname{Fr}(A) \times \overline{B} \right) \cup \left( \overline{A} \times \operatorname{Fr}(B) \right) \).
\end{proof}

\begin{problem}{III.4.18}
The exterior, \( \operatorname{Ext}(A) \), of a set \(A \subset X\) is defined by \( \operatorname{Ext}(A) = \operatorname{Int}(\mathscr{C}A) \). Prove: (a) \( \operatorname{Ext}(A \cup B) = \operatorname{Ext}(A) \cap \operatorname{Ext}(B) \); (b) \( A \cap \operatorname{Ext}(A) = \varnothing \); (c) \( X = \operatorname{Ext}(\varnothing) \); (d) \( \operatorname{Ext}(\mathscr{C}\operatorname{Ext}(A)) = \operatorname{Ext}(A) \).
\end{problem}

\begin{proof}
	\begin{enumerate}[label={(\alph*)},leftmargin=*,itemsep=0pt]
		\item \begingroup
		      \allowdisplaybreaks%
		      \begin{align*}
			      \operatorname{Ext}(A \cup B) & = \operatorname{Int}(\mathscr{C}(A \cup B))                              \\
			                                   & = \operatorname{Int}(\mathscr{C}(A) \cap \mathscr{C}(B))                 \\
			                                   & = \operatorname{Int}(\mathscr{C}A) \cap \operatorname{Int}(\mathscr{C}B) \\
			                                   & = \operatorname{Ext}(A) \cap \operatorname{Ext}(B).
		      \end{align*}
		      \endgroup
		\item \( \operatorname{Ext}(A) = \operatorname{Int}(\mathscr{C}A) = \mathscr{C}\overline{A} \subset \mathscr{C}A \) so \( A \cap \operatorname{Ext}(A) \subset A \cap \mathscr{C}A = \varnothing \).

		      Therefore \( A \cap \operatorname{Ext}(A) = \varnothing \).
		\item \( \operatorname{Ext}(\varnothing) = \operatorname{Int}(\mathscr{C}\varnothing) = \operatorname{Int}(X) = X \).
		\item \( \operatorname{Ext}(\mathscr{C}\operatorname{Ext}(A)) = \operatorname{Ext}(A) = \operatorname{Int}(\operatorname{Ext}(A)) = \operatorname{Ext}(A) \).
	\end{enumerate}
\end{proof}

\begin{problem}{III.4.19}
If every countable subset of a space is closed, is the topology necessarily discrete?
\end{problem}

\begin{proof}
	No. It is not necessary. Here we provide a counterexample.

	Let \(X\) be an uncountable set. We define
	\[
		\mathscr{T} = \left\{ \varnothing, X, \text{ all subsets of \(X\) whose complement is countable } \right\}
	\]

	If \( {(A_{\alpha})}_{\alpha\in \mathscr{A}} \) is a family of elements of \( \mathscr{T} \) then
	\[
		X - \bigcup_{\alpha\in\mathscr{A}} A_{\alpha} = \bigcap_{\alpha\in\mathscr{A}}(X - A_{\alpha}) \text{ is countable}
	\]

	so \( \bigcup_{\alpha\in\mathscr{A}} A_{\alpha} \in \mathscr{T} \). Moreover, if \( A, B \in \mathscr{T} \) then
	\[
		X - (A \cap B) = (X - A) \cup (X - B) \text{ is countable}
	\]

	so \( A \cap B \in \mathscr{T} \). Hence \( \mathscr{T} \) is a topology on \(X\).

	Every singleton subset of \(X\) is closed. Let \(a, b\) be two distinct points of \(X\). There exist neighborhoods \(A \ni a\) and \(B \ni b\) such that \(b \notin A\) and \(a \notin B\). The complement \( X - (A \cap B) = (X - A) \cup (X - B) \) is countable so \( A \cap B \) is nonempty. Hence \( X \) is not discrete as any two distinct points doesn't have disjoint neighborhoods.
\end{proof}

\begin{problem}{III.4.20}
A point \(a \in A\) is called isolated whenever \(a \in A - A^{\prime}\). A set is called perfect if it is closed and has no isolated points. Prove: (1) If \(A\) has no isolated points, then \(\overline{A}\) is perfect. (2) If a space \(X\) has no isolated points, then every open set in \(X\) also have no isolated points.
\end{problem}

\begin{quotation}
	If a space \(X\) has no isolated points, then it is not necessary that every dense subset of \(X\) has no isolated points. A sufficient condition for this is \(X\) being a \(T_{1}\) space (for every pair of distinct points, each has a neighborhood not containing the other point).
\end{quotation}

\begin{proof}
	\begin{enumerate}[label={(\arabic*)}]
		\item \( \overline{A} \) is closed by definition. Let \( x \in \overline{A} \).

		      If \( x \in A \) then \(x\) is not an isolated point of \(A\) hence not an isolated point of \(\overline{A}\).

		      If \( x \notin A \) then \( x \in A^{\prime} \) so every neighborhood of \(x\) intersects \( A = A - \left\{x\right\} \), hence \(x\) is not an isolated point of \(\overline{A}\).

		      Therefore \(\overline{A}\) doesn't have any isolated points, thus \(\overline{A}\) is perfect if \(A\) has no isolated points.
		\item Let \(U\) be an open set of \(X\) and \(x\) an element of \(U\). For every neighborhood \(V\) of \(x\), one has \( (U \cap V) - \left\{x\right\} \ne \varnothing \) because \(X\) has no isolated points. Hence every neigborhood of \(x\) intersects \( U - \left\{x\right\} \), which means every element of \(U\) is a cluster point, hence \(U\) has no isolated point.
	\end{enumerate}
\end{proof}

\begin{problem}{III.4.21}
Call a set residual if its complement is dense, and call it nowhere dense if its closure has empty interior. Prove: (1) A nowhere dense set is a redisual set. (2) \(A\) is nowhere dense if and only if \( A \subset \overline{\mathscr{C}(\overline{A})} \). (3) The union of a residual and a nowhere dense set is a residual set. (4) The boundary of a closed (or open) set is nowhere dense. (5) For any set \(A\), both \(A \cap \overline{\mathscr{C}(A)}\) and \( \overline{A} \cap \mathscr{C}(A) \) are residual. (6) The boundary of any set is the union of two residual sets.
\end{problem}

\begin{proof}
	\begin{enumerate}[label={(\arabic*)},itemsep=0pt,leftmargin=*]
		\item Let \(A\) be a nowhere dense subset of a topological space. From the definition of nowhere dense set, \( \operatorname{Int}(\overline{A}) = \varnothing \).

		      The complement of \( \overline{\mathscr{C}(A)} \) is the interior of \(A\). Moreover, \( \operatorname{Int}(A) \subset \operatorname{Int}(\overline{A}) = \varnothing \) so \( \operatorname{Int}(A) = \varnothing \). Hence \( \overline{\mathscr{C}(A)} = X \), which means \(A\) is also a residual set.
		\item If \(A\) is nowhere dense then \( \operatorname{Int}(\overline{A}) = \varnothing \). Since \( \overline{\mathscr{C}(\overline{A})} = \mathscr{C}(\operatorname{Int}(\overline{A})) \) then \( \overline{\mathscr{C}(\overline{A})} \) is the entire ambient space, which means \( A \subset \overline{\mathscr{C}(\overline{A})} \).

		      Conversely, suppose that \( A \subset \overline{\mathscr{C}(\overline{A})} \). Because \( \overline{\mathscr{C}(\overline{A})} = \mathscr{C}(\operatorname{Int}(\overline{A})) \) then \( A \subset \mathscr{C}(\operatorname{Int}(\overline{A})) \), so \( \operatorname{Int}(\overline{A}) \subset \mathscr{C}A \), which means \( \operatorname{Int}(\overline{A}) \subset \operatorname{Int}(\mathscr{C}A) = \mathscr{C}(\overline{A}) \). Since \( \operatorname{Int}(\overline{A}) \subset \overline{A} \) and \( \mathscr{C}(\overline{A}) \) are disjoint, we deduce that \( \operatorname{Int}(\overline{A}) = \varnothing \). Hence \(A\) is nowhere dense.
		\item Let \(A\) be a residual set and \(B\) a nowhere dense set.

		      \( \mathscr{C}A \) is dense by definition.

		      \( \varnothing = \operatorname{Int}(\overline{B}) = \mathscr{C}(\overline{\mathscr{C}(\overline{B})}) \) so \( \mathscr{C}(\overline{B}) \) is dense. Moreover, \( \mathscr{C}(B) \supset \mathscr{C}(\overline{B}) \) so \( \mathscr{C} \) is also dense.

		      \( \mathscr{C}(A \cup B) = \mathscr{C}(A) \cap \mathscr{C}(B) \supset \mathscr{C}(A) \cap \mathscr{C}(\overline{B}) \). The set \( \mathscr{C}(\overline{B}) \) is open so according to Problem~\ref{problem:III.4.13}, \( \overline{\mathscr{C}(A) \cap \mathscr{C}(\overline{B})} = \overline{\mathscr{C}\overline{B}} \). Hence \(\mathscr{C}(A) \cap \mathscr{C}(\overline{B})\) is dense, which implies that \(\mathscr{C}(A) \cap \mathscr{C}(B)\) is dense. Therefore \(A \cup B\) is a residual set.
		\item Let \(A\) be a subset of a topological space. Because \( \operatorname{Fr}(A) = \overline{A} \cap \overline{\mathscr{C}A} \), the boundary of \(A\) is a closed set.

		      If \(A\) is closed then \( \operatorname{Fr}(A) \subset \overline{A} \). Every neighborhood of every point in \( \operatorname{Fr}(A) \) intersects \(A\) and \(\mathscr{C}A\). Moreover, \( \mathscr{C}A \) and \( A = \overline{A} \) are disjoint so \( \mathscr{C}A \) and \( \operatorname{Fr}(A) \subset \overline{A} \) are disjoint. Hence \( \operatorname{Fr}(A) \) has no interior point if \(A\) is closed.

		      If \(A\) is open then \(\mathscr{C}A\) is closed and \(\operatorname{Fr}(A) = \operatorname{Fr}(\mathscr{C}A)\) so \( \operatorname{Fr}(A) \) has no interior point.

		      Thus \( \operatorname{Fr}(A) \) has no interior point whenever \(A\) is closed (or open). Moreover, \( \operatorname{Fr}(A) \) is closed, so \( \operatorname{Int}(\overline{\operatorname{Fr}(A)}) = \operatorname{Int}(\operatorname{Fr}(A)) = \varnothing \), which means \( \operatorname{Fr}(A) \) is nowhere dense.
		\item We will show that the complement of these sets are dense.
		      \begingroup
		      \allowdisplaybreaks%
		      \begin{align*}
			      \overline{\mathscr{C}(A \cap \overline{\mathscr{C}(A)})} & = \overline{\mathscr{C}(A) \cup \mathscr{C}(\overline{\mathscr{C}(A)})}                            & \text{(De Morgan's law)}                   \\
			                                                               & = \overline{\mathscr{C}A} \cup \overline{\mathscr{C}(\overline{\mathscr{C}A})}                     & \text{(closure and union are commutative)} \\
			                                                               & = \overline{\mathscr{C}A} \cup \overline{\operatorname{Int}(A)}                                                                                 \\
			                                                               & = \operatorname{Int(\mathscr{C}A)} \cup \operatorname{Fr}(A) \cup \overline{\operatorname{Int}(A)}                                              \\
			                                                               & \supset \operatorname{Int}(\mathscr{C}A) \cup \operatorname{Fr}(A) \cup \operatorname{Int}(A) = X
		      \end{align*}
		      \endgroup

		      so the complement of \( A \cap \overline{\mathscr{C}A} \) is dense, which means \( A \cap \overline{\mathscr{C}A} \) is residual.

		      By interchanging \( A \) and \( \mathscr{C}A \), we deduce that \( \overline{A} \cap \mathscr{C}(A) \) is residual.
		\item Let \(A\) be any set.
		      \begingroup
		      \allowdisplaybreaks%
		      \begin{align*}
			      \operatorname{Fr}(A) & = \overline{A} \cap \overline{\mathscr{C}(A)}                                                                                                       \\
			                           & = (\overline{A} \cap \overline{\mathscr{C}(A)}) \cap (A \cup \mathscr{C}(A))                                                                        \\
			                           & = (\overline{A} \cap \overline{\mathscr{C}(A)} \cap A) \cup (\overline{A} \cap \overline{\mathscr{C}(A)} \cap \mathscr{C}(A))                       \\
			                           & = \underbrace{(A \cap \overline{\mathscr{C}(A)})}_{\text{residual set}} \cup \underbrace{(\overline{A} \cap \mathscr{C}(A))}_{\text{residual set}}.
		      \end{align*}
		      \endgroup

		      Hence \( \operatorname{Fr}(A) \) is a union of two residual sets.
	\end{enumerate}
\end{proof}

\begin{problem}{III.4.22}
An open set \(U\) is called regular if \(U = \operatorname{Int}(\overline{U})\); a closed set \(A\) is called regular if \(A = \overline{\operatorname{Int}(A)}\). Prove:
\begin{enumerate}[leftmargin=*,label={(\alph*)},itemsep=0pt]
	\item If \(A\) is closed, then \(\operatorname{Int}(A)\) is a regular open set.
	\item If \(U\) is open, then \(\overline{U}\) is a regular closed set.
	\item The complement of a regular open (closed) set is a regular closed (open) set.
	\item If \(U, V\) are regular open sets, then \(U \subset V\) if and only if \( \overline{U} \subset \overline{V} \).
	\item If \(A, B\) are regular closed set, then \( A \subset B \) if and only if \( \operatorname{Int}(A) \subset \operatorname{Int}(B) \).
	\item If \(A, B\) are regular closed sets, so also is \(A \cup B\).
	\item If \(U, V\) are regular open sets, so also is \(U \cap V\).
\end{enumerate}
\end{problem}

\begin{proof}
	\begin{enumerate}[leftmargin=*,label={(\alph*)},itemsep=0pt]
		\item Let \( B = \operatorname{Int}(A) \) then \(B\) is open.

		      \( \operatorname{Int}(\overline{B}) \) is the largest open set contained in \( \overline{B} \) so \( B \subset \operatorname{Int}(\overline{B}) \).

		      %   \( B = \operatorname{Int}(A) \subset A \) so \( \overline{B} \subset A \) (as \(A\) is closed), which implies \( \mathscr{C}(\overline{B}) \supset \mathscr{C}(A) \), therefore \( \overline{\mathscr{C}(\overline{B})} \supset \overline{\mathscr{C}(A)} \). Since \( \overline{\mathscr{C}(\overline{B})} = \mathscr{C}(\operatorname{Int}(\overline{B})) \) and \( \overline{\mathscr{C}A} = \mathscr{C}(\operatorname{Int}(A)) = \mathscr{C}(B) \), we deduce that \( \mathscr{C}(\operatorname{Int}(\overline{B})) \supset \mathscr{C}(B) \), which precisely means \( \operatorname{Int}(\overline{B}) \subset B \) (because larger sets have smaller complement and vice versa).

		      \( B = \operatorname{Int}(A) \subset A \) so \( \overline{B} \subset A \) (as \(A\) is closed), which implies \( \operatorname{Int}(\overline{B}) \subset \operatorname{Int}(A) = B \).

		      Thus \( B = \operatorname{Int}(\overline{B}) \), so \( B \) is a regular open set.
		\item Let \( V = \overline{U} \) then \( V \) is closed.

		      \( \overline{\operatorname{Int}(V)} \) is the smallest closed set containing \( \operatorname{Int}(V) \) and \( \operatorname{Int}(V) \subset V \) so \( \overline{\operatorname{Int}(V)} \subset V \).

		      \( V = \overline{U} \supset U \) so \( \operatorname{Int}(V) \supset U \) (as \(U\) is open), which implies \( \overline{\operatorname{Int}(V)} \supset \overline{U} = V \).

		      Hence \( V = \overline{\operatorname{Int}(V)} \), which means \( V \) is a regular closed set.
		\item For any set \(S\), one has \( \mathscr{C}(\operatorname{Int}(\overline{S})) = \overline{\mathscr{C}(\overline{S})} = \overline{\operatorname{Int}(\mathscr{C}(S))} \).
		      \[
			      S = \operatorname{Int}(\overline{S}) \iff \mathscr{C}(S) = \mathscr{C}(\operatorname{Int}(\overline{S})) \iff \mathscr{C}(S) = \overline{\operatorname{Int}(\mathscr{C}(S))}
		      \]

		      so \( S \) is a regular open set if and only if its complement is a regular closed set.

		      Therefore the complement of a regular open set is a regular closed set and vice versa.
		\item \( U \subset V \) implies \( \overline{U} \subset \overline{V} \).

		      Conversely, \( \overline{U} \subset \overline{V} \) implies \( U = \operatorname{Int}(\overline{U}) \subset \operatorname{Int}(\overline{V}) = V \).

		\item \( A \subset B \) implies \( \operatorname{Int}(A) \subset \operatorname{Int}(B) \).

		      Conversely, \( \operatorname{Int}(A) \subset \operatorname{Int}(B) \) implies \( A = \overline{\operatorname{Int}(A)} \subset \overline{\operatorname{Int}(B)} = B \).
		\item If \( A, B \) are regular closed sets, then \( A \cup B \) is closed.

		      \( A \cup B = \overline{\operatorname{Int}(A)} \cup \overline{\operatorname{Int}(B)} = \overline{\operatorname{Int}(A) \cup \operatorname{Int}(B)} \subset \overline{\operatorname{Int}(A \cup B)} \) since \( \operatorname{Int}(A) \cup \operatorname{Int}(B) \subset \operatorname{Int}(A \cup B) \).

		      \( \operatorname{Int}(A \cup B) \subset A \cup B \) so \( \overline{\operatorname{Int}(A \cup B)} \subset \overline{A \cup B} = A \cup B \).

		      Hence \( A \cup B = \overline{\operatorname{Int}(A \cup B)} \), which means \( A \cup B \) is a regular closed set.
		\item If \( U, V \) are regular open sets, then \( U \cap V \) is open.

		      \( U \cap V = \operatorname{Int}(\overline{U}) \cap \operatorname{Int}(\overline{V}) = \operatorname{Int}(\overline{U} \cap \overline{V}) \supset \operatorname{Int}(\overline{U \cap V}) \) since \( \overline{U} \cap \overline{V} \supset \overline{U \cap V} \).

		      \( \overline{U \cap V} \supset U \cap V \) so \( \operatorname{Int}(\overline{U \cap V}) \supset \operatorname{Int}(U \cap V) = U \cap V \).

		      Hence \( U \cap V = \operatorname{Int}(\overline{U \cap V}) \), which means \( U \cap V \) is a regular open set.
	\end{enumerate}
\end{proof}

\section{Topologizing with preassigned elementary operations}

\begin{proposition}{5.2}
	Let \(X\) be a set, and \( \gamma: \mathscr{P}(X) \to \mathscr{P}(X) \) a map such that:
	\begin{enumerate}[label={(\arabic*)}]
		\item \( \gamma(\varnothing) = \varnothing \).
		\item \( \gamma \circ \gamma (A) \subset A \cup \gamma(A) \).
		\item \( \gamma(A \cup B) = \gamma(A) \cup \gamma(B) \).
		\item For each \( x \in X, x \notin \gamma(\left\{x\right\}) \).
	\end{enumerate}

	Then \( \mathscr{T} = \left\{ \mathscr{C}(A \cup \gamma(A)) \mid A \in \mathscr{P}(X) \right\} \) is a topology, and \( A^{\prime} = \gamma(A) \).
\end{proposition}

\begin{proof}
	Firstly, we show that
	\[
		A \subset B \implies \gamma(A) \subset \gamma(B).
	\]

	If \( A \subset B \) then \( B = A \cup B \), from which we deduce that \( \gamma(A) \subset \gamma(A) \cup \gamma(B) = \gamma(A \cup B) = \gamma(B) \).

	Consequently,
	\[ A \subset B \implies A \cup \gamma(A) \subset B \cup \gamma(B). \]

	\begin{enumerate}[label={(\roman*)}]
		\item \( X = \mathscr{C}(\varnothing \cup \gamma(\varnothing)) \) so \( X \in \mathscr{T} \).

		      \( \varnothing = \mathscr{C}(X) = \mathscr{C}(X \cup \gamma(X)) \) so \( \varnothing \in \mathscr{T} \).
		\item According to De Morgan's law,
		      \[  \mathscr{C}(A \cup \gamma(A)) \cap \mathscr{C}(B \cup \gamma(B)) = \mathscr{C}((A \cup \gamma(A)) \cup (B \cup \gamma(B))) = \mathscr{C}((A \cup B) \cup \gamma(A \cup B)) \]

		      Hence \(\mathscr{C}(A \cup \gamma(A)) \cap \mathscr{C}(B \cup \gamma(B)) \in \mathscr{T}\).
		\item Let \( S = \bigcup_{\alpha} \mathscr{C}(A_{\alpha} \cup \gamma(A_{\alpha})) \) then \( \mathscr{C}S = \mathscr{C}\bigcup_{\alpha} \mathscr{C}(A_{\alpha} \cup \gamma(A_{\alpha})) = \bigcap_{\alpha}(A_{\alpha} \cup \gamma(A_{\alpha})) \).

		      Hence \( \gamma(\mathscr{C}S) \subset \gamma(A_{\alpha} \cup \gamma(A_{\alpha})) \) for every \( \alpha \) as \( \mathscr{C}S \subset A_{\alpha} \cup \gamma(A_{\alpha}) \), which implies that
		      \[
			      \gamma(\mathscr{C}S) \subset \gamma(A_{\alpha} \cup \gamma(A_{\alpha})) = \gamma(A_{\alpha}) \cup \gamma\circ\gamma(A_{\alpha}) \subset \gamma(A_{\alpha}) \cup (A_{\alpha} \cup \gamma(A_{\alpha})) = A_{\alpha} \cup \gamma(A_{\alpha})
		      \]

		      for every \(\alpha\), so
		      \[
			      \gamma(\mathscr{C}S) \subset \bigcap_{\alpha} (A_{\alpha} \cup \gamma(A_{\alpha})) = \mathscr{C}S.
		      \]

		      Therefore \( \mathscr{C}S = \mathscr{C}(S) \cup \gamma(\mathscr{C}S) \), which means \( S = \mathscr{C}(\mathscr{C}(S) \cup \gamma(\mathscr{C}S)) \), so \( S \in \mathscr{T} \).
	\end{enumerate}

	Hence \( \mathscr{T} \) is a topology on \(X\).

	According to Proposition 5.1, \( A \cup \gamma(A) = \overline{A} \).

	If \( x \in A^{\prime} \) then \( x \in \overline{A - \left\{x\right\}} = (A - \left\{x\right\}) \cup \gamma(A - \left\{x\right\}) \), which means \( x \in \gamma(A - \left\{x\right\}) \subset \gamma(A) \). Hence \( A^{\prime} \subset \gamma(A) \).

	Let \( x \in \gamma(A) \). Assume that there exists a neighborhood \(U\) of \(x\) such that \( U \cap (A - \left\{x\right\}) = \varnothing \). Therefore \( A - \left\{x\right\} \subset \mathscr{C}U \).
	\[
		\gamma(A) = \gamma((A - \left\{x\right\}) \cup \left\{x\right\}) = \gamma(A - \left\{x\right\}) \cup \gamma(\left\{x\right\})
	\]

	Since \( x \notin \gamma(\left\{x\right\}) \), it follows that \( x \in \gamma(A - \left\{ x \right\}) \subset \gamma(\mathscr{C}U) \). Moreover, \( \mathscr{C}U \) is closed so \( \gamma(\mathscr{C}U) \subset \overline{\mathscr{C}U} = \mathscr{C}U \), which means \( x \in \mathscr{C}U \), this is a contradiction as \( x \in U \).

	Hence every neighborhood of \(x\) intersects \( A - \left\{x\right\} \) non-trivially, so \( \gamma(A) \subset A^{\prime} \).

	Thus \( A^{\prime} = \gamma(A) \).
\end{proof}

\begin{proposition}{5.3}
	Let \(X\) be a set, and \( \eta: \mathscr{P}(X) \to \mathscr{P}(X) \) a map such that:
	\begin{enumerate}[label={(\arabic*)}]
		\item \( \eta(X) = X \).
		\item \( \eta(A) \subset A \).
		\item \( \eta \circ \eta(A) = \eta(A) \).
		\item \( \eta(A \cap B) = \eta(A) \cap \eta(B) \).
	\end{enumerate}

	Then \( \mathscr{T} = \left\{ \eta(A) \mid A \in \mathscr{P}(X) \right\} \) is a topology, and \( \operatorname{Int}(A) = \eta(A) \).
\end{proposition}

\begin{proof}
	If \( A \subset B \) then
	\[
		\eta(A) = \eta(A \cap B) = \eta(A) \cap \eta(B) \subset \eta(B).
	\]

	Therefore \( A \subset B \implies \eta(A) \subset \eta(B) \).
	\begin{enumerate}[label={(\roman*)}]
		\item \( X = \eta(X) \) so \( X \in \mathscr{T} \).

		      \( \eta(\varnothing) \subset \varnothing \) so \( \varnothing = \eta(\varnothing) \in \mathscr{T} \).
		\item \( \eta(A) \cap \eta(B) = \eta(A \cap B) \in \mathscr{T} \).
		\item Let \( S = \bigcup_{\alpha} \eta(A_{\alpha}) \) then
		      \[
			      \eta(S) = \eta\left( \bigcup_{\alpha} \eta(A_{\alpha}) \right) \supset \eta(\eta(A_{\alpha}))
		      \]

		      for every \(\alpha\), from which we obtain that
		      \[
			      \eta(S) \supset \bigcup_{\alpha}\eta(\eta(A_{\alpha})) = \bigcup_{\alpha} \eta(A_{\alpha}) = S.
		      \]

		      Since \( \eta(S) \subset S \), we conclude that \( S = \eta(S) \), which means \( S \in \mathscr{T} \).
	\end{enumerate}

	Hence \( \mathscr{T} \) is a topology on \(X\).

	\( \operatorname{Int}(A) \) is the largest open set contained in \( A \) so \( \eta(A) \subset \operatorname{Int}(A) \).

	\( \operatorname{Int}(A) \) is open in \(X\) so \( \operatorname{Int}(A) = \eta(B) \) for some \( B \subset X \). Since \( A \supset \operatorname{Int}(A) \)
	\[
		\eta(A) \supset \eta(\operatorname{Int}(A)) = \eta\circ\eta(B) = \eta(B) = \operatorname{Int}(A).
	\]

	Hence \( \operatorname{Int}(A) = \eta(A) \).
\end{proof}

\begin{proposition}{5.4}
	Let \(X\) be a set, and \( \beta: \mathscr{P}(X) \to \mathscr{P}(X) \) a map such that:
	\begin{enumerate}[label={(\arabic*)}]
		\item \( \beta(\varnothing) = \varnothing \).
		\item \( \beta(A) = \beta(\mathscr{C}A) \).
		\item \( \beta \circ \beta(A) \subset \beta(A) \).
		\item \( A \cap B \cap \beta(A \cap B) = A \cap B \cap \left[\beta(A) \cup \beta(B)\right] \).
	\end{enumerate}

	Then \( \mathscr{T} = \left\{ \mathscr{C}(A \cup \beta(A)) \mid A \in \mathscr{P}(X) \right\} \) is a topology, and \( \operatorname{Fr}(A) = \beta(A) \).
\end{proposition}

\begin{proof}
	We show that \( U \in \mathscr{T} \) if and only if \( U \cap \beta(U) = \varnothing \).

	If \( U \in \mathscr{T} \) then \( U = \mathscr{C}(A \cup \beta(A)) \) for some \( A \subset X \).
	\begingroup
	\allowdisplaybreaks%
	\begin{align*}
		U \cap \beta(U) & = \mathscr{C}(A) \cap \mathscr{C}(\beta(A)) \cap \beta(\mathscr{C}(A) \cap \mathscr{C}(\beta(A)))                       \\
		                & = \mathscr{C}(A) \cap \mathscr{C}(\beta(A)) \cap \left[ \beta(\mathscr{C}(A)) \cup \beta(\mathscr{C}(\beta(A))) \right] \\
		                & = \mathscr{C}(A) \cap \mathscr{C}(\beta(A)) \cap \left[ \beta(A) \cup \beta(\beta(A)) \right]                           \\
		                & = \mathscr{C}(A) \cap \mathscr{C}(\beta(A)) \cap \beta(A)                                                               \\
		                & = \mathscr{C}(A) \cap \varnothing                                                                                       \\
		                & = \varnothing.
	\end{align*}
	\endgroup

	Conversely, \( U \cap \beta(U) = \varnothing \) implies \( \beta(U) \subset \mathscr{C}U \), so
	\[
		\mathscr{C}(\mathscr{C}(U) \cup \beta(\mathscr{C}U)) = \mathscr{C}(\mathscr{C}(U) \cup \beta(U)) = \mathscr{C}(\mathscr{C}(U)) = U
	\]

	which means \( U \in \mathscr{T} \).

	Therefore \( \mathscr{T} = \left\{ U \subset X \mid U \cap \beta(U) = \varnothing \right\} \).

	Also we use the following lemma: \textbf{If \( A \cap \beta(A) = \varnothing \) then \( A \cap \beta(B) = \varnothing \) whenever \( A \subset B \). }

	If \( A \subset B \) then \( A = A\cap B \) and
	\begingroup
	\allowdisplaybreaks%
	\begin{align*}
		\varnothing = A \cap \beta(A) & = A \cap B \cap \beta(A \cap B)                       \\
		                              & = A \cap B \cap \left[ \beta(A) \cup \beta(B) \right] \\
		                              & = A \cap B \cap \beta(A) \cup A \cap B \cap \beta(B)  \\
		                              & = A \cap B \cap \beta(B)                              \\
		                              & = A \cap \beta(B).
	\end{align*}
	\endgroup

	\begin{enumerate}[label={(\roman*)}]
		\item \( X = \mathscr{C}(\varnothing \cup \beta(\varnothing)) \) so \( X \in \mathscr{T} \).

		      \( \varnothing = \mathscr{C}(X) = \mathscr{C}(X \cup \beta(X)) \) so \( \varnothing \in \mathscr{T} \).
		\item Let \( U, V \in \mathscr{T} \).
		      \begingroup
		      \allowdisplaybreaks%
		      \begin{align*}
			      U \cap V \cap \beta(U \cap V) & = U \cap V \cap \left[\beta(U) \cup \beta(V)\right]      \\
			                                    & = (U \cap V \cap \beta(U)) \cup (U \cap V \cap \beta(V)) \\
			                                    & = \varnothing \cup \varnothing                           \\
			                                    & = \varnothing.
		      \end{align*}
		      \endgroup
		\item Let \( \left\{ A_{\alpha} \right\} \) be a family of members of \( \mathscr{T} \) and \( A = \bigcup_{\alpha} A_{\alpha} \)
		      \begingroup
		      \allowdisplaybreaks%
		      \begin{align*}
			      \left(\bigcup_{\alpha} A_{\alpha}\right) \cap \beta\left(\bigcup_{\alpha} A_{\alpha}\right) & = \left(\bigcup_{\alpha} A_{\alpha}\right) \cap \beta(A) \\
			                                                                                                  & = \bigcup_{\alpha} (A_{\alpha} \cap \beta(A))            \\
			                                                                                                  & = \bigcup_{\alpha} \varnothing                           \\
			                                                                                                  & = \varnothing
		      \end{align*}
		      \endgroup

		      Hence \( A \in \mathscr{T} \).
	\end{enumerate}

	Thus \( \mathscr{T} \) is a topology on \(X\).

	\( \mathscr{C}(A \cup \beta(A)) \) is open so \( A \cup \beta(A) \) is closed, so \( \overline{A} \subset A \cup \beta(A) \) as \( \overline{A} \) is the smallest closed set containing \(A\).
	\begingroup
	\allowdisplaybreaks%
	\begin{align*}
		\operatorname{Fr}(A) & = \overline{A} \cap \overline{\mathscr{C}A}                            \\
		                     & \subset (A \cup \beta(A)) \cap (\mathscr{C}A \cup \beta(\mathscr{C}A)) \\
		                     & = (A \cup \beta(A)) \cap (\mathscr{C}A \cup \beta(A))                  \\
		                     & = (A \cap \mathscr{C}A) \cup \beta(A)                                  \\
		                     & = \beta(A)
	\end{align*}
	\endgroup

	so \( \operatorname{Fr}(A) \subset \beta(A) \).

	We prove \( \beta(A) \subset \operatorname{Fr}(A) \) using contradiction. Assume that there exists \( x \in \beta(A) \) and a neighborhood \( U \) of \( x \) such that \( U \cap A = \varnothing \) or \( U \cap \mathscr{C}A = \varnothing \).

	\( U \) is open so \( U \cap \beta(U) = \varnothing \).

	If \( U \cap A = \varnothing \) then \( U \subset \mathscr{C}A \) so \( U \cap \beta(A) = U \cap \beta(\mathscr{C}A) = \varnothing \).

	If \( U \cap \mathscr{C}A = \varnothing \) then \( U \subset A \) so \( U \cap \beta(\mathscr{C}A) = U \cap \beta(A) = \varnothing \).

	This is a contradiction since \( x \in \beta(A) \) and \( U \cap \beta(A) = \varnothing \).

	Hence for every \( x \in \beta(A) \), for every neighborhood \( U \) of \(x\), one has \( U \cap A \ne \varnothing \) and \( U \cap \mathscr{C}A \ne \varnothing \), this means \( x \in \operatorname{Fr}(A) \). Therefore \( \beta(A) \subset \operatorname{Fr}(A) \).

	So \( \operatorname{Fr}(A) = \beta(A) \).
\end{proof}

\begin{proposition}{5.5}
	Let \(X\) be a set, and \(\mathscr{A} \subset \mathscr{P}(X)\) a family with the properties:
	\begin{enumerate}[label={(\arabic*)}]
		\item \( \varnothing, X \) belong to \(\mathscr{A}\).
		\item The intersection of any family of members of \(\mathscr{A}\) is also a member of \(\mathscr{A}\).
		\item The finite union of members of \(\mathscr{A}\) belongs to \(\mathscr{A}\).
	\end{enumerate}

	The \( \mathscr{T} = \left\{ \mathscr{C}A \mid A \in \mathscr{A} \right\} \) is a topology, in which \( \mathscr{A} \) is the complete family of closed sets.
\end{proposition}

\begin{proof}
	\( \varnothing, X \in \mathscr{T} \).

	\( \mathscr{C}A \cap \mathscr{C}B = \mathscr{C}(A \cup B) \) so \( \mathscr{C}A \cap \mathscr{C}B \in \mathscr{T} \) whenever \( A, B \in \mathscr{A} \).

	\( \bigcup_{\alpha} \mathscr{C}A_{\alpha} = \mathscr{C}\bigcap_{\alpha}A_{\alpha} \in \mathscr{T} \) whenever \( A_{\alpha} \in \mathscr{A} \).

	Hence \( \mathscr{T} \) is a topology.

	By the definition of closed set, \( \mathscr{A} \) is the complete family of closed sets.
\end{proof}

\begin{problem}{III.5.1}
Let \(A \mapsto u(A)\) and \(A \mapsto v(A)\) be two closure operations. Assume that \( v\circ u(A) \) is \(u\)-closed. Prove that \( A \mapsto v\circ u(A) \) is a closure operation and that \(v\circ u(A)\) is in fact the intersection of all sets containing \(A\) that are closed in both \(u\) and \(v\). Finally, show that \(u\circ v(A) \subset v\circ u(A)\).
\end{problem}

\begin{proof}
	\begin{enumerate}[label={(\arabic*)}]
		\item \( v\circ u(\varnothing) = v(u(\varnothing)) = v(\varnothing) = \varnothing \)
		\item \( A \subset u(A) \subset v(u(A)) = v\circ u(A) \).
		\item \( v\circ u \circ v\circ u(A) = v\circ v\circ u(A) \) because \( v\circ u(A) \) is \(u\)-closed. Because \( v\circ u(A) \) is \(v\)-closed, it follows that \( v\circ v\circ u(A) = v\circ u(A) \). Hence \( v\circ u \circ v\circ u(A) = v\circ u(A) \).
		\item \( v\circ u(A \cup B) = v(u(A \cup B)) = v(u(A) \cup u(B)) = v(u(A)) \cup v(u(B)) = v\circ u(A) \cup v\circ u(B) \).
	\end{enumerate}

	Therefore \( v\circ u \) is a closure operation.

	Let \(B\) be a superset of \(A\) that is closed in both \(u\) and \(v\), then \( v\circ u(A) \subset v\circ u(B) = B \), which means \( v\circ u(A) \) is the smallest set that is closed in both \(u\) and \(v\) and contains \(A\). Hence \( v\circ u(A) \) is the intersection of all sets containing \(A\) that are closed in both \(u\) and \(v\).

	One has \( A \subset u(A) \) so \( v(A) \subset v\circ u(A) \). Therefore \( u \circ v(A) \subset u\circ v\circ u(A) \). Since \( v\circ u(A) \) is \(u\)-closed, we deduce that \( u\circ v\circ u(A) = v\circ u(A) \). Hence \( u\circ v(A) \subset v\circ u(A) \).
\end{proof}

\begin{problem}{III.5.2}
Let \( \varphi: X \to \mathscr{P}(Y) \) be a map of a set \(X\) into the set \( \mathscr{P}(Y) \) \textcolor{red}{such that \( \varphi(x) \ne \varnothing \) for every \( x \in X \)}. For \( A \subset X \), let \( \varphi(A) := \bigcup \left\{ \varphi(x) \mid x \in A \right\} \); and for \( B \subset Y \), let \( \varphi^{-1}(B) := \left\{ x \mid (\varphi(x) \subset B) \land (\varphi(x) \ne \varnothing) \right\} \). Show that \( u(A) = \varphi^{-1}\circ\varphi(A) \) satisfies (1), (2), (3) and (i) of Proposition III.5.1.
\end{problem}

\begin{proof}
	\begin{enumerate}[label={(\arabic*)}]
		\item \( \varphi(\varnothing) = \varnothing \) and \( \varphi^{-1}(\varnothing) = \varnothing \) by definition. Hence \( \varphi^{-1} \circ \varphi(\varnothing) = \varnothing \).
		\item Let \( x \in A \) then \( \varphi(x) \subset \varphi(A) \) and \( \varphi(x) \ne \varnothing \) so \( x \in \varphi^{-1}\circ \varphi(A) = u(A) \).

		      Hence \( A \subset \varphi^{-1} \circ \varphi(A) \).
		\item From (2), it follows that \( u(A) \subset u \circ u(A) \).

		      Let \( x \in u \circ u(A) = \varphi^{-1} \circ \varphi \circ \varphi^{-1} \circ \varphi(A) \) then \( \varphi(x) \subset \varphi \circ \varphi^{-1} \circ \varphi(A) \) and \( \varphi(x) \ne \varnothing \).

		      \( \varphi(x) \subset \varphi \circ \varphi^{-1} \circ \varphi(A) \) means \( x \in \varphi^{-1} \circ \varphi(A) \) by definition.

		      Therefore \( u \circ u(A) \subset u(A) \). Hence \( u \circ u(A) = u(A) \).
	\end{enumerate}

	Suppose that \( A_{1} \subset A_{2} \subset X \) then \( \varphi(A_{1}) \subset \varphi(A_{2}) \) by definition.

	Therefore \( \varphi^{-1} \circ \varphi(A_{1}) \subset \varphi^{-1} \circ \varphi(A_{2}) \), which means \( u(A_{1}) \subset u(A_{2}) \).

	Hence \( (A_{1} \subset A_{2}) \implies (u(A_{1}) \subset u(A_{2})) \).
\end{proof}

\begin{problem}{III.5.3}
Let \(X\) be a space, and let \( \tau \) be an operation associating with each pair of subsets \(A\) and \(B\) a set \(\tau(A, B) \subset X\) subject to the conditions:
\begin{enumerate}[label={\alph*.}]
	\item \( \tau(A, B \cup C) \cup \tau(B, C\cup A) = \tau(A\cup B, C) \cup \tau(A, B) \).
	\item \( \tau(\varnothing, X) = \varnothing \).
	\item \( \tau(\overline{A}, \overline{\mathscr{C}A}) \subset \overline{A} \).
	\item \( \tau(A, B) \subset A \cup B \).
\end{enumerate}

Show that, necessarily, \( \tau(A, B) = (A \cap \overline{B}) \cup (\overline{A} \cap B) \).
\end{problem}

\begin{proof}
	% TODO
\end{proof}

\section{\( G_{\sigma}, F_{\sigma} \) and Borel sets}

\begin{problem}{III.6.1}\label{problem:III.6.1}
A set is bivalent if it is both an \(F_{\sigma}\) and a \(G_{\sigma}\). Show that the complements, finite unions, and finite intersections of bivalent sets are bivalent.
\end{problem}

\begin{proof}
	Let \(B\) be a bivalent set then \(B\) is both an \(F_{\sigma}\) and a \(G_{\sigma}\) set. Since the complement of an \(F_{\sigma}\) is a \(G_{\sigma}\) and the complement of a \(G_{\sigma}\) is an \(F_{\sigma}\), it follows that the complement of \(B\) is bivalent.

	Let \( B_{1}, \ldots, B_{n} \) be bivalent sets then \( \bigcup^{n}_{i=1}B_{i} \) is a bivalent, since the union of at most countable \(F_{\sigma}\)-sets is an \(F_{\sigma}\)-set and the union of finitely many \(G_{\sigma}\)-sets is a \(G_{\sigma}\)-set.
\end{proof}

For the remaining problems (in this section), we assume that all closed sets are \(G_{\delta}\)-sets.

\begin{problem}{III.6.2}\label{problem:III.6.2}
Prove: Every \(F_{\sigma}\) is the disjoint union of bivalent sets.
\end{problem}

\begin{proof}
	Let \(F\) be an \(F_{\sigma}\)-set then \( F = \bigcup^{\infty}_{1} F_{i} \) in which each \(F_{i}\) is closed.

	Define \(A_{1} = F_{1}\) and \(A_{n+1} = F_{n+1} - \bigcup^{n}_{1} F_{i}\) then \( A_{1}, A_{2}, \ldots \) are pairwise disjoint and
	\[
		\bigcup^{\infty}_{1} A_{i} = \bigcup^{\infty}_{1} F_{i}.
	\]

	Each \(F_{i}\) is bivalent (due to \(F_{i}\) being closed and the assumption after Problem~\ref{problem:III.6.1}) so each \(A_{i}\) is bivalent.
\end{proof}

\begin{problem}{III.6.3}\label{problem:III.6.3}
Prove: For every sequence \( \left\{ F_{i} \right\} \) of \(F_{\sigma}\)-sets, there exists a pairwise disjoint sequence \( \left\{ H_{i} \right\} \) of \(F_{\sigma}\)-sets with \(H_{i} \subset F_{i}\) for every \(i\) and \( \bigcup H_{i} = \bigcup F_{i} \).
\end{problem}

\begin{proof}
	According to Problem~\ref{problem:III.6.2}, \( F_{i} \) is a disjoint union of bivalent sets, \( F_{i} = \bigcup^{\infty}_{j=1} F_{i,j} \).

	Define \( H_{i} = \bigcup^{\infty}_{j=1} \left(F_{i,j} - \bigcup_{(k,\ell) < (i,j)} F_{k,\ell}\right) \) in which \( (k,\ell) < (i,j) \) if \( k < i \) or \( k = i \) and \( \ell < j \) (the dictionary order).

	Then \( H_{i} \subset \bigcup^{\infty}_{j=1} F_{i,j} = F_{i} \), \( \left\{ H_{i} \right\} \) are pairwise disjoint, and \( \bigcup H_{i} = \bigcup F_{i} \).
\end{proof}

\begin{problem}{III.6.4}\label{problem:III.6.4}
Prove: If \( \left\{ G_{i} \right\} \) is any sequence of \(G_{\delta}\)-sets with \( \bigcap^{\infty}_{1} G_{i} = \varnothing \), there exists a sequence of bivalent sets \( \left\{ B_{i} \right\} \) with \(G_{i} \subset B_{i}\) and \( \bigcap^{\infty}_{1} B_{i} = \varnothing \).
\end{problem}

\begin{proof}
	Let \( F_{i} = \mathscr{C}G_{i} \) then \( F_{i} \) is an \( F_{\sigma} \)-set.

	According to Problem~\ref{problem:III.6.3}, there exists a pairwise disjoint sequence \( \left\{ H_{i} \right\} \) of \(F_{\sigma}\)-sets with \(H_{i} \subset F_{i}\) for every \(i\) and \( \bigcup H_{i} = \bigcup F_{i} \). Let \( B_{i} = \mathscr{C}H_{i} \) then \( G_{i} \subset B_{i} \) and \( \bigcap^{\infty}_{1} B_{i} = \bigcap^{\infty}_{i=1} G_{i} = \varnothing \).
\end{proof}

\begin{problem}{III.6.5}
Prove: (a) If \(G, H\) are disjoint \(G_{\delta}\)-sets, there exists a bivalent set \(B\) with \(H \subset B\) and \( B \cap G = \varnothing \).

(b) If \(G\) is a \(G_{\sigma}\) and \(F\) is an \(F_{\sigma}\) such that \(G \subset F\), there exists a bivalent \(B\) with \( G \subset B \subset F \).
\end{problem}

\begin{proof}
	\begin{enumerate}[label={(\alph*)}]
		\item Because
		      \[
			      H \cap G \cap \cdots \cap G \cap \cdots = \varnothing
		      \]

		      there exists a sequence of bivalent sets \( \left\{ B_{i} \right\} \) with \(G \subset B_{i}\) for \(i>1\) and \(H \subset B_{1}\) and \( \bigcap^{\infty}_{1} B_{i} = \varnothing \), according to Problem~\ref{problem:III.6.4}.

		      \(B_{1}\) is a bivalent set, \( H \subset B_{1} \) and \( B_{1} \cap G = B_{1} \cap G \cap \cdots \cap G \cap \cdots \subset \bigcap^{\infty}_{i=1} B_{i} = \varnothing \).

		      Let \( B = B_{1} \) then \(B\) is a bivalent, \( H \subset B \) and \( B \cap G = \varnothing \).
		\item \( G \subset F \) so \( G \) and \( \mathscr{C}F \) are disjoint. Moreover, they are \( G_{\delta} \)-sets.

		      According to part (a), there is a bivalent set \(B\) such that \( G \subset B \) and \( B \cap \mathscr{C}F = \varnothing \).

		      \( B \cap \mathscr{C}F = \varnothing \) is equivalent to \( B \subset F \). Hence \( G \subset B \subset F \).
	\end{enumerate}
\end{proof}

\section{Relativization}

\begin{problem}{III.7.1}
In \( {\tilde{E}}^{1} \), show that for any set \(A\), \(\sup A \in \overline{A}\).
\end{problem}

\begin{proof}
	If \( A \) is empty then \( \sup A = -\infty \in \tilde{E}^{1} \).

	Otherwise, \( A \) is nonempty then either it has an upper bound or not. If \(A\) is nonempty and has an upper bound then \(A\) has a least upper bound (the least upper bound property of \(\mathbb{R}\)), so \( \sup A \) is well-defined and \( \sup A \in E^{1} \subset \tilde{E}^{1} \). If \( A \) is nonempty and doesn't have an upper bound then \( \sup A = +\infty \in \tilde{E}^{1} \).
\end{proof}

\begin{problem}{III.7.2}
Describe the relative topology of \( \left\{ z \mid \left\vert{z}\right\vert = 1 \right\} \) as a subspace of \(E^{2}\).
\end{problem}

\begin{proof}
	A basis for the relative topology of the given set is \( \left\{ \exp(\iota t) \mid t \in \openinterval{a,b} \right\} \)
\end{proof}

\begin{problem}{III.7.3}
Show that the rationals, as a subspace of \(E^{1}\), do not have the discrete topology.
\end{problem}

\begin{proof}
	Let \(U\) be a nonempty open subset of the rationals (the set of rationals is considered a subspace of \(E^{1}\)). From the definition of subspace topologies, there exists an open subset \(V \subset E^{1}\) such that \(U = V \cap \mathbb{Q}\). Since \(V\) is an open subset of \(E^{1}\), it contains some open interval \(\openinterval{a, b}\). Every open interval contains infinitely many rationals, hence \(U\) contains infinitely many rationals.

	Therefore every nonempty open subset of the rationals is not a singleton, which means the rationals, as a subspace of \(E^{1}\), don't have the discrete topology.
\end{proof}

\begin{problem}{III.7.4}
Let \(K \subset E^{1}\) be the set of irrationals in \( \openinterval{0,1} \). Expressing each \(x, y\in K\) as decimals, define \( d(x, y) = 1/n \) if \(x\) and \(y\) have their first \( (n - 1) \) digits identical, and their \(n\) th digits different. Let \( B(x; 1/n) = \left\{ y \in K \mid d(x, y) < 1/n \right\} \). Show that the \( \left\{ B(x; 1/n) \mid x \in K, n \in \mathbb{Z}^{+} \right\} \) is a basis for a topology \( \mathscr{T} \) on \(K\), and prove that \( \mathscr{T} \) coinicides with the induced topology of \(K\) as a subspace of \(E^{1}\).
\end{problem}

\begin{proof}
	By definition, for every \( x, y \in K \), \( d(x, y) = 0 \) if and only if \( x = y \). Moreover, \( d(x, y) = d(y, x) \) for every \( x, y \in K \).

	Let \( x, y, z \in K \), we aim to prove that \( d(x, y) + d(y, z) \ge d(x, z) \). If \( x = y \) or \( y = z \) or \( x = z \) then the inequality is true. Assume that \( x, y, z \) are pairwise distinct.

	Let \( d(x, y) = 1/n \) and \( d(y, z) = 1/m \). Without loss of generality, assume that \( m \ge n \), then \( d(x, z) \le 1/n \), which implies \( d(x, z) \le d(x, y) + d(y, z) \). Therefore \( (K, d) \) is a metric space, so \( \left\{ B(x; 1/n) \mid x \in K, n \in \mathbb{Z}^{+} \right\} \) is a basis for the metric topology on \( K \).

	Consider an open ball \( B(x; 1/n) \) and the first \( n - 1 \) digits in the decimal expansion of \( x \) is \( a_{1}, a_{2}, \ldots, a_{n-1} \). Let \( a = 0.a_{1}a_{2}\cdots a_{n-1} \) and \( b = a + \sum^{\infty}_{k=n}\dfrac{9}{10^{k}} \) then \( B(x; 1/n) = K \cap \openinterval{a, b} \), which means \( B(x; 1/n) \) is an open set in the induced topology of \( K \). Therefore the induced topology of \(K\) is finer than the metric topology on \( K \).

	Conversely, consider \( \openinterval{a, b} \cap K \) for some open interval \( \openinterval{a, b} \). Let \( x \in \openinterval{a, b} \cap K \) then there exists \( \varepsilon > 0 \) such that \( \openinterval{x - \varepsilon, x + \varepsilon} \subset \openinterval{a, b} \). If \( d(x, y) = \dfrac{1}{n} \) then \( \left\vert{x - y}\right\vert \le 10^{1-n} \). There exists a positive integer \(N\) such that \( 10^{1-N} < \varepsilon \) so \( d(x, y) < 1/n \) whenever \( \left\vert{x - y}\right\vert \le 10^{1-n} \), which means
	\[
		x \in \openinterval{x - 10^{1-N}, x + 10^{1-N}} \cap K \subset \openinterval{x - \varepsilon, x + \varepsilon} \cap K \subset \openinterval{a, b} \cap K
	\]

	so the induced topology of \( K \) is coarser than the metric topology on \( K \).

	Hence \( \mathscr{T} \) coincides with the induced topology of \(K\) as a subspace of \( E^{1} \).
\end{proof}

\begin{problem}{III.7.5}
Let \(A \subset B\) be open in \(B\). Show: For any set \( S \), \( A \cap S \) is open in \( B \cap S \).
\end{problem}

\begin{proof}
	Let \( \mathscr{B} \) be a basis for the topology on \(B\) then \( \left\{ U \cap S \mid U \in \mathscr{B} \right\} \) is a basis for the subspace \( B \cap S \).

	\( A \) is open in \( B \) so there exists a collection \( {\left\{ U_{i} \right\}}_{i\in I} \) of elements of \( \mathscr{B} \) such that \( A = \bigcup_{i\in I} U_{i} \). Therefore
	\[
		A \cap S = \left(\bigcup_{i\in I} U_{i}\right) \cap S = \bigcup_{i\in I} (U_{i} \cap S)
	\]

	which means \( A \cap S \) is open in \( B \cap S \).
\end{proof}

\begin{problem}{III.7.6}
Let \( Y_{1}, Y_{2} \) be (not necessarily disjoint) subspaces of \(X\), and \( A \subset Y_{1} \cap Y_{2} \). Assume that \( A \) is open (closed) in \(Y_{1}\) and open (closed) in \(Y_{2}\). Prove: \( A \) is open (closed) in \( Y_{1} \cup Y_{2} \).
\end{problem}

\begin{proof}
	If \( A \) is open (closed) in \( Y_{1} \) and \( Y_{2} \) then there exist open (closed) sets \( U_{1}, U_{2} \subset X \) such that \( A = Y_{1} \cap U_{1} = Y_{2} \cap U_{2} \). Besides \( U_{1} \cap U_{2} \) is open (closed) in \(X\) and
	\begingroup
	\allowdisplaybreaks%
	\begin{align*}
		(Y_{1} \cup Y_{2}) \cap (U_{1} \cap U_{2}) & = (Y_{1} \cap U_{1} \cap U_{2}) \cup (Y_{2} \cap U_{1} \cap U_{2}) \\
		                                           & = (A \cap U_{2}) \cup (A \cap U_{1})                               \\
		                                           & = A \cap (U_{1} \cup U_{2})                                        \\
		                                           & = A.
	\end{align*}
	\endgroup

	Hence \( A \) is open (closed) in \( Y_{1} \cup Y_{2} \).
\end{proof}

\begin{problem}{III.7.7}
Let \(A \subset X\) be closed and \( U \subset A \) open in \(A\). Let \(V\) be any set open in \(X\) with \( U \subset V \). Prove: \( U \cup (V - A) \) is open in \(X\).
\end{problem}

\begin{proof}
	\( U \subset A \) is open in \(A\) so there is an open set \( W \subset X \) such that \( U = W \cap A \).
	\begingroup
	\allowdisplaybreaks%
	\begin{align*}
		U \cup (V - A) & = U \cup (V \cap (X - A))                         \\
		               & = (U \cup V) \cap (U \cup (X - A))                \\
		               & = V \cap (U \cup (X - A))                         \\
		               & = V \cap ((W \cap A) \cup (X - A))                \\
		               & = V \cap ((W \cup (X - A)) \cap (A \cup (X - A))) \\
		               & = V \cap ((W \cup (X - A)) \cap X)                \\
		               & = V \cap (W \cup (X - A)).
	\end{align*}
	\endgroup

	Since \( V, W, X - A \) are open in \( X \), it follows that \( U \cup (V - A) \) is open in \( X \).
\end{proof}

\begin{problem}{III.7.8}
\begin{enumerate}[label={(\alph*)}]
	\item Let \(D\) be dense in \(X\). Give an example to show \( D \cap A \) need not be dense in \(A\).
	\item Prove: If \(A\) is dense in \(B \subset X\), then \(A\) is dense in \( \overline{B} \).
\end{enumerate}
\end{problem}

\begin{proof}
	\begin{enumerate}[label={(\alph*)}]
		\item Let \( X = \mathbb{R}, D = \mathbb{Q}, A = \mathbb{R} - \mathbb{Q} \).

		      \( D \) is dense in \(X\) but \( D \cap A = \varnothing \) is not dense in \( A \).
		\item Let \( U \) be a nonempty open subset of \( \overline{B} \) then \( U \) intersects \( B \). \( U \cap B \) is a nonempty open subset of \( B \). Because \( A \) is dense in \( B \) then \( U \cap B \) intersects \( A \). Therefore \( (U \cap B) \cap A \ne \varnothing \). Moreover, \( (U \cap B) \cap A = U \cap (B \cap A) = U \cap A \) as \( A \subset B \). Hence \( U \) intersects \( A \), which means any nonempty open subset of \( \overline{B} \) intersects \( A \), which implies that \( A \) is dense in \( \overline{B} \).
	\end{enumerate}
\end{proof}

\begin{problem}{III.7.9}
Let \(X\) be the set of Problem~\ref{problem:III.3.6}, with the topology \( \mathscr{T}_{R} \). Let \( A \subset X \), and using the induced ordering on \(A\), obtain \( \mathscr{T}_{R}(A) \). Show that \( \mathscr{T}_{R}(A) \) coincides with the induced topology on \(A\) as a subspace of \(X\).
\end{problem}

\begin{proof}
	Let \( \mathscr{T}_{A} \) be the induced topology on \(A\) as a subspace of \(X\).

	\textbf{Prove that \( \mathscr{T}_{A} \subset \mathscr{T}_{R}(A) \).}

	Let \( U \in \mathscr{T}_{A} \) then there exists an open subset \( V \subset X \) such that \( U = A \cap V \).

	From Problem~\ref{problem:III.3.6}, \( G \in \mathscr{T}_{R} \) if and only if \( x \in G \implies U_{R}(x) \in \mathscr{T}_{R} \).

	Let \( x \in U = A \cap V \). Note that \( V \in \mathscr{T}_{R} \). Apply the only if part, we deduce that
	\[
		\left\{ y \mid x \prec_{A} y \right\} = \left\{ y \mid x \prec y \right\} \cap A \in \mathscr{T}_{R}(A)
	\]

	which means \( U \in \mathscr{T}_{R}(A) \), according to the if part. Hence \( \mathscr{T}_{A} \subset \mathscr{T}_{R}(A) \).

	\textbf{Prove that \( \mathscr{T}_{R}(A) \subset \mathscr{T}_{A} \).}

	Let \( U \in \mathscr{T}_{R}(A) \) and \( x \in U \). From Problem~\ref{problem:III.3.6}, it follows that \( \left\{ y \mid x \prec_{A} y \right\} \in \mathscr{T}_{R}(A) \). Moreover
	\[
		\left\{ y \mid x \prec_{A} y \right\} = \underbrace{\left\{ y \mid x \prec y \right\}}_{\in \mathscr{T}_{R}} \cap A
	\]

	so \( \left\{ y \mid x \prec_{A} y \right\} \in \mathscr{T}_{A} \). Therefore \( U \in \mathscr{T}_{A} \), which implies \( \mathscr{T}_{R}(A) \subset \mathscr{T}_{A} \).

	Thus the two topologies are identical.
\end{proof}

\begin{problem}{III.7.10}
For any linearly ordered set \(X\), let \( \mathscr{T}_{0}(X) \) be the topology with subbasis all sets of form \( \left\{ x \mid x > a \right\}, \left\{ x \mid x < b \right\} \). In \( E^{1} \), \( \mathscr{T}_{0}(E^{1}) \) is the Euclidean topology. Let \( A = \left\{0\right\} \cup \left\{ x \mid \left\vert{x}\right\vert > 1 \right\} \). Show that its topology \( \mathscr{T}_{0}(A) \) as a linearly ordered set does not coincide with its topology as a subspace of \(E^{1}\).
\end{problem}

\begin{proof}
	Denote by \( \mathscr{T} \) the induced topology of \( A \) as a subspace of \( E^{1} \) then \( \left\{0\right\} \in \mathscr{T} \).

	If \( U \subset A \) is open and contains \( 0 \) then \( U \) is of the form
	\[
		\left\{ x \mid x > a \right\} \cup \left\{0\right\} \cup \left\{ x \mid x < b \right\} = \left\{0\right\}
	\]

	in which \( a < -1 \) and \( b > 1 \). Hence \( \left\{0\right\} \) is not open in \( \mathscr{T}_{0}(A) \).

	Thus \( \mathscr{T}_{0}(A) \ne \mathscr{T} \).
\end{proof}

\begin{problem}{III.7.11}\label{problem:III.7.11}
Let \(X\) be any space, and let \(Y\) be a closed subspace. Let \(A\subset X\) by any set, and let \(H\) be a neighborhood of \( A \cap Y \) in \(Y\). Prove: \( A \cap \overline{Y - H} = \varnothing \).
\end{problem}

\begin{proof}
	\( H \) is open in \( Y \) so there is an open subset \( U \subset X \) such that \( H = Y \cap U \).
	\[
		A \cap \overline{Y - H} = A \cap \overline{Y - Y \cap U} = A \cap \overline{Y - U}.
	\]

	Since \( Y - U = Y \cap (X - U) \) is closed in \(X\) (as \(Y, X - U\) are closed in \(X\)), then \( \overline{Y - U} = Y - U \). Hence
	\[
		A \cap \overline{Y - U} = A \cap (Y - U) = A \cap Y \cap (X - U).
	\]

	Besides, \( A \cap Y \subset H \subset U \) so \( A \cap Y \cap (X - U) = \varnothing \). Thus \( A \cap \overline{Y - H} = \varnothing \).
\end{proof}

\begin{problem}{III.7.12}
Let \(X\) be any space, and assume that \(X = E_{1} \cup E_{2}\), where \(E_{1}\) and \(E_{2}\) are closed in \(X\). Let \( B \subset E_{1} \) be such that \( B \cap E_{2} \subset Q \) where \( Q \) is open in \(E_{2}\). Prove: \(B \subset \operatorname{Int}(E_{1} \cup Q)\).
\end{problem}

\begin{proof}
	According to Problem~\ref{problem:III.7.11}, \( B \cap \overline{E_{2} - Q} = \varnothing \), from which we deduce that \( B \subset X - \overline{E_{2} - Q} \). Moreover
	\[
		X - \overline{E_{2} - Q} = \operatorname{Int}(X - (E_{2} - Q)) = \operatorname{Int}((X - E_{2}) \cup Q) \subset \operatorname{Int}(E_{1} \cup Q).
	\]

	Hence \( B \subset \operatorname{Int}(E_{1} \cup Q) \).
\end{proof}

\section{Continuous maps}

\begin{problem}{III.8.1}
Let \( \mathscr{S} \) be Sierpinski space and let \(2\) be the discrete space \( \left\{ 0, 1 \right\} \). Let \( f: \mathscr{S} \to 2 \) be the identity map. Show that \( f \) is not continuous, but that \( f^{-1}: 2 \to \mathscr{S} \) is.
\end{problem}

\begin{proof}
	The preimage of \( \left\{ 1 \right\} \), which is open in \(2\), under \( f \) is not open in \( \mathscr{S} \) so \( f \) is not continuous.

	The preimages of \( \varnothing, \left\{ 0 \right\}, \left\{ 0, 1 \right\} \) (these are the open sets in \(\mathscr{S}\)) under \( f \) are \( \varnothing, \left\{ 0 \right\}, \left\{ 0, 1 \right\} \) and they are open in \( 2 \). Hence \( f^{-1} \) is continuous.
\end{proof}

\begin{problem}{III.8.2}
Let \(c\) be the characteristic function of \( \halfopenright{0, 1} \subset E^{1} \). Is \( c\vert_{\closedinterval{0, 1}} \) continuous at \( x = 0 \)? At \( x = 1 \)? On \( \halfopenright{0, 1} \)?
\end{problem}

\begin{proof}
	If \(U\) be a neighborhood at \( c(0) = 1 \) then the preimage of \(U\) under \( c\vert_{\closedinterval{0,1}} \) is \( \halfopenright{0,1} \) (if \( 0 \notin U \)) or \( \closedinterval{0, 1} \) (if \( 0 \in U \)) which are open in \( \closedinterval{0,1} \). Hence \( c\vert_{\closedinterval{0,1}} \) is continuous at \( x = 0 \).

	If \( U \) is a neighborhood at \( c(1) = 0 \) then the preimage of \(U\) under \( c\vert_{\closedinterval{0,1}} \) is
	\begin{itemize}[itemsep=0pt,topsep=0pt]
		\item \( \closedinterval{0,1} \) if \( 1 \in U \),
		\item \( \left\{1\right\} \) if \( 1 \notin U \).
	\end{itemize}

	So \( c\vert_{\closedinterval{0,1}} \) is not continuous at \( x = 1 \).

	Let \( x_{0} \in \halfopenright{0, 1} \). If \( U \) is a neighborhood at \( c(x_{0}) = 1 \) then the preimage of \( U \) under \( c\vert_{\closedinterval{0,1}} \) is \( \halfopenright{0,1} \) (if \( 0 \notin U \)) or \( \closedinterval{0,1} \) (if \( 0 \in U \)). Hence \( c\vert_{\closedinterval{0,1}} \) is continuous on \( \halfopenright{0,1} \).
\end{proof}

\begin{problem}{III.8.3}
Let \(X\) be any space and \(c_{A}\) the characteristic function of \(A \subset X\). Show that \( c_{A}: X \to E^{1} \) is continuous if and only if \(A\) is both open and closed in \(X\).
\end{problem}

\begin{proof}
	Suppose that \( c_{A} \) is continuous. The preimage of \( \openinterval{-1/2, 1/2} \) (this is a neighborhood of \(0\) but doesn't contain \(1\)) under \( c_{A} \) is \( X - A \). The preimage of \( \openinterval{1/2, 3/2} \) (this is a neighborhood of \(1\) but doesn't contain \(0\)) under \( c_{A} \) is \( A \). Hence \( A, X - A \) are open, which means \(A\) is both open and closed in \(X\).

	Suppose that \( A \) is both open and closed in \(X\). Let \( U \) be an open set in \( E^{1} \).

	If \( 0 \notin U, 1 \notin U \) then \( c_{A}^{-1}(U) = \varnothing \) (open in \(X\)).

	If \( 0 \notin U, 1 \in U \) then \( c_{A}^{-1}(U) = A \) (open in \(X\)).

	If \( 0 \in U, 1 \notin U \) then \( c_{A}^{-1}(U) = X - A \) (open in \(X\)).

	If \( 0 \in U, 1 \in U \) then \( c_{A}^{-1}(U) = X \) (open in \(X\)).

	Hence \( c_{A} \) is continuous.
\end{proof}

\begin{problem}{III.8.4}
Let \(C\) be the set of Section 3, Example 7.
\begin{enumerate}[label={(\alph*)}]
	\item Define \( \varphi: C \to E^{1} \) by \( \varphi(f) = f(1) \). Show that \( \varphi \) is continuous in the \( \mathscr{U} \) topology, but is \textit{not} continuous in the \( \mathscr{M} \) topology.
	\item Define \( \psi: C \to E^{1} \) by
	      \[
		      \psi(f) = \int_{0}^{1} f(x) dx.
	      \]

	      Show that \(\psi\) is continuous in both the \(\mathscr{M}\) and \(\mathscr{U}\) topologies.
	\item Are either of these maps continuous in the \( \mathscr{L} \) topology (Problem~\ref{problem:III.3.5})?
\end{enumerate}
\end{problem}

\begin{proof}
	\begin{enumerate}[label={(\alph*)}]
		\item Let \( \openinterval{a,b} \) be an open interval in \( E^{1} \).
		      \[
			      \varphi^{-1}(\openinterval{a,b}) = \left\{ f \in C \mid f(1) \in \openinterval{a,b} \right\}
		      \]

		      For each \( f \in \varphi^{-1}(\openinterval{a, b}) \), define \( r = \min\left\{ f(1) - a, b - f(1) \right\} \). For every \( g \in U(f, r) \)
		      \[
			      \left\vert f(1) - g(1) \right\vert \le \sup_{x} \left\vert f - g \right\vert < r
		      \]

		      so \( g(1) \in \openinterval{a,b} \), which means \( g \in \varphi^{-1}(\openinterval{a, b}) \). Hence \( U(f, r) \subset \varphi^{-1}(\openinterval{a, b}) \), so \( \varphi^{-1}(\openinterval{a, b}) \) is open in \( C \) with the \(\mathscr{U}\) topology. Therefore \( \varphi \) is continuous in the \(\mathscr{U}\) topology.

		      To show that \( \varphi \) is not continuous in the \(\mathscr{M}\) topology, we consider the open interval \( \openinterval{-1, 1} \). From the definition, the zero function \( f(x) \equiv 0 \) is an element of \( \varphi^{-1}(\openinterval{-1, 1}) \). Let \( r > 0 \), consider \( M(f, r) \).

		      Define \( f_{n}: \closedinterval{0,1} \to E^{1} \) by
		      \[
			      f_{n}(x) = \begin{cases}
				      0                & \text{if } 0 \le x < 1 - 1/n,   \\
				      rn(1 + n(x - 1)) & \text{if } 1 - 1/n \le x \le 1.
			      \end{cases}
		      \]

		      then \( f_{n} \in M(f, r) \) as \( \int_{0}^{1} \left\vert f_{n} - f \right\vert = \int_{0}^{1} \left\vert f_{n} \right\vert = \dfrac{1}{2}\cdot\dfrac{1}{n}\cdot rn = \dfrac{r}{2} < r \). However, \( f_{n}(1) = rn \) so \( f_{n}(1) \notin \varphi^{-1}(\openinterval{-1, 1}) \) whenever \( n > \dfrac{1}{r} \). Therefore \( M(f, r) \) is not contained in \( \varphi^{-1}(\openinterval{-1, 1}) \) for any \( r > 0 \), which means \( \varphi^{-1}(\openinterval{-1, 1}) \) is not open. Thus \( \varphi \) is not continuous in the \( \mathscr{M} \) topology.
		\item Let \( \openinterval{a,b} \) be an open interval in \( E^{1} \) and \( f \in \psi^{-1}(\openinterval{a,b}) \) then \( \int_{0}^{1} f \in \openinterval{a,b} \).

		      Define \( r = \min\left\{ -a + \int_{0}^{1}f, b - \int_{0}^{1}f  \right\} \) then for any \( g \in U(f, r) \), one has
		      \[
			      \int_{0}^{1}\left\vert f - g \right\vert \le \sup\limits_{x\in\closedinterval{0,1}} \left\vert f - g \right\vert < r
		      \]

		      so
		      \[
			      \left\vert \int_{0}^{1} g - \int_{0}^{1} f \right\vert \le \int_{0}^{1} \left\vert f - g \right\vert < r
		      \]

		      which means \( \int_{0}^{1} g \in \openinterval{a,b} \), so \( U(f, r) \subset \psi^{-1}(\openinterval{a, b}) \). Hence \( \psi \) is continuous in the \( \mathscr{U} \) topology.

		      On the other hand, for any \( g \in M(f, r) \)
		      \[
			      \left\vert \int_{0}^{1} g - \int_{0}^{1} f \right\vert \le \int_{0}^{1} \left\vert f - g \right\vert < r
		      \]

		      so \( \psi(g) = \int_{0}^{1} g \in \openinterval{a, b} \), which implies \( M(f, r) \subset \psi^{-1}(\openinterval{a, b}) \). Hence \( \psi \) is continuous in the \( \mathscr{M} \) topology.
		\item Let \( \openinterval{a, b} \) be an open interval in \( E^{1} \) and \( f \in \varphi^{-1}(\openinterval{a, b}) \) then \( f(1) \in \openinterval{a, b} \). Define \( \varepsilon = \min\left\{ f(1) - a, b - f(1) \right\} \) then \( U_{(x_{1} = 1, \varepsilon)}(f) \subset \varphi^{-1}(\openinterval{a, b}) \), so \( \varphi^{-1}(\openinterval{a, b}) \) is open in the \( \mathscr{L} \) topology. Hence \( \varphi \) is continuous in the \( \mathscr{L} \) topology.

		      However, \( \psi \) is not continuous in the \( \mathscr{L} \) topology. Counterexample: Let \( f \) be the zero function.\@\( x_{1}, \ldots, x_{n} \in \closedinterval{0,1} \) and \( \varepsilon > 0 \). Construct a function \( g \in C \) such that \( g(x_{i}) = f(x_{i}) \) and the graph of \( g \) has saw-tooth shape.
	\end{enumerate}
\end{proof}

\begin{problem}{III.8.5}
Under what conditions is the bijective map \( 1: (X, \mathscr{T}_{1}) \to (X, \mathscr{T}_{2}) \) not continuous, and with inverse not continuous.
\end{problem}

\begin{proof}
	When \( \mathscr{T}_{1} \) and \( \mathscr{T}_{2} \) are not comparable then \( 1 \) is not continuous and its inverse is not continuous.
\end{proof}

\begin{problem}{III.8.6}
Let \( X \) be the space in Problem~\ref{problem:III.1.1}. State a necessary and sufficient condition that \( f: X \to X \) be continuous.
\end{problem}

\begin{proof}
	\( f: X \to X \) is continuous if and only if the preimage of any finite subset of \(X\) under \(f\) is finite.
\end{proof}

\begin{problem}{III.8.7}\label{problem:III.8.7}
Let \( X \) be the space in Problem~\ref{problem:III.1.4}. Show that \( f: X \to X \) is continuous if and only if it is order-preserving.
\end{problem}

\begin{proof}
	Suppose that \( f: X \to X \) is continuous. Let \( x, y \in X \) such that \( y \prec x \). Let \( U = \left\{ z \mid z \prec f(x) \right\} \) then \( U \) is open. Since \( f \) is continuous, then \( f^{-1}(U) \) is open. Besides, \( x \in f^{-1}(U) \) so \( y \in f^{-1}(U) \) by the definition of openness in \( X \) (see Problem~\ref{problem:III.1.4}). Hence \( f(y) \in U \), which implies \( f(y) \prec f(x) \).

	Suppose that \( f \) is order-preserving. A basis for the topology on \( X \) is
	\[
		\left\{ U_{a} \mid a \in X \right\}
	\]

	in which \( U_{a} = \left\{ x \mid x \prec a \right\} \). Since \( f \) is order-preserving, \( f^{-1}(U_{a}) \) is also a basic open set for each \( a \in X \) (here I make use of the anti-symmetry of \( \prec \)). Therefore \( f \) is continuous.
\end{proof}

\begin{problem}{III.8.8}
Let \( \mathbb{Z}^{+} \) be taken with the topology of Problem~\ref{problem:III.1.5}. Show that \( f: \mathbb{Z}^{+} \to \mathbb{Z}^{+} \) is continuous if and only if \( (m \text{ divides } n) \implies (f(m) \text{ divides } f(n)) \).
\end{problem}

\begin{proof}
	Divisibility is an order relation on \( \mathbb{Z}^{+} \) so according to Problem~\ref{problem:III.8.7}, \( f: \mathbb{Z}^{+} \to \mathbb{Z}^{+} \) is continuous if and only if \( f \) preserves divisibility.
\end{proof}

\begin{problem}{III.8.9}
Let \( X \) be the set of Problem~\ref{problem:III.3.6}, with topology \( \mathscr{T}_{R} \). Derive a necessary and sufficient condition for continuity of \( f: X \to X \).
\end{problem}

\begin{proof}
	The open sets in \( X \) are characterized by the following property: \( U \in \mathscr{T}_{R} \) if and only if \( x \in U \land (y \prec x) \implies y \in U \). According to Problem~\ref{problem:III.8.7}, \( f: X \to X \) is continuous if and only if \( f \) is order-preserving.
\end{proof}

\begin{problem}{III.8.10}
Prove that the following three statements are equivalent:
\begin{enumerate}[label={(\alph*)}]
	\item \( f: X \to Y \) is continuous.
	\item \( f(A^{\prime}) \subset \overline{f(A)} \) for each \( A \subset X \).
	\item \( \operatorname{Fr}\left\lbrack f^{-1}(B) \right\rbrack \subset f^{-1}\left\lbrack \operatorname{Fr}(B) \right\rbrack \) for each \( B \subset Y \).
\end{enumerate}
\end{problem}

\begin{proof}
	(a) \( \implies \) (b) Let \( y \in f(A^{\prime}) \) then there exists \( x \in A^{\prime} \) such that \( f(x) = y \). For every neighborhood \( U \) of \( f(x) \), \( f^{-1}(U) \) is open and a neighborhood of \( x \), so \( f^{-1}(U) \) intersects \( A - \left\{ x \right\} \) as \( x \) is a cluster point of \( A \). Moreover \( f(f^{-1}(U)) \subset U \) and \( f(A - \left\{x\right\}) \subset f(A) \) so \( U \) intersects \( f(A) \). Hence \( y = f(x) \in \overline{f(A)} \), which means \( f(A^{\prime}) \subset \overline{f(A)} \) for each \( A \subset X \).

	(b) \( \implies \) (a) For each \( A \subset X \), \( f(A^{\prime}) \subset \overline{f(A)} \) so
	\[
		f(\overline{A}) = f(A \cup A^{\prime}) = f(A) \cup f(A^{\prime}) \subset \overline{f(A)} \cup \overline{f(A)} = \overline{f(A)}.
	\]

	Hence \( f \) is continuous.

	(a) \( \implies \) (c) For each \( B \subset Y \), \( f^{-1}(\overline{B}) \) is closed in \( X \) as \( f \) is continuous and \( \overline{B} \) is closed in \( Y \). Since \( B \subset \overline{B} \) then \( f^{-1}(B) \subset f^{-1}(\overline{B}) \). Moreover, \( \overline{f^{-1}(B)} \) is the intersection of all closed sets containing \( f^{-1}(B) \) so \( \overline{f^{-1}(B)} \subset f^{-1}(\overline{B}) \). Hence
	\begingroup
	\allowdisplaybreaks%
	\begin{align*}
		\operatorname{Fr}\left\lbrack f^{-1}(B) \right\rbrack & = \overline{f^{-1}(B)} \cap \overline{f^{-1}(Y - B)}       \\
		                                                      & \subset f^{-1}(\overline{B}) \cap f^{-1}(\overline{Y - B}) \\
		                                                      & = f^{-1}(\overline{B} \cap \overline{Y - B})               \\
		                                                      & = f^{-1}\left\lbrack \operatorname{Fr}(B) \right\rbrack.
	\end{align*}
	\endgroup

	(c) \( \implies \) (a) Let \( B \) be a closed set in \( Y \) then \( B = \overline{B} \) and
	\[
		\operatorname{Fr}\left\lbrack f^{-1}(B) \right\rbrack \subset f^{-1}\left\lbrack \operatorname{Fr}(B) \right\rbrack \subset f^{-1}(\overline{B}) = f^{-1}(B)
	\]

	so \( f^{-1}(B) \) contains all of its boundary points, which implies \( f^{-1}(B) \) is closed in \(X\). Hence \( f \) is continuous.
\end{proof}

\begin{problem}{III.8.11}
Let \( X = X_{1} \cup X_{2} \) and \( f: X \to Y \). Assume \( f\vert_{X_{1}} \) and \( f\vert_{X_{2}} \) to be continuous at \( x \in X_{1} \cap X_{2} \). Show that \( f \) is continuous at \( x \).
\end{problem}

\begin{proof}
	Let \( V \) be a neighborhood of \( f(x) \) then there exist a neighborhood \( U_{1} \) of \( x \) in \( X_{1} \) such that \( f(U_{1}) \subset V \) and a neighborhood \( U_{2} \) of \( x \) in \( X_{2} \) such that \( f(U_{2}) \subset V \).

	\( U_{1} = W_{1} \cap X_{1} \) for some open set \( W_{1} \subset X \) and \( U_{2} = W_{2} \cap X_{2} \) for some open set \( W_{2} \subset X \). Let \( U = W_{1} \cap W_{2} \) then \( U \) is a neighborhood of \( x \).

	Let \( t \in U \). The point \( t \) is in \( X_{1} \) or \( X_{2} \) as their union is \( X \). If \( t \in X_{1} \) then \( t \in W_{1} \cap X_{1} = U_{1} \), so \( f(t) \in f(U_{1}) \subset V \). If \( t \in X_{2} \) then \( t \in W_{2} \cap X_{2} = U_{2} \), so \( f(t) \in f(U_{2}) \subset V \). Hence \( f(U) \subset V \).

	Thus \( f \) is continuous at \( x \).
\end{proof}

\begin{problem}{III.8.12}
Let \( f: X \to Y \) be a map and \( A \subseteq X \). Give an example showing \( f\vert_{A} \) continuous, although \( f \) is not continuous at any point of \( A \).
\end{problem}

\begin{proof}
	Let \( f: \mathbb{R} \to \mathbb{R} \) be the Dirichlet function, which means \( f \) is the indicator function of \( \mathbb{Q} \) in \( \mathbb{R} \). Then \( f\vert_{\mathbb{Q}} \) is continuous as it is a constant function but \( f \) is not continuous at any point of \( \mathbb{Q} \).
\end{proof}

\begin{problem}{III.8.13}
Let \( f: X \to Y \) be continuous. If \( B \subseteq Y \) is a \( G_{\delta} \) (resp. \( F_{\sigma} \)), show that \( f^{-1}(B) \) is also a \( G_{\delta} \) (resp. \( F_{\sigma} \)).
\end{problem}

\begin{proof}
	If \( B \) is a \( G_{\delta} \) set then \( B = \bigcap^{\infty}_{n=1} G_{n} \) in which each \( G_{n} \) is open in \( Y \). Because
	\[
		f^{-1}(B) = f^{-1}\left( \bigcap^{\infty}_{n=1} G_{n} \right) = \bigcap^{\infty}_{n=1} f^{-1}(G_{n})
	\]

	and \( f \) is continuous, then \( f^{-1}(B) \) is also a \( G_{\delta} \) set.

	If \( B \) is a \( F_{\delta} \) set then \( B = \bigcup^{\infty}_{n=1} F_{n} \) in which each \( F_{n} \) is closed in \( Y \). Because
	\[
		f^{-1}(B) = f^{-1}\left( \bigcup^{\infty}_{n=1} F_{n} \right) = \bigcup^{\infty}_{n=1} f^{-1}(F_{n})
	\]

	and \( f \) is continuous, then \( f^{-1}(B) \) is also a \( F_{\delta} \) set.
\end{proof}

\begin{problem}{III.8.14}
Construct an example of a map \( f: X \times X \to E^{1} \) continuous in each variable separately but not continuous on \( X \times X \) (use \( X = E^{1} \) for simplicity).
\end{problem}

\begin{proof}
	\[
		f(x, y) = \begin{cases}
			\dfrac{xy}{x^{2} + y^{2}} & (x, y) \ne (0, 0) \\
			0                         & (x, y) = (0, 0)
		\end{cases}
	\]

	then \( x \mapsto f(x, y) \) and \( y \mapsto f(x, y) \) are continuous but \( f \) is not continuous.
\end{proof}

\section{Piecewise definition of maps}

\section{Continuous maps into \( E^{1} \)}

\section{Open maps and closed maps}

\section{Homeomorphism}
