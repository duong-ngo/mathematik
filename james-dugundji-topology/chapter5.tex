\chapter{Connectedness}

\section{Connectedness}

\begin{problem}{V.1.1}
Show that a discrete space having more than one point is never connected and that a space having indiscrete topology is always connected.
\end{problem}

\begin{proof}
	Let \( Y \) be a topological space.

	Assume that \( Y \) has the discrete topology and has more than one point. Let \( y \) be a point in \( Y \) then \( \left\{ y \right\}, Y - \left\{ y \right\} \) are nonempty disjoint open sets in \( Y \) and \( Y = \left\{ y \right\} \cup (Y - \left\{ y \right\}) \). Hence \( Y \) is not connected.

	Assume that \( Y \) has the indiscrete topology. If \( Y \) is empty then it is connected. Otherwise, let \( f: Y \to 2 \) be a continuous map and \( y \in Y \). Because \( f \) is continuous, the preimage \( f^{-1}(f(y)) \) is open in \( Y \). Moreover, \( f^{-1}(f(y)) \) is nonempty and \( Y \) has the indiscrete topology so \( f^{-1}(f(y)) = Y \), which means \( f \) is a constant map, hence not surjective. Thus \( Y \) is connected.
\end{proof}

\begin{problem}{V.1.2}
Show that the extended real line \( \tilde{E}^{1} \) is connected.
\end{problem}

\begin{proof}
	\( E^{1} \) as a subspace of \( \tilde{E}^{1} \) has the Euclidean topology, so \( E^{1} \) is connected. Moreover, the closure of \( E^{1} \) in \( \tilde{E}^{1} \) is \( \tilde{E}^{1} \) itself. Since the closure of a connected set is connected, we conclude that \( \tilde{E}^{1} \) is connected.
\end{proof}

\begin{problem}{V.1.3}\label{problem:V.1.3}
Let \( X \) be an infinite set, with topology \( \mathscr{T} = \left\{ \varnothing \right\} \cup \left\{ A \mid \mathscr{C}A \text{ is finite} \right\} \). Show that \( X \) is connected.
\end{problem}

\begin{proof}
	Assume that \( X \) is not connected then there is a nonempty open set \( U \subset X \) such that \( U \) is closed. According to the definition of \( \mathscr{T} \), \( \mathscr{C}U \) is finite and \( U \) is also finite, which means \( X \) is finite, which is a contradiction. Hence \( X \) is connected.
\end{proof}

\begin{problem}{V.1.4}
Let \( (X, \mathscr{T}) \) be connected, and \( \mathscr{T}_{1} \subset \mathscr{T} \). Prove that \( (X, \mathscr{T}_{1}) \) is connected.
\end{problem}

\begin{proof}
	Let \( f: (X, \mathscr{T}_{1}) \to 2 \) be a continuous map.

	\( 1: (X, \mathscr{T}) \to (X, \mathscr{T}_{1}) \) is continuous, so \( f \circ 1 \) is continuous. If \( f \) is surjective then \( f \circ 1 \) is surjective. Since \( f \circ 1: (X, \mathscr{T}) \to 2 \) is continuous and \( (X, \mathscr{T}) \) is connected, \( f \circ 1 \) is not surjective. Hence \( f \) is not surjective.

	Thus \( (X, \mathscr{T}_{1}) \) is connected.
\end{proof}

\begin{problem}{V.1.5}
Let \( \left\{ A_{i} \mid i \in \mathbb{Z}^{+} \right\} \) be connected sets in \( Y \), with \( A_{i} \cap A_{i+1} \ne \varnothing \) for each \(i\). Prove: \( \bigcup_{i} A_{i} \) is connected.
\end{problem}

\begin{proof}
	Let \( f: \bigcup_{i} A_{i} \to 2 \) be a continuous map and \( x \in A_{1} \)

	\( A_{1} \) is connected so \( f\vert_{A_{1}} \) is not surjective, hence \( f(a_{1}) = f(x) \) for every \( a_{1} \in A_{1} \).

	Assume that \( f(a_{n}) = f(x) \) for every \( a_{n} \in A_{n} \). Let \( a \) be a point in \( A_{n} \cap A_{n+1} \) then \( f(a) = f(x) \). Because \( f\vert_{A_{n+1}} \) is not surjective, it is a constant map, therefore, \( f(a_{n+1}) = f(x) \) for every \( a_{n+1} \in A_{n+1} \).

	By the principle of mathematical induction, \( f\vert_{A_{n}} = f\vert_{\left\{x\right\}} \) for every positive integer \( n \). Thus \( f: \bigcup_{i} A_{i} \to 2 \) is a constant map, hence not surjective, which implies that \( \bigcup_{i} A_{i} \) is connected.
\end{proof}

\begin{problem}{V.1.6}\label{problem:V.1.6}
Let \( \left\{ A_{\alpha} \mid \alpha \in \mathscr{A} \right\} \) be a family of connected subsets of \( Y \), and assume that there exists a connected set \(A\) with \( A \cap A_{\alpha} \ne \varnothing \) for each \( A_{\alpha} \). Show that \( A \cup \bigcup_{\alpha} A_{\alpha} \) is connected.
\end{problem}

\begin{proof}
	\( A, A_{\alpha} \) are connected sets and intersecting so \( A \cup A_{\alpha} \) is a connected set for each \( \alpha \in \mathscr{A} \).

	The connected sets \( A \cup A_{\alpha} \) have at least one common point so their union is a connected set. Hence \( A \cup \bigcup_{\alpha} A_{\alpha} \) is connected.
\end{proof}

\begin{problem}{V.1.7}
Let \( \left\{ A_{\alpha} \mid \alpha \in \mathscr{A} \right\} \) be any family of connected sets. Assume that any two of them have nonempty intersection. Prove that \( \bigcup_{\alpha} A_{\alpha} \) is connected.
\end{problem}

\begin{proof}
	Let \( \alpha_{0} \in \mathscr{A}, x_{0} \in A_{\alpha_{0}} \). Apply Problem~\ref{problem:V.1.6} to \( \left\{ A_{\alpha} \mid \alpha \in \mathscr{A}, \alpha \ne \alpha_{0} \right\} \) and \( A_{\alpha_{0}} \), it follows that \( \bigcup_{\alpha} A_{\alpha} \) is connected.
\end{proof}

\begin{problem}{V.1.8}
Prove:
\begin{enumerate}[label={(\alph*)}]
	\item \(Y\) is connected if and only if every open covering \( \left\{ U_{\alpha} \mid \alpha \in \mathscr{A} \right\} \) of \(Y\) has the following property: For each pair of sets \( U_{\alpha_{1}}, U_{\alpha_{n}} \), there are finitely many \( U_{\alpha_{2}}, \ldots, U_{\alpha_{n-1}} \) such that \( U_{\alpha_{i}} \cap U_{\alpha_{i+1}} \ne \varnothing, i = 1, \ldots, n - 1 \).
	\item \(Y\) is connected if and only if every nbd-finite closed covering has the same property as in (a).
\end{enumerate}
\end{problem}

\begin{proof}
	\begin{enumerate}[label={(\alph*)}]
		\item Suppose \( Y \) is a connected space.

		      Let \( \left\{ U_{\alpha} \mid \alpha \in \mathscr{A} \right\} \) be \textbf{an open covering} of \( Y \). We define a relation \( \sim \) on \( Y \) as follows: \( x \sim y \) if there are finitely many \( U_{\alpha_{1}}, \ldots, U_{\alpha_{n}} \) such that \( U_{\alpha_{i}} \cap U_{\alpha_{i+1}} \ne \varnothing, i = 1, \ldots, n - 1 \), \( x \in U_{\alpha_{1}}, y \in U_{\alpha_{n}} \). Evidently, \( \sim \) is an equivalence relation.

		      Let \( a \in Y \) and \( {[a]}_{\sim} \) be the equivalence class containing \( a \). If \( b \in {[a]}_{\sim} \) then there are finitely many \( U_{\alpha_{1}}, \ldots, U_{\alpha_{n}} \) such that \( U_{\alpha_{i}} \cap U_{\alpha_{i+1}} \ne \varnothing, i = 1, \ldots, n - 1 \), \( a \in U_{\alpha_{1}}, b \in U_{\alpha_{n}} \). Hence \( b \in U_{\alpha_{n}} \subset {[a]}_{\sim} \) so the equivalence class \( {[a]}_{\sim} \) is open in \( Y \).

		      Note that the set of equivalence classes constitutes a partition of \( Y \) so \( {[a]}_{\sim} \) is also closed in \( Y \) for each \( a \in Y \). Since \( Y \) is connected and \( {[a]}_{\sim} \) is open and closed in \( Y \), we conclude that \( Y = {[a]}_{\sim} \).

		      Thus for each pair of sets \( U_{\alpha_{1}}, U_{\alpha_{n}} \) in \( \left\{ U_{\alpha} \mid \alpha \in \mathscr{A} \right\} \), there are finitely many \( U_{\alpha_{2}}, \ldots, U_{\alpha_{n-1}} \) such that \( U_{\alpha_{i}} \cap U_{\alpha_{i+1}} \ne \varnothing, i = 1, \ldots, n - 1 \).

		      Suppose that \( Y \) is not connected then there exists a nonempty open set \( A \subset Y \) such that \( A \ne Y \) and \( Y - A \) is open in \( Y \). The open covering \( \left\{ A, Y - A \right\} \) doesn't have the mentioned property.
		\item Suppose \( Y \) is a connected space.

		      Let \( \left\{ U_{\alpha} \mid \alpha \in \mathscr{A} \right\} \) be \textbf{a nbd-finite closed} covering of \( Y \). We define a relation \( \sim \) on \( Y \) as follows: \( x \sim y \) if there are finitely many \( U_{\alpha_{1}}, \ldots, U_{\alpha_{n}} \) such that \( U_{\alpha_{i}} \cap U_{\alpha_{i+1}} \ne \varnothing, i = 1, \ldots, n - 1 \), \( x \in U_{\alpha_{1}}, y \in U_{\alpha_{n}} \). Evidently, \( \sim \) is an equivalence relation.

		      Let \( a \in Y \) and \( {[a]}_{\sim} \) be the equivalence class containing \( a \). If \( b \in {[a]}_{\sim} \) then there are finitely many \( U_{\alpha_{1}}, \ldots, U_{\alpha_{n}} \) such that \( U_{\alpha_{i}} \cap U_{\alpha_{i+1}} \ne \varnothing, i = 1, \ldots, n - 1 \), \( a \in U_{\alpha_{1}}, b \in U_{\alpha_{n}} \). Hence \( b \in U_{\alpha_{n}} \subset {[a]}_{\sim} \) so the equivalence class \( {[a]}_{\sim} \) is the union of some closed sets in the given nbd-finite closed covering. Hence \( {[a]}_{\sim} \) is closed in \( Y \).

		      The union of the equivalence classes other than \( {[a]}_{\sim} \) is closed in \( Y \) as any subcover of \( \left\{ U_{\alpha} \mid \alpha \in \mathscr{A} \right\} \) is nbd-finite and closed. Therefore \( {[a]}_{\sim} \) is open in \( Y \). Hence \( {[a]}_{\sim} = Y \), which means for each pair of sets \( U_{\alpha_{1}}, U_{\alpha_{n}} \) in \( \left\{ U_{\alpha} \mid \alpha \in \mathscr{A} \right\} \), there are finitely many \( U_{\alpha_{2}}, \ldots, U_{\alpha_{n-1}} \) such that \( U_{\alpha_{i}} \cap U_{\alpha_{i+1}} \ne \varnothing, i = 1, \ldots, n - 1 \).

		      Suppose that \( Y \) is not connected then there exists a nonempty open set \( A \subset Y \) such that \( A \ne Y \) and \( Y - A \) is open in \( Y \). Hence \( A, Y - A \) are closed in \( Y \) and the nbd-finite closed covering \( \left\{ A, Y - A \right\} \) doesn't have the mentioned property.
	\end{enumerate}
\end{proof}

\begin{problem}{V.1.9}
\begin{enumerate}[label={(\alph*)}]
	\item Let \(Y\) be a space and \(A \subset Y\) any subset. Let \(C \subset Y\) be connected, containing points of \(A\) and points not in \(A\). Prove: \(C\) must contain points of the boundary of \(A\).
	\item Why is
	      \[
		      A = \left\{ (x, y, 0) \in E^{3} \mid x^{2} + y^{2} \le 1 \right\}
	      \]

	      and
	      \[
		      C = \left\{ (0, 0, z) \mid -1 \le z \le 1 \right\} \subset E^{3}
	      \]

	      \textit{not} a counterexample to this result?
\end{enumerate}
\end{problem}

\begin{proof}
	\begin{enumerate}[label={(\alph*)}]
		\item Assume that \( C \cap \operatorname{Fr}(A) = \varnothing \) then
		      \begingroup
		      \allowdisplaybreaks%
		      \begin{align*}
			      C & = C \cap Y                                                                                       \\
			        & = C \cap (\operatorname{Int}(A) \cup \operatorname{Fr}(A) \cup \operatorname{Int}(\mathscr{C}A)) \\
			        & = (C \cap \operatorname{Int}(A)) \cup (C \cap \operatorname{Int}(\mathscr{C}A))
		      \end{align*}
		      \endgroup

		      Because \( C \cap A \ne \varnothing \) and \( C \cap \operatorname{Fr}(A) = \varnothing \), it follows that \( C \cap \operatorname{Int}(A) \ne \varnothing \). Similarly, \( C \cap \operatorname{Int}(\mathscr{C}A) \ne \varnothing \). Hence two open sets \( \operatorname{Int}(A) \) and \( \operatorname{Int}(\mathscr{C}A) \) disconnect \( C \), which contradicts connectedness of \(C\).

		      Thus \( C \cap \operatorname{Fr}(A) \ne \varnothing \).
		\item It is not a counterexample because every point of \(A\) is a boundary point of \(A\) and \( C \cap A \ne \varnothing \).
	\end{enumerate}
\end{proof}

\begin{problem}{V.1.10}
For each pair of positive integers \( a, b \), let \( U(a, b) = \left\{ an + b \mid n \in \mathbb{Z} \right\} \cap \mathbb{Z}^{+} \). Prove that \( \left\{ U(a, b) \mid \text{all } (a, b) \text{ such that } a \text{ is relatively prime to } b \right\} \) is a basis for a topology \( \mathscr{T} \) in \( \mathbb{Z}^{+} \). Using this topology, show:
\begin{enumerate}[label={(\alph*)}]
	\item For each prime \(p\), the set \( \left\{ kp \mid k \in \mathbb{Z}^{+}  \right\} \) is closed in \( \mathbb{Z}^{+} \).
	\item If \(P\) is the set of all primes, then \( \operatorname{Int}(P) = \varnothing \).
	\item \( (\mathbb{Z}^{+}, \mathscr{T}) \) is connected.
\end{enumerate}
\end{problem}

\begin{proof}
	Assume that \( U(a, b) \cap U(c, d) \ne \varnothing \) in which \( a, b \)  are relatively prime and \( c, d \) are relatively prime. Let \( x_{0} \) be the smallest element of \( U(a, b) \cap U(c, d) \) then \( a, x_{0} \) are relatively prime and \( c, x_{0} \) are relatively prime, so \( \operatorname{lcm}(a, c), x_{0} \) are relatively prime. Moreover
	\[
		U(\operatorname{lcm}(a, c), x_{0}) = U(a, b) \cap U(c, d)
	\]

	so the given collection is indeed a basis for a topology \( \mathscr{T} \) in \( \mathbb{Z}^{+} \).

	\begin{enumerate}[label={(\alph*)}]
		\item For each prime \(p\), the sets
		      \[
			      \left\{ kp \mid k \in \mathbb{Z}^{+} \right\}, U(p, 1), \ldots, U(p, p - 1)
		      \]

		      constitute a partition of \( \mathbb{Z}^{+} \). Therefore \( \left\{ kp \mid k \in \mathbb{Z}^{+} \right\} \) is closed in \( \mathbb{Z}^{+} \).
		\item Assume that there is a prime \( p \in \operatorname{Int}(P) \) then there exists \( U(a, b) \) such that
		      \[
			      p \in U(a, b) \subset \operatorname{Int}(P)
		      \]

		      then \( p + ap \in U(a, b) \subset \operatorname{Int}(P) \subset P \), which means \( p(1 + a) \) is a prime, which is a contradiction. Hence \( \operatorname{Int}(P) = \varnothing \).
		\item Assume that \( (\mathbb{Z}^{+}, \mathscr{T}) \) is not connected, then there is a nonempty open set \( W \subset \mathbb{Z}^{+} \) such that the complement of \( W \) in \( \mathbb{Z}^{+} \) is also nonempty and open. So there exist \( U(a_{1}, b_{1}) \) such that \( W \cap U(a_{1}, b_{1}) = \varnothing \) and \( U(a_{2}, b_{2}) \) such that \( (\mathbb{Z}^{+} - W) \cap U(a_{2}, b_{2}) = \varnothing \).

		      Suppose that there is a multiple \( ka_{1} \) of \( a_{1} \) in \( W \). Then there exists \( U(a_{2}, b_{2}) \subset W \) such that \( ka_{1} \in U(c, d) \) so there exists an integer \( n \) such that \( ka_{1} = cn + d \). Because \( \gcd(c, d) = 1 \), then \( \gcd(ka_{1}, c) = 1 \), so \( \gcd(a_{1}, c) = 1 \), which means \( U(a_{1}, b_{1}) \cap U(c, d) \ne \varnothing \) due to the Chinese remainder theorem. This is a contradiction since \( W \cap U(a_{1}, b_{1}) = \varnothing \).

		      Hence \( W \) contains no multiple of \( a_{1} \). Similarly, \( \mathbb{Z}^{+} - W \) contains no multiple of \( a_{2} \). Hence every multiple of \( a_{1} \) is in \( \mathbb{Z}^{+} - W \) and every multiple of \( a_{2} \) is in \( W \). On the other hand, every common multiple of \( a_{1} \) and \( a_{2} \) is in \( W \) and \( \mathbb{Z}^{+} - W \), which is a contradiction.

		      Thus \( (\mathbb{Z}^{+}, \mathscr{T}) \) is connected.
	\end{enumerate}
\end{proof}

\section{Applications}

\begin{problem}{V.2.1}
Prove: \( S^{n} \) is connected for all \( n \ge 1 \).
\end{problem}

\begin{proof}
	The map \( f: \closedinterval{0,1} \to S^{1} \) defined by \( f(x) = (\cos(2\pi x), \sin(2\pi x)) \) is continuous. Moreover, \( \closedinterval{0,1} \) is connected so \( S^{1} \) is connected.

	Assume that \( S^{n} \) is connected. In \( \mathbb{R}^{n+2} \), the subspace
	\[
		S^{n+1} \cap \text{(hyperspace \( x_{n+1} = 0 \))} = \left\{ x \in \mathbb{R}^{n+2} \mid x_{n+2} = 0 \land \left\vert x \right\vert = 1 \right\} \cong S^{n}
	\]

	Let \( N = (0, 0, \ldots, 1) \in S^{n+1} \) and \( u \) is a unit vector such that \( u_{n+1} = 0 \) then \( \anglebracket{n, u} = 0 \). Let \( \mathcal{U} \) be the collection of such unit vectors then
	\[
		S^{n+1} \cap \text{(hyperplane \(\anglebracket{u, x} = 0\))} \cong S^{n}
	\]

	because an orthogonal transformation in \( \operatorname{O}(n+2) \) that sends \( N \) to \( u \) is a linear isomorphism (hence homeomorphism, with the Euclidean topology, since any linear map is continuous) and the restriction of \( f_{u} \) to \( S^{n+1} \cap \text{(hyperspace \( x_{n+1} = 0 \))} \) maps \( S^{n+1} \cap \text{(hyperspace \( x_{n+1} = 0 \))} \) to \( S^{n+1} \cap \text{(hyperplane \(\anglebracket{u, x} = 0\))} \).

	Moreover \( S^{n+1} \cap \text{(hyperplane \(\anglebracket{u, x} = 0\))} \) passes through \( \pm n \), therefore the union of all \( S^{n+1} \cap \text{(hyperplane \(\anglebracket{u, x} = 0\))} \) is connected, which means \( S^{n+1} \) is connected.

	By the principle of mathematical induction, \( S^{n} \) is connected for all \( n \ge 1 \).
\end{proof}

\begin{problem}{V.2.2}
Prove: \( I \) is not homeomorphic to \( S^{1} \); and also \( \halfopenright{0, 2\pi} \) is not homeomorphic to \( S^{1} \).
\end{problem}

\begin{proof}
	Assume that \( I \cong S^{1} \) then there exists a homeomorphism \( \varphi: I \cong S^{1} \). The restriction \( \varphi\vert_{I - \left\{ 1/2 \right\}}: \halfopenright{0, 1/2} \cup \halfopenleft{1/2, 1} \to S^{1} - \left\{ \varphi(1/2) \right\} \) is also a homeomorphism. However, \( \halfopenright{0, 1/2} \cup \halfopenleft{1/2, 1} \) is not connected and \( S^{1} - \left\{ \varphi(1/2) \right\} \) is connected, which is a contradiction. Thus \( I \) is not homeomorphic to \( S^{1} \).

	Assume that \( \halfopenright{0, 2\pi} \cong S^{1} \) then there exists a homeomorphism \( \varphi: \halfopenright{0, 2\pi} \cong S^{1} \). The restriction \( \varphi\vert_{\halfopenright{0, 2\pi} - \left\{\pi\right\}}: \halfopenright{0, 2\pi} \cup \openinterval{\pi, 2\pi} \to S^{1} - \left\{ \varphi(\pi) \right\} \) is then a homeomorphism. However, \( \halfopenright{0, 2\pi} \cup \openinterval{\pi, 2\pi} \) is not connected and \( S^{1} - \left\{ \varphi(\pi) \right\} \) is connected, which is a contradiction. Thus \( \halfopenright{0, 2\pi} \) is not homeomorphic to \( S^{1} \).
\end{proof}

\begin{problem}{V.2.3}
Show that \( E^{1} \) is not homeomorphic to \( \tilde{E}^{1} \).
\end{problem}

\begin{proof}
	\( f: \tilde{E}^{1} \to \closedinterval{-1, 1} \) given by \( f(x) = \dfrac{x}{\left\vert x \right\vert + 1} \), \( f(\infty) = 1, f(-\infty) = -1 \) is a homeomorphism. Therefore \( \tilde{E}^{1} \cong \closedinterval{-1, 1} \) and \( E^{1} \cong \openinterval{-1, 1} \).

	Assume that \( E^{1} \cong \tilde{E}^{1} \) then \( \openinterval{-1, 1} \cong \closedinterval{-1, 1} \). There exists a homeomorphism \( \varphi: \closedinterval{-1, 1} \cong \openinterval{-1, 1} \). The restriction of \( \varphi \) to \( \halfopenright{-1, 1} \) maps \( \halfopenright{-1, 1} \) to \( \openinterval{-1, \varphi(1)} \cup \openinterval{\varphi(1), 1} \). However, \( \halfopenright{-1, 1} \) is connected and \( \openinterval{-1, \varphi(1)} \cup \openinterval{\varphi(1), 1} \) is not connected, which is a contradiction. Thus \( E^{1} \) is not homeomorphic to \( \tilde{E}^{1} \).
\end{proof}

\begin{problem}{V.2.4}
Prove that \( S^{n} \) and \( S^{1} \) are not homeomorphic \textcolor{red}{if \( n > 1 \)}.
\end{problem}

\begin{proof}
	Let \( N = (0, 0, \ldots, 1) \in S^{n} \), we will show that \( S^{n} - \left\{ \pm N \right\} \) is connected. For each \( P \in S^{n} - \left\{ \pm N \right\} \), there is a unique point \( P_{0} \) on the equator such that \( P_{0}, \pm N, P \) are coplanar and \( P, P_{0} \) are on the same side of \( ON \). The line segment \( PP_{0} \) is connected. Define a map \( f_{P}: PP_{0} \to S^{n} - \left\{ \pm N \right\} \) by \( f_{P}(X) \) is the intersection of \( S^{n} \) and the ray \( OX \). This is a continuous map so \( f_{P}(PP_{0}) \) is connected and contains \( f_{P}(P_{0}) = P_{0} \). Hence
	\[
		S^{n} - \left\{\pm N\right\} = \bigcup_{P \in S^{n} - \left\{\pm N\right\}} (\text{the equator} \cup f_{P}(PP_{0}))
	\]

	is connected.

	Assume that there is a homeomorphism \( \varphi: S^{n} \cong S^{1} \). Then \( S^{n} - \left\{ \pm N \right\} \) and \( S^{1} - \left\{ \varphi(N), \varphi(-N) \right\} \) are homeomorphic. However \( S^{n} - \left\{ \pm N \right\} \) is connected and \( S^{1} - \left\{ \varphi(N), \varphi(-N) \right\} \) is not connected.

	Thus \( S^{n} \) and \( S^{1} \) are not homeomorphic if \( n > 1 \).
\end{proof}

\begin{problem}{V.2.5}
Let \( n \ge 2 \), and in \( E^{n} \), show:
\begin{enumerate}[label={(\alph*)}]
	\item If \( A_{1} = \left\{ x \in E^{n} \mid \text{all coordinates of \(x\) are rational} \right\} \), then \( E^{n} - A_{1} \) is connected.
	\item If \( A_{2} = \left\{ x \in E^{n} \mid \text{all coordinates of \(x\) are irrational} \right\} \), then \( A_{2} \) is not connected.
	\item If \( A_{3} = \left\{ x \in E^{n} \mid \text{at least one coordinate is irrational} \right\} \), then \( A_{3} \) is connected.
\end{enumerate}
\end{problem}

\begin{proof}
	If \( A \) is a countable subset of \( E^{n} \) then \( E^{n} - A \) is connected for \( n > 1 \).
	\begin{enumerate}[label={(\alph*)}]
		\item \( A_{1} \) is countable so \( E^{n} - A_{1} \) is connected.
		\item Assume that \( A_{2} = \underbrace{(\mathbb{R} - \mathbb{Q}) \times \cdots \times (\mathbb{R} - \mathbb{Q})}_{n} \) is connected then \( p_{1}(A_{2}) = \mathbb{R} - \mathbb{Q} \) is connected, in which \( p_{1} \) is the canonical projection \( p_{1}: E^{n} \to E^{1} \). Let \( r \) be a rational number then \( \mathbb{R} - \mathbb{Q} = ((\mathbb{R} - \mathbb{Q}) \cap \openinterval{-\infty, r}) \cup ((\mathbb{R} - \mathbb{Q}) \cap \openinterval{r, \infty}) \), which means \( \mathbb{R} - \mathbb{Q} \) is not connected. Thus \( A_{2} \) is not connected.
		\item Let \( x, y \) be two distinct points in \( A_{3} \) then either
		      \begin{itemize}
			      \item \( x, y \) have irrational coordinates at different positions.

			            Without loss of generality, assume that \( x_{1}, y_{2} \) are irrational. Let \( z \in A_{3} \) such that \( z_{1} = x_{1}, z_{2} = y_{2} \) then the line segments \( xz, zy \subset A_{3} \), which means \( xz \cup zy \subset A_{3} \).
			      \item \( x, y \) have irrational coordinates at the same position.

			            Without loss of generality, assume that \( x_{1}, y_{1} \) are irrational. Let \( z \in A_{3} \) such that \( z_{2} \) is irrational. According to the previous case, there are a connected set containing \( x, z \), a connected set containing \( y, z \) and they are contained in \( A_{3} \). Hence there is a connected set containing \( x, y \) and contained in \( A_{3} \).
		      \end{itemize}

		      Fix a point \( x \in A_{3} \) then for every \( y \in A_{3} \), there is a connected set containing \( x, y \) and contained in \( A_{3} \). Hence \( A_{3} \) is connected.
	\end{enumerate}
\end{proof}

\begin{problem}{V.2.6}
Show: If \( A \subset E^{n} \) is connected and non-empty then \( \aleph(A) = \mathfrak{c} \) or \(1\).
\end{problem}

\begin{proof}
	Either \( p_{i}(A) \) are singletons or at least one \( p_{i}(A) \) is not singleton.

	If the former is the case then \( A \) is a singleton, which means \( \aleph(A) = 1 \).

	Otherwise, \( p_{i}(A) \) is an interval for some \(i\) and \( \aleph(\pi_{i}(A)) = \mathfrak{c} \), which means \( \aleph(A) \ge \mathfrak{c} \). On the other hand, \( \aleph(A) \le \aleph(\mathbb{R}^{n}) = \mathfrak{c} \). Hence \( \aleph(A) = \mathfrak{c} \).

	Thus \( \aleph(A) = \mathfrak{c} \) or \(1\).
\end{proof}

\begin{problem}{V.2.7}
Let \( \mathscr{A} \subset \mathscr{P}(E^{2}) \) be the family of all connected sets. Find \( \aleph(\mathscr{A}) \).
\end{problem}

\begin{proof}
	The set \( U = E^{2} - (\closedinterval{0, 1} \times \left\{0\right\}) \) is connected and so is its closure \( E^{2} \). For any subset \( S \) of \( \closedinterval{0,1} \), the set \( U \cup (S \times \left\{0\right\}) \) is connected as \( U \subset U \cup (S \times \left\{0\right\}) \subset \overline{U} \). Hence \( \aleph(\mathscr{A}) \ge 2^{\mathfrak{c}} \).

	Moreover, \( \aleph(\mathscr{A}) \le \aleph(\mathscr{P}(E^{2})) = 2^{\mathfrak{c}} \) so \( \aleph(\mathscr{A}) = 2^{\mathfrak{c}} \), according to Bernstein-Schr\"{o}der theorem.

	In general, the cardinality of the family of all connected sets in \( \mathbb{R}^{n} \) (for \(n > 1\)) is \( 2^{\mathfrak{c}} \).
\end{proof}

\begin{problem}{V.2.8}
In Problem~\ref{problem:V.1.3}, show that the only continuous real-valued functions on \(X\) are constant functions.
\end{problem}

\begin{proof}
	Assume that there is a continuous function \( f: X \to E^{1} \) that is nonconstant then there are \( x, y \in X \) such that \( f(x) \ne f(y) \). There are two disjoint open intervals containing \( f(x), f(y) \). The preimages of these open intervals are then nonempty and disjoint in \( X \), which is a contradiction since \( X \) is a connected space. Thus the only continuous real-valued functions on \(X\) are constant functions.
\end{proof}

\begin{problem}{V.2.9}
Show that the analogue of the Bernstein-Schr\"{o}der theorem for topological spaces is not valid, by exhibiting two spaces \( X, Y \) such that \( X \) is homeomorphic to a subset of \( Y \) and \( Y \) is homeomorphic to a subset of \( X \), although \( X \) and \( Y \) are not homeomorphic.
\end{problem}

\begin{proof}
	\( \openinterval{-1, 1} \) and \( \closedinterval{-1, 1} \) are not homeomorphic, however:

	\( \openinterval{-1, 1} \) is homeomorphic to \( \openinterval{-1, 1} \), a proper subset of \( \closedinterval{-1, 1} \).

	\( \closedinterval{-1, 1} \) is homeomorphic to \( \closedinterval{-1/2, 1/2} \), a proper subset of \( \openinterval{-1, 1} \).
\end{proof}

\section{Components}

\section{Local Connectedness}

\section{Path-Connectedness}

