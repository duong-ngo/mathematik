\chapter{Connectedness}

\section{Connectedness}

\begin{problem}{V.1.1}
Show that a discrete space having more than one point is never connected and that a space having indiscrete topology is always connected.
\end{problem}

\begin{proof}
	Let \( Y \) be a topological space.

	Assume that \( Y \) has the discrete topology and has more than one point. Let \( y \) be a point in \( Y \) then \( \left\{ y \right\}, Y - \left\{ y \right\} \) are nonempty disjoint open sets in \( Y \) and \( Y = \left\{ y \right\} \cup (Y - \left\{ y \right\}) \). Hence \( Y \) is not connected.

	Assume that \( Y \) has the indiscrete topology. If \( Y \) is empty then it is connected. Otherwise, let \( f: Y \to 2 \) be a continuous map and \( y \in Y \). Because \( f \) is continuous, the preimage \( f^{-1}(f(y)) \) is open in \( Y \). Moreover, \( f^{-1}(f(y)) \) is nonempty and \( Y \) has the indiscrete topology so \( f^{-1}(f(y)) = Y \), which means \( f \) is a constant map, hence not surjective. Thus \( Y \) is connected.
\end{proof}

\begin{problem}{V.1.2}
Show that the extended real line \( \tilde{E}^{1} \) is connected.
\end{problem}

\begin{proof}
	\( E^{1} \) as a subspace of \( \tilde{E}^{1} \) has the Euclidean topology, so \( E^{1} \) is connected. Moreover, the closure of \( E^{1} \) in \( \tilde{E}^{1} \) is \( \tilde{E}^{1} \) itself. Since the closure of a connected set is connected, we conclude that \( \tilde{E}^{1} \) is connected.
\end{proof}

\begin{problem}{V.1.3}\label{problem:V.1.3}
Let \( X \) be an infinite set, with topology \( \mathscr{T} = \left\{ \varnothing \right\} \cup \left\{ A \mid \mathscr{C}A \text{ is finite} \right\} \). Show that \( X \) is connected.
\end{problem}

\begin{proof}
	Assume that \( X \) is not connected then there is a nonempty open set \( U \subset X \) such that \( U \) is closed. According to the definition of \( \mathscr{T} \), \( \mathscr{C}U \) is finite and \( U \) is also finite, which means \( X \) is finite, which is a contradiction. Hence \( X \) is connected.
\end{proof}

\begin{problem}{V.1.4}
Let \( (X, \mathscr{T}) \) be connected, and \( \mathscr{T}_{1} \subset \mathscr{T} \). Prove that \( (X, \mathscr{T}_{1}) \) is connected.
\end{problem}

\begin{proof}
	Let \( f: (X, \mathscr{T}_{1}) \to 2 \) be a continuous map.

	\( 1: (X, \mathscr{T}) \to (X, \mathscr{T}_{1}) \) is continuous, so \( f \circ 1 \) is continuous. If \( f \) is surjective then \( f \circ 1 \) is surjective. Since \( f \circ 1: (X, \mathscr{T}) \to 2 \) is continuous and \( (X, \mathscr{T}) \) is connected, \( f \circ 1 \) is not surjective. Hence \( f \) is not surjective.

	Thus \( (X, \mathscr{T}_{1}) \) is connected.
\end{proof}

\begin{problem}{V.1.5}
Let \( \left\{ A_{i} \mid i \in \mathbb{Z}^{+} \right\} \) be connected sets in \( Y \), with \( A_{i} \cap A_{i+1} \ne \varnothing \) for each \(i\). Prove: \( \bigcup_{i} A_{i} \) is connected.
\end{problem}

\begin{proof}
	Let \( f: \bigcup_{i} A_{i} \to 2 \) be a continuous map and \( x \in A_{1} \)

	\( A_{1} \) is connected so \( f\vert_{A_{1}} \) is not surjective, hence \( f(a_{1}) = f(x) \) for every \( a_{1} \in A_{1} \).

	Assume that \( f(a_{n}) = f(x) \) for every \( a_{n} \in A_{n} \). Let \( a \) be a point in \( A_{n} \cap A_{n+1} \) then \( f(a) = f(x) \). Because \( f\vert_{A_{n+1}} \) is not surjective, it is a constant map, therefore, \( f(a_{n+1}) = f(x) \) for every \( a_{n+1} \in A_{n+1} \).

	By the principle of mathematical induction, \( f\vert_{A_{n}} = f\vert_{\left\{x\right\}} \) for every positive integer \( n \). Thus \( f: \bigcup_{i} A_{i} \to 2 \) is a constant map, hence not surjective, which implies that \( \bigcup_{i} A_{i} \) is connected.
\end{proof}

\begin{problem}{V.1.6}\label{problem:V.1.6}
Let \( \left\{ A_{\alpha} \mid \alpha \in \mathscr{A} \right\} \) be a family of connected subsets of \( Y \), and assume that there exists a connected set \(A\) with \( A \cap A_{\alpha} \ne \varnothing \) for each \( A_{\alpha} \). Show that \( A \cup \bigcup_{\alpha} A_{\alpha} \) is connected.
\end{problem}

\begin{proof}
	\( A, A_{\alpha} \) are connected sets and intersecting so \( A \cup A_{\alpha} \) is a connected set for each \( \alpha \in \mathscr{A} \).

	The connected sets \( A \cup A_{\alpha} \) have at least one common point so their union is a connected set. Hence \( A \cup \bigcup_{\alpha} A_{\alpha} \) is connected.
\end{proof}

\begin{problem}{V.1.7}
Let \( \left\{ A_{\alpha} \mid \alpha \in \mathscr{A} \right\} \) be any family of connected sets. Assume that any two of them have nonempty intersection. Prove that \( \bigcup_{\alpha} A_{\alpha} \) is connected.
\end{problem}

\begin{proof}
	Let \( \alpha_{0} \in \mathscr{A}, x_{0} \in A_{\alpha_{0}} \). Apply Problem~\ref{problem:V.1.6} to \( \left\{ A_{\alpha} \mid \alpha \in \mathscr{A}, \alpha \ne \alpha_{0} \right\} \) and \( A_{\alpha_{0}} \), it follows that \( \bigcup_{\alpha} A_{\alpha} \) is connected.
\end{proof}

\begin{problem}{V.1.8}
Prove:
\begin{enumerate}[label={(\alph*)}]
	\item \(Y\) is connected if and only if every open covering \( \left\{ U_{\alpha} \mid \alpha \in \mathscr{A} \right\} \) of \(Y\) has the following property: For each pair of sets \( U_{\alpha_{1}}, U_{\alpha_{n}} \), there are finitely many \( U_{\alpha_{2}}, \ldots, U_{\alpha_{n-1}} \) such that \( U_{\alpha_{i}} \cap U_{\alpha_{i+1}} \ne \varnothing, i = 1, \ldots, n - 1 \).
	\item \(Y\) is connected if and only if every nbd-finite closed covering has the same property as in (a).
\end{enumerate}
\end{problem}

\begin{proof}
	\begin{enumerate}[label={(\alph*)}]
		\item Suppose \( Y \) is a connected space.

		      Let \( \left\{ U_{\alpha} \mid \alpha \in \mathscr{A} \right\} \) be \textbf{an open covering} of \( Y \). We define a relation \( \sim \) on \( Y \) as follows: \( x \sim y \) if there are finitely many \( U_{\alpha_{1}}, \ldots, U_{\alpha_{n}} \) such that \( U_{\alpha_{i}} \cap U_{\alpha_{i+1}} \ne \varnothing, i = 1, \ldots, n - 1 \), \( x \in U_{\alpha_{1}}, y \in U_{\alpha_{n}} \). Evidently, \( \sim \) is an equivalence relation.

		      Let \( a \in Y \) and \( {[a]}_{\sim} \) be the equivalence class containing \( a \). If \( b \in {[a]}_{\sim} \) then there are finitely many \( U_{\alpha_{1}}, \ldots, U_{\alpha_{n}} \) such that \( U_{\alpha_{i}} \cap U_{\alpha_{i+1}} \ne \varnothing, i = 1, \ldots, n - 1 \), \( a \in U_{\alpha_{1}}, b \in U_{\alpha_{n}} \). Hence \( b \in U_{\alpha_{n}} \subset {[a]}_{\sim} \) so the equivalence class \( {[a]}_{\sim} \) is open in \( Y \).

		      Note that the set of equivalence classes constitutes a partition of \( Y \) so \( {[a]}_{\sim} \) is also closed in \( Y \) for each \( a \in Y \). Since \( Y \) is connected and \( {[a]}_{\sim} \) is open and closed in \( Y \), we conclude that \( Y = {[a]}_{\sim} \).

		      Thus for each pair of sets \( U_{\alpha_{1}}, U_{\alpha_{n}} \) in \( \left\{ U_{\alpha} \mid \alpha \in \mathscr{A} \right\} \), there are finitely many \( U_{\alpha_{2}}, \ldots, U_{\alpha_{n-1}} \) such that \( U_{\alpha_{i}} \cap U_{\alpha_{i+1}} \ne \varnothing, i = 1, \ldots, n - 1 \).

		      Suppose that \( Y \) is not connected then there exists a nonempty open set \( A \subset Y \) such that \( A \ne Y \) and \( Y - A \) is open in \( Y \). The open covering \( \left\{ A, Y - A \right\} \) doesn't have the mentioned property.
		\item Suppose \( Y \) is a connected space.

		      Let \( \left\{ U_{\alpha} \mid \alpha \in \mathscr{A} \right\} \) be \textbf{a nbd-finite closed} covering of \( Y \). We define a relation \( \sim \) on \( Y \) as follows: \( x \sim y \) if there are finitely many \( U_{\alpha_{1}}, \ldots, U_{\alpha_{n}} \) such that \( U_{\alpha_{i}} \cap U_{\alpha_{i+1}} \ne \varnothing, i = 1, \ldots, n - 1 \), \( x \in U_{\alpha_{1}}, y \in U_{\alpha_{n}} \). Evidently, \( \sim \) is an equivalence relation.

		      Let \( a \in Y \) and \( {[a]}_{\sim} \) be the equivalence class containing \( a \). If \( b \in {[a]}_{\sim} \) then there are finitely many \( U_{\alpha_{1}}, \ldots, U_{\alpha_{n}} \) such that \( U_{\alpha_{i}} \cap U_{\alpha_{i+1}} \ne \varnothing, i = 1, \ldots, n - 1 \), \( a \in U_{\alpha_{1}}, b \in U_{\alpha_{n}} \). Hence \( b \in U_{\alpha_{n}} \subset {[a]}_{\sim} \) so the equivalence class \( {[a]}_{\sim} \) is the union of some closed sets in the given nbd-finite closed covering. Hence \( {[a]}_{\sim} \) is closed in \( Y \).

		      The union of the equivalence classes other than \( {[a]}_{\sim} \) is closed in \( Y \) as any subcover of \( \left\{ U_{\alpha} \mid \alpha \in \mathscr{A} \right\} \) is nbd-finite and closed. Therefore \( {[a]}_{\sim} \) is open in \( Y \). Hence \( {[a]}_{\sim} = Y \), which means for each pair of sets \( U_{\alpha_{1}}, U_{\alpha_{n}} \) in \( \left\{ U_{\alpha} \mid \alpha \in \mathscr{A} \right\} \), there are finitely many \( U_{\alpha_{2}}, \ldots, U_{\alpha_{n-1}} \) such that \( U_{\alpha_{i}} \cap U_{\alpha_{i+1}} \ne \varnothing, i = 1, \ldots, n - 1 \).

		      Suppose that \( Y \) is not connected then there exists a nonempty open set \( A \subset Y \) such that \( A \ne Y \) and \( Y - A \) is open in \( Y \). Hence \( A, Y - A \) are closed in \( Y \) and the nbd-finite closed covering \( \left\{ A, Y - A \right\} \) doesn't have the mentioned property.
	\end{enumerate}
\end{proof}

\begin{problem}{V.1.9}
\begin{enumerate}[label={(\alph*)}]
	\item Let \(Y\) be a space and \(A \subset Y\) any subset. Let \(C \subset Y\) be connected, containing points of \(A\) and points not in \(A\). Prove: \(C\) must contain points of the boundary of \(A\).
	\item Why is
	      \[
		      A = \left\{ (x, y, 0) \in E^{3} \mid x^{2} + y^{2} \le 1 \right\}
	      \]

	      and
	      \[
		      C = \left\{ (0, 0, z) \mid -1 \le z \le 1 \right\} \subset E^{3}
	      \]

	      \textit{not} a counterexample to this result?
\end{enumerate}
\end{problem}

\begin{proof}
	\begin{enumerate}[label={(\alph*)}]
		\item Assume that \( C \cap \operatorname{Fr}(A) = \varnothing \) then
		      \begingroup
		      \allowdisplaybreaks%
		      \begin{align*}
			      C & = C \cap Y                                                                                       \\
			        & = C \cap (\operatorname{Int}(A) \cup \operatorname{Fr}(A) \cup \operatorname{Int}(\mathscr{C}A)) \\
			        & = (C \cap \operatorname{Int}(A)) \cup (C \cap \operatorname{Int}(\mathscr{C}A))
		      \end{align*}
		      \endgroup

		      Because \( C \cap A \ne \varnothing \) and \( C \cap \operatorname{Fr}(A) = \varnothing \), it follows that \( C \cap \operatorname{Int}(A) \ne \varnothing \). Similarly, \( C \cap \operatorname{Int}(\mathscr{C}A) \ne \varnothing \). Hence two open sets \( \operatorname{Int}(A) \) and \( \operatorname{Int}(\mathscr{C}A) \) disconnect \( C \), which contradicts connectedness of \(C\).

		      Thus \( C \cap \operatorname{Fr}(A) \ne \varnothing \).
		\item It is not a counterexample because every point of \(A\) is a boundary point of \(A\) and \( C \cap A \ne \varnothing \).
	\end{enumerate}
\end{proof}

\begin{problem}{V.1.10}
For each pair of positive integers \( a, b \), let \( U(a, b) = \left\{ an + b \mid n \in \mathbb{Z} \right\} \cap \mathbb{Z}^{+} \). Prove that \( \left\{ U(a, b) \mid \text{all } (a, b) \text{ such that } a \text{ is relatively prime to } b \right\} \) is a basis for a topology \( \mathscr{T} \) in \( \mathbb{Z}^{+} \). Using this topology, show:
\begin{enumerate}[label={(\alph*)}]
	\item For each prime \(p\), the set \( \left\{ kp \mid k \in \mathbb{Z}^{+}  \right\} \) is closed in \( \mathbb{Z}^{+} \).
	\item If \(P\) is the set of all primes, then \( \operatorname{Int}(P) = \varnothing \).
	\item \( (\mathbb{Z}^{+}, \mathscr{T}) \) is connected.
\end{enumerate}
\end{problem}

\begin{proof}
	Assume that \( U(a, b) \cap U(c, d) \ne \varnothing \) in which \( a, b \)  are relatively prime and \( c, d \) are relatively prime. Let \( x_{0} \) be the smallest element of \( U(a, b) \cap U(c, d) \) then \( a, x_{0} \) are relatively prime and \( c, x_{0} \) are relatively prime, so \( \operatorname{lcm}(a, c), x_{0} \) are relatively prime. Moreover
	\[
		U(\operatorname{lcm}(a, c), x_{0}) = U(a, b) \cap U(c, d)
	\]

	so the given collection is indeed a basis for a topology \( \mathscr{T} \) in \( \mathbb{Z}^{+} \).

	\begin{enumerate}[label={(\alph*)}]
		\item For each prime \(p\), the sets
		      \[
			      \left\{ kp \mid k \in \mathbb{Z}^{+} \right\}, U(p, 1), \ldots, U(p, p - 1)
		      \]

		      constitute a partition of \( \mathbb{Z}^{+} \). Therefore \( \left\{ kp \mid k \in \mathbb{Z}^{+} \right\} \) is closed in \( \mathbb{Z}^{+} \).
		\item Assume that there is a prime \( p \in \operatorname{Int}(P) \) then there exists \( U(a, b) \) such that
		      \[
			      p \in U(a, b) \subset \operatorname{Int}(P)
		      \]

		      then \( p + ap \in U(a, b) \subset \operatorname{Int}(P) \subset P \), which means \( p(1 + a) \) is a prime, which is a contradiction. Hence \( \operatorname{Int}(P) = \varnothing \).
		\item Assume that \( (\mathbb{Z}^{+}, \mathscr{T}) \) is not connected, then there is a nonempty open set \( W \subset \mathbb{Z}^{+} \) such that the complement of \( W \) in \( \mathbb{Z}^{+} \) is also nonempty and open. So there exist \( U(a_{1}, b_{1}) \) such that \( W \cap U(a_{1}, b_{1}) = \varnothing \) and \( U(a_{2}, b_{2}) \) such that \( (\mathbb{Z}^{+} - W) \cap U(a_{2}, b_{2}) = \varnothing \).

		      Suppose that there is a multiple \( ka_{1} \) of \( a_{1} \) in \( W \). Then there exists \( U(a_{2}, b_{2}) \subset W \) such that \( ka_{1} \in U(c, d) \) so there exists an integer \( n \) such that \( ka_{1} = cn + d \). Because \( \gcd(c, d) = 1 \), then \( \gcd(ka_{1}, c) = 1 \), so \( \gcd(a_{1}, c) = 1 \), which means \( U(a_{1}, b_{1}) \cap U(c, d) \ne \varnothing \) due to the Chinese remainder theorem. This is a contradiction since \( W \cap U(a_{1}, b_{1}) = \varnothing \).

		      Hence \( W \) contains no multiple of \( a_{1} \). Similarly, \( \mathbb{Z}^{+} - W \) contains no multiple of \( a_{2} \). Hence every multiple of \( a_{1} \) is in \( \mathbb{Z}^{+} - W \) and every multiple of \( a_{2} \) is in \( W \). On the other hand, every common multiple of \( a_{1} \) and \( a_{2} \) is in \( W \) and \( \mathbb{Z}^{+} - W \), which is a contradiction.

		      Thus \( (\mathbb{Z}^{+}, \mathscr{T}) \) is connected.
	\end{enumerate}
\end{proof}

\section{Applications}

\begin{problem}{V.2.1}
Prove: \( S^{n} \) is connected for all \( n \ge 1 \).
\end{problem}

\begin{proof}
	The map \( f: \closedinterval{0,1} \to S^{1} \) defined by \( f(x) = (\cos(2\pi x), \sin(2\pi x)) \) is continuous. Moreover, \( \closedinterval{0,1} \) is connected so \( S^{1} \) is connected.

	Assume that \( S^{n} \) is connected. In \( \mathbb{R}^{n+2} \), the subspace
	\[
		S^{n+1} \cap \text{(hyperspace \( x_{n+1} = 0 \))} = \left\{ x \in \mathbb{R}^{n+2} \mid x_{n+2} = 0 \land \left\vert x \right\vert = 1 \right\} \cong S^{n}
	\]

	Let \( N = (0, 0, \ldots, 1) \in S^{n+1} \) and \( u \) is a unit vector such that \( u_{n+1} = 0 \) then \( \anglebracket{n, u} = 0 \). Let \( \mathcal{U} \) be the collection of such unit vectors then
	\[
		S^{n+1} \cap \text{(hyperplane \(\anglebracket{u, x} = 0\))} \cong S^{n}
	\]

	because an orthogonal transformation in \( \operatorname{O}(n+2) \) that sends \( N \) to \( u \) is a linear isomorphism (hence homeomorphism, with the Euclidean topology, since any linear map is continuous) and the restriction of \( f_{u} \) to \( S^{n+1} \cap \text{(hyperspace \( x_{n+1} = 0 \))} \) maps \( S^{n+1} \cap \text{(hyperspace \( x_{n+1} = 0 \))} \) to \( S^{n+1} \cap \text{(hyperplane \(\anglebracket{u, x} = 0\))} \).

	Moreover \( S^{n+1} \cap \text{(hyperplane \(\anglebracket{u, x} = 0\))} \) passes through \( \pm n \), therefore the union of all \( S^{n+1} \cap \text{(hyperplane \(\anglebracket{u, x} = 0\))} \) is connected, which means \( S^{n+1} \) is connected.

	By the principle of mathematical induction, \( S^{n} \) is connected for all \( n \ge 1 \).
\end{proof}

\begin{problem}{V.2.2}
Prove: \( I \) is not homeomorphic to \( S^{1} \); and also \( \halfopenright{0, 2\pi} \) is not homeomorphic to \( S^{1} \).
\end{problem}

\begin{proof}
	Assume that \( I \cong S^{1} \) then there exists a homeomorphism \( \varphi: I \cong S^{1} \). The restriction \( \varphi\vert_{I - \left\{ 1/2 \right\}}: \halfopenright{0, 1/2} \cup \halfopenleft{1/2, 1} \to S^{1} - \left\{ \varphi(1/2) \right\} \) is also a homeomorphism. However, \( \halfopenright{0, 1/2} \cup \halfopenleft{1/2, 1} \) is not connected and \( S^{1} - \left\{ \varphi(1/2) \right\} \) is connected, which is a contradiction. Thus \( I \) is not homeomorphic to \( S^{1} \).

	Assume that \( \halfopenright{0, 2\pi} \cong S^{1} \) then there exists a homeomorphism \( \varphi: \halfopenright{0, 2\pi} \cong S^{1} \). The restriction \( \varphi\vert_{\halfopenright{0, 2\pi} - \left\{\pi\right\}}: \halfopenright{0, 2\pi} \cup \openinterval{\pi, 2\pi} \to S^{1} - \left\{ \varphi(\pi) \right\} \) is then a homeomorphism. However, \( \halfopenright{0, 2\pi} \cup \openinterval{\pi, 2\pi} \) is not connected and \( S^{1} - \left\{ \varphi(\pi) \right\} \) is connected, which is a contradiction. Thus \( \halfopenright{0, 2\pi} \) is not homeomorphic to \( S^{1} \).
\end{proof}

\begin{problem}{V.2.3}
Show that \( E^{1} \) is not homeomorphic to \( \tilde{E}^{1} \).
\end{problem}

\begin{proof}
	\( f: \tilde{E}^{1} \to \closedinterval{-1, 1} \) given by \( f(x) = \dfrac{x}{\left\vert x \right\vert + 1} \), \( f(\infty) = 1, f(-\infty) = -1 \) is a homeomorphism. Therefore \( \tilde{E}^{1} \cong \closedinterval{-1, 1} \) and \( E^{1} \cong \openinterval{-1, 1} \).

	Assume that \( E^{1} \cong \tilde{E}^{1} \) then \( \openinterval{-1, 1} \cong \closedinterval{-1, 1} \). There exists a homeomorphism \( \varphi: \closedinterval{-1, 1} \cong \openinterval{-1, 1} \). The restriction of \( \varphi \) to \( \halfopenright{-1, 1} \) maps \( \halfopenright{-1, 1} \) to \( \openinterval{-1, \varphi(1)} \cup \openinterval{\varphi(1), 1} \). However, \( \halfopenright{-1, 1} \) is connected and \( \openinterval{-1, \varphi(1)} \cup \openinterval{\varphi(1), 1} \) is not connected, which is a contradiction. Thus \( E^{1} \) is not homeomorphic to \( \tilde{E}^{1} \).
\end{proof}

\begin{problem}{V.2.4}
Prove that \( S^{n} \) and \( S^{1} \) are not homeomorphic \textcolor{red}{if \( n > 1 \)}.
\end{problem}

\begin{proof}
	Let \( N = (0, 0, \ldots, 1) \in S^{n} \), we will show that \( S^{n} - \left\{ \pm N \right\} \) is connected. For each \( P \in S^{n} - \left\{ \pm N \right\} \), there is a unique point \( P_{0} \) on the equator such that \( P_{0}, \pm N, P \) are coplanar and \( P, P_{0} \) are on the same side of \( ON \). The line segment \( PP_{0} \) is connected. Define a map \( f_{P}: PP_{0} \to S^{n} - \left\{ \pm N \right\} \) by \( f_{P}(X) \) is the intersection of \( S^{n} \) and the ray \( OX \). This is a continuous map so \( f_{P}(PP_{0}) \) is connected and contains \( f_{P}(P_{0}) = P_{0} \). Hence
	\[
		S^{n} - \left\{\pm N\right\} = \bigcup_{P \in S^{n} - \left\{\pm N\right\}} (\text{the equator} \cup f_{P}(PP_{0}))
	\]

	is connected.

	Assume that there is a homeomorphism \( \varphi: S^{n} \cong S^{1} \). Then \( S^{n} - \left\{ \pm N \right\} \) and \( S^{1} - \left\{ \varphi(N), \varphi(-N) \right\} \) are homeomorphic. However \( S^{n} - \left\{ \pm N \right\} \) is connected and \( S^{1} - \left\{ \varphi(N), \varphi(-N) \right\} \) is not connected.

	Thus \( S^{n} \) and \( S^{1} \) are not homeomorphic if \( n > 1 \).
\end{proof}

\begin{problem}{V.2.5}
Let \( n \ge 2 \), and in \( E^{n} \), show:
\begin{enumerate}[label={(\alph*)}]
	\item If \( A_{1} = \left\{ x \in E^{n} \mid \text{all coordinates of \(x\) are rational} \right\} \), then \( E^{n} - A_{1} \) is connected.
	\item If \( A_{2} = \left\{ x \in E^{n} \mid \text{all coordinates of \(x\) are irrational} \right\} \), then \( A_{2} \) is not connected.
	\item If \( A_{3} = \left\{ x \in E^{n} \mid \text{at least one coordinate is irrational} \right\} \), then \( A_{3} \) is connected.
\end{enumerate}
\end{problem}

\begin{proof}
	If \( A \) is a countable subset of \( E^{n} \) then \( E^{n} - A \) is connected for \( n > 1 \).
	\begin{enumerate}[label={(\alph*)}]
		\item \( A_{1} \) is countable so \( E^{n} - A_{1} \) is connected.
		\item Assume that \( A_{2} = \underbrace{(\mathbb{R} - \mathbb{Q}) \times \cdots \times (\mathbb{R} - \mathbb{Q})}_{n} \) is connected then \( p_{1}(A_{2}) = \mathbb{R} - \mathbb{Q} \) is connected, in which \( p_{1} \) is the canonical projection \( p_{1}: E^{n} \to E^{1} \). Let \( r \) be a rational number then \( \mathbb{R} - \mathbb{Q} = ((\mathbb{R} - \mathbb{Q}) \cap \openinterval{-\infty, r}) \cup ((\mathbb{R} - \mathbb{Q}) \cap \openinterval{r, \infty}) \), which means \( \mathbb{R} - \mathbb{Q} \) is not connected. Thus \( A_{2} \) is not connected.
		\item Let \( x, y \) be two distinct points in \( A_{3} \) then either
		      \begin{itemize}
			      \item \( x, y \) have irrational coordinates at different positions.

			            Without loss of generality, assume that \( x_{1}, y_{2} \) are irrational. Let \( z \in A_{3} \) such that \( z_{1} = x_{1}, z_{2} = y_{2} \) then the line segments \( xz, zy \subset A_{3} \), which means \( xz \cup zy \subset A_{3} \).
			      \item \( x, y \) have irrational coordinates at the same position.

			            Without loss of generality, assume that \( x_{1}, y_{1} \) are irrational. Let \( z \in A_{3} \) such that \( z_{2} \) is irrational. According to the previous case, there are a connected set containing \( x, z \), a connected set containing \( y, z \) and they are contained in \( A_{3} \). Hence there is a connected set containing \( x, y \) and contained in \( A_{3} \).
		      \end{itemize}

		      Fix a point \( x \in A_{3} \) then for every \( y \in A_{3} \), there is a connected set containing \( x, y \) and contained in \( A_{3} \). Hence \( A_{3} \) is connected.
	\end{enumerate}
\end{proof}

\begin{problem}{V.2.6}
Show: If \( A \subset E^{n} \) is connected and non-empty then \( \aleph(A) = \mathfrak{c} \) or \(1\).
\end{problem}

\begin{proof}
	Either \( p_{i}(A) \) are singletons or at least one \( p_{i}(A) \) is not singleton.

	If the former is the case then \( A \) is a singleton, which means \( \aleph(A) = 1 \).

	Otherwise, \( p_{i}(A) \) is an interval for some \(i\) and \( \aleph(\pi_{i}(A)) = \mathfrak{c} \), which means \( \aleph(A) \ge \mathfrak{c} \). On the other hand, \( \aleph(A) \le \aleph(\mathbb{R}^{n}) = \mathfrak{c} \). Hence \( \aleph(A) = \mathfrak{c} \).

	Thus \( \aleph(A) = \mathfrak{c} \) or \(1\).
\end{proof}

\begin{problem}{V.2.7}
Let \( \mathscr{A} \subset \mathscr{P}(E^{2}) \) be the family of all connected sets. Find \( \aleph(\mathscr{A}) \).
\end{problem}

\begin{proof}
	The set \( U = E^{2} - (\closedinterval{0, 1} \times \left\{0\right\}) \) is connected and so is its closure \( E^{2} \). For any subset \( S \) of \( \closedinterval{0,1} \), the set \( U \cup (S \times \left\{0\right\}) \) is connected as \( U \subset U \cup (S \times \left\{0\right\}) \subset \overline{U} \). Hence \( \aleph(\mathscr{A}) \ge 2^{\mathfrak{c}} \).

	Moreover, \( \aleph(\mathscr{A}) \le \aleph(\mathscr{P}(E^{2})) = 2^{\mathfrak{c}} \) so \( \aleph(\mathscr{A}) = 2^{\mathfrak{c}} \), according to Bernstein-Schr\"{o}der theorem.

	In general, the cardinality of the family of all connected sets in \( \mathbb{R}^{n} \) (for \(n > 1\)) is \( 2^{\mathfrak{c}} \).
\end{proof}

\begin{problem}{V.2.8}
In Problem~\ref{problem:V.1.3}, show that the only continuous real-valued functions on \(X\) are constant functions.
\end{problem}

\begin{proof}
	Assume that there is a continuous function \( f: X \to E^{1} \) that is nonconstant then there are \( x, y \in X \) such that \( f(x) \ne f(y) \). There are two disjoint open intervals containing \( f(x), f(y) \). The preimages of these open intervals are then nonempty and disjoint in \( X \), which is a contradiction since \( X \) is a connected space. Thus the only continuous real-valued functions on \(X\) are constant functions.
\end{proof}

\begin{problem}{V.2.9}
Show that the analogue of the Bernstein-Schr\"{o}der theorem for topological spaces is not valid, by exhibiting two spaces \( X, Y \) such that \( X \) is homeomorphic to a subset of \( Y \) and \( Y \) is homeomorphic to a subset of \( X \), although \( X \) and \( Y \) are not homeomorphic.
\end{problem}

\begin{proof}
	\( \openinterval{-1, 1} \) and \( \closedinterval{-1, 1} \) are not homeomorphic, however:

	\( \openinterval{-1, 1} \) is homeomorphic to \( \openinterval{-1, 1} \), a proper subset of \( \closedinterval{-1, 1} \).

	\( \closedinterval{-1, 1} \) is homeomorphic to \( \closedinterval{-1/2, 1/2} \), a proper subset of \( \openinterval{-1, 1} \).
\end{proof}

\section{Components}

\begin{problem}{V.3.1}
In a space \(X\), define \( x \sim y \) if \( x \) and \( y \) are contained in a connected set. Show that this is an equivalence relation. What are the equivalence classes?
\end{problem}

\begin{proof}
	\( \sim \) is reflexive and symmetric. If \( x \sim y \) and \( y \sim z \) then \( x, y \) are contained in a connected set \( U \) and \( y, z \) are contained in a connected set \( V \). Two connected sets \( U, V \) are not disjoint so \( U \cup V \) is connected, which implies \( x \sim z \) as \( x, z \in U \cup V \). Hence \( \sim \) is transitive, from which we conclude that \( \sim \) is an equivalence relation.

	The equivalence classes are components of \(X\).
\end{proof}

\begin{problem}{V.3.2}
Let \(A \subset Y\), where both \(A\) and \(Y\) are connected. Let \(U\) be any set both open and closed in \(Y - A\). Prove that \( A \cup U \) is connected.
\end{problem}

\begin{quotation}
	We prove this lemma: Let \( X \) be a connected space, \( C \) a connected subset of \( X \). If \( M, N \) are separated sets such that \( X - C = M \cup N \) then \( C \cup M, C \cup N \) are connected. Moreover, if \( C \) is closed then \( C \cup M, C \cup N \) are closed.

	\begin{proof}
		Assume that \( C \cup M = A \cup B \) in which \( A, B \) are separated. Since \( C \subset A \cup B \) and \( C \) is connected, it follows that \( C \cap A = \varnothing \) or \( C \cap B = \varnothing \). Without loss of generality, assume that \( C \cap A = \varnothing \) then \( A \subset M \) since \( A \subset C \cup M \).

		\( M \) and \( N \) are separated and \( A \subset M \) so \( A, N \) are separated. Therefore \( A, N \cup B \) are separated.
		\[
			X = C \cup M \cup N = A \cup B \cup N
		\]

		so \( A \) or \( B \cup N \) is empty, as \( X \) is connected. Hence \( A \) or \( B \) is empty, which means \( C \cup M \) is connnected. Similarly, \( C \cup N \) is connected.

		If \( C \) is closed then
		\[
			\overline{C \cup M} = \overline{C} \cup \overline{M} = (C \cup \overline{M}) \cap (C \cup M \cup N) = C \cup M
		\]

		which means \( C \cup M \) is closed.
	\end{proof}
\end{quotation}

\begin{proof}
	Let \( V \) be the complement of \( U \) in \( Y - A \). Two sets \( U, V \) are closed in \( Y - A \) so
	\begingroup
	\allowdisplaybreaks%
	\begin{align*}
		U & = \operatorname{cl}_{Y - A}(U) = \operatorname{cl}_{Y}(U) \cap (Y - A) \\
		V & = \operatorname{cl}_{Y - A}(V) = \operatorname{cl}_{Y}(V) \cap (Y - A)
	\end{align*}
	\endgroup

	Since \( U, V \subset Y - A \)
	\begingroup
	\allowdisplaybreaks%
	\begin{align*}
		U \cap \operatorname{cl}_{Y}(V) & = U \cap (Y - A) \cap \operatorname{cl}_{Y}(V) = U \cap \operatorname{cl}_{Y - A}(V) = U \cap V = \varnothing, \\
		\operatorname{cl}_{Y}(U) \cap V & = \operatorname{cl}_{Y}(U) \cap (Y - A) \cap V = \operatorname{cl}_{Y - A}(U) \cap V = U \cap V = \varnothing.
	\end{align*}
	\endgroup

	Hence \( U, V \) are separated sets. According to the previous lemma, we conclude that \( A \cup U, A \cup V \) are connected sets.
\end{proof}

\begin{proof}[Another proof]
	Assume that
	\[
		A\cup U = ((A \cup U) \cap V_{1}) \cup ((A \cup U) \cap V_{2}) = (A \cup U) \cap (V_{1} \cup V_{2})
	\]

	in which \( V_{1}, V_{2} \) are open in \(Y\) and \( V_{1} \cap V_{2} \cap (A \cup U) = \varnothing \).

	Since \( A \subset A \cup U = ((A \cup U) \cap V_{1}) \cup ((A \cup U) \cap V_{2}) \), \( A \) is connected, and \( ((A \cup U) \cap V_{1}) \), \( ((A \cup U) \cap V_{2}) \) are disjoint, we deduce that either \( A \subset (A \cup U) \cap V_{1} \) or \( A \subset (A \cup U) \cap V_{2} \).

	Without loss of generality, suppose that \( A \subset (A \cup U) \cap V_{2} \) then \( A \subset V_{2} \) since \( A \cap U = \varnothing \). Therefore
	\[
		\varnothing = A \cap (A \cup U) \cap V_{1} = (A \cap A \cap V_{1}) \cup (A \cap U \cap V_{1}) = A \cap V_{1}.
	\]

	\( U \) is open in \( Y - A \) so there exists an open set \( U_{0} \) in \(X\) such that \( U = U_{0} \cap (Y - A) \). So \( U = U_{0} - A \).

	If \( x \in U_{0} \cap V_{1} \) then \( x \in U_{0} \) and \( x \in Y - A \), which implies \( x \in U_{0} \cap (Y - A) = U \). Hence \( U_{0} \cap V_{1} \subset U \).

	\( U_{0} \cap V_{1} \) is an open set in \( X \) and
	\[
		(A \cup U) \cap (U_{0} \cap V_{1}) = U \cap U_{0} \cap V_{1} = U \cap V_{1} = (A \cup U) \cap V_{1}
	\]

	so we can assume that \( V_{1} \subset U \) by replacing \( V_{1} \) with \( U_{0} \cap V_{1} \).

	\( U \) is also closed in \( Y - A \) so \( (Y - A) - U \) is open in \( Y - A \). Hence there exists an open set \( U_{1} \) in \( Y \) such that
	\[
		(Y - A) - U = (Y - A) \cap U_{1} = U_{1} - A.
	\]

	Moreover
	\begingroup
	\allowdisplaybreaks%
	\begin{align*}
		(V_{2} \cup U_{1}) \cap (A \cup U) & = (V_{2} \cap A) \cup (U_{1} \cap A) \cup (V_{2} \cap U) \cup (U_{1} \cap U) \\
		                                   & = A \cup (U_{1} \cap A) \cup (V_{2} \cap U) \cup \varnothing                 \\
		                                   & = A \cup (V_{2} \cap U)                                                      \\
		                                   & = (A \cup V_{2}) \cap (A \cup U)                                             \\
		                                   & = V_{2} \cup (A \cup U)
	\end{align*}
	\endgroup

	so we can replace \( V_{2} \) with \( V_{2} \cup U_{1} \).

	\( (Y - A) - U \subset U_{1} \subset V_{2} \) so \( Y \subset (A \cup U) \cup V_{2} \). Besides, \( A \cup U = (A \cup U) \cap (V_{1} \cup V_{2}) \subset V_{1} \cup V_{2} \).

	Therefore \( Y \subset V_{1} \cup V_{2} \), which means \( Y = V_{1} \cup V_{2} \). On the other hand, \( V_{1} \subset U \)
	\begingroup
	\allowdisplaybreaks%
	\begin{align*}
		\varnothing & = V_{1} \cap V_{2} \cap (A \cup U)                         \\
		            & = (V_{1} \cap V_{2} \cap A) \cup (V_{1} \cap V_{2} \cap U) \\
		            & = (V_{1} \cap V_{2} \cap A) \cup (V_{1} \cap V_{2})        \\
		            & = V_{1} \cap V_{2}
	\end{align*}
	\endgroup

	so \( Y \) is the union of disjoint open sets \( V_{1}, V_{2} \). Since \( Y \) is connected, either \( V_{1} \) or \( V_{2} \) is empty.

	\bigskip

	Now assume that there exist disjoint open sets \( W_{1}, W_{2} \) in \( A \cup U \) such that \( A\cup U = W_{1} \cup W_{2} \). There exist open sets \( V_{1}, V_{2} \) in \( Y \) such that \( W_{1} = (A\cup U) \cap V_{1} \) and \( W_{2} = (A\cup U) \cap V_{2} \). According to the previous paragraph, we can assume that \( V_{1} \cup V_{2} = Y \) and \( V_{1} \cap V_{2} = \varnothing \), from which we deduce that \( V_{1} = \varnothing \) or \( V_{2} = \varnothing \), which means \( W_{1} = \varnothing \) or \( W_{2} = \varnothing \). Hence \( A \cup U \) is connected.
\end{proof}

\begin{problem}{V.3.3}
Let \( A \subset Y \) where both \( A \) and \( Y \) are connected. Let \( C \) be any component of \( Y - A \). Show that \( Y - C \) is connected.
\end{problem}

\begin{proof}
	Assume that \( Y - C \) is the union of separated sets \( U, V \), which means
	\[
		Y - C = U \cup V,\qquad U \cap \overline{V} = \overline{U} \cap V = \varnothing.
	\]

	\( A \subset U \cup V \) and \( A \) is connected so either \( A \subset U \) or \( A \subset V \). Without loss of generality, assume that \( A \subset U \) then \( A \cap (C \cup V) = \varnothing \). Therefore \( C \subset C \cup V \subset Y - A \). According to the previous lemma, \( C \cup V \) is connected. Since \( C \) is a component in \( Y - A \), it follows that \( C = C \cup V \). As \( C \cap V = \varnothing \), we deduce that \( V = \varnothing \).

	Thus \( Y - C \) is connected.
\end{proof}

\begin{problem}{V.3.4}\label{problem:V.3.4}
Let \( B \subset Y \) be a \textcolor{red}{nonempty} connected set both open and closed in \( Y \). Prove: \(B\) is a component of \(Y\).
\end{problem}

\begin{proof}
	\( B \) is connected so it is contained in a component \( C \) of \( Y \).

	\( B = C \cap B \) is open and closed in \( C \) as \( B \) is open and closed in \( Y \). Because \( C \) is connected and \( B \) is nonempty, open and closed in \( C \), it follows that \( B = C \). Hence \( B \) is a component of \( Y \).
\end{proof}

\begin{problem}{V.3.5}
In a space \(X\), define \( x \sim y \) if there is no decomposition of \(X\) into two disjoint open sets, one of which contains \(x\), and the other \(y\).
\begin{enumerate}[label={(\alph*)}]
	\item Prove that this is an equivalence relation in \(X\). The equivalence classes are called the quasi-component of \(X\).
	\item Prove each quasi-component is the intersection of all the open closed sets containing a given element.
	\item Prove: Each component is contained in a quasi-component.
	\item In \(E^{2}\) let \(L_{1}\) be the line \( x = 1 \) and \(L_{2}\) the line \(x = -1\). For each \( n \in \mathbb{Z}^{+} \), let \( R_{n} \) be the rectangle \( \left\{ (x, y) \mid \left\vert x\right\vert \le n/(n + 1), \left\vert y\right\vert \le n \right\} \). Finally, let \(Y\) be the subspace \( L_{1} \cup L_{2} \cup \bigcup_{n} \operatorname{Fr}(R_{n}) \) of \( E^{2} \). Show: The component of \( (1, 0) \) is \( L_{1} \) and the quasi-component of \( (1, 0) \) is \( L_{1} \cup L_{2} \).
	\item Show that if \( x_{1}, x_{2} \) (respectively, \( y_{1}, y_{2} \)) belong to a quasi-component of \(X\) (respectively, \(Y\)), then \( (x_{1}, y_{1}) \) and \( (x_{2}, y_{2}) \) belong to a quasi-component of \( X \times Y \).
\end{enumerate}
\end{problem}

\begin{proof}
	\begin{enumerate}[label={(\alph*)}]
		\item From the definition, it follows that \( \sim \) is reflexive and symmetric.

		      Assume that there exist \( x, y, z \in X \) such that \( x \sim y, y \sim z \) and there is a decomposition of \(X\) into two disjoint open sets \( U, V \) such that \( x \in U \) and \( z \in V \). Since \( x \sim y \) and \( y \sim z \), it follows that \( y \notin V \) and \( y \notin U \), which means \( y \notin X \), which is a contradiction. Therefore \( \sim \) is transitive.

		      Thus \( \sim \) is an equivalence relation in \(X\).
		\item Let \( Q \) be a quasi-component in \( X \), \( x \in Q \) and \( S \) be any open closed set containing \( x \). Suppose that \( x \sim y \) then \( y \in S \) by the definition of \( \sim \). Thus each quasi-component is the intersection of all the open closed sets containing an element.
		\item Let \( C \) be a component of \( X \) and \( x, y \in C \). Assume that \( x, y \) are not in a quasi-component then there is a decomposition of \( X \) into two disjoint open sets \( U, V \) such that \( x \in U, y \in V \). Therefore \( C = (C \cap U) \cup (C \cap V) \), which contradicts the connectedness of \( C \). Hence each component is contained in a quasi-component.
		\item Each \( \operatorname{Fr}(R_{n}) \) is connected, open and closed in \( Y \) so each \( \operatorname{Fr}(R_{n}) \) is a component of \( Y \), according to Problem~\ref{problem:V.3.4}.

		      \( L_{1}, L_{2} \) are connected sets in \( Y \). From the previous paragraph, either \( L_{1}, L_{2} \) are two components of \( Y \) or \( L_{1} \cup L_{2} \) is a component of \( Y \).

		      The induced topology for \( L_{1} \cup L_{2} \), as a subspace of \( Y \) is the same as the indcued topology for \( L_{1} \cup L_{2} \), as a subspace of \( E^{2} \). Since \( L_{1} \cup L_{2} \) is not connected in \( E^{2} \), we deduce that \( L_{1} \cup L_{2} \) is not a component of \( Y \).

		      Hence \( L_{1}, L_{2} \) are two distinct components of \( Y \). Thus the components of \( Y \) are \( L_{1}, L_{2}, \operatorname{Fr}(R_{n}) \).

		      \bigskip
		      Each \( \operatorname{Fr}(R_{n}) \) is an open and closed component in \( Y \) so each \( \operatorname{Fr}(R_{n}) \) is a quasi-component in \( Y \).

		      \( L_{1} \) is not open in \( Y \) as every neighborhood of any point in \( L_{1} \) intersects infinitely many \( \operatorname{Fr}(R_{n}) \). Therefore \( L_{1} \) is not a quasi-component of \( Y \). Since each quasi-component is a union of components, \( L_{1} \cup L_{2} \) is a quasi-component.
		\item \( \left\{ x_{1} \right\} \times Y \) and \( Y \) are homeomorphic, \( X \) and \( X \times \left\{ y_{2} \right\} \) are homeomorphic.

		      Let \( A, B \) be a clopen decomposition of \( X \times Y \) then \( A \cap (\left\{ x_{1} \right\} \times Y) \) and \( B \cap (\left\{ x_{1} \right\} \times Y) \) is a clopen decomposition of \( \left\{ x_{1} \right\} \times Y \).

		      As \( y_{1}, y_{2} \) are in the same quasi-component of \( Y \), \( (x_{1}, y_{1}), (x_{1}, y_{2}) \) are in the same quasi-component of \( \left\{ x_{1} \right\} \times Y \), which means \( (x_{1}, y_{1}), (x_{1}, y_{2}) \) are both in \( C_{1} \) or \( C_{2} \). Without loss of generality, suppose that they are both in \( C_{1} \).

		      Similarly, \( A \cap (X \times \left\{ y_{2} \right\}) \), \( B \cap (X \times \left\{ y_{2} \right\}) \) is a clopen decomposition of \( X \times \left\{ y_{2} \right\} \). Two points \( x_{1}, x_{2} \) are in the same quasi-component of \( X \), so \( (x_{1}, y_{2}) \) and \( (x_{2}, y_{2}) \) are in the same quasi-component of \( X \times \left\{ y_{2} \right\} \), which means \( (x_{1}, y_{2}) \) and \( (x_{2}, y_{2}) \) are both in \( C_{1} \) or \( C_{2} \). Since \( (x_{1}, y_{2}) \in C_{1} \), \( (x_{2}, y_{2}) \in C_{2} \).

		      Hence whenever \( (x_{1}, y_{1}) \in H \) for some clopen set \( H \) in \( X \times Y \), one also has \( (x_{2}, y_{2}) \in H \). According to part (c), \( (x_{1}, y_{1}) \) and \( (x_{2}, y_{2}) \) are in the same quasi-component of \( X \times Y \).
	\end{enumerate}
\end{proof}

\section{Local Connectedness}

\begin{problem}{V.4.1}
Prove that an open set in \( E^{n} \) can have at most countably many components. Give an example to show that this is no longer true for closed sets.
\end{problem}

\begin{proof}
	\( E^{n} \) has a countable basis \( \mathscr{B} \). Let \( U \) be an open set in \( E^{n} \). Assume that \( U \) has uncountably many components \( C_{i}(x_{i}) \) for \( i \in I \).

	\( E^{n} \) is locally connected so each component of the open set \( U \) is an open set. Therefore \( C_{i}(x_{i}) \) contains a basis element \( B_{i} \in \mathscr{B} \). Since the components \( C_{i}(x_{i}) \) of \( U \) are disjoint and there are uncountably many of them, then \( \left\{ B_{i} \right\} \) is uncountable, which contradicts the countability of \( \mathscr{B} \).

	Thus every open set in \( E^{n} \) can have at most countably many components.

	\bigskip

	In \( E^{1} \), the Cantor ternary set \( C \) is uncountable, closed, and totally disconnected, hence it has uncountably many components.

	In \( E^{n} \), the set \( C \times \underbrace{\left\{ 0 \right\} \times \cdots \times \left\{ 0 \right\}}_{n-1} \) is uncountable, closed, and totally disconnected, so it has uncountably many components.
\end{proof}

\begin{problem}{V.4.2}
Let \( Y \) be locally connected and let \( U \) be a component of the open \( G \subset Y \). Show \( G \cap \operatorname{Fr}(U) = \varnothing \).
\end{problem}

\begin{proof}
	\( Y \) is locally connected and \( G \subset Y \) is open so the component \( U \) of \( G \) is an open set in \( Y \). Since \( U \) is a component of \( G \), \( \operatorname{cl}_{G}(U) = U \).
	\begingroup
	\allowdisplaybreaks%
	\begin{align*}
		G \cap \operatorname{Fr}(U) & = G \cap \operatorname{cl}_{Y}(U) \cap \operatorname{cl}_{Y}(Y - U)        \\
		                            & = G \cap \operatorname{cl}_{Y}(U) \cap G \cap \operatorname{cl}_{Y}(Y - U) \\
		                            & = \operatorname{cl}_{G}(U) \cap G \cap (Y - U)                             \\
		                            & = U \cap G \cap (Y - U)                                                    \\
		                            & = U \cap (Y - U)                                                           \\
		                            & = \varnothing.
	\end{align*}
	\endgroup
\end{proof}

\begin{problem}{V.4.3}
Let \( Y \) be locally connected and \( A \subset Y \) arbitrary. Let \( C \) be a component of \( A \). Prove:
\begin{enumerate}[label={(\alph*)}]
	\item \(\operatorname{Int}(C) = C \cap \operatorname{Int}(A)\),
	\item \(\operatorname{Fr}(C) \subset \operatorname{Fr}(A)\),
	\item If \( A \) is closed, then \(\operatorname{Fr}(C) = C \cap \operatorname{Fr}(A)\).
\end{enumerate}
\end{problem}

\begin{proof}
	\begin{enumerate}[label={(\alph*)}]
		\item By the definition of interiors, \( \operatorname{Int}(C) \subset C \). Besides, \( \operatorname{Int}(C) \subset C \subset A \) and \( \operatorname{Int}(A) \) is the largest open set contained in \( A \), it follows that \( \operatorname{Int}(C) \subset \operatorname{Int}(A) \). Therefore \( \operatorname{Int}(C) \subset C \cap \operatorname{Int}(A) \). So far, we haven't used the local connectedness of \( Y \).

		      For each \( y \in C \cap \operatorname{Int}(A) \), there is a (connected) basic open set \( U \) such that \( y \in U \subset \operatorname{Int}(A) \) because \( Y \) is locally connected. Since \( U \) is connected and \( y \in U \subset \operatorname{Int}(A) \subset A \), it follows that \( U \subset C \) as \( C \) is the component of \( y \) in \( A \). Moreover, \( U \) is open so \( y \in U \subset \operatorname{Int}(C) \). Hence \( C \cap \operatorname{Int}(A) \subset \operatorname{Int}(C) \).

		      Thus \( \operatorname{Int}(C) = C \cap \operatorname{Int}(A) \).
		\item Let \( y \in \operatorname{Fr}(C) \) then every neighborhood \( N \) of \( y \) intersects \( C \) and \( Y - C \). Since \( Y \) is locally connected, there is a connected basic open set \( B \) such that \( y \in B \subset N \). As \( y \) is a boundary point of \( C \), the connected open set \( B \) intersects \( C \) and \( Y - C \). If \( B \) is contained in \( A \) then it contradicts the maximality of the component \( C \) in \( A \) because \( C \cup B \) is connected. Therefore \( B \) intersects \( Y - A \), which means \( N \) intersects \( A \) (because \(N\) intersects \(A\)) and \( Y - A \). Hence \( y \in \operatorname{Fr}(A) \).

		      Thus \( \operatorname{Fr}(C) \subset \operatorname{Fr}(A) \).
		\item Let \( y \in C \cap \operatorname{Fr}(A) \). Let \( N \) be a neighborhood of \( y \). As \( Y \) is locally connected, there is a connected open set \( B \) such that \( y \in B \subset N \). This connected open set \( B \) intersects both \( A \) and \( Y - A \), hence \( B \) intersects \( Y - C \). Hence \( N \) intersects \( C \) and \( Y - C \). Thus \( y \in \operatorname{Fr}(C) \) and \( C \cap \operatorname{Fr}(A) \subset \operatorname{Fr}(C) \). Note that we haven't used the closedness of \(A\) yet.

		      \( C \) is closed in \( A \) (because \(C\) is a component of \(A\)) and \( A \) is closed in \( Y \) so \( C \) is closed in \( Y \). Therefore \( \operatorname{Fr}(C) \subset C \), and together with part (b), we deduce that \( \operatorname{Fr}(C) \subset C \cap \operatorname{Fr}(A) \).

		      Thus \( \operatorname{Fr}(C) = C \cap \operatorname{Fr}(A) \).
	\end{enumerate}
\end{proof}

\begin{problem}{V.4.4}
Let \( Y \) be locally connected, and \( A \subset Y \) arbitrary. If \(\operatorname{Fr}(A)\) is locally connected, prove that \( \overline{A} \) is locally connected.
\end{problem}

\begin{quotation}
	In this problem and the next one, we use the following lemma: Let \( X \) be a topological space and \( A_{\alpha} \subset X \) are disjoint locally connected subspaces of \( X \). Then \( \bigcup_{\alpha\in\mathscr{A}} A_{\alpha} \) is locally connected (with the induced topology from \( X \)).

	\begin{proof}
		Let \( \mathscr{B}_{\alpha} \) be a basis for \( A_{\alpha} \) consisting of connected open sets in \( A_{\alpha} \). Define \( \mathscr{B} = \bigcup_{\alpha\in\mathscr{A}} \mathscr{B}_{\alpha} \).

		Let \( U \) be an open set in \( \bigcup_{\alpha\in\mathscr{A}} A_{\alpha} \) then \( U \cap A_{\alpha} \) is open in \( A_{\alpha} \) and can be written as a union of members of \( \mathscr{B}_{\alpha} \), for each \( \alpha \in \mathscr{A} \). Hence \( \mathscr{B} \) is a basis for \( \bigcup_{\alpha\in\mathscr{A}} A_{\alpha} \) and \( \bigcup_{\alpha\in\mathscr{A}} A_{\alpha} \) is locally connected.
	\end{proof}
\end{quotation}

\begin{proof}
	\( \operatorname{Int}(A) \) is open in \( Y \) and \( Y \) is locally connected so \( \operatorname{Int}(A) \) is locally connected.

	\( \overline{A} = \operatorname{Int}(A) \cup \operatorname{Fr}(A) \). Since \( \operatorname{Int}(A) \) and \( \operatorname{Fr}(A) \) are disjoint and locally connected, it follows from the above lemma that \( \overline{A} \) is locally connected.
\end{proof}

\begin{problem}{V.4.5}
Let \( Y \) be locally connected, \( Y = A \cup B \), where \( A, B \) are closed and \( A \cap B \) is locally connected. Prove that both \( A \) and \( B \) are locally connected.
\end{problem}

\begin{quotation}
	The following proof implicitly uses the transitivity of subspaces.
\end{quotation}

\begin{proof}
	\( A, B \) are closed so \( A \cap B \) is closed.
	\[
		Y - A = (A \cup B) - A = B - A = B - (A \cap B)
	\]

	is open in \( Y \). Because \( Y \) is locally connected, then the open set \( Y - A \) (with the induced topology) is locally connected.
	\[
		B = (B - (A \cap B)) \cup (A \cap B) = (Y - A) \cup (A \cap B)
	\]

	and \( Y - A, A \cap B \) are disjoint, locally connected, so \( B \) is locally connected, according to the above lemma. Similarly, \( A \) is locally connected.
\end{proof}

\begin{problem}{V.4.6}
Let \(Y\) be locally connected, \( A \subset Y \) arbitrary. Let \( S \subset A \) be connected and open in \(A\). Show \( S = A \cap U \), where \( U \) is connected and open in \( Y \).
\end{problem}

\begin{proof}
	If \( S \) is the emptyset, let \( U = \varnothing \) then \( S = A \cap U \).

	Assume that \( S \) is nonempty. As \( S \) is open in \( A \), there exists an open set \( U \) in \( Y \) such that \( S = A \cap U \). Because \( U \) is open in \( Y \) and \( Y \) is locally connected, \( U = \bigcup_{\alpha\in\mathscr{A}} B_{\alpha} \) in which each \( B_{\alpha} \) is a connected open set. Without loss of generality, we can assume that \( B_{\alpha} \cap S \ne \varnothing \) for every \( \alpha \in \mathscr{A} \).

	\( S \cup B_{\alpha} \) is connected for each \( \alpha \in \mathscr{A} \) because \( S \) and \( B_{\alpha} \) are connected and intersecting. Since \( \varnothing \ne S \subset \bigcap_{\alpha\in\mathscr{A}} (S \cup B_{\alpha}) \) then \( \bigcup_{\alpha\in\mathscr{A}} (S \cup B_{\alpha}) \) is connected. Moreover
	\[
		\bigcup_{\alpha\in\mathscr{A}} (S \cup B_{\alpha}) = S \cup \bigcup_{\alpha\in\mathscr{A}} B_{\alpha} = S \cup U = U
	\]

	which implies that \( U \) is connected. Thus \( S = A \cap U \), in which \( U \) is connected and open in \( Y \).
\end{proof}

\begin{problem}{V.4.7}
Let \( Y \) be locally connected, but not connected. Show that a decomposition of \( Y \) into two nonempty disjoint open sets can always be accomplished by taking any component as one of the sets and all the rest as the other set.
\end{problem}

\begin{proof}
	Let \( C \) be a component of \( Y \) then \( C \) is closed in \( Y \). Because \( Y \) is locally connected, \( C \) is open in \( Y \). Since \( Y \) is not connected, \( Y - C \) is nonempty. Hence \( Y = C \cup (Y - C) \) which means \( Y \) can be decomposed into the union of two disjoint open sets, in which one is a component and the rest is the complement.
\end{proof}

\section{Path-Connectedness}

\begin{problem}{V.5.1}
Show: If any one of the conditions in Proposition 5.4 holds, then the path components of \(Y\) coincide with the components of \(Y\).
\end{problem}

\begin{quotation}
	Proposition 5.4: The following two properties of a space \(Y\) are equivalent:
	\begin{enumerate}[label={(\arabic*)}]
		\item Each path component is open (and therefore also closed).
		\item Each point of \(Y\) has a path-connected neighborhood.
	\end{enumerate}
\end{quotation}

\begin{proof}
	If any one of the conditions holds, then each path component of \( Y \) is open and closed.

	Let \( P \) be a path component of \( Y \). Since \( P \) is open, closed, and connected (path-connectedness implies connected), we conclude that \( P \) is a component of \( Y \), according to Problem~\ref{problem:V.3.4}. Thus the path components of \(Y\) coincide with the components of \(Y\).
\end{proof}

\begin{problem}{V.5.2}
Let \( \left\{ Y_{\alpha} \mid \alpha \in \mathscr{A} \right\} \) be a family of spaces. Show \( \prod_{\alpha} Y_{\alpha} \) is path-connected if and only if each \( Y_{\alpha} \) is path-connected.
\end{problem}

\begin{proof}
	Suppose that \( \prod_{\alpha} Y_{\alpha} \) is path-connected. Since each canonical projection \( p_{\alpha}: \prod_{\alpha} Y_{\alpha} \to Y_{\alpha} \) is continuous, we conclude that each \( Y_{\alpha} \) is path-connected.

	Conversely, suppose that each \( Y_{\alpha} \) is path-connected. Let \( y, \hat{y} \) be a point of \( \prod_{\alpha} Y_{\alpha} \). For each \( \alpha \), there is a path \( f_{\alpha}: \closedinterval{0, 1} \to Y_{\alpha} \) connecting \( y_{\alpha} \) and \( \hat{y}_{\alpha} \). Define \( f: \closedinterval{0, 1} \to \prod_{\alpha} Y_{\alpha} \) by \( p_{\alpha}(f(t)) = f_{\alpha}(t) \) for each \( t \in \closedinterval{0,1}, \alpha \) then \( f \) is continuous as \( p_{\alpha} \circ f \) is continuous for each \( \alpha \). Therefore \( f \) is a path connecting \( y_{\alpha} \) and \( \hat{y}_{\alpha} \).
\end{proof}

\begin{problem}{V.5.3}
Let \(X\) be the connected set in Problem~\ref{problem:V.1.3} \textcolor{red}{and \( X \) is countable}. Show that \(X\) is totally pathwise disconnected.
\end{problem}

\begin{proof}
	This is a consequence of Sierpiński's theorem on partitions of the unit interval into closed sets.
\end{proof}
