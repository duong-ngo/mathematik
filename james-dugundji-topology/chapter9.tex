\chapter{Metric Spaces}

\section{Metric on Sets}

\begin{problem}{IX.1.1}
Let \( d: Y \times Y \to E^{1} \) satisfy: (1) \( d(x, y) = 0 \) if and only if \( x = y \), and (2) \( d(x, y) \le d(z, x) + d(z, y) \) for all \( x, y, z \). Prove: \( d \) is a metric.
\end{problem}

\begin{proof}
	From (2), it follows that
	\begingroup
	\allowdisplaybreaks%
	\begin{align*}
		d(x, y) & \le d(y, x) + d(y, y) = d(y, x) \\
		d(y, x) & \le d(x, y) + d(x, x) = d(x, y)
	\end{align*}
	\endgroup

	for every \( x, y \in Y \) so \( d(x, y) = d(y, x) \) for every \( x, y \in Y \). Moreover, for every \( x, y \in Y \)
	\[
		0 = d(y, y) \le d(x, x) + d(x, y) = d(x, y)
	\]

	and for every \( x, y, z \in Y \)
	\[
		d(x, y) \le d(z, x) + d(z, y) = d(x, z) + d(z, y)
	\]

	so \( d \) is positive-definite, symmetric, and satisfies the triangle inequality. Thus \( d \) is a metric.
\end{proof}

\begin{problem}{IX.1.2}
Let \( d_{i}, i = 1, 2, \ldots, n \) be \( n \) metrics in a set \( Y \). For any constants \( a_{i} \ge 0 \), not all zero, show that \( \sum^{n}_{1} a_{i} d_{i}(x, y) \) is a metric.
\end{problem}

\begin{proof}
	Let \( d(x, y) = \sum^{n}_{1} a_{i} d_{i}(x, y) \). As each \( d_{i}(x, y) \) is nonnegative for every \( x, y \in Y \), \( d(x, y) \) is nonnegative for every \( x, y \in Y \).

	If \( x = y \) then \( d(x, y) = \sum^{n}_{1} a_{i} d_{i}(x, y) = \sum^{n}_{1} a_{i}\cdot 0 = 0 \). Conversely, if \( d(x, y) = \sum^{n}_{1} a_{i} d_{i}(x, y) = 0 \) then \( a_{i} d_{i}(x, y) = 0 \) for every \( i \). Since \( a_{i} \) are not all zero, there exists \( i \) such that \( a_{i} > 0 \), so \( d_{i}(x, y) = 0 \), which implies \( x = y \). Hence \( x = y \iff d(x, y) = 0 \).

	Each \( d_{i} \) is symmetric so \( d \) is symmetric.

	For every \( x, y, z \in Y \)
	\begingroup
	\allowdisplaybreaks%
	\begin{align*}
		d(x, y) & = \sum^{n}_{1} a_{i}d_{i}(x, y)                                   \\
		        & \le \sum^{n}_{1} a_{i}(d_{i}(x, z) + d_{i}(z, y))                 \\
		        & = \sum^{n}_{1} a_{i} d_{i}(x, z) + \sum^{n}_{1} a_{i} d_{i}(z, y) \\
		        & = d(x, z) + d(z, y).
	\end{align*}
	\endgroup

	Thus \( d \) is a metric.
\end{proof}

\section{Topology Induced by a Metric}

\begin{problem}{IX.2.1}
Give an example to show that in a metric space \( (X, \mathscr{T}(d)) \), it is \textit{not} necessarily true that (a) \( \overline{B_{d}(a, r)} = \left\{ x \mid d(x, a) \le r \right\} \), and (b) \( \operatorname{Fr}[B_{d}(a, r)] = \left\{ x \mid d(x, a) = r \right\} \).
\end{problem}

\begin{proof}
	Let \( (X, \mathscr{T}(d)) \) be a metric space with the discrete metric and \( X \) has at least two elements then
	\[
		\overline{B_{d}(a, 1)} = \overline{ \left\{ a \right\} } = \left\{ a \right\} \ne X = \left\{ x \mid d(x, a) \le 1 \right\}
	\]

	and
	\[
		\operatorname{Fr}[B_{d}(a, 1)] = \varnothing \ne X - \left\{ a \right\} = \left\{ x \mid d(x, a) = 1 \right\}. \qedhere
	\]
\end{proof}

\begin{problem}{IX.2.2}
Prove that a pseudometric \( d \) in \( Y \) induces a topology that has \( \left\{ B_{d}(y, r) \mid y \in Y, r > 0 \right\} \) as basis.
\end{problem}

\begin{proof}
	\( \left\{ B_{d}(y, r) \mid y \in Y, r > 0 \right\} \) is a covering of \( Y \).

	Assume that \( y \in B_{d}(y_{1}, r_{1}) \cap B_{d}(y_{2}, r_{2}) \). Define \( r = \min\left\{ r_{1} - d(y, y_{1}), r_{2} - d(y, y_{2}) \right\} \) then for every \( x \in B_{d}(y, r) \)
	\[
		\begin{split}
			d(x, y_{1}) \le d(x, y) + d(y, y_{1}) < r + d(y, y_{1}) \le r_{1} \\
			d(x, y_{2}) \le d(x, y) + d(y, y_{2}) < r + d(y, y_{2}) \le r_{2}
		\end{split}
	\]

	which means \( B_{d}(y, r) \subset B_{d}(y_{1}, r_{1}) \cap B_{d}(y_{2}, r_{2}) \).

	Hence \( \left\{ B_{d}(y, r) \mid y \in Y, r > 0 \right\} \) is a basis for a topology on \( Y \).
\end{proof}

\section{Equivalent Metrics}

\begin{problem}{IX.3.1}
Let \( (Y, \mathscr{T}(d)) \) be a metric space. Show that
\[
	\rho(x, y) = \dfrac{d(x, y)}{1 + d(x, y)}
\]

is a metric in \( Y \) and that \( \rho \sim d \).
\end{problem}

\begin{proof}
	\( \rho(x, y) = 0 \iff d(x, y) = 0 \iff x = y \).

	\( \rho(x, y) = \dfrac{d(x, y)}{1 + d(x, y)} = \dfrac{d(y, x)}{1 + d(y, x)} = \rho(y, x) \).
	\begingroup
	\allowdisplaybreaks%
	\begin{align*}
		\rho(x, y) & = \dfrac{d(x, y)}{1 + d(x, y)}                                                    \\
		           & \le \dfrac{d(x, z) + d(z, y)}{1 + d(x, z) + d(z, y)}                              \\
		           & = \dfrac{d(x, z)}{1 + d(x, z) + d(z, y)} + \dfrac{d(z, y)}{1 + d(x, z) + d(z, y)} \\
		           & \le \dfrac{d(x, z)}{1 + d(x, z)} + \dfrac{d(z, y)}{1 + d(z, y)}                   \\
		           & = \rho(x, z) + \rho(z, y)
	\end{align*}
	\endgroup

	Hence \( \rho \) is a metric in \( Y \).
	\bigskip

	Let \( \varepsilon > 0 \) and \( a \in Y \) then
	\begin{itemize}
		\item \( B_{\rho}\left(a, \dfrac{\varepsilon}{1 + \varepsilon}\right) \subset B_{d}(a, \varepsilon) \).
		\item If \( \varepsilon \ge 1 \) then \( B_{d}(a, 1) \subset B_{\rho}(a, \varepsilon) \).
		\item If \( 0 < \varepsilon < 1 \) then \( B_{d}\left( a, \dfrac{\varepsilon}{1 - \varepsilon} \right) \subset B_{\rho}(a, \varepsilon) \).
	\end{itemize}

	Therefore \( \rho \sim d \).
\end{proof}

\begin{problem}{IX.3.2}
Let \( Y_{1}, \ldots, Y_{n} \) be metric spaces, and \( d_{1}, \ldots, d_{n} \) be metrics for these spaces. Show that
\[
	d[(x_{1}, \ldots, x_{n}), (y_{1}, \ldots, y_{n})] = \max\left\{ d_{i}(x_{i}, y_{i}) \mid i = 1, \ldots, n \right\}
\]

is a metric for the space \( \prod^{n}_{1} Y_{i} \).
\end{problem}

\begin{proof}
	\begingroup
	\allowdisplaybreaks%
	\begin{align*}
		d[(x_{1}, \ldots, x_{n}), (y_{1}, \ldots, y_{n})] = 0 & \iff d_{i}(x_{i}, y_{i}) = 0 \quad \forall i = 1, \ldots, n \\
		                                                      & \iff (x_{1}, \ldots, x_{n}) = (y_{1}, \ldots, y_{n}).
	\end{align*}
	\endgroup

	\begingroup
	\allowdisplaybreaks%
	\begin{align*}
		d[(x_{1}, \ldots, x_{n}), (y_{1}, \ldots, y_{n})] & = \max\left\{ d_{i}(x_{i}, y_{i}) \mid i = 1, \ldots, n \right\} \\
		                                                  & = \max\left\{ d_{i}(y_{i}, x_{i}) \mid i = 1, \ldots, n \right\} \\
		                                                  & = d[(y_{1}, \ldots, y_{n}), (x_{1}, \ldots, x_{n})].
	\end{align*}
	\endgroup

	For every \( i = 1, \ldots, n \)
	\[
		d[(x_{1}, \ldots, x_{n}), (z_{1}, \ldots, z_{n})] + d[(z_{1}, \ldots, z_{n}), (y_{1}, \ldots, y_{n})] \ge d_{i}(x_{i}, z_{i}) + d_{i}(z_{i}, y_{i}) \ge d_{i}(x_{i}, y_{i})
	\]

	so
	\[
		d[(x_{1}, \ldots, x_{n}), (z_{1}, \ldots, z_{n})] + d[(z_{1}, \ldots, z_{n}), (y_{1}, \ldots, y_{n})] \ge d[(x_{1}, \ldots, x_{n}), (y_{1}, \ldots, y_{n})].
	\]

	Thus \( d \) is a metric for the space \( \prod^{n}_{1} Y_{i} \).
\end{proof}

\begin{problem}{IX.3.3}
Let \( X = \openinterval{0, 1} \subset E^{1} \). Show:
\begin{enumerate}[label={(\alph*)}]
	\item \( d(x, y) = \left\vert x^{-1} - y^{-1} \right\vert \) is a metric on \( X \).
	\item \( d \) is equivalent to the usual metric \( d_{0}(x, y) = \left\vert x - y \right\vert \) on \( X \).
	\item There exists no metric on \( E^{1} \) that coincides with metric \( d \) on \( X \).
\end{enumerate}
\end{problem}

\begin{proof}
	\begin{enumerate}[label={(\alph*)}]
		\item \( d(x, y) = 0 \iff \left\vert x^{-1} - y^{-1} \right\vert = 0 \iff x = y \).

		      \( d(x, y) = \left\vert x^{-1} - y^{-1} \right\vert = \left\vert y^{-1} - x^{-1} \right\vert = d(y, x) \).

		      \( d(x, y) =  \left\vert x^{-1} - y^{-1} \right\vert \le \left\vert x^{-1} - z^{-1} \right\vert + \left\vert z^{-1} - y^{-1} \right\vert = d(x, z) + d(z, y) \).

		      Therefore \( d \) is a metric on \( X \).
		\item For every \( \varepsilon > 0 \) and \( a \in X \)
		      \begin{itemize}
			      \item \( B_{d}(a, \varepsilon) \subset B_{d_{0}}(a, \varepsilon) \)
			      \item \( B_{d_{0}}\left(a, \dfrac{\varepsilon a^{2}}{\varepsilon a + 1}\right) \subset B_{d}(a, \varepsilon) \)
		      \end{itemize}

		      Therefore \( d \sim d_{0} \).
		\item % TODO
	\end{enumerate}
\end{proof}

\section{Continuity of the Distance}

\section{Properties of Metric Topologies}

\section{Maps of Metric Spaces into Affine Spaces}

\section{Cartesian Products of Metric Spaces}

\section{The Space \( \ell^{2}(\mathscr{A}) \); Hilbert Cube}

\section{Metrization of Topological Spaces}

\section{Gauge Spaces}

\section{Uniform Spaces}
