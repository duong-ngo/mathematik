\chapter{Identification Topology; Weak Topology}

\section{Identification Topology}

\begin{problem}{VI.1.1}\label{problem:VI.1.1}
Reversing the situation treated in the text, let \(X\) be a set, \( (Y, \mathscr{T}) \) a space, and \( p: X \to Y \) a surjective map. Prove:
\begin{enumerate}[label={(\alph*)}]
	\item \( \mathscr{T}_{X} = \left\{ p^{-1}(U) \mid U \text{ open in } Y \right\} \) is a topology in \( X \).
	\item \( p: (X, \mathscr{T}_{X}) \to (Y, \mathscr{T}) \) is continuous, open, and closed.
\end{enumerate}
\end{problem}

\begin{proof}
	\begin{enumerate}[label={(\alph*)}]
		\item \( \mathscr{T}_{X} \) contains \( \varnothing, X \) as \( p^{-1}(\varnothing) = \varnothing \) and \( p^{-1}(Y) = X \).

		      If \( {\left\{ U_{\alpha} \right\}}_{\alpha\in\mathscr{A}} \) is a collection of open sets in \( Y \), then
		      \[
			      \bigcup_{\alpha\in\mathscr{A}} p^{-1}(U_{\alpha}) = p^{-1}\left(\bigcup_{\alpha\in\mathscr{A}} U_{\alpha}\right)
		      \]

		      so \( \mathscr{T}_{X} \) is closed under arbitrary union.

		      If \( U_{1}, \ldots, U_{n} \) are open sets in \( Y \) then
		      \[
			      \bigcap^{n}_{i=1} p^{-1}(U_{i}) = p^{-1}\left(\bigcap^{n}_{i=1} U_{i}\right)
		      \]

		      so \( \mathscr{T}_{X} \) is closed under finite intersection.

		      Hence \( \mathscr{T}_{X} \) is a topology in \( X \).
		\item For each open set \( U \) in \( Y \), \( p^{-1}(U) \in \mathscr{T}_{X} \) so \( p \) is continuous.

		      Let \( V \) be an open set in \( X \). Then there is an open set \( U \) in \( Y \) such that \( V = p^{-1}(U) \). Hence \( p(V) = pp^{-1}(U) = U \) because \( p \) is surjective. So \( p \) is an open map.

		      Let \( W \) be a closed set in \( X \) then \( X - W \) is open and there exists an open set \( U \) in \( Y \) such that \( X - W = p^{-1}(U) \). Therefore
		      \[
			      W = X - p^{-1}(U) = p^{-1}(Y) - p^{-1}(U) = p^{-1}(Y - U)
		      \]

		      which implies that \( p(W) = pp^{-1}(Y - U) = Y - U \), which is closed in \( Y \). So \( p \) is a closed map.

		      Thus \( p \) is a continuous, open, and closed map.
	\end{enumerate}
\end{proof}

\begin{problem}{VI.1.2}\label{problem:VI.1.2}
For each \( \alpha \in \mathscr{A} \), let \( p_{\alpha}: X_{\alpha} \to Y_{\alpha} \) be a continuous, open surjection. Show that \( \prod_{\alpha} p_{\alpha}: \prod_{\alpha} X_{\alpha} \to \prod_{\alpha} Y_{\alpha} \) is an identification.
\end{problem}

\begin{proof}
	For the sake of brevity, denote \( p = \prod_{\alpha} p_{\alpha} \). By definition, \( p \) is surjective.

	\( p_{Y_{\alpha}} \circ p \) is continuous for each projection \( p_{Y_{\alpha}}: \prod_{\alpha} Y_{\alpha} \to Y_{\alpha} \) so \( p \) is continuous.

	Let \( \prod_{\alpha} U_{\alpha} \) be a basic open set in \( \prod_{\alpha} X_{\alpha} \), which means \( U_{\alpha} = X_{\alpha} \) for all but finitely many \( \alpha \) and \( U_{\alpha} \) is open in \( X_{\alpha} \) for every \( \alpha \). Because \( p_{\alpha} \) is an open surjection for each \( \alpha \), the image
	\[
		p\left( \prod_{\alpha} U_{\alpha} \right) = \prod_{\alpha} p_{\alpha}(U_{\alpha})
	\]

	is open in \( \prod_{\alpha} Y_{\alpha} \) as \( p_{\alpha}(U_{\alpha}) \) is open in \( Y_{\alpha} \) and \( p_{\alpha}(U_{\alpha}) = Y_{\alpha} \) for all but finitely many \( \alpha \). Hence \( p \) is an open map.

	\( p \) is a continuous, open surjection so \( p \) is an identification.
\end{proof}

\begin{problem}{VI.1.3}\label{problem:VI.1.3}
Let \( X \) be a space and \( A \subset X \) a subspace. Assume that there exists a continuous \( r: X \to A \) such that \( r\vert_{A} = 1_{A} \) (such a map is called a \textit{retraction} of \(X\) onto \(A\)). Show that \( r \) is an identification.
\end{problem}

\begin{proof}
	By definition, \( r \) is continuous and surjective. Let \( f: A \xhookrightarrow{} X \) be the inclusion map.

	\( f \) is continuous and \( r \circ f = 1_{A} \) so \( r \) is an identification.
\end{proof}

\begin{problem}{VI.1.4}\label{problem:VI.1.4}
Let \( X \) be any set. Given any family \( \left\{ (Y_{\alpha}, \mathscr{T}_{\alpha}), f_{\alpha} \mid \alpha \in \mathscr{A} \right\} \) of spaces and maps \( f_{\alpha}: X \to Y_{\alpha} \), the ``projective limit topology of \(X\) determined by this family'' is \( \bigvee_{\alpha} f_{\alpha}^{-1}(\mathscr{T}_{\alpha}) \) (see Problem~\ref{problem:III.3.8}). Prove:
\begin{enumerate}[label={(\alph*)}]
	\item If \( j: X \to \prod_{\alpha} Y_{\alpha} \) is the map \( j(x) = \left\{ f_{\alpha}(x) \right\} \), then \( \bigvee_{\alpha} f_{\alpha}^{-1}(\mathscr{T}_{\alpha}) \) is the topology in \(X\) determined by \(j\) as in Problem~\ref{problem:VI.1.1}.
	\item If whenever \( x \ne x^{\prime} \), there is some index \( \alpha \) such that \( f_{\alpha}(x) \ne f_{\alpha}(x^{\prime}) \), then \( j \) is an embedding.
\end{enumerate}
\end{problem}

\begin{proof}
	\begin{enumerate}[label={(\alph*)}]
		\item Let \( \prod_{\alpha} U_{\alpha} \) be a subbasic open set in \( \prod_{\alpha} Y_{\alpha} \) then \( U_{\alpha} = Y_{\alpha} \) for every \( \alpha \) but one \( \beta \in \mathscr{A} \).
		      \[
			      j^{-1}\left( \prod_{\alpha} U_{\alpha} \right) = \bigcap_{\alpha} f_{\alpha}^{-1}(U_{\alpha}) = f_{\beta}^{-1}(U_{\beta}) \in \bigvee_{\alpha} f_{\alpha}^{-1}(\mathscr{T}_{\alpha})
		      \]

		      Hence \( j \) is continuous, which means if \( j^{-1}(U) \) is open whenever \( U \subset \prod_{\alpha} Y_{\alpha} \) is open.

		      Let \( V \) be an open set in \( X \). According to the definition of the topology \( \bigvee_{\alpha} f_{\alpha}^{-1}(\mathscr{T}_{\alpha}) \), \( V \) can be written as a union of finite intersection of elements in \( \bigcup_{\alpha} f_{\alpha}^{-1}(\mathscr{T}_{\alpha}) \), which means
		      \[
			      V = \bigcup_{i\in I} V_{i}
		      \]

		      where each \( V_{i} \) is a finite intersection of elements in \( \bigcup_{\alpha} f_{\alpha}^{-1}(\mathscr{T}_{\alpha}) \).
		      \[
			      V_{i} = \bigcap^{n_{i}}_{k=1} f_{\alpha_{k}}^{-1}(U_{\alpha_{k}}) = \bigcap^{n_{i}}_{k=1} j^{-1}\left( U_{\alpha_{k}} \times \prod_{\alpha \ne \alpha_{k}} Y_{\alpha} \right) = j^{-1}\left( \bigcap^{n_{i}}_{k=1} U_{\alpha_{k}} \times \prod_{\alpha \ne \alpha_{k}} Y_{\alpha} \right) = j^{-1}(W_{i})
		      \]

		      where \( U_{\alpha_{k}} \) is open in \( Y_{\alpha_{k}} \). So
		      \[
			      V = \bigcup_{i\in I} j^{-1}(W_{i}) = j^{-1}\left( \bigcup_{i\in I} W_{i} \right)
		      \]

		      which means \( V \) is the preimage of an open set in \( \prod_{\alpha} Y_{\alpha} \).

		      Thus \( \bigvee_{\alpha} f_{\alpha}^{-1}(\mathscr{T}_{\alpha}) \) is the same as the topology in \( X \) determined by \( j \) as in Problem~\ref{problem:VI.1.1}.
		\item According to Problem~\ref{problem:VI.1.1}, \( j \) is continuous, open, and closed.

		      Whenever \( x \ne x^{\prime} \), there is some index \( \alpha \) such that \( f_{\alpha}(x) \ne f_{\alpha}(x^{\prime}) \), then \( j(x) \ne j(x^{\prime}) \), which implies \( j \) is injective.

		      A continuous, open, injective map is an embedding so \( j \) is an embedding.
	\end{enumerate}
\end{proof}

\section{Subspaces}

\begin{problem}{VI.2.1}
Let \(X\) have the projective limit topology (Problem~\ref{problem:VI.1.4}) determined by
\[
	\left\{ Y_{\alpha}, f_{\alpha} \mid \alpha \in \mathscr{A} \right\}
\]

and let \( A \subset X \). Prove: The subspace topology of \(A\) is the projective limit topology determined by the maps \( f_{\alpha}\vert_{A} \).
\end{problem}

\begin{proof}
	The projective limit topology on \( A \) determined by the maps \( f_{\alpha}\vert_{A} \) has subbasis
	\[
		\bigcup_{\alpha} {(f_{\alpha}\vert_{A})}^{-1}(\mathscr{T}_{\alpha})
	\]

	Let \( V \) be an open set in \( A \) (with the projective limit topology) then
	\[
		V = \bigcup_{i\in I} V_{i}
	\]

	in which each \( V_{i} \) is the intersection of finitely many elements of \( \bigcup_{\alpha} {(f_{\alpha}\vert_{A})}^{-1}(\mathscr{T}_{\alpha}) \). So there exist \( \alpha_{i_{1}}, \ldots, \alpha_{i_{n(i)}} \in \mathscr{A} \) such that
	\[
		V_{i} = \bigcap^{n(i)}_{k=1} {(f_{\alpha_{k}}\vert_{A})}^{-1}(U_{\alpha_{k}})
	\]

	Hence
	\begingroup
	\allowdisplaybreaks%
	\begin{align*}
		V_{i} & = \bigcap^{n(i)}_{k=1} (A \cap f_{\alpha_{k}}^{-1}(U_{\alpha_{k}}))                                                             \\
		      & = A \cap \bigcap^{n(i)}_{k=1} f_{\alpha_{k}}^{-1}(U_{\alpha_{k}})                                                               \\
		      & = A \cap \bigcap^{n(i)}_{k=1} j^{-1}\left( U_{\alpha_{k}} \times \prod_{\alpha \ne \alpha_{k}} Y_{\alpha} \right)               \\
		      & = A \cap j^{-1}\left( \bigcap^{n(i)}_{k=1} \left( U_{\alpha_{k}} \times \prod_{\alpha\ne\alpha_{k}} Y_{\alpha} \right) \right).
	\end{align*}
	\endgroup

	Therefore
	\begingroup
	\allowdisplaybreaks%
	\begin{align*}
		V & = A \cap \bigcup_{i\in I} j^{-1}\left( \bigcap^{n(i)}_{k=1} \left( U_{\alpha_{k}} \times \prod_{\alpha\ne\alpha_{k}} Y_{\alpha} \right) \right) \\
		  & = A \cap j^{-1}\left( \bigcup_{i\in J} \bigcap^{n(i)}_{k=1} \left( U_{\alpha_{k}} \times \prod_{\alpha\ne\alpha_{k}} Y_{\alpha} \right) \right)
	\end{align*}
	\endgroup

	Hence \( V \) is in the subspace topology of \( A \).

	Conversely, one can show that if \( V \) is in the subspace topology of \( A \), then \( V \) is also in the projective limit topology on \( A \) determinded by the maps \( f_{\alpha}\vert_{A} \).

	Thus the projective limit topology on \( A \) determinded by the maps \( f_{\alpha}\vert_{A} \) and the subspace topology on \( A \) coincide.
\end{proof}

\section{General Theorems}

\begin{problem}{VI.3.1}
Let \( p: X \to Y \) be a continuous open (or closed) surjection, and assume that each fiber \( p^{-1}(y) \) is connected. For any \( F \subset Y \), show that \( F \) is connected if and only if \( p^{-1}(F) \) is connected.
\end{problem}

\begin{proof}
	By Proposition 2.1, \( p\vert_{p^{-1}(F)}: p^{-1}(F) \to F \) is an identification because \( p \) is an identification which is also an open (or closed) map. Denote \( q = p\vert_{p^{-1}(F)} \).

	If \( p^{-1}(F) \) is connected then \( F = p(p^{-1}(F)) \) is connected, as \( p \) is a continuous surjection.

	If \( p^{-1}(F) \) is not connected then there is a continuous surjection \( h: p^{-1}(F) \to 2 \). As each fiber of \( q \) (each fiber of \(q \) is a fiber of \(p\)) is connected, the restriction of \( h \) to each fiber is a constant map. Therefore \( hq^{-1}: F \to 2 \) is a continuous surjection, according to the transgression property, which means \( F \) is not connected.
\end{proof}

\begin{problem}{VI.3.2}
Let \( X \) have the projective limit topology \( \mathscr{T} \) determined by the family
\[
	\left\{ (Y_{\alpha}, \mathscr{T}_{\alpha}), f_{\alpha} \mid \alpha \in \mathscr{A} \right\}
\]

Assume that each \( \mathscr{T}_{\alpha} \) is the projective limit topology determined by a family
\[
	\left\{ (Z_{\alpha, \beta}, \mathscr{T}_{\alpha,\beta}), g_{\alpha,\beta} \mid \beta \in \mathscr{B} \right\}.
\]

Prove: \( \mathscr{T} \) is the projective limit topology determined by
\[
	\left\{ (Z_{\alpha,\beta}, \mathscr{T}_{\alpha,\beta}), g_{\alpha,\beta} \circ f_{\alpha} \mid (\alpha, \beta) \in \mathscr{A} \times \mathscr{B} \right\}.
\]
\end{problem}

\begin{proof}
	Denote by \( \widetilde{\mathscr{T}} \) the projective limit topology determined by
	\[
		\left\{ (Z_{\alpha,\beta}, \mathscr{T}_{\alpha,\beta}), g_{\alpha,\beta} \circ f_{\alpha} \mid (\alpha, \beta) \in \mathscr{A} \times \mathscr{B} \right\}.
	\]

	Let \( h: X \to \prod_{\alpha} Y_{\alpha} \) be the map \( h(x) = {\left\{ f_{\alpha}(x) \right\}}_{\alpha} \) then
	\[
		\mathscr{T} = \left\{ h^{-1}(U) \mid U \text{ open in } \prod_{\alpha}Y_{\alpha} \right\}
	\]

	according to Problem~\ref{problem:VI.1.4}.

	For each \( \alpha \), let \( h_{\alpha}: Y_{\alpha} \to \prod_{\beta} Z_{\alpha,\beta} \) be the map \( h_{\alpha}(x) = {\left\{ g_{\alpha,\beta}(x) \right\}}_{\beta} \) then
	\[
		\mathscr{T}_{\alpha} = \left\{ h_{\alpha}^{-1}(U) \mid U \text{ open in } \prod_{\beta} Z_{\alpha,\beta} \right\}
	\]

	according to Problem~\ref{problem:VI.1.4}.

	Let \( \ell: (X, \widetilde{\mathscr{T}}) \to \prod_{\alpha,\beta} Z_{\alpha,\beta} \) be the map \( \ell(x) = {\left\{ g_{\alpha,\beta}(f_{\alpha}(x)) \right\}}_{\alpha,\beta} \) then
	\[
		\widetilde{\mathscr{T}} = \left\{ \ell^{-1}(U) \mid U \text{ open in } \prod_{\alpha,\beta} Z_{\alpha,\beta} \right\}
	\]

	according to Problem~\ref{problem:VI.1.4}.

	Note that \( f_{\alpha} = p_{\alpha} \circ h \) and \( g_{\alpha,\beta} = p_{\alpha,\beta} \circ h_{\alpha} \) in which \( p_{\alpha}: \prod_{\alpha} Y_{\alpha} \to Y_{\alpha} \) and \( p_{\alpha,\beta}: \prod_{\beta} Z_{\alpha,\beta} \to Z_{\alpha,\beta} \) are projection maps. Denote by \( q_{\alpha,\beta} \) the projection map \( \prod_{\alpha,\beta} W_{\alpha,\beta} \to W_{\alpha,\beta} \).
	\[
		\begin{tikzcd}
			&& {\prod_{\alpha} Y_{\alpha}} \\
			\\
			X && {Y_{\alpha}} && {\prod_{\beta}Z_{\alpha,\beta}} && {Z_{\alpha,\beta}}
			\arrow["{p_{\alpha}}", from=1-3, to=3-3]
			\arrow["h", from=3-1, to=1-3]
			\arrow["{f_{\alpha}}"', from=3-1, to=3-3]
			\arrow["{h_{\alpha}}"', from=3-3, to=3-5]
			\arrow["{g_{\alpha,\beta}}"', bend right, from=3-3, to=3-7]
			\arrow["{p_{\alpha,\beta}}"', from=3-5, to=3-7]
		\end{tikzcd}
	\]

	\[
		\begin{tikzcd}
			X && {\prod_{\alpha,\beta} Z_{\alpha,\beta}} && {Z_{\alpha,\beta}}
			\arrow["\ell", from=1-1, to=1-3]
			\arrow["{g_{\alpha,\beta} \circ f_{\alpha}}"', bend right, from=1-1, to=1-5]
			\arrow["{q_{\alpha,\beta}}", from=1-3, to=1-5]
		\end{tikzcd}
	\]

	Let \( U \in \mathscr{T} \) then there exists \( V \) open in \( \prod_{\alpha} Y_{\alpha} \) such that \( U = h^{-1}(V) \) (see Problem~\ref{problem:VI.1.4} and~\ref{problem:VI.1.1}). One can write \( V \) in terms of subbasic elements as follows
	\[
		V = \bigcup_{i\in I} \bigcap^{n(i)}_{k=1} p_{\alpha_{k}}^{-1}(V_{\alpha_{k}})
	\]

	in which \( V_{\alpha_{k}} \) is open in \( Y_{\alpha_{k}} \).

	As \( \mathscr{T}_{\alpha} \) is the projective limit topology on \( Y_{\alpha} \) determined by the maps \( g_{\alpha,\beta}: Y_{\alpha} \to Z_{\alpha,\beta} \), there is an open set \( W_{\alpha_{k}} \) in \( \prod_{\beta} Z_{\alpha_{k},\beta} \) such that \( V_{\alpha_{k}} = h_{\alpha}^{-1}(W_{\alpha_{k}}) \). The open set \( W_{\alpha_{k}} \) can be written in terms of subbasic elements as follows
	\[
		W_{\alpha_{k}} = \bigcup_{j \in J} \bigcap^{n(j)}_{r=1} p_{\alpha_{k},\beta_{r}}^{-1}(W_{\alpha_{k}, \beta_{r}})
	\]

	in which \( W_{\alpha_{k}, \beta_{r}} \) is open in \( Z_{\alpha_{k}, \beta_{r}} \).
	\begingroup
	\allowdisplaybreaks%
	\begin{align*}
		V             & = \bigcup_{i\in I} \bigcap^{n(i)}_{k=1} p_{\alpha_{k}}^{-1}(V_{\alpha_{k}})                                                                                                                         \\
		              & = \bigcup_{i\in I} \bigcap^{n(i)}_{k=1} p_{\alpha_{k}}^{-1}\left( h^{-1}_{\alpha_{k}}(W_{\alpha_{k}}) \right)                                                                                       \\
		              & = \bigcup_{i\in I} \bigcap^{n(i)}_{k=1} {(h_{\alpha_{k}} \circ p_{\alpha_{k}})}^{-1}(W_{\alpha_{k}})                                                                                                \\
		              & = \bigcup_{i\in I} \bigcap^{n(i)}_{k=1} {(h_{\alpha_{k}} \circ p_{\alpha_{k}})}^{-1} \left( \bigcup_{j \in J} \bigcap^{n(j)}_{r=1} p_{\alpha_{k},\beta_{r}}^{-1}(W_{\alpha_{k}, \beta_{r}}) \right) \\
		              & = \bigcup_{i\in I} \bigcap^{n(i)}_{k=1} \bigcup_{j\in J} \bigcap^{n(j)}_{r=1} {(p_{\alpha_{k},\beta_{r}} \circ h_{\alpha_{k}} \circ p_{\alpha_{k}})}^{-1}(W_{\alpha_{k},\beta_{r}})                 \\
		U = h^{-1}(V) & = \bigcup_{i\in I} \bigcap^{n(i)}_{k=1} \bigcup_{j\in J} \bigcap^{n(j)}_{r=1} {(p_{\alpha_{k},\beta_{r}} \circ h_{\alpha_{k}} \circ p_{\alpha_{k}} \circ h)}^{-1}(W_{\alpha_{k},\beta_{r}})         \\
		              & = \bigcup_{i\in I} \bigcap^{n(i)}_{k=1} \bigcup_{j\in J} \bigcap^{n(j)}_{r=1} {(g_{\alpha_{k},\beta_{r}} \circ f_{\alpha_{k}})}^{-1}(W_{\alpha_{k},\beta_{r}})                                      \\
		              & = \bigcup_{i\in I} \bigcap^{n(i)}_{k=1} \bigcup_{j\in J} \bigcap^{n(j)}_{r=1} {(q_{\alpha_{k},\beta_{r}} \circ \ell)}^{-1}(W_{\alpha_{k},\beta_{r}})                                                \\
		              & = \bigcup_{i\in I} \bigcap^{n(i)}_{k=1} \bigcup_{j\in J} \bigcap^{n(j)}_{r=1} \ell^{-1}q_{\alpha_{k},\beta_{r}}^{-1}(W_{\alpha_{k},\beta_{r}})                                                      \\
		              & = \ell^{-1}\left( \bigcup_{i\in I} \bigcap^{n(i)}_{k=1} \bigcup_{j\in J} \bigcap^{n(j)}_{r=1} q_{\alpha_{k},\beta_{r}}^{-1}(W_{\alpha_{k},\beta_{r}}) \right) \in \widetilde{\mathscr{T}}
	\end{align*}
	\endgroup

	Hence \( U \in \widetilde{\mathscr{T}} \), which means \( \mathscr{T} \subset \widetilde{\mathscr{T}} \).

	\bigskip
	Conversely, let \( U \in \widetilde{\mathscr{T}} \) then there exists \( W \) open in \( \prod_{\alpha,\beta} Z_{\alpha,\beta} \) such that \( U = \ell^{-1}(W) \).

	\( W \) can be written in terms of subbasic elements.
	\begingroup
	\allowdisplaybreaks%
	\begin{align*}
		U & = \ell^{-1}(W) = \ell^{-1}\left( \bigcup_{i\in I}\bigcap^{n(i)}_{r=1} q^{-1}_{\alpha_{r},\beta_{r}}(W_{\alpha_{r},\beta_{r}}) \right) \\
		  & = \bigcup_{i\in I}\bigcap^{n(i)}_{r=1} \ell^{-1}q^{-1}_{\alpha_{r},\beta_{r}}(W_{\alpha_{r},\beta_{r}})                               \\
		  & = \bigcup_{i\in I}\bigcap^{n(i)}_{r=1} {(q_{\alpha_{r},\beta_{r}}\circ \ell)}^{-1}(W_{\alpha_{r},\beta_{r}})                          \\
		  & = \bigcup_{i\in I}\bigcap^{n(i)}_{r=1} {(g_{\alpha_{r},\beta_{r}}\circ f_{\alpha_{r}})}^{-1}(W_{\alpha_{r},\beta_{r}})                \\
		  & = \bigcup_{i\in I}\bigcap^{n(i)}_{r=1} f_{\alpha_{r}}^{-1}(g_{\alpha_{r},\beta_{r}}^{-1}(W_{\alpha_{r},\beta_{r}})) \in \mathscr{T}
	\end{align*}
	\endgroup

	so \( \widetilde{\mathscr{T}} \subset \mathscr{T} \).

	Thus \( \mathscr{T} = \widetilde{\mathscr{T}} \).
\end{proof}

\begin{problem}{VI.3.3}
Let \(X\) have the projective limit topology determined by \( \left\{ Y_{\alpha}, f_{\alpha} \mid \alpha \in \mathscr{A} \right\} \). Prove: \( f: Z \to X \) is continuous if and only if each \( f_{\alpha} \circ f \) is continuous.
\end{problem}

\begin{proof}
	For each \( \alpha \), the map \( f_{\alpha}: X \to Y_{\alpha} \) is continuous.

	If \( f \) is continuous then each \( f_{\alpha} \circ f \) is continuous.

	Conversely, assume that each \( f_{\alpha} \circ f \) is continuous. Let \( U \) be an open set in \( X \).
	\[
		\bigcup_{\alpha} f_{\alpha}^{-1}(\mathscr{T}_{\alpha})
	\]

	is a subbasis for the projective limit topology on \( X \). Therefore \( U \) can be written as
	\[
		U = \bigcup_{\gamma} \bigcap^{n(\gamma)}_{k=1} f_{\gamma,k}^{-1}(U_{\gamma,k})
	\]

	in which \( U_{\gamma,k} \) is open in \( Y_{\gamma,k} \) so
	\begingroup
	\allowdisplaybreaks%
	\begin{align*}
		f^{-1}(U) & = f^{-1}\left( \bigcup_{\gamma} \bigcap^{n(\gamma)}_{k=1} f_{\gamma,k}^{-1}(U_{\gamma,k}) \right) \\
		          & = \bigcup_{\gamma} \bigcap^{n(\gamma)}_{k=1} f^{-1}(f_{\gamma,k}^{-1}(U_{\gamma,k}))              \\
		          & = \bigcup_{\gamma} \bigcap^{n(\gamma)}_{k=1} {(f_{\gamma,k} \circ f)}^{-1}(U_{\gamma,k})
	\end{align*}
	\endgroup

	\( {(f_{\gamma,k} \circ f)}^{-1}(U_{\gamma,k}) \) is open as \( U_{\gamma,k} \) is open in \( Y_{\gamma,k} \) and \( f_{\gamma,k} \circ f \) is continuous. Hence \( f^{-1}(U) \) is open (finite intersection then arbitrary union), so \( f \) is continuous.

	Thus \( f \) is continuous if and only if each \( f_{\alpha} \circ f \) is continuous.
\end{proof}

\section{Spaces with Equivalence Relations}

\begin{problem}{VI.4.1}
Let \( p: X \to X/R \) be an open (or closed) map, and \( B \subset X/R \) any subset. Show that \( B \) is homeomorphic to \( p^{-1}(B)/R_{0} \), where \( R_{0} \) is the restriction of \( R \) on \( p^{-1}(B) \).
\end{problem}

\begin{proof}
	Let \( q = p\vert_{p^{-1}(B)}: p^{-1}(B) \to B \) and \( r: p^{-1}(B) \to p^{-1}(B)/R_{0} \).

	\( p \) is an open (or closed) identification so \( q \) is an identification. Also, \( r \) is an identification.
	\[
		\begin{tikzcd}
			{p^{-1}(B)} && B \\
			\\
			{p^{-1}(B)/R_{0}}
			\arrow["q", from=1-1, to=1-3]
			\arrow["r"', from=1-1, to=3-1]
			\arrow["{rq^{-1}}", from=1-3, to=3-1]
			\arrow["{qr^{-1}}"{description}, shift left=3, from=3-1, to=1-3]
		\end{tikzcd}
	\]

	\( q \) is constant on each fiber of \( r \) and \( r \) is constant on each fiber of \( q \). Therefore the induced maps \( qr^{-1} \) and \( rq^{-1} \) are continuous. On the other hand, \( qr^{-1} \) and \( rq^{-1} \) are inverses of each other so they are homeomorphisms, which implies \( B \cong p^{-1}(B)/R_{0} \).
\end{proof}

\begin{problem}{VI.4.2}
Give an example showing that if \( A \subset X \) is not open or closed, then \( X - A \) need not be homeomorphic to the complement of \( [A] \) in \( X/A \).
\end{problem}

\begin{proof}
	Let \( X = \mathbb{R} \) and \( A = \mathbb{Q} \) then \( A \) is not open or closed in \( X \).

	The restriction of \( q: X \to X/A \) to \( q\vert_{X - A}: X - A \to X/A - \left\{ [A] \right\} \) is a continuous bijection. Let \( O = \openinterval{0, 1} \cap (\mathbb{R} - \mathbb{Q}) = \openinterval{0, 1} \cap (X - A) \) then \( O \) is open in \( X - A \). We will show that \( q(O) \) is not open in \( q(X - A) \).

	Assume that \( q(O) \) is open in \( q(X - A) \) then there exists an open set \( U \subset X/A \) such that \( q(O) = q(X - A) \cap U \).

	If \( U \) doesn't contain \( [A] \) then the preimage \( q^{-1}(U) \) doesn't contain any rational numbers, hence not open, which is a contradiction since \( U \) is open in \( X/A \) and \( q \) is continuous. Therefore \( U \) contains \( [A] \), so \( q^{-1}(U) \) contains \( \mathbb{Q} \), and
	\[
		q^{-1}(U) = q^{-1}(\left\{ [A] \right\} \cup q(O)) = q^{-1}(\left\{ [A] \right\}) \cup q^{-1}(q(O)) = \mathbb{Q} \cup O
	\]

	which is not open in \( X = \mathbb{R} \). This is a contradiction as \( q^{-1}(U) \) is simutaneously open and non-open.

	Thus \( q \) is not an open map, so it is not a homeomorphism.
\end{proof}

\begin{problem}{VI.4.3}
These problems will be much easier after studying compactness. (Use now Problem~\ref{problem:III.9.1}.)
\begin{enumerate}[label={(\alph*)}]
	\item In \( I^{2} \), let \( (0, y) \sim (1, y) \). Show \( I^{2}/R \cong \) the cylinder \( S^{1} \times I \).
	\item In \( I^{2} \), let \( (0, y) \sim (1, 1 - y) \). Show \( I^{2}/R \cong \) M\"{o}bius band.
	\item In \( I^{2} \), let \( \operatorname{Fr}(I^{2}) \sim (0, 0) \). Show \( I^{2}/R \cong S^{2} \).
	\item In \( I^{2} \), let \( (0, y) \sim (1, y), (x, 0) \sim (x, 1) \). Show \( I^{2}/R \cong \) the torus \( S^{1} \times S^{1} \).
\end{enumerate}
\end{problem}

\begin{proof}
	I have no idea other than using compactness and the closed map lemma.
\end{proof}

\begin{problem}{VI.4.4}
Let \( R \) be an equivalence relation in \( X \). For each \( A \subset X \) define \( C(A) = \left\{ x \in X \mid \exists a \in A: x R a \right\} \). Show that \( p(U) \) is open in \( X/R \) if and only if \( C(U) \) is open in \( X \).
\end{problem}

\begin{proof}
	According to the definition of \( C \), \( C(U) = p^{-1}(p(U)) \). Due to the definition of identification maps, \( p(U) \) is open in \( X/R \) if and only if \( p^{-1}(p(U)) \) is open in \( X \).

	Thus \( p(U) \) is open in \( X/R \) if and only if \( C(U) \) is open in \( X \).
\end{proof}

\begin{problem}{VI.4.5}
Let \( R, S \) be two equivalence relations in \( X \), and such that \( S \subset R \) (see Problem~\ref{problem:I.7.6}). Prove that \( (X/S)/(R/S) \cong X/R \).
\end{problem}

\begin{proof}
	Let \( p_{R}: X \to X/R \), \( p_{S}: X \to X/S \), and \( p: X/S \to (X/S)/(R/S) \) be projection maps.

	According to the transgression property, there is an induced continuous map \( q = p_{R}p_{S}^{-1} \) such that \( q \circ p_{S} = p_{R} \). According to Chapter I, section 7, there exists a unique map \( q \) that commutes the following diagram.
	\[
		\begin{tikzcd}
			X && X \\
			\\
			{X/S} && {X/R}
			\arrow["1", from=1-1, to=1-3]
			\arrow["{p_{S}}"', from=1-1, to=3-1]
			\arrow["{p_{R}}", from=1-3, to=3-3]
			\arrow["q"', from=3-1, to=3-3]
		\end{tikzcd}
	\]

	For each subset \( U \subset X/R \), \( p_{R}^{-1}(U) \) is \( p_{S} \)-saturated because \( S \subset R \). If \( U \) is open then \( p_{R}^{-1}(U) \) is open in \( X \) and \( p_{S} \)-saturated so \( p_{S}(p_{R}^{-1}(U)) \) is open in \( X/S \). Conversely, if \( q^{-1}(U) \) is open in \( X/S \) then \( p_{S}^{-1}(q^{-1}(U)) = {(q \circ p_{S})}^{-1}(U) = p_{R}^{-1}(U) \) is open in \( X \), which means \( U \) is open in \( X/R \). Hence \( q \) is an identification.

	According to the definition of \( R/S \) (see Problem~\ref{problem:I.7.5} and~\ref{problem:I.7.6})
	\[
		p(Sa) = p(Sb) \iff (Sa)R/S(Sb) \iff q(Sa) = q(Sb)
	\]

	which means \( q \) is constant on each fiber of \( p \) and \( p \) is constant on each fiber of \( q \)
	\[
		\begin{tikzcd}
			{X/S} && {X/R} \\
			\\
			{(X/S)/(R/S)}
			\arrow["q", from=1-1, to=1-3]
			\arrow["p"', from=1-1, to=3-1]
			\arrow["\cong"', from=3-1, to=1-3]
		\end{tikzcd}
	\]

	so \( (X/S)/(R/S) \cong X/R \).
\end{proof}

\begin{problem}{VI.4.6}
Let \( 0 \) be the origin in \( E^{3} \). In \( E^{3} - \left\{ 0 \right\} \), define \( x R y \) if \( x \) and \( y \) lie on a line through the origin. Show that \( R \) is an equivalence relation; \( (E^{3} - \left\{ 0 \right\})/R \) is called the projective plane \( P^{2} \). Call \textit{line} in \( P^{2} \) any set \( A \) such that \( p^{-1}(A) \) is a plane in \( E^{3} - \left\{ 0 \right\} \) going through the origin. Show that a line in \( P^{2} \) is homeomorphic to \( S^{1} \).
\end{problem}

\begin{proof}
	We will show that \( p: E^{3} - \left\{ 0 \right\} \to (E^{3} - \left\{ 0 \right\})/R \) an open map. Let \( U \) be an open set in \( E^{3} - \left\{ 0 \right\} \). The image \( p(U) \) is open in \( P^{2} \) if and only if \( p^{-1}(p(U)) \) is open in \( E^{3} - \left\{ 0 \right\} \).
	\[
		p^{-1}(p(U)) = \bigcup_{t \in E^{1} - \left\{ 0 \right\}} \left\{ tu \mid u \in U \right\}
	\]

	is open as \( \left\{ tu \mid u \in U \right\} \) is homeomorphic to \( U \) for each \( t \in E^{1} - \left\{ 0 \right\} \). Hence \( p \) is an open map.

	Let \( A \) be any line in \( P^{2} \) then \( p^{-1}(A) \) is a plane in \( E^{3} - \left\{ 0 \right\} \) going through the origin. Because \( p \) is an open map, \( p\vert_{p^{-1}(A)}: p^{-1}(A) \to A \) is an identification.

	Let's define \( f: E^{3} - \left\{ 0 \right\} \to E^{3} - \left\{ 0 \right\} \) by \( f(x) = x/\left\vert x \right\vert \) is continuous and open, so \( f: p^{-1}(A) \to f(p^{-1}(A)) \) is an identification. Moreover, \( f(p^{-1}(A)) \cong S^{1} \). The map \( g: S^{1} \to S^{1}/\left\{ x \sim -x \right\} \) is an open continuous surjection.

	The identifications \( g \circ (f\vert_{p^{-1}(A)}) \) and \( p\vert_{p^{-1}(A)} \) are constant on each fiber of the other map so \( A \) and \( S^{1}/\left\{ x \sim -x \right\} \) are homeomorphic.

	The homeomorphism of \( S^{1} \) and \( S^{1}/\left\{ x \sim -x \right\} \) follows from the map \( z \to z^{2} \).

	Thus \( A \cong S^{1} \).
\end{proof}

\begin{problem}{VI.4.7}
Let \( V^{2} = \left\{ x \in E^{2} \mid \left\vert x \right\vert \le 1 \right\} \). Generate an equivalence relation by \( x R y \) if \( \left\vert x \right\vert = \left\vert y \right\vert = 1 \), and \( x, y \) are diametrically opposite. Show \( V^{2}/R \) is homeomorphic to \( P^{2} \).
\end{problem}

\begin{proof}
	\( P^{2} \) is homeomorphic to \( S^{2}/\left\{ x \sim -x \right\} \).

	Let \( US^{2} \) be the upper hemisphere (\( x_{3} \ge 0 \)) then \( V^{2} \cong US^{2} \) and \( US^{2}/\left\{ x \sim -x \right\} \cong V^{2}/R \).

	Thus \( P^{2} \cong S^{2}/\left\{ x \sim -x \right\} \cong US^{2}/\left\{ x \sim -x \right\} \cong V^{2}/R \).
\end{proof}

\section{Cones and Suspensions}

\begin{quotation}
	The following two problems show that: The cone functor \( T \) takes closed embeddings to closed embeddings but doesn't necessarily take embeddings to embeddings.
\end{quotation}

\begin{problem}{VI.5.1}
If \( A \subset X \) is closed, prove that \( TA \) is homeomorphic to a closed subspace of \( TX \).
\end{problem}

\begin{proof}
	Denote \( q: X \times I \to TX \) the identification map that defines \( TX \) and \( p: A \times I \to TA \) the identification map that defines \( TA \).

	Consider the set \( q(A \times I) \).
	\begingroup
	\allowdisplaybreaks%
	\begin{align*}
		q^{-1}(q(A \times I)) & = q^{-1}(q(A \times \halfopenright{0, 1} \cup A \times 1))    \\
		                      & = q^{-1}(q(A \times \halfopenright{0, 1}) \cup q(A \times 1)) \\
		                      & = A \times \halfopenright{0, 1} \cup X \times 1               \\
		                      & = A \times I \cup X \times 1
	\end{align*}
	\endgroup

	is closed in \( X \times I \). Therefore \( q(A \times I) \) is closed in \( TX \).

	Let \( f: A \to X \) be the inclusion map then \( \bar{f}: A \times I \to X \times I \) is also an inclusion map. Pass to the quotient, we get a continuous injection \( Tf: TA \to TX \).

	Let \( C \subseteq TA \) be a closed subset then \( p^{-1}(C) \) is closed in \( A \times I \), hence \( p^{-1}(C) \) is closed in \( X \times I \).

	Note that \( p^{-1}(C) \subseteq_{\text{closed}} A \times I \subseteq_{\text{closed}} X \times I \) and \( q^{-1}(Tf(C)) = q^{-1}(q(p^{-1}(C))) \).

	If \( C \) doesn't contain any point of the form \( \left\langle x, 1 \right\rangle \) then
	\[
		q^{-1}(Tf(C)) = p^{-1}(C)
	\]

	which means \( Tf(C) \) is closed in \( TX \).

	If \( C \) contains some point of the form \( \left\langle x, 1 \right\rangle \) then
	\[
		q^{-1}(Tf(C)) = p^{-1}(C) \cup X \times 1
	\]

	this implies \( Tf(C) \) is closed in \( TX \).

	Moreover, \( Tf(C) \subseteq TA \) so \( Tf(C) \) is also closed in \( TA \).

	Hence \( Tf: TA \to TX \) is a closed embedding.
\end{proof}

\begin{problem}{VI.5.2}
Let \( i: \operatorname{Int}(I) \to I \) be the inclusion map. Show that the map \( Ti: T[\operatorname{Int}(I)] \to TI \) is not an embedding.
\end{problem}

\begin{proof}
	This idea is from \href{https://math.stackexchange.com/a/4879627}{Thorgott on Math StackExchange}.

	\( Ti \) is injective and continuous.

	I don't want to make confusion so I introduce new notations: \( X = I \) and \( U = \operatorname{Int}(I) \).

	\( p: X \times I \to TX \) and \( q: U \times I \to TU \) are identification maps that induce the cones.

	The set \( A = \left\{ (s, t): s < t, s \in U, t \in I \right\} \) is open in \( U \times I \) and \( q \)-saturated (because \( A \) contains \( U \times \left\{ 1 \right\} \)) so \( B = q(A) \) is open in \( TU \). We will show that \( Ti(B) \) is not open in \( Ti(TU) \).

	Assume that \( Ti(B) \) is open in \( Ti(TU) \) then there exists an open set \( W \subset TX \) such that \( Ti(B) = Ti(TU) \cap W \).
	\[
		q^{-1}(Ti(B)) = (U \times I) \cap q^{-1}(W) = (U \times I) \cap p^{-1}(W).
	\]

	\( p^{-1}(W) \) is open in \( X \times I \) and contains \( X \times \left\{ 1 \right\} \) so \( p^{-1}(W) \) contains \( U \times \halfopenleft{1 - \varepsilon, 1} \) for some \( \varepsilon \in \openinterval{0, 1} \). Therefore \( q^{-1}(Ti(B)) \) contains \( U \times \halfopenleft{1 - \varepsilon, 1} \) and \( A \) contains \( U \times \halfopenleft{1 - \varepsilon, 1} \), which contradicts the definition of \( A \).

	Thus \( Ti \) is not an embedding.
\end{proof}

\section{Attaching of Spaces}

\begin{example}{3}
	Attaching \( X \times I \times Y \) to the free union \( X + Y \) by \( (x, 0, y) \mapsto x, (x, 1, y) \mapsto y \) gives a space called the join \( X \ast Y \) of \( X \) and \( Y \). Prove that \( X \ast y_{0} \cong TX \) and \( X \ast S^{0} \cong SX \).
\end{example}

\begin{proof}
	\( TX = (X \times I)/(X \times \left\{ 1 \right\}) \) and let \( T: X \to TX \) be the identification map.

	Let \( J: X \times I \times y_{0} \to X \ast y_{0} \) be the identification map.
	\[\begin{tikzcd}
			{X\times I} && TX \\
			\\
			{X \times I \times \left\{ y_{0} \right\} } && {X \ast \left\{ y_{0} \right\}}
			\arrow["T", from=1-1, to=1-3]
			\arrow["f"', from=1-1, to=3-1]
			\arrow["{J \circ f}"', from=1-1, to=3-3]
			\arrow["J"', from=3-1, to=3-3]
		\end{tikzcd}\]

	\( f: X \times I \to X \times I \times y_{0} \) given by \( f(x, t) = (x, t, y_{0}) \) is a homeomorphism, so \( J \circ f \) is an identification map.
	\[
		T(x, t) = T(x^{\prime}, t^{\prime})\iff \left\lbrack\begin{array}{ll} x = x^{\prime}, t = t^{\prime} \ne 1 \\
			t = t^{\prime} = 1\end{array}\right. \iff J \circ f(x, t) = J \circ f(x^{\prime}, t^{\prime})
	\]

	so \( TX \cong X \ast y_{0} \).

	\bigskip

	Let \( S^{0} = \left\{ -1; 1 \right\} \) (the 1-sphere) then
	\begingroup
	\allowdisplaybreaks%
	\begin{align*}
		X \ast S^{0} & \cong (X \ast \left\{1\right\}) \cup_{f} (X \ast \left\{-1\right\}) \cong SX
	\end{align*}
	\endgroup

	where \( f: X \times \left\{0\right\} \to X \times \left\{0\right\} \) is the identity map.
\end{proof}

\begin{proposition}{6.2}
	Let \( p: X + Y \to X \cup_{f} Y \) be the projection, and let \( C \subset X + Y \) be such that \( C \cap X \) is closed in \( X \). Then \( p(C) \) is closed in \( X \cup_{f} Y \) if and only if \( (C \cap Y) \cup f(C \cap A) \) is closed in \( Y \).
\end{proposition}

\begin{proof}
	\( f: A \to Y \) is the continuous in which \( A \subset X \) is closed.

	Suppose that \( x \in p^{-1}(p(C)) \) then \( p(x) \in p(C) \). Either
	\begin{itemize}
		\item \( x \in X \).

		      If this is the case then \( x \in C \cap X \) or \( x \in f^{-1}(C \cap Y) \) or \( x \in f^{-1}(f(C \cap A)) \).
		\item \( x \in Y \).

		      If this is the case then \( x \in f(C \cap A) \) or \( C \cap Y \).
	\end{itemize}

	Conversely, if \( x \in (C \cap X) \cup f^{-1}(C \cap Y) \cup f^{-1}(f(C \cap A)) \cup f(C \cap A) \) then \( x \in p^{-1}(p(C)) \). Hence
	\begingroup
	\allowdisplaybreaks%
	\begin{align*}
		p^{-1}(p(C)) & = \underbrace{(C \cap X) \cup f^{-1}(C \cap Y) \cup f^{-1}(f(C \cap A))}_{\subset X} \cup \underbrace{(C \cap Y) \cup f(C \cap A)}_{\subset Y} \\
		             & = \underbrace{(C \cap X) \cup f^{-1}[(C \cap Y) \cup f(C \cap A)]}_{\subset X} \cup \underbrace{(C \cap Y) \cup f(C \cap A)}_{\subset Y}
	\end{align*}
	\endgroup

	So
	\begingroup
	\allowdisplaybreaks%
	\begin{align*}
		\textcolor{blue}{p^{-1}(p(C)) \cap Y} & = (C \cap Y) \cup f(C \cap A)                                    \\
		p^{-1}(p(C)) \cap X                   & = (C \cap X) \cup f^{-1}[(C \cap Y) \cup f(C \cap A)]            \\
		                                      & = (C \cap X) \cup f^{-1}[\textcolor{blue}{p^{-1}(p(C)) \cap Y}].
	\end{align*}
	\endgroup

	If \( p(C) \) is closed in \( X \cup_{f} Y \) then \( (C \cap Y) \cup f(C \cap A) \) is closed in \( Y \).

	If \( (C \cap Y) \cup f(C \cap A) \) is closed in \( Y \) then \( p^{-1}(p(C)) \cap Y \) is closed in \( Y \). Since \( f \) is continuous
	\[
		f^{-1}[p^{-1}(p(C)) \cap Y] \operatorname{\subset}\limits_{\text{closed}} A \operatorname{\subset}\limits_{\text{closed}} X
	\]

	then \( p^{-1}(p(C)) \cap X \) is closed in \( X \). Hence \( p^{-1}(p(C)) \) is closed in \( X + Y \), which means \( p(C) \) is closed in \( X \cup_{f} Y \).
\end{proof}

\begin{proposition}{6.4}
	Let \( X \) be attached to \( Y \) by \( f: A \to Y \). Let \( X_{1} \subset X \) and \( Y_{1} \subset Y \) be closed subsets such that \( f(A \cap X_{1}) \subset Y_{1} \), and attach \( X_{1} \) to \( Y_{1} \) by \( f_{1} = f\vert_{A \cap X_{1}} \). Then \( X_{1} \cup_{f_{1}} Y_{1} \) is homeomorphic to a closed subset of \( X \cup_{f} Y \).
\end{proposition}

\begin{proof}
	Let \( p: X + Y \to X \cup_{f} Y \) and \( p_{1}: X_{1} + Y_{1} \to X_{1} \cup_{f_{1}} Y_{1} \) and \( i: X_{1} + Y_{1} \to X + Y \) the inclusion map. The map \( p \circ i \circ p_{1}^{-1} \) is single-valued and the following diagram commutes
	\begin{figure}[htp]
		\centering
		\begin{tikzcd}
			{X_{1} + Y_{1}} &&& {X + Y} \\
			\\
			{X_{1} \cup_{f_{1}} Y_{1}} &&& {X \cup_{f} Y}
			\arrow["i", from=1-1, to=1-4]
			\arrow["{p_{1}}"', from=1-1, to=3-1]
			\arrow["p", from=1-4, to=3-4]
			\arrow["{p\circ i\circ p_{1}^{-1}}"', from=3-1, to=3-4]
		\end{tikzcd}
	\end{figure}

	Because \( i \) is continuous and relation-preserving, we deduce that \( p \circ i \circ p_{1}^{-1} \) is continuous. Evidently, \( p \circ i \circ p_{1}^{-1} \) is injective. We will show that this map is also a closed map.

	Every closed set in \( X_{1} \cup_{f_{1}} Y_{1} \) is the image of a closed and saturated set \( C_{1} \) under the map \( p_{1} \) (\(p_{1}(C_{1})\) is closed in \( X_{1} \cup_{f_{1}} Y_{1} \) and \(C_{1}\) is \(p_{1}\)-saturated).

	\( C_{1} \subset X_{1} + Y_{1} \) and is closed so \( C_{1} \cap X_{1} \) is closed. Moreover \( C_{1} \cap X = C_{1} \cap X_{1} \subset X_{1} \subset X \) is closed in \( X_{1} \) and \( X \). According to Proposition 6.2, \( p_{1}(C_{1}) \) is closed in \( X_{1} \cup_{f_{1}} Y_{1} \) so \( (C_{1} \cap Y_{1}) \cup f_{1}(C_{1} \cap A) \) is closed in \( Y_{1} \), hence closed in \( Y \). Moreover
	\[
		(C_{1} \cap Y_{1}) \cup f_{1}(C_{1} \cap A) = (C_{1} \cap Y) \cup f(C_{1} \cap A)
	\]

	is closed in \( Y \) so according to Proposition 6.2, \( p(C_{1}) \) is closed in \( X \cup_{f} Y \).
	\[
		(p \circ i \circ p_{1}^{-1})(p_{1}(C_{1})) = (p \circ i)(C_{1}) = p(C_{1}).
	\]

	Hence \( p \circ i \circ p_{1}^{-1} \) is a closed map, so it is a homeomorphism onto its image.

	Thus \( X_{1} \cup_{f_{1}} Y_{1} \) is embedded as a closed set of \( X \cup_{f} Y \).
\end{proof}

\begin{problem}{VI.6.1}
Let \( A \subset X \) and \( B \subset Y \) be closed. Show that \( A \ast B \) can be identified with a closed subspace of \( X \ast Y \) (it is a convention that \( \varnothing \ast Y = Y, X \ast \varnothing = X \)).
\end{problem}

\begin{proof}
	The join \( X \ast Y \) is obtained from attaching \( X \times I \times Y \) to \( X + Y \) by the continuous map \( f: X \times \left\{0, 1\right\} \times Y \to X + Y \) given by \( f(x, 0, y) = x, f(x, 1, y) = y \).

	Let \( p: X \times I \times Y + (X + Y) \to X \ast Y \) and \( q: A \times I \times B + (A + B) \to A \ast B \) be the identification maps.
	\begingroup
	\allowdisplaybreaks%
	\begin{align*}
		p^{-1}(p(A \times I \times B)) & = p^{-1}(p(A \times \openinterval{0,1} \times B) \cup p(A \times \left\{0\right\} \times B) \cup p(A \times \left\{1\right\} \times B)) \\
		                               & = A \times \openinterval{0, 1} \cup A \times \left\{0\right\} \times Y \cup X \times \left\{1\right\} \times B                          \\
		                               & = A \times I \times B \cup A \times \left\{0\right\} \times Y \cup X \times \left\{1\right\} \times B
	\end{align*}
	\endgroup

	is closed in \( X \times I \times Y + (X + Y) \) so \( p(A \times I \times B) \) is closed in \( X \ast Y \).

	The inclusion maps \( \iota_{A}: A \to X \) and \( \iota_{B}: B \to Y \) induce the continuous map \( f: A \ast B \to X \ast Y \). The map \( f \) is also injective.

	Let \( C \subseteq A \ast B \) be a closed subset then \( q^{-1}(C) \) is closed in \( A \times I \times B + (A + B) \). Moreover \( A \times I \times B + (A + B) \) is closed in \( X \times I \times Y + (X + Y) \) so \( q^{-1}(C) \) is closed in \( X \times I \times Y + (X + Y) \), and \( q^{-1}(C) \cap X \times I \times Y \) is closed in \( X \times I \times Y \) so \( p(q^{-1}(C)) \) is closed in \( X \ast Y \), according to Theorem VI.6.2.

	On the other hand, \( f(C) = p(q^{-1}(C)) \) so \( f \) is a closed map. Therefore \( f \) is an embedding.

	Thus \( A \ast B \) is homeomorphic to a closed subset of \( X \ast Y \).
\end{proof}

\begin{problem}{VI.6.2}
Prove: \( X \ast Y \cong Y \ast X \).
\end{problem}

\begin{proof}
	\( X \ast Y \) is a quotient space of \( X \times I \times Y + (X + Y) \); \( Y \ast X \) is a quotient space of \( Y \times I \times X + (X + Y) \).

	The map \( f: X \times I \times Y + (X + Y) \to Y \times I \times X + (X + Y) \) defined by
	\[
		\begin{cases}
			f(x, t, y) = (x, t, y) & (x, t, y) \in X \times I \times Y, \\
			f(x) = x               & x \in X,                           \\
			f(y) = y               & y \in Y,
		\end{cases}
	\]

	is a homeomorphism. Two homeomorphisms \( f, f^{-1} \) preserve relations, so \( X \ast Y \cong Y \ast X \).

	\bigskip
\end{proof}

%

\begin{problem}{VI.6.3}
Use 6.4 to prove that \( TX \) can be considered a closed subspace of \( SX \).
\end{problem}

\begin{proof}
	Remind the definitions of \( SX \) and \( TX \)
	\[
		SX = (X \times \left\lbrack -1, 1 \right\rbrack) \cup_{f} \left\{ y_{0}, y_{1} \right\}
	\]

	where \( f: X \times \left\{ -1; 1 \right\} \to \left\{ y_{0}, y_{1} \right\} \), \( f(x, -1) = y_{0}, f(x, 1) = y_{1} \).

	\[
		TX = (X \times \left\lbrack 0, 1 \right\rbrack) \cup_{f_{1}} \left\{ y_{1} \right\}
	\]

	where \( f_{1} = f\vert_{X \times \left\{ 1 \right\}} \) and \( f_{1}: X \times \left\{ 1 \right\} \to \left\{ y_{1} \right\} \).

	According to Theorem VI.6.4, \( TX \) is homeomorphic to a closed subspace of \( SX \).
\end{proof}

\begin{problem}{VI.6.4}
Attach \( TX \) to \( TX \) by \( f\left\langle x, 0\right\rangle = \left\langle x, 0\right\rangle \). Show that \( TX \cup_{f} TX \cong SX \).
\end{problem}

\begin{quotation}
	One can show that \( U \subseteq X \times [-1, 1] \) is closed if and only if \( U \cap X \times [-1, 0] \) is closed in \( X \times [-1, 0] \) and \( U \cap X \times [0, 1] \) is closed in \( X \times [0, 1] \). From this, one can deduce that \( X \times \left\lbrack -1, 1 \right\rbrack \cong X \times \left\lbrack -1, 0 \right\rbrack \cup_{f} X \times \left\lbrack 0, 1 \right\rbrack \). In general, any topological space is coherent with any locally finite closed cover (particularly, any finite closed cover).
\end{quotation}

\begin{proof}
	To avoid ambiguity, define
	\[
		\begin{cases}
			TX_{+} = (X \times \left\lbrack 0, 1 \right\rbrack)/(X \times \left\{1\right\}) \\
			TX_{-} = (X \times \left\lbrack -1, 0 \right\rbrack)/(X \times \left\{-1\right\})
		\end{cases}
	\]

	then \( TX, TX_{+}, TX_{-} \) are homeomorphic.

	\( TX_{-} \cup_{f} TX_{+} \) is a quotient space of \( TX_{-} + TX_{+} \).

	\( TX_{-} + TX_{+} \) is a quotient space of \( X \times \left\lbrack -1, 0 \right\rbrack + X \times \left\lbrack 0, 1 \right\rbrack \).

	\( SX \) is a quotient space of \( X \times \left\lbrack -1, 1 \right\rbrack \).

	\( X \times \left\lbrack -1, 1 \right\rbrack \cong X \times \left\lbrack -1, 0 \right\rbrack \cup_{f} X \times \left\lbrack 0, 1 \right\rbrack \) is a quotient space of \( X \times \left\lbrack -1, 0 \right\rbrack + X \times \left\lbrack 0, 1 \right\rbrack \).

	Moreover, two identification maps \( X \times \left\lbrack -1, 0 \right\rbrack + X \times \left\lbrack 0, 1 \right\rbrack \to TX_{-} \cup_{f} TX_{+} \) and \( X \times \left\lbrack -1, 0 \right\rbrack + X \times \left\lbrack 0, 1 \right\rbrack \to SX \) have the same identifications so \( TX_{-} \cup_{f} TX_{+} \cong SX \).
\end{proof}

\begin{problem}{VI.6.5}
Let \(X\) be attached to \(Y\) by \(f: A \to Y\). Let \( V \subset Y \) be open, and let \( U \subset X \) be open, such that \( f^{-1}(V) = U \cap A \). Prove: \( p(V \cup U) \) is open in \( X \cup_{f} Y \).
\end{problem}

\begin{quotation}
	\( p: X + Y \to X \cup_{f} Y \) is the identification map that induces the attaching space.
\end{quotation}

\begin{proof}
	In Proposition 6.2, we showed that
	\[
		p^{-1}p(C) = C \cup f(C \cap A) \cup f^{-1}[f(C \cap A)] \cup f^{-1}(C \cap Y)
	\]

	so
	\begingroup
	\allowdisplaybreaks%
	\begin{align*}
		p^{-1}(p(U \cup V)) & = (U \cup V) \cup f(U \cap A) \cup f^{-1}[f(U \cap A)] \cup f^{-1}(V) \\
		                    & = U \cup V \cup f(U \cap A) \cup f^{-1}[f(U \cap A)] \cup (U \cap A)  \\
		                    & = U \cup V \cup f(U \cap A) \cup f^{-1}[f(U \cap A)].
	\end{align*}
	\endgroup

	On the other hand
	\begingroup
	\allowdisplaybreaks%
	\begin{align*}
		f(U \cap A)         & = f(f^{-1}(V) \cap A) = V \cap f(A) \subset V, \\
		f^{-1}[f(U \cap A)] & \subseteq f^{-1}(V) = U \cap A \subset U.
	\end{align*}
	\endgroup

	Therefore \( p^{-1}p(U \cup V) = U \cup V \cup f(U \cap A) \cup f^{-1}[f(U \cap A)] = U \cup V \), which means \( p^{-1}p(U \cup V) \) is open. Because \( p \) is an identifcaton map, \( p(U \cup V) \) is open in \( X \cup_{f} Y \).
\end{proof}

\begin{problem}{VI.6.6}
Let \( F \subset X \cup_{f} Y \) be a given set. Assume that \( V \subset Y, U \subset X \) are open and satisfy: (a) \( Y \cap p^{-1}(F) \subset V \); (b) \( X \cap p^{-1}[F \cup p(V)] \subset U \). Show that \( p[(U - A) \cup V] \) is open in \( X \cup_{f} Y \) and contains \( F \).
\end{problem}

\begin{quotation}
	I couldn't prove that \( p[(U - A) \cup V] \) is open in \( X \cup_{f} Y \).
\end{quotation}

\begin{proof}
	\begingroup
	\allowdisplaybreaks%
	\begin{align*}
		p(Y) \cap F     & = p(Y \cap p^{-1}(F)) \subset p(V),                                                          \\
		p(X - A) \cap F & \subset p(X - A) \cap (F \cup p(V)) = p[(X - A) \cap p^{-1}(F \cup p(V))] \subseteq p(U - A)
	\end{align*}
	\endgroup

	so \( F = (p(Y) \cap F) \cup (p(X - A) \cap F) \subset p(V) \cup p(U - A) = p((U - A) \cup V) \).
	\begingroup
	\allowdisplaybreaks%
	\begin{align*}
		p^{-1}p[(U - A) \cup V] \cap Y & = (((U - A) \cup V) \cap Y) \cup f[((U - A) \cup V) \cap A] = V,     \\
		p^{-1}p[(U - A) \cup V] \cap X & = (((U - A) \cup V) \cap X) \cup f^{-1}(V) = (U - A) \cup f^{-1}(V).
	\end{align*}
	\endgroup

	% TODO
\end{proof}

\section{The Relation \(K(f)\) of Continuous Maps}

\begin{problem}{VI.7.1}
Show that \( I^{n} \) is a quotient space of the Cantor set.
\end{problem}

\begin{quotation}
	The following proof makes use of the closed map lemma, which uses compactness, which is not yet introduced.
\end{quotation}

\begin{proof}
	Let \( C \) be the Cantor set.

	According to Proposition IV.4.3, there exists a continuous surjection \( p: C \to I^{n} \).

	Because \( C \) is compact and \( I^{n} \) and Hausdorff so \( p \) is closed, according to the closed map lemma. Therefore \( p \) is an identification map, which means \( I^{n} \) is a quotient space of the Cantor set.
\end{proof}

\begin{problem}{VI.7.2}
If \( Y \) is a quotient space \( X \), and \( Z \) is a quotient space of \( Y \), prove that \( Z \) is homeomorphic to a quotient space of \( X \).
\end{problem}

\begin{proof}
	Let \( p: X \to Y \) and \( q: Y \to Z \) be the identification maps.
	\begingroup
	\allowdisplaybreaks%
	\begin{align*}
		U \subseteq_{\text{open}} Z & \iff q^{-1}(U) \subseteq_{\text{open}} Y             \\
		                            & \iff p^{-1}[q^{-1}(U)] \subseteq_{\text{open}} X     \\
		                            & \iff {(q \circ p)}^{-1}(U) \subseteq_{\text{open}} X
	\end{align*}
	\endgroup

	so \( q \circ p \) is a quotient map, which means \( Z \) is a quotient space of \( X \).
\end{proof}

\begin{problem}{VI.7.3}
Let \( R_{1}, R_{2} \) be two relations in a space \( X \) such that \( x R_{1} x^{\prime} \implies x R_{2} x^{\prime} \) for every pair \( x, x^{\prime} \). Show that \( X/R_{2} \) is a quotient space of \( X/R_{1} \).
\end{problem}

\begin{proof}
	Let \( f: X \to X \) be the identity map of \( X \).
	\[
		\begin{tikzcd}
			X && X \\
			\\
			{X/R_{1}} && {X/R_{2}}
			\arrow["f", from=1-1, to=1-3]
			\arrow["{p_{1}}"', from=1-1, to=3-1]
			\arrow["{p_{2}}", from=1-3, to=3-3]
			\arrow["{f_{\ast}}"', from=3-1, to=3-3]
		\end{tikzcd}
	\]

	\( f \) preserves relation so there exists an induce map \( f_{\ast} \) that commutes the above diagram. Since \( f \) is an identification (because it is a homeomorphism), \( f_{\ast} \) is also an identification. Hence \( X/R_{2} \) is a quotient space of \( X/R_{1} \).
\end{proof}

\begin{problem}{VI.7.4}
Let \( A \subset X \) be a retract of \( X \), and let \( r: X \to A \) be the map in Problem~\ref{problem:VI.1.3}. Show \( A \cong X/K(r) \).
\end{problem}

\begin{proof}
	In Problem~\ref{problem:VI.1.3}, we proved that \( r \) is an identification, so \( A \cong X/K(r) \).
\end{proof}

\section{Weak Topologies (Coherent Topologies)}

\begin{problem}{VI.8.1}
If we drop both requirements in VI.8.1, does the process still determine a topology?
\end{problem}

\begin{quotation}
	Yes. However, the requirements allow us to prove more fruitful results.
\end{quotation}

\begin{proof}
	Assume that the process drops both requirements.

	If \( U_{i} \in \mathscr{T}(\mathfrak{A}) \) for every \( i \in I \) then
	\[
		A_{\alpha} \cap \bigcup_{i \in I} U_{i} = \bigcup_{i\in I} A_{\alpha} \cap U_{i}
	\]

	is open in \( A_{\alpha} \) for every \( \alpha \in \mathscr{A} \). Therefore \( \bigcup_{i \in I} U_{i} \in \mathscr{T}(\mathfrak{A}) \).

	If \( U, V \in \mathscr{T}(\mathfrak{A}) \) then
	\[
		U \cap V \cap A_{\alpha} = (U \cap A_{\alpha}) \cap (V \cap A_{\alpha})
	\]

	is open in \( A_{\alpha} \) for every \( \alpha \in \mathscr{A} \). Therefore \( U \cap V \in \mathscr{T}(\mathfrak{A}) \).

	Hence the process still determine a topology even if we drop both requirements in VI.8.1.
\end{proof}

\begin{problem}{VI.8.2}\label{problem:VI.8.2}
Let both \( X \) and \( Y \) have weak topology, with that for \( X \) determined by \( \left\{ A_{\alpha} \mid \alpha \in \mathscr{A} \right\} \) and that for \( Y \) determined by \( \left\{ B_{\beta} \mid \beta \in \mathscr{B} \right\} \). Let \( X \times Y \) be the Cartesian product topology, and let \( (X \times Y, \mathscr{T}) \) be the weak topology in \( X \times Y \) determined by \( \left\{ A_{\alpha} \times B_{\beta} \mid (\alpha, \beta) \in \mathscr{A} \times \mathscr{B} \right\} \). Show that the map \( \ell: (X \times Y, \mathscr{T}) \to X \times Y \) is continuous.
\end{problem}

\begin{proof}
	Let \( U \times Y, X \times V \) be subbasic open sets in \( X \times Y \).
	\[
		\begin{split}
			\ell^{-1}(U \times Y) \cap A_{\alpha} \times B_{\beta} = (U \cap A_{\alpha}) \times B_{\beta} \subseteq_{\text{open}} A_{\alpha} \times B_{\beta}, \\
			\ell^{-1}(X \times V) \cap A_{\alpha} \times B_{\beta} = A_{\alpha} \times (V \cap B_{\beta}) \subseteq_{\text{open}} A_{\alpha} \times B_{\beta}.
		\end{split}
	\]

	Therefore the preimages of subbasic open sets in \( X \times Y \) are open in \( (X \times Y, \mathscr{T}) \), which means \( \ell \) is continuous.
\end{proof}

\begin{problem}{VI.8.3}
In Problem~\ref{problem:VI.8.2}, let both \( \left\{ A_{\alpha} \mid \alpha \in \mathscr{A} \right\} \) and \( \left\{ B_{\beta} \mid \beta \in \mathscr{B} \right\} \) be open coverings or nbd-finite closed coverings. Show that \( \ell \) is then always a homeomorphism.
\end{problem}

\begin{proof}
	If \( \left\{ A_{\alpha} \mid \alpha \in \mathscr{A} \right\} \) and \( \left\{ B_{\beta} \mid \beta \in \mathscr{B} \right\} \) are open coverings (nbd-finite closed coverings) then \( \left\{ A_{\alpha} \times B_{\beta} \mid (\alpha, \beta) \in \mathscr{A} \times \mathscr{B} \right\} \) is also an open covering (nbd-finite closed covering). Therefore the topology of \( X \times Y \) is exactly the weak topology induced by \( \left\{ A_{\alpha} \times B_{\beta} \mid (\alpha, \beta) \in \mathscr{A} \times \mathscr{B} \right\} \), according to Theorem III.9.3.
\end{proof}

\begin{problem}{VI.8.4}
In Problem~\ref{problem:VI.8.2}, assume that each \( x \in X \) is in the interior of some \( A_{\alpha} \) and also that each \( y \in Y \) is in the interior of some \( B_{\beta} \). Show that the map \( \ell: (X \times Y, \mathscr{T}) \to X \times Y \) is a homeomorphism.
\end{problem}

\begin{proof}
	For each \( (x, y) \in (X \times Y, \mathscr{T}) \), there exist \( A_{\alpha}, B_{\beta} \) such that
	\[
		x \in \operatorname{Int}(A_{\alpha}), y \in \operatorname{Int}(B_{\beta}).
	\]

	The product \( \operatorname{Int} A_{\alpha} \times \operatorname{Int} B_{\beta} \) is open in \( (X \times Y, \mathscr{T}) \) and \( X \times Y \).

	If \( U \subseteq_{\text{open}} \operatorname{Int} A_{\alpha} \times \operatorname{Int} B_{\beta} \) then \( \ell(U) \subseteq_{\text{open}} \operatorname{Int} A_{\alpha} \times \operatorname{Int} B_{\beta} \). Hence \( \ell \) is a local homeomorphism, hence an open map.

	Thus \( \ell \) is a homeomorphism.
\end{proof}

\begin{problem}{VI.8.5}
Let \( \left\{ Y_{\alpha} \mid \alpha \in \mathscr{A} \right\} \) be any family of spaces, and \( \left\{ b_{\alpha}^{0} \right\} \) a fixed point in \( \prod_{\alpha} Y_{\alpha} \). Let \( PY_{\alpha} \) be the subset of all points in the set \( \prod_{\alpha} Y_{\alpha} \) having at most finitely many coordinates different from \( \left\{ b_{\alpha}^{0} \right\} \). For each finite \( \mathscr{F} \subset \mathscr{A} \), let \( s(\mathscr{F}) \) be the slice through \( \left\{ b_{\alpha}^{0} \right\} \) parallel to \( \prod_{\alpha\in\mathscr{F}} Y_{\alpha} \). Take each \( s(\mathscr{F}) \) with the Cartesian product topology, and let \( PY_{\alpha} \) be given the weak topology determined by the family \( \left\{ s(\mathscr{F}) \right\} \). Prove:
\begin{enumerate}[itemsep=0pt,label={(\arabic*)}]
	\item Each projection \( p_{\beta}\vert_{PY_{\alpha}}: PY_{\alpha} \to Y_{\beta} \) is a continuous open map.
	\item If \( s(b; a) \) is the slice in \( \prod_{\alpha} Y_{\alpha} \) through \( \left\{ b_{\alpha} \right\} \) parallel to \( Y_{\alpha} \), the map \( s_{\alpha}: Y_{\alpha} \to s(b; \alpha) \) is continuous.
	\item This type of Cartesian product is in general not associative (hint: use Dowker's example).
\end{enumerate}
\end{problem}

\begin{quotation}
	I skipped this problem.
\end{quotation}

\begin{proof}
	% TODO
\end{proof}
