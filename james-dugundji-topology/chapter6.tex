\chapter{Identification Topology; Weak Topology}

\section{Identification Topology}

\begin{problem}{VI.1.1}\label{problem:VI.1.1}
Reversing the situation treated in the text, let \(X\) be a set, \( (Y, \mathscr{T}) \) a space, and \( p: X \to Y \) a surjective map. Prove:
\begin{enumerate}[label={(\alph*)}]
	\item \( \mathscr{T}_{X} = \left\{ p^{-1}(U) \mid U \text{ open in } Y \right\} \) is a topology in \( X \).
	\item \( p: (X, \mathscr{T}_{X}) \to (Y, \mathscr{T}) \) is continuous, open, and closed.
\end{enumerate}
\end{problem}

\begin{proof}
	\begin{enumerate}[label={(\alph*)}]
		\item \( \mathscr{T}_{X} \) contains \( \varnothing, X \) as \( p^{-1}(\varnothing) = \varnothing \) and \( p^{-1}(Y) = X \).

		      If \( {\left\{ U_{\alpha} \right\}}_{\alpha\in\mathscr{A}} \) is a collection of open sets in \( Y \), then
		      \[
			      \bigcup_{\alpha\in\mathscr{A}} p^{-1}(U_{\alpha}) = p^{-1}\left(\bigcup_{\alpha\in\mathscr{A}} U_{\alpha}\right)
		      \]

		      so \( \mathscr{T}_{X} \) is closed under arbitrary union.

		      If \( U_{1}, \ldots, U_{n} \) are open sets in \( Y \) then
		      \[
			      \bigcap^{n}_{i=1} p^{-1}(U_{i}) = p^{-1}\left(\bigcap^{n}_{i=1} U_{i}\right)
		      \]

		      so \( \mathscr{T}_{X} \) is closed under finite intersection.

		      Hence \( \mathscr{T}_{X} \) is a topology in \( X \).
		\item For each open set \( U \) in \( Y \), \( p^{-1}(U) \in \mathscr{T}_{X} \) so \( p \) is continuous.

		      Let \( V \) be an open set in \( X \). Then there is an open set \( U \) in \( Y \) such that \( V = p^{-1}(U) \). Hence \( p(V) = pp^{-1}(U) = U \) because \( p \) is surjective. So \( p \) is an open map.

		      Let \( W \) be a closed set in \( X \) then \( X - W \) is open and there exists an open set \( U \) in \( Y \) such that \( X - W = p^{-1}(U) \). Therefore
		      \[
			      W = X - p^{-1}(U) = p^{-1}(Y) - p^{-1}(U) = p^{-1}(Y - U)
		      \]

		      which implies that \( p(W) = pp^{-1}(Y - U) = Y - U \), which is closed in \( Y \). So \( p \) is a closed map.

		      Thus \( p \) is a continuous, open, and closed map.
	\end{enumerate}
\end{proof}

\begin{problem}{VI.1.2}
For each \( \alpha \in \mathscr{A} \), let \( p_{\alpha}: X_{\alpha} \to Y_{\alpha} \) be a continuous, open surjection. Show that \( \prod_{\alpha} p_{\alpha}: \prod_{\alpha} X_{\alpha} \to \prod_{\alpha} Y_{\alpha} \) is an identification.
\end{problem}

\begin{proof}
	For the sake of brevity, denote \( p = \prod_{\alpha} p_{\alpha} \). By definition, \( p \) is surjective.

	\( p_{Y_{\alpha}} \circ p \) is continuous for each projection \( p_{Y_{\alpha}}: \prod_{\alpha} Y_{\alpha} \to Y_{\alpha} \) so \( p \) is continuous.

	Let \( \prod_{\alpha} U_{\alpha} \) be a basic open set in \( \prod_{\alpha} X_{\alpha} \), which means \( U_{\alpha} = X_{\alpha} \) for all but finitely many \( \alpha \) and \( U_{\alpha} \) is open in \( X_{\alpha} \) for every \( \alpha \). Because \( p_{\alpha} \) is an open surjection for each \( \alpha \), the image
	\[
		p\left( \prod_{\alpha} U_{\alpha} \right) = \prod_{\alpha} p_{\alpha}(U_{\alpha})
	\]

	is open in \( \prod_{\alpha} Y_{\alpha} \) as \( p_{\alpha}(U_{\alpha}) \) is open in \( Y_{\alpha} \) and \( p_{\alpha}(U_{\alpha}) = Y_{\alpha} \) for all but finitely many \( \alpha \). Hence \( p \) is an open map.

	\( p \) is a continuous, open surjection so \( p \) is an identification.
\end{proof}

\begin{problem}{VI.1.3}
Let \( X \) be a space and \( A \subset X \) a subspace. Assume that there exists a continuous \( r: X \to A \) such that \( r\vert_{A} = 1_{A} \) (such a map is called a \textit{retraction} of \(X\) onto \(A\)). Show that \( r \) is an identification.
\end{problem}

\begin{proof}
	By definition, \( r \) is continuous and surjective. Let \( f: A \xhookrightarrow{} X \) be the inclusion map.

	\( f \) is continuous and \( r \circ f = 1_{A} \) so \( r \) is an identification.
\end{proof}

\begin{problem}{VI.1.4}\label{problem:VI.1.4}
Let \( X \) be any set. Given any family \( \left\{ (Y_{\alpha}, \mathscr{T}_{\alpha}), f_{\alpha} \mid \alpha \in \mathscr{A} \right\} \) of spaces and maps \( f_{\alpha}: X \to Y_{\alpha} \), the ``projective limit topology of \(X\) determined by this family'' is \( \bigvee_{\alpha} f_{\alpha}^{-1}(\mathscr{T}_{\alpha}) \) (see Problem~\ref{problem:III.3.8}). Prove:
\begin{enumerate}[label={(\alph*)}]
	\item If \( j: X \to \prod_{\alpha} Y_{\alpha} \) is the map \( j(x) = \left\{ f_{\alpha}(x) \right\} \), then \( \bigvee_{\alpha} f_{\alpha}^{-1}(\mathscr{T}_{\alpha}) \) is the topology in \(X\) determined by \(j\) as in Problem~\ref{problem:VI.1.1}.
	\item If whenever \( x \ne x^{\prime} \), there is some index \( \alpha \) such that \( f_{\alpha}(x) \ne f_{\alpha}(x^{\prime}) \), then \( j \) is an embedding.
\end{enumerate}
\end{problem}

\begin{proof}
	\begin{enumerate}[label={(\alph*)}]
		\item Let \( \prod_{\alpha} U_{\alpha} \) be a subbasic open set in \( \prod_{\alpha} Y_{\alpha} \) then \( U_{\alpha} = Y_{\alpha} \) for every \( \alpha \) but one \( \beta \in \mathscr{A} \).
		      \[
			      j^{-1}\left( \prod_{\alpha} U_{\alpha} \right) = \bigcap_{\alpha} f_{\alpha}^{-1}(U_{\alpha}) = f_{\beta}^{-1}(U_{\beta}) \in \bigvee_{\alpha} f_{\alpha}^{-1}(\mathscr{T}_{\alpha})
		      \]

		      Hence \( j \) is continuous, which means if \( j^{-1}(U) \) is open whenever \( U \subset \prod_{\alpha} Y_{\alpha} \) is open.

		      Let \( V \) be an open set in \( X \). According to the definition of the topology \( \bigvee_{\alpha} f_{\alpha}^{-1}(\mathscr{T}_{\alpha}) \), \( V \) can be written as a union of finite intersection of elements in \( \bigcup_{\alpha} f_{\alpha}^{-1}(\mathscr{T}_{\alpha}) \), which means
		      \[
			      V = \bigcup_{i\in I} V_{i}
		      \]

		      where each \( V_{i} \) is a finite intersection of elements in \( \bigcup_{\alpha} f_{\alpha}^{-1}(\mathscr{T}_{\alpha}) \).
		      \[
			      V_{i} = \bigcap^{n_{i}}_{k=1} f_{\alpha_{k}}^{-1}(U_{\alpha_{k}}) = \bigcap^{n_{i}}_{k=1} j^{-1}\left( U_{\alpha_{k}} \times \prod_{\alpha \ne \alpha_{k}} Y_{\alpha} \right) = j^{-1}\left( \bigcap^{n_{i}}_{k=1} U_{\alpha_{k}} \times \prod_{\alpha \ne \alpha_{k}} Y_{\alpha} \right) = j^{-1}(W_{i})
		      \]

		      where \( U_{\alpha_{k}} \) is open in \( Y_{\alpha_{k}} \). So
		      \[
			      V = \bigcup_{i\in I} j^{-1}(W_{i}) = j^{-1}\left( \bigcup_{i\in I} W_{i} \right)
		      \]

		      which means \( V \) is the preimage of an open set in \( \prod_{\alpha} Y_{\alpha} \).

		      Thus \( \bigvee_{\alpha} f_{\alpha}^{-1}(\mathscr{T}_{\alpha}) \) is the same as the topology in \( X \) determined by \( j \) as in Problem~\ref{problem:VI.1.1}.
		\item According to Problem~\ref{problem:VI.1.1}, \( j \) is continuous, open, and closed.

		      Whenever \( x \ne x^{\prime} \), there is some index \( \alpha \) such that \( f_{\alpha}(x) \ne f_{\alpha}(x^{\prime}) \), then \( j(x) \ne j(x^{\prime}) \), which implies \( j \) is injective.

		      A continuous, open, injective map is an embedding so \( j \) is an embedding.
	\end{enumerate}
\end{proof}

\section{Subspaces}

\begin{problem}{VI.2.1}
Let \(X\) have the projective limit topology (Problem~\ref{problem:VI.1.4}) determined by
\[
	\left\{ Y_{\alpha}, f_{\alpha} \mid \alpha \in \mathscr{A} \right\}
\]

and let \( A \subset X \). Prove: The subspace topology of \(A\) is the projective limit topology determined by the maps \( f_{\alpha}\vert_{A} \).
\end{problem}

\begin{proof}
	The projective limit topology on \( A \) determined by the maps \( f_{\alpha}\vert_{A} \) has subbasis
	\[
		\bigcup_{\alpha} {(f_{\alpha}\vert_{A})}^{-1}(\mathscr{T}_{\alpha})
	\]

	Let \( V \) be an open set in \( A \) (with the projective limit topology) then
	\[
		V = \bigcup_{i\in I} V_{i}
	\]

	in which each \( V_{i} \) is the intersection of finitely many elements of \( \bigcup_{\alpha} {(f_{\alpha}\vert_{A})}^{-1}(\mathscr{T}_{\alpha}) \). So there exist \( \alpha_{i_{1}}, \ldots, \alpha_{i_{n(i)}} \in \mathscr{A} \) such that
	\[
		V_{i} = \bigcap^{n(i)}_{k=1} {(f_{\alpha_{k}}\vert_{A})}^{-1}(U_{\alpha_{k}})
	\]

	Hence
	\begingroup
	\allowdisplaybreaks%
	\begin{align*}
		V_{i} & = \bigcap^{n(i)}_{k=1} (A \cap f_{\alpha_{k}}^{-1}(U_{\alpha_{k}}))                                                             \\
		      & = A \cap \bigcap^{n(i)}_{k=1} f_{\alpha_{k}}^{-1}(U_{\alpha_{k}})                                                               \\
		      & = A \cap \bigcap^{n(i)}_{k=1} j^{-1}\left( U_{\alpha_{k}} \times \prod_{\alpha \ne \alpha_{k}} Y_{\alpha} \right)               \\
		      & = A \cap j^{-1}\left( \bigcap^{n(i)}_{k=1} \left( U_{\alpha_{k}} \times \prod_{\alpha\ne\alpha_{k}} Y_{\alpha} \right) \right).
	\end{align*}
	\endgroup

	Therefore
	\begingroup
	\allowdisplaybreaks%
	\begin{align*}
		V & = A \cap \bigcup_{i\in I} j^{-1}\left( \bigcap^{n(i)}_{k=1} \left( U_{\alpha_{k}} \times \prod_{\alpha\ne\alpha_{k}} Y_{\alpha} \right) \right) \\
		  & = A \cap j^{-1}\left( \bigcup_{i\in J} \bigcap^{n(i)}_{k=1} \left( U_{\alpha_{k}} \times \prod_{\alpha\ne\alpha_{k}} Y_{\alpha} \right) \right)
	\end{align*}
	\endgroup

	Hence \( V \) is in the subspace topology of \( A \).

	Conversely, one can show that if \( V \) is in the subspace topology of \( A \), then \( V \) is also in the projective limit topology on \( A \) determinded by the maps \( f_{\alpha}\vert_{A} \).

	Thus the projective limit topology on \( A \) determinded by the maps \( f_{\alpha}\vert_{A} \) and the subspace topology on \( A \) coincide.
\end{proof}

\section{General Theorems}

\begin{problem}{VI.3.1}
Let \( p: X \to Y \) be a continuous open (or closed) surjection, and assume that each fiber \( p^{-1}(y) \) is connected. For any \( F \subset Y \), show that \( F \) is connected if and only if \( p^{-1}(F) \) is connected.
\end{problem}

\begin{proof}
	By Proposition 2.1, \( p\vert_{p^{-1}(F)}: p^{-1}(F) \to F \) is an identification because \( p \) is an identification which is also an open (or closed) map. Denote \( q = p\vert_{p^{-1}(F)} \).

	If \( p^{-1}(F) \) is connected then \( F = p(p^{-1}(F)) \) is connected, as \( p \) is a continuous surjection.

	If \( p^{-1}(F) \) is not connected then there is a continuous surjection \( h: p^{-1}(F) \to 2 \). As each fiber of \( q \) (each fiber of \(q \) is a fiber of \(p\)) is connected, the restriction of \( h \) to each fiber is a constant map. Therefore \( hq^{-1}: F \to 2 \) is a continuous surjection, according to the transgression property, which means \( F \) is not connected.
\end{proof}

\begin{problem}{VI.3.2}
Let \( X \) have the projective limit topology \( \mathscr{T} \) determined by the family
\[
	\left\{ (Y_{\alpha}, \mathscr{T}_{\alpha}), f_{\alpha} \mid \alpha \in \mathscr{A} \right\}
\]

Assume that each \( \mathscr{T}_{\alpha} \) is the projective limit topology determined by a family
\[
	\left\{ (Z_{\alpha, \beta}, \mathscr{T}_{\alpha,\beta}), g_{\alpha,\beta} \mid \beta \in \mathscr{B} \right\}.
\]

Prove: \( \mathscr{T} \) is the projective limit topology determined by
\[
	\left\{ (Z_{\alpha,\beta}, \mathscr{T}_{\alpha,\beta}), g_{\alpha,\beta} \circ f_{\alpha} \mid (\alpha, \beta) \in \mathscr{A} \times \mathscr{B} \right\}.
\]
\end{problem}

\begin{proof}
	Denote by \( \widetilde{\mathscr{T}} \) the projective limit topology determined by
	\[
		\left\{ (Z_{\alpha,\beta}, \mathscr{T}_{\alpha,\beta}), g_{\alpha,\beta} \circ f_{\alpha} \mid (\alpha, \beta) \in \mathscr{A} \times \mathscr{B} \right\}.
	\]

	Let \( h: X \to \prod_{\alpha} Y_{\alpha} \) be the map \( h(x) = {\left\{ f_{\alpha}(x) \right\}}_{\alpha} \) then
	\[
		\mathscr{T} = \left\{ h^{-1}(U) \mid U \text{ open in } \prod_{\alpha}Y_{\alpha} \right\}
	\]

	according to Problem~\ref{problem:VI.1.4}.

	For each \( \alpha \), let \( h_{\alpha}: Y_{\alpha} \to \prod_{\beta} Z_{\alpha,\beta} \) be the map \( h_{\alpha}(x) = {\left\{ g_{\alpha,\beta}(x) \right\}}_{\beta} \) then
	\[
		\mathscr{T}_{\alpha} = \left\{ h_{\alpha}^{-1}(U) \mid U \text{ open in } \prod_{\beta} Z_{\alpha,\beta} \right\}
	\]

	according to Problem~\ref{problem:VI.1.4}.

	Let \( \ell: (X, \widetilde{\mathscr{T}}) \to \prod_{\alpha,\beta} Z_{\alpha,\beta} \) be the map \( \ell(x) = {\left\{ g_{\alpha,\beta}(f_{\alpha}(x)) \right\}}_{\alpha,\beta} \) then
	\[
		\widetilde{\mathscr{T}} = \left\{ \ell^{-1}(U) \mid U \text{ open in } \prod_{\alpha,\beta} Z_{\alpha,\beta} \right\}
	\]

	according to Problem~\ref{problem:VI.1.4}.

	Note that \( f_{\alpha} = p_{\alpha} \circ h \) and \( g_{\alpha,\beta} = p_{\alpha,\beta} \circ h_{\alpha} \) in which \( p_{\alpha}: \prod_{\alpha} Y_{\alpha} \to Y_{\alpha} \) and \( p_{\alpha,\beta}: \prod_{\beta} Z_{\alpha,\beta} \to Z_{\alpha,\beta} \) are projection maps. Denote by \( q_{\alpha,\beta} \) the projection map \( \prod_{\alpha,\beta} W_{\alpha,\beta} \to W_{\alpha,\beta} \).
	\[
		\begin{tikzcd}
			&& {\prod_{\alpha} Y_{\alpha}} \\
			\\
			X && {Y_{\alpha}} && {\prod_{\beta}Z_{\alpha,\beta}} && {Z_{\alpha,\beta}}
			\arrow["{p_{\alpha}}", from=1-3, to=3-3]
			\arrow["h", from=3-1, to=1-3]
			\arrow["{f_{\alpha}}"', from=3-1, to=3-3]
			\arrow["{h_{\alpha}}"', from=3-3, to=3-5]
			\arrow["{g_{\alpha,\beta}}"', bend right, from=3-3, to=3-7]
			\arrow["{p_{\alpha,\beta}}"', from=3-5, to=3-7]
		\end{tikzcd}
	\]

	\[
		\begin{tikzcd}
			X && {\prod_{\alpha,\beta} Z_{\alpha,\beta}} && {Z_{\alpha,\beta}}
			\arrow["\ell", from=1-1, to=1-3]
			\arrow["{g_{\alpha,\beta} \circ f_{\alpha}}"', bend right, from=1-1, to=1-5]
			\arrow["{q_{\alpha,\beta}}", from=1-3, to=1-5]
		\end{tikzcd}
	\]

	Let \( U \in \mathscr{T} \) then there exists \( V \) open in \( \prod_{\alpha} Y_{\alpha} \) such that \( U = h^{-1}(V) \) (see Problem~\ref{problem:VI.1.4} and~\ref{problem:VI.1.1}). One can write \( V \) in terms of subbasic elements as follows
	\[
		V = \bigcup_{i\in I} \bigcap^{n(i)}_{k=1} p_{\alpha_{k}}^{-1}(V_{\alpha_{k}})
	\]

	in which \( V_{\alpha_{k}} \) is open in \( Y_{\alpha_{k}} \).

	As \( \mathscr{T}_{\alpha} \) is the projective limit topology on \( Y_{\alpha} \) determined by the maps \( g_{\alpha,\beta}: Y_{\alpha} \to Z_{\alpha,\beta} \), there is an open set \( W_{\alpha_{k}} \) in \( \prod_{\beta} Z_{\alpha_{k},\beta} \) such that \( V_{\alpha_{k}} = h_{\alpha}^{-1}(W_{\alpha_{k}}) \). The open set \( W_{\alpha_{k}} \) can be written in terms of subbasic elements as follows
	\[
		W_{\alpha_{k}} = \bigcup_{j \in J} \bigcap^{n(j)}_{r=1} p_{\alpha_{k},\beta_{r}}^{-1}(W_{\alpha_{k}, \beta_{r}})
	\]

	in which \( W_{\alpha_{k}, \beta_{r}} \) is open in \( Z_{\alpha_{k}, \beta_{r}} \).
	\begingroup
	\allowdisplaybreaks%
	\begin{align*}
		V             & = \bigcup_{i\in I} \bigcap^{n(i)}_{k=1} p_{\alpha_{k}}^{-1}(V_{\alpha_{k}})                                                                                                                         \\
		              & = \bigcup_{i\in I} \bigcap^{n(i)}_{k=1} p_{\alpha_{k}}^{-1}\left( h^{-1}_{\alpha_{k}}(W_{\alpha_{k}}) \right)                                                                                       \\
		              & = \bigcup_{i\in I} \bigcap^{n(i)}_{k=1} {(h_{\alpha_{k}} \circ p_{\alpha_{k}})}^{-1}(W_{\alpha_{k}})                                                                                                \\
		              & = \bigcup_{i\in I} \bigcap^{n(i)}_{k=1} {(h_{\alpha_{k}} \circ p_{\alpha_{k}})}^{-1} \left( \bigcup_{j \in J} \bigcap^{n(j)}_{r=1} p_{\alpha_{k},\beta_{r}}^{-1}(W_{\alpha_{k}, \beta_{r}}) \right) \\
		              & = \bigcup_{i\in I} \bigcap^{n(i)}_{k=1} \bigcup_{j\in J} \bigcap^{n(j)}_{r=1} {(p_{\alpha_{k},\beta_{r}} \circ h_{\alpha_{k}} \circ p_{\alpha_{k}})}^{-1}(W_{\alpha_{k},\beta_{r}})                 \\
		U = h^{-1}(V) & = \bigcup_{i\in I} \bigcap^{n(i)}_{k=1} \bigcup_{j\in J} \bigcap^{n(j)}_{r=1} {(p_{\alpha_{k},\beta_{r}} \circ h_{\alpha_{k}} \circ p_{\alpha_{k}} \circ h)}^{-1}(W_{\alpha_{k},\beta_{r}})         \\
		              & = \bigcup_{i\in I} \bigcap^{n(i)}_{k=1} \bigcup_{j\in J} \bigcap^{n(j)}_{r=1} {(g_{\alpha_{k},\beta_{r}} \circ f_{\alpha_{k}})}^{-1}(W_{\alpha_{k},\beta_{r}})                                      \\
		              & = \bigcup_{i\in I} \bigcap^{n(i)}_{k=1} \bigcup_{j\in J} \bigcap^{n(j)}_{r=1} {(q_{\alpha_{k},\beta_{r}} \circ \ell)}^{-1}(W_{\alpha_{k},\beta_{r}})                                                \\
		              & = \bigcup_{i\in I} \bigcap^{n(i)}_{k=1} \bigcup_{j\in J} \bigcap^{n(j)}_{r=1} \ell^{-1}q_{\alpha_{k},\beta_{r}}^{-1}(W_{\alpha_{k},\beta_{r}})                                                      \\
		              & = \ell^{-1}\left( \bigcup_{i\in I} \bigcap^{n(i)}_{k=1} \bigcup_{j\in J} \bigcap^{n(j)}_{r=1} q_{\alpha_{k},\beta_{r}}^{-1}(W_{\alpha_{k},\beta_{r}}) \right) \in \widetilde{\mathscr{T}}
	\end{align*}
	\endgroup

	Hence \( U \in \widetilde{\mathscr{T}} \), which means \( \mathscr{T} \subset \widetilde{\mathscr{T}} \).

	\bigskip
	Conversely, let \( U \in \widetilde{\mathscr{T}} \) then there exists \( W \) open in \( \prod_{\alpha,\beta} Z_{\alpha,\beta} \) such that \( U = \ell^{-1}(W) \).

	\( W \) can be written in terms of subbasic elements.
	\begingroup
	\allowdisplaybreaks%
	\begin{align*}
		U & = \ell^{-1}(W) = \ell^{-1}\left( \bigcup_{i\in I}\bigcap^{n(i)}_{r=1} q^{-1}_{\alpha_{r},\beta_{r}}(W_{\alpha_{r},\beta_{r}}) \right) \\
		  & = \bigcup_{i\in I}\bigcap^{n(i)}_{r=1} \ell^{-1}q^{-1}_{\alpha_{r},\beta_{r}}(W_{\alpha_{r},\beta_{r}})                               \\
		  & = \bigcup_{i\in I}\bigcap^{n(i)}_{r=1} {(q_{\alpha_{r},\beta_{r}}\circ \ell)}^{-1}(W_{\alpha_{r},\beta_{r}})                          \\
		  & = \bigcup_{i\in I}\bigcap^{n(i)}_{r=1} {(g_{\alpha_{r},\beta_{r}}\circ f_{\alpha_{r}})}^{-1}(W_{\alpha_{r},\beta_{r}})                \\
		  & = \bigcup_{i\in I}\bigcap^{n(i)}_{r=1} f_{\alpha_{r}}^{-1}(g_{\alpha_{r},\beta_{r}}^{-1}(W_{\alpha_{r},\beta_{r}})) \in \mathscr{T}
	\end{align*}
	\endgroup

	so \( \widetilde{\mathscr{T}} \subset \mathscr{T} \).

	Thus \( \mathscr{T} = \widetilde{\mathscr{T}} \).
\end{proof}

\begin{problem}{VI.3.3}
Let \(X\) have the projective limit topology determined by \( \left\{ Y_{\alpha}, f_{\alpha} \mid \alpha \in \mathscr{A} \right\} \). Prove: \( f: Z \to X \) is continuous if and only if each \( f_{\alpha} \circ f \) is continuous.
\end{problem}

\begin{proof}
	For each \( \alpha \), the map \( f_{\alpha}: X \to Y_{\alpha} \) is continuous.

	If \( f \) is continuous then each \( f_{\alpha} \circ f \) is continuous.

	Conversely, assume that each \( f_{\alpha} \circ f \) is continuous. Let \( U \) be an open set in \( X \).
	\[
		\bigcup_{\alpha} f_{\alpha}^{-1}(\mathscr{T}_{\alpha})
	\]

	is a subbasis for the projective limit topology on \( X \). Therefore \( U \) can be written as
	\[
		U = \bigcup_{\gamma} \bigcap^{n(\gamma)}_{k=1} f_{\gamma,k}^{-1}(U_{\gamma,k})
	\]

	in which \( U_{\gamma,k} \) is open in \( Y_{\gamma,k} \) so
	\begingroup
	\allowdisplaybreaks%
	\begin{align*}
		f^{-1}(U) & = f^{-1}\left( \bigcup_{\gamma} \bigcap^{n(\gamma)}_{k=1} f_{\gamma,k}^{-1}(U_{\gamma,k}) \right) \\
		          & = \bigcup_{\gamma} \bigcap^{n(\gamma)}_{k=1} f^{-1}(f_{\gamma,k}^{-1}(U_{\gamma,k}))              \\
		          & = \bigcup_{\gamma} \bigcap^{n(\gamma)}_{k=1} {(f_{\gamma,k} \circ f)}^{-1}(U_{\gamma,k})
	\end{align*}
	\endgroup

	\( {(f_{\gamma,k} \circ f)}^{-1}(U_{\gamma,k}) \) is open as \( U_{\gamma,k} \) is open in \( Y_{\gamma,k} \) and \( f_{\gamma,k} \circ f \) is continuous. Hence \( f^{-1}(U) \) is open (finite intersection then arbitrary union), so \( f \) is continuous.

	Thus \( f \) is continuous if and only if each \( f_{\alpha} \circ f \) is continuous.
\end{proof}

\section{Spaces with Equivalence Relations}

\begin{problem}{VI.4.1}
Let \( p: X \to X/R \) be an open (or closed) map, and \( B \subset X/R \) any subset. Show that \( B \) is homeomorphic to \( p^{-1}(B)/R_{0} \), where \( R_{0} \) is the restriction of \( R \) on \( p^{-1}(B) \).
\end{problem}

\begin{proof}
	Let \( q = p\vert_{p^{-1}(B)}: p^{-1}(B) \to B \) and \( r: p^{-1}(B) \to p^{-1}(B)/R_{0} \).

	\( p \) is an open (or closed) identification so \( q \) is an identification. Also, \( r \) is an identification.
	\[
		\begin{tikzcd}
			{p^{-1}(B)} && B \\
			\\
			{p^{-1}(B)/R_{0}}
			\arrow["q", from=1-1, to=1-3]
			\arrow["r"', from=1-1, to=3-1]
			\arrow["{rq^{-1}}", from=1-3, to=3-1]
			\arrow["{qr^{-1}}"{description}, shift left=3, from=3-1, to=1-3]
		\end{tikzcd}
	\]

	\( q \) is constant on each fiber of \( r \) and \( r \) is constant on each fiber of \( q \). Therefore the induced maps \( qr^{-1} \) and \( rq^{-1} \) are continuous. On the other hand, \( qr^{-1} \) and \( rq^{-1} \) are inverses of each other so they are homeomorphisms, which implies \( B \cong p^{-1}(B)/R_{0} \).
\end{proof}

\begin{problem}{VI.4.2}
Give an example showing that if \( A \subset X \) is not open or closed, then \( X - A \) need not be homeomorphic to the complement of \( [A] \) in \( X/A \).
\end{problem}

\begin{proof}
	Let \( X = \mathbb{R} \) and \( A = \mathbb{Q} \) then \( A \) is not open or closed in \( X \).

	The restriction of \( q: X \to X/A \) to \( q\vert_{X - A}: X - A \to X/A - \left\{ [A] \right\} \) is a continuous bijection. Let \( O = \openinterval{0, 1} \cap (\mathbb{R} - \mathbb{Q}) = \openinterval{0, 1} \cap (X - A) \) then \( O \) is open in \( X - A \). We will show that \( q(O) \) is not open in \( q(X - A) \).

	Assume that \( q(O) \) is open in \( q(X - A) \) then there exists an open set \( U \subset X/A \) such that \( q(O) = q(X - A) \cap U \).

	If \( U \) doesn't contain \( [A] \) then the preimage \( q^{-1}(U) \) doesn't contain any rational numbers, hence not open, which is a contradiction since \( U \) is open in \( X/A \) and \( q \) is continuous. Therefore \( U \) contains \( [A] \), so \( q^{-1}(U) \) contains \( \mathbb{Q} \), and
	\[
		q^{-1}(U) = q^{-1}(\left\{ [A] \right\} \cup q(O)) = q^{-1}(\left\{ [A] \right\}) \cup q^{-1}(q(O)) = \mathbb{Q} \cup O
	\]

	which is not open in \( X = \mathbb{R} \). This is a contradiction as \( q^{-1}(U) \) is simutaneously open and non-open.

	Thus \( q \) is not an open map, so it is not a homeomorphism.
\end{proof}

\begin{problem}{VI.4.3}
These problems will be much easier after studying compactness. (Use now Problem~\ref{problem:III.9.1}.)
\begin{enumerate}[label={(\alph*)}]
	\item In \( I^{2} \), let \( (0, y) \sim (1, y) \). Show \( I^{2}/R \cong \) the cylinder \( S^{1} \times I \).
	\item In \( I^{2} \), let \( (0, y) \sim (1, 1 - y) \). Show \( I^{2}/R \cong \) M\"{o}bius band.
	\item In \( I^{2} \), let \( \operatorname{Fr}(I^{2}) \sim (0, 0) \). Show \( I^{2}/R \cong S^{2} \).
	\item In \( I^{2} \), let \( (0, y) \sim (1, y), (x, 0) \sim (x, 1) \). Show \( I^{2}/R \cong \) the torus \( S^{1} \times S^{1} \).
\end{enumerate}
\end{problem}

\begin{proof}
	I have no idea other than using compactness and the closed map lemma.
\end{proof}

\begin{problem}{VI.4.4}
Let \( R \) be an equivalence relation in \( X \). For each \( A \subset X \) define \( C(A) = \left\{ x \in X \mid \exists a \in A: x R a \right\} \). Show that \( p(U) \) is open in \( X/R \) if and only if \( C(U) \) is open in \( X \).
\end{problem}

\begin{proof}
	According to the definition of \( C \), \( C(U) = p^{-1}(p(U)) \). Due to the definition of identification maps, \( p(U) \) is open in \( X/R \) if and only if \( p^{-1}(p(U)) \) is open in \( X \).

	Thus \( p(U) \) is open in \( X/R \) if and only if \( C(U) \) is open in \( X \).
\end{proof}

\begin{problem}{VI.4.5}
Let \( R, S \) be two equivalence relations in \( X \), and such that \( S \subset R \) (see Problem~\ref{problem:I.7.6}). Prove that \( (X/S)/(R/S) \cong X/R \).
\end{problem}

\begin{proof}
	Let \( p_{R}: X \to X/R \), \( p_{S}: X \to X/S \), and \( p: X/S \to (X/S)/(R/S) \) be projection maps.

	According to the transgression property, there is an induced continuous map \( q = p_{R}p_{S}^{-1} \) such that \( q \circ p_{S} = p_{R} \). According to Chapter I, section 7, there exists a unique map \( q \) that commutes the following diagram.
	\[
		\begin{tikzcd}
			X && X \\
			\\
			{X/S} && {X/R}
			\arrow["1", from=1-1, to=1-3]
			\arrow["{p_{S}}"', from=1-1, to=3-1]
			\arrow["{p_{R}}", from=1-3, to=3-3]
			\arrow["q"', from=3-1, to=3-3]
		\end{tikzcd}
	\]

	For each subset \( U \subset X/R \), \( p_{R}^{-1}(U) \) is \( p_{S} \)-saturated because \( S \subset R \). If \( U \) is open then \( p_{R}^{-1}(U) \) is open in \( X \) and \( p_{S} \)-saturated so \( p_{S}(p_{R}^{-1}(U)) \) is open in \( X/S \). Conversely, if \( q^{-1}(U) \) is open in \( X/S \) then \( p_{S}^{-1}(q^{-1}(U)) = {(q \circ p_{S})}^{-1}(U) = p_{R}^{-1}(U) \) is open in \( X \), which means \( U \) is open in \( X/R \). Hence \( q \) is an identification.

	According to the definition of \( R/S \) (see Problem~\ref{problem:I.7.5} and~\ref{problem:I.7.6})
	\[
		p(Sa) = p(Sb) \iff (Sa)R/S(Sb) \iff q(Sa) = q(Sb)
	\]

	which means \( q \) is constant on each fiber of \( p \) and \( p \) is constant on each fiber of \( q \)
	\[
		\begin{tikzcd}
			{X/S} && {X/R} \\
			\\
			{(X/S)/(R/S)}
			\arrow["q", from=1-1, to=1-3]
			\arrow["p"', from=1-1, to=3-1]
			\arrow["\cong"', from=3-1, to=1-3]
		\end{tikzcd}
	\]

	so \( (X/S)/(R/S) \cong X/R \).
\end{proof}

\begin{problem}{VI.4.6}
Let \( 0 \) be the origin in \( E^{3} \). In \( E^{3} - \left\{ 0 \right\} \), define \( x R y \) if \( x \) and \( y \) lie on a line through the origin. Show that \( R \) is an equivalence relation; \( (E^{3} - \left\{ 0 \right\})/R \) is called the projective plane \( P^{2} \). Call \textit{line} in \( P^{2} \) any set \( A \) such that \( p^{-1}(A) \) is a plane in \( E^{3} - \left\{ 0 \right\} \) going through the origin. Show that a line in \( P^{2} \) is homeomorphic to \( S^{1} \).
\end{problem}

\begin{proof}
	We will show that \( p: E^{3} - \left\{ 0 \right\} \to (E^{3} - \left\{ 0 \right\})/R \) an open map. Let \( U \) be an open set in \( E^{3} - \left\{ 0 \right\} \). The image \( p(U) \) is open in \( P^{2} \) if and only if \( p^{-1}(p(U)) \) is open in \( E^{3} - \left\{ 0 \right\} \).
	\[
		p^{-1}(p(U)) = \bigcup_{t \in E^{1} - \left\{ 0 \right\}} \left\{ tu \mid u \in U \right\}
	\]

	is open as \( \left\{ tu \mid u \in U \right\} \) is homeomorphic to \( U \) for each \( t \in E^{1} - \left\{ 0 \right\} \). Hence \( p \) is an open map.

	Let \( A \) be any line in \( P^{2} \) then \( p^{-1}(A) \) is a plane in \( E^{3} - \left\{ 0 \right\} \) going through the origin. Because \( p \) is an open map, \( p\vert_{p^{-1}(A)}: p^{-1}(A) \to A \) is an identification.

	Let's define \( f: E^{3} - \left\{ 0 \right\} \to E^{3} - \left\{ 0 \right\} \) by \( f(x) = x/\left\vert x \right\vert \) is continuous and open, so \( f: p^{-1}(A) \to f(p^{-1}(A)) \) is an identification. Moreover, \( f(p^{-1}(A)) \cong S^{1} \). The map \( g: S^{1} \to S^{1}/\left\{ x \sim -x \right\} \) is an open continuous surjection.

	The identifications \( g \circ (f\vert_{p^{-1}(A)}) \) and \( p\vert_{p^{-1}(A)} \) are constant on each fiber of the other map so \( A \) and \( S^{1}/\left\{ x \sim -x \right\} \) are homeomorphic.

	The homeomorphism of \( S^{1} \) and \( S^{1}/\left\{ x \sim -x \right\} \) follows from the map \( z \to z^{2} \).

	Thus \( A \cong S^{1} \).
\end{proof}

\begin{problem}{VI.4.7}
Let \( V^{2} = \left\{ x \in E^{2} \mid \left\vert x \right\vert \le 1 \right\} \). Generate an equivalence relation by \( x R y \) if \( \left\vert x \right\vert = \left\vert y \right\vert = 1 \), and \( x, y \) are diametrically opposite. Show \( V^{2}/R \) is homeomorphic to \( P^{2} \).
\end{problem}

\begin{proof}
	\( P^{2} \) is homeomorphic to \( S^{2}/\left\{ x \sim -x \right\} \).

	Let \( US^{2} \) be the upper hemisphere (\( x_{3} \ge 0 \)) then \( V^{2} \cong US^{2} \) and \( US^{2}/\left\{ x \sim -x \right\} \cong V^{2}/R \).

	Thus \( P^{2} \cong S^{2}/\left\{ x \sim -x \right\} \cong US^{2}/\left\{ x \sim -x \right\} \cong V^{2}/R \).
\end{proof}

\section{Cones and Suspensions}

\begin{problem}{VI.5.1}
If \( A \subset X \) is closed, prove that \( TA \) is homeomorphic to a closed subspace of \( TX \).
\end{problem}

\begin{proof}
	Denote \( q: X \times I \to TX \) and \( p: A \times I \to TA \).

	We will show that \( q(A \times I) \) is closed in \( TX \).
	\[
		q^{-1}(q(A \times I)) = (A \times I) \cup (X \times \left\{ 1 \right\})
	\]

	and \( A \times I, X \times \left\{ 1 \right\} \) are closed in \( X \times I \) so \( q(A \times I) \) is closed in \( TX \).

	Let \( p(a, t) \) be a point in \( TA \) then \( q(p^{-1}(p(a, t))) = \left\{ q(a, t) \right\} \), hence we can define a map \( f: TA \to q(A \times I) \) by \( f(p(a, t)) = q(a, t) \). This map \( f \) is bijective and \( f \circ p = q, p = f^{-1} \circ q \). As \( f^{-1} \circ q \) is continuous, we deduce that \( f^{-1} \) is continuous (Theorem 3.1). Besides, \( q \) is constant on each fiber of \( p \) so \( f \) is continuous (Theorem 3.2 on Transgression). Therefore \( f \) is a homeomorphism.

	Thus \( TA \) is homeomorphic to \( q(A\times I) \), which is a closed subspace of \( TX \).
\end{proof}

\begin{problem}{VI.5.2}
Let \( i: \operatorname{Int}(I) \to I \) be the inclusion map. Show that the map \( Ti: T[\operatorname{Int}(I)] \to TI \) is not an embedding.
\end{problem}

\begin{proof}
	% TODO
\end{proof}

\section{Attaching of Spaces}

\section{The Relation \(K(f)\) of Continuous Maps}

\section{Weak Topologies}
