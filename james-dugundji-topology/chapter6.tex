\chapter{Identification Topology; Weak Topology}

\section{Identification Topology}

\begin{problem}{VI.1.1}\label{problem:VI.1.1}
Reversing the situation treated in the text, let \(X\) be a set, \( (Y, \mathscr{T}) \) a space, and \( p: X \to Y \) a surjective map. Prove:
\begin{enumerate}[label={(\alph*)}]
	\item \( \mathscr{T}_{X} = \left\{ p^{-1}(U) \mid U \text{ open in } Y \right\} \) is a topology in \( X \).
	\item \( p: (X, \mathscr{T}_{X}) \to (Y, \mathscr{T}) \) is continuous, open, and closed.
\end{enumerate}
\end{problem}

\begin{proof}
	\begin{enumerate}[label={(\alph*)}]
		\item \( \mathscr{T}_{X} \) contains \( \varnothing, X \) as \( p^{-1}(\varnothing) = \varnothing \) and \( p^{-1}(Y) = X \).

		      If \( {\left\{ U_{\alpha} \right\}}_{\alpha\in\mathscr{A}} \) is a collection of open sets in \( Y \), then
		      \[
			      \bigcup_{\alpha\in\mathscr{A}} p^{-1}(U_{\alpha}) = p^{-1}\left(\bigcup_{\alpha\in\mathscr{A}} U_{\alpha}\right)
		      \]

		      so \( \mathscr{T}_{X} \) is closed under arbitrary union.

		      If \( U_{1}, \ldots, U_{n} \) are open sets in \( Y \) then
		      \[
			      \bigcap^{n}_{i=1} p^{-1}(U_{i}) = p^{-1}\left(\bigcap^{n}_{i=1} U_{i}\right)
		      \]

		      so \( \mathscr{T}_{X} \) is closed under finite intersection.

		      Hence \( \mathscr{T}_{X} \) is a topology in \( X \).
		\item For each open set \( U \) in \( Y \), \( p^{-1}(U) \in \mathscr{T}_{X} \) so \( p \) is continuous.

		      Let \( V \) be an open set in \( X \). Then there is an open set \( U \) in \( Y \) such that \( V = p^{-1}(U) \). Hence \( p(V) = pp^{-1}(U) = U \) because \( p \) is surjective. So \( p \) is an open map.

		      Let \( W \) be a closed set in \( X \) then \( X - W \) is open and there exists an open set \( U \) in \( Y \) such that \( X - W = p^{-1}(U) \). Therefore
		      \[
			      W = X - p^{-1}(U) = p^{-1}(Y) - p^{-1}(U) = p^{-1}(Y - U)
		      \]

		      which implies that \( p(W) = pp^{-1}(Y - U) = Y - U \), which is closed in \( Y \). So \( p \) is a closed map.

		      Thus \( p \) is a continuous, open, and closed map.
	\end{enumerate}
\end{proof}

\begin{problem}{VI.1.2}
For each \( \alpha \in \mathscr{A} \), let \( p_{\alpha}: X_{\alpha} \to Y_{\alpha} \) be a continuous, open surjection. Show that \( \prod_{\alpha} p_{\alpha}: \prod_{\alpha} X_{\alpha} \to \prod_{\alpha} Y_{\alpha} \) is an identification.
\end{problem}

\begin{proof}
	For the sake of brevity, denote \( p = \prod_{\alpha} p_{\alpha} \). By definition, \( p \) is surjective.

	\( p_{Y_{\alpha}} \circ p \) is continuous for each projection \( p_{Y_{\alpha}}: \prod_{\alpha} Y_{\alpha} \to Y_{\alpha} \) so \( p \) is continuous.

	Let \( \prod_{\alpha} U_{\alpha} \) be a basic open set in \( \prod_{\alpha} X_{\alpha} \), which means \( U_{\alpha} = X_{\alpha} \) for all but finitely many \( \alpha \) and \( U_{\alpha} \) is open in \( X_{\alpha} \) for every \( \alpha \). Because \( p_{\alpha} \) is an open surjection for each \( \alpha \), the image
	\[
		p\left( \prod_{\alpha} U_{\alpha} \right) = \prod_{\alpha} p_{\alpha}(U_{\alpha})
	\]

	is open in \( \prod_{\alpha} Y_{\alpha} \) as \( p_{\alpha}(U_{\alpha}) \) is open in \( Y_{\alpha} \) and \( p_{\alpha}(U_{\alpha}) = Y_{\alpha} \) for all but finitely many \( \alpha \). Hence \( p \) is an open map.

	\( p \) is a continuous, open surjection so \( p \) is an identification.
\end{proof}

\begin{problem}{VI.1.3}
Let \( X \) be a space and \( A \subset X \) a subspace. Assume that there exists a continuous \( r: X \to A \) such that \( r\vert_{A} = 1_{A} \) (such a map is called a \textit{retraction} of \(X\) onto \(A\)). Show that \( r \) is an identification.
\end{problem}

\begin{proof}
	By definition, \( r \) is continuous and surjective. Let \( f: A \xhookrightarrow{} X \) be the inclusion map.

	\( f \) is continuous and \( r \circ f = 1_{A} \) so \( r \) is an identification.
\end{proof}

\begin{problem}{VI.1.4}\label{problem:VI.1.4}
Let \( X \) be any set. Given any family \( \left\{ (Y_{\alpha}, \mathscr{T}_{\alpha}), f_{\alpha} \mid \alpha \in \mathscr{A} \right\} \) of spaces and maps \( f_{\alpha}: X \to Y_{\alpha} \), the ``projective limit topology of \(X\) determined by this family'' is \( \bigvee_{\alpha} f_{\alpha}^{-1}(\mathscr{T}_{\alpha}) \) (see Problem~\ref{problem:III.3.8}). Prove:
\begin{enumerate}[label={(\alph*)}]
	\item If \( j: X \to \prod_{\alpha} Y_{\alpha} \) is the map \( j(x) = \left\{ f_{\alpha}(x) \right\} \), then \( \bigvee_{\alpha} f_{\alpha}^{-1}(\mathscr{T}_{\alpha}) \) is the topology in \(X\) determined by \(j\) as in Problem~\ref{problem:VI.1.1}.
	\item If whenever \( x \ne x^{\prime} \), there is some index \( \alpha \) such that \( f_{\alpha}(x) \ne f_{\alpha}(x^{\prime}) \), then \( j \) is an embedding.
\end{enumerate}
\end{problem}

\begin{proof}
	\begin{enumerate}[label={(\alph*)}]
		\item Let \( \prod_{\alpha} U_{\alpha} \) be a subbasic open set in \( \prod_{\alpha} Y_{\alpha} \) then \( U_{\alpha} = Y_{\alpha} \) for every \( \alpha \) but one \( \beta \in \mathscr{A} \).
		      \[
			      j^{-1}\left( \prod_{\alpha} U_{\alpha} \right) = \bigcap_{\alpha} f_{\alpha}^{-1}(U_{\alpha}) = f_{\beta}^{-1}(U_{\beta}) \in \bigvee_{\alpha} f_{\alpha}^{-1}(\mathscr{T}_{\alpha})
		      \]

		      Hence \( j \) is continuous, which means if \( j^{-1}(U) \) is open whenever \( U \subset \prod_{\alpha} Y_{\alpha} \) is open.

		      Let \( V \) be an open set in \( X \). According to the definition of the topology \( \bigvee_{\alpha} f_{\alpha}^{-1}(\mathscr{T}_{\alpha}) \), \( V \) can be written as a union of finite intersection of elements in \( \bigcup_{\alpha} f_{\alpha}^{-1}(\mathscr{T}_{\alpha}) \), which means
		      \[
			      V = \bigcup_{i\in I} V_{i}
		      \]

		      where each \( V_{i} \) is a finite intersection of elements in \( \bigcup_{\alpha} f_{\alpha}^{-1}(\mathscr{T}_{\alpha}) \).
		      \[
			      V_{i} = \bigcap^{n_{i}}_{k=1} f_{\alpha_{k}}^{-1}(U_{\alpha_{k}}) = \bigcap^{n_{i}}_{k=1} j^{-1}\left( U_{\alpha_{k}} \times \prod_{\alpha \ne \alpha_{k}} Y_{\alpha} \right) = j^{-1}\left( \bigcap^{n_{i}}_{k=1} U_{\alpha_{k}} \times \prod_{\alpha \ne \alpha_{k}} Y_{\alpha} \right) = j^{-1}(W_{i})
		      \]

		      where \( U_{\alpha_{k}} \) is open in \( Y_{\alpha_{k}} \). So
		      \[
			      V = \bigcup_{i\in I} j^{-1}(W_{i}) = j^{-1}\left( \bigcup_{i\in I} W_{i} \right)
		      \]

		      which means \( V \) is the preimage of an open set in \( \prod_{\alpha} Y_{\alpha} \).

		      Thus \( \bigvee_{\alpha} f_{\alpha}^{-1}(\mathscr{T}_{\alpha}) \) is the same as the topology in \( X \) determined by \( j \) as in Problem~\ref{problem:VI.1.1}.
		\item According to Problem~\ref{problem:VI.1.1}, \( j \) is continuous, open, and closed.

		      Whenever \( x \ne x^{\prime} \), there is some index \( \alpha \) such that \( f_{\alpha}(x) \ne f_{\alpha}(x^{\prime}) \), then \( j(x) \ne j(x^{\prime}) \), which implies \( j \) is injective.

		      A continuous, open, injective map is an embedding so \( j \) is an embedding.
	\end{enumerate}
\end{proof}

\section{Subspaces}

\begin{problem}{VI.2.1}
Let \(X\) have the projective limit topology (Problem~\ref{problem:VI.1.4}) determined by
\[
	\left\{ Y_{\alpha}, f_{\alpha} \mid \alpha \in \mathscr{A} \right\}
\]

and let \( A \subset X \). Prove: The subspace topology of \(A\) is the projective limit topology determined by the maps \( f_{\alpha}\vert_{A} \).
\end{problem}

\begin{proof}
	The projective limit topology on \( A \) determined by the maps \( f_{\alpha}\vert_{A} \) has subbasis
	\[
		\bigcup_{\alpha} {(f_{\alpha}\vert_{A})}^{-1}(\mathscr{T}_{\alpha})
	\]

	Let \( V \) be an open set in \( A \) (with the projective limit topology) then
	\[
		V = \bigcup_{i\in I} V_{i}
	\]

	in which each \( V_{i} \) is the intersection of finitely many elements of \( \bigcup_{\alpha} {(f_{\alpha}\vert_{A})}^{-1}(\mathscr{T}_{\alpha}) \). So there exist \( \alpha_{i_{1}}, \ldots, \alpha_{i_{n(i)}} \in \mathscr{A} \) such that
	\[
		V_{i} = \bigcap^{n(i)}_{k=1} {(f_{\alpha_{k}}\vert_{A})}^{-1}(U_{\alpha_{k}})
	\]

	Hence
	\begingroup
	\allowdisplaybreaks%
	\begin{align*}
		V_{i} & = \bigcap^{n(i)}_{k=1} (A \cap f_{\alpha_{k}}^{-1}(U_{\alpha_{k}}))                                                             \\
		      & = A \cap \bigcap^{n(i)}_{k=1} f_{\alpha_{k}}^{-1}(U_{\alpha_{k}})                                                               \\
		      & = A \cap \bigcap^{n(i)}_{k=1} j^{-1}\left( U_{\alpha_{k}} \times \prod_{\alpha \ne \alpha_{k}} Y_{\alpha} \right)               \\
		      & = A \cap j^{-1}\left( \bigcap^{n(i)}_{k=1} \left( U_{\alpha_{k}} \times \prod_{\alpha\ne\alpha_{k}} Y_{\alpha} \right) \right).
	\end{align*}
	\endgroup

	Therefore
	\begingroup
	\allowdisplaybreaks%
	\begin{align*}
		V & = A \cap \bigcup_{i\in I} j^{-1}\left( \bigcap^{n(i)}_{k=1} \left( U_{\alpha_{k}} \times \prod_{\alpha\ne\alpha_{k}} Y_{\alpha} \right) \right) \\
		  & = A \cap j^{-1}\left( \bigcup_{i\in J} \bigcap^{n(i)}_{k=1} \left( U_{\alpha_{k}} \times \prod_{\alpha\ne\alpha_{k}} Y_{\alpha} \right) \right)
	\end{align*}
	\endgroup

	Hence \( V \) is in the subspace topology of \( A \).

	Conversely, one can show that if \( V \) is in the subspace topology of \( A \), then \( V \) is also in the projective limit topology on \( A \) determinded by the maps \( f_{\alpha}\vert_{A} \).

	Thus the projective limit topology on \( A \) determinded by the maps \( f_{\alpha}\vert_{A} \) and the subspace topology on \( A \) coincide.
\end{proof}

\section{General Theorems}

\begin{problem}{VI.3.1}
Let \( p: X \to Y \) be a continuous open (or closed) surjection, and assume that each fiber \( p^{-1}(y) \) is connected. For any \( F \subset Y \), show that \( F \) is connected if and only if \( p^{-1}(F) \) is connected.
\end{problem}

\begin{proof}
	By Proposition 2.1, \( p\vert_{p^{-1}(F)}: p^{-1}(F) \to F \) is an identification because \( p \) is an identification which is also an open (or closed) map. Denote \( q = p\vert_{p^{-1}(F)} \).

	If \( p^{-1}(F) \) is connected then \( F = p(p^{-1}(F)) \) is connected, as \( p \) is a continuous surjection.

	If \( p^{-1}(F) \) is not connected then there is a continuous surjection \( h: p^{-1}(F) \to 2 \). As each fiber of \( q \) (each fiber of \(q \) is a fiber of \(p\)) is connected, the restriction of \( h \) to each fiber is a constant map. Therefore \( hq^{-1}: F \to 2 \) is a continuous surjection, according to the transgression property, which means \( F \) is not connected.
\end{proof}

\begin{problem}{VI.3.2}
Let \( X \) have the projective limit topology \( \mathscr{T} \) determined by the family
\[
	\left\{ (Y_{\alpha}, \mathscr{T}_{\alpha}), f_{\alpha} \mid \alpha \in \mathscr{A} \right\}
\]

Assume that each \( \mathscr{T}_{\alpha} \) is the projective limit topology determined by a family
\[
	\left\{ (Z_{\alpha, \beta}, \mathscr{T}_{\alpha,\beta}), g_{\alpha,\beta} \mid \beta \in \mathscr{B} \right\}.
\]

Prove: \( \mathscr{T} \) is the projective limit topology determined by
\[
	\left\{ (Z_{\alpha,\beta}, \mathscr{T}_{\alpha,\beta}), g_{\alpha,\beta} \circ f_{\alpha} \mid (\alpha, \beta) \in \mathscr{A} \times \mathscr{B} \right\}.
\]
\end{problem}

\begin{proof}
	Denote by \( \widetilde{\mathscr{T}} \) the projective limit topology determined by
	\[
		\left\{ (Z_{\alpha,\beta}, \mathscr{T}_{\alpha,\beta}), g_{\alpha,\beta} \circ f_{\alpha} \mid (\alpha, \beta) \in \mathscr{A} \times \mathscr{B} \right\}.
	\]

	Let \( h: X \to \prod_{\alpha} Y_{\alpha} \) be the map \( h(x) = {\left\{ f_{\alpha}(x) \right\}}_{\alpha} \) then
	\[
		\mathscr{T} = \left\{ h^{-1}(U) \mid U \text{ open in } \prod_{\alpha}Y_{\alpha} \right\}
	\]

	according to Problem~\ref{problem:VI.1.4}.

	For each \( \alpha \), let \( h_{\alpha}: Y_{\alpha} \to \prod_{\beta} Z_{\alpha,\beta} \) be the map \( h_{\alpha}(x) = {\left\{ g_{\alpha,\beta}(x) \right\}}_{\beta} \) then
	\[
		\mathscr{T}_{\alpha} = \left\{ h_{\alpha}^{-1}(U) \mid U \text{ open in } \prod_{\beta} Z_{\alpha,\beta} \right\}
	\]

	according to Problem~\ref{problem:VI.1.4}.

	Let \( \ell: (X, \widetilde{\mathscr{T}}) \to \prod_{\alpha,\beta} Z_{\alpha,\beta} \) be the map \( \ell(x) = {\left\{ g_{\alpha,\beta}(f_{\alpha}(x)) \right\}}_{\alpha,\beta} \) then
	\[
		\widetilde{\mathscr{T}} = \left\{ \ell^{-1}(U) \mid U \text{ open in } \prod_{\alpha,\beta} Z_{\alpha,\beta} \right\}
	\]

	according to Problem~\ref{problem:VI.1.4}.

	Note that \( f_{\alpha} = p_{\alpha} \circ h \) and \( g_{\alpha,\beta} = p_{\alpha,\beta} \circ h_{\alpha} \) in which \( p_{\alpha}: \prod_{\alpha} Y_{\alpha} \to Y_{\alpha} \) and \( p_{\alpha,\beta}: \prod_{\beta} Z_{\alpha,\beta} \to Z_{\alpha,\beta} \) are projection maps. Denote by \( q_{\alpha,\beta} \) the projection map \( \prod_{\alpha,\beta} W_{\alpha,\beta} \to W_{\alpha,\beta} \).
	\[
		\begin{tikzcd}
			&& {\prod_{\alpha} Y_{\alpha}} \\
			\\
			X && {Y_{\alpha}} && {\prod_{\beta}Z_{\alpha,\beta}} && {Z_{\alpha,\beta}}
			\arrow["{p_{\alpha}}", from=1-3, to=3-3]
			\arrow["h", from=3-1, to=1-3]
			\arrow["{f_{\alpha}}"', from=3-1, to=3-3]
			\arrow["{h_{\alpha}}"', from=3-3, to=3-5]
			\arrow["{g_{\alpha,\beta}}"', bend right, from=3-3, to=3-7]
			\arrow["{p_{\alpha,\beta}}"', from=3-5, to=3-7]
		\end{tikzcd}
	\]

	\[
		\begin{tikzcd}
			X && {\prod_{\alpha,\beta} Z_{\alpha,\beta}} && {Z_{\alpha,\beta}}
			\arrow["\ell", from=1-1, to=1-3]
			\arrow["{g_{\alpha,\beta} \circ f_{\alpha}}"', bend right, from=1-1, to=1-5]
			\arrow["{q_{\alpha,\beta}}", from=1-3, to=1-5]
		\end{tikzcd}
	\]

	Let \( U \in \mathscr{T} \) then there exists \( V \) open in \( \prod_{\alpha} Y_{\alpha} \) such that \( U = h^{-1}(V) \) (see Problem~\ref{problem:VI.1.4} and~\ref{problem:VI.1.1}). One can write \( V \) in terms of subbasic elements as follows
	\[
		V = \bigcup_{i\in I} \bigcap^{n(i)}_{k=1} p_{\alpha_{k}}^{-1}(V_{\alpha_{k}})
	\]

	in which \( V_{\alpha_{k}} \) is open in \( Y_{\alpha_{k}} \).

	As \( \mathscr{T}_{\alpha} \) is the projective limit topology on \( Y_{\alpha} \) determined by the maps \( g_{\alpha,\beta}: Y_{\alpha} \to Z_{\alpha,\beta} \), there is an open set \( W_{\alpha_{k}} \) in \( \prod_{\beta} Z_{\alpha_{k},\beta} \) such that \( V_{\alpha_{k}} = h_{\alpha}^{-1}(W_{\alpha_{k}}) \). The open set \( W_{\alpha_{k}} \) can be written in terms of subbasic elements as follows
	\[
		W_{\alpha_{k}} = \bigcup_{j \in J} \bigcap^{n(j)}_{r=1} p_{\alpha_{k},\beta_{r}}^{-1}(W_{\alpha_{k}, \beta_{r}})
	\]

	in which \( W_{\alpha_{k}, \beta_{r}} \) is open in \( Z_{\alpha_{k}, \beta_{r}} \).
	\begingroup
	\allowdisplaybreaks%
	\begin{align*}
		V             & = \bigcup_{i\in I} \bigcap^{n(i)}_{k=1} p_{\alpha_{k}}^{-1}(V_{\alpha_{k}})                                                                                                                         \\
		              & = \bigcup_{i\in I} \bigcap^{n(i)}_{k=1} p_{\alpha_{k}}^{-1}\left( h^{-1}_{\alpha_{k}}(W_{\alpha_{k}}) \right)                                                                                       \\
		              & = \bigcup_{i\in I} \bigcap^{n(i)}_{k=1} {(h_{\alpha_{k}} \circ p_{\alpha_{k}})}^{-1}(W_{\alpha_{k}})                                                                                                \\
		              & = \bigcup_{i\in I} \bigcap^{n(i)}_{k=1} {(h_{\alpha_{k}} \circ p_{\alpha_{k}})}^{-1} \left( \bigcup_{j \in J} \bigcap^{n(j)}_{r=1} p_{\alpha_{k},\beta_{r}}^{-1}(W_{\alpha_{k}, \beta_{r}}) \right) \\
		              & = \bigcup_{i\in I} \bigcap^{n(i)}_{k=1} \bigcup_{j\in J} \bigcap^{n(j)}_{r=1} {(p_{\alpha_{k},\beta_{r}} \circ h_{\alpha_{k}} \circ p_{\alpha_{k}})}^{-1}(W_{\alpha_{k},\beta_{r}})                 \\
		U = h^{-1}(V) & = \bigcup_{i\in I} \bigcap^{n(i)}_{k=1} \bigcup_{j\in J} \bigcap^{n(j)}_{r=1} {(p_{\alpha_{k},\beta_{r}} \circ h_{\alpha_{k}} \circ p_{\alpha_{k}} \circ h)}^{-1}(W_{\alpha_{k},\beta_{r}})         \\
		              & = \bigcup_{i\in I} \bigcap^{n(i)}_{k=1} \bigcup_{j\in J} \bigcap^{n(j)}_{r=1} {(g_{\alpha_{k},\beta_{r}} \circ f_{\alpha_{k}})}^{-1}(W_{\alpha_{k},\beta_{r}})                                      \\
		              & = \bigcup_{i\in I} \bigcap^{n(i)}_{k=1} \bigcup_{j\in J} \bigcap^{n(j)}_{r=1} {(q_{\alpha_{k},\beta_{r}} \circ \ell)}^{-1}(W_{\alpha_{k},\beta_{r}})                                                \\
		              & = \bigcup_{i\in I} \bigcap^{n(i)}_{k=1} \bigcup_{j\in J} \bigcap^{n(j)}_{r=1} \ell^{-1}q_{\alpha_{k},\beta_{r}}^{-1}(W_{\alpha_{k},\beta_{r}})                                                      \\
		              & = \ell^{-1}\left( \bigcup_{i\in I} \bigcap^{n(i)}_{k=1} \bigcup_{j\in J} \bigcap^{n(j)}_{r=1} q_{\alpha_{k},\beta_{r}}^{-1}(W_{\alpha_{k},\beta_{r}}) \right) \in \widetilde{\mathscr{T}}
	\end{align*}
	\endgroup

	Hence \( U \in \widetilde{\mathscr{T}} \), which means \( \mathscr{T} \subset \widetilde{\mathscr{T}} \).

	\bigskip
	Conversely, let \( U \in \widetilde{\mathscr{T}} \) then there exists \( W \) open in \( \prod_{\alpha,\beta} Z_{\alpha,\beta} \) such that \( U = \ell^{-1}(W) \).

	\( W \) can be written in terms of subbasic elements.
	\begingroup
	\allowdisplaybreaks%
	\begin{align*}
		U & = \ell^{-1}(W) = \ell^{-1}\left( \bigcup_{i\in I}\bigcap^{n(i)}_{r=1} q^{-1}_{\alpha_{r},\beta_{r}}(W_{\alpha_{r},\beta_{r}}) \right) \\
		  & = \bigcup_{i\in I}\bigcap^{n(i)}_{r=1} \ell^{-1}q^{-1}_{\alpha_{r},\beta_{r}}(W_{\alpha_{r},\beta_{r}})                               \\
		  & = \bigcup_{i\in I}\bigcap^{n(i)}_{r=1} {(q_{\alpha_{r},\beta_{r}}\circ \ell)}^{-1}(W_{\alpha_{r},\beta_{r}})                          \\
		  & = \bigcup_{i\in I}\bigcap^{n(i)}_{r=1} {(g_{\alpha_{r},\beta_{r}}\circ f_{\alpha_{r}})}^{-1}(W_{\alpha_{r},\beta_{r}})                \\
		  & = \bigcup_{i\in I}\bigcap^{n(i)}_{r=1} f_{\alpha_{r}}^{-1}(g_{\alpha_{r},\beta_{r}}^{-1}(W_{\alpha_{r},\beta_{r}})) \in \mathscr{T}
	\end{align*}
	\endgroup

	so \( \widetilde{\mathscr{T}} \subset \mathscr{T} \).

	Thus \( \mathscr{T} = \widetilde{\mathscr{T}} \).
\end{proof}

\begin{problem}{VI.3.3}
Let \(X\) have the projective limit topology determined by \( \left\{ Y_{\alpha}, f_{\alpha} \mid \alpha \in \mathscr{A} \right\} \). Prove: \( f: Z \to X \) is continuous if and only if each \( f_{\alpha} \circ f \) is continuous.
\end{problem}

\begin{proof}
	For each \( \alpha \), the map \( f_{\alpha}: X \to Y_{\alpha} \) is continuous.

	If \( f \) is continuous then each \( f_{\alpha} \circ f \) is continuous.

	Conversely, assume that each \( f_{\alpha} \circ f \) is continuous. Let \( U \) be an open set in \( X \).
	\[
		\bigcup_{\alpha} f_{\alpha}^{-1}(\mathscr{T}_{\alpha})
	\]

	is a subbasis for the projective limit topology on \( X \). Therefore \( U \) can be written as
	\[
		U = \bigcup_{\gamma} \bigcap^{n(\gamma)}_{k=1} f_{\gamma,k}^{-1}(U_{\gamma,k})
	\]

	in which \( U_{\gamma,k} \) is open in \( Y_{\gamma,k} \) so
	\begingroup
	\allowdisplaybreaks%
	\begin{align*}
		f^{-1}(U) & = f^{-1}\left( \bigcup_{\gamma} \bigcap^{n(\gamma)}_{k=1} f_{\gamma,k}^{-1}(U_{\gamma,k}) \right) \\
		          & = \bigcup_{\gamma} \bigcap^{n(\gamma)}_{k=1} f^{-1}(f_{\gamma,k}^{-1}(U_{\gamma,k}))              \\
		          & = \bigcup_{\gamma} \bigcap^{n(\gamma)}_{k=1} {(f_{\gamma,k} \circ f)}^{-1}(U_{\gamma,k})
	\end{align*}
	\endgroup

	\( {(f_{\gamma,k} \circ f)}^{-1}(U_{\gamma,k}) \) is open as \( U_{\gamma,k} \) is open in \( Y_{\gamma,k} \) and \( f_{\gamma,k} \circ f \) is continuous. Hence \( f^{-1}(U) \) is open (finite intersection then arbitrary union), so \( f \) is continuous.

	Thus \( f \) is continuous if and only if each \( f_{\alpha} \circ f \) is continuous.
\end{proof}

\section{Spaces with Equivalence Relations}

\section{Cones and Suspensions}

\section{Attaching of Spaces}

\section{The Relation \(K(f)\) of Continuous Maps}

\section{Weak Topologies}
