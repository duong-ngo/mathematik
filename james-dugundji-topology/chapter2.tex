\chapter{Ordinals and Cardinals}

\section{Orderings}

\begin{problem}{II.1.1}
Let \( \Delta \) be the diagonal in \( A\times A \). Show that \( R \subset A\times A \) is a preorder if and only if \( \Delta \subset R \) and \( R \circ R = R \).
\end{problem}

\begin{proof}
	Suppose that \( R \) is a preorder. For every \( x \in A \), one has \( (x, x) \in \Delta \) and \( (x, x) \in R \), so \( \Delta \subset R \). Moreover
	\begin{itemize}
		\item If \( x(R \circ R) y \) then there exists \( z \in A \)  such that \( xRz \) and \( zRy \). From the transitivity of \( R \), we deduce that \( xRy \).
		\item If \( xRy \) then \( xRx \) (reflexivity of \(R\)) and \( xRy \), which implies that \( x(R\circ R) y \).
	\end{itemize}

	So \( R \circ R = R \).

	Conversely, suppose that \( \Delta \subset R \) and \( R \circ R = R \). The condition \( \Delta \subset R \) precisely means \( R \) is reflexive. The condition \( R \circ R = R \) implies that \( R \) is transitive. Therefore \( R \) is a preorder.
\end{proof}

\begin{problem}{II.1.2}
In \( \mathbb{Z}^{+} \), define \( m \prec n \) if \( n \) divides \( m \). Show that this is a partial ordering, that every chain has an upper bound, and determine the set of maximal elements.
\end{problem}

\begin{proof}
	For every \( n \in \mathbb{Z}^{+} \), \( n \) divides \( n \) so \( n \prec n \), which means \( \prec \) is reflexive. If \( a \prec b \) and \( b \prec c \) then there exist \( p, q \in \mathbb{Z}^{+} \) such that \( a = pb, b = qc \) so \( a = (pq)c \), which implies \( a \prec c \). Hence \( \prec \) is transitive. Moreover, if \( a \prec b \) and \( b \prec a \), there exist \( p, q \in \mathbb{Z}^{+} \) such that \( a = pb, b = qa \), so \( pq = 1 \), which means \( p = q = 1 \), so \( a = b \), which implies the antisymmetric of \( \prec \). Thus \( \prec \) is a partial ordering.

	Suppose \( A \) is a chain of \( \mathbb{Z}^{+} \). For every \( a \in A \), \( 1 \) divides \( a \) so \( 1 \) is an upper bound of \( A \). Therefore every chain of \( (\mathbb{Z}^{+}, \prec) \) has an upper bound.

	Because \( \mathbb{Z}^{+} \) with the usual order relation \( \le \) is well-ordering then \( A \) has a first element \( a_{0} \). Since \( (A, \prec) \) is a chain, then \( \forall a \in A: a \prec a_{0} \lor a_{0} \prec a \). If \( a_{0} \prec a \) for some \( a \in A \) then there exists \( p \in \mathbb{Z}^{+} \) such that \( a_{0} = pa \). On the one hand, \( a_{0} = pa \) implies \( a_{0} \ge a \). On the other hand, \( a_{0} \leq a \). Therefore \( a = a_{0} \). Thus \( a_{0} \) is also a maximal element of the chain \( A \) in \( (\mathbb{Z}^{+}, \prec) \), which means the maximal element of \( (A, \prec) \) is the first element of \( (A, \le) \).
\end{proof}

\begin{problem}{II.1.3}
Let \( \mathscr{F} \) be the set of all real-valued functions of a real variable. Show that by defining \( f \prec g \) to mean ``\( \forall x: f(x) \leq g(x) \),\@'' \( (\mathscr{F}, \prec) \) is a partial ordered set. If \( f \prec^{\prime} g \) denotes
\[
	(f = g) \text{ or } \left(\lim\limits_{x\to\infty} \dfrac{f(x)}{g(x)} = 0\right),
\]

is \( (\mathscr{F}, \prec^{\prime}) \) partially ordered?
\end{problem}

\begin{proof}
	For every \( f \in \mathscr{F} \), \( f \prec f \) since \( \forall x: f(x) \le f(x) \), so \( \prec \) is reflexive.

	If \( f \prec g \) and \( g \prec h \) then \( \forall x: f(x) \le g(x) \land g(x) \le h(x) \), from which we deduce that \( \forall x: f(x) \le h(x) \). Hence \( f \prec h \), so \( \prec \) is transitive.

	If \( f \prec g \) and \( g \prec f \) then \( \forall x: f(x) \le g(x) \land g(x) \le f(x) \), from which one has \( \forall x: f(x) = g(x) \), which precisely means \( f = g \). So \( \prec \) is antisymmetric.

	Hence \( (\mathscr{F}, \prec) \) is a partial ordered set.

	\bigskip

	For every \( f \in \mathscr{F} \), \( f = f \) so \( f \prec^{\prime} f \).

	If \( f \prec^{\prime} g \) and \( g \prec^{\prime} h \) then the following cases are exhaustive:
	\begin{itemize}
		\item \( f = g \) and \( g = h \). Then \( f = h \), which implies \( f \prec^{\prime} h \).
		\item \( \lim\limits_{x\to\infty} \dfrac{f(x)}{g(x)} = 0 \) and \( g = h \). Then \( \lim\limits_{x\to\infty} \dfrac{f(x)}{h(x)} = 0 \), which implies \( f \prec^{\prime} h \).
		\item \( f = g \) and \( \lim\limits_{x\to\infty} \dfrac{g(x)}{h(x)} = 0 \). Then \( \lim\limits_{x\to\infty } \dfrac{f(x)}{h(x)} = 0 \), which implies \( f \prec^{\prime} h \).
		\item \( \lim\limits_{x\to\infty} \dfrac{f(x)}{g(x)} = 0 \) and \( \lim\limits_{x\to\infty} \dfrac{g(x)}{h(x)} = 0 \). Then \( \lim\limits_{x\to\infty} \dfrac{f(x)}{h(x)} = \left(\lim\limits_{x\to\infty} \dfrac{f(x)}{g(x)}\right)\cdot \left(\lim\limits_{x\to\infty} \dfrac{g(x)}{h(x)}\right) = 0 \cdot 0 = 0 \), which implies \( f \prec^{\prime} h \).
	\end{itemize}

	Suppose that there exists \( f, g \in \mathscr{F} \) such that \( f \prec^{\prime} g \) and \( g \prec^{\prime} f \) but \( f \ne g \). From the definition of \( \prec^{\prime} \), it follows that \( \lim\limits_{x\to\infty} \dfrac{f(x)}{g(x)} = \lim\limits_{x\to\infty} \dfrac{g(x)}{f(x)} = 0 \) and
	\[
		0 = \left(\lim\limits_{x\to\infty} \dfrac{f(x)}{g(x)}\right)\cdot \left(\lim\limits_{x\to\infty} \dfrac{g(x)}{f(x)}\right) = \lim\limits_{x\to\infty} \dfrac{f(x)g(x)}{g(x)f(x)} = 1
	\]

	which is a contradiction. Hence \( \prec^{\prime} \) is antisymmetric.

	Thus \( (\mathscr{F}, \prec^{\prime}) \) is also partially ordered.
\end{proof}

\begin{problem}{II.1.4}
A partially ordered set is a lattice if each pair of its elements has a least upper bound and a greatest lower bound. Is \((\mathscr{F}, \prec)\) in Problem 3 a lattice? Is \((\mathscr{P}(X), \subset)\) a lattice? In \((\mathscr{P}(A \times A), \subset)\), is the set of all transitive relations a lattice? Determine also if the set of all partial orders, preorders, and well-orders are each lattices.
\end{problem}

\begin{proof}
	For any two functions \( f, g \in \mathscr{F} \), we define \( f \lor g, f \land g \in \mathscr{F} \) as follows:
	\[
		(f \lor g)(x) = \max\left\{ f(x), g(x) \right\},\qquad (f \land g)(x) = \min\left\{ f(x), g(x) \right\}.
	\]

	Then \( f \lor g \) and \( f \land g \) are evidently upper bound and lower bound of \( \left\{ f, g \right\} \). Let \( h \) be a function in \( \mathscr{F} \). If \( h \) is an upper bound of \( \left\{ f, g \right\} \) then \( \forall x: h(x) \ge f(x) \land h(x) \ge g(x) \), which implies \( \forall x: h(x) \ge (f \lor g)(x) \), equivalently, \( (f \lor g) \prec h \). If \( h \) is a lower bound of \( \left\{ f, g \right\} \) then \( \forall x: h(x) \le f(x) \land h(x) \le g(x) \), which means \( \forall x: h(x) \le (f \land g)(x) \), so \( h \prec (f \land g) \). Hence \( f\lor g \) is a least upper bound of \( \left\{ f, g \right\} \) and \( f\land g \) is a greatest lower bound of \( \left\{ f, g \right\} \). Thus \( (\mathscr{F}, \prec) \) is a lattice.

	\( (\mathscr{P}(X), \subset) \) is also a lattice. A least upper bound of \( \left\{ A, B \right\} \) is \( A \cup B \) and a greatest lower bound of \( \left\{ A, B \right\} \) is \( A \cap B \).

	In \( (\mathscr{P}(A\times A), \subset) \), the set of all transitive relations is a lattice. A greatest lower bound of \( \left\{ R, S \right\} \) is \( R \cap S \) and a least upper bound of \( \left\{ R, S \right\} \) is the intersection of all transitive relations containing \( R \) and \( S \)  (there is at least one transitive relation containing \( R \) and \( S \) and it is \( A\times A \)).

	In \( (\mathscr{P}(A\times A), \subset) \), the set of all reflexive relations is a lattice. A greatest lower bound of \( \left\{ R, S \right\} \) is \( R \cap S \) and a least upper bound of \( \left\{ R, S \right\} \) is the intersection of all reflexive relations containing \( R \) and \( S \).

	In \( (\mathscr{P}(A\times A), \subset) \), the set of all antisymmetric relations is a lattice. A greatest lower bound of \( \left\{ R, S \right\} \) is \( R \cap S \) and a least upper bound of \( \left\{ R, S \right\} \) is the intersection of all antisymmetric relations containing \( R \) and \( S \).

	Therefore in \( (\mathscr{P}(A\times A), \subset) \), the set of all partial preorders, orders are each lattices.

	However, in \( (\mathscr{P}(A\times A), \subset) \), the set of all well-orders is not necessarily a lattice. For example, consider \( A = \left\{ 1, 2 \right\} \) and
	\[
		\begin{split}
			R = \left\{ (1, 1), (2, 2), (1, 2) \right\}, \\
			S = \left\{ (1, 1), (2, 2), (2, 1) \right\},
		\end{split}
	\]

	are well-orders. However, \( \left\{ R, S \right\} \) doesn't have a lower bound (which is a well-order) hence it doesn't have a greatest lower bound.
\end{proof}

\begin{problem}{II.1.5}\label{problem:II.1.5}
Let \(A\) be the set of all infinite sequences of real numbers. Order \(A\) lexicographically; that is, \((a_{1}, a_{2}, \cdots) \prec (b_{1}, b_{2}, \cdots)\) if either \(a_{i} = b_{i}\) for all \(i\), or \(a_{n} < b_{n}\) at the first place \(n\) where they differ. Show that this is a total ordering in \(A\). Is the conventional ordering of the rationals the same as the lexicographic ordering of their decimal expansions?
\end{problem}

\begin{proof}
	For every sequence \( (a_{1}, a_{2}, \cdots) \), one has \( (a_{1}, a_{2}, \cdots) \prec (a_{1}, a_{2}, \cdots) \) by definition so \( \prec \) is reflexive.

	Suppose \( (a_{1}, a_{2}, \cdots) \prec (b_{1}, b_{2}, \cdots) \) and \( (b_{1}, b_{2}, \cdots) \prec (c_{1}, c_{2}, \cdots) \). The following cases are exhaustive.
	\begin{itemize}
		\item \( a_{i} = b_{i} \) for all \( i \) and \( b_{i} = c_{i} \) for all \( i \). Then \( a_{i} = c_{i} \) for all \( i \).
		\item \( a_{i} = b_{i} \) for all \( i \) and there exists a positive integer \( n \) such that \( b_{n} < c_{n} \) and \( b_{i} = c_{i} \) for all \( i < n \). Then \( a_{n} < c_{n} \) and \( a_{i} = c_{i} \) for all \( i < n \).
		\item There exists a positive integer \( n \) such that \( a_{n} < b_{n} \) and \( a_{i} = b_{i} \) for all \( i < n \) and \( b_{i} = c_{i} \) for all \( i \). Then \( a_{n} < c_{n} \) and \( a_{i} = c_{i} \) for all \( i < n \).
		\item There exists a positive integer \( n \) such that \( a_{n} < b_{n} \) and \( a_{i} = b_{i} \) for all \( i < n \) and there exists a positive integer \( m \) such that \( b_{m} < c_{m} \) and \( b_{i} = c_{i} \) for all \( i < m \). If \( n > m \) then \( a_{m} = b_{m} < c_{m} \) and \( a_{i} = b_{i} \) for all \( i < m \). If \( n \le m \) then \( a_{n} < b_{n} = c_{n} \) and \( a_{i} = c_{i} \) for all \( i < n \).
	\end{itemize}

	In either cases, one has \( (a_{1}, a_{2}, \cdots) \prec (c_{1}, c_{2}, \cdots) \), so \( \prec \) is transitive.

	Suppose \( (a_{1}, a_{2}, \cdots) \prec (b_{1}, b_{2}, \cdots) \) and \( (b_{1}, b_{2}, \cdots) \prec (a_{1}, a_{2}, \cdots) \). Assume for the sake of contrary that there exists a positive integer \( n \) such that \( a_{n} \ne b_{n} \). By the well-ordering property of the positive integers, we can assume without loss of generality that \( n \) is the smallest positive integer such that \( a_{n} \ne b_{n} \). Therefore \( a_{i} = b_{i} \) for all \( i < n \). However, either \( a_{n} < b_{n} \) or \( b_{n} < a_{n} \) so either \( (a_{1}, a_{2}, \cdots) \prec (b_{1}, b_{2}, \cdots) \) or \( (b_{1}, b_{2}, \cdots) \prec (a_{1}, a_{2}, \cdots) \), which is a contradiction. Hence \( a_{i} = b_{i} \) for all \( i \), which implies that \( \prec \) is antisymmetric.

	Therefore \( \prec \) is a partial order.

	Consider two sequences \( (a_{1}, a_{2}, \cdots) \) and \( (b_{1}, b_{2}, \cdots) \). Either \( a_{i} = b_{i} \) for all \( i \) or there exists a positive integer \( i \) such that \( a_{i} \ne b_{i} \). If the former is the case then \( (a_{1}, a_{2}, \cdots) \prec (b_{1}, b_{2}, \cdots) \). If the latter is the case then we use the least positive integer \( n \) such that \( a_{n} \ne b_{n} \). Since \( a_{i} = b_{i} \) for all \( i < n \) and either \( a_{n} < b_{n} \) or \( b_{n} < a_{n} \), we deduce that either \( (a_{1}, a_{2}, \cdots) \prec (b_{1}, b_{2}, \cdots) \) or \( (b_{1}, b_{2}, \cdots) \prec (a_{1}, a_{2}, \cdots) \). Hence \( (a_{1}, a_{2}, \cdots) \prec (b_{1}, b_{2}, \cdots) \) or \( (b_{1}, b_{2}, \cdots) \prec (a_{1}, a_{2}, \cdots) \), which means \( \prec \) is a total order.

	Thus \( A \) is a total ordered set.

	\bigskip

	The conventional ordering of the rationals is \textbf{not} the same as the lexicographic ordering of their decimal expansions because
	\[
		1.0999\ldots = 1.0\overline{9} = 1.1
	\]

	but \( 1.0\overline{9} \prec 1.1 \) and \( 1.0\overline{9} \ne 1.1 \) (with respect to the lexicographic ordering).
\end{proof}

\begin{problem}{II.1.6}
Let \(X\) be a totally ordered set. A pair of subsets \(A, B\) satisfying
\begin{enumerate}[label={(\alph*)}]
	\item \(A \cup B = X\),
	\item \(A \cap B = \varnothing\), and
	\item \((a \in A) \land (b \in B) \implies a \prec b\)
\end{enumerate}

is called a cut in \(X\). If \(A, B\), and \(A^{\prime}, B^{\prime}\) are cuts, show that \((A \subset A^{\prime}) \lor (A^{\prime} \subset A)\).
\end{problem}

\begin{proof}
	Firstly, we show that if \(A, B\) is a cut and \( a \in A \) then \( c \prec a \land c \ne a \implies c \in A \). Assume that \( c \prec a, c \ne a \) and \( c \notin A \) then \( c \prec a, c \ne a \) and \( c \in B \). From the definition of cuts, it follows that \( a \prec c \). From the antisymmetry of \( \prec \), it follows that \( a = c \), which is a contradiction.

	Given two cuts \(A, B\) and \(A^{\prime}, B^{\prime}\), either \( A \subset A^{\prime} \) or \( A \not\subset A^{\prime} \). If \( A \not\subset A^{\prime} \) then there exists \( x \in A \) such that \( x \notin A^{\prime} \). Since \( x \notin A^{\prime} \), we deduce that \( x \in B^{\prime} \). Therefore \( x \) follows every element of \( A^{\prime} \), which means \( A^{\prime} \subset A \). Thus \((A \subset A^{\prime}) \lor (A^{\prime} \subset A)\).
\end{proof}

\begin{problem}{II.1.7}
Let \(R \subset A \times A\) be a well-order. Show that unless \(A\) is a finite set, \(R^{-1}\) is not a well-order.
\end{problem}

\begin{proof}
	Assume that \( A \) is infinite. We define a map \( f: \mathbb{N} \to A \) as follows: \( f(0) \) is the first element of \( A \) (with respect to the well-order \( R \)) and \( f(n) \) is the first element of \( A  - \left\{ f(i) \mid i < n \right\} \) (with respect to the well-order \( R \)). The map \( f \) is an injection, hence \( f(\mathbb{N}) \) is an infinite subset of \( A \). For every element \( f(n) \in f(\mathbb{N}) \), the element \( f(n + 1) \) strictly follows \( f(n) \), so \( f(\mathbb{N}) \) has no maximal element, which precisely means \( f(\mathbb{N}) \) doesn't have a first element with respect to the order \( R^{-1} \). Therefore \( R^{-1} \) is not a well-order.
\end{proof}

\begin{problem}{II.1.8}
Show that if \(A\) is a finite set, each total ordering is a well-ordering.
\end{problem}

\begin{proof}
	Every singleton subset of \(A\) has a first element.

	Assume that every subset of \(n\) elements of \(A\) has a first element where \(n\) is a positive integer. Consider a subset \( B \subset A \) having \( n + 1 \) elements. Let \( b \) be an element of \( B \) then \( B - \left\{ b \right\} \) has \( n \) elements. By the inductive hypothesis, \( B - \left\{ b \right\} \) has a first element \( b_{0} \). Since \(A\) is totally ordered, \( b \) precedes \( b_{0} \) or \( b_{0} \) precedes \( b \). Therefore \( b \) or \( b_{0} \) is a first element of \( B \).

	By the principle of mathematical induction and the finiteness of \(A\), we conclude that every nonempty subset of \(A\) has a first element with respect to any given total ordering, hence each total ordering on a finite set is a well-ordering.
\end{proof}

\begin{problem}{II.1.9}\label{problem:II.1.9}
Let \(A, B\) be well-ordered. Show that lexicographic ordering in \(A \times B\) is also a well-ordering.
\end{problem}

\begin{proof}
	Denote the well-orders on \(A, B\) by \( \prec_{A}, \prec_{B} \), respectively, and the lexicographic ordering in \(A\times B\) by \(\prec\).

	For all \( (a, b) \in A \times B \), \( (a, b) \prec (a, b) \), so \( \prec \) is reflexive.

	If \( (a_{1}, b_{1}), (a_{2}, b_{2}), (a_{3}, b_{3}) \in A\times B \) such that \( (a_{1}, b_{1}) \prec (a_{2}, b_{2}) \) and \( (a_{2}, b_{2}) \prec (a_{3}, b_{3}) \) then \( a_{1} \prec_{A} a_{2} \) and \( a_{2} \prec a_{3} \). If \( a_{1} \ne a_{3} \) then \( (a_{1}, b_{1}) \prec (a_{3}, b_{3}) \). If \( a_{1} = a_{3} \) then \( a_{1} \prec_{A} a_{2} \) and \( a_{2} \prec_{A} a_{1} \), from which we deduce that \( a_{1} = a_{2} \) and \( a_{2} = a_{3} \). Therefore \( b_{1} \prec_{B} b_{2} \) and \( b_{2} \prec_{B} b_{3} \), which implies \( b_{1} \prec_{B} b_{3} \), so \( (a_{1}, b_{1}) \prec (a_{3}, b_{3}) \). Hence \( \prec \) is transitive.

	If \( (a_{1}, b_{1}), (a_{2}, b_{2}) \in A\times B \) such that \( (a_{1}, b_{1}) \prec (a_{2}, b_{2}) \) and \( (a_{2}, b_{2}) \prec (a_{1}, b_{1}) \) then \( a_{1} \prec_{A} a_{2} \) and \( a_{2} \prec_{A} a_{1} \) so \( a_{1} = a_{2} \). From the definition of lexicographic orderings, it follows that \( b_{1} \prec_{B} b_{2} \) and \( b_{2} \prec_{B} b_{1} \), so \( b_{1} = b_{2} \). Therefore \( (a_{1}, b_{1}) = (a_{2}, b_{2}) \), hence \( \prec \) is antisymmetric.

	Hence \( \prec \) is a partial order on \( A\times B \).

	Let \( C \) be a nonempty subset of \( A\times B \). Denote by \( \pi_{A} \) the canonical projection \( A\times B \to A \). The image \( \pi_{A}(C) \subset A \) is nonempty so it has a first element, namely, \( a_{0} \). Consider the subset \( D = \left\{ b \in B \mid (a_{0}, b) \in C \right\} \), \( D \) is nonempty because \( \pi_{A}(C) \) is nonempty. Since \( B \) is well-ordered, \( D \) has a first element, namely, \( b_{0} \). By the definition of lexicographic orderings, \( (a_{0}, b_{0}) \) is a first element in \( C \). Hence the lexicographic ordering in \( A\times B \) is a well-ordering.
\end{proof}

\begin{problem}{II.1.10}\label{problem:II.1.10}
Let \(A\) be well-ordered. Show that there does not exist any sequence \(\{a_{n} \mid n \in \mathbb{N}\}\) with \((a_{n+1} \prec a_{n}) \land (a_{n+1} \neq a_{n})\) for each \(n\).
\end{problem}


\begin{proof}
	Assume that there exists such a sequence \( {(a_{n})}_{n\in\mathbb{N}} \). Because \( A \) is well-ordered, the set \( \left\{ a_{n} \mid n \in \mathbb{N} \right\} \) has a minimal element, so there is some \( n \in \mathbb{N} \) such that \( a_{n} \) is equal to the minimal element of \( \left\{ a_{n} \mid n \in \mathbb{N} \right\} \). Therefore \( a_{n} \prec a_{n+1} \), which contradicts the definition of the sequence \( {(a_{n})}_{n\in\mathbb{N}} \). Hence there does not exist such a sequence.
\end{proof}

\begin{problem}{II.1.11}
Let \(\{A_{\alpha} \mid \alpha \in \mathscr{A}\}\) be a family of well-ordered sets {\color{red} (all but finitely many \( A_{\alpha} \) are non-singletons)}, and assume that \(\mathscr{A}\) is also well-ordered. Order \(\prod_{\alpha} A_\alpha\) lexicographically (as in Problem~\ref{problem:II.1.5}, this means: if \(\beta\) is the first element in \(\{\alpha \in \mathscr{A} \mid p_{\alpha}(x) \neq p_{\alpha}(y)\}\), then \(x \prec y\) if and only if \(p_{\beta}(x) \prec p_{\beta}(y)\)). Using Problem~\ref{problem:II.1.10}, show that this is a well-ordering in \(\prod_{\alpha} A_{\alpha}\) if and only if \(\mathscr{A}\) is a finite set.
\end{problem}

\begin{proof}
	This proof implicitly assumes the axiom of choice.

	Using the result of Problem~\ref{problem:II.1.9} and mathematical induction, one can show that if \( \mathscr{A} \) is a finite set then the given relation is a well-ordering in \( \prod_{\alpha} A_{\alpha} \).

	Conversely, suppose that the given relation is a well-ordering in \( \prod_{\alpha} A_{\alpha} \). Suppose on the contrary that \( \mathscr{A} \) is infinite. Let \( \mathscr{B} = \left\{ \alpha \in \mathscr{A} \mid A_{\alpha} \text{ is not a singleton } \right\} \). The map \( f: \prod_{\alpha\in \mathscr{A}} A_{\alpha} \to \prod_{\alpha\in \mathscr{B}} A_{\alpha} \) defined by \( f(c)(\alpha) = c(\alpha) \) for every \( c \in \prod_{\alpha\in\mathscr{A}} A_{\alpha} \) and \( \alpha \in \mathscr{B} \) is a relation-preserving bijection. Therefore it suffices to assume without loss of generality that every \( A_{\alpha} \) is non-singleton.

	In each \( A_{\alpha} \), let \( a_{\alpha} \) be the minimal element of \( A_{\alpha} \) and \( b_{\alpha} \) the successor of \( a_{\alpha} \). We use the sequence \( {(\alpha_{n})}_{n\in\mathbb{N}} \) in which \( \alpha_{0} \) is the minimal element of \( \mathscr{A} \) and \( \alpha_{n+1} \) is the minimal element of \( \mathscr{A} - \left\{ \alpha \in \mathscr{A} \mid \alpha_{n} \prec \alpha \land \alpha \ne \alpha_{n} \right\} \). Let's define a sequence \( {(f_{n})}_{n\in\mathbb{N}} \) in \( \prod_{\alpha} A_{\alpha} \) as follows:
	\[
		f_{n}(\alpha) = \begin{cases}
			a_{\alpha} & \alpha \ne \alpha_{n} \\
			b_{\alpha} & \alpha = \alpha_{n}
		\end{cases}
	\]

	then \( {(f_{n})}_{n\in\mathbb{N}} \) is a decreasing sequence, which contradicts the result in Problem~\ref{problem:II.1.10}.

	Thus the given ordering is a well-ordering in \(\prod_{\alpha} A_{\alpha}\) if and only if \(\mathscr{A}\) is a finite set.
\end{proof}

\section{Zorn's Lemma; Zermelo's Theorem}

\begin{problem}{II.2.1}
Prove the following extension of Zorn's lemma: If \( X \) is preordered and if each chain in \(X\) has an upper bound, then for each given \( x_{0} \in X \) there exists a maximal \( m \) with \( x_{0} \prec m \).
\end{problem}

\begin{proof}
	Define \( A = \left\{ x \in X \mid x_{0} \prec x \right\} \).

	\( A \) is nonempty as it contains \( x_{0} \) and \( A \) is partially ordered by \( \prec \). Every chain in \( A \) is a chain in \( X \) hence it has an upper bound in \( X \). From the definition of \( A \) and the transitivity of \( \prec \), this upper bound is also in \( A \), so every chain in \( A \) has an upper bound in \( A \).

	By Zorn's lemma, \( A \) has a maximal element \( m \). Because of the definition of \( A \), \( \neg (m \prec x) \) for every \( x \in X - A \). Hence \( m \) is a maximal element of \( X \) with \( x_{0} \prec m \).
\end{proof}

\begin{problem}{II.2.2}
Prove the following equivalent to the axiom of choice.
\begin{enumerate}[label={(\alph*)},leftmargin=*]
	\item If \(X\) is a preordered set, and if each well-ordered subset has an upper bound, then \(X\) has at least one maximal element. [Hint: One way to prove this is to require that a \( \varphi \)-tower be well-ordered by inclusion, rather than merely totally ordered, and then modify the proofs in 2.1, 2.3 accordingly.]
	\item If \( F: X \to \mathscr{P}(Y) \) is any mapping such that \( F(x) \ne \varnothing \) for each \( x \in X \), then there exists an \( f: X \to Y \) such that \( f(x) \in F(x) \) for each \( x \in X \).
	\item If \( X \) is a partially ordered set, then each chain in \(X\) is contained in a maximal chain.
	\item If \( \mathscr{P}(X) \) is partially ordered by inclusion, then each subset \( P \subset \mathscr{P}(X) \) of finite character contains a maximal element. [\( P \subset \mathscr{P}(X) \) is of finite character if \( A \in P \iff \text{(every finite subset of \(A\) belongs to \(P\))} \)]
\end{enumerate}
\end{problem}

\begin{proof}
	Assume the axiom of choice, then the Zorn's lemma holds.

	\begin{enumerate}[label={(\alph*)}]
		\item Let \( \mathcal{W} \) be the set of well-ordered subsets of \( X \). From the hypothesis, each element of \( \mathcal{W} \) has an upper bound in \( X \). Define a relation \( \le \) on \( \mathcal{W} \) in which \( W_{1} \le W_{2} \) if and only if \( W_{1} \) is an initial interval (segment) of \( W_{2} \) then \( \le \) is a partial ordering on \( \mathcal{W} \). Assume that \( \mathscr{C} \) is a chain in \( \mathcal{W} \) then \( \bigcup_{C \in \mathscr{C}} C \) is also a well-ordered subset of \( X \) and moreover an upper bound of the chain \( \mathscr{C} \). From the Zorn's lemma, it follows that \( \mathcal{W} \) has a maximal element \( W \). The set \( W \) has an upper bound \( w \). If \( w \) is not a maximal element of \( X \) then there exists \( x \in X \) such that \( w \ne x \land  w \prec x \) then \( W < W \cup \left\{ x \right\} \in \mathcal{W} \), which contradicts the maximality of \( W \). Hence \( X \) has a maximal element.
		\item From the axiom of choice, there exists a set \( S \) such that \( S \cap F(x) \) has exactly one element for each \( x \in X \). Define \( f: X \to Y \) such that \( f(x) \) is the only element of \( S \cap F(x) \), then \( f(x) \in F(x) \) for each \( x \in X \).
		\item Let \( Y \) be the set of all chains of \( X \) then \( Y \) is partially ordered by inclusion. Consider a chain \( {(Z_{\alpha})}_{\alpha\in\mathscr{A}} \) in \( Y \), we will show that \( \bigcup_{\alpha\in\mathscr{A}} Z_{\alpha} \) is a chain in \( X \). Let \( x, y \) be any two elements of \( \bigcup_{\alpha\in\mathscr{A}} Z_{\alpha} \) then there exist \( i, j \in \mathscr{A} \) such that \( x \in Z_{i}, y \in Z_{j} \). Since \( {(Z_{\alpha})}_{\alpha\in\mathscr{A}} \) is a chain in \( Y \) then \( Z_{i} \subset Z_{j} \) or \( Z_{j} \subset Z_{i} \). Hence \( x \) and \( y \) are comparable, so \( \bigcup_{\alpha\in\mathscr{A}} Z_{\alpha} \) is a chain in \( X \). Moreover, \( \bigcup_{\alpha\in\mathscr{A}} Z_{\alpha} \) is an upper bound of \( {(Z_{\alpha})}_{\alpha\in\mathscr{A}} \) so every chain in \( Y \) has an upper bound. Apply Zorn's lemma to \( Y \), we obtain that it has a maximal element, which precisely means each chain in \( X \) is contained in a maximal chain.
		\item \( P \) is a partially ordered set by inclusion. Let \( \mathcal{A} \) be a chain in \( P \). Consider the set \( \bigcup_{A\in\mathcal{A}} A \subset X \) and an arbitrary finite subset \( F \) of it. Since \( \mathcal{A} \) is a chain, there exists an element \( B \) of \( \mathcal{A} \) contaning \( F \) (one can prove this using mathematical induction on the number of elements of \( F \)). Therefore \( B \in P \), which means any finite subset of \( \bigcup_{A\in\mathcal{A}} A \) is an element of \( P \). Hence \( \bigcup_{A\in\mathcal{A}} A \in P \) and \( \bigcup_{A\in\mathcal{A}} A \) is an upper bound of the chain \( \mathcal{A} \). From the Zorn's lemma, it follows that \( P \) has a maximal element.
	\end{enumerate}

	Conversely,
	\begin{enumerate}[label={(\alph*)}]
		\item Let \( X \) be a partially ordered set and each of its chain has an upper bound.

		      Let \( \mathcal{W} \) be the set of well-ordered subsets of \( X \) and we define a relation \( \prec \) on \( \mathcal{W} \) in which \( A \prec B \) if and only if \( A \) is an initial interval (segment) of \( B \). Then \( \prec \) is a preordering on \( \mathcal{W} \). Let \( \mathscr{A} \) be a well-ordered subset of \( \mathcal{W} \) then \( \bigcup_{A\in \mathscr{A}} A \) is a well-ordered subset of \(X\) and an upper bound of \( \mathscr{A} \) by \( \prec \). Hence \( \mathcal{W} \) has a maximal element \( W \subset X \). Let \( w \) be an upper bound of \( W \) and suppose on the contrary that \( w \) is not a maximal element of \( X \) then there exists \( x \in X \) such that \( x \ne w \) and \( w \prec x \). Hence \( W \subsetneq W \cup \left\{ x \right\} \in \mathcal{W} \) which contradicts the maximality of \( W \). Therefore \( X \) has a maximal element, which implies the Zorn's lemma.
		\item Let \( {(A_{\alpha})}_{\alpha\in\mathscr{A}} \) be a family of nonempty sets (it is a map \( F: \mathscr{A} \to \mathscr{P}\left(\bigcup_{\alpha\in\mathscr{A}} A_{\alpha}\right) \)). From (b), there exists a map \( f: \mathscr{A} \to \bigcup_{\alpha\in\mathscr{A}} A_{\alpha} \) such that \( f(\alpha) \in F(\alpha) = A_{\alpha} \) for every \( \alpha \in \mathscr{A} \). Hence the axiom of choice holds.
		\item Let \( X \) be a partially ordered set and each of its chain has an upper bound.

		      Let \( A \) be a chain in \( X \). From (c), \( A \) is contained in a maximal chain \( C \subset X \). From the hypothesis, \( C \) has an upper bound \( x \). Suppose on the contrary that \( x \) is not a maximal element of \( X \) then there exists \( y \in X \) such that \( x \ne y \) and \( x \leq y \). So \( C \subsetneq C \cup \left\{ y \right\} \subset X \) and \( C \cup \left\{ y \right\} \) is a chain which is a proper superset of \( C \), and this contradicts the maximality of \( C \). Hence \( x \) is a maximal element of \( X \), so the Zorn's lemma holds.
		\item Let \( X \) be a partially ordered set and each of its chain has an upper bound.

		      Let \( Y \subset \mathscr{P}(X) \) be the set of all chains in \( X \). If \( A \in Y \) then \( A \) is a chain in \( X \), so every finite subset of \( A \) is a chain of \( X \) hence contained in \( Y \). Therefore \( Y \) is of finite character, which implies \( Y \) has a maximal element. So \( X \) has a maximal chain. From Hausdorff maximal principle (c) we conclude that the Zorn's lemma holds.
	\end{enumerate}

	Hence the axiom of choice and the four given statements are equivalent.
\end{proof}

\begin{problem}{II.2.3}
Let \( \mathscr{P}(X) \) be partially ordered by inclusion, and let \( \mathscr{A} \subset \mathscr{P}(X) \) be a maximal chain. In \( X \), define \( x \prec y \) if either \( x = y \) or \( \exists A \in \mathscr{A}: (x \in A) \land (y \notin A) \). Show that this is a total order in \( X \). If, in addition, for each \( \mathscr{L} \subset \mathscr{A} \), it is true that
\[
	\bigcap \left\{ A \mid A \in \mathscr{L} \right\} \in \mathscr{A},
\]

show that \( \prec \) is a well-order.
\end{problem}

\begin{proof}
	\textbf{Partial order}

	\( \prec \) is reflexive by definition.

	Suppose that \( x \prec y \) and \( y \prec z \). The following cases are exhaustive:
	\begin{itemize}[itemsep=0pt]
		\item \( x = y, y = z \) then \( x = z \), so \( x \prec z \).
		\item \( x = y, y \ne z \) then \( x \prec z \).
		\item \( x \ne y, y = z \) then \( x \prec z \).
		\item \( x \ne y, y \ne z \) then there exist \( A, B \in \mathscr{A} \) such that \( x \in A, y \notin A \) and \( y \in B, z \notin B \). Since \( \mathscr{A} \) is a chain, one has \( A \subset B \) or \( B \subset A \). Because \( y \notin A, y \in B \), we deduce that \( A \subset B \). From \( z \notin B \) and \( A \subset B \), it follows that \( z \notin A \). So \( x \in A \) and \( z \notin A \), which means \( x \prec z \).
	\end{itemize}

	In either cases, one has \( x \prec z \), so \( \prec \) is transitive.

	Suppose that \( x \prec y \) and \( y \prec x \). If \( x \ne y \) then there exist \( A_{x}, A_{y} \in \mathscr{A} \) such that \( x \in A_{x} \), \( y \notin A_{x} \), \( y \in A_{y} \), \( y \notin A_{x} \). Therefore \( A_{x}, A_{y} \) are not related by inclusion, which is a contradiction since \( \mathscr{A} \) is a chain. Hence \( x = y \), which means \( \prec \) is antisymmetric.

	Hence \( \prec \) is a partial order.

	\textbf{Total order}

	Let \( x \) be an element of \( X \) and define
	\[
		A_{x} = \bigcup \left\{ A \in \mathscr{A} \mid x \notin A \right\} \in \mathscr{P}(X).
	\]

	For each \( A \in \mathscr{A} \)
	\begin{itemize}[itemsep=0pt]
		\item either \( x \notin A \) --- This implies \( A \subset A_{x} \).
		\item or \( x \in A \) --- This implies \( B \subset A \) whenever \( x \notin B \in \mathscr{A} \) as \( \mathscr{A} \) is totally ordered. Therefore \( A_{x} \subset A \).
	\end{itemize}

	Since \( \mathscr{A} \) is a maximal chain, we deduce that \( A_{x} \in \mathscr{A} \). Moreover, from its definition, it follows that \( A_{x} \) is the largest element of \( \mathscr{A} \) not containing \( x \).

	Let \( x, y \) be two elements of \( X \) then \( A_{x}, A_{y} \in \mathscr{A} \), so \( A_{x} \subset A_{y} \) or \( A_{y} \subset A_{x} \). If \( A_{x} \ne A_{y} \) and \( A_{x} \subset A_{y} \) then \( x \prec y \). If \( A_{x} \ne A_{y} \) and \( A_{y} \subset A_{x} \) then \( y \prec x \).

	If \( A_{x} = A_{y} \), assume for the sake of contradiction that \( x \ne y \). From the maximality of \( \mathscr{A} \), we deduce that \( A_{x} \cup \left\{ x \right\} \in \mathscr{A} \), because any element of \( \mathscr{A} \) containing \(x\) also contains \(A_{x}\). However, \( A_{x} = A_{y} \) is the largest element of \( \mathscr{A} \) not containing \( y \) and \( A_{x} \cup \left\{ x \right\} \) is an element of \( \mathscr{A} \) not containing \( y \) and strictly larger than \( A_{y} \), which is a contradiction. Hence \( x = y \).

	Therefore \( \prec \) is a total order in \( X \).

	\textbf{Well order}

	Let \(S\) be a nonempty subset of \(X\).

	Define a map \( f: X \to \mathscr{A} \) by \( f(x) = A_{x} \). According to the above proof, we show that \( f \) is injective and \( x \prec y \iff A_{x} \subset A_{y} \). Let \( \mathscr{B} = \left\{ A_{x} \mid x \in S \right\} \subset \mathscr{A} \) then \( A_{S} = \bigcap \left\{ A \mid A \in \mathscr{B} \right\} \in \mathscr{A} \). The set \( A_{S} \) is the largest element of \( \mathscr{A} \) that doesn't contain any element of \( S \).

	Assume that there doesn't exist an element \( x\in X \) such that \( A_{S} \cup \left\{ x \right\} \in \mathscr{A} \) then for any \( A \in \mathscr{A} \) that is strictly larger than \( A_{S} \), \( A - A_{S} \) has more than one element, this contradicts the maximality of \( \mathscr{A} \) as we can add another set that is strictly between \( A_{S} \) and \( A \). Hence there exists \( s \in X \) such that \( A_{x} \cup \left\{ s \right\} \in \mathscr{A} \).

	Since \( A_{x} \) is the largest element of \( \mathscr{A} \) that doesn't contain any element of \( S \) and \( A_{x} \subsetneq A_{x} \cup \left\{ s \right\} \in \mathscr{A} \), we conclude that \( s \in S \). Moreover, \( s \) is the smallest element of \( S \), as \( f(s) = A_{S} \subset A_{x} = f(x) \) for every \( x \in S \).

	Thus \( S \) has a smallest element, so \( \prec \) is a well order in \(X\).
\end{proof}

\begin{problem}{II.2.4}
Let \(X\) be a partially ordered by \( R \subset X \times X \). Show that there exists a total order \( R^{\prime} \supset R \).
\end{problem}

\begin{proof}
	If \( R \) is already a total order in \( X \) then by picking \( R^{\prime} = R \), we are done.

	Otherwise, there exist \( x, y \in X \) which are not comparable. Define
	\[
		S = R \cup \left\{ (x, y) \right\} \cup \left\{ (p, q) \mid pRx \land yRq \right\}
	\]

	then \( S \) is reflexive, anti-symmetric and transitive. So \( S \) is a proper extension of \( R \).

	Let \( \mathcal{O} \) be the set of partial orders in \( X \) containing \( R \) and order \( \mathcal{O} \) by inclusion. Let \( \mathscr{C} \) be a chain in \( \mathcal{O} \) and \( B = \bigcup \left\{ A \mid A \in \mathscr{C} \right\} \). We show that \( B \in \mathcal{O} \).
	\begin{itemize}[itemsep=0pt]
		\item The union of reflexive relations is a reflexive relation, so \( B \) is reflexive.
		\item If \( x, y \in B \) and \( x B y, y B x \) then there exist \( S_{1}, S_{2} \in \mathscr{C} \) such that \( x S_{1} y \) and \( y S_{2} x \). Since \( \mathscr{C} \) is totally ordered then \( S_{1} \subset S_{2} \) or \( S_{2} \subset S_{1} \). The first case implies \( (x S_{2} y) \land (y S_{2} x) \) so \( x = y \) and the latter implies \( (y S_{1} x) \land (x S_{1} y) \) so \( x = y \). Hence \( B \) is anti-symmetric.
		\item If \( x, y, z \in B \) and \( x B y, y B z \) then there exist \( S_{1}, S_{2} \in \mathscr{C} \) such that \( x S_{1} y \) and \( y S_{2} z \). Since \( \mathscr{C} \) is totally ordered then \( S_{1} \subset S_{2} \) or \( S_{2} \subset S_{1} \). The first case implies \( x S_{2} z \) and the latter implies \( x S_{1} z \). Hence \( B \) is transitive.
		\item \( B \) contains \( R \).
	\end{itemize}

	Hence \( B \in \mathcal{O} \) and \( B \) is an upper bound of \( \mathscr{C} \). According to Zorn's lemma, \( \mathcal{O} \) has a maximal element \( R^{\prime} \). The maximality of \( R^{\prime} \) means it is a total order in \( X \).
\end{proof}

\begin{problem}{II.2.5}
Let \( A \) be partially ordered. A set \( B \subset A \) is called cofinal in \(A\) if
\[
	\forall a\in A\; \exists b\in B: a \prec b.
\]

Prove that every totally ordered set has a cofinal well-ordered subset.
\end{problem}

\begin{proof}
	Let \( X \) be a totally ordered set by an order relation \(\prec\). Let \( \mathscr{A} \) be the set of well-ordered subset of \( X \). Define a relation \( \le \) in \( \mathscr{A} \) by \( A \le B \) if and only if \( A \) is an initial interval (segment) of \( B \). Consider a chain \( \mathscr{C} \) in \( \mathscr{A} \), the set \( \bigcup\left\{ C \mid C \in \mathscr{C} \right\} \) is also a well-ordered subset of \( X \). Hence \( \mathscr{C} \) has an upper bound in \( \mathscr{A} \). According to Zorn's lemma, \( \mathscr{A} \) has a maximal element \( A \).

	Assume that \( A \) is not cofinal in \( X \) then there exists \( x \in X \) such that \( \neg (x \prec a) \) for every \( a \in A \), then \( A \cup \left\{ x \right\} \) is a well-ordered subset of \( X \) and this contradicts the maximality of \( A \). Hence \( A \) is a cofinal well-ordered subset of \( X \).
\end{proof}

\begin{problem}{II.2.6}
A partially ordered set is ``of type \(\omega\)'' if each chain \(C\) has a cofinal sequence \( \left\{ c_{i} \mid i \in \mathbb{N} \right\} \). Prove that if, in a partially ordered set of type \(\omega\), each ascending sequence has an upper bound, then there is a maximal element.
\end{problem}

\begin{proof}
	Let \(X\) be a partially ordered set of type \(\omega\) such that each ascending sequence in \(X\) has an upper bound.

	Let \( C \) be a chain of \( X \) then \( C \) has a cofinal sequence \( \left\{ c_{i} \mid i \in \mathbb{N} \right\} \). We define recursively a sequence \( \left\{ d_{i} \mid i \in \mathbb{N} \right\} \) as follows:
	\[
		\begin{cases}
			d_{0} = c_{0} \\
			d_{n+1} = \max\left\{ d_{n}, c_{n+1} \right\}
		\end{cases}
	\]

	then \( \left\{ d_{i} \mid i \in \mathbb{N} \right\} \) is an ascending sequence. This sequence is well-defined since \( C \) is totally ordered. Moreover, \( c_{n} \leq d_{n} \) for every \( n \in \mathbb{N} \) so \( \left\{ d_{i} \mid i \in \mathbb{N} \right\} \) is also an ascending cofinal sequence in \( C \). Since each ascending sequence in \( X \) has an upper bound and \( \left\{ d_{i} \mid i \in \mathbb{N} \right\} \) is a cofinal sequence in \( C \), we deduce that \( C \) has an upper bound. Hence every chain of \(X\) has an upper bound. According to Zorn's lemma,
\end{proof}

\begin{problem}{II.2.7}
Let \(X\) be a preordered set, and assume that each chain in \(X\) has an upper bound. Let \(f: X \to X\) satisfy \( x \prec f(x) \) for each \( x \in X \). Prove: there exists at least one \(x_{0}\) such that \( f(x_{0}) \prec x_{0} \) also.
\end{problem}

\begin{proof}
\end{proof}

\begin{problem}{II.2.8}
Let \( \mathbb{R} \) be the additive group of reals, and \( \mathbb{C} \) the additive group of complex numbers. Prove that the groups \( \mathbb{R} \) and \( \mathbb{C} \) are isomorphic.
\end{problem}

\begin{proof}
\end{proof}

\section{Ordinals}

\section{Comparability of Ordinals}

\section{Transfinite induction and Construction}

\section{Ordinal numbers}

\section{Cardinals}

\section{Cardinal Arithmetic}

\section{The ordinal number omega}
