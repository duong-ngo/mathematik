\documentclass[class=understanding-analysis,crop=false]{standalone}

\begin{document}

\chapter{The Real Numbers}

\section{Discussion: The Irrationality of $\sqrt{2}$}

\section{Some Preliminaries}

\begin{exercise}
    \begin{enumerate}[label = (\alph*)]
        \item Prove that $\sqrt{3}$ is irrational. Does a similar argument work to show $\sqrt{6}$ is irrational?
        \item Where does the proof of Theorem 1.1.1 break down if we try to use it to prove $\sqrt{4}$ is irrational?
    \end{enumerate}
\end{exercise}

\begin{proof}
    \begin{enumerate}
        \item Suppose that $\sqrt{3}$ is rational. Then there exists two coprime integers $a$, $b$ such that $a, b\ne 0$ and $a^{2} = 3b^{2}$.
            \par $3b^{2}$ is divided by 3 then so is $a^{2}$ and $a$. Let $a = 3a_{1}$, which leads to $3a_{1}^{2} = b^{2}$.
            \par $3a_{1}^{2}$ is divided by 3 then so is $b^{2}$ and $b$.
            \par This implies that $a$ and $b$ are divided by 3, which means $a$ and $b$ are not coprime.
            \par Hence $\sqrt{3}$ is irrational.
            \bigskip
            \par Suppose that $\sqrt{6}$ is rational. Then there exists two coprime integers $a$, $b$ such that $a, b \ne 0$ and $a^{2} = 6b^{2}$.
            \par $3b^{2}$ is divided by 3 then so is $a^{2}$ and $a$. Let $a = 3a_{1}$, then $3a_{1}^{2} = 2b^{2}$.
            \par $3a_{1}^{2}$ is divided by 3 then so is $b^{2}$ and $b$.
            \par Therefore, $a$ and $b$ are divided by 3, which means they are not coprime.
            \par Hence $\sqrt{6}$ is irrational.
        \item When we try to prove $\sqrt{4}$ is irrational by contradiction, the proof breaks down after we conclude that the numerator ($a$) is divided by 2 and reduce the equation $a^{2} = 4b^{2}$.
    \end{enumerate}
\end{proof}

\begin{exercise}
    \par Show that there is no rational number $r$ satisfying $2^{r} = 3$.
\end{exercise}

\begin{proof}
    \par Suppose that $r = \dfrac{p}{q}$, where $p, q\ne 0$ and $p$, $q$ are coprime.
    \par If $\dfrac{p}{q} \le 0$, then $2^{r} \le 1$. Therefore, $\frac{p}{q} > 0$.
    \par Without loss of generality, suppose that $p, q > 0$.
    \begin{align*}
        2^{p/q} & = 3 \\
        \Longleftrightarrow 2^{p} & = 3^{q}
    \end{align*}
    \par $p$ are positive integers, therefore $2^{p}$ is divided by 2. But $3^{q}$ is not divided by 2.
    \par Hence $r$ is irrational.
\end{proof}

\begin{exercise}
    \par Decide which of the following represent true statements about the nature of sets. For any that are false, provide a specific example where the statement in question does not hold.
    \begin{enumerate}[label = (\alph*)]
        \item If $A_{1}\supseteq A_{2}\supseteq A_{3}\supseteq A_{4} \cdots$ are all sets containing an infinite number of elements, then the intersection $\bigcap^{\infty}_{n=1}A_{n}$ is infinite as well.
        \item If $A_{1}\supseteq A_{2}\supseteq A_{3}\supseteq A_{4} \cdots$ are all finite, nonempty sets of real numbers, then the intersection $\bigcap^{\infty}_{n=1}A_{n}$ is finite and nonempty.
        \item $A\cap (B\cup C) = (A\cap B)\cup C$.
        \item $A\cap (B\cap C) = (A\cap B)\cap C$.
        \item $A\cap (B\cup C) = (A\cap B)\cup (A\cap C)$.
    \end{enumerate}
\end{exercise}

\begin{proof}
    \begin{enumerate}[label = (\alph*)]
        \item This is false.
            \par For example, let $A_{n} = \{ x : x\in\mathbb{N}, x\ge n \}$.
            \par Then $\bigcap^{\infty}_{n=1}A_{n} = \emptyset$.
        \item This is true.
            \par Suppose that $A_{1}$ has $a_{1}$ elements.
            \par The number of elements of $A_{2}$ ($a_{2}$) does not exceed the number of elements of $A_{1}$.
            \par The number of elements of $A_{n+1}$ ($a_{n+1}$) does not exceed the number of elements of $A_{n}$ ($a_{n}$).
            \par $a_{1}, a_{2}, \ldots$ are positive integers and they don't exceed $a_{1}$. Hence there exists $a_{k}$ such that $a_{k}$ is the smallest, and consequently, $a_{k} = a_{k+1} = a_{k+2} = \cdots$
            \par This implies that $\bigcap^{\infty}_{n=1}A_{n} = A_{k}$, where $A_{k}$ is finite and nonempty.
        \item This is false.
            \par For example, $A = \emptyset$, and $B$, $C$ are nonempty set, then $A\cap (B\cup C)$ is empty while $(A\cap B)\cup C$ is not.
        \item This is true.
        \item This is true.
    \end{enumerate}
\end{proof}

\begin{exercise}
    \par Produce an infinite collection of sets $A_{1}$, $A_{2}$, $A_{3}$, \ldots with the property that every $A_{i}$ has an infinite number of elements, $A_{i}\cap A_{j} = \emptyset$ for all $i\ne j$, and $\bigcup^{\infty}_{i=1}A_{i} = \mathbb{N}$.
\end{exercise}

\begin{proof}
    \par $A_{1}$ is the set of integers $x$, where $2 \nmid x$.
    \par $A_{2}$ is the set of integers $x$, where $2 \mid x$ and $2 \nmid x$.
    \par $A_{n}$ is the set of integers $x$, where $2^{n-1} \mid x$ and $2^{n} \nmid x$.
    \par For all $n\in\mathbb{N}$, $A_{n}$ has infinite number of elements.
    \par $A_{i}\cap A_{j} = \emptyset$, which follows the definition of these sets, and $\bigcup^{\infty}_{i=1}A_{i} = \bigsqcup^{\infty}_{i=1}A_{i} = \mathbb{N}$.
\end{proof}

\begin{exercise}[De Morgan's Laws]
    \par Let $A$ and $B$ be subsets of $\mathbb{R}$.
    \begin{enumerate}[label = (\alph*)]
        \item If $x\in (A\cap B){}^{c}$, explain why $x\in A^{c}\cup B^{c}$. This shows that $(A\cap C){}^{c} \subseteq A^{c}\cup B^{c}$.
        \item Prove the reverse inclusion $(A\cap B){}^{c}\supseteq A^{c}\cup B^{c}$, and conclude that $(A\cap B){}^{c} = A^{c}\cup B^{c}$.
        \item Show $(A\cup B){}^{c} = A^{c}\cap B^{c}$ by demonstrating inclusion both ways.
    \end{enumerate}
\end{exercise}

\begin{proof}
    \begin{enumerate}[label = (\alph*)]
        \item $x\in (A\cap B){}^{c}$, then $x\notin A\cap B$.
            \[
                \Rightarrow
                \begin{sqcases}
                    x\in A^{c}\cap B \\
                    x\in B^{c}\cap A \\
                    x\in A^{c}\cap B^{c}
                \end{sqcases}
                \Rightarrow
                \begin{sqcases}
                    x\in A^{c}, x\notin B^{c} \\
                    x\in B^{c}, x\notin A^{c} \\
                    x\in A^{c}\cap B^{c}
                \end{sqcases}
                \Rightarrow x\in A^{c}\cup B^{c}.
            \]
            \par So $x\in (A\cap B){}^{c}$ implies $x\in A^{c}\cup B^{c}$, therefore $(A\cap B){}^{c}\subseteq A^{c}\cup B^{c}$.
        \item $x\in A^{c}\cup B^{c}$, then $x\in A^{c}$ or $x\in B^{c}$
            \[
                \Rightarrow
                \begin{sqcases}
                    x\in A^{c} \\
                    x\in B^{c}
                \end{sqcases}
                \Rightarrow
                \begin{sqcases}
                    x\notin A, x\in B \\
                    x\notin B, x\in A \\
                    x\notin A, x\notin B
                \end{sqcases}
                \Rightarrow
                x\notin A\cap B
                \Rightarrow
                x\in (A\cap B){}^{c}
            \]
            \par So $A^{c}\cup B^{c}\subseteq (A\cap B){}^{c}$. These two inclusions implies that $A^{c}\cup B^{c} = (A\cap B){}^{c}$
        \item
            \[
                x\in (A\cup B){}^{c}
                \Leftrightarrow
                x\notin A\cup B
                \Leftrightarrow
                \begin{cases}
                    x\notin A\cap B^{c} \\
                    x\notin A^{c}\cap B \\
                    x\notin A\cap B
                \end{cases}
                \Leftrightarrow
                \begin{cases}
                    x\notin A\cap B^{c}, x\notin A\cap B \\
                    x\notin A^{c}\cap B, x\notin A\cap B
                \end{cases}
                \Leftrightarrow
                \begin{cases}
                    x\notin A \\
                    x\notin B
                \end{cases}
                \Leftrightarrow
                x\in A^{c}\cap B^{c}.
            \]
            \par Hence $(A\cup B){}^{c}\subseteq A^{c}\cap B^{c}$, $A^{c}\cap B^{c}\subseteq (A\cup B){}^{c}$, therefore $(A\cup B){}^{c} = A^{c}\cap B^{c}$.
    \end{enumerate}
\end{proof}

\begin{exercise}
    \begin{enumerate}[label = (\alph*)]
        \item Verify the triangle inequality in the special case where $a$ and $b$ have the same sign.
        \item Find an efficient proof for all the cases at once by first demonstrating $(a + b){}^{2} = (\abs{a} + \abs{b}){}^{2}$.
        \item Prove $\abs{a - b}\le \abs{a-c} + \abs{c-d} + \abs{d-b}$ for all $a$, $b$, $c$, and $d$.
        \item Prove $\abs{\abs{a} - \abs{b}}\le \abs{a-b}$.
    \end{enumerate}
\end{exercise}

\begin{proof}
    \begin{enumerate}[label = (\alph*)]
        \item If $a\ge 0$ and $b\ge 0$, $\abs{a} + \abs{b} = a + b = \abs{a + b}$.
            \par If $a < 0$ and $b < 0$, $\abs{a} + \abs{b} = (-a) + (-b) = -(a + b) = \abs{a+b}$.
        \item
            \begin{align*}
                &\abs{ab} \ge ab \\
                \Longleftrightarrow& \abs{a}\cdot\abs{b} \ge ab \\
                \Longleftrightarrow& 2\abs{a}\cdot\abs{b} \ge 2ab \\
                \Longleftrightarrow& \abs{a}^{2} + 2\abs{a}\cdot\abs{b} + \abs{b}^{2} \ge a^{2} + 2ab + b^{2} \\
                \Longleftrightarrow& (\abs{a} + \abs{b}){}^{2} \ge (a+b){}^{2} \\
                \Longleftrightarrow& \abs{a} + \abs{b} \ge \abs{a + b}.
            \end{align*}
            \par The equality holds iff $ab\ge 0$.
        \item By triangle inequality:
            \begin{align*}
                \abs{a - b} & = \abs{(a - c) + (c - b)} \\
                            & \le \abs{a - c} + \abs{c - b} \\
                            & = \abs{a - c} + \abs{(c - d) + (d - b)} \\
                            & \le \abs{a - c} + \abs{c - d} + \abs{d - b}.
            \end{align*}
        \item
            \begin{align*}
                & \abs{a} = \abs{(a - b) + b} \le \abs{a - b} + \abs{b} \Rightarrow \abs{a} - \abs{b} \le \abs{a - b} \\
                & \abs{b} = \abs{(b - a) + a} \le \abs{b - a} + \abs{a} \Rightarrow \abs{b} - \abs{a} \le \abs{a - b}
            \end{align*}
            \par Therefore, $\abs{\abs{a} - \abs{b}}\le \abs{a - b}$.
    \end{enumerate}
\end{proof}

\begin{exercise}
    \par Given a function $f$ and a subset $A$ of its domain, let $f(A)$ represent the range of $f$ over the set $A$; that is, $f(A) = \{ f(x) : x\in A \}$.
    \begin{enumerate}[label = (\alph*)]
        \item Let $f(x) = x^{2}$. If $A = [0, 2]$ and $B = [1, 4]$, find $f(A)$ and $f(B)$. Does $f(A\cap B) = f(A)\cap f(B)$ in this cases? Does $f(A\cup B) = f(A)\cup f(B)$?
        \item Find two sets $A$ and $B$ for which $f(A\cap B) \ne f(A)\cap f(B)$.
        \item Show that, for an arbitrary function $g: \mathbb{R}\to\mathbb{R}$, it it always true that $g(A\cap B) \subseteq g(A)\cap g(B)$ for all sets $A, B\subseteq\mathbb{R}$.
        \item Form and prove a conjecture about the relationship between $g(A\cup B)$ and $g(A)\cup g(B)$ for an arbitrary function $g$.
    \end{enumerate}
\end{exercise}

\begin{proof}
    \begin{enumerate}[label = (\alph*)]
        \item $f(A) = [0, 4]$, $f(B) = [1, 16]$.
            \par $f(A)\cap f(B) = [1, 4]$, $A\cap B = [1, 2]\Rightarrow f(A\cap B) = [1, 4]$. So $f(A\cap B) = f(A)\cap f(B)$, in this case.
            \par $A\cup B = [0, 4]\Rightarrow f(A\cup B) = [0, 16]$
            \par $f(A)\cup f(B) = [0, 4]\cup [1, 16] = [0, 16]$.
            \par Therefore, $f(A\cup B) = f(A)\cup f(B)$, in this case.
        \item Let $A = [-2, -1]$ and $B = [1, 2]$.
            \par $f(A) = f(B) = [1, 4]$, so $f(A)\cap f(B) = [1, 4]$.
            \par $A\cap B = \emptyset\Rightarrow f(A\cap B) = \emptyset\ne [1, 4] = f(A)\cap f(B)$.
        \item If $A\cap B = \emptyset$, the formula is valid, since $f(A\cap B) = \emptyset$ and empty set is subset of any set.
            \par Otherwise, let $y\in g(A\cap B)$. By definition of $g(A\cap B)$, there exists $x\in A\cap B$ such that $g(x) = y$.
            \par $x\in A\cap B$ means $x\in A$ and $x\in B$. This implies that $f(x)\in f(A)$, $f(x)\in f(B)$, so $y = f(x) \in f(A)\cap f(B)$.
            \par Hence $f(A\cap B)\subseteq f(A)\cap f(B)$.
        \item Conjecture that $g(A\cup B) = g(A)\cup g(B)$.
            \par If $A$ nor $B$ is empty set, the equality holds.
            \par Otherwise:
            \begin{itemize}
                \item Let $y\in g(A\cup B)$, then there exists $x\in A\cup B$ such that $g(x) = y$.
                    \par Since $x\in A\cup B$, then $y = g(x) \in g(A)$ or $g(B)$, equivalently, $y\in g(A)\cup g(B)$.
                    \par So $g(A\cup B)\subseteq g(A)\cup g(B)$.
                \item Let $z\in g(A)\cup g(B)$
                    \par If $z\in g(A)$, then $z$ has preimage in $A$. If $z\in g(B)$, then $z$ has preimage in $B$.
                    \par Therefore, $z$ has preimage in $A\cup B$ - let it be $x$. Since $x\in A\cup B$, then $z = g(x)\in g(A\cup B)$.
                    \par So $g(A)\cup g(B)\subseteq g(A\cup B)$.
            \end{itemize}
            \par Thus, $g(A)\cup g(B) = g(A\cup B)$.
    \end{enumerate}
\end{proof}



\section{The Axiom of Completeness}

\section{Consequences of Completeness}

\section{Cardinality}

\section{Cantor's Theorem}

\section{Epilogue}


\end{document}
