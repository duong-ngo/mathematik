\documentclass[12pt]{book}

\usepackage[left=2cm,right=2cm,top=2.5cm,bottom=2.5cm]{geometry}
\usepackage[unicode=true,colorlinks=true,linkcolor=blue]{hyperref}

\usepackage{amsmath}
\usepackage{amsfonts}
\usepackage{amssymb}
\usepackage{amsthm}
\usepackage{amscd}
\usepackage{mathtools}
\usepackage{mathrsfs}
\usepackage{cases}

\usepackage{fancyhdr}
\usepackage{xcolor}
\usepackage{titlesec}
\usepackage{indentfirst}
\usepackage{chngcntr}
\usepackage{caption}
\usepackage{subcaption}
\usepackage{booktabs}
\usepackage[inline]{enumitem}
\usepackage{setspace}

\setcounter{chapter}{0}

\pagestyle{fancy}
\setlength{\headheight}{16pt}
\lhead{}
\rhead{\leftmark}
\lfoot{}
\cfoot{\thepage}
\rfoot{}

\titleformat{\chapter}[display]{\flushright\bf\huge}{\chaptertitlename\,\thechapter}{10pt}{}
\titleformat{\section}{\bf\Large}{\thesection}{10pt}{}
\titleformat{\subsection}{\bf\large}{\thesubsection}{10pt}{}
\titleformat{\subsubsection}{\bf\large}{\thesubsubsection}{10pt}{}

\captionsetup{labelfont={bf},labelsep=period}
\counterwithin{figure}{chapter}
\counterwithin{table}{chapter}

\theoremstyle{definition}
\newtheorem{note}{Note}
\counterwithin{note}{chapter}

\newtheorem{theorem}{Theorem}
\counterwithin{theorem}{chapter}
\newtheorem{definition}[theorem]{Definition}
\newtheorem{example}[theorem]{Example}
\newtheorem{lemma}[theorem]{Lemma}

\setstretch{1.4142}
\theoremstyle{definition}
\newtheorem{innercustomgeneric}{\customgenericname}
\providecommand{\customgenericname}{}
\newcommand{\newcustomtheorem}[2]{%
  \newenvironment{#1}[1]
  {%
   \renewcommand\customgenericname{#2}%
   \renewcommand\theinnercustomgeneric{##1}%
   \innercustomgeneric%
  }
  {\endinnercustomgeneric}
}

\newcustomtheorem{exercise}{Exercise}
\newcustomtheorem{problem}{Problem}

\newenvironment{sqcases}{%
    \matrix@check\sqcases\env@sqcases
}{%
    \endarray\right.%
}
\def\env@sqcases{%
\let\@ifnextchar\new@ifnextchar
\left\lbrack{}
\def\arraystretch{1.2}%
\array{@{}l@{\quad}l@{}}%
}
\newcommand{\rank}{\text{rank}}
\newcommand{\abs}[1]{\left\vert{#1}\right\vert}
\newcommand{\norm}[1]{\left\Vert{#1}\right\Vert}
\newcommand{\anglebracket}[1]{\left\langle{#1}\right\rangle}
\newcommand{\innerprod}[1]{\left\langle{#1}\right\rangle}
\newcommand{\tuple}[1]{\left({#1}\right)}
\newcommand{\set}[1]{\left\{{#1}\right\}}
\newcommand{\powerset}[1]{\mathcal{P}\left({#1}\right)}
\newcommand{\mapsfrom}{\mathrel{\reflectbox{\ensuremath{\mapsto}}}}
\newcommand{\longmapsfrom}{\mathrel{\reflectbox{\ensuremath{\longmapsto}}}}
\newcommand{\floor}[1]{\left\lfloor{#1}\right\rfloor}
\newcommand{\ceiling}[1]{\left\lceil{#1}\right\rceil}
\newcommand{\fraction}[1]{\left\{{#1}\right\}}
\newcommand{\openinterval}[1]{\left]{#1}\right[}
\newcommand{\closedinterval}[1]{\left[{#1}\right]}
\newcommand{\halfopenleft}[1]{\left]{#1}\right]}
\newcommand{\halfopenright}[1]{\left[{#1}\right[}

\title{Kristopher Tapp's Differential Geometry of Curves and Surfaces: Notes and Exercises}
\author{Ngo Quang Duong}
\date{\today}

\begin{document}

\maketitle

\tableofcontents

\chapter{First Examples}

\section{The Simplest Examples}

\section{Linear Systems with Constant Coefficients}

\chapter{Smooth Maps}

\section{Smooth Functions and Smooth Maps}

\section{Partitions of Unity}

\documentclass[class=linearalgebra,crop=false]{standalone}

\newcommand{\sgn}[1]{\text{sgn}\left({#1}\right)}
\setcounter{lemma}{0}

\begin{document}

\chapter{Định thức và hệ phương trình tuyến tính}

\par Thực hiện các phép nhân sau đây, viết các phép thế thu được thành tích của những xích rời rạc và tính dấu của chúng.

\begin{exercise}
    $
        \begin{pmatrix}
            1 & 2 & 3 & 4 & 5 \\
            2 & 4 & 5 & 1 & 3
        \end{pmatrix}
        \begin{pmatrix}
            1 & 2 & 3 & 4 & 5 \\
            4 & 3 & 5 & 1 & 2
        \end{pmatrix}
    $.
\end{exercise}

\begin{proof}[Lời giải]
    \[
        \begin{pmatrix}
            1 & 2 & 3 & 4 & 5 \\
            2 & 4 & 5 & 1 & 3
        \end{pmatrix}
        \begin{pmatrix}
            1 & 2 & 3 & 4 & 5 \\
            4 & 3 & 5 & 1 & 2
        \end{pmatrix}
        =
        \begin{pmatrix}
            4 & 3 & 5 & 1 & 2 \\
            1 & 5 & 3 & 2 & 4
        \end{pmatrix}
        \begin{pmatrix}
            1 & 2 & 3 & 4 & 5 \\
            4 & 3 & 5 & 1 & 2
        \end{pmatrix}
        =
        \begin{pmatrix}
            1 & 2 & 3 & 4 & 5 \\
            1 & 5 & 3 & 2 & 4
        \end{pmatrix}.
    \]
    \[
        \begin{pmatrix}
            1 & 2 & 3 & 4 & 5 \\
            1 & 5 & 3 & 2 & 4
        \end{pmatrix}
        =
        (1)(2,5,4)(3).
    \]
    \[
        \sgn{
            \begin{matrix}
                1 & 2 & 3 & 4 & 5 \\
                1 & 5 & 3 & 2 & 4
            \end{matrix}
        }
        = \sgn{1}\sgn{2,5,4}\sgn{3}
        = 1.
    \]
\end{proof}

\begin{exercise}
    $
        \begin{pmatrix}
            1 & 2 & 3 & 4 & 5 \\
            3 & 5 & 4 & 1 & 2
        \end{pmatrix}
        \begin{pmatrix}
            1 & 2 & 3 & 4 & 5 \\
            4 & 3 & 1 & 5 & 2
        \end{pmatrix}
    $.
\end{exercise}

\begin{proof}[Lời giải]
    \[
        \begin{pmatrix}
            1 & 2 & 3 & 4 & 5 \\
            3 & 5 & 4 & 1 & 2
        \end{pmatrix}
        \begin{pmatrix}
            1 & 2 & 3 & 4 & 5 \\
            4 & 3 & 1 & 5 & 2
        \end{pmatrix}
        =
        \begin{pmatrix}
            4 & 3 & 1 & 5 & 2 \\
            1 & 4 & 3 & 2 & 5
        \end{pmatrix}
        \begin{pmatrix}
            1 & 2 & 3 & 4 & 5 \\
            4 & 3 & 1 & 5 & 2
        \end{pmatrix}
        =
        \begin{pmatrix}
            1 & 2 & 3 & 4 & 5 \\
            1 & 4 & 3 & 2 & 5
        \end{pmatrix}.
    \]
    \[
        \begin{pmatrix}
            1 & 2 & 3 & 4 & 5 \\
            1 & 4 & 3 & 2 & 5
        \end{pmatrix}
        =
        (1)(2,4)(3)(5).
    \]
    \[
        \sgn{
            \begin{matrix}
                1 & 2 & 3 & 4 & 5 \\
                1 & 4 & 3 & 2 & 5
            \end{matrix}
        }
        = \sgn{1}\sgn{2,4}\sgn{3}\sgn{5}
        = -1.
    \]
\end{proof}

\begin{exercise}
    $(1,2)(2,3)\ldots (n-1,n)$.
\end{exercise}

\begin{lemma}\label{chapter3:cycles-product}
    $(a_{1}, a_{2}, \ldots, a_{k})(a_{k},a_{k+1}) = (a_{1},a_{2},\ldots, a_{k+1})$.
\end{lemma}

\begin{proof}[Chứng minh bổ đề]
    \par Xét dãy
        \[
            a_{1}, a_{2}, \ldots, a_{k-1}, a_{k}, a_{k+1}.
        \]
    \par Sau khi tác động bằng $(a_{k},a_{k+1})$, dãy trên trở thành:
        \[
            a_{1}, a_{2}, \ldots, a_{k-1}, a_{k+1}, a_{k}.
        \]
    \par Sau khi tác động bằng $(a_{1}, a_{2}, \ldots, a_{k})$, dãy trên (liên trên) trở thành:
        \[
            a_{2}, a_{3}, \ldots, a_{k}, a_{k+1}, a_{1}.
        \]
    \par Theo định nghĩa về xích, ta có điều phải chứng minh.
\end{proof}

\begin{proof}[Lời giải]
    \par Theo bổ đề~\ref{chapter3:cycles-product}:
        \[
            (1,2)(2,3)\ldots (n-1,n) = (1,2,\ldots,n)
            =
            \begin{pmatrix}
                1 & 2 & \cdots & n-1 & n \\
                2 & 3 & \cdots & n   & 1
            \end{pmatrix}
        \]
    \par $(1,2,\ldots, n)$ chính là một xích.
        \[
            \sgn{1,2,\ldots,n} = \sgn{1,2}\sgn{2,3}\ldots\sgn{n-1,n} = (-1){}^{n-1}.
        \]
\end{proof}

\begin{exercise}
    $(1,2,3)(2,3,4)(3,4,5)\ldots (n-2,n-1,n)$.
\end{exercise}

\begin{proof}[Lời giải]
    \par Theo bổ đề~\ref{chapter3:cycles-product}, nếu $n > 3$:
    \begin{align*}
        (1,2,3)(2,3,4)(3,4,5)\ldots (n-2,n-1,n)
        & = (1,2)(2,3)(2,3)(3,4)(3,4)(4,5) \ldots (n-2,n-1)(n-1,n) \\
        & = (1,2)(2,3){}^{2}(3,4){}^{2}\ldots (n-2,n-1){}^{2}(n-1,n) \\
        & = (1,2)(n-1,n)\qquad\text{(đây là 2 xích rời nhau)} \\
        & =
        \begin{pmatrix}
            1 & 2 & 3 & \cdots & n-2 & n-1 & n   \\
            2 & 1 & 3 & \cdots & n-2 & n   & n-1
        \end{pmatrix}.
    \end{align*}
    \par Nếu $n = 3$:
    \[
        (1,2,3) =
        \begin{pmatrix}
            1 & 2 & 3 \\
            2 & 3 & 1
        \end{pmatrix}.
    \]
    \par Trong cả hai trường hợp, dấu của phép thế (kết quả) là 1.
\end{proof}

\begin{exercise}
    Cho hai cách sắp thành dãy $a_{1}$, $a_{2}$, \ldots, $a_{n}$ và $b_{1}$, $b_{2}$, \ldots, $b_{n}$ của $n$ số tự nhiên đầu tiên. Chứng minh rằng có thể đưa cách sắp này về cách sắp kia bằng cách sử dụng không quá $n-1$ phép thế sơ cấp.
\end{exercise}

\begin{lemma}\label{chapter3:product-of-disjoint-cycles}
    Mọi phép thế đều có thể được viết dưới dạng tích của các xích rời nhau.
\end{lemma}

\begin{proof}[Chứng minh bổ đề~\ref{chapter3:product-of-disjoint-cycles}]
\end{proof}

\begin{lemma}\label{chapter3:product-of-transpositions}
    Một xích độ dài $k$ ($k > 1$) có thể viết được dưới dạng tích của $k-1$ phép thế sơ cấp.
\end{lemma}

\begin{proof}[Chứng minh bổ đề~\ref{chapter3:product-of-transpositions}]
\end{proof}

\begin{proof}
\end{proof}

\end{document}

\chapter{Cartesian products}

\section{Cartesian product topology}

\begin{problem}{IV.1.1}
Let \( \left\{ Y_{\alpha} \mid \alpha \in \mathscr{A} \right\} \) be a family of spaces. Assume that each \( Y_{\alpha} \) has a basis of cardinal number \( \le \aleph \). What is the cardinal of a basis for \( \prod_{\alpha} Y_{\alpha} \)?
\end{problem}

\begin{proof}
	% TODO
	It is \( \aleph \cdot \aleph(\mathscr{A}) \).
\end{proof}

\begin{problem}{IV.1.2}
Let \( \aleph(\mathscr{A}) \) be arbitrary and \( \prod_{\alpha} A_{\alpha} \subset \prod_{\alpha} Y_{\alpha} \). If all but at most finitely many factors \( A_{\alpha} = Y_{\alpha} \), prove \( \operatorname{Int}\left( \prod_{\alpha} A_{\alpha} \right) = \prod_{\alpha} \operatorname{Int}(A_{\alpha}) \).
\end{problem}

\begin{proof}
	Let \( \beta \in \mathscr{A} \). We will show that \( \operatorname{Int}\left( \prod_{\alpha} A_{\alpha} \right) = \prod_{\alpha} \operatorname{Int}(A_{\alpha}) \) when \( A_{\alpha} = Y_{\alpha} \) for all \(\alpha \ne \beta\).
	\begingroup
	\allowdisplaybreaks%
	\begin{align*}
		\operatorname{Int}\left( \prod_{\alpha} A_{\alpha} \right) & = \mathscr{C}\overline{\prod_{\alpha} Y_{\alpha} - \prod_{\alpha}A_{\alpha}}                         \\
		                                                           & = \mathscr{C}\overline{\mathscr{C}A_{\beta} \times \prod_{\alpha \ne \beta} Y_{\alpha}}              \\
		                                                           & = \mathscr{C}\left( \overline{\mathscr{C}A_{\beta}} \times \prod_{\alpha\ne\beta} Y_{\alpha} \right) \\
		                                                           & = \mathscr{C}\overline{\mathscr{C}A_{\beta}} \times \prod_{\alpha\ne\beta} Y_{\alpha}                \\
		                                                           & = \operatorname{Int}(A_{\beta}) \times \prod_{\alpha\ne\beta} Y_{\alpha}                             \\
		                                                           & = \prod_{\alpha} \operatorname{Int}(A_{\alpha}).
	\end{align*}
	\endgroup

	Now let \( \mathscr{B} \) be a finite subset of \( \mathscr{A} \) and \( A_{\alpha} = Y_{\alpha} \) whenever \( \alpha \notin \mathscr{B} \). From the previous case, we deduce that
	\begingroup
	\allowdisplaybreaks%
	\begin{align*}
		\operatorname{Int}\left( \prod_{\alpha} A_{\alpha} \right) & = \operatorname{Int}\left( \bigcap_{\beta \in \mathscr{B}} A_{\beta} \times \prod_{\alpha\ne\beta} Y_{\alpha} \right) \\
		                                                           & = \bigcap_{\beta \in \mathscr{B}} \operatorname{Int}\left( A_{\beta} \times \prod_{\alpha\ne\beta} Y_{\alpha} \right) \\
		                                                           & = \bigcap_{\beta \in \mathscr{B}} \operatorname{Int}(A_{\beta}) \times \prod_{\alpha\ne\beta} Y_{\alpha}              \\
		                                                           & = \prod_{\alpha} \operatorname{Int}(A_{\alpha}).
	\end{align*}
	\endgroup
\end{proof}

\begin{problem}{IV.1.3}
Let \( R \) be the real numbers with upper-limit topology (Chapter III, Section 3, Example 4). Show that \( R \times R \) is not a discrete space, but that \( A = \left\{ (x, y) \mid x + y = 1 \right\} \), as a subspace of \( R \times R \), has the discrete topology.
\end{problem}

\begin{proof}
	The singleton \( \left\{ (0, 0) \right\} \) is not open in \( R \times R \) as it doesn't contain any basic open set of the for \( \halfopenleft{a, b} \times \halfopenleft{c, d} \).

	On the other hand, if \( (x, y) \in A \) then \( \left\{ (x, y) \right\} = A \cap (\halfopenleft{x - 1, x} \times \halfopenleft{y - 1, y}) \) so \( \left\{ (x, y) \right\} \) is open in \( A \). Thus \( A \) has the discrete topology.
\end{proof}

\begin{problem}{IV.1.4}
Prove: \( \prod_{\alpha} A_{\alpha} \) is dense in \( \prod_{\alpha} Y_{\alpha} \) if and only if each \( A_{\alpha} \subset Y_{\alpha} \) is dense.
\end{problem}

\begin{proof}
	Because \( \overline{\prod_{\alpha} A_{\alpha}} = \prod_{\alpha} \overline{A_{\alpha}} \), the result follows.
\end{proof}

\section{Continuity of maps}

\begin{problem}{IV.2.1}
Prove: The cartesian product topology in \( \prod_{\alpha} Y_{\alpha} \) is the smallest topology for which all projections \( p_{\beta}: \prod_{\alpha} Y_{\alpha} \to Y_{\beta} \) are continuous.
\end{problem}

\begin{proof}
	Let \( \mathscr{T} \) be a topology on \( \prod_{\alpha} Y_{\alpha} \) such that all projections \( p_{\beta} \) are continuous. We need to show that \( \mathscr{T} \) contains the cartesian product topology.

	For each open set \( U_{\beta} \subset Y_{\beta} \), \( \left\langle U_{\beta} \right\rangle = p_{\beta}^{-1}(U_{\beta}) \in \mathscr{T} \) because \( p_{\beta} \) is continuous. Therefore the cartesian product topology is contained in \( \mathscr{T} \).

	Thus the cartesian product topology is the smallest topology for which all projections are continuous.
\end{proof}

\begin{problem}{IV.2.2}
Let \( \left\{ Y_{\alpha} \mid \alpha \in \mathscr{A} \right\} \) be a family of spaces. For each \( \mathscr{B} \subset \mathscr{A} \), let
\[
	p_{\mathscr{B}}: \prod_{\alpha \in \mathscr{A}} Y_{\alpha} \to \prod_{\beta \in \mathscr{B}} Y_{\beta}
\]

be the projection. Let \( A \subset \prod_{\alpha} Y_{\alpha} \) be closed. Prove:
\[
	A = \bigcap_{\mathscr{B} \text{ finite}} p_{\mathscr{B}}^{-1}\left( p_{\mathscr{B}}(A) \right).
\]
\end{problem}

\begin{proof}
	For each \( \mathscr{B} \subset \mathscr{A} \), \( p_{\mathscr{B}}^{-1}\left( p_{\mathscr{B}}(A) \right) \supset A \) so
	\[
		\bigcap_{\mathscr{B} \text{ finite}} p_{\mathscr{B}}^{-1}\left( p_{\mathscr{B}}(A) \right) \supset A
	\]

	Let \( c \in \bigcap_{\mathscr{B} \text{ finite}} p_{\mathscr{B}}^{-1}\left( p_{\mathscr{B}}(A) \right) \). Suppose on the contrary that \( c \notin A \). Since \( A \) is closed, there exists a basic open set \( U = \left\langle U_{\alpha_{1}}, \ldots, U_{\alpha_{n}} \right\rangle \) that contains \( c \) and is contained in \( \mathscr{C}A \). Let \( \mathscr{B} = \left\{ \alpha_{1}, \ldots, \alpha_{n} \right\} \). Because \( c \in p_{\mathscr{B}}^{-1}(p_{\mathscr{B}}(A)) \), \( p_{\mathscr{B}}(c) \in p_{\mathscr{B}}(A) \), so there exists \( f \in A \) such that \( p_{\mathscr{B}}(c) = p_{\mathscr{B}}(f) \). This means \( p_{\alpha_{i}}(f) = f(\alpha_{i}) = c(\alpha_{i}) \in U_{\alpha_{i}} \) for each \( i = 1, \ldots, n \), so \( f \in \left\langle U_{\alpha_{i}} \right\rangle \), which means \( f \in U = \bigcap^{n}_{i=1} \left\langle U_{\alpha_{i}} \right\rangle \). Therefore \( f \in A \cap U \), which contradicts \( U \subset \mathscr{C}A \). Thus \( c \in A \) and we conclude that \( \bigcap_{\mathscr{B} \text{ finite}} p_{\mathscr{B}}^{-1}\left( p_{\mathscr{B}}(A) \right) \subset A \).

	Hence \( A = \bigcap_{\mathscr{B} \text{ finite}} p_{\mathscr{B}}^{-1}\left( p_{\mathscr{B}}(A) \right) \).
\end{proof}


\section{Slices in Cartesian Products}

\begin{problem}{IV.3.1}
Let \( \mathscr{S} \) be the Sierpiński space, and \( \mathscr{S} \times \left\{ 0 \right\} \) the slice in \( \mathscr{S} \times \mathscr{S} \) parallel to the first factor. Is \( \mathscr{S} \times \left\{ 0 \right\} \) closed in \( \mathscr{S} \times \mathscr{S} \)?
\end{problem}

\begin{proof}
	The complement of \( \mathscr{S} \times \left\{0\right\} \) in \( \mathscr{S} \times \mathscr{S} \) is \( \mathscr{S} \times \left\{1\right\} \), which is not open. Hence \( \mathscr{S} \times \left\{ 0 \right\} \) is not closed in \( \mathscr{S} \times \mathscr{S} \).
\end{proof}

\section{Peano curves}


\newpage
\chapter{Rings and Fields}

\chapter{Identification Topology; Weak Topology}

\section{Identification Topology}

\begin{problem}{VI.1.1}\label{problem:VI.1.1}
Reversing the situation treated in the text, let \(X\) be a set, \( (Y, \mathscr{T}) \) a space, and \( p: X \to Y \) a surjective map. Prove:
\begin{enumerate}[label={(\alph*)}]
	\item \( \mathscr{T}_{X} = \left\{ p^{-1}(U) \mid U \text{ open in } Y \right\} \) is a topology in \( X \).
	\item \( p: (X, \mathscr{T}_{X}) \to (Y, \mathscr{T}) \) is continuous, open, and closed.
\end{enumerate}
\end{problem}

\begin{proof}
	\begin{enumerate}[label={(\alph*)}]
		\item \( \mathscr{T}_{X} \) contains \( \varnothing, X \) as \( p^{-1}(\varnothing) = \varnothing \) and \( p^{-1}(Y) = X \).

		      If \( {\left\{ U_{\alpha} \right\}}_{\alpha\in\mathscr{A}} \) is a collection of open sets in \( Y \), then
		      \[
			      \bigcup_{\alpha\in\mathscr{A}} p^{-1}(U_{\alpha}) = p^{-1}\left(\bigcup_{\alpha\in\mathscr{A}} U_{\alpha}\right)
		      \]

		      so \( \mathscr{T}_{X} \) is closed under arbitrary union.

		      If \( U_{1}, \ldots, U_{n} \) are open sets in \( Y \) then
		      \[
			      \bigcap^{n}_{i=1} p^{-1}(U_{i}) = p^{-1}\left(\bigcap^{n}_{i=1} U_{i}\right)
		      \]

		      so \( \mathscr{T}_{X} \) is closed under finite intersection.

		      Hence \( \mathscr{T}_{X} \) is a topology in \( X \).
		\item For each open set \( U \) in \( Y \), \( p^{-1}(U) \in \mathscr{T}_{X} \) so \( p \) is continuous.

		      Let \( V \) be an open set in \( X \). Then there is an open set \( U \) in \( Y \) such that \( V = p^{-1}(U) \). Hence \( p(V) = pp^{-1}(U) = U \) because \( p \) is surjective. So \( p \) is an open map.

		      Let \( W \) be a closed set in \( X \) then \( X - W \) is open and there exists an open set \( U \) in \( Y \) such that \( X - W = p^{-1}(U) \). Therefore
		      \[
			      W = X - p^{-1}(U) = p^{-1}(Y) - p^{-1}(U) = p^{-1}(Y - U)
		      \]

		      which implies that \( p(W) = pp^{-1}(Y - U) = Y - U \), which is closed in \( Y \). So \( p \) is a closed map.

		      Thus \( p \) is a continuous, open, and closed map.
	\end{enumerate}
\end{proof}

\begin{problem}{VI.1.2}
For each \( \alpha \in \mathscr{A} \), let \( p_{\alpha}: X_{\alpha} \to Y_{\alpha} \) be a continuous, open surjection. Show that \( \prod_{\alpha} p_{\alpha}: \prod_{\alpha} X_{\alpha} \to \prod_{\alpha} Y_{\alpha} \) is an identification.
\end{problem}

\begin{proof}
	For the sake of brevity, denote \( p = \prod_{\alpha} p_{\alpha} \). By definition, \( p \) is surjective.

	\( p_{Y_{\alpha}} \circ p \) is continuous for each projection \( p_{Y_{\alpha}}: \prod_{\alpha} Y_{\alpha} \to Y_{\alpha} \) so \( p \) is continuous.

	Let \( \prod_{\alpha} U_{\alpha} \) be a basic open set in \( \prod_{\alpha} X_{\alpha} \), which means \( U_{\alpha} = X_{\alpha} \) for all but finitely many \( \alpha \) and \( U_{\alpha} \) is open in \( X_{\alpha} \) for every \( \alpha \). Because \( p_{\alpha} \) is an open surjection for each \( \alpha \), the image
	\[
		p\left( \prod_{\alpha} U_{\alpha} \right) = \prod_{\alpha} p_{\alpha}(U_{\alpha})
	\]

	is open in \( \prod_{\alpha} Y_{\alpha} \) as \( p_{\alpha}(U_{\alpha}) \) is open in \( Y_{\alpha} \) and \( p_{\alpha}(U_{\alpha}) = Y_{\alpha} \) for all but finitely many \( \alpha \). Hence \( p \) is an open map.

	\( p \) is a continuous, open surjection so \( p \) is an identification.
\end{proof}

\begin{problem}{VI.1.3}
Let \( X \) be a space and \( A \subset X \) a subspace. Assume that there exists a continuous \( r: X \to A \) such that \( r\vert_{A} = 1_{A} \) (such a map is called a \textit{retraction} of \(X\) onto \(A\)). Show that \( r \) is an identification.
\end{problem}

\begin{proof}
	By definition, \( r \) is continuous and surjective. Let \( f: A \xhookrightarrow{} X \) be the inclusion map.

	\( f \) is continuous and \( r \circ f = 1_{A} \) so \( r \) is an identification.
\end{proof}

\begin{problem}{VI.1.4}\label{problem:VI.1.4}
Let \( X \) be any set. Given any family \( \left\{ (Y_{\alpha}, \mathscr{T}_{\alpha}), f_{\alpha} \mid \alpha \in \mathscr{A} \right\} \) of spaces and maps \( f_{\alpha}: X \to Y_{\alpha} \), the ``projective limit topology of \(X\) determined by this family'' is \( \bigvee_{\alpha} f_{\alpha}^{-1}(\mathscr{T}_{\alpha}) \) (see Problem~\ref{problem:III.3.8}). Prove:
\begin{enumerate}[label={(\alph*)}]
	\item If \( j: X \to \prod_{\alpha} Y_{\alpha} \) is the map \( j(x) = \left\{ f_{\alpha}(x) \right\} \), then \( \bigvee_{\alpha} f_{\alpha}^{-1}(\mathscr{T}_{\alpha}) \) is the topology in \(X\) determined by \(j\) as in Problem~\ref{problem:VI.1.1}.
	\item If whenever \( x \ne x^{\prime} \), there is some index \( \alpha \) such that \( f_{\alpha}(x) \ne f_{\alpha}(x^{\prime}) \), then \( j \) is an embedding.
\end{enumerate}
\end{problem}

\begin{proof}
	\begin{enumerate}[label={(\alph*)}]
		\item Let \( \prod_{\alpha} U_{\alpha} \) be a subbasic open set in \( \prod_{\alpha} Y_{\alpha} \) then \( U_{\alpha} = Y_{\alpha} \) for every \( \alpha \) but one \( \beta \in \mathscr{A} \).
		      \[
			      j^{-1}\left( \prod_{\alpha} U_{\alpha} \right) = \bigcap_{\alpha} f_{\alpha}^{-1}(U_{\alpha}) = f_{\beta}^{-1}(U_{\beta}) \in \bigvee_{\alpha} f_{\alpha}^{-1}(\mathscr{T}_{\alpha})
		      \]

		      Hence \( j \) is continuous, which means if \( j^{-1}(U) \) is open whenever \( U \subset \prod_{\alpha} Y_{\alpha} \) is open.

		      Let \( V \) be an open set in \( X \). According to the definition of the topology \( \bigvee_{\alpha} f_{\alpha}^{-1}(\mathscr{T}_{\alpha}) \), \( V \) can be written as a union of finite intersection of elements in \( \bigcup_{\alpha} f_{\alpha}^{-1}(\mathscr{T}_{\alpha}) \), which means
		      \[
			      V = \bigcup_{i\in I} V_{i}
		      \]

		      where each \( V_{i} \) is a finite intersection of elements in \( \bigcup_{\alpha} f_{\alpha}^{-1}(\mathscr{T}_{\alpha}) \).
		      \[
			      V_{i} = \bigcap^{n_{i}}_{k=1} f_{\alpha_{k}}^{-1}(U_{\alpha_{k}}) = \bigcap^{n_{i}}_{k=1} j^{-1}\left( U_{\alpha_{k}} \times \prod_{\alpha \ne \alpha_{k}} Y_{\alpha} \right) = j^{-1}\left( \bigcap^{n_{i}}_{k=1} U_{\alpha_{k}} \times \prod_{\alpha \ne \alpha_{k}} Y_{\alpha} \right) = j^{-1}(W_{i})
		      \]

		      where \( U_{\alpha_{k}} \) is open in \( Y_{\alpha_{k}} \). So
		      \[
			      V = \bigcup_{i\in I} j^{-1}(W_{i}) = j^{-1}\left( \bigcup_{i\in I} W_{i} \right)
		      \]

		      which means \( V \) is the preimage of an open set in \( \prod_{\alpha} Y_{\alpha} \).

		      Thus \( \bigvee_{\alpha} f_{\alpha}^{-1}(\mathscr{T}_{\alpha}) \) is the same as the topology in \( X \) determined by \( j \) as in Problem~\ref{problem:VI.1.1}.
		\item According to Problem~\ref{problem:VI.1.1}, \( j \) is continuous, open, and closed.

		      Whenever \( x \ne x^{\prime} \), there is some index \( \alpha \) such that \( f_{\alpha}(x) \ne f_{\alpha}(x^{\prime}) \), then \( j(x) \ne j(x^{\prime}) \), which implies \( j \) is injective.

		      A continuous, open, injective map is an embedding so \( j \) is an embedding.
	\end{enumerate}
\end{proof}

\section{Subspaces}

\begin{problem}{VI.2.1}
Let \(X\) have the projective limit topology (Problem~\ref{problem:VI.1.4}) determined by
\[
	\left\{ Y_{\alpha}, f_{\alpha} \mid \alpha \in \mathscr{A} \right\}
\]

and let \( A \subset X \). Prove: The subspace topology of \(A\) is the projective limit topology determined by the maps \( f_{\alpha}\vert_{A} \).
\end{problem}

\begin{proof}
	The projective limit topology on \( A \) determined by the maps \( f_{\alpha}\vert_{A} \) has subbasis
	\[
		\bigcup_{\alpha} {(f_{\alpha}\vert_{A})}^{-1}(\mathscr{T}_{\alpha})
	\]

	Let \( V \) be an open set in \( A \) (with the projective limit topology) then
	\[
		V = \bigcup_{i\in I} V_{i}
	\]

	in which each \( V_{i} \) is the intersection of finitely many elements of \( \bigcup_{\alpha} {(f_{\alpha}\vert_{A})}^{-1}(\mathscr{T}_{\alpha}) \). So there exist \( \alpha_{i_{1}}, \ldots, \alpha_{i_{n(i)}} \in \mathscr{A} \) such that
	\[
		V_{i} = \bigcap^{n(i)}_{k=1} {(f_{\alpha_{k}}\vert_{A})}^{-1}(U_{\alpha_{k}})
	\]

	Hence
	\begingroup
	\allowdisplaybreaks%
	\begin{align*}
		V_{i} & = \bigcap^{n(i)}_{k=1} (A \cap f_{\alpha_{k}}^{-1}(U_{\alpha_{k}}))                                                             \\
		      & = A \cap \bigcap^{n(i)}_{k=1} f_{\alpha_{k}}^{-1}(U_{\alpha_{k}})                                                               \\
		      & = A \cap \bigcap^{n(i)}_{k=1} j^{-1}\left( U_{\alpha_{k}} \times \prod_{\alpha \ne \alpha_{k}} Y_{\alpha} \right)               \\
		      & = A \cap j^{-1}\left( \bigcap^{n(i)}_{k=1} \left( U_{\alpha_{k}} \times \prod_{\alpha\ne\alpha_{k}} Y_{\alpha} \right) \right).
	\end{align*}
	\endgroup

	Therefore
	\begingroup
	\allowdisplaybreaks%
	\begin{align*}
		V & = A \cap \bigcup_{i\in I} j^{-1}\left( \bigcap^{n(i)}_{k=1} \left( U_{\alpha_{k}} \times \prod_{\alpha\ne\alpha_{k}} Y_{\alpha} \right) \right) \\
		  & = A \cap j^{-1}\left( \bigcup_{i\in J} \bigcap^{n(i)}_{k=1} \left( U_{\alpha_{k}} \times \prod_{\alpha\ne\alpha_{k}} Y_{\alpha} \right) \right)
	\end{align*}
	\endgroup

	Hence \( V \) is in the subspace topology of \( A \).

	Conversely, one can show that if \( V \) is in the subspace topology of \( A \), then \( V \) is also in the projective limit topology on \( A \) determinded by the maps \( f_{\alpha}\vert_{A} \).

	Thus the projective limit topology on \( A \) determinded by the maps \( f_{\alpha}\vert_{A} \) and the subspace topology on \( A \) coincide.
\end{proof}

\section{General Theorems}

\begin{problem}{VI.3.1}
Let \( p: X \to Y \) be a continuous open (or closed) surjection, and assume that each fiber \( p^{-1}(y) \) is connected. For any \( F \subset Y \), show that \( F \) is connected if and only if \( p^{-1}(F) \) is connected.
\end{problem}

\begin{proof}
	By Proposition 2.1, \( p\vert_{p^{-1}(F)}: p^{-1}(F) \to F \) is an identification because \( p \) is an identification which is also an open (or closed) map. Denote \( q = p\vert_{p^{-1}(F)} \).

	If \( p^{-1}(F) \) is connected then \( F = p(p^{-1}(F)) \) is connected, as \( p \) is a continuous surjection.

	If \( p^{-1}(F) \) is not connected then there is a continuous surjection \( h: p^{-1}(F) \to 2 \). As each fiber of \( q \) (each fiber of \(q \) is a fiber of \(p\)) is connected, the restriction of \( h \) to each fiber is a constant map. Therefore \( hq^{-1}: F \to 2 \) is a continuous surjection, according to the transgression property, which means \( F \) is not connected.
\end{proof}

\begin{problem}{VI.3.2}
Let \( X \) have the projective limit topology \( \mathscr{T} \) determined by the family
\[
	\left\{ (Y_{\alpha}, \mathscr{T}_{\alpha}), f_{\alpha} \mid \alpha \in \mathscr{A} \right\}
\]

Assume that each \( \mathscr{T}_{\alpha} \) is the projective limit topology determined by a family
\[
	\left\{ (Z_{\alpha, \beta}, \mathscr{T}_{\alpha,\beta}), g_{\alpha,\beta} \mid \beta \in \mathscr{B} \right\}.
\]

Prove: \( \mathscr{T} \) is the projective limit topology determined by
\[
	\left\{ (Z_{\alpha,\beta}, \mathscr{T}_{\alpha,\beta}), g_{\alpha,\beta} \circ f_{\alpha} \mid (\alpha, \beta) \in \mathscr{A} \times \mathscr{B} \right\}.
\]
\end{problem}

\begin{proof}
	Denote by \( \widetilde{\mathscr{T}} \) the projective limit topology determined by
	\[
		\left\{ (Z_{\alpha,\beta}, \mathscr{T}_{\alpha,\beta}), g_{\alpha,\beta} \circ f_{\alpha} \mid (\alpha, \beta) \in \mathscr{A} \times \mathscr{B} \right\}.
	\]

	Let \( h: X \to \prod_{\alpha} Y_{\alpha} \) be the map \( h(x) = {\left\{ f_{\alpha}(x) \right\}}_{\alpha} \) then
	\[
		\mathscr{T} = \left\{ h^{-1}(U) \mid U \text{ open in } \prod_{\alpha}Y_{\alpha} \right\}
	\]

	according to Problem~\ref{problem:VI.1.4}.

	For each \( \alpha \), let \( h_{\alpha}: Y_{\alpha} \to \prod_{\beta} Z_{\alpha,\beta} \) be the map \( h_{\alpha}(x) = {\left\{ g_{\alpha,\beta}(x) \right\}}_{\beta} \) then
	\[
		\mathscr{T}_{\alpha} = \left\{ h_{\alpha}^{-1}(U) \mid U \text{ open in } \prod_{\beta} Z_{\alpha,\beta} \right\}
	\]

	according to Problem~\ref{problem:VI.1.4}.

	Let \( \ell: (X, \widetilde{\mathscr{T}}) \to \prod_{\alpha,\beta} Z_{\alpha,\beta} \) be the map \( \ell(x) = {\left\{ g_{\alpha,\beta}(f_{\alpha}(x)) \right\}}_{\alpha,\beta} \) then
	\[
		\widetilde{\mathscr{T}} = \left\{ \ell^{-1}(U) \mid U \text{ open in } \prod_{\alpha,\beta} Z_{\alpha,\beta} \right\}
	\]

	according to Problem~\ref{problem:VI.1.4}.

	Note that \( f_{\alpha} = p_{\alpha} \circ h \) and \( g_{\alpha,\beta} = p_{\alpha,\beta} \circ h_{\alpha} \) in which \( p_{\alpha}: \prod_{\alpha} Y_{\alpha} \to Y_{\alpha} \) and \( p_{\alpha,\beta}: \prod_{\beta} Z_{\alpha,\beta} \to Z_{\alpha,\beta} \) are projection maps. Denote by \( q_{\alpha,\beta} \) the projection map \( \prod_{\alpha,\beta} W_{\alpha,\beta} \to W_{\alpha,\beta} \).
	\[
		\begin{tikzcd}
			&& {\prod_{\alpha} Y_{\alpha}} \\
			\\
			X && {Y_{\alpha}} && {\prod_{\beta}Z_{\alpha,\beta}} && {Z_{\alpha,\beta}}
			\arrow["{p_{\alpha}}", from=1-3, to=3-3]
			\arrow["h", from=3-1, to=1-3]
			\arrow["{f_{\alpha}}"', from=3-1, to=3-3]
			\arrow["{h_{\alpha}}"', from=3-3, to=3-5]
			\arrow["{g_{\alpha,\beta}}"', bend right, from=3-3, to=3-7]
			\arrow["{p_{\alpha,\beta}}"', from=3-5, to=3-7]
		\end{tikzcd}
	\]

	\[
		\begin{tikzcd}
			X && {\prod_{\alpha,\beta} Z_{\alpha,\beta}} && {Z_{\alpha,\beta}}
			\arrow["\ell", from=1-1, to=1-3]
			\arrow["{g_{\alpha,\beta} \circ f_{\alpha}}"', bend right, from=1-1, to=1-5]
			\arrow["{q_{\alpha,\beta}}", from=1-3, to=1-5]
		\end{tikzcd}
	\]

	Let \( U \in \mathscr{T} \) then there exists \( V \) open in \( \prod_{\alpha} Y_{\alpha} \) such that \( U = h^{-1}(V) \) (see Problem~\ref{problem:VI.1.4} and~\ref{problem:VI.1.1}). One can write \( V \) in terms of subbasic elements as follows
	\[
		V = \bigcup_{i\in I} \bigcap^{n(i)}_{k=1} p_{\alpha_{k}}^{-1}(V_{\alpha_{k}})
	\]

	in which \( V_{\alpha_{k}} \) is open in \( Y_{\alpha_{k}} \).

	As \( \mathscr{T}_{\alpha} \) is the projective limit topology on \( Y_{\alpha} \) determined by the maps \( g_{\alpha,\beta}: Y_{\alpha} \to Z_{\alpha,\beta} \), there is an open set \( W_{\alpha_{k}} \) in \( \prod_{\beta} Z_{\alpha_{k},\beta} \) such that \( V_{\alpha_{k}} = h_{\alpha}^{-1}(W_{\alpha_{k}}) \). The open set \( W_{\alpha_{k}} \) can be written in terms of subbasic elements as follows
	\[
		W_{\alpha_{k}} = \bigcup_{j \in J} \bigcap^{n(j)}_{r=1} p_{\alpha_{k},\beta_{r}}^{-1}(W_{\alpha_{k}, \beta_{r}})
	\]

	in which \( W_{\alpha_{k}, \beta_{r}} \) is open in \( Z_{\alpha_{k}, \beta_{r}} \).
	\begingroup
	\allowdisplaybreaks%
	\begin{align*}
		V             & = \bigcup_{i\in I} \bigcap^{n(i)}_{k=1} p_{\alpha_{k}}^{-1}(V_{\alpha_{k}})                                                                                                                         \\
		              & = \bigcup_{i\in I} \bigcap^{n(i)}_{k=1} p_{\alpha_{k}}^{-1}\left( h^{-1}_{\alpha_{k}}(W_{\alpha_{k}}) \right)                                                                                       \\
		              & = \bigcup_{i\in I} \bigcap^{n(i)}_{k=1} {(h_{\alpha_{k}} \circ p_{\alpha_{k}})}^{-1}(W_{\alpha_{k}})                                                                                                \\
		              & = \bigcup_{i\in I} \bigcap^{n(i)}_{k=1} {(h_{\alpha_{k}} \circ p_{\alpha_{k}})}^{-1} \left( \bigcup_{j \in J} \bigcap^{n(j)}_{r=1} p_{\alpha_{k},\beta_{r}}^{-1}(W_{\alpha_{k}, \beta_{r}}) \right) \\
		              & = \bigcup_{i\in I} \bigcap^{n(i)}_{k=1} \bigcup_{j\in J} \bigcap^{n(j)}_{r=1} {(p_{\alpha_{k},\beta_{r}} \circ h_{\alpha_{k}} \circ p_{\alpha_{k}})}^{-1}(W_{\alpha_{k},\beta_{r}})                 \\
		U = h^{-1}(V) & = \bigcup_{i\in I} \bigcap^{n(i)}_{k=1} \bigcup_{j\in J} \bigcap^{n(j)}_{r=1} {(p_{\alpha_{k},\beta_{r}} \circ h_{\alpha_{k}} \circ p_{\alpha_{k}} \circ h)}^{-1}(W_{\alpha_{k},\beta_{r}})         \\
		              & = \bigcup_{i\in I} \bigcap^{n(i)}_{k=1} \bigcup_{j\in J} \bigcap^{n(j)}_{r=1} {(g_{\alpha_{k},\beta_{r}} \circ f_{\alpha_{k}})}^{-1}(W_{\alpha_{k},\beta_{r}})                                      \\
		              & = \bigcup_{i\in I} \bigcap^{n(i)}_{k=1} \bigcup_{j\in J} \bigcap^{n(j)}_{r=1} {(q_{\alpha_{k},\beta_{r}} \circ \ell)}^{-1}(W_{\alpha_{k},\beta_{r}})                                                \\
		              & = \bigcup_{i\in I} \bigcap^{n(i)}_{k=1} \bigcup_{j\in J} \bigcap^{n(j)}_{r=1} \ell^{-1}q_{\alpha_{k},\beta_{r}}^{-1}(W_{\alpha_{k},\beta_{r}})                                                      \\
		              & = \ell^{-1}\left( \bigcup_{i\in I} \bigcap^{n(i)}_{k=1} \bigcup_{j\in J} \bigcap^{n(j)}_{r=1} q_{\alpha_{k},\beta_{r}}^{-1}(W_{\alpha_{k},\beta_{r}}) \right) \in \widetilde{\mathscr{T}}
	\end{align*}
	\endgroup

	Hence \( U \in \widetilde{\mathscr{T}} \), which means \( \mathscr{T} \subset \widetilde{\mathscr{T}} \).

	\bigskip
	Conversely, let \( U \in \widetilde{\mathscr{T}} \) then there exists \( W \) open in \( \prod_{\alpha,\beta} Z_{\alpha,\beta} \) such that \( U = \ell^{-1}(W) \).

	\( W \) can be written in terms of subbasic elements.
	\begingroup
	\allowdisplaybreaks%
	\begin{align*}
		U & = \ell^{-1}(W) = \ell^{-1}\left( \bigcup_{i\in I}\bigcap^{n(i)}_{r=1} q^{-1}_{\alpha_{r},\beta_{r}}(W_{\alpha_{r},\beta_{r}}) \right) \\
		  & = \bigcup_{i\in I}\bigcap^{n(i)}_{r=1} \ell^{-1}q^{-1}_{\alpha_{r},\beta_{r}}(W_{\alpha_{r},\beta_{r}})                               \\
		  & = \bigcup_{i\in I}\bigcap^{n(i)}_{r=1} {(q_{\alpha_{r},\beta_{r}}\circ \ell)}^{-1}(W_{\alpha_{r},\beta_{r}})                          \\
		  & = \bigcup_{i\in I}\bigcap^{n(i)}_{r=1} {(g_{\alpha_{r},\beta_{r}}\circ f_{\alpha_{r}})}^{-1}(W_{\alpha_{r},\beta_{r}})                \\
		  & = \bigcup_{i\in I}\bigcap^{n(i)}_{r=1} f_{\alpha_{r}}^{-1}(g_{\alpha_{r},\beta_{r}}^{-1}(W_{\alpha_{r},\beta_{r}})) \in \mathscr{T}
	\end{align*}
	\endgroup

	so \( \widetilde{\mathscr{T}} \subset \mathscr{T} \).

	Thus \( \mathscr{T} = \widetilde{\mathscr{T}} \).
\end{proof}

\begin{problem}{VI.3.3}
Let \(X\) have the projective limit topology determined by \( \left\{ Y_{\alpha}, f_{\alpha} \mid \alpha \in \mathscr{A} \right\} \). Prove: \( f: Z \to X \) is continuous if and only if each \( f_{\alpha} \circ f \) is continuous.
\end{problem}

\begin{proof}
	For each \( \alpha \), the map \( f_{\alpha}: X \to Y_{\alpha} \) is continuous.

	If \( f \) is continuous then each \( f_{\alpha} \circ f \) is continuous.

	Conversely, assume that each \( f_{\alpha} \circ f \) is continuous. Let \( U \) be an open set in \( X \).
	\[
		\bigcup_{\alpha} f_{\alpha}^{-1}(\mathscr{T}_{\alpha})
	\]

	is a subbasis for the projective limit topology on \( X \). Therefore \( U \) can be written as
	\[
		U = \bigcup_{\gamma} \bigcap^{n(\gamma)}_{k=1} f_{\gamma,k}^{-1}(U_{\gamma,k})
	\]

	in which \( U_{\gamma,k} \) is open in \( Y_{\gamma,k} \) so
	\begingroup
	\allowdisplaybreaks%
	\begin{align*}
		f^{-1}(U) & = f^{-1}\left( \bigcup_{\gamma} \bigcap^{n(\gamma)}_{k=1} f_{\gamma,k}^{-1}(U_{\gamma,k}) \right) \\
		          & = \bigcup_{\gamma} \bigcap^{n(\gamma)}_{k=1} f^{-1}(f_{\gamma,k}^{-1}(U_{\gamma,k}))              \\
		          & = \bigcup_{\gamma} \bigcap^{n(\gamma)}_{k=1} {(f_{\gamma,k} \circ f)}^{-1}(U_{\gamma,k})
	\end{align*}
	\endgroup

	\( {(f_{\gamma,k} \circ f)}^{-1}(U_{\gamma,k}) \) is open as \( U_{\gamma,k} \) is open in \( Y_{\gamma,k} \) and \( f_{\gamma,k} \circ f \) is continuous. Hence \( f^{-1}(U) \) is open (finite intersection then arbitrary union), so \( f \) is continuous.

	Thus \( f \) is continuous if and only if each \( f_{\alpha} \circ f \) is continuous.
\end{proof}

\section{Spaces with Equivalence Relations}

\section{Cones and Suspensions}

\section{Attaching of Spaces}

\section{The Relation \(K(f)\) of Continuous Maps}

\section{Weak Topologies}


\end{document}
