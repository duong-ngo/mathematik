\chapter{General Topology}

\section{Metric Spaces}

\subsection*{Problems}

\begin{enumerate}[itemsep=0pt,label={\arabic*.}]
	\item Consider the set \( X \) of all continuous real-valued functions on \( [0, 1] \). Show that
	      \[
		      \operatorname{dist}(f, g) = \int_{0}^{1} \left\vert f(x) - g(x) \right\vert dx
	      \]

	      defines a metric on \( X \). Is this still the case if continuity is weakened to integrability?

	      \begin{proof}
		      If \( f, g \) are continuous then so is \( \left\vert f(x) - g(x) \right\vert \), hence \( \operatorname{dist}(f, g) \) is well-defined.

		      \( \left\vert f(x) - g(x) \right\vert \ge 0 \) for all \( x \in [0, 1] \) so \( \operatorname{dist}(f, g) = \int_{0}^{1} \left\vert f(x) - g(x) \right\vert dx \ge 0 \). Moreover, if \( \operatorname{dist}(f, g) = \int_{0}^{1} \left\vert f(x) - g(x) \right\vert dx = 0 \) then \( f(x) = g(x) \) for every \( x \in [0, 1] \). Hence \( \operatorname{dist} \) is positive definite.

		      \( \int_{0}^{1} \left\vert f(x) - g(x) \right\vert dx = \int_{0}^{1} \left\vert g(x) - f(x) \right\vert dx \) so \( \operatorname{dist} \) is symmetric.

		      For any continuous real-valued functions \( f, g, h \) on \( [0, 1] \) and \( x \in [0, 1] \)
		      \[
			      \left\vert f(x) - g(x) \right\vert \left\vert f(x) - h(x) \right\vert + \left\vert h(x) - g(x) \right\vert
		      \]

		      so
		      \[
			      \int_{0}^{1} \left\vert f(x) - g(x) \right\vert dx \le \int_{0}^{1} \left\vert f(x) - h(x) \right\vert dx + \int_{0}^{1} \left\vert h(x) - g(x) \right\vert dx
		      \]

		      which means \( \operatorname{dist}(f, g) \le \operatorname{dist}(f, h) + \operatorname{dist}(h, g) \).

		      Thus \( \operatorname{dist} \) defines a metric on \( X \).

		      If continuity is weakened to integrability then \( \operatorname{dist} \) is not positive definite anymore and it becomes a pseudometric.
	      \end{proof}
	\item If \( X \) is a metric space and \( x_{0} \) is a given point in \( X \), show that the function \( f: X \to \mathbb{R} \) given by \( f(x) = \operatorname{dist}(x, x_{0}) \) is continuous.
	      \begin{proof}
		      Let \( \varepsilon > 0 \) and choose \( \delta = \varepsilon \).

		      If \( \operatorname{dist}(x, y) < \delta \) then \( \left\vert f(x) - f(y) \right\vert = \left\vert \operatorname{dist}(x, x_{0}) - \operatorname{dist}(y, x_{0}) \right\vert \le \operatorname{dist}(x, y) < \varepsilon \).

		      So \( f \) is continuous at every point \( x \), hence continuous (in fact, this proof shows that \( f \) is Lipschitz continuous).
	      \end{proof}
	\item If \( A \) is a subset of a metric space \( X \) then define a real-valued function \( d \) on \( X \) by \( d(x) = \operatorname{dist}(x, A) = \inf\left\{ \operatorname{dist}(x, y) \mid y \in A \right\} \). Show that \( d \) is continuous.
	      \begin{proof}
		      For every \( x_{1}, x_{2} \in X \) and \( a \in A \),
		      \[
			      d(x_{1}) \le \operatorname{dist}(x_{1}, a) \le \operatorname{dist}(x_{1}, x_{2}) + \operatorname{dist}(x_{2}, a)
		      \]

		      so
		      \[
			      d(x_{1}) - \operatorname{dist}(x_{2}, a) \le \operatorname{dist}(x_{1}, x_{2}).
		      \]

		      Moreover
		      \[
			      \sup\left\{ d(x_{1}) - \operatorname{dist}(x_{2}, a) \mid a \in A \right\} = d(x_{1}) - \inf\left\{ \operatorname{dist}(x_{2}, a) \mid a \in A \right\} = d(x_{1}) - d(x_{2})
		      \]

		      and \( \operatorname{dist}(x_{1}, x_{2}) \) is an upper bound of \( \left\{ d(x_{1}) - \operatorname{dist}(x_{2}, a) \mid a \in A \right\} \). Therefore \( d(x_{1}) - d(x_{2}) \le \operatorname{dist}(x_{1}, x_{2}) \). Analogously, \( d(x_{2}) - d(x_{1}) \le \operatorname{dist}(x_{1}, x_{2}) \), so \( \left\vert d(x_{1}) - d(x_{2}) \right\vert \le \operatorname{dist}(x_{1}, x_{2}) \).

		      Let \( \varepsilon > 0 \) and \( \delta = \varepsilon \).

		      If \( \operatorname{dist}(x_{1}, x_{2}) < \delta \) then \( \left\vert d(x_{1}) - d(x_{2}) \right\vert \le \operatorname{dist}(x_{1}, x_{2}) < \varepsilon \). Hence \( d \) is continuous (in fact, it is also Lipschitz continuous).
	      \end{proof}
\end{enumerate}

\section{Topological Spaces}

\subsection*{Problems}

\begin{enumerate}[itemsep=0pt,label={\arabic*.}]
	\item Show that in a topological space \( X \):
	      \begin{enumerate}[itemsep=0pt,label={(\alph*)}]
		      \item the union of two closed sets is closed;
		      \item the intersection of any collection of closed sets is closed; and
		      \item the empty set \( \varnothing \) and whole space \( X \) are closed.
	      \end{enumerate}

	      \begin{proof}
		      The proof is straightforward as ``closed sets'' are complements of ``open sets'' and one can use De Morgan's formulae.
	      \end{proof}
	\item Consider the topology on the real line generated by the half open intervals \( \halfopenright{x, y} \) \textit{together with} those of the form \( \halfopenleft{x, y} \). Show that this coincides with the discrete topology.

	      \begin{proof}
		      Let \( a \) be a real number, then \( \halfopenleft{a - 1, a}, \halfopenright{a, a + 1} \) are open sets, so \( \left\{ a \right\} = \halfopenleft{a - 1, a} \cap \halfopenright{a, a + 1} \) is open. Hence this coincides with the discrete topology (every subset of \(\mathbb{R}\) is open).
	      \end{proof}
	\item Show that the space \( \Omega \cup \left\{ \Omega \right\} \) in the order topology cannot be given a metric consistent with its topology.
	      \begin{proof}
		      Assume that there is a countable neighborhood basis \( \mathbf{N}(\Omega) \) at \( \Omega \).

		      Each \( U \in \mathbf{N}(\Omega) \) has a least element \( x \) such that \( \halfopenleft{x, \Omega} \subset U \) (since \( \Omega \cup \left\{ \Omega \right\} \) is well-ordered). Consider the set \( S \) consisting of such elements. The ordinal \( \sup S + 1 \) is also countable and \( \halfopenleft{\sup S + 1, \Omega} \) is not contained in any member of \( \mathbf{N}(\Omega) \), which is a contradiction.

		      Hence \( \Omega \cup \left\{ \Omega \right\} \) is not first-countable, so it cannot be given a metric consistent with the order topology on it.
	      \end{proof}
	\item If \( f: X \to Y \) is a function between topological spaces, and \( f^{-1}(U) \) is open for each open \( U \) in some subbasis for the topology of \( Y \), show that \( f \) is continuous.
	      \begin{proof}
		      Let \( \mathbf{S} \) be a subbasis for the topology of \( Y \) and \( f^{-1}(U) \) is open for each \( U \in \mathbf{S} \).

		      Let \( \mathbf{B} \) be the collection of the intersections of finite elements in \( \mathbf{S} \). For each \( B \in \mathbf{B} \), there exist \( U_{1}, \ldots, U_{n} \in \mathbf{S} \) such that \( B = \bigcap_{i=1}^{n} U_{i} \).
		      \[
			      f^{-1}(B) = f^{-1}\left(\bigcap_{i=1}^{n} U_{i}\right) = \bigcap_{i=1}^{n} f^{-1}(U_{i})
		      \]

		      is then open in \( X \).

		      For each open set \( V \) in \( Y \), there exists a collection of basic open sets \( {(B_{\alpha})}_{\alpha \in \mathscr{A}} \) in \( \mathbf{B} \) such that \( V = \bigcup_{\alpha} B_{\alpha} \).
		      \[
			      f^{-1}(V) = f^{-1}\left( \bigcup_{\alpha \in \mathscr{A}} B_{\alpha} \right) = \bigcup_{\alpha \in \mathscr{A}} f^{-1}(B_{\alpha})
		      \]

		      is then open in \( X \).

		      Thus \( f \) is continuous.
	      \end{proof}
	\item Suppose that \( S \) is a set and that we are given, for each \( x \in S \), a collection \( \mathbf{N}(x) \) of subsets of \( S \) satisfying:
	      \begin{enumerate}[itemsep=0pt,label={(\arabic*)}]
		      \item \( N \in \mathbf{N}(x) \implies x \in N \);
		      \item \( N, M \in \mathbf{N}(x) \implies \exists P \in \mathbf{N}(x) \ni P \subseteq N \cap M \); and
		      \item \( x \in S \implies \mathbf{N}(x) \ne \varnothing \).
	      \end{enumerate}

	      Then show that there is a unique topology on \( S \) such that \( \mathbf{N}(x) \) is a neighborhood basis at \( x \), for each \( x \in S \). (Thus a topology can be defined by the specification of such a collection of neighborhoods at each point.)

	      \begin{proof}
		      Let \( \mathcal{O} \) be the collection of subsets of \( S \) such that for each \( O \in \mathcal{O} \), for each \( x \in O \), there exists \( N \in \mathbf{N}(x) \) contained in \( O \).

		      Vacuously, \( \varnothing \in \mathcal{O} \). By definition, \( S \in \mathcal{O} \), the union of arbitrarily many sets in \( \mathcal{O} \) is also in \( \mathcal{O} \). From (2), it follows that the intersection of two sets in \( \mathcal{O} \) is again in \( \mathcal{O} \). So the given topology is such that \( \mathbf{N}(x) \) is a neighborhood basis at \( x \), for each \( x \in S \).

		      Assume that \( \mathcal{O}^{\prime} \) is a topology such that \( \mathbf{N}(x) \) is a neighborhood basis at \( x \), for each \( x \in S \).

		      Let \( U \in \mathcal{O}^{\prime} \) then \( U \) is either empty or nonempty. If \( U \) is empty then \( U \in \mathcal{O} \). Otherwise, for each \( x \in U \), there exists \( N \in \mathbf{N}(x) \) such that \( x \in N \subseteq U \) because \( \mathbf{N}(x) \) is a neighborhood basis at \( x \) in \( (S, \mathcal{O}^{\prime}) \). Hence \( \mathcal{O}^{\prime} \subseteq \mathcal{O} \).

		      Conversely, let \( U \in \mathcal{O} \). For every \( x \in U \), there exists \( N \in \mathbf{N}(x) \) such that \( x \in N \subseteq U \). Since \( \mathbf{N}(x) \) is a neighborhood basis at \( x \), we conclude that \( x \) is in the \( \mathcal{O}^{\prime} \)-interior of \( U \). Hence \( \mathcal{O} \subseteq \mathcal{O}^{\prime} \).

		      Thus \( \mathcal{O} = \mathcal{O}^{\prime} \).
	      \end{proof}
\end{enumerate}

\section{Subspaces}

\begin{enumerate}[itemsep=0pt,label={\arabic*.}]
	\item Let \( X \) be a topological space and \( A, B \subseteq X \).
	      \begin{enumerate}[itemsep=0pt,label={(\alph*)}]
		      \item Show that
		            \[
			            \operatorname{int}(A) = \left\{ a \in X \mid \exists U \text{ open } \ni a \in U \subseteq A \right\}
		            \]

		            and
		            \[
			            \overline{A} = \left\{ x \in X \mid \forall U \text{ open with } x \in U, U \cap A \ne \varnothing \right\}.
		            \]
		      \item Show that \( A \) is open iff \( A = \operatorname{int}(A) \) and that \( A \) is closed iff \( A = \overline{A} \).
		      \item Show that \( X - \operatorname{int}(A) = \overline{X - A} \) and that \( X - \overline{A} = \operatorname{int}(X - A) \).
		      \item Show that \( \operatorname{int}(A \cap B) = \operatorname{int}(A) \cap \operatorname{int}(B) \) and that \( \overline{A \cup B} = \overline{A} \cup \overline{B} \).
		      \item Show that
		            \begin{gather*}
			            \bigcap \operatorname{int}(A_{\alpha}) \supseteq \operatorname{int}\left(\bigcap A_{\alpha}\right) = \operatorname{int}\left(  \bigcap \operatorname{int}(A_{\alpha}) \right), \\
			            \bigcup \overline{A_{\alpha}} \subseteq \overline{\bigcup A_{\alpha}} = \overline{\bigcup \overline{A_{\alpha}}}, \\
			            \bigcup \operatorname{int}(A_{\alpha}) \subseteq \operatorname{int}\left(\bigcup A_{\alpha}\right), \\
			            \bigcap \overline{A_{\alpha}} \supseteq \overline{\bigcap A_{\alpha}},
		            \end{gather*}

		            and give examples showing that these inclusions need not be equalities.
		      \item Show \( A \subseteq B \implies [\overline{A} \subseteq \overline{B} \text{ and } \operatorname{int}(A) \subseteq \operatorname{int}(B)] \).
	      \end{enumerate}

	      \begin{proof}
		      \begin{enumerate}[itemsep=0pt,label={(\alph*)}]
			      \item \( \left\{ a \in X \mid \exists U \text{ open } \ni a \in U \subseteq A \right\} \) is the union of all open sets contained in \( A \). Therefore
			            \[
				            \left\{ a \in X \mid \exists U \text{ open } \ni a \in U \subseteq A \right\} \subseteq \operatorname{int}(A)
			            \]

			            because \( \operatorname{int}(A) \) is the largest open set contained in \( A \). Moreover, by the definition of \( \left\{ a \in X \mid \exists U \text{ open } \ni a \in U \subseteq A \right\} \), this set contains \( \operatorname{int}(A) \).

			            Thus \( \operatorname{int}(A) = \left\{ a \in X \mid \exists U \text{ open } \ni a \in U \subseteq A \right\} \).
			            \begingroup
			            \allowdisplaybreaks%
			            \begin{align*}
				            x \notin \overline{A} & \iff x \in X - \overline{A}                                                                                 \\
				                                  & \iff \exists U \text{ open with } x \in U, U \subseteq X - A                                                \\
				                                  & \iff \exists U \text{ open with } x \in U, U \cap A = \varnothing                                           \\
				                                  & \iff x \notin \left\{ x \in X \mid \forall U \text{ open with } x \in U, U \cap A \ne \varnothing \right\}.
			            \end{align*}
			            \endgroup

			            Thus \( \overline{A} = \left\{ x \in X \mid \forall U \text{ open with } x \in U, U \cap A \ne \varnothing \right\} \).
			      \item If \( A = \operatorname{int}(A) \) then \( A \) is open. Conversely, if \( A \) is open, then \( A = \operatorname{int}(A) \) because \( A \) is the largest open set contained in \( A \).

			            If \( A = \overline{A} \) then \( A \) is closed. Conversely, if \( A \) is closed, then \( A = \overline{A} \) because \( A \) is the smallest closed set containing \( A \).
			      \item \begingroup
			            \allowdisplaybreaks%
			            \begin{align*}
				            x \in X - \operatorname{int}(A) & \iff x \notin \operatorname{int}(A)                                      \\
				                                            & \iff \forall U \text{ open } \ni x \in U \nsubseteq A                    \\
				                                            & \iff \forall U \text{ open } \ni x \in U, U \cap (X - A) \ne \varnothing \\
				                                            & \iff x \in \overline{X - A}.                                             \\
				            x \in X - \overline{A}          & \iff x \notin \overline{A}                                               \\
				                                            & \iff \exists U \text{ open with } x \in U, U \cap A = \varnothing        \\
				                                            & \iff \exists U \text{ open with } x \in U, U \subseteq X - A             \\
				                                            & \iff x \in \operatorname{int}(X - A).
			            \end{align*}
			            \endgroup

			            Thus \( X - \operatorname{int}(A) = \overline{X - A} \) and \( X - \overline{A} = \operatorname{int}(X - A) \).
			      \item \( \operatorname{int}(A) \cap \operatorname{int}(B) \) is an open set contained in \( A \cap B \) so \( \operatorname{int}(A) \cap \operatorname{int}(B) \subseteq \operatorname{int}(A \cap B) \). Conversely, if \( x \in \operatorname{int}(A \cap B) \) then there exists an open set \( U \ni x \) such that \( x \in U \subseteq A \cap B \), so \( x \in \operatorname{int}(A) \) and \( x \in \operatorname{int}(B) \). Hence \( \operatorname{int}(A \cap B) = \operatorname{int}(A) \cap \operatorname{int}(B) \).

			            \( \overline{A} \cup \overline{B} \) is a closed set containg \( A \cup B \) so \( \overline{A \cup B} \subseteq \overline{A} \cup \overline{B} \). Conversely, if \( x \in \overline{A} \cup \overline{B} \) then every neighborhood of \( x \) intersects \( A \), or every neighborhood of \( x \) intersects \( B \), which means \( x \in \overline{A \cup B} \). Hence \( \overline{A \cup B} = \overline{A} \cup \overline{B} \).
			      \item \begin{itemize}
				            \item If \( x \in \operatorname{int}\left( \bigcap A_{\alpha} \right) \) then there exists an open set \( U \) such that \( x \in U \subseteq \bigcap A_{\alpha} \), which means \( x \in \operatorname{int}(A_{\alpha}) \) for every \( \alpha \). Hence \( \bigcap \operatorname{int}(A_{\alpha}) \supseteq \operatorname{int}\left(\bigcap A_{\alpha}\right) \) and \( \operatorname{int}\left(\bigcap A_{\alpha}\right) \subseteq \operatorname{int}\left( \bigcap \operatorname{int}(A_{\alpha}) \right) \).

				                  Conversely, if \( x \in \operatorname{int}\left( \bigcap \operatorname{int}(A_{\alpha}) \right) \) then there exists an open set \( U \) such that \( x \in U \subseteq \bigcap \operatorname{int}(A_{\alpha}) \). Therefore \( x \in U \subseteq \bigcap A_{\alpha} \), which means \( x \in \operatorname{int}\left(\bigcap A_{\alpha}\right) \).

				                  Thus \( \operatorname{int}\left(\bigcap A_{\alpha}\right) = \operatorname{int}\left(  \bigcap \operatorname{int}(A_{\alpha}) \right) \).

				                  An example to show that the inclusion need not be equality:
				                  \begingroup
				                  \allowdisplaybreaks%
				                  \begin{align*}
					                  \bigcap_{n \in \mathbb{Z}^{+}} \operatorname{int}([-1/n, 1+1/n]) & = \bigcap_{n \in \mathbb{Z}^{+}} \openinterval{-1/n, 1+1/n} = [0, 1]                                         \\
					                                                                                   & \supsetneq \openinterval{0, 1} = \operatorname{int}\left(\bigcap_{n \in \mathbb{Z}^{+}} [-1/n, 1+1/n]\right)
				                  \end{align*}
				                  \endgroup
				            \item If \( x \in \bigcup \overline{A_{\alpha}} \) then there exists \( \alpha \) such that \( x \in \overline{A_{\alpha}} \). For every open set \( U \ni x \), \( U \cap A_{\alpha} \ne \varnothing \) so \( U \cap \bigcup A_{\alpha} \ne \varnothing \). Therefore \( x \in \overline{\bigcup A_{\alpha}} \), so \( \bigcup \overline{A_{\alpha}} \subseteq \overline{\bigcup A_{\alpha}} \).

				                  If \( x \in \overline{\bigcup A_{\alpha}} \) then every open set \( U \ni x \) intersects \( \bigcup A_{\alpha} \), so \( U \) intersects \( \bigcup \overline{A_{\alpha}} \), which means \( x \in \overline{\bigcup \overline{A_{\alpha}}} \).

				                  Conversely, if \( x \in \overline{\bigcup \overline{A_{\alpha}}} \) then every open set \( U \ni x \), \( U \cap \bigcup \overline{A_{\alpha}} = \varnothing \) so there exists \( \alpha \) such that \( U \cap \overline{A_{\alpha}} \ne \varnothing \). Therefore \( x \notin \operatorname{int}(X - A_{\alpha}) \), which means \( x \in \overline{A_{\alpha}} \) and \( U \cap A_{\alpha} \ne \varnothing \). Therefore \( U \cap \bigcup A_{\alpha} \ne \varnothing \). Hence \( x \in \overline{\bigcup A_{\alpha}} \).

				                  Thus \( \overline{\bigcup A_{\alpha}} = \overline{\bigcup \overline{A_{\alpha}}} \).

				                  An example to show that the inclusion need not be equality:
				                  \begingroup
				                  \allowdisplaybreaks%
				                  \begin{align*}
					                  \bigcup_{n\in\mathbb{Z}^{+}} \overline{\openinterval{1/n, 1 - 1/n}} & = \bigcup_{n\in\mathbb{Z}^{+}} \closedinterval{1/n, 1 - 1/n} = \openinterval{0, 1}                                                      \\
					                                                                                      & \subsetneq \closedinterval{0, 1} = \overline{\openinterval{0,1}} = \overline{\bigcup_{n\in\mathbb{Z}^{+}} \openinterval{1/n, 1 - 1/n}}.
				                  \end{align*}
				                  \endgroup
				            \item \( \operatorname{int}(A_{\alpha}) \subseteq A_{\alpha} \subseteq \bigcup A_{\alpha} \) for every \( \alpha \) so \( \operatorname{int}(A_{\alpha}) \subseteq \operatorname{int}\left( \bigcup A_{\alpha} \right) \) for every \( \alpha \). Therefore \( \bigcup \operatorname{int}(A_{\alpha}) \subseteq \operatorname{int}\left(\bigcup A_{\alpha}\right) \).

				                  An example to show that the inclusion need not be equality:
				                  \begingroup
				                  \allowdisplaybreaks%
				                  \begin{align*}
					                  \operatorname{int}(\closedinterval{-1, 0}) \cup \operatorname{int}(\closedinterval{0, 1}) & = \openinterval{-1, 0} \cup \openinterval{0, 1}                                                                                                      \\
					                                                                                                            & \subsetneq \openinterval{-1, 1} = \operatorname{int}(\closedinterval{-1, 1}) = \operatorname{int}(\closedinterval{-1,0} \cup \closedinterval{0, 1}).
				                  \end{align*}
				                  \endgroup
				            \item \( \overline{A_{\alpha}} \supseteq A_{\alpha} \supseteq \bigcap A_{\alpha} \) so \( \overline{A_{\alpha}} \supseteq \overline{\bigcap A_{\alpha}} \) for every \( \alpha \). Therefore \( \bigcap \overline{A_{\alpha}} \supseteq \overline{\bigcap A_{\alpha}} \).

				                  An example to show that the inclusion need not be equality:
				                  \begingroup
				                  \allowdisplaybreaks%
				                  \begin{align*}
					                  \overline{\openinterval{-1, 0}} \cap \overline{\openinterval{0, 1}} & = \closedinterval{-1, 0} \cap \closedinterval{0, 1} = \left\{ 0 \right\}          \\
					                                                                                      & \supsetneq \varnothing = \overline{\openinterval{-1,0} \cap \openinterval{0, 1}}.
				                  \end{align*}
				                  \endgroup
			            \end{itemize}
			      \item Assume \( A \subseteq B \).

			            \( \operatorname{int}(A) \subseteq A \subseteq B \) and \( \operatorname{int}(A) \) is open so \( \operatorname{int}(A) \) is not larger than the largest open set contained in \( B \), which means \( \operatorname{int}(A) \subseteq \operatorname{int}(B) \).

			            \( \overline{B} \supseteq B \supseteq A \) and \( \overline{B} \) is closed so \( \overline{B} \) is not smaller than the smallest closed set containing \( A \), which implies \( \overline{B} \supseteq \overline{A} \).
		      \end{enumerate}
	      \end{proof}
	\item For \( A \subseteq X \), a topological space, show that \( X \) is the \textit{disjoint} union of \( \operatorname{int}(A), \operatorname{bdry}(A) \), and \( X - \overline{A} \).
	      \begin{proof}
		      From the definition of interior, closure, and boundary, \( \overline{A} = \operatorname{int}(A) \cup \operatorname{bdry}(A) \)

		      Let \( x \in \overline{A} \), the following cases are disjoint and exhausted:
		      \begin{itemize}
			      \item There exists an open set \( U \ni x \) such that \( U \subseteq A \) --- This means \( x \in \operatorname{int}(A) \).
			      \item For all open sets \( U \ni x \), \( U \cap (X - A) \ne \varnothing \) --- This means \( x \in \operatorname{bdry}(A) \).
		      \end{itemize}

		      So \( \operatorname{int}(A) \) and \( \operatorname{bdry}(A) \) are disjoint.

		      Hence \( X \) is the disjoint union of \( \operatorname{int}(A), \operatorname{bdry}(A) \), and \( X - \overline{A} \).
	      \end{proof}
	\item Show that a metric space is second countable iff it has a countable dense set.
	      \begin{proof}
		      This proof assumes the axiom of countable choice.

		      Let \( (M, d) \) be a metric space.

		      Assume that \( M \) is second countable. Let \( \mathbf{B} \) be a countable basis for \( M \). For each \( B \in \mathbf{B} \), there exists \( x_{B} \in B \) and let \( D = \left\{ x_{B} \mid B \in \mathbf{B} \right\} \). Each nonempty open set \( U \) in \( M \) is a union of basic open sets in \( \mathbf{B} \), so \( U \cap D \ne \varnothing \), which means \( \overline{D} = M \). Hence \( M \) has a countable dense set.

		      Assume that \( M \) has a countable dense set \( D \). Define \( \mathbf{B} \) to be the set of open balls where each ball is centered at a point in \( D \) and has rational radius then \( \mathbf{B} \) is countable.

		      Let \( U \) be a nonempty open set in \( M \). For every \( x \in U \), there exists \( \varepsilon \) such that \( B_{\varepsilon}(x) \subseteq U \). Since \( D \) is dense in \( M \), there exists \( y \in B_{\varepsilon/2}(x) \cap D \). Let \( r \) be a rational number such that \( \operatorname{dist}(x, y) < r < \varepsilon/2 \). For every \( z \in B_{r}(y) \)
		      \[
			      \operatorname{dist}(x, z) \le \operatorname{dist}(x, y) + \operatorname{dist}(y, z) < \operatorname{dist}(x, y) + r < 2r < \varepsilon.
		      \]

		      Moreover, \( x \in B_{r}(y) \) so \( x \in B_{r}(y) \subseteq U \), where \( B_{r}(y) \in \mathbf{B} \). So \( \mathbf{B} \) is a countable basis for \( M \), which means \( M \) is second countable.
	      \end{proof}
	\item Show that the union of two nowhere dense sets is nowhere dense.
	      \begin{proof}
		      Let \( A, B \) be two nowhere dense sets then \( \operatorname{int}(\overline{A}) = \operatorname{int}(\overline{B}) = \varnothing \).

		      \( X = X - \operatorname{int}(\overline{A}) = \overline{X - \overline{A}} \) and \( X = X - \operatorname{int}(\overline{B}) = \overline{X - \overline{B}} \) so \( X - \overline{A} \) and \( X - \overline{B} \) are dense.

		      Let \( U \) be a nonempty open set, then \( U \cap (X - \overline{A}) \) is a nonempty open set, so \( U \cap (X - \overline{A}) \cap (X - \overline{B}) \) is nonempty.
		      \[
			      \varnothing \ne U \cap (X - \overline{A}) \cap (X - \overline{B}) = U \cap (X - (\overline{A} \cup \overline{B})) = U \cap (X - \overline{A \cup B}).
		      \]

		      Therefore \( X - \overline{A \cup B} \) is dense in \( X \) and
		      \[
			      \operatorname{int}(\overline{A \cup B}) = X - \overline{X - \overline{A \cup B}} = \varnothing.
		      \]

		      Thus \( A \cup B \) is nowhere dense.
	      \end{proof}
	\item A topological space \( X \) is said to be ``irreducible'' if, whenever \( X = F \cup G \) with \( F \) and \( G \) closed, then either \( X = F \) or \( X = G \). A subspace is irreducible if it is so in the subspace topology. Show that if \( X \) is irreducible and \( U \subseteq X \) is open, then \( U \) is irreducible.
	      \begin{proof}
		      If \( U = \varnothing \) then it is irreducible.

		      If \( U \) is nonempty, assume that \( U = A \cup B \), where \( A, B \) are closed subsets of \( U \).

		      There exists closed subsets \( F, G \) of \( X \) such that \( A = U \cap F \) and \( B = U \cap G \).

		      \( X = F \cup G \cup (X - A) \) so \( X = F \) or \( X = G \cup (X - A) \).

		      If \( X = F \) then \( U = U \cap X = U \cap F = A \).

		      Otherwise, \( X = G \cup (X - A) \) implies \( X = G \) because \( X \) is irreducible. So \( B = U \cap G = U \cap X = U \).

		      Hence \( U \) is irreducible.
	      \end{proof}
	\item A ``Zariski space'' is a topological space with the property that every descending chain \( F_{1} \supseteq F_{2} \supseteq F_{3} \supseteq \cdots \) of closed sets is eventually constant. Show that every Zariski space can be expressed as a finite union \( X = \bigcup_{i=1}^{n} Y_{i} \) where the \( Y_{i} \) are closed and irreducible and \( Y_{i} \nsubseteq Y_{j} \) for \( i \ne j \). Also show that this decomposition is unique up to order.
	      \begin{proof}
		      A topological space is irreducible iff any two nonempty open subsets are intersecting.

		      Let \( S \) be a nonempty irreducible subspace of \( X \). We consider the set \( \mathcal{I} \) of irreducible subspaces of \( X \) containing \( S \) then \( \mathcal{I} \) is nonempty for \( S \in \mathcal{I} \).

		      Let \( (B_{\alpha}) \) be a chain in \( \mathcal{I} \) and \( B = \bigcup B_{\alpha} \). Let \( U, V \) be two nonempty open sets in \( B \). There are \( B_{\beta}, B_{\gamma} \) in the chain such that \( U \cap B_{\beta} \ne \varnothing \) and \( V \cap B_{\gamma} \ne \varnothing \). Because \( B_{\beta}, B_{\gamma} \) belong to a chain, then \( B_{\beta} \subseteq B_{\gamma} \) or \( B_{\gamma} \subseteq B_{\beta} \). Without loss of generality, assume that \( B_{\beta} \subseteq B_{\gamma} \) then \( U \cap B_{\gamma} \ne \varnothing \). Because \( B_{\gamma} \) is nonempty, \( U \cap B_{\gamma} \cap V \cap B_{\gamma} \) is nonempty, so \( U \cap V \) is nonempty. Hence \( B \) is irreducible.

		      So every chain in \( \mathcal{I} \) has an upper bound. Therefore \( \mathcal{I} \) has a maximal irreducible set.

		      \bigskip

		      Now we will show that the closure of an irreducible set is also irreducible.

		      Assume \( S \subseteq X \) is irreducible and let \( U, V \) be two nonempty open sets in \( \overline{S} \). Because \( U, V \) are nonempty, then \( U \cap S \ne \varnothing, V \cap S \ne \varnothing \), because the neighborhood of any point in \( \overline{S} \) intersects \( S \). As \( S \) is irreducible, \( U \cap S, V \cap S \) are intersecting, so \( U \cap V \ne \varnothing \). Therefore \( \overline{S} \) is irreducible.

		      Consequently, every maximal irreducible set is closed.

		      \bigskip

		      In a Zariski space \( X \), every singleton subset is irreducible, hence contained in some maximal irreducible set.

		      Let \( \mathscr{C} \) be the collection of closed subsets of \( X \) which is not a finite union of irreducible sets. Assume for the sake of contrary that \( \mathscr{C} \) is nonempty. By the descending chain condition and Zorn's lemma, we deduce that \( \mathscr{C} \) has a minimal element, denote it by \( C \). Since \( C \) is not a finite union of irreducible sets so \( C \) is not irreducible, hence \( C = C_{1} \cup C_{2} \) for some proper closed subsets \( C_{1}, C_{2} \) in \( C \). Because \( C_{i} \subseteq_{\text{closed}} C \subseteq_{\text{closed}} X \), it follows that \( C_{1}, C_{2} \) are closed in \( X \). Due to the minimality of \( C \), both \( C_{1}, C_{2} \) are finite unions of irreducible sets. Therefore \( C = C_{1} \cup C_{2} \) is a finite union of irreducible sets, which is a contradiction. Hence \( \mathscr{C} \) is empty.

		      We proved that every closed subset of \( X \) is a finite union of irreducible sets, so \( X \) is a finite union of irreducible sets. Because every irreducible set is contained in some maximal irreducible set, \( X \) is a finite union of maximal irreducible sets. According to the maximality, \( X \) is a finite union of maximal irreducible sets where each maximal irreducible set is not contained in any others.

		      Assume that \( X = \bigcup_{i=1}^{n} Y_{i} = \bigcup_{j=1}^{m} Z_{j} \) where \( Y_{i}, Z_{j} \) are maximal irreducible sets. For every \( Y_{i} \), we have \( Y_{i} = \bigcup_{j=1}^{m} Y_{i} \cap Z_{j} \). Because \( Y_{i} \cap Z_{j} \) are closed in \( Y_{i} \) and \( Y_{i} \) is irreducible, \( Y_{i} \) is contained in some \( Z_{j} \). Since these are maximal irreducible setss, \( Y_{i} = Z_{j} \). Hence each maximal irreducible set in the first decomposition must be equal to one in the second decomposition and vice versa, from which we conclude that the two decomposition are the same up to order.
	      \end{proof}
	\item Let \( X \) be the real line with the topology for which the open sets are \( \varnothing \) together with the complements of finite subsets. Show that \( X \) is an irreducible Zariski space.
	      \begin{proof}
		      Assume that \( F_{1} \supseteq F_{2} \supseteq F_{3} \supseteq \cdots \) is a descending chain of closed sets in \( X \).

		      If all of the closed sets in the given chain is infinite then the chain is eventually constant.

		      Otherwise, there exists a finite closed set in the chain, so the chain is eventually constant.

		      Therefore \( X \) is a Zariski space.

		      Assume that \( X = A \cup B \) where \( A, B \) are closed subsets of \( X \). According to the definition of the topology on \( X \), a closed set in \( X \) is either finite or identical to \( X \). If \( A \) is finite then \( B \) must not be finite because \( A \cup B \) is infinite, so \( B = X \). Otherwise, \( A = X \). So \( X \) is irreducible.

		      Thus \( X \) is an irreducible Zariski space.
	      \end{proof}
	\item Let \( X = A \cup B \), where \( A \) and \( B \) are closed. Let \( f: X \to Y \) be a function. If the restrictions of \( f \) to \( A \) and \( B \) are both continuous then show that \( f \) is continuous.
	      \begin{proof}
		      Let \( G \) be a closed set in \( Y \).

		      \( f\vert_{A}, f\vert_{B} \) are continuous so \( {(f\vert_{A})}^{-1}(G) \subseteq A \) is closed and \( {(f\vert_{B})}^{-1}(G) \subseteq B \) is closed.
		      \[
			      f^{-1}(G) = (f^{-1}(G) \cap A) \cup (f^{-1}(G) \cap B) = {(f\vert_{A})}^{-1}(G) \cup {(f\vert_{B})}^{-1}(G).
		      \]

		      Because \( {(f\vert_{A})}^{-1}(G) \subseteq_{\text{closed}} A \subseteq_{\text{closed}} X \) and \( {(f\vert_{B})}^{-1}(G) \subseteq_{\text{closed}} B \subseteq_{\text{closed}} X \), it follows that \( {(f\vert_{A})}^{-1}(G), {(f\vert_{B})}^{-1}(G) \) are closed in \( X \). Hence \( f^{-1}(G) \) is closed in \( X \).

		      Thus \( f \) is continuous.
	      \end{proof}
\end{enumerate}

\section{Connectivity and Components}

\begin{enumerate}[label={\arabic*.}]
	\item If \( A \) is a connected subset of the topological space \( X \) and if \( A \subseteq B \subseteq \overline{A} \) then show that \( B \) is connected.
	      \begin{proof}
		      Let \( f: X \to 2 \) be a continuous map onto a discrete space of two elements \( 0 \) and \( 1 \).

		      Because \( A \) is connected, the restriction \( f\vert_{A} \) is a constant map. If \( x \in B \), the preimage \( f^{-1}(f(x)) \) is an open neighborhood of \( x \) and it intersects \( A \). Let \( y \) be a common point of \( A \) and \( f^{-1}(f(x)) \) then \( f(y) = f(x) \). Therefore \( f\vert_{B} \) are constant maps.

		      Hence \( B \) is connected.
	      \end{proof}
	\item A space \( X \) is said to be ``locally connected'' if for each \( x \in X \) and each neighborhood \( N \) of \( x \), there is a connected neighborhood \( V \) of \( x \) with \( V \subseteq N \). If \( X \) is locally connected, show that its components are open and equal its quasi-components.
	      \begin{proof}
		      Let \( C \) be a component of \( X \).

		      For each \( x \in C \), there is a connected neighborhood \( V \) with \( x \in V \subseteq C \). Therefore \( x \in \operatorname{int}(C) \) for every \( x \in C \), so \( C \) is open.

		      In a locally connected space, every component is clopen.

		      Each component of \( X \) is contained in a unique quasi-component. Let \( D \) be the quasi-component containing \( C \) and \( x \in C \). From the definition of a quasi-component
		      \[ D = \bigcap \left\{ f^{-1}(f(x)) \mid f \text{ is a continuous discrete-valued map} \right\}. \]

		      \( C \subseteq_{\text{open}} X \) and \( C \subseteq D \) so \( C \subseteq_{\text{open}} D \) and \( D - C \subseteq_{\text{closed}} D \). Moreover, \( C \subseteq_{\text{closed}} X \) so \( C \subseteq_{\text{closed}} D \). So \( C \) and \( D - C \) are closed in \( D \).

		      Assume that there exists \( y \in D - C \) then let \( d_{0}: C \to \left\{ 0, 1 \right\} \) and \( d_{1}: D - C \to \left\{ 0, 1 \right\} \) be continuous constant maps such that \( d_{0} \equiv 0 \) and \( d_{1} \equiv 1 \). Since the domains of these two maps are disjoint, then there exists a continuous map \( d: D \to \left\{ 0, 1 \right\} \) such that \( d\vert_{C} = d_{0} \) and \( d\vert_{D - C} = d_{1} \), according to the gluing lemma (Chapter 1, Section 3, Problem 8). Therefore \( d \) is a continuous discrete-valued map on \( D \) which is not a constant map and this contradicts the definition of a quasi-component.

		      Hence \( D = C \), so in a locally connected space, every component equals to the quasi-component containing it.
	      \end{proof}
	\item Show that the unit interval \( [0, 1] \) in the real number is connected.
	      \begin{proof}
		      Suppose on the contrary that \( \openinterval{0, 1} \) is not connected, then there exists disjoint proper open subsets \( U, V \) of \( \openinterval{0, 1} \) such that \( \openinterval{0, 1} = U \cup V \).

		      Without loss of generality, assume that \( 1 \in V \). Let \( x = \sup U \) then \( x \) is a limit point of \( U \), so \( x \) is in the closure of \( U \) in \( \openinterval{0, 1} \). Moreover, \( U = [0,1] - V \) is closed so \( x \in U \).

		      Since \( U \subseteq \openinterval{0, 1} \) is open then \( U \) is open in \( \mathbb{R}^{1} \), so there exists \( \varepsilon > 0 \) such that \( \openinterval{x - \varepsilon, x + \varepsilon} \subseteq U \subseteq \openinterval{0, 1} \), which contradicts \( x \) being the supremum of \( U \).

		      Hence \( \openinterval{0, 1} \) is connected, and \( [0, 1] = \overline{\openinterval{0, 1}} \) is connected.
	      \end{proof}
	\item Consider the subspace \( X \) of the unit square in the plane consisting of the vertical line segments \( \left\{ 1/n \right\} \times [0,1] \) for \( n = 1, 2, 3, \ldots \), and the two points \( (0, 0) \) and \( (0, 1) \). Show that the latter two points are components of \( X \) but not quasi-components. Show that the two point set \( \left\{ (0, 0), (0, 1) \right\} \) is a quasi-component which is not connected.
	      \begin{proof}
		      Each \( \left\{ 1/n \right\} \times [0, 1] \) is connected as it is homeomorphic to \( [0, 1] \). Assume that \( \left\{ 1/n \right\} \times [0, 1] \) is not a component of \( X \) then there exists \( X \ni (x, y) \notin \left\{ 1/n \right\} \times [0, 1] \) such that \( \left\{ 1/n \right\} \times [0, 1] \cup \left\{ (x, y) \right\} \) is connected.

		      Either \( x < 1/n \) or \( x > 1/n \). Let \( c \) be a real number between \( x \) and \( 1/n \). The set \( S = \left\{ 1/n \right\} \times [0, 1] \cup \left\{ (x, y) \right\} \) is disconnected by \( \left\{ (p_{1}, p_{2}) \in \mathbb{R}^{2} \mid p_{1} > c \right\} \cap S \) and \( \left\{ (p_{1}, p_{2}) \in \mathbb{R}^{2} \mid p_{1} < c \right\} \cap S \). Hence \( \left\{ 1/n \right\} \times [0, 1] \) is a component of \( X \).

		      Let \( C \) be the component containing \( (0, 0) \) then \( C \) is disjoint from \( \left\{ 1/n \right\} \times [0, 1] \) for every positive integer \( n \). However, \( \left\{ (0, 0), (0, 1) \right\} \) is not connected since it is disconnected by
		      \[ \left\{ (p_{1}, p_{2}) \in \mathbb{R}^{2} \mid p_{2} < 1/2 \right\} \cap \left\{ (0, 0), (0, 1) \right\} \text{  and  } \left\{ (p_{1}, p_{2}) \in \mathbb{R}^{2} \mid p_{2} > 1/2 \right\} \cap \left\{ (0, 0), (0, 1) \right\} \]

		      so \( \left\{ (0, 0) \right\} \) and \( \left\{ (0, 1) \right\} \) are components of \( X \).

		      Each vertical line \( \left\{ 1/n \right\} \times [0, 1] \) is clopen in \( X \) as
		      \begingroup
		      \allowdisplaybreaks%
		      \begin{align*}
			      \left\{ 1/n \right\} \times [0, 1] & = [1/n - 1/2n(n+1), 1/n + 1/2n(n+1)] \times \mathbb{R} \cap X              \\
			                                         & = \openinterval{1/n - 1/2n(n+1), 1/n + 1/2n(n+1)} \times \mathbb{R} \cap X
		      \end{align*}
		      \endgroup

		      so they are quasi-components of \( X \) (Each component is contained in a quasi-component and each quasi-component containing a point is the intersection of all clopen sets containing the point.)

		      Every neighborhood of \( (0, 0) \) in \( X \) contains the intersection of \( X \) and an open ball centered at \( (0, 0) \). Because any open ball centered at \( (0, 0) \) intersects infinitely many vertical line segments \( \left\{ 1/n \right\} \times [0, 1] \) then \( \left\{ (0, 0) \right\} \) is not open in \( X \). Therefore \( \left\{ (0, 0) \right\} \) is not a quasi-component. Similarly, \( \left\{ (0, 1) \right\} \) is not a quasi-component.

		      Together with each \( \left\{ 1/n \right\} \times [0,1] \) being a quasi-component, we deduce that \( \left\{ (0, 0), (0, 1) \right\} \) is a quasi-component.
	      \end{proof}
	\item A topological space \( X \) is said to be ``path-connected'' if for any two points \( p \) and \( q \) in \( X \) there exists a map \( \lambda: [0, 1] \to X \) with \( \lambda(0) = p \) and \( \lambda(1) = q \). A space \( X \) is ``locally path-connected'' if every neighborhood of any point contains a path-connected neighborhood. A ``path component'' is a maximal path-connected subset. Show that:
	      \begin{enumerate}[itemsep=0pt,label={(\alph*)}]
		      \item a path-connected space is connected;
		      \item a space is the disjoint union of its path components;
		      \item a path component of a space is contained in some component;
		      \item the path components of a locally path-connected space are clopen, and coincide with the components;
		      \item the space with exactly two points \( p \) and \( q \) and open sets \( \varnothing, \left\{ p \right\}, \left\{ p, q \right\} \) (only) is path-connected; and
		      \item the subspace of the plane consisting of \( \left\{ 0 \right\} \times [-1, 1] \cup \left\{ (x, \sin(1/x)) \mid x > 0 \right\} \) is connected but not path-connected.
	      \end{enumerate}

	      \begin{proof}
		      \begin{enumerate}[itemsep=0pt,label={(\alph*)}]
			      \item Assume \( X \) is a path-connected space.

			            Let \( x_{0} \) be a point in \( X \). For every \( p \in X \), there exists a continuous map \( f_{p}: [0, 1] \to X \) with \( f_{p}(0) = x_{0}, f_{p}(1) = p \). The image \( f_{p}([0, 1]) \) is connected because \( [0, 1] \) is connected and \( f_{p} \) is continuous.
			            \[
				            X = \bigcup_{p} f_{p}([0, 1])
			            \]

			            and the images \( f_{p}([0,1]) \) share a common point \( x_{0} \) so \( X \) is connected.
			      \item If two path components are intersecting then their union is path-connected, so they are identical, due to the maximality of path components.

			            Consider an arbitrary point \( p \in X \). The union of all path-connected sets containing \( p \) is path-connected and a path component of \( X \). Therefore each point of \( X \) is contained in a path component.

			            Thus every space is the disjoint union of its path components.
			      \item Every path component is connected, hence is contained in some component.
			      \item Let \( A \) be a path component of some locally path-connected space \( X \).

			            Every point in \( A \) has a path-connected neighborhood \( N \) so \( N \subseteq A \). Therefore \( A \) is open in \( X \). Since \( X \) is the disjoint union of its path components, and each path component is open, so each path component is also closed. Hence each path component of \( X \) is clopen.

			            Let \( C \) be a component of \( X \) then \( C \) is a union of some path components. If \( C \) contains more than one path components then \( C \) is not connected (because the path components in \( X \) are open and disjoint). Therefore \( C \) is also path-connected, hence a path component.
			      \item Define a map \( f: [0, 1] \to \left\{ p, q \right\} \) by
			            \[
				            f(t) = \begin{cases}
					            p & t < 1  \\
					            q & t = 1.
				            \end{cases}
			            \]

			            \( f \) is continuous because the preimage of every closed set in \( \left\{ p, q \right\} \) is closed in \( [0, 1] \). So \( f \) is a path from \( p \) to \( q \). Hence the given two-point space is path-connected.
			      \item The set \( \left\{ (x, \sin(1/x)) \mid x > 0 \right\} \) is connected so its closure
			            \[
				            \left\{ 0 \right\} \times [-1, 1] \cup \left\{ (x, \sin(1/x)) \mid x > 0 \right\}
			            \]

			            is connected.

			            Assume there is a continuous map \( \gamma: [0, 1] \to \left\{ 0 \right\} \times [-1, 1] \cup \left\{ (x, \sin(1/x)) \mid x > 0 \right\} \) where \( \gamma(0) = (0, 0) \) and \( \gamma(1) = (1/\pi, 0) \).

			            The projection maps \( p_{1}: (x, y) \mapsto x \) and \( p_{2}: (x, y) \mapsto y \) are continuous so \( p_{1} \circ \gamma \) and \( p_{2} \circ \gamma \) are continuous. Therefore \( p_{1} \circ \gamma(t) \) can take on any values of the form \( \dfrac{1}{\pm\pi/2 + 2n\pi} \) (\( n \in \mathbb{Z}^{+} \)), so \( p_{2} \circ \gamma \) takes on the values \( \pm 1 \) in any open neighborhood \( \halfopenright{0, b} \). So there is no neighborhood of \( 0 \) that is carried into \( \openinterval{-1/2, 1/2} \) by \( p_{2} \circ \gamma \), which means \( p_{2} \circ \gamma \) is not continuous at \( 0 \), hence a contradiction.

			            Thus \( \left\{ 0 \right\} \times [-1, 1] \cup \left\{ (x, \sin(1/x)) \mid x > 0 \right\} \) is not path-connected.
		      \end{enumerate}
	      \end{proof}
\end{enumerate}

\section{Separation Axioms}

Proposition 5.2 doesn't need the Hausdorff property.

\begin{enumerate}[itemsep=0pt,label={\arabic*.}]
	\item Give an example of a space that is not \( \mathrm{T}_{0} \), and an example of a \( \mathrm{T}_{0} \)-space that is not \( \mathrm{T}_{1} \).
	      \begin{proof}
		      Every space with the trivial topology and consisting of at least two points is not \( \mathrm{T}_{0} \).

		      The Sierpinski space with the underlying set \( \left\{ p, q \right\} \) and open sets being precisely \( \varnothing, \left\{ p \right\}, \left\{ p, q \right\} \) is a \( \mathrm{T}_{0} \)-space that is not \( \mathrm{T}_{1} \).
	      \end{proof}
	\item Show that a finite \( \mathrm{T}_{1} \)-space is discrete.
	      \begin{proof}
		      Let \( X \) be a finite \( \mathrm{T}_{1} \)-space. In a \( \mathrm{T}_{1} \)-space, every singleton set is closed. Since \( X \) is finite, every subset of \( X \) is closed, so \( X \) is discrete.
	      \end{proof}
	\item Consider the set \( \omega \) of natural numbers together with two other points named \( x, y \). Put a partial ordering on this set which orders \( \omega \) as usual and makes both \( x \) and \( y \) greater than any integer, but does not order \( x \) against \( y \). Give this the strong order topology. Show it is \( \mathrm{T}_{1} \) but not Hausdorff.
	      \begin{proof}
		      Let \( X = \omega \cup \left\{ x, y \right\} \).

		      Since \( X \) has the strong order topology, then every singleton subset of \( X \) is closed, so it is \( \mathrm{T}_{1} \).

		      Every open neighborhood of \( x \) contains a subbasis \( \left\{ a \mid a < x \right\} \) and every open neighborhood of \( y \) contains a subbasis \( \left\{ b \mid b < y \right\} \) where \( a, b \in \omega \). These open neighborhoods have at least one common point, namely, \( \max\left\{ a, b \right\} \) so they are not disjoint.

		      Hence \( X \) is not Hausdorff.
	      \end{proof}
	\item Consider the space \( X \) whose point set is the plane but whose open sets are given by the basis consisting of the usual open sets in the plane together with the sets \( \left\{ (x, y) \mid x^{2} + y^{2} < a, y \ne 0 \right\} \cup \left\{ (0, 0) \right\} \) for all \( a > 0 \). Show that \( X \) is Hausdorff but not regular.
	      \begin{proof}
		      The topology is finer than the Euclidean topology (which is Hausdorff) so \( X \) is Hausdorff.

		      Let \( V \) be a closed neighborhood of the origin then \( V \) contains an open disk centered at the origin. However, the neighborhood \( \left\{ (x, y) \mid x^{2} + y^{2} < a, y \ne 0 \right\} \cup \left\{ (0, 0) \right\} \) of the origin doesn't contain any open disk centered at the origin. So the closed neighborhoods of the origin don't form a neighborhood basis at the origin. Therefore \( X \) is not regular.
	      \end{proof}
	\item Show that a subspace of a Hausdorff space is Hausdorff.
	      \begin{proof}
		      Let \( A \) be a subspace of a Hausdorff space \( X \).

		      If \( x, y \) are distinct points in \( A \) then there are disjoint open sets \( U, V \subseteq X \) such that \( x \in U, y \in V \). So \( U \cap A, V \cap A \) are disjoint open neighborhoods of \( x, y \) in \( A \).

		      Therefore \( A \) is Hausdorff.
	      \end{proof}
	\item Show that a space is normal iff for any sets \( U \) open and \( C \) closed with \( C \subseteq U \) there is an open set \( V \) with \( C \subseteq V \subseteq \overline{V} \subseteq U \).
	      \begin{proof}
		      Suppose \( X \) is normal. Let \( U \) be an open set and \( C \) is closed with \( C \subseteq U \) then \( X - U \) and \( C \) are disjoint closed sets. Since \( X \) is normal, there are disjoint open sets \( V, W \) such that \( X - U \subseteq W \) and \( C \subseteq V \). Therefore \( C \subseteq V \subseteq X - W \subseteq U \). Since \( X - W \) is closed and containing \( V \), then \( V \subseteq \overline{V} \subseteq X - W \). Hence \( C \subseteq V \subseteq \overline{V} \subseteq U \).

		      Conversely, suppose that for any sets \( U \) open and \( C \) closed with \( C \subseteq U \) there is an open set \( V \) with \( C \subseteq V \subseteq \overline{V} \subseteq U \). Let \( F, G \) be disjoint closed sets in \( X \) then \( F \subseteq X - G \) and \( X - G \) is open. There is an open set \( B \) such that \( F \subseteq B \subseteq \overline{B} \subseteq X - G \), so \( F \subseteq B \) and \( G \subseteq X - \overline{B} \), where \( B, X - \overline{B} \) are disjoint open sets. Hence \( X \) is normal.
	      \end{proof}
	\item Show that there is a smallest topology on the real numbers such that every singleton is closed. Which of the separation axioms does it satisfy?
	      \begin{proof}
		      The intersection of all topologies on the real numbers, in which every singleton is closed, is also a topology satisfying the property, hence it is the smallest topology (satisfying the property).

		      This topology is identical to the cofinite topology on the real numbers.

		      This topology satisfies \( \mathrm{T}_{0}, \mathrm{T}_{1} \) by its definition but it is not \( \mathrm{T}_{2}, \mathrm{T}_{3}, \mathrm{T}_{4} \).
	      \end{proof}
	\item Show that if a Zariski space (see Section 3, Problem 6) is Hausdorff then it is finite.
	      \begin{proof}
		      Let \( X \) be a Zariski space then \( X = \bigcup_{i=1}^{n} Y_{i} \) with each \( Y_{i} \) is a closed maximal irreducible set.

		      \( X \) is Hausdorff so \( Y_{i} \) is Hausdorff. An irreducible subspace, which is Hausdorff, is singleton. Therefore \( X \) is finite.
	      \end{proof}
	\item Show that a metric space is normal.
	      \begin{proof}
		      Let \( M \) be a metric space and \( F, G \) disjoint closed subsets of \( M \).

		      Define \( d(x, A) = \inf\left\{ \operatorname{dist}(x, a) \mid a \in A \right\} \). If \( d(x, A) = 0 \) then \( x \) is a limit point of \( A \). The function \( d: x \mapsto d(x, A) \) is continuous.

		      Let \( f: X \to \mathbb{R} \) be the function given by \( f(x) = \dfrac{d(x, F)}{d(x, F) + d(x, G)} \). Because \( F, G \) are disjoint closed sets, \( f \) is well-defined for \( d(x, F) + d(x, G) \ne 0 \).

		      \( f \) is continuous and \( f^{-1}(\openinterval{-\infty, 1/2}), f^{-1}(\openinterval{1/2, \infty}) \) are disjoint open sets containing \( A, B \), respectively, for \( f(A) = \left\{ 0 \right\} \) and \( f(B) = \left\{ 1 \right\} \).

		      Thus \( M \) is normal.
	      \end{proof}
\end{enumerate}

\section{Nets (Moore-Smith Convergence)}

\begin{enumerate}[itemsep=0pt,label={\arabic*.}]
	\item Show that a sequence is a universal net if and only if it is eventually constant.
	      \begin{proof}

	      \end{proof}
	\item Consider the space \( X = \Omega \cup \left\{ \Omega \right\} \) of ordinals up to and including the first uncountable ordinal \( \Omega \) with the order topology. Show explicity that there is a net in \( \Omega \) which converges to \( \left\{ \Omega \right\} \) but there is no \textit{sequence} which does so.
	      \begin{proof}

	      \end{proof}
	\item Prove Proposition 6.14. ``A subnet of a universal net is universal.\@''
	      \begin{proof}

	      \end{proof}
	\item Let \( H \) be a dense set in the topological space \( X \) and let \( f: H \to Y \) be a map with \( Y \) regular. Let \( g: X \to Y \) be a \textit{function}. Suppose that for any net \( \left\{ h_{\alpha} \right\} \) in \( H \) with \( h_{\alpha} \to x \in X \) we have \( f(h_{\alpha}) \to g(x) \). Then show that \( g: X \to Y \) is continuous. Also show that the condition of regularity on \( Y \) is needed by giving a counterexample without it.
	      \begin{proof}

	      \end{proof}
\end{enumerate}

\section{Compactness}

\begin{enumerate}[itemsep=0pt,label={\arabic*.}]
	\item Give a direct proof of ``Every net in a compact space has a convergent subnet.\@''
	      \begin{proof}

	      \end{proof}
	\item Let \( X \) be a compact space and let \( \left\{ C_{\alpha} \mid \alpha \in A \right\} \) be a collection of closed sets, closed with respect to finite intersections. Let \( C = \bigcap C_{\alpha} \) and suppose that \( C \subseteq U \) with \( U \) open. Show that \( C_{\alpha} \subseteq U \) for some \( \alpha \).
	      \begin{proof}

	      \end{proof}
	\item Given an example showing that the hypothesis, in Theorem 7.13, that \( f \) is closed, cannot be dropped.
	      \begin{proof}

	      \end{proof}
\end{enumerate}

\section{Products}

\section{Metric Spaces Again}

\section{Existence of Real-valued Functions}

\section{Locally Compact Spaces}

\section{Paracompact Spaces}

\section{Quotient Spaces}

\section{Homotopy}

\section{Topological Groups}

\section{Convex Bodies}

\section{The Baire Category Theorem}
