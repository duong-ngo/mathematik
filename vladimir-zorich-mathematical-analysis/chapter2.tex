\chapter{The Real Numbers}

\section{The Axiom System and Some General Properties of the Set
of Real Numbers}

\section{The Most Important Classes of Real Numbers and Computational Aspects of Operations with Real Numbers}

% chapter2:section2:exercise1
\begin{exercise}
    Using the principle of induction, show that
    \begin{enumerate}[label={(\alph*)}]
        \item the sum $x_{1} + \cdots + x_{n}$ of real numbers is defined independently of the insertion of parentheses to specify the order of addition;
        \item the same is true of the product $x_{1}\cdots x_{n}$;
        \item $\abs{x_{1} + \cdots + x_{n}}\leq \abs{x_{1}} + \cdots + \abs{x_{n}}$
        \item $\abs{x_{1}\cdots x_{n}} = \abs{x_{1}}\cdots \abs{x_{n}}$.
        \item $\left((m, n\in\mathbb{N})\land (m < n)\right)\implies \left(\right)$
        \item ${(1 + x)}^{n}\geq 1 + nx$ for $x > -1$ and $n\in\mathbb{N}$, equality holding only when $n = 1$ or $x = 0$ (Bernoulli's inequality);
        \item ${(a + b)}^{n} = a^{n} + \frac{n}{1!}a^{n-1}b + \frac{n(n-1)}{2!}a^{n-2}b^{2} + \cdots + \frac{n(n-1)\cdots 2}{(n-1)!}ab^{n-1} + b^{n}$ (Newton's binomial formula).
    \end{enumerate}
\end{exercise}

\begin{proof}
    I skip this exercise.
\end{proof}
\newpage

% chapter2:section2:exercise2
\begin{exercise}
    \begin{enumerate}[label={(\alph*)}]
        \item Verify that $\mathbb{Z}$ and $\mathbb{Q}$ are inductive sets.
        \item Give examples of inductive sets different from $\mathbb{N}$, $\mathbb{Z}$, $\mathbb{Q}$, and $\mathbb{R}$.
    \end{enumerate}
\end{exercise}

\begin{proof}
    I skip this exercise.
\end{proof}
\newpage

% chapter2:section2:exercise3
\begin{exercise}
    Show that an inductive set is not bounded above.
\end{exercise}

\begin{proof}
\end{proof}
\newpage

% chapter2:section2:exercise4
\begin{exercise}
    \begin{enumerate}[label={(\alph*)}]
        \item Show that an inductive set is infinite (that is, equipotent with one of its subsets different from itself).
        \item The set $E_{n} = \{ x\in\mathbb{N} \mid x\leq n \}$ is finite. (We denote $\operatorname{card} E_{n}$ by $n$.)
    \end{enumerate}
\end{exercise}

\begin{proof}
\end{proof}
\newpage

% chapter2:section2:exercise5
\begin{exercise}[The Euclidean algorithm]
    \begin{enumerate}[label={(\alph*)}]
        \item Let $m, n \in \mathbb{N}$ and $m > n$. Their greatest common divisor ($\text{gcd}(m, n) = d \in \mathbb{N}$) can be found in a finite number of steps using the following algorithm of Euclid involving successive divisions with remainder.
        \begin{align*}
            m & = q_{1}n + r_{1} & (r_{1} < n) \\
            n & = q_{2}r_{1} + r_{2} & (r_{2} < r_{1}) \\
            r_{1} & = q_{3}r_{2} + r_{3} & (r_{3} < r_{2}) \\
                  & \vdots \\
            r_{k-1} & = q_{k+1}r_{k} + 0.
        \end{align*}

        Then $d = r_{k}$.
        \item If $d = \text{gcd}(m, n)$, one can choose numbers $p, q \in \mathbb{Z}$ such that $pm + qn = d$; in particular, if $m$ and $n$ are relatively prime, then $pm + qn = 1$.
    \end{enumerate}
\end{exercise}

\begin{proof}
\end{proof}
\newpage

% chapter2:section2:exercise6
\begin{exercise}
    Try to give your own proof of the fundamental theorem of arithmetic (Para-
    graph a in Sect. 2.2.2).
\end{exercise}

\begin{proof}
\end{proof}
\newpage

% chapter2:section2:exercise7
\begin{exercise}
    Show that if the product $m\cdot n$ of natural numbers is divisible by a prime $p$, that is, $m\cdot n = p\cdot k$, where $k \in \mathbb{N}$, then either $m$ or $n$ is divisible by $p$.
\end{exercise}

\begin{proof}
\end{proof}
\newpage

% chapter2:section2:exercise8
\begin{exercise}
    It follows from the fundamental theorem of arithmetic that the set of prime numbers is infinite.
\end{exercise}

\begin{proof}
\end{proof}
\newpage

% chapter2:section2:exercise9
\begin{exercise}
    Show that if the natural number $n$ is not of the form $k^{m}$, where $k, m \in \mathbb{N}$, then the equation $x^{m} = n$ has no rational roots.
\end{exercise}

\begin{proof}
\end{proof}
\newpage

% chapter2:section2:exercise10
\begin{exercise}
    Show that the expression of a rational number in any $q$-ary computation system is periodic, that is, starting from some rank it consists of periodically repeating groups of digits.
\end{exercise}

\begin{proof}
\end{proof}
\newpage

% chapter2:section2:exercise11
\begin{exercise}
\end{exercise}

\begin{proof}
\end{proof}
\newpage

% chapter2:section2:exercise12
\begin{exercise}
    Knowing that $\frac{m}{n} := m \cdot n$ ``rules'' for addition, multiplication, and division of fractions, and also the condition for two fractions to be equal.
\end{exercise}

\begin{proof}
    I skip this exercise.
\end{proof}
\newpage

% chapter2:section2:exercise13
\begin{exercise}
    Verify that the rational numbers $\mathbb{Q}$ satisfy all the axioms for real numbers except the axiom of completeness.
\end{exercise}

\begin{proof}
\end{proof}
\newpage

% chapter2:section2:exercise14
\begin{exercise}
    Adopting the geometric model of the set of real numbers (the real line), show how to construct the numbers $a + b$, $a - b$, $ab$, and $\frac{a}{b}$ in this model.
\end{exercise}

\begin{proof}
\end{proof}
\newpage

% chapter2:section2:exercise15
\begin{exercise}
\end{exercise}

\begin{proof}
\end{proof}
\newpage

% chapter2:section2:exercise16
\begin{exercise}
\end{exercise}

\begin{proof}
\end{proof}
\newpage

% chapter2:section2:exercise17
\begin{exercise}
\end{exercise}

\begin{proof}
\end{proof}
\newpage

% chapter2:section2:exercise18
\begin{exercise}
    Let $-A$ be the set of numbers of the form $-a$, where $a \in A \subset R$. Show that $\sup(-A) = -\inf A$.
\end{exercise}

\begin{proof}
\end{proof}
\newpage

% chapter2:section2:exercise19
\begin{exercise}
\end{exercise}

\begin{proof}
\end{proof}
\newpage

% chapter2:section2:exercise20
\begin{exercise}
\end{exercise}

\begin{proof}
\end{proof}
\newpage

% chapter2:section2:exercise21
\begin{exercise}
\end{exercise}

\begin{proof}
\end{proof}
\newpage

% chapter2:section2:exercise22
\begin{exercise}
\end{exercise}

\begin{proof}
\end{proof}
\newpage

% chapter2:section2:exercise23
\begin{exercise}
    Show that if $\mathbb{R}$ and $\mathbb{R}'$ are two models of the set of real numbers and $f: \mathbb{R}\to \mathbb{R}'$ is a mapping such that $f(x + y) = f(x) + f(y)$ and $f(x\cdot y) = f(x)\cdot f(y)$ for any $x, y\in \mathbb{R}$, then
    \begin{enumerate}[label={(\alph*)}]
        \item $f(0) = 0'$;
        \item $f(1) = 1'$ if $f(x)\not\equiv 0'$, which we shell henceforth assume;
        \item $f(m) = m'$ where $m\in\mathbb{Z}$ and $m'\in\mathbb{Z}'$, and the mapping $f: \mathbb{Z}\to \mathbb{Z}'$ is injective and preserves the order.
        \item $f(\frac{m}{n}) = \frac{m'}{n'}$, where $m, n\in\mathbb{Z}$, $n\ne 0$, $m',n'\in\mathbb{Z}'$, $n'\ne 0'$, $f(m) = m'$, $f(n) = n'$. Thus $f: \mathbb{Q}\to \mathbb{Q}'$ is a bijection that preserves order.
        \item $f: \mathbb{R}\to \mathbb{R}'$ is a bijection that preserves order.
    \end{enumerate}
\end{exercise}

\begin{proof}
\end{proof}
\newpage

% chapter2:section2:exercise24
\begin{exercise}
    On the basis of the preceding exercise and the axiom of completeness, show
that the axiom system for the set of real numbers determines it completely up to an
isomorphism (method of realizing it), that is, if $\mathbb{R}$ and $\mathbb{R}'$ are two sets satisfying these axioms, then there exists a one-to-one correspondence $f: \mathbb{R}\to \mathbb{R}'$ that preserves the arithmetic operations and the order: $f(x + y) = f(x) + f(y)$, $f(x\cdot y) = f(x)\cdot f(y)$, and $(x\leq y) \Longleftarrow (f(x) \leq f(y))$.
\end{exercise}

\begin{proof}
\end{proof}
\newpage

% chapter2:section2:exercise25
\begin{exercise}
\end{exercise}

\begin{proof}
\end{proof}
\newpage

% chapter2:section2:exercise26
\begin{exercise}
\end{exercise}

\begin{proof}
\end{proof}
\newpage

% chapter2:section2:exercise27
\begin{exercise}
    Write ${(100)}_{10}$ in the binary and ternary systems.
\end{exercise}

\begin{proof}
    \begin{align*}
        {(100)}_{10} & = 1\cdot 2^{6} + 1\cdot 2^{5} + 0\cdot 2^{4} + 0\cdot 2^{3} + 1\cdot 2^{2} + 0\cdot 2^{1} + 0\cdot 2^{0} \\
                     & = {(1100100)}_{2}, \\
        {(100)}_{10} & = 1\cdot 3^{4} + 2\cdot 3^{2} + 0\cdot 3^{1} + 1\cdot 3^{0} \\
                     & = {(1201)}_{3}.
    \end{align*}
\end{proof}
\newpage

% chapter2:section2:exercise28
\begin{exercise}
\end{exercise}

\begin{proof}
\end{proof}
\newpage

% chapter2:section2:exercise29
\begin{exercise}
    What is the smallest number of questions to be answered ``yes'' or ``no'' that one must pose in order to be sure of determining a 7-digit telephone number?
\end{exercise}

\begin{proof}
    I skip this exercise.
\end{proof}
\newpage

% chapter2:section2:exercise30
\begin{exercise}
\end{exercise}

\begin{proof}
\end{proof}
\newpage

\section{Basic Lemmas Connected with the Completeness of the Real Numbers}

% chapter2:section3:exercise1
\begin{exercise}
\end{exercise}

\begin{proof}
\end{proof}
\newpage

% chapter2:section3:exercise2
\begin{exercise}
\end{exercise}

\begin{proof}
\end{proof}
\newpage

% chapter2:section3:exercise3
\begin{exercise}
\end{exercise}

\begin{proof}
\end{proof}
\newpage

% chapter2:section3:exercise4
\begin{exercise}
\end{exercise}

\begin{proof}
\end{proof}
\newpage

\section{Countable and Uncountable Sets}

% chapter2:section4:exercise1
\begin{exercise}
\end{exercise}

\begin{proof}
\end{proof}
\newpage

% chapter2:section4:exercise2
\begin{exercise}
\end{exercise}

\begin{proof}
\end{proof}
\newpage

% chapter2:section4:exercise3
\begin{exercise}
\end{exercise}

\begin{proof}
\end{proof}
\newpage

% chapter2:section4:exercise4
\begin{exercise}
\end{exercise}

\begin{proof}
\end{proof}
\newpage

% chapter2:section4:exercise5
\begin{exercise}
\end{exercise}

\begin{proof}
\end{proof}
\newpage

% chapter2:section4:exercise6
\begin{exercise}
\end{exercise}

\begin{proof}
\end{proof}
\newpage

% chapter2:section4:exercise7
\begin{exercise}
\end{exercise}

\begin{proof}
\end{proof}
\newpage

% chapter2:section4:exercise8
\begin{exercise}
\end{exercise}

\begin{proof}
\end{proof}
\newpage
